\documentclass[11pt]{report}
%%%%%%%%%%%%%%%%%%%%%%%%%%%%%%%%%%%%%%%%%%%%%%%%%%%%%%%%%%%%%%%%%%%%%%%%%%%%%%%%%%%%%%%%%%%%%%%%%%%%%%%%%%%%%%%%%%%%%%%%%%%%
%\usepackage[none,dcucite,abbr]{harvard}
\usepackage[T1]{fontenc}
\usepackage[latin1]{inputenc}
%\usepackage{harvard}
\usepackage{graphicx}
\usepackage{amsmath,amsthm,amsfonts,amssymb}
\usepackage{uathesis}
\usepackage{lscape}
\usepackage{longtable}
\usepackage{pstricks,pst-node}
\usepackage[mathscr]{euscript}
\usepackage[arrow,matrix,tips]{xy}



%\floatstyle{boxed}
%\newfloat{QuantumCircuit}{htp}{qci}[chapter]
%\floatname{QuantumCircuit}{Circuit}

%\setlength{\textheight}{8.5in}
%\setlength{\textwidth}{6.5in}
%\setlength{\oddsidemargin}{-.3in}
%\setlength{\evensidemargin}{-.3in}
% macros for processing of the literate files

% reference formats - 
\labelformat{chapter}{chapter~#1}
\labelformat{section}{section~#1}
\labelformat{subsection}{sub-section~#1}
\labelformat{subsubsection}{sub-sub-section~#1}
\labelformat{equation}{equation~(#1)}
\labelformat{table}{table~#1}
\labelformat{figure}{figure~#1}


\DefineVerbatimEnvironment%
 {happycodefirst}{Verbatim}{gobble=2,numbers=left,firstnumber=1,fontsize=\footnotesize}
\DefineVerbatimEnvironment%
 {qplcodethesis}{Verbatim}{commandchars=\\\{\},numbers=left,firstnumber=1,fontsize=\footnotesize}
\DefineVerbatimEnvironment%
 {bnf}{Verbatim}{commandchars=\\\{\},fontfamily=courier,fontsize=\footnotesize}
\DefineVerbatimEnvironment%
 {happycodecont}{Verbatim}{gobble=2,numbers=left,firstnumber=last,fontsize=\footnotesize}
% import and redefinition of general macros
\input{/home/grads/gilesb/latexFiles/macros}
\newcommand{\inflist}[1]{\ensuremath{\mathbb{IL}({#1})}}
\renewcommand{\incsec}[1]{\subsection{#1}}
% end of redefinitions.
\newcommand{\incsubsec}[1]{\subsubsection{#1}}
\newcommand{\incsubsubsec}[1]{\paragraph{#1}}
\newcommand{\qplcode}[1]{\textbf{#1}}
\DefineVerbatimEnvironment%
{code}{Verbatim}{numbers=left,fontsize=\footnotesize,firstnumber=1}
\newcommand{\CodeResetNumbers}{\RecustomVerbatimEnvironment%
{code}{Verbatim}{numbers=left,fontsize=\footnotesize,firstnumber=1}}
\newcommand{\CodeContinueNumbers}{\RecustomVerbatimEnvironment%
{code}{Verbatim}{numbers=left,fontsize=\footnotesize,firstnumber=last}}
\renewcommand{\C}{\ensuremath{\mathbb{C}}}

%    Q-circuit version 2
%    Copyright (C) 2004  Steve Flammia & Bryan Eastin
%    Last modified on: 9/16/2011
%
%    This program is free software; you can redistribute it and/or modify
%    it under the terms of the GNU General Public License as published by
%    the Free Software Foundation; either version 2 of the License, or
%    (at your option) any later version.
%
%    This program is distributed in the hope that it will be useful,
%    but WITHOUT ANY WARRANTY; without even the implied warranty of
%    MERCHANTABILITY or FITNESS FOR A PARTICULAR PURPOSE.  See the
%    GNU General Public License for more details.
%
%    You should have received a copy of the GNU General Public License
%    along with this program; if not, write to the Free Software
%    Foundation, Inc., 59 Temple Place, Suite 330, Boston, MA  02111-1307  USA

% Thanks to the Xy-pic guys, Kristoffer H Rose, Ross Moore, and Daniel Müllner,
% for their help in making Qcircuit work with Xy-pic version 3.8.
% Thanks also to Dave Clader, Andrew Childs, Rafael Possignolo, Tyson Williams,
% Sergio Boixo, Cris Moore, Jonas Anderson, and Stephan Mertens for helping us test
% and/or develop the new version.

\usepackage{xy}
\xyoption{matrix}
\xyoption{frame}
\xyoption{arrow}
\xyoption{arc}

\usepackage{ifpdf}
\ifpdf
\else
\PackageWarningNoLine{Qcircuit}{Qcircuit is loading in Postscript mode.  The Xy-pic options ps and dvips will be loaded.  If you wish to use other Postscript drivers for Xy-pic, you must modify the code in Qcircuit.tex}
%    The following options load the drivers most commonly required to
%    get proper Postscript output from Xy-pic.  Should these fail to work,
%    try replacing the following two lines with some of the other options
%    given in the Xy-pic reference manual.
\xyoption{ps}
\xyoption{dvips}
\fi

% The following resets Xy-pic matrix alignment to the pre-3.8 default, as
% required by Qcircuit.
\entrymodifiers={!C\entrybox}

\newcommand{\bra}[1]{\ensuremath{\left\langle{#1}\right\vert}\xspace}
\newcommand{\ket}[1]{\ensuremath{\left\vert{#1}\right\rangle}\xspace}
    % Defines Dirac notation. %7/5/07 added extra braces so that the commands will work in subscripts.
\newcommand{\qw}[1][-1]{\ar @{-} [0,#1]}
    % Defines a wire that connects horizontally.  By default it connects to the object on the left of the current object.
    % WARNING: Wire commands must appear after the gate in any given entry.
\newcommand{\dqw}[1][-1]{\ar @{.} [0,#1]}
    % Defines a dotted wire that connects horizontally.  By default it connects to the object on the left of the current object.
    % WARNING: Wire commands must appear after the gate in any given entry.
\newcommand{\qwx}[1][-1]{\ar @{-} [#1,0]}
    % Defines a wire that connects vertically.  By default it connects to the object above the current object.
    % WARNING: Wire commands must appear after the gate in any given entry.
\newcommand{\cw}[1][-1]{\ar @{=} [0,#1]}
    % Defines a classical wire that connects horizontally.  By default it connects to the object on the left of the current object.
    % WARNING: Wire commands must appear after the gate in any given entry.
\newcommand{\cwx}[1][-1]{\ar @{=} [#1,0]}
    % Defines a classical wire that connects vertically.  By default it connects to the object above the current object.
    % WARNING: Wire commands must appear after the gate in any given entry.
\newcommand{\gate}[1]{*+<.6em>{#1} \POS ="i","i"+UR;"i"+UL **\dir{-};"i"+DL **\dir{-};"i"+DR **\dir{-};"i"+UR **\dir{-},"i" \qw}
    % Boxes the argument, making a gate.
\newcommand{\meter}{*=<1.8em,1.4em>{\xy ="j","j"-<.778em,.322em>;{"j"+<.778em,-.322em> \ellipse ur,_{}},"j"-<0em,.4em>;p+<.5em,.9em> **\dir{-},"j"+<2.2em,2.2em>*{},"j"-<2.2em,2.2em>*{} \endxy} \POS ="i","i"+UR;"i"+UL **\dir{-};"i"+DL **\dir{-};"i"+DR **\dir{-};"i"+UR **\dir{-},"i" \qw}
    % Inserts a measurement meter.
    % In case you're wondering, the constants .778em and .322em specify
    % one quarter of a circle with radius 1.1em.
    % The points added at + and - <2.2em,2.2em> are there to strech the
    % canvas, ensuring that the size is unaffected by erratic spacing issues
    % with the arc.
\newcommand{\measure}[1]{*+[F-:<.9em>]{#1} \qw}
    % Inserts a measurement bubble with user defined text.
\newcommand{\measuretab}[1]{*{\xy*+<.6em>{#1}="e";"e"+UL;"e"+UR **\dir{-};"e"+DR **\dir{-};"e"+DL **\dir{-};"e"+LC-<.5em,0em> **\dir{-};"e"+UL **\dir{-} \endxy} \qw}
    % Inserts a measurement tab with user defined text.
\newcommand{\measureD}[1]{*{\xy*+=<0em,.1em>{#1}="e";"e"+UR+<0em,.25em>;"e"+UL+<-.5em,.25em> **\dir{-};"e"+DL+<-.5em,-.25em> **\dir{-};"e"+DR+<0em,-.25em> **\dir{-};{"e"+UR+<0em,.25em>\ellipse^{}};"e"+C:,+(0,1)*{} \endxy} \qw}
    % Inserts a D-shaped measurement gate with user defined text.
\newcommand{\multimeasure}[2]{*+<1em,.9em>{\hphantom{#2}} \qw \POS[0,0].[#1,0];p !C *{#2},p \drop\frm<.9em>{-}}
    % Draws a multiple qubit measurement bubble starting at the current position and spanning #1 additional gates below.
    % #2 gives the label for the gate.
    % You must use an argument of the same width as #2 in \ghost for the wires to connect properly on the lower lines.
\newcommand{\multimeasureD}[2]{*+<1em,.9em>{\hphantom{#2}} \POS [0,0]="i",[0,0].[#1,0]="e",!C *{#2},"e"+UR-<.8em,0em>;"e"+UL **\dir{-};"e"+DL **\dir{-};"e"+DR+<-.8em,0em> **\dir{-};{"e"+DR+<0em,.8em>\ellipse^{}};"e"+UR+<0em,-.8em> **\dir{-};{"e"+UR-<.8em,0em>\ellipse^{}},"i" \qw}
    % Draws a multiple qubit D-shaped measurement gate starting at the current position and spanning #1 additional gates below.
    % #2 gives the label for the gate.
    % You must use an argument of the same width as #2 in \ghost for the wires to connect properly on the lower lines.
\newcommand{\control}{*!<0em,.025em>-=-<.2em>{\bullet}}
    % Inserts an unconnected control.
\newcommand{\controlo}{*+<.01em>{\xy -<.095em>*\xycircle<.19em>{} \endxy}}
    % Inserts a unconnected control-on-0.
\newcommand{\ctrl}[1]{\control \qwx[#1] \qw}
    % Inserts a control and connects it to the object #1 wires below.
\newcommand{\ctrlo}[1]{\controlo \qwx[#1] \qw}
    % Inserts a control-on-0 and connects it to the object #1 wires below.
\newcommand{\targ}{*+<.02em,.02em>{\xy ="i","i"-<.39em,0em>;"i"+<.39em,0em> **\dir{-}, "i"-<0em,.39em>;"i"+<0em,.39em> **\dir{-},"i"*\xycircle<.4em>{} \endxy} \qw}
    % Inserts a CNOT target.
\newcommand{\qswap}{*=<0em>{\times} \qw}
    % Inserts half a swap gate.
    % Must be connected to the other swap with \qwx.
\newcommand{\multigate}[2]{*+<1em,.9em>{\hphantom{#2}} \POS [0,0]="i",[0,0].[#1,0]="e",!C *{#2},"e"+UR;"e"+UL **\dir{-};"e"+DL **\dir{-};"e"+DR **\dir{-};"e"+UR **\dir{-},"i" \qw}
    % Draws a multiple qubit gate starting at the current position and spanning #1 additional gates below.
    % #2 gives the label for the gate.
    % You must use an argument of the same width as #2 in \ghost for the wires to connect properly on the lower lines.
\newcommand{\ghost}[1]{*+<1em,.9em>{\hphantom{#1}} \qw}
    % Leaves space for \multigate on wires other than the one on which \multigate appears.  Without this command wires will cross your gate.
    % #1 should match the second argument in the corresponding \multigate.
\newcommand{\push}[1]{*{#1}}
    % Inserts #1, overriding the default that causes entries to have zero size.  This command takes the place of a gate.
    % Like a gate, it must precede any wire commands.
    % \push is useful for forcing columns apart.
    % NOTE: It might be useful to know that a gate is about 1.3 times the height of its contents.  I.e. \gate{M} is 1.3em tall.
    % WARNING: \push must appear before any wire commands and may not appear in an entry with a gate or label.
\newcommand{\gategroup}[6]{\POS"#1,#2"."#3,#2"."#1,#4"."#3,#4"!C*+<#5>\frm{#6}}
    % Constructs a box or bracket enclosing the square block spanning rows #1-#3 and columns=#2-#4.
    % The block is given a margin #5/2, so #5 should be a valid length.
    % #6 can take the following arguments -- or . or _\} or ^\} or \{ or \} or _) or ^) or ( or ) where the first two options yield dashed and
    % dotted boxes respectively, and the last eight options yield bottom, top, left, and right braces of the curly or normal variety.  See the Xy-pic reference manual for more options.
    % \gategroup can appear at the end of any gate entry, but it's good form to pick either the last entry or one of the corner gates.
    % BUG: \gategroup uses the four corner gates to determine the size of the bounding box.  Other gates may stick out of that box.  See \prop.

\newcommand{\rstick}[1]{*!L!<-.5em,0em>=<0em>{#1}}
    % Centers the left side of #1 in the cell.  Intended for lining up wire labels.  Note that non-gates have default size zero.
\newcommand{\lstick}[1]{*!R!<.5em,0em>=<0em>{#1}}
    % Centers the right side of #1 in the cell.  Intended for lining up wire labels.  Note that non-gates have default size zero.
\newcommand{\ustick}[1]{*!D!<0em,-.5em>=<0em>{#1}}
    % Centers the bottom of #1 in the cell.  Intended for lining up wire labels.  Note that non-gates have default size zero.
\newcommand{\dstick}[1]{*!U!<0em,.5em>=<0em>{#1}}
    % Centers the top of #1 in the cell.  Intended for lining up wire labels.  Note that non-gates have default size zero.
\newcommand{\Qcircuit}{\xymatrix @*=<0em>}
\newcommand{\Qcircuitnocompile}{\xymatrixnocompile @*=<0em>}
    % Defines \Qcircuit as an \xymatrix with entries of default size 0em.
\newcommand{\link}[2]{\ar @{-} [#1,#2]}
    % Draws a wire or connecting line to the element #1 rows down and #2 columns forward.
\newcommand{\pureghost}[1]{*+<1em,.9em>{\hphantom{#1}}}
    % Same as \ghost except it omits the wire leading to the left.
 % for quantum circuits

\newcommand{\marginnote}[1]{\marginpar{\tiny #1}}



\lstdefinestyle{hskl}{language=Haskell,
%  basicstyle=\small, % swap this and the following line for prop. font
  basicstyle=\small,
%  keywordstyle=\underbar, % these look ugly
  identifierstyle=\itshape,
  commentstyle=\ttfamily,
%  commentstyle=\underbar
  flexiblecolumns=false,
  basewidth={0.5em,0.45em},
  morekeywords={Map},
  % The following replace compound charcters like ->
  % Something is missing - someone kindly sent me an email which 
  % I've lost - but it's obvious how to add more.
  literate={-}{{$-$}}1 {+}{{$+$}}1 {/}{{$/$}}1 
	   {*}{{$\times$}}1 {=}{{$=$}}1
	   {\%}{{$\%$}}1 {=}{{$=$}}1
           {>}{{$>$}}1 {<}{{$<$}}1 
           {>>}{{$\gg$}}2 {<<}{{$\ll$}}2
           {=>}{{$\Rightarrow$}}2 {>=}{{$\geq$}}2 {<-}{{$\leftarrow$}}2
           {<=}{{$\leq$}}2
           {<==}{{$\Longleftarrow$}}3 
           {=/=}{{$\neq$}}3 
} 
\lstdefinestyle{origlinqpl}{language=lqpl,basicstyle=\footnotesize\ttfamily,%
	numbers=left,%
        numberstyle=\tiny,%
        numbersep=6pt }
\lstdefinestyle{linqpl}{language=lqpl,basicstyle=\footnotesize,%
	numbers=left,%
        numberstyle=\tiny,%
        numbersep=6pt,%
	escapechar=`,%
  literate={-}{{$-$}}1 {+}{{$+$}}1 {/}{{$/$}}1%
	   {*}{{$\times$}}1 {=}{{$=$}}1%
	   {\%}{{$\%$}}1 {=}{{$=$}}1%
           {>}{{$>$}}1 {<}{{$<$}}1%
           {>>}{{$\gg$}}2 {<<}{{$\ll$}}2%
           {=>}{{$\Rightarrow$}}2 {>=}{{$\geq$}}2%
           {=<}{{$\leq$}}2%
           {<=}{{$\Leftarrow$}}3%
           {=/=}{{$\neq$}}3 }
\lstdefinestyle{inlinqpl}{language=lqpl,basicstyle=\footnotesize\ttfamily,%
	escapechar=`}

\newcommand{\lbl}{{\cdptr}}
\newcommand{\cd}{\ensuremath{\mathcal{C}}}
\newcommand{\lblcd}{{\cd}}
%\newcommand{\lblcd}{\ensuremath{@\lbl}}
\newcommand{\cdptr}{\ensuremath{\triangleright\cd}}
\newcommand{\TODO}[1]{\begin{quote}TODO: \textbf{#1}\end{quote}}
\newcommand{\lqpl}{L-QPL}
\newcommand{\linearqpl}{Linear QPL}

\newcommand{\inlqpl}[1]{\protect{\lstinline[style=inlinqpl]!#1!}}
\newcommand{\inlhskl}[1]{\lstinline[style=hskl]!#1!}


\newcommand{\Had}{\text{Hadamard}}
\newcommand{\nottr}{\text{Not}}
\newcommand{\Cnot}{controlled{-}\nottr}
\newcommand{\Z}{\ensuremath{Z}}
\newcommand{\stacknode}[1]{\ensuremath{\mathrm{\texttt{#1}}}}
\newcommand{\Int}{\stacknode{Int}}
\newcommand{\Datatype}{\stacknode{datatype}}
\newcommand{\Bool}{\stacknode{Bool}}

\newcommand{\vc}[1]{\ensuremath{\mathbold{#1}}}
\newcommand{\syntacticset}[1]{\ensuremath{\mathbold{#1}}}

\newcommand{\n}{\syntacticset{N}}
\newcommand{\T}{\syntacticset{T}}
\newcommand{\Q}{\syntacticset{Q}}
\newcommand{\R}{\syntacticset{R}}
\newcommand{\Data}{\syntacticset{Data}}
\newcommand{\Qloc}{\syntacticset{Qloc}}
\newcommand{\Cloc}{\syntacticset{Cloc}}
\newcommand{\Loc}{\syntacticset{Loc}}
\newcommand{\Aexp}{\syntacticset{Aexp}}
\newcommand{\Bexp}{\syntacticset{Bexp}}
\newcommand{\Cexp}{\syntacticset{Cexp}}
\newcommand{\Stm}{\syntacticset{Stm}}
\newcommand{\eval}[3]{\ensuremath{\langle{#1},{#2}\rangle\to{#3}}}
\newcommand{\true}{\ensuremath{\mathbold{true}}}
\newcommand{\false}{\ensuremath{\mathbold{false}}}


\newcommand{\qcons}[1]{\texttt{#1}}
\newcommand{\qtype}[1]{\texttt{\bf{#1}}}
\newcommand{\bms}{\specialcat{Bqsm}}
\newcommand{\lbms}{\specialcat{lBqsm}}
\newcommand{\cms}{\specialcat{Cqsm}}
\newcommand{\ms}{\specialcat{QSM}}

\newcommand{\qsp}{\ensuremath{\to}}

\newcommand{\qsnodeij}[4]{\ensuremath{#1\{{#2}\qsp {#3}\}^{#4}}}

\newcommand{\qsbit}[4]{\ensuremath{#1\{0\qsp {#2};1\qsp {#3}\}^{#4}}}



\newcommand{\qsqbit}[6]{\ensuremath{#1\{00\qsp {#2};01\qsp {#3};10\qsp {#4};11\qsp {#5}\}^{#6}}}

\newcommand{\qsbitUp}[4]{\ensuremath{#1\left\{\begin{array}{l}%
                	0\qsp {#2}\\%
                        1\qsp {#3}%
                        \end{array}%
	             \right\}^{#4}}}


\newcommand{\qsqbitUp}[6]{\ensuremath{#1\left\{\begin{array}{ll}%
                	00\qsp {#2};&01\qsp {#3}\\%
                        10\qsp {#4};&11\qsp {#5}%
                        \end{array}%
	             \right\}^{#6}}}
\newcommand{\qsqbitAllUp}[6]{\ensuremath{#1\left\{\begin{array}{l}%
                	00\qsp {#2}\\%
                        01\qsp {#3}\\%
                        10\qsp {#4}\\%
                        11\qsp {#5}%
                        \end{array}%
	             \right\}^{#6}}}

\newcommand{\eentail}{\ensuremath{\Vdash\!\!\dashv}}
\newcommand{\ientail}{\ensuremath{\Vdash}}
\newcommand{\qentail}{\ensuremath{\vdash}}
\newcommand{\qcontext}{\ensuremath{\, |\, }}
\newcommand{\qcontrolled}[2]{\ensuremath{{#1}\Leftarrow{#2}}}
\newcommand{\qvec}[1]{\ensuremath{\widetilde{#1}}}
\newcommand{\qop}[2]{\ensuremath{{#1}\cdot{#2}}}
\newcommand{\qopseq}[2]{\ensuremath{{#1};{#2}}}

\newcommand{\qresultsind}[3]{\ensuremath{{#2}\overset{#1}{\leadsto}{#3}}}
\newcommand{\qresultsin}[2]{\qresultsind{}{#1}{#2}}
\newcommand{\qresultsindUp}[3]{\ensuremath{\begin{array}{l}%
             {#2}\overset{#1}{\leadsto}\\%
	\qquad{#3}}%
          \end{array}}
\newcommand{\qresultsinUp}[2]{\qresultsindUp{}{#1}{#2}}
\newcommand{\stackthree}[3]{\ensuremath{\begin{array}{l}{#1}\\{#2}\\{#3}\end{array}}}
\newcommand{\stackthreec}[3]{\ensuremath{\begin{array}{c}{#1}\\{#2}\\{#3}\end{array}}}
\newcommand{\stacktwo}[2]{\ensuremath{\begin{array}{l}{#1}\\{#2}\end{array}}}
\newcommand{\stacktwoc}[2]{\ensuremath{\begin{array}{c}{#1}\\{#2}\end{array}}}
\newcommand{\qmeasnob}[3]{{\begin{singlespace}\stackthree{\text{meas }{#1}:}{\quad \ket{0}=>#2}{\quad \ket{1}=>#3}\end{singlespace}}}
\newcommand{\qmeas}[3]{{\begin{singlespace}\left\{\stackthree{\text{meas }{#1}:}{\ \ket{0}=>#2}{\ \ket{1}=>#3}\right\}\end{singlespace}}}
\newcommand{\qcompose}[2]{\ensuremath{{#1};{#2}}}
\newcommand{\qtensor}[2]{\ensuremath{{#1};;{#2}}}
\newcommand{\qins}{\ensuremath{\mathscr{I}}}
\newcommand{\qmod}{\ensuremath{\mathscr{M}}}
\newcommand{\qcase}[4]{{\begin{singlespace}\left\{\stacktwo{\text{case }{#1}:}{\{{\ {#2}=>{#3}}\}_{#4}}\right\}\end{singlespace}}}
\newcommand{\quse}[2]{{\left\{\text{use }{#1}:\ \{{#2}\}\right\}}}
\newcommand{\tcls}{\ensuremath{\tau_C}}
\newcommand{\semins}[1]{\texttt{#1}}
\newcommand{\qifelse}[6]{{\begin{singlespace}\left\{\stackthree{\text{if }{#1}=>{#2}}{{\{\ {#3}=>{#4}}\}_{#5}}{\text{else }=>{#6}}\right\}\end{singlespace}}}


\newcommand{\qsmins}[1]{\texttt{#1}}

\newcommand{\qsminsparm}[1]{\ensuremath{#1}}
\newcommand{\qsminswithp}[2]{\texttt{#1}\qsminsparm{\ #2}}
\newcommand{\dmpelemqc}{\text{\texttt{Qc}}}
\newcommand{\qsmbool}[1]{\text{\texttt{#1}}}
\newcommand{\qsmfalse}{\qsmbool{False}}
\newcommand{\qsmtrue}{\qsmbool{True}}
\newcommand{\qstackMod}[1]{\texttt{#1}}

\newcommand{\terminalio}[1]{\texttt{#1}}

\newcommand{\nowiregate}[1]{*{\xy *+<.6em>{#1};p\save+LU;+RU **\dir{-}\restore\save+RU;+RD **\dir{-}\restore\save+RD;+LD **\dir{-}\restore\POS+LD;+LU **\dir{-}\endxy}}

\newcommand{\interpsem}[1]{\ensuremath{\left\llbracket {#1} \right\rrbracket}}

\newcommand{\Inflist}{\text{Inflist}}

\newcommand{\trspace}{\ensuremath{\qquad\qquad\qquad\qquad\qquad}}
\newcommand{\trbigspace}{\ensuremath{\qquad\qquad\qquad\qquad\qquad\qquad\qquad}}

\newcommand{\trspacefour}{\ensuremath{\qquad\qquad\qquad\qquad}}
\newcommand{\trspacethree}{\ensuremath{\qquad\qquad\qquad}}
\newcommand{\trspacetwo}{\ensuremath{\qquad\qquad}}

\newcommand{\ilsep}{\ensuremath{\blacktriangleright}}

\newcommand{\lqplmodifier}[1]{\text{\texttt{#1}}}
\newcommand{\IdOnly}{\lqplmodifier{IdOnly}}
\newcommand{\Left}{\lqplmodifier{Left}}
\newcommand{\Right}{\lqplmodifier{Right}}
        
\begin{document}
\nocite{bib:allen1980,bib:hughes1972,bib:luthar1980,bib:polcino1972,bib:sehgal78}
\degree{\MSc}%

\dept{Mathematics}%
%

\field{Algebra}%
%
\permanentaddress{93 Fonda Green \\
Calgary, AB \\
Canada, T2A 5S4}%
%

\examiners{S. K. Sehgal, Gerald Gliff, A. Weiss, A. Al-Sallam}

\convocationseason{Fall}%
%

%BeginExpansion
%\frontpiece
THE UNIVERSITY OF ALBERTA



\title{The Unit Groups of Certain Group Rings}
\author{Brett G. Giles}
\admin
\doublespacing
\begin{abstract}
This thesis deals with the problem of describing the unit group of specific group rings over the integers. 
After a brief introduction to remind one of some of the properties of the group ring we start to discuss 
the unit groups. A few of the basic results as presented by Sehgal 
\cite{bib:sehgal78}, Chapter 2, are shown. 
After this introduction we talk about specifics.

The first method to determine a unit group is then discussed. It is a general method, applicable to
any group.  However, in practice we see that it is unsuitable for any but a small number of groups.
In this section we talk in an expository manner as the proofs of the results are normally very
dependant on the particular group. The ones presented by this method later on are \(S_3, D_4, D_6\)
and \(A_4\).

Next, we present the method for groups of order $p^3$, where $p$ is an odd prime. 
These come from
the paper by Ritter and Sehgal  \cite{bib:ritter1981}.
 In this paper they also present a method for determining the
unit group of a particular group of order $p^n$. This will not be treated here. I consider both
non-abelian groups of order $p^3$ and descriptions of the unit groups of both of their respective
group rings are presented. Later on in the paper I present the method as applied to the groups of
order 27.

The last theoretical results are on determining the unit group of the group ring over a group of
order $pq$ where $p\equiv 1(\mod{q})$. These results come from a 
paper by Luthar  \cite{bib:luthar1980}. No practical
examples are done of this method.

The next part deals with presenting actual groups and determining the unit structure of their
integral group ring. The first two, $S_3$ and $D_4$ are from previous authors. The first was done by
Hughes and Pearson \cite{bib:hughes1972},
 the second by C. Polcino-Milies \cite{bib:polcino1972}. The next, $D_6$, is new. The last one
is merely an expository account of Allen and Hobby's \cite{bib:allen1980}
 rendering of \( \mathcal{U} ( \mathbb{Z}
\mathrm{A}_4)\). Our other concrete examples deal with the two non-abelian groups of order 27 as
presented in  \cite{bib:ritter1981} by Ritter and Sehgal.  
\end{abstract}

\onehalfspacing %everything from here on will be 1.5 spaced


\tableofcontents
\listoffigures
\listoftables

\bodyoftext 
\chapter{Preliminaries} 
\section{Generalities} 
Throughout this paper the term \emph{group ring} shall be taken as follows: The group ring $KG$ of
the group $G$ over the ring $K$ (which posesses an identity) is the ring of all formal sums
\[\alpha=\sum \lambda(g)g \] where \(g \in G\) and \( \lambda(g) \in K\) so that
 \(\supp(a)= \{g \mid \lambda(g) \neq 0\}\) is a finite set.

The ring structure is inherited from the structures of the group and the ring. The operations on 
\(KG\) are defined as follows:
\begin{align*}
\sum\lambda(g)g &= \sum\mu(g)g &\iff \forall  g \in G, \lambda(g) = \mu(g)\\
\sum \lambda(g)g + \sum \mu(g)g &= \sum(\lambda(g)+\mu(g))g\\
\sum \lambda(g)g \centerdot \sum \mu(g)g &= \sum \nu(g)g
\end{align*}
where \( \nu(g) = \sum \lambda(x)\mu(y) \) with the last sum being over all  \( (x,y) \in  
G\times  G \mathrm{\ with} \  x\centerdot y = g\).

In this paper, we will be investigating the
\emph{unit group} of the group ring. 
\begin{definition}
The \emph{unit group} of a group ring \KG\ is the group of all elements invertible under
the multiplication of \KG. 
\end{definition}

It should be noted that the multiplicative identity in \KG{} is the
element \(1e\), where \(1\) is the multiplicitive identity of \(K\) and
 \(e\) the group identity. Obviously, the unit group of \KG\ contains
 \( \pm G\). Further definition of the unit group becomes more interesting.

\section{Notation}

Throughout this paper the following will be assumed.
\begin{description}
\item[\integers{}] the ring of integers.
\item[\rat] the ring of rational numbers.
\item[\( R_n \)] the ring of \(n\) by \(n\) matrices with entries from \(R\).
\item[\(\ggen{c} \)] The group generated by the element \(c\).  (There may be be more than one element 
for generating.)
\item[\(\Delta (G,N)\)] \(\ggen{x-1 : x \in N} \mathrm{ in }\ \integers G\) where $N$ 
is normal in $G$. This is also \[\left\{\sum_{g \in G} \mu(g)g 
 | \sum_{x \in N} \mu(gx) = 0\,\  \forall  g \in G\right\}.\]
\end{description}

\section{General results}

In this section, I will present a few of the results from Sehgal \cite{bib:sehgal78} in order to give the reader a
feel as to what the unit group of a group ring may become at times.

\begin{proposition}

If $G$ is abelian of order $n$, then \[\mathscr{U}\integers{}G = \pm{}G\times F\] where $F$ is a
free abelian group of a determinable order. This order is dependent on the number of cyclic
subgroups of various orders in $G$.
\end{proposition}

The above theorem helps a great deal when dealing with groups, as it is often possible to get a
factor or sub-group of your group to fall in this particular category. In particular, this
proposition can be used to show that the unit group of \(\integers{}\ggen{x}\) where \(x^2=1\) is just
\(\pm{}\ggen{x}\) This fact is used later.

This proposition along with others can be used to prove the following more powerful theorem.


\begin{theorem}

If $G$ is a torsion group\footnote{See definition \ref{defn:torsionGroup} in the appendix.} then \(\mathscr{U}\integers{}G = \pm{}G\) if and only if $G$ is one of
\begin{enumerate}
\item an abelian group with \(G^4 = \{1\}\) or
\item an abelian group with \(G^6 = \{1\}\) or
\item a Hamiltonian\footnote{See definition \ref{defn:hamiltonianGroup}}
 2-group.
\end{enumerate}
\end{theorem}

This theorem completely characterizes torsion groups that have trivial
\(\mathscr{U}\integers{}G\).

In conclusion, we note that there are many different areas one 
can explore when determining unit
groups.  These range from finding unit groups of particular 
integral group rings to determining when
torsion-free groups have trivial unit groups in their 
integral group ring.

\chapter{Theoretical considerations}
\section{Representation theory method}

This method enables us to give a concrete description of 
the unit groups of certain group rings
involving \integers.  The first two, $S_3$ and $D_4$ 
were done by Hughes and Pearson, and Polcino-Milies,
respectively. The third, which is $D_6$, was done 
by myself by extending the method of Polcino-Milies. 
The last one, $A_4$, which is included only in an 
expository way, was done by Allen and
Hobby. The general manner in which to apply this method is described below.

Consider the group G. We may use representation theory 
to determine its non-equivalent irreducible
representations. Call these $\theta{}_i$. These will be maps 
from \(\rat{}G\) to matrices over 
\rat.

If one takes these \(\theta{}_i\) that we have obtained, one may now define a map \(\theta:
\rat{}G \rightarrow \rat_{i1}\bigoplus\ldots\bigoplus \rat_{in}\), by, if
\(a \in \rat{}G\), then \(\theta(a) = (\theta_1(a),\ldots,\theta_n(a))\). 
Considering both sides of the mapping as vector spaces over \rat, it is readily seen that
$\theta$ is a linear mapping. Let $A$ be the matrix of $\theta$. It will be noticed that
\(\theta\integers \subset \bigoplus\integers_n\). Next, by using the matrix $A$ and its inverse we
will be able to deduce a system of linear congruences that give us the restrictions needed for an
element of \(\bigoplus\integers_n\) to be in \(\integers{}G\). These, together with the fact that a
matrix with coefficients in \integers{} has to have an integral determinant in order to have an
integral inverse determine the proper group.

Naturally one sees that a significant problem with this method is the size of the matrix $A$
involved. It is a square matrix of size \(o(G)\). Another problem is that we have no guarantee that the
system of congruences will lead to a usable situation.

A further difficulty with the method is that the end result may turn out to be a direct product of
matrix rings.  In this case, it is just as difficult to determine properties from this description as
it was from the original.

Despite these difficulties, the method is sufficiently useful enough to apply it to a few groups of
small size.

\section{Second Method - Groups of order $p^3$}

In this section, we intend to study the unit groups of the integral group rings of groups of order
\(p^3\), where $p$ is an odd prime.  The first type, the commutative groups of order \(p^3\), are uninteresting
as their unit groups are just + G.

In dealing with the non-commutative groups of order \(p^3\), we note that there are two non-isomorphic
groups of that order. They are:
\[H=\ggen{a,b \vert a^{p^2}=e=b^p, b^{-1}ab=a^{p+1}} \mathrm{and}\]
\[G=\ggen{a,b,c \vert (a,b) = a^{-1}b^{-1}ab = c, ca=ac, cb=bc, a^p = e = b^p = c^p}.\]

Throughout this section we reserve the letters G and H to mean these two groups.

\subsection{Fibre product}

We must now define a concept that we will be using throughout this section --- that of the fibre
product.

To understand what a fibre product of rings is, consider the ring $R$ with the two ideals $I,J$ of $R$
such that $I\cap{}J = 0$.  Then we have the diagram
\[
\xymatrix{
R \ar[r] \ar[d] &
   R/J \ar[d]^{\tilde{\ }} \\
R/I \ar[r]_{\hat{\ }} &
   R/(I+J)\\
}
\]
%\comdiagxy{R}{R/J}{R/I}{R/(I+J)}

Then $R$ is the fibre product of $I$ and $J$ in the sense that
\[R\cong\{(\alpha,\beta)\vert \alpha{}\in{} R/I, \beta{}\in{} R/J, \hat{\alpha}=\tilde{\beta}\},\]
where $\hat{}$ is the map from $R/I$ to $R/(I+J)$ and $\tilde{}$ is the map from $R/J$ to $R/(I+J)$.

From the above we can deduce a fibre product of the unit groups as is given by the following diagram.
\comdiagxy{\mathscr{U}(R)}{\mathscr{U}(R/J)}{\mathscr{U}(R/I)}{\mathscr{U}(R/(I+J))}

\subsection{Use of Fibre Product}

In this section, we are going to apply this to the Group Ring
\(\integers{}X\) with $J = \Delta(X,N)$ as the kernel of the natural
homomorphism of $\integers{}X\longrightarrow{}\integers{}X/N$ with $N$ normal in $X$ and 
\(I = \hat{N}\integers{}G\) where \(N=\sum_{x\in{}N}x\). From here on in, we shall write $\hat{x}$ 
 for $\ggen{\hat{x}}$.

\subsubsection{Pseudo-diagonals of matrices.}
        
To present the proofs of this section, we will need to number certain matrices by their
pseudo-diagonals. If the $n\times{}n$ matrix $A=[a_{ij}]$ is considered, then the $j$-th
pseudo-diagonal is given by the elements \[a_{1,j+1}, a_{2,j+2}, \ldots, a_{n-1,j-1}, a_{nj}\] where
the second subscript is considered $\mod{n}$ and $j=0,1,2,3,...,n-1$.

Some of the matrices that we will be dealing with from here on will be numbered via their
pseudo-diagonals.  As an example the matrix \(B = [b_{i,j}]\), if numbered by pseudo-diagonals means
that the element $b_{i,j}$ is located on the $i$-th pseudo-diagonal at the $j$-th spot. When using
this indexing scheme, we have $0\leq{}i,j\leq{}n-1$.

For convenience, since we will be dealing with many diagonal-like matrices, we introduce the
following notation: 
\[
A =  \mathrm{PDIAG}_i(\series{0}{n-1})
\] 
will represent the $n\times{}n$ matrix that has 
the elements \(x_0,\ldots,x_{n-1}\) on the $i$-th pseudo-diagonal and
zeroes elsewhere. If we are talking of the $0$-th pseudo-diagonal, we then mean the main diagonal and
refer to it as \[A =  \mathrm{DIAG}(\series{0}{n-1})\] 

\subsection{Preliminary propositions}

Throughout this section we let $\omega$ denote a primitive $p$-th root of unity.

\begin{proposition}
Suppose \(\series{0}{p-1}\ \in\ \integers[\omega]\) then there exists \(t_i\ \in\
\integers[\omega]\) satisfying
\[\sum^{p-1}_{i=0} t_i\omega^{ij}=x_j, 0\leq{}j\leq{}p-1\] if and only if
\[\sum^{p-1}_{i=0}x_i\omega^{ki} \in{} p \integers[\omega]\ \forall\ 0\leq{}k\leq{p-1}\]
\end{proposition}

\begin{proof}
The system of equations is equivalent to
\[[\series[t]{0}{p-1}]W = [\series{0}{p-1}]\]
where $W= [a_{kl}]$, with \(a_{kl} = \omega^{(k-1)(l-1)}\). Now as $W$ is a character matrix, the
orthogonality relations of a primitive root of unity tell us that\[W^{-1} = \frac{1}{p}[a_{kl}^{-1}]\].

Our system is equivalent to \[[\series[t]{0}{p-1}] = [\series{0}{p-1}]W^{-1}.\]


Therefore, there exists a solution if and only if we have
\[\frac{1}{p}\sum_{i=0}^{p-1}\omega^{ik}x_i \in\ \integers[\omega]\] for all \(0\leq{}k\leq{}p-1\).
This, of course, is the same as saying \[\sum_{i=0}^{p-1}\omega^{ik}x_i \in\
p\integers[\omega]\forall{}\ 0\leq{}k\leq{}p-1\].
\end{proof}

\begin{proposition}

Let $A$ and $B$ be $p \times\ p$ matrices over $\rat[\omega]$, with $A = \mathrm{DIAG}(1, \omega, 
\omega^2, \ldots, \omega^{p-1})$ and $B$ as the matrix with ones
on pseudo-diagonal number one, (where numbering of pseudo-diagonals start at zero) and zeroes
elsewhere, that is $B = \mathrm{PDIAG}_1(1, 1, \ldots, 1)$.

Then the $\integers[\omega]$ span of the matrices $\{A^iB^j, 0\leq{}i,j\leq{}p-1\}$ consists of all elements of the form $M= [x_{ij}]$,
where the numbering is via the pseudo-diagonals, and such that for each $j,k$ with  $0\leq{}i,j\leq{}p-1$
 we have\[\sum_{i=0}^{p-1}\ x_{ji}\omega^{ki}\  \in\ p\integers[\omega]\].
\end{proposition}

\begin{proof}

We see that for a fixed $j$, the matrices $\{A^iB^j, 0\leq{}i,j\leq{}p-1\}$, have non-zero entries only in the $j$-th
psuedo-diagonal. The vector (\series{0}{p-1}) in $\integers[\omega]$ is a diagonal in the span of $\{A^iB^j\}$ if and only
if there exists $t_i\ \in\ \integers[\omega]$ such that
\[\sum_{i=0}^{p-1}\ t_iA^i\ = \ \mathrm{DIAG}(\series{0}{p-1})\]

This, we see, means that\[\sum_{i=0}^{p-1}\ t_i\omega^{ji}\ = \ x_j, 0\leq{}i,j\leq{}p-1 \]
and now applying the last proposition we now have our answer.
\end{proof}

\begin{proposition}

Let $o_1 \subset\ o_2$ be \integers-orders in a rational algebra. Then, if an element $\alpha\ \in\ o_1$
 has  an inverse in $o_2$,
then $\alpha$ already has an inverse in $o_1$.
\end{proposition}
\begin{proof}

As groups, we have that the indices behave in the following manner:
\[(0_2\ :\ \alpha{}o_1)\ =\ (\alpha{}o_2\ :\ \alpha{}o_1)\ \leq{}\  (o_2\ :\ o_1)\]
which implies that $\alpha{}o_1 \ = o_1$, and therefore shows us that $\alpha$ is a unit in $o_1$.
\end{proof}

Now we digress before continuing with our list of propositions.  Let us recall our group $H$ with the
two generators $a$ and $b$. Let us write $c = a^{-1}b^{-1}ab = a^{p^{2}}$.
Then we have $H' = \ggen{c}$ of order $p$. Thus $\widetilde{H} = 
H/\ggen{c}=\ggen{\tilde{a}} \times \ggen{\tilde{b}}$. As before, $\omega$ is a primitive $p$-th 
root of unity.  Then \[\rat{}H \cong{} \rat{}\widetilde{H} \oplus 
\rat{}(\omega)_p\] 
As well as this we have 
\[\rat{}\widetilde{H} \cong{} \rat{}H/\Delta(H,\ggen{c}) \cong{} 
\rat{}H\hat{c} \]and
\[\rat(\omega)_{p} \cong{} \rat{}H/\hat{c}\rat{}H.\]

Clearly this gives us
\[\integers{}H / \Delta{}(H,\ggen{c}) \cong{} \integers{}\widetilde{H} \] and the mapping
\[\integers{}H \rightarrow\integers\widetilde{H} \oplus \integers[\omega{}]_{p} ,\]
with the projection onto the first component. Let us proceed with the calculation of the map with
respect to the second component. We easily see that
\[\tilde{c}\integers{}H +(1-c) \integers{}H = p\integers{}H + (1 - c)\integers{}H\]
as $c$ is the sum of $p$ elements all of which are units of \(\integers{}H\). (They are contained in $+ H$). 
Also, it is obvious that
\[\tilde{c}\integers{}H \cap (1-c) \integers{}H = 0 .\]

Thus, it is obvious that we have the fibre product
\comdiagxy{\integers{}H}{\integers{}\widetilde{H}}%
{\integers{}H/\tilde{c}\integers{}H}{\integers{}\widetilde{H}/p\integers{}\widetilde{H}}
%\comdiagxy{1}{2}{3}{4}
with all the maps being natural.


The $p\times p$ matrices \(A =  \mathrm{PDIAG}_1(1,1,\ldots{},1,\omega{})\) and
$ B = \mathrm{DIAG}(1,\omega{},\omega{}^2,\ldots,\omega{}^{p-1})$ obviously satisfy the
relations $A^p=\omega{}I, B^p = I, B^{-1}AB = A^{p^{2}+1}$.

The matrices $ \{B^jA^i \vert 0\leq{}i,j\leq{}p-1\}$ are linearly independant over 
$\integers{}[\omega{}]$ as is shown in the following: Assume 
\[\sum_{i,j}z_{ij}B^iA^j = 0.\]
As $A^j$ has non-zero entries only  in the $j$-th pseudo-diagonal we have
\[\sum_{i}z_{ij}B^iA^j = 0\, \forall{} j,\ 0\leq{}j\leq{}p-1.\]
As $A$ is a non-singular matrix we have that
\[\sum_{i}z_{ij}B{i} = 0,\]
which immediately implies that $z_{ij} = 0\ \forall 0\leq{}i,j\leq{}p-1$.

Let $T=\integers{}H/\hat{c}\integers{}H$, and $S_p$ be the
$\integers{}[\omega{}]-$span of the matrices $\{B^iA^j\, \vert\, 0\leq{}i,j\leq{}p-1\}$,
from above. The claim here is that $T \cong{} S_p$. Consider the map
\[
\phi{}: \integers{}H \rightarrow{}S_p,\qquad \phi(a) = A,\qquad \phi(b) = B.\]
As $\hat{c} = (1+c+c^2+...+c^{p-1})$ is mapped by $\phi$ to 
$(1+\omega{}+\omega{}^2+...+\omega{}^{p-1})I$, which is zero, we have the induced map
$\phi{}_0: T \rightarrow{} S_p.$


Since $\phi{}(a^{pk}a^ib^j) = \omega{}^kA^iB^j$, we have that $\phi{}$ is onto $S_p$. 
Also we see that $\phi{}_0$ is $1\leftrightarrow{}1$, for if we tensor
both $T$ and $S_p$ with $\rat{}$ we see that both of them have $\integers{}-$dimension
 of $p^n-p^{n-1}$

\begin{proposition}
A matrix $Z \in{} \integers{}[\omega{}]$ is in $S_p$ if and only if the matrix $X = Z'$ satisfies
\begin{equation}
\sum_{i=0}^{p-1} x_{j,i}\omega{}^{ki} \in{} p\iom, 
\forall 0\leq{}j,k\leq{}p-1, \omega^p = 1.
\label{eqn:spcondition}\\
\end{equation}
where $Z'$ is obtained from $Z$ by dividing all the entries below the main diagonal by $\omega{}$.
\end{proposition}
\begin{proof}

We make the observation that $A^i = \mathrm{PDIAG}_i(1,1,\ldots,1,\omega,\ldots,\omega)$, 
where $\omega$ is repeated $i$ times. Thus, to compute $S_p$ it is enough to
find the span of $\{B^jA^i \vert 0\leq{}i,j\leq{}p-1\}$ separately for each $i$. 
Therefore, all we need do is to find all $\integers{}[\omega]$ vectors of the form 
$(z_{i0},\ldots,z_{ip-1})$ such that
\[\sum_{j=0}^{p-1} t_jB_j\mathrm{DIAG}(1,1,\ldots,1,\omega,\ldots,\omega) = \mathrm{DIAG}(z_{i0},\ldots,z_{ip-1})\]
which is equivalent to
\[\sum_{j=0}^{p-1} t_jB_j = \mathrm{DIAG}(z_{i0},\ldots,z_{ip-i-1},\omega^{-1}z_{ip-i},\ldots,\omega^{-1}z_{ip-1})\]
and the result therefore follows from proposition 2.
\end{proof}

Now let us consider a fibre product of $\integers{}H$. We have the diagram
\[
\xymatrix {
\integers{}H \ar[r] \ar[d] 
&\integers\hat{H} \ar[d]\\
T \ar[r]^{\theta{}_1} \ar[d]_{\phi{}_0} 
&(\integers/p\integers)\hat{H}\\
S_p \ar[ur]_{\phi{}_1 = \theta{}_1{\phi{}_0}^{-1}}&
}\]
with all the unablelled maps natural,and $\phi_1$ defined as above. Then the diagram above is
commutative.

Consider $\phi_1$. If $M \in S_p$ we wish to write $M$ as $\sum\alpha_{ij}B^iA^j, 
\alpha_{ij}\in\integers{}[\omega], 0 \le i,j \le p-1$. Let $M'$
be obtained from $M$
by dividing all the elements below the main diagonal by $\omega$.  Then the $j$-th pseudo-diagonal 
$\series[x]{j,0}{j,p-1}$ of $M'$ is the same as the main diagonal of $\sum\alpha_{ij}B^iA^j$.

We then have
\[[\series[\alpha]{0,j}{p-1,j}] W = [\series{j,0}{j,p-1}]\]
where $W=[\omega{}^{ij}], 0\le{}i,j\le{}p-1$. Therefore,
\[\alpha_{i,j} = \frac{1}{p}\sum_{k}\omega^{-ij}x_{i,j}.\]
Recalling that we have
\[M = \sum \alpha_{i,j}B^{i}A^{j}\]
then from the above commutative diagram we have
\[\phi_{1}(M) = \sum\tilde{\alpha}_{i,j}b^{i}a^{j}\]
where $\tilde{\alpha}_{i,j}$  is obtained from $\alpha_{i,j}$ by taking
$\omega = 1$ and going $\mod{p}$.

In consideration of Proposition 3, we have the following theorem.

\subsection{Main result for type 1 groups of order $p^3$}

\begin{theorem}
\begin{enumerate}
\item{}$\integers{}H\cong \{(\alpha,M) \in \integers\widetilde{H} 
\times \integers{}[\omega{}]_p \vert M'$ satisfies Equation \ref{eqn:spcondition} and 
$\theta_2(\alpha) = \phi_1(M)\}.$
\item{}$\mathscr{U}\integers{}H\cong \{(\alpha,M) \in 
\mathscr{U}\integers\widetilde{H} 
\times \integers{}[\omega{}]_p \vert M$ is a unit of $\integers{}[\omega{}]_p, M'$
 satisfies Equation \ref{eqn:spcondition} and 
$\theta_2(\alpha) = \phi_1(M)\}.$
\end{enumerate}
\end{theorem}

In the theorem we have,
\begin{itemize}
\item{} $M'$ is obtained from $M$ by dividing every element below the main diagonal by $\omega$
where $\omega$ is a
primitive $p$-th root of unity.
\item{} Equation  \ref{eqn:spcondition} is the one we have encountered many times already
\[\spcondition
\]
where $\{x_{i,j}\}$ are numbered according to the pseudo-diagonals of $M'$.
\item{}  $\theta_2:\integers\widetilde{H} \rightarrow (\integers/p\integers)
\widetilde{H}$ is  the natural map $\mod{p}$.
\item{}$\phi_1(M) =\sum_{i,j}\tilde{\alpha}_{i,j}\tilde{b}^i\tilde{a}^j$
 where $\alpha_{i,j} = \frac{1}{p}\sum_k\omega^{-ik}x_{jk} 
\in \integers{}[\omega{}]$ and $\tilde{\alpha}_{i,j}$ is
obtained from $\alpha_{i,j}$ by
putting $\omega = 1$ and going $\mod{p}$.
\end{itemize}


\subsection{Groups of type 2 of order $p^3$}
We now consider our second group of order 3 which we refer to as 
$G$. Recall
\begin{equation}
G= \ggen{a,b,c \vert (a,b) = a^{-1}b^{-1}ab = c, 
ca=ac, cb=bc, a^p = e = b^p = c^p}\label{eqn:definitionofg}\\
\end{equation}
We note that the factor commutator group, $\widetilde{G} = G/\ggen{c}$ is elementary
abelian of order $p^2, \widetilde{G} = \ggen{\tilde{a}} \times \ggen{\tilde{b}}$. This gives us the 
decomposition
\[\rat\cong\rat\widetilde{G} \oplus\rat(\omega)_p.\]
In fact, we have
\begin{equation*}
\rat\widetilde{G} \cong \rat{}G/\Delta(G,\ggen{c})\cong
\rat{}G\tilde{c}\ \mathrm{and}\ \\
\rat(\omega)_p\cong\rat{}G/\tilde{c}\rat{}G.\\
\end{equation*}
Clearly this gives us
\begin{equation*}
\integers{}G/\Delta(G,\ggen{c})\cong\integers\widetilde{G}\ 
\mathrm{and \ the\ mapping}\ \\
\integers{}G\rightarrow\integers\widetilde{G}\oplus\integers{}[\omega]_p\\
\end{equation*}
with the mapping being projection onto the first component. Our next step will 
be, as before, to compute the
the projection into the second component. We consider the fibre 
product diagram
\[
\xymatrix {
\integers{}G \ar[r] \ar[d] 
&\integers\widetilde{G} \ar[d]\\
\integers{}G/\hat{c}\integers{}G \ar[r]_{\theta{}_1} 
&(\integers/p\integers)\widetilde{G}\\
}\]
where all the maps are the natural projections excepting the map $\theta_1$ which is
\begin{align*}
\theta_1:\integers{}G/\hat{c}\integers{}G &\rightarrow (\integers/p\integers)\widetilde{G}\ \mathrm{with} \\
\theta_1(\sum{}zc^ia^jb^k) &= \sum\tilde{z}\tilde{a}^j\tilde{b}^k,\ z \in \integers{}.\\
\end{align*}
\subsection{Twisting of Group Rings}
In order to continue, we need to introduce the notion of \emph{twisted }
group rings at this
point. A twisted group ring is constructed in the same manner as 
an ordinary group ring
except that the definition of multiplication differs. In the 
twisted group ring, there is a
twisting factor that is used to commute elements.


Let us take the twisted group ring of a ring $R$ and a group $G$. 
This is then written as
$R\circ{}G$. The twisting may (and quite often is) be multiplication by a 
particular element of R.
Therefore, if $a,b \in G, r \in R$, we may define the 
multiplication of $a$ and $b$ in the
twisted group ring as $ab = rba$.

Using the notation we have just introduced, it is easy to see 
that $\integers{}G/\hat{c}\integers{}G$ is
isomorphic as a ring to the twisted group ring $\integers{}[\omega]\circ\widetilde{G}$ with
$ \tilde{b}\tilde{a} = \omega\tilde{a}\tilde{b}$. After this
identification, we see that the map $\theta_1$ may be written as
\[\theta_1(\sum\alpha\tilde{a}^i\tilde{b}^j) = \sum\tilde{\alpha}\tilde{a}^i\tilde{}b^j, \alpha \in \integers{}[\omega].\]
where we get $\tilde{\alpha}$ from $\alpha$ by substituting $\omega = 1$ and going $\mod{p}$.

Let us now define the map $\phi_0$ from $\integers{}[\omega]\circ\widetilde{G}$
to $\integers{}[\omega]_p$ by
\begin{equation*}
\tilde{a}\rightarrow A = \mathrm{DIAG}(l,\omega,\ldots,\omega^{p-1}),\\
\tilde{b} \rightarrow B = \mathrm{PDIAG}_1(l,\ldots,1).\\
\end{equation*}
We note that $BA = \omega{}AB$ and that $A^p = I = B^p$.

We claim that $\{A^iB^j\}$ is a linearly independent set over $\integers{}[\omega]$
for $0\le{}i,j\le{}p-1$. This
can easily be seen as follows. Let $\alpha_{i,j}$ be in $\integers{}[\omega]$ for 
$0\le{}i,j\le{}p-1$. Then 
we see that % Is this B sub j or B sup j?
\[\sum_i\sum_j\alpha_{i,j}A^iB^j = 0 \Longrightarrow \sum_i\alpha_{i,j}A^iB^j = 0,\]
as $B^j$ has non-zero entries only in the $j$-th diagonal. Since $B$ 
is non-singular, we may remove
the $B^j$ from the above result, which implies that   $\alpha_{i,j}= 0$ for all 
$0\le{}i,j\le{}p-1$. Therefore, we now
have that if $S_p = \mathrm{span}\{A^iB^j, 0\le{}i,j\le{}p-1\},$ then
\[\integers{}G/\hat{c}\integers{}G \cong \integers{}[\omega]\circ\widetilde{G}
\cong S_p.\]
From proposition 2 it follows that
\[S_p = \{M\in\iom_p \mid M \mathrm{satisfies }\ \sum_{i=0}^{p-1}
x_j,i\omega^{ki} \in p\iom\ \forall\, 0\le i,j\le p-l\}.
\]

Let us consider the following fibre product diagram and extension 
to $S_p$.
\[
\xymatrix{
\intG \ar[rr] \ar[d] &&\intGtil \ar[d]\\
\iom\circ\widetilde{G}\cong\intG/\hat{c}\intG \ar[rr]_{\theta_1} \ar[d]_{\phi_0}&&
    (\integers/p\integers)\widetilde{G}\\
S_p \ar[urr]_{\phi_1}\\
}
\]

In the above, the map from $S_p$ to $(\integers/p\integers)\widetilde{G}$ is denoted by $\phi_1$ and the 
map from $\intG/\hat{c}\intG$ to
$S_p$ is denoted by $\phi_0$. Obviously, as before in a similar diagram, 
we define $\phi_1$ to be the map
$\theta_1\phi_0^{-1}$.


As before, given M in $S_p$ we wish to find $\phi_0^{-1}(M)\in \iom
\circ \widetilde{G}$. Let $M = \{x^{j,i}\}$ be
numbered by the pseudo-diagonals. We wish to find $a_i,j\in \iom$ suc that 
$\sum_{i,j}a_{i,j}A^iB^j = M$. It is
necessary to find $a_{i,j}$ such that
\[
\sum_ia_{i,j}A^i=\mathrm{DIAG}(\series{j,0}{j,p-1}),\ 0\le j\le p-1.
\]

Again as before we get this is equivalent to the matrix equation
\[
[\series[a]{0,j}{p-1,j}]W = [\series{j,0}{j,p-1}]
\]
where $W = [w_{i,j}]$ is numbered by columns and rows starting at zero and 
$w_{i,j} = \omega^{ij}$. Again, by
following the previous procedure, we see that
\[
a_{i,j} = (\frac{1}{p})\sum_{k=0}^{p-1}\omega^{-ki}x_{j,k}.
\]

From the above we have that   
\[
\phi_0^{-1}(M) = \sum_{i,j}a_{i,j}\tilde{a}^i\tilde{b}^j \quad
\mathrm{and}\quad
\phi_1(M) = \sum_{i,j}\tilde{a}_{i,j}\tilde{a}^i\tilde{b}^j
\]
where $\tilde{a}_{i,j}$ is obtained from $a_{i,j}$ by substituting $\omega = 1$
 and going 
$\mod p$.


        The above constitutes the first part of the next theorem. The 
second part follows
directly from proposition 3 as $S_p$ is an order in $\iom_p$.



\subsection{Main result for type 2 groups of order $p^3$}
\begin{theorem}~\\
\begin{enumerate}
\item{}\( \intG \cong \{(\alpha,M) \in \intGtil\times\iom_p \vert M'\) satisfies
condition \ref{eqn:spcondition} and \(\theta_2(\alpha)=\phi_1(M)\}.\)
\item \(\ug\intG \cong \{(\alpha,M) \in \ug\intGtil\times\iom_p \vert M\) is
a unit of \(\iom_p, M\) satisfies  \ref{eqn:spcondition} and 
 \(\theta_2(\alpha)=\phi_1(M)\}.\)
\end{enumerate}
\end{theorem}

In the theorem we have,
\begin{enumerate}
\item \( \theta_2:\intGtil\rightarrow (\integers/p\integers)\widetilde{G}\) is the natural map mod $p$;

\item The condition  \ref{eqn:spcondition} is the one we have encountered many times 
\begin{equation*}
\spcondition
\end{equation*}
        where  \(\{x_{j,i}\}\) are numbered according to the pseudo-diagonals of $M$.

\item \(\phi_1(M) = \sum_{i,j}\tilde{a}_{i,j}\tilde{a}^i\tilde{b}^j\)
 where \( a_{i,j} = \frac{1}{p}\sum_k\omega^{-ki}x_{j,k} \in \iom \)
and \tildeaij is obtained from \aij by putting $\omega=1$ and going $\mod{p}$.
\end{enumerate}


\subsection{Concluding remarks on groups of order $p^3$}
        This concludes our section on groups of order $p^3$. It should be 
noted that there are
striking similarities in the derivations for both types of 
groups. However, the proofs do need
to be presented separately due to the underlying differences. In 
a paper by Dr. S. Sehgal and
Dr. J\"urgen Ritter, \cite{bib:ritter1981} the method presented here for type 1 groups 
of order \(p^3\) is extended to
similar type groups of order $p^n$.


        At the end of this paper, I will present examples of the methods 
shown in this section,
using groups of order 27.



\section{Third method - Groups of order $pq$}\label{sec:thirdmethod}
\subsection{Generalities}\label{sec:thirdmethod:generalities}
        In this section, we shall study the unit group of groups of order 
$pq$, where 
$p$ and $q$ are
primes $p\equiv 1 (\mod{q})$. We shall restrict our attention to the 
non-abelian group of this order. In
order to refresh your memory, we have
\begin{equation}
 G = \ggen{a,b\ \vert\ a^p = b^q, bab^{-1} = a^j \notequiv{}\ 1 \mod{p},\ 
 j^q \equiv   1 \mod{p}}.
\end{equation}

In this case, we will need to consider the set of normalized 
units of \intG{} given by the map:
\[
\sigma:\ug{\intG} \to \ug{\integers{}[b]}
\]
where
\[
\sigma(a) = 1, \ \sigma(b)=b.
\]


        The kernel of this homomorphism is obviously those elements of
\ug{\intG} 
for which the
coefficient of $a^k$ is 1 and the power of b is 0. Let us denote the 
kernel by $\mathscr{N}$. This set shall be
the group of {\it{normalized}} units of \intG. We note that any unit of \intG{}
can be written as the
product of a normalized unit and a unit of \intb.


        Now since the units of \intb{} are known to us and we have the 
equation
\[        ba = a^jb\]
we only need to determine the set of normalized units.


        Let $\omega$ be a primitive $p$-th root of unity. Then the 
field $k = \rat(\omega)$  
is a cyclic extension
of \rat{} of order $p-1$. Let $k_0$ be the fixed field of the
 automorphism $t:\omega \to \omega^j$. Then $k_0$ is of degree
$(p-1)/q$ over \rat. For any element $\alpha$ of $k$ let
\[\alpha^t, \alpha^{(2)t}, \ldots , \alpha^{(q-1)t}\]
be the succesive applications of the automorphism $t$.


        Let $R$ and $R_0$ denote the rings of integers of $k$ and $k_0$, 
respectively. We note that $R$ is
a free $R_0$-module with the basis\[1,\chi,\supseries[\chi]{2}{q-1}\]
where
\[\chi=\omega-1\]
is the prime    in $R$ over the rational prime $p$. The corresponding 
prime in $R_0$ is
\[X_0 = (\omega-1)(\omega^2-1),\ldots,(\omega^q-1).\]
As $(\omega^i-1)/(\omega-1)$ is a unit, we have that $\chi_0 = \chi^q$ as ideals.


Recalling from number theory that in the above situation we have that
\[\integers/p\integers\cong{}R_0/\chi_0R_0\cong  R/\chi{} R.\]
Therefore, in particular, we have that each element of $R$ modulo $\chi$ 
is congruent to a rational
integer.


        At this point, we present a lemma.
\begin{lemma}
Suppose $\alpha\in{}R$, with $\alpha\equiv{}s\pmod{x}$, with 
$s\in\integers$. Then $\alpha$ can be written 
uniquely as
\[\alpha = a_0+a_1\omega+\cdots+a_{p-1}\omega^{p-1}\]
with $\sum{}a_i = s$, and the $a_i\in\integers$.
\end{lemma}

\begin{proof}
Write \[\alpha =  c0+c_1\omega+c_2\omega^2+\cdots+c_{p-1}\omega^{p-1}\] 
where the $c_i$ are rational integers. Since $a\equiv  S \pmod{\chi}$, we 
therefore have
\begin{equation}
       c0+c_1+c_2+\cdots+c_{p-1} \equiv S \pmod{p}.\label{eqn:cisequivSmodp}
\end{equation}

In full generality the first equation may be rewritten as\[
\alpha = (c_0+m)+(c_1+m)\omega+(c_2+m)\omega^2+\cdots+(c_{p-1}+m)\omega^{p-1}
\]where m is a rational integer. The sum of the new coefficients is 
$s$ if and only if\[ c0+c_1+c_2+\cdots+c_{p-1}+pm = s.\]
Considering \ref{eqn:cisequivSmodp} above, the relation gives us a unique value for m 
such that the equality does
hold.
\end{proof}
\subsection{The unit group as matrices over $R$}.

        Recall that it is obvious that $\intG\cong\intab$ and, therefore, 
an element $x$ of \intG{} can
be written as\[x(a,b) = x_0(a) + x_1(a)b + \cdots + x_{q-1}(a)b^{q-1}\]
where $x_i(a)$ is an element of \inta. Therefore, if $x,y,z \in \intG$, 
written as above, and $z=x\cdot y$, then we would have (upon recalling the definition of 
multiplication in a group ring)
the equations
\begin{align*}
z_0(a) &=& x_0(a)y_0(a) + x_1(a)y_{q-1}(a^j) +\cdots+ x_{q-1}(a)y_1(a^{j^{q-1}})\\
z1(a) &=& x_0(a)y_1 (a) + x_1(a)y_0(a^j) +\cdots + x_{q-1}(a)y_2(a^{j^{q-1}})\\
&\vdots&\\
z_{q-1} (a) &=& x_0(a)y_{q-1}(a) + x_1(a)y_{q-2}(a^j) + \cdots + x_{q-1}(a)y_0(a^{j^{q-1}})
\end{align*}

 Recalling that$\inta \cong \intgrgen{X}/X^{p-1}$ we can therefore associate 
to an
element $x(a,b) \in\intG$ the 
elements\[\alpha_0 = x_0(\omega), \alpha_1 = x_1(\omega),\ldots,
 \alpha_{q-1} = x_{q-1}(\omega),\]
with $\alpha_i \in R$.


        Consider the matrix
\[
A = 
\begin{bmatrix}
\alpha_0 &\alpha_1 &\cdots&\alpha_{q-1}\\
\alpha_{q-1}^t &\alpha_0^t &\cdots&\alpha_{q-2}^t\\
&&\vdots& \\
 \alpha_1^{(q-1)t} &\alpha_2^{(q-1)t} &\cdots&\alpha_0^{(q-1)t}
\end{bmatrix}
\]

with entries in $R$. We shall call matrices with this form matrices 
of type 1.


        From our previous calculations, we see that the obvious map 
$x(a,b)\to A$ is a
homomorphism from \intG{} into the matrices of type 1.


        Let us consider what happens when $A$ is invertible in $R$. Denoting 
the first row of $A^{-1}$
by \series[\beta]{0}{q-1} and we would then have the system of equations
\begin{align*}
\beta_0\alpha_0 + \beta_1\alpha^t_{q-1}+\cdots+\beta_{q-1}\alpha_1^{(q-1)t}&=&1\\
\beta_0\alpha_1 + \beta_1\alpha^t_0 + \cdots+\beta_{q-1}\alpha_2^{(q-1)t}&=&0\\
&\vdots&\\
\beta_0\alpha_{q-1} + \beta_1\alpha^t_{q-2} + \cdots+\beta_{q-1}\alpha_0^{(q-1)t}&=&0
\end{align*}
If we apply the automorphism $t:\omega\to\omega^j$ to these relations 
successively we see that
\[
A^{-1} = 
\begin{bmatrix}
\beta_0 &\beta_1 &\cdots&\beta_{q-1}\\
\beta_{q-1}^t &\beta_0^t &\cdots&\beta_{q-2}^t\\
&&\vdots& \\
 \beta_1^{(q-1)t} &\beta_2^{(q-1)t} &\cdots&\beta_0^{(q-1)t}
\end{bmatrix}
\]
is once again of the same type. Therefore, our conclusion is that 
the invertible matrices of the
type 1 form a group.


        Let us restrict the homomorphism from \intG{} to type 1 matrices down 
to $\mathscr{N}$, the group
of normalized units of \intG{}. Let us further restrict it to 
matrices of type 1 that satisfy the
following conditions:
\begin{align}
&\alpha_0\equiv1,\alpha_1\equiv0,\cdots,\alpha_{q-1}\equiv0 \pmod{\chi}\mathrm{ or }
A\equiv I \pmod{\chi}\label{eqn:cond1fortype1}\\
&\det{A}\mathrm{\ is\ a\ unit\ in\ }R_0\label{eqn:cond2fortype1}.
\end{align}

We conjecture that the homomorphism is actually an isomorphism. 
First, we show
that it is one-to-one. Let $x(a,b)$ be in $\mathscr{N}$, with $x(a,b)$ being 
mapped to the identity. 
Then\[x_0(\omega)=1, x_1(\omega)=0,\cdots,x_{q-1}(\omega)=0.\]
Since $x(a,b)$ is a normalized unit we also have
\[x_0(1)=1, x_1(1)=0,\cdots,x_{q-1}(1)=0\]
which means that\[
x_0(a)=1, x1(a)=0,\cdots,x_{q-1}(a)=0.\]
That shows that the map is one-to-one.

Now we turn our attention to the matter of onto. Let $A$ be 
invertible of type 1 and
satisfy conditions \ref{eqn:cond1fortype1} and 
\ref{eqn:cond2fortype1}. Let $B$ be the inverse of $A$. Let
\[\alpha_0 = x_0(\omega), \alpha_1 = x1(\omega),\cdots,
\alpha_{q-1} = x_{q-1}(\omega).\]
where the $x_i(X)$ are polynomials with rational integer 
coefficients of degree $\le{}p-1$, the sum of
the coefficients of $x_i=1$ or $=0$, corresponding to $i=0$, or 
$i=1,2,\cdots,q-1$. Form the element
\[        x(a,b) = x_0(a) + x1(a)b + \cdots + x_{q-1}(a)b^{q-1}\]
and the corresponding element for $B$
\[     y(a,b) =   y_0(a) + y_1(a)b +\cdots+ y_{q-1}(a)b^{q-1}\]
which is        derived in a similiar manner.

Since $AB = I$ we have
\[\sum_{l+m\equiv i\pmod{p}}x_l(\omega)y_m(\omega) = 1 \mathrm{\ or\ } 0\]
according as to $i=0$ or $i=1,2,3,\cdots,q-1$.
We also see that
\[\sum_{l+m\equiv i\pmod{p}}x_l(1)y_m(1) = 1 \mathrm{\ or\ } 0\]
according as to $i=0$ or $i=1,2,3,\cdots,q-1$. Therefore, $x(a,b)y(a,b) = 
1$. In the same manner as
above we see that $y(a,b)x(a,b) = 1$.


        Therefore, we have proven that the group $\mathscr{N}$
 of normalized units of 
\intG{} is isomorphic
to the type 1 matrices in $R$ that satisfy conditions  \ref{eqn:cond1fortype1} and 
\ref{eqn:cond2fortype1}.

\subsection{The unit group as matrices over $R_0$}
        In this section, we continue from where we left off at the end of 
the last section and
extend our description of $\mathscr{N}$.

  Let us put
\begin{align*}
\delta&=&(X-\chi^t)(X-\chi^{2t}) \cdots (X-\chi^{(q-1)t})\\
&=&X^{q-1}+\delta_1X^{q-2}+\cdots+\delta_{q-1}.
\end{align*}
Also let\[\delta=\delta{\chi}.\]
Since we obviously have $X-\chi \equiv X\pmod{\chi}$, we have that
\[N_{k/k_0}(X-\chi) \equiv X \pmod{\chi_0},\]
and, therefore, if we compare coefficients we will have
\begin{equation}
\delta_i\equiv \chi^i\pmod{\chi_0}.
\label{eqn:deltaiequiv}
\end{equation}

If one lets $\delta_0 = 1$ then let
\[
P= 
\begin{bmatrix}
1&\chi^1 &\cdots&\chi^{q-1}\\
1&\chi^t &\cdots&(\chi^t)^{q-2}\\
&&\vdots& \\
1& \chi^{(q-1)t} &\cdots&(\chi^{(q-1)t})^{q-1}
\end{bmatrix}
\]
With some work, we can see that $P^{-1}$ is $[p_{i,j}], 1\le{} i,j \le{}q$
where the numbering 
is by rows and
columns and $p_{i,j} = (\delta_{q-i}/\delta)^{(i-1)t}$, where $a^{0t}$
is, of course, just $a$.

        Let $E = \mathrm{PDIAG}_1(1,1,\cdots,1)$ be a $q \times q$
 matrix. It is quite obvious that
$E^{-1} = \mathrm{PDIAG}_{q-1}(1,1,\cdots,1) =  E^{q-1}$. 
For a matrix $M$ with entries in $k$, we shall denote 
by $M'$ the
matrix obtained from $M$ by applying the automorphism $t$ to the 
entries of $M$. In consideration
of this, it is obvious that $P' = EP$, and $(P^{-1})' = P^{-1}E^{-1}$. 
Therefore if $A$ has its entries in $k$, then
$P^{-1}AP$ has entries in $k_0$ if and only if
$A' =  E A E^{-1}$.

This is equivalent to the matrix $A$ being a type 1 matrix from the 
previous section. That is
\[
A = 
\begin{bmatrix}
\alpha_0 &\alpha_1 &\cdots&\alpha_{q-1}\\
\alpha_{q-1}^t &\alpha_0^t &\cdots&\alpha_{q-2}^t\\
&&\vdots& \\
 \alpha_1^{(q-1)t} &\alpha_2^{(q-1)t} &\cdots&\alpha_0^{(q-1)t}
\end{bmatrix}
\]
where the elements are in $k$.


It then follows that the map from $A$ to $X = P^{-1} A P$ is an 
isomorphism of the ring of
matrices $A$ of type 1 having entries in $k$ with the ring of $q \times q$ 
matrices $X$ with entries in $k_0$.


Let $X =  [x_{i,j}], 0\le i,j\le q-1$. Suppose that $X$ with entries in $R_0$ 
satisfies the congruence
\begin{equation}
 X\equiv 
\begin{bmatrix}
1&0\\
*&1
\end{bmatrix}
\pmod{\chi_0(=\chi^{q-1})}.
\label{eqn:Xissubdiagonal}
\end{equation}



Then the entries in the first row of the corresponding matrix 
$A = P X P^{-1}$ are
\begin{align*}
\alpha_0 &=& \frac{1}{\delta}\left( x_0(\chi)(\delta_{q-1}) + x_1(\chi)(\delta_{q-2}) 
+\cdots + x_{q-1}(\chi)(\delta_0)\right) =\frac{\beta_0}{\delta} \\
\alpha_1 &=&  \frac{1}{\delta^t}\left( x_0(\chi)(\delta_{q-1})^t + 
x_1(\chi)(\delta_{q-2})^t 
+\cdots + x_{q-1}(\chi)(\delta_0)^t\right)=\frac{\beta_1}{\delta^t} \\
&&\vdots \\
\alpha_{q-1}&=& \frac{1}{\delta^{(q-1)t}}\left( x_0(\chi)(\delta_{q-1})^{(q-1)t} +
% x_1(\chi)(\delta_{q-2})^{(q-1)t} +
\cdots + x_{q-1}(\chi)(\delta_0)^{(q-1)t}\right)=\frac{\beta_{q-1}}{\delta^{(q-1)t}}
\end{align*} 
where for \boundsq{i},
\[
x_i(\chi) = x_{0,i} + x_{1,i}\chi +\cdots+ x_{q-1,1}\chi^{q-1}\equiv 
\chi^i\pmod{\chi^{i+1}},
\]

This implies, if we consider the congruences involving the $\delta_i$'s
and the congruences that one
can derive by successive applications of the automorphism $t$ to 
them we have\[\beta_0\equiv q\chi^{q-1}\pmod{\chi}\]
and
\[\beta_i\equiv (\chi^{it})^{q-1}+\chi(\chi^{it})^{q-2}+
\cdots+\chi^{q-2}(\chi^{it})^{q-1}+\chi^{q-1}\pmod{\chi}\]

        Since $\delta$ and its conjugates via the isomorphism $t$ are all 
associates of $\chi^{q-1}$ this means
that the $\alpha_i$ are all elements of $R$. As well, $\pmod{\chi}$ we will have 
the following hold true:
\begin{align*}
\frac{\delta}{\chi^{q-1}} &=&\left(1=(\chi^t/\chi)\right) 
\left(1=(\chi^{2t}/\chi)\right)\cdots \left(1=(\chi^{(q-1)t}/\chi)\right)\\
&\equiv &(j-1)(j^2-1)\cdots(j^{q-1}-1)\\
&\equiv &q
\end{align*}

Therefore, we have that\[  \alpha_0 = \frac{\beta_0}{\delta_0}\equiv 1 \pmod{\chi}.\]

Since
\[(\frac{\chi^t}{\chi})^q-1 = \left(\frac{\omega^j-1}{\omega-1}\right)^q-1 \equiv
j^q - 1 \equiv 0 \pmod{\chi},\]
therefore, $(\chi^t)^q-\chi^q\equiv 0 \pmod{\chi^{q+1}}$, and therefore, 
noticing that 
$\chi^t-\chi$ is an associate of $\chi$, we have
\[\beta_1\equiv\left(\frac{(\chi^t)^q-\chi^q}{\chi^t-\chi}\right)
\equiv 0 \pmod{\chi^q}\]
giving us the relation
\[        \alpha_1 =\frac{\beta_1}{\delta^t}\equiv 0\pmod{\chi}.\]

Similarly, we can continue to show that $\alpha_i\equiv0 \pmod{\chi}$ for 
$2\le i \le q-1$.


        Now let us do the converse. Suppose we have a matrix $A$ of type 1 
with entries in $R$
that satisfy the conditions
\[\alpha_i\equiv 0\mathrm{\ or\ }1 \pmod{\chi}\]
depending as $i=0$ or $i=1,2,\cdots,q-1$. Let $X = P^{-1} A P$. Number the 
matrix $X$ as above, $X=[x_{i,j}]$ for \boundsq{i,j},
the numbering according to columns and rows. In 
consideration of the above, we
note that
\begin{align*}
x_{i,j}&=&\sum_{u,v}\left((\frac{\delta_{q-i-1}}{\delta})^{(u)t}
(\alpha_{q-u+v})^{(u)t}(\chi^j)^{(v)t}\right)\\
&=&\sum_u\left(\sum_v(\frac{\delta_{q-i-1}}{\delta})
(\alpha_{q-u+v})(\chi^j)^{(v-u)t}\right)^{(u)t} \\
&=&\sum_u\left(\sum_v(\frac{\delta_{q-i-1}}{\delta})
(\alpha_{v})(\chi^j)^{(v)t}\right)^{(u)t} \\
&=&\mathrm{Tr}\left((\frac{\delta_{q-i-1}}{\delta})
(\sum_v\alpha_{v}(\chi^j)^{(v)t})\right),
\end{align*}

where Tr is the trace of $k$ over $k_0$. Since $\delta$ is the different of 
the extension $k$ over $k_0$, we have
that $x_{i,j}$ is in $R_0$. As well, in view of the congruences involving 
the $\delta_i$ and the $\alpha_i$, we would have
that if $j\ge i$,
\begin{align*}
x_{i,j}&=&\mathrm{Tr}\left(\left(\frac{\chi^{q-i-1}}{\delta}\right)
\sum_v\alpha_v(\chi^j)^{(v)t}\right)\\
 &=&\mathrm{Tr}\left(\frac{\chi_{q-i-1}}{\delta}\alpha_0\chi^j\right)\\
&=&\mathrm{Tr}\left(\frac{\chi_{q-1+j-i}}{\delta}\right)\\
&\equiv& 1 \mathrm{\ or \ } 0 \pmod{\chi_0}
\end{align*}
according to whether $i =j$ or $i < j$. This means that the matrix $X$ 
has entries in $R_0$ satisfying
the conditions in equivalence \ref{eqn:Xissubdiagonal}. 
Therefore, we have proven the following 
theorem.

\begin{theorem}
The group of normalized units $\mathscr{N}$ of \intG{}
 is isomorphic to the group 
of $q \times q$ matrices $X$ in $R_0$
that are invertible in $R_0$ and satisfying the congruences
\begin{equation*}
 X\equiv 
\begin{bmatrix}
1&0\\
*&1
\end{bmatrix}
\pmod{\chi_0(=\chi^{q-1})}.
\end{equation*}
\end{theorem}

        This concludes our section on the groups of order $pq$. The rest of 
the thesis deals with
concrete examples.
\chapter{Applications of Representation Method.}


In this section, we shall use the first method described in this 
paper to describe the unit
groups of \st, \df, \ds, and, as well, as give a brief expository 
account of the method findings
for \af.



\section{The Unit Group of \intsth.}
        The first thing we do is consider a map deduced from a 
representation of \st{} given by
\begin{align*}
\theta(1 2) &=& \begin{pmatrix}
1&-1&
\begin{pmatrix}
1&-1\\
0&-1
\end{pmatrix}
\end{pmatrix}\\
\theta(1 2 3) &=& \begin{pmatrix}
1&1&
\begin{pmatrix}
0&-1\\
1&-1
\end{pmatrix}
\end{pmatrix}
\end{align*}
As one can see from the above this gives rise to a map
\[\theta:\rat~\st\to\rat\oplus\rat\oplus\rat_2.\]

We define the map by linear extension using the convention that 
cycles multiply from the
right to the left (for example $(1 2)(1 2 3) = (2 3)$). Since we 
have defined $\theta$ by linear extension
we see that this map is a homomorphism.


        Let $B_1 = \{e, (1 2), (2 3), (1 3), (1 2 3), (1 3 2)\}$ be a basis of 
\rat\st{} and let $\alpha=(\alpha_1,\series[\alpha]{2}{6})$ be an element of \rat\st{}
 with respect to the basis $B_1$. 
Similarly let $B_2$ be the canonical
basis for $\rat\oplus\rat\oplus\rat_2$ with $X = (x_1,x_2,\cdots,x_6)$
 being an element of that 
space with respect to $B_2$.

        Then we may consider $x=\theta\alpha=\alpha A$ where
\[A=
\begin{bmatrix}
1&1&1&0&0&1\\
1&-1&1&-1&0&-1\\
1&-1&-1&0&-1&1\\
1&-1&0&1&1&0\\
1&1&0&-1&1&-1\\
1&1&-1&1&-1&0
\end{bmatrix}
\]
and upon further calculation we see that
\begin{gather*}
A^{-1}=\frac{1}{6}
\begin{bmatrix}
1&1&1&1&1&1\\
        1& -1&    -1   &   -1&      1  &1\\
                2 & 2  &  -2     & 0& -2   & 0\\
	0&  0 &   -2    &  2& -2  &  2\\
                0& -2&    0&       2&2&      -2\\
                2& -2  &  2     &0&0& 2
\end{bmatrix}
\end{gather*}
        As $A$ is invertible, it is readily seen that $\theta$ is an isomorphism. 
Moreover from the
elements of $A$ it is readily seen that
\[
\theta\intgr{\st} \subset\integers\oplus\integers\oplus\integers_2.\]

 Furthermore, if we consider $A^{-1}$, then for $x_i\in\intgr{\st}$, we have
\begin{gather*}
\theta^{-1}x\in\intgr{\st}\iff\\
\begin{matrix}
&x_1&+x_2&+2x_3&&&+2x_6&\equiv0\pmod{6}\\
&x_1  &-x_2&+2x_3& &-2x_5  &-2x_6&\equiv0\pmod{6}\\
&x_1  &-x_2  &-2x_3  &-2x_4&&+2x_6&\equiv0\pmod{6}\\
&x_1  &-x_2&&+2x_4&+2x_5&&\equiv0\pmod{6}\\
&x_1&+x_2  &-2x_3  &-2x_4&+2x_5&&\equiv 0\pmod{6}\\
&x_1&+x_2 &&+2x_4  &-2x_5 &-2x_6&\equiv0\pmod{6}
\end{matrix}
\end{gather*}
After simple row reduction we see that this reduces to the 
following set of three equations.
\[
\begin{matrix}
&x_1&+x_2&&+2x_4&+4x_5&+4x_6&\equiv0\pmod{6}\\
      &  &4x_2&&     &+4x_5&+2x_6&\equiv0\pmod{6}\\
&     &           &4x_3 &+2x_4&+4x_5 &+2x_6&\equiv 0\pmod{6}
\end{matrix}
\]

The second and third reduce respectively to:
\[
\begin{matrix}
&x_2&&\equiv &x_6 &-x_5\pmod{3}\\
&x_4&+x_5&\equiv&x_3&+x_5\pmod{3}
\end{matrix}
\]
Inspection of the       first equation shows us that
\[x_1+x_2\equiv0\pmod{2}\]
If we also consider our first equation as an equation modulo $3$ 
and combine it with the result
of the second equation we get
\[
x_1\equiv x_4+x_6\pmod{3}.\]

Therefore, the final result is as follows:
\[
\begin{matrix}
&x_1&+x_2&\equiv&0&&\pmod{2}\\
&&x_2&\equiv&x_6 - &x_5&\pmod{3}\\
&x_1&&\equiv&x_3+&x_5&\pmod{3}\\
&&&\equiv&x_4+&x_6&\pmod{3}
\end{matrix}
\]

 Keeping all this in mind, we next consider the projection operator 
$\phi:\rat\oplus\rat\oplus\rat_2\to\rat_2$. Then we can see 
that
\[\phi\theta\intsth = \left\{
\begin{pmatrix}
x_3&x_4\\
x_5&x_6
\end{pmatrix} \left| x_3+x_5 \equiv x_4+x_6\pmod{6}\right. \right\}\]


Let us call this space $\Y$.

                Let $X=(x1,\cdots,x6)\in\theta\intsth$, with
$ x_6\equiv x_3+x_5-x_4\pmod{3}$. Let $\delta=x_3x_6-x_4x_5$.
Consider
\[(x_3-x_4)(x_3+x_5) = x_3(x_3+x_5-x_4)-x_4x_5 \equiv
 x_3x_6-x_4x_5\pmod{3} \equiv\delta\pmod{3} \]


In consideration of our row reduced equations it follows that 
$X^{-1}$ exists and is in $\theta\intsth{} \iff
x_3x_6-x_4x_5=\delta=\pm 1, x_1 = 1, x_2 = \delta x_1$.

We see by composition of maps, that $\phi\theta$ is a homomorphism of
\intsth{}
into $\Y$, and
therefore induces a homomorphisim of the unit group of \intsth{} into 
the unit group of $\Y$. We
will now show that this induced homomorphism is one to one and 
onto, proving that it is an
isomorphism.

        Let
\[Z=
\begin{pmatrix}
x_3&x_4 \\
x_5&x_6
\end{pmatrix} \in \ugrp{\Y}.
\]
Also let $\delta=x_3x_6-x_4x_5= \pm 1$ and if $x_1,x_2 \in \{-1,0,1\}$
 are defined by the congruences
\begin{align*}
x_2 &\equiv x_6-x_5 \pmod{3}\ \textrm{ and}\\
x_1 &\equiv x_3+x_5 \pmod{3}
\end{align*}
then neither $x_1$ nor $x_2$ is zero and all the above conditions are 
satisified. Therefore $\alpha=\theta^{-1}X$ is a
unit in \intsth{} with $\phi\theta\alpha=U$. Therefore, by the 
above we have that $\phi\theta$
is one to one and onto,
therefore we have

\begin{theorem}
The unit group of \intsth{} is isomorphic to:
\[\left\{
\begin{pmatrix}
a&b\\
c&d
\end{pmatrix} \in \ugrp{\integers_2} \left| a+c \equiv b+d \pmod{3} \right. \right\}
\]
\end{theorem}


\section{The units of \intdfr}

In this section, we will determine the group of units of \intdfr,
where \dfr{} is the dihedral
group of order 8. This group is determined by the generators $a,b$ 
together with the relations
\[a^4=b^2=baba= 1.\]
Before proceeding with the construction, we give a few 
definitions. The homomorphism
$\xi:\intG\to\integers$ with $\xi(g) = 1$ for all $g\in G$
 is called the \emph{augmentation function}. 
Denote by $V(\intG)$ the
normal subgroup of the units $u \in \intG$  such that $\xi(u) = 1$.
 If $u$ is 
in $V(\intG)$, it is called a
\emph{normalized unit}. Finally an automorphism $\theta$ of \intG{}
 is said to be 
normalized if $\xi\theta(g) = 1$ for all $g \in G$.

Now, as before, we determine our map from $\rat{}\dfr{}$
 to a direct sum of 
matrix rings over \rat. In this case, the map will be
\[
\theta:\rat \dfr{} \to \rat\oplus\rat\oplus\rat\oplus\rat\oplus\rat_2
\]
given by
\begin{align*}
\theta(a) &=\left(1,1,-1,-1,
\begin{pmatrix}
0 &-1\\
1&0
\end{pmatrix} \right) \ \textrm{ and} \\
\theta(b) &=\left(1,-1,1,-1,
\begin{pmatrix}
0 &1\\
1&0
\end{pmatrix} \right) 
\end{align*}

In the same manner as before, consider $\dfr{}$ as the basis of
\rat\dfr{} and use the canonical basis for the right hand side of the
above mapping. Then the map $\theta$ can be represented by $A$ where 
$A$
is the following matrix.
\[
A=\begin{bmatrix}
1&1&1&1&1&0&0&1\\
1& 1&        -1&      -1&      0&       -1&      1&       0\\
1& 1&         1&       1&       -1&      0&       0&       -1\\
1& 1&        -1&      -1&      0&       1&       -1&      0\\
1& -1&        1&       -1&      0&               1&     0\\
1& -1&        -1&      1&       -1&      0&       0&       1\\
1& -1&        1&       -1&      0&       -1&      -1&      0\\
1& -1&        -1&      1&       1&       0&       0&       -1
\end{bmatrix}
\]

Continuing our calculations we find that $A^{-1}$ is $\frac{1}{8}$
 times the
following matrix.
\[A^{-1} = \frac{1}{8}\begin{bmatrix}
1&1&1&1&1&1&1&1\\
1&1&1&1&-1&-1&-1&-1\\
1&-1&1&-1&1&-1&1&-1\\
1&-1&1&-1&-1&1&-1&1\\
        2& 0&     -2&      0&       0& 2& 0& 2\\
        0& -2&    0&       2&       2&  0&    -2&      0\\
        0&  2&    0&       -2&      2&  0&    -2&      0\\
        2&  0&    -2&      0&       0&  2&    0&       -2
\end{bmatrix}
\]

In the same manner as before, we see that if
\[X=\left(x_1, x_2, x_3, x_4, 
\begin{pmatrix}
x_5&x_6\\
x7&x_8
\end{pmatrix}
\right)\]
with $X \in \integers\oplus\integers\oplus
\integers\oplus\integers\oplus\integers_2$ then X belongs to$\theta(\intdfr)$ iff
\[\begin{matrix}
x_1+&x_2+&x_3+&x_4+&2x_5&& +&2x8&\equiv 0\pmod{8}\\
&&\bullet\\
&&\bullet\\
&&\bullet\\
x_1 - &x_2- &x_3+&x_4+&2x_5&&-&2x_8&\equiv 0\pmod{8}
\end{matrix}
\]
Applying row reduction in the same manner as the last example we 
get
\[
\begin{matrix}
        i)      &x_1+&x_2+&x_3+&x_4+2&x_5&& +&2x_8&\equiv0\pmod{8}\\
        ii)     &&x_2+&x_3&&&&   +&2x_8&=0\pmod{4}\\
        iii)    &&   &x_3 - &x_4 - &x_5 - &x_6- &x_7+&x_8&\equiv0\pmod{4}\\
        iv)     &&&&x_4+&x_5&   +&x_7&   &\equiv0\pmod{2}\\
        v)      &&&&&x_5&&      +&x_8&\equiv0\pmod{2}\\
        vi)     &&&&&&x_6+&x_7&   &\equiv0\pmod{2}
\end{matrix}
\]        
In addition, if $X$ is to belong to $\theta(\ug{\intdfr})$, then $x_i= \pm 1$
 for $i=1,2,3,4$ and we must
have $x_5x_8-x_6x_7= \pm 1$.

Consider $\scrx \in GL(2,\integers)$, with
\[ \scrx =\mattbt{x_5}{x_6}{x_7}{x_8} \]
which satisfy equations $v)$ and $vi)$ above. Then there exists $x_i, i=1,2,3,4$
with $X=\left(x_1, x_2, x_3, x_4, \scrx \in \ug{\intdfr}\right) \iff$ one of $a)$, 
$b)$ or $c)$ below hold.
\[\begin{matrix}
a)&x_8\equiv 1\pmod{2};& x_5+x_6+x_7-x_8\equiv 0\pmod{4};& x_5+x_8\equiv 2\pmod{4}\\
b)&x_8\equiv 1\pmod{2};& x_5+x_6+x_7-x_8\equiv 2\pmod{4};& x_5+x_8\equiv 0\pmod{4}\\
c)&x_8\equiv 0\pmod{2};& x_5+x_8\equiv 0 \pmod{4}; &
\end{matrix}
\]

Let us do the calculations that show this. Consider: if $X$ is in 
$\theta(\ug{\dfr})$, then, as noted
before, we must have $x_1,x_2,x_3$ and $x_4 = \pm 1$. Let $\delta = \pm 1$. Then 
there are only certain combinations of the above that
 satisfy equations $i)$ through $vi)$. 
In particular, it should be noted that either all of $x_1$ to $x_4$
 are either of the same sign or there are two of one sign and
two of the other. To see this, consider equations $i)$ and $v)$. If 
three of $x_1$ to $x_4$ were positive 1
and the fourth negative 1, then we would have that $x_5 + x_8 \equiv 1 \pmod{4}$. This is an obvious contradiction to equation $v)$.

Therefore, let us consider the cases separately. If $x_1 = x_2 = x_3 
= x_4 = \delta$, then equation $i)$
implies that 
\begin{align*}
2x_5 + 2x_8 &\equiv 4\pmod{8} \mathrm{\ which\ implies\ that}\\
x_6+x_8&\equiv0\pmod{4}.
\end{align*}
Equation $iii)$ tells us that
\begin{align*}
-x_5-x_6-x_7+x_8&\equiv 0\pmod{4} \mathrm{\ which\ implies\ that}\\
x_5+x_6+x_7-x_8&\equiv 0\pmod{4}.
\end{align*}
Finally, equation $ii)$ gives us that
\begin{align*}
x_8&\equiv 1\pmod{2}.
\end{align*}
It can be readily seen that the above is condition $a)$.

Let us now consider $x_1=x_2= -x_3=-x_4=\delta$. Here we see that equation $i)$ 
implies
\begin{align*}
x_5+x_8&\equiv 0\pmod{4}.
\end{align*}
Equation $ii)$ implies
\begin{align*}
x_8&\equiv 0\pmod{2}.
\end{align*}
We see that we now have condition $c)$. In addition, however, we 
can see that equation $iii)$
implies that
\begin{align*}
x_5+x_6+x_7-x_8&\equiv 0\pmod{4}.
\end{align*}

Now suppose that $x_1 = -x_2 = x_3 = -x_4 = \delta$. Then, we will have 
equation $i)$ implying
\begin{align*}
x_5+x_8\equiv 0\pmod{4}.
\end{align*}
Equation $ii)$ gives us
\begin{align*}
x_8&\equiv 0\pmod{2}.
\end{align*}


Again, we have condition c). This time, though, Equation $iii)$
shows us
\begin{align*}
x_5+x_6+x_6-x_8&\equiv2\pmod{4}.
\end{align*}

The last combination to consider is $x_1=-x_2=-x_3=x_4=\delta$.
In this case we get equation $i)$showing that
\begin{align*}
x_5+x_8&\equiv 0 \pmod{4}.
\end{align*}
Equation $ii)$ implies that
\begin{align*}
x_8&\equiv 1 \pmod{2}.
\end{align*}
Equation $iii)$ implies that
\begin{align*}
x_5+x_6+x_7-x_8&\equiv 2 \pmod{4}.
\end{align*}

This finally is condition b).

To show the other direction of the if and only if above is quite 
simple given the above calculations. What one needs to do is to simply choose the 
particular $x_i\ (i=1,2,3,4)$ as given
above. These will then satisfy all the equations.

Let us denote by $\Omega$ those matrices of $GL(2,\integers)$ that satisfy 
equations $v)$ and $vi)$ above
and any one of a), b) or c). It is obvious by the linearity of 
the constraints that $\Omega$ is a
subgroup of $GL(2,\integers)$. For any element $X \in \Omega$ we can see by the 
above computations that there
are exactly two elements of $\theta\ug{\intdfr}$ with $X$ as the last member.

Let $\delta=\pm 1$ as above.

If a) holds, then $\mathscr{X}=(\delta,\delta,\delta,\delta,X) \in \theta\ug{\intdfr}$.


If b) holds, then $\mathscr{X}=(\delta,-\delta,-\delta,\delta,X) \in \theta\ug{\intdfr}$.


If c) holds, there are two cases to consider. If we have the 
first, which is
\begin{align*}
x_5+x_6+x_7-x_8&\equiv 0 \pmod{4}.
\end{align*}
holding, then
\begin{align*}
\mathscr{X}=(\delta,\delta,-\delta,-\delta,X) \in \theta\ug{\intdfr}.
\end{align*}
Otherwise, we have
\begin{align*}
x_5+x_6+x_7-x_8&\equiv 2 \pmod{4}.
\end{align*}
holding which gives us
\begin{align*}
\mathscr{X}=(\delta,-\delta,\delta,-\delta,X) \in \theta\ug{\intdfr}.
\end{align*}

Now, we are in a position to describe the unit group of \intdfr{}. If 
we choose $\alpha \in \ug{\intdfr}$ such that 
 $\theta(\alpha) = (x_1,x_2,x_3,x_4,X)$, then we can observe that 
$\xi(\alpha)=x_1$. From this and the preceding
information it is easy to see that since for any element $X$ in $\Omega$ 
there are exactly two elements
in $\theta\ug{\intdfr}$, we have the following theorem.

\begin{theorem}
\begin{align*}
\theta\ug{\intdfr} \cong \{\pm 1\}\times\Omega.
\end{align*}
\end{theorem}


\section{Units of \intdsx.}
In this section, we will show how a minor extension to the method 
can be used to
describe groups of higher orders. At this point, we will 
determine the unit group of \intdsx
where \dsx is the dihedral group of order 12 given by the 
generators $a,b$ together with the
relations
\begin{align*}
a^2=b^6=ab^2ab^2=e
\end{align*}

Now if we applied the method of this section blindly we would be 
dealing with matrices of
order 12. This, to say the least, is inelegant. In addition to 
this the description of \ug{\intdsx}
would include a direct product of two matrix groups. This does 
not give us a very satisfying
description, as determining properties of this type of product is 
not very easy.

Instead, what we do in this section is to consider $\dsx \cong \ctwo \times \st$
where $\ctwo$ is the cyclic
group of order 2,$={ +1,-1}$ under multiplication. \st is, as 
before, the symmetric group on 3
elements. Keeping this in mind, we have $\intdsx \cong (\intctwo)\st$, where 
\intctwo{} is the group ring of the
group \ctwo{} over the ring \integers, and the whole thing is the group ring 
of the group \st{} over the ring
\intctwo. The isomorphism of the above group rings is a well known 
theorem (See Sehgal\cite{bib:sehgal78}).

At this point, let us denote $\rat \ctwo$ by $\mathscr{R}$. Then what we intend to do 
is to apply the
method used previously, replacing \rat{} by  $\mathscr{R}$ and \integers{} by \intctwo. This 
brings us to the point where we may now define the map $\theta$ as
\begin{align*}
\theta \mathscr{R}\st\rightarrow\scr\oplus\scr\oplus\scr
\end{align*}

where the mapping is defined by
\begin{align*}
\theta(1\ 2)&=& \begin{pmatrix}
1&-1&
\begin{pmatrix}
1&-1\\
0&-1
\end{pmatrix}
\end{pmatrix}
\end{align*}
and
\begin{align*}
\theta(1\ 2\ 3)&=& \begin{pmatrix}
1&1&
\begin{pmatrix}
0&-1\\
1&-1
\end{pmatrix}
\end{pmatrix}
\end{align*}

At this point, one must keep in mind that the elements of the 
above vectors are in \scr{}
not in \rat{} as before.

Now, let us consider the two modules that we have. The first \scr\st{}
is obviously a free
module over \scr{} with basis the elements of \st. The second is again a 
free module over \scr{} being
a direct sum of matrix rings over \scr. We will use the standard 
basis for this module.


        Considering this, we may look at $\theta$ as a module homomorphism 
between two free
modules. We may therefore represent it as a matrix $A$, in this 
case, a $6\times 6$ matrix. Obviously
this matrix is going to have the same entries as the one we 
obtained when looking at \intsth{},
with the distinction that these elements will be in \scr{} and not 
just in \rat. For the sake of
convenience I will rewrite $A$ and $A^{-1}$.
\begin{align*}
A&=&\begin{bmatrix}
1&1&1&0&0&1\\
        1& -1&    1&       -1&      0&       -1\\
        1& -1&    -1&      0&       -1&      1\\
        1& -1&    0&       1&       1&       0\\
        1&  1&    0&       -1&      1&       -1\\
        1&  1&    -1&      1&       -1&      0
\end{bmatrix}
\end{align*}
and
\begin{align*}
A^{-1}&=&\frac{1}{6}\begin{bmatrix}
1&1&1&1&1&1\\
        1& -1&    -1&      -1&      1&       1\\
        2&  2&    -2&      0&       -2&      0\\
        0& 0&     -2&      2&       -2&      2\\
        0& -2&    0&       2&       2&       -2\\
        2& -2&    2&       0&       0&       2
\end{bmatrix}
\end{align*}

As before, since A is invertible we see that $\theta$ is an isomorphism. 
Also, as before, we are led to a
set of six congruences that describe when an element of 
$(\intctwo)\oplus(\intctwo)\oplus(\intctwo)_2$ mapped by $\theta^{-1}$

is in $(\intctwo)\st$. It is not neccessary to re-write the original 
equations. The result after row
reduction are the equations:
\begin{align*}
x_1+ x_2 & \equiv 0 & \pmod{2}\\
x_2 & \equiv x_6-x_5 & \pmod{3}\\
x_1\equiv x_3+x_5 & \equiv x_4+x_6 & \pmod{3}
\end{align*}
At the risk of being repititious, we note that the above 
congruences are in \intctwo.
Namely that we are considering modulo the ideals generated by $2$ 
or $3$ in \intctwo in the above
equations.

Let $\phi$ denote the projection map of $\scr\oplus\scr\oplus\scr_2$ onto $\scr_2$.
 Then we have that
\begin{align*}
\phi \theta((\intctwo)\st) = 
\begin{pmatrix}    
x_3&x_4\\
x_5&x_6
\end{pmatrix} : x_3+x_5&\equiv x_5+x_6\pmod{3}
\end{align*}

Let us call the above set $\mathscr{Y}$.

Let us now pause for a moment to consider some of the properties 
of \intctwo, where $e$ is
the identity of \ctwo{} and $\eta$ is the other element with $\eta^2=e$. The 
units of the group ring \intctwo{} are
easy to determine as they are only $\pm\ctwo$. Therefore, for a matrix 
to be in the units of $(intctwo)_2$
the determinant must be one of $\pm e$ or $\pm \eta$.

Let us continue. If $X= (x_1,\cdots,x_6)$ is in $\theta(\intctwo)\st$, then, if we 
let $\delta$ represent any one of
$\pm e, \pm \eta$, then as was calculated before, we have that $X^{-1}$ exists 
and is in  $\theta(\intctwo)\st$ if and only if
\begin{align*}
x_3x_6-x_4x_5=\delta, &x_1=\delta, &x_2= \pm\delta.
\end{align*}
The mapping $\phi\theta$ is a ring homomorphism of $(\intctwo)\st$ into $\mathscr{Y}$
 and thus induces a homorphism from $\ug{(\intctwo)\st}\rightarrow\ug{\mathscr{Y}}$. 
I conjecture that this is an isomorphism onto. To see this let
\begin{align*}
\mu&=&\begin{pmatrix}
x_3&x_4\\
x_5&x_6
\end{pmatrix} 
\in \ug{\mathscr{Y}}.
\end{align*}
Then $\delta =x_3x_6-x_4x_5=\pm e$ or $\pm \eta$  and if we choose 
$x_1,x_2$ which are in $\{e, 0, -e, \eta, -\eta\}$ to satisfy
\begin{align*}
x_2&\equiv x_6 - x_5&\pmod{3}\\
x_1&\equiv x_3+x_5&\pmod{3},
\end{align*}
it follows from this that neither $x_1$ nor $x_2$ is 0 and that the 
above conditions are all satisfied.
Thus $\alpha=\theta^{-1}X$ is a unit in $(\intctwo)\st$ with 
$\phi\theta\alpha=\mu$. Furthermore, we can 
see that from the
preceding conditions that $\phi\theta$ is one-to-one.

Therefore, remembering that $\intdsx \cong (\intctwo)\st$ and that
$\ug{\intdsx} \cong \ug{(\intctwo)\st}$ we have
the following theorem.

\begin{theorem}
\begin{align*}
\ug{\intdsx}\cong \left\{
\begin{pmatrix}
a&b\\
c&d
\end{pmatrix}
\in \ug{\intctwo} : a+c \equiv b+d \pmod{3}\right\}.
\end{align*}
\end{theorem}

\section{The Unit Group of \intaf,(expository)}
In this section, we will present the charachterization of $\ug{\intaf}$
as presented by Allen
and Hobby\cite{bib:allen1980}. We will not include the proofs, as it is felt that 
nothing new is to be gained from rehashing the method.

As is known \af{} has 4 irreducible representations, call them $\theta_i,i = 
1,2,3,4$. Also \af{} is
generated by the two elements $a=(1 2)(3 4)$ and $b=(1 2 3)$. The 
representations $\theta_i, i=1,2,3$ are easily described as follows. 
$\theta_i(a)=1$ and $\theta_i(b)=\omega^{i-1}$. The fourth, 
$\theta_4$ is given as follows
\begin{align*}
\theta_4(a) &=\begin{bmatrix}
-1&0&1\\
-1&0&1\\
-1&1&0
\end{bmatrix} \mathrm{\ and\ }\\
\theta_4(b) &=\begin{bmatrix}
-1&1&0\\
-1&0&0\\
-1&0&1
\end{bmatrix}
\end{align*}
Let $\theta:\rat\af \rightarrow \rat\oplus\rat\oplus\rat\oplus\rat_3$
 be defined by $\theta(r) = (\theta_1(r),\theta_2(r),\theta_3(r),\theta_4(r))$.
 Then we may use this map as in the preceding sections to determine the unit 
group. The characterization
arrived at by Allen and Hobby is as follows.

\begin{theorem}
\begin{align*}
\ug{\intaf}\cong \{\pm 1\} \times \{X \in SL(3,\integers)\}
\end{align*}
 where $X$ satisfies the below 
conditions 1), 2) and 3)

\begin{itemize}
\item{}Every column sum of $X$ is congruent to $1\pmod{4}$.
\item{}No row contains all odd elements.
\item{}One pseudo-trace is congruent to $-1\pmod{4}$ while the other two 
are congruent to $0\pmod{4}$.
\end{itemize}
\end{theorem}

There are three pseudo-traces on a three by three matrix, they 
are the sums of the
elements on the pseudo-diagonals that have been discussed 
previously.
\chapter{Groups of order $27$}
In this section, we will use the method presented earlier in the 
paper to determine the unit
group of the integral group rings of the two non-commutative 
groups of order 27.

The structures of these two groups are:
\begin{align*}
G = \langle a,b | (a,b) = c, ca = ac, cb = bc, a3 = b^3 = c^3 = 1\rangle,\mathrm{\ and}\\
H = \langle a,b | a^9 = b^3 = 1, b^{-1}ab = a^4\rangle.
\end{align*}
As is true in general, we have that  
$\widetilde{G} = \langle\tilde{a}\rangle\times\langle\tilde{b}\rangle$, and 
$\widetilde{H} = \langle\tilde{a}\rangle\times\langle\tilde{b}\rangle$ 
are both
elementary abelian 3-groups. Also, it is well known that
 $\ug{\intgr{\tilde{G}}} = \pm \tilde{G}$ 
and $\ug{intgr{\tilde{H}}} = \pm \tilde{H}$. The
object of this section is to give a concrete description of both 
$\ug{\intG}$ and $\ug{\intgr{H}}$.
\section{First group of order 27.}
Let us consider $G$ first. Referring back to our general proof we 
would let
$A=\mathrm{DIAG}(1,\omega,\omega^2)$ and $B = \mathrm{PDIAG}_1$. 
Our fibre product diagram would become
\[
\xymatrix{
\intG \ar[r] \ar[d] &
\intgr{\tilde{G}}  \ar[d] \\
\intgrgen{\omega}^\circ\tilde{G}\cong\intG/\tilde{c}\intG \ar[r] \ar[d]_{\phi_0} &
(\integers/3\integers)\tilde{G} \\
S_p \ar[ur]^{\phi_1}\\
}
\]

The maps $\phi_0$ and $\phi_1$ are defined in the same way as 
they were in general.

Specializing the condition (*) from the general theorem we see 
that the matrix
\begin{align*}
M&=&\begin{bmatrix}
x_{0,0}&x_{1,0}&x_{2,0}\\
x_{2,1}&x_{0,1}&x_{1,1}\\
x_{1,2}&x_{2,2}&x_{0,2}
\end{bmatrix} \in \intgrgen{\omega}_3
\end{align*}
belongs to our $S_p$ if and only if for each $i, 0\leq i \leq 2$,
the conditions
\begin{align*}
x_{i,0} + x_{i,1} + x_{i,2} &\in & 3\intgrgen{\omega} \\
x_{i,0} + x_{i,1}\omega + x_{i,0}\omega^2 &\in & 3\intgrgen{\omega} \\
x_{i,0} + x_{i,1}\omega^2 + x_{i,2}\omega &\in & 3\intgrgen{\omega}
\end{align*}
hold. To find $\phi_1(M)$, we need $a_{i,j} \in \intgrgen{\omega}$ 
such that $M=\sum a_{i,j}A^iB^j$. This will give us the matrix
equations
\begin{equation*}
\begin{bmatrix}
a_{0,j}\\
a_{1,j}\\
a_{2,j}
\end{bmatrix} =\frac{1}{3}
\begin{bmatrix}
1&1&1\\
1&\omega^2&\omega\\
1&\omega&\omega^2
\end{bmatrix}
\begin{bmatrix}
x_{j,0}\\
x_{j,1}\\
x_{j,2}
\end{bmatrix}
\end{equation*}
and this is equivalent to
\begin{equation}\label{eqn:grp27}
\begin{split}
a_{0,j} &= \frac{1}{3}(x_{j,0}+x_{j,1}+x_{j,2})\\
a_{1,j} &= \frac{1}{3}(x_{j,0}+\omega^2 x_{j,1}+ \omega x_{j,2})\\
a_{2,j} &= \frac{1}{3}(x_{j,0}+\omega x_{j,1}+ \omega^2 x_{j,2})
\end{split}
\end{equation}

From our previous work we know that 
$\phi_1(M) = \sum \tildeaij \tilde{a}^i \tilde{b}^j$.
Also, we know that the units
of \intG{} are pairs $(\alpha,M)$, with 
$\alpha \in \ug{\intgr{\tilde{G}}}$ and $M$ in $S_p$ 
with $\phi_1(M) = \phi_2(\alpha)$.
However, since we know that $\ug{\intgr{\tilde{G}}}  = \pm \tilde{G}$, 
we need matrices $M$ such that
\begin{equation*}
\phi_1(M) = \sum \tildeaij \tilde{a}^i \tilde{b}^j = \theta_2(\pm a^m b^n)
\end{equation*}
for some $m,n$. If we put $\pi = \omega -1$ we then  get
\begin{itemize}
\item{}For two values of $i$ and all $j$, $a_{i,j} \equiv 0 \pmod{\pi}$.
\item{}Considering the third value for $i$, either 
\begin{align*}
a_{i,0} &= \pm 1, & a_{i,1}\equiv a_{i,2} & \equiv 0 \pmod{\pi} \mathrm{\ or}\\
a_{i,1} &= \pm 1, & a_{i,0}\equiv a_{i,2} & \equiv 0 \pmod{\pi} \mathrm{\ or}\\
a_{i,2} &= \pm 1, & a_{i,1}\equiv a_{i,1} & \equiv 0 \pmod{\pi}
\end{align*}
\end{itemize}
        This proves the following theorem.
\begin{theorem}
\begin{equation*}
\ug{\intG} \cong \left\{
M\in\ug{\intgrgen{\omega}}_3 |
M \mathrm{\ satisfies\ 1.\ and\ 2.\ above\ where\ the\ }a_{i,j}
\mathrm{\ are\ given\ by\ equation\ }\ref{eqn:grp27}
\right\}
\end{equation*}
\end{theorem}
        
From the above it is clear that the matrices in $\ug{\intgrgen{\omega}}_3 $
 which are congruent to $I \pmod{\pi^3}$ are contained in $\ug{\intG}$
 and therefore, $\ug{\intG}$ is a 
congruence subgroup in $SL(3,\intgrgen{\omega})$.

\section{Second group of order 27.}
        Now, let us describe $\ug{\intgr{H}}$ If we have a matrix
\begin{equation*}
X = Z' = \begin{bmatrix}
x_{0,0}&x_{1,0}&x_{2,0}\\
x_{2,1}&x_{0,1}&x_{1,1}\\
x_{1,2}&x_{2,2}&x_{0,2}
\end{bmatrix}
\end{equation*}
satisfying (*) then the corresponding matrix in $S_p$ is
\begin{equation*}
Z = \begin{bmatrix}
x_{0,0}&x_{1,0}&x_{2,0}\\
\omega x_{2,1}&x_{0,1}&x_{1,1}\\
\omega x_{1,2}&\omega x_{2,2}&x_{0,2}
\end{bmatrix}
\end{equation*}

If we write $A = \sum a_{i,j}B^iA^j$, then $\phi_1(Z) = \pm h, 
h \in H$  if and only if the matrix X satisfies 1. and 2.
from above. Then we have the the following theorem.

\begin{theorem}
\begin{equation*}
\ug{\intgr{H}} \cong
\left\{Z \in \ug{\intgrgen{\omega}}_3 |
Z' \mathrm{\ satisfies\ 1.\ and\ 2.\ above\ where\ the\ }a_{i,j}
\mathrm{\ are\ given\ by\ equation\ }\ref{eqn:grp27}
\right\}
\end{equation*}
\end{theorem}
\appendix
\chapter{Definitions}
\begin{definition}\label{defn:abelianGroup}
 $G$ is an \emph{abelian group} when all elementsts of $G$ commute.
\end{definition}
\begin{definition}\label{defn:hamiltonianGroup}
$G$ is \emph{Hamiltonian} when it is non-abelian and where each subgroup is
normal.  All such groups are of the form $G = Q_8 \times E \times D$ where
$Q_8$ is the quaternion group of order 8, $D$ is a torsion group and $E$ is a
direct sum of a finite number of copies of the cyclic group $C_2$.
A Hamiltonian 2-group is a group of the 
form $Q_8 \times E $ where \(E^2 = 1\)).
\end{definition}
\begin{definition}\label{defn:normalSubGroup}
$N$ is a \emph{normal} subgroup of $G$ when it is invariant under conjugation. 
That is $\forall g\in G, n\in N, g^{-1}ng \in N$.
\end{definition}
\begin{definition}\label{defn:quaternionGroup}
A non-abelian group of size 8 denoted as $Q_8$ 
with the elements $\{1, -1, i, -i, j, -j, k, -k\}$
 where $i^2 = j^2 = k^2 = ijk = -1$ 
and $(-1)x = x(-1) = -x\ \forall x \in Q_8$. This is a Hamiltonian group 
and is included in all Hamiltonian groups.
\end{definition}
\begin{definition}\label{defn:torsionGroup}
 $G$ is a \emph{torsion group} when it is an abelian group and
every element of $G$ has 
finite order. This is sometimes referred to as \emph{periodic}.
\end{definition}


\bibliography{tunvyrs}
\bibliographystyle{plain}
\end{document}