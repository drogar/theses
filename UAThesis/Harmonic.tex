\documentclass{article}
\usepackage[T1]{fontenc}
\usepackage[latin1]{inputenc}
\usepackage{amsmath, amsthm, amssymb}
\usepackage{amsfonts}
\usepackage{graphicx}

\input{mathmacs}

\newtheorem{theorem}{Theorem}
\newtheorem{acknowledgement}[theorem]{Acknowledgement}
\newtheorem{algorithm}[theorem]{Algorithm}
\newtheorem{axiom}[theorem]{Axiom}
\newtheorem{case}[theorem]{Case}
\newtheorem{claim}[theorem]{Claim}
\newtheorem{conclusion}[theorem]{Conclusion}
\newtheorem{condition}[theorem]{Condition}
\newtheorem{conjecture}[theorem]{Conjecture}
\newtheorem{corollary}[theorem]{Corollary}
\newtheorem{criterion}[theorem]{Criterion}
\newtheorem{definition}[theorem]{Definition}
\newtheorem{example}[theorem]{Example}
\newtheorem{exercise}[theorem]{Exercise}
\newtheorem{lemma}[theorem]{Lemma}
\newtheorem{notation}[theorem]{Notation}
\newtheorem{problem}[theorem]{Problem}
\newtheorem{proposition}[theorem]{Proposition}
\newtheorem{remark}[theorem]{Remark}
\newtheorem{solution}[theorem]{Solution}
\newtheorem{summary}[theorem]{Summary}


\begin{document}

\title{Harmonic Analysis on Groups}


	Harmonic analysis can be said to comprise the study of
Function spaces over a topological group $G$.  In this essay, we
will mainly consider \LG\  and its extensions and what sort of
relation \LG\  considered as a ring has to \(G\).


\begin{definition}  A topological group is a group \(G\) together with
a topology (which is Hausdorff), such that the map \(m:(G \times G)\rightarrow G\)
given by \(m(x,y)= xy^1\) is continuous.
\end{definition}

Examples:
\begin{enumerate}
\item \((R,+)\) usual topology is a topological group.
\item 	A Group with discrete topology is a topological group.
\end{enumerate}

	Let \(G \) be locally compact throughout the rest of
this paper. Let \(\mathfrak{S} \) be the \(\sigma \) algebra generated by all 
\emph{compact} subsets of \(G \).  A. Haar proved in 1934 that there is a measure
defined on  \(\mathfrak{S} \) such that \(m(E) = m(xE) = m(Ex) = m(E^{-1} \).  In 1934
Von Neuman gave a definition of an invariant integral on compact
\( G\) (an invariant integral is one for which \( \int_G f(x)dx = \int_G f(ax)dx =
\int_G f(xa)dx = \int_G f(x^{-1})dx\) and the other axioms of integrations such
as additivity, multiplication by constants hold).  We will not
give the proof of this here as it is rather long and does not fit
in well with the essay.  For a reference, see \cite{bib:pontryagin}   pp 192-201.


Examples of Haar measures on groups are:
\begin{enumerate}
\item Usual Lebesque measure on \((R,+)\)
\item Counting measure on finite, discrete group (integration is 
equivalent to summation).
\end{enumerate}


	Now, we can define an integral of a complex
valued function by the general methods as set out in \cite{bib:royden}
namely, define simple functions and so on.  As this has been
done many time8 elsewhere we will not repeat it here.

Now, let us restrict ourselves to locally compact
abelian\footnote{LCA will mean locally compact abelian} groups.

\begin{lemma}
Let \(m\) be a Haar measure on a LCA group \(G\), then
if \(V\) is a non empty open subset of \(G\), then \(m(V)> 0\).
\end{lemma}
\begin{proof}
Suppose \(m(V)= 0\). Let \(K \subset G\) be compact.  The
translates of \(V\) form an open cover of \(K\) and therefore there is
a finite sub cover.  Hence \(m(K)=0\), this then implies \(m(E)=0\)
for all Borel sets \(E \subset G\), which is a contradiction. \(\square\)
\end{proof}
\begin{lemma}
If \(m, m'\) are two Haar measures on a LCA group
then \(m'(E)=bm(E)\).
\end{lemma}
\begin{proof}
Let \(g \in C_{k}(G)\) (functions with compact support)
such that \(\int gdm = 1.\)
Let \(b = \int_G g(-x)dm'(x).\)

For all \(f \in C_k (G)\)
\begin{eqnarray}
\int_G fdm' & = & \int_G g(y)dm(y) \int_G f(x)dm'(x)\\
&= &	\int_G g(y)dm(y) \int_G f(x+y)dm'(x)\\
&= & \int_G dm'(x) \int_G g(y)\centerdot f(x+y)dm(y) \label{eqn:needfth1}\\
&=& \int_G dm'(x)\int_G g(y-x) \centerdot f(y)dm(y)\\
&=& \int_G f(y)dm(y) \int_G g(y-x)dm'(x) \label{eqn:needfth2}\\
&=&b \int_G f(y)dm(y)\\
\end{eqnarray}

Equations \ref{eqn:needfth1} and \ref{eqn:needfth2}  follow by
Fubini's theorem as \(g(y)\centerdot f(x+y) \in C_k (G \times G))\),
therefore \(m' = bm.\)
\end{proof}


	Thus, we can speak of \emph{The} Haar measure on a group
by normalizing.  If \(G\) is compact \( m(G) = 1\), if \(G\) is discrete, let
\(m(x) = 1\) (\(x \) is any singleton in \(G\)).  If \(G\) is finite use the latter
definition.  In place of writing \(\int_G fdm\) we may write \(\int_G fdx\).

\begin{definition}
 For \(f\) and \(g\), Borel functions, define the convolution

of \(f\) and \(g\) as \[ f \ast g = \int_G f(x-y)g(y)dy \mathrm{where}
\int_G |f(x-y)g(y)|dy < \infty \label{eqn:convolution}\]
\end{definition}

Let us consider \LG, the function space with the
range of a function defined on \(\mathfrak{C}\).  It is clear that the \(L^1\)
norms are translation invariant as integration is invariant.

\begin{theorem}
\begin{enumerate}
\item \((f\ast g)(x) = (g\ast f)(x)\) (provided that both
are defined at x).
\item if \(f, g, h\) are in \LG\  then \[(f \ast g) \ast h = f\ast (g\ast h)\]
\item if \(f, g\) are in \LG\ then \ref{eqn:convolution} 
holds for almost all \(x\) in \(G\),
\[f\ast g \in \LG \mathrm{and} ||f\ast g ||_1 \leq ||f||_1 \centerdot ||g||_1\]
\end{enumerate}
\end{theorem}

\begin{proof}
\begin{enumerate}
\item \( (f\ast g)(x) =\int_G f(x-y)g(x)dy \)
replace \(y\) by \(y+x\) and we have
\(f*g = \int_G f(-y)\centerdot  g(y+z)dy\)
and since \(mE = m(-E)\)
\(f*g = \int_G f(y)\centerdot g(-y+x)dy = g \ast f\)
\item	To prove part two we use Fubini's theorem
\begin{eqnarray*}
& (f\ast (g\ast h))(x) = \int_G f(x-z)(g\ast h)(z)dz \\
&=&\int_G f(x-z)\int_G g(z-y)h(y)dydz \\
&=&\int_G f(x-z-y)g(z)h(y)dzdy\\
&=&\int_G (f\ast g)(x-y)h(y)dy = ((f\ast g)\ast h)x
\end{eqnarray*}
\item Again we will use Fubini's theorem, but
first we need to show that \(f(x-y)g(y)\) is a Borel function on \(G \times G\).

	Let \(V)\) be open in the complex plane,\( E = f^{-1} (V)
E' = E \times G, E'' = {(x,y) :  x-y \in E} E'\) is a Borel set in\( G \times G\) and
\( \eta: G \times G \rightarrow G \times G\) defined by
\( \eta(x,y) = (x+y, y)\) is a homeomorphism
which maps\( E'\) onto\( E''\), \(\therefore E''\) is also a Borel set. 
 Since \(f(x-y) \in V \iff (x,y) \in E''\), \(f\) is a Borel function 
and therefore so is \(f(x-y)\centerdot  g(y)\).

By	a simple application of Fubini's theorem we see that
\[ \int_G \int_G |f(y-x)g(y)| dydx = ||f||_1 \centerdot ||g||_1\]
\end{enumerate}
\end{proof}


		~	f(y~x)g(y)~ dl'
Let q(x) =

q(x)~ L~(G) and q(x)~ m for almost all x.
exists for almost all x and since  f~ g(x))4

~	tI~l11 1~1{L
t
From the above, we see that L (G) forms a

Comutative Banach Algebra with ~ as the multiplication. Let us
investigate this structure a little further.  A natural question
is, does there exist a multiplicative identity in L (G)?  The
answer is given in the following;
it follows that

Therefore,  (f~g)(x)
q(x) we have that
Theorem 2  L6]  If G is discrete I~(G) possess a unit.


Proof
M(x) = 1.

	Let the Haar measure be normalized such that
Then (f*g)(x) = Z f(x-y)g(y)
	define e(O) = 1,	e(x) = 0 x ~ 0

then e~L1(G) and  f = f~e = e~f. 0


Next we shall show that if G is not discrete

1
Then L (G) has no unit element but a few definitions and a lemma
is needed.

Let e (g) be an arbitrary L~ function such that:

(1)	e~(g)7O
-6-

(2)  e  (g) = 0

(3)  ~IeAtI =1

example is  charA (g)
m(A)


Definition 3.  Let g be a fixed eleiient of G and ~   a neighbou~

hood base for ~ero, with every U ~ ~ associate the function   ~ U

We say these functions fomi a ~ net contractin~ to the element ~.


Let C ~Ube an arbitrary ~ net contracting to

~	G.  ThenforallfEL'(G)

lim  IIf%  ~f*e~~lI=0
U~o

where f~(y) = f(y-x)

and this limit relation is satisfied at any given f, uniformly
for all ~ nets contracting to g~and all g0 ~ G.
L~a ~ [2]
the element


L~~of         For every ~ ? 0 there exists a neighbourhood of 
zero,
suchthat   ~if(x)-f(x-y)I dx ~�

(This is the absolute continuity of the integral with respect to
the Haar measure which we will take as given).


f* e~0+UIi
	�(x-y~e  (	A
	) -	~ y)dytdx

Now making the translation ~  x + g
y ~ y + g
Now we are ready

-7-

we have

A = ~~f~~(x) - f(x~y)Je~~(y+g0)dy~ dx = A
	now observing that  e	(
~	y+g0) =~~(y)
arid applying Fubini's theorem we have

A ~f( f lf(~) - f(x-y)  dx)e~(y)dy ~

~	~~~f(x) - f(xy)(dx ~

for all  UC UL  no matter what g0 is or
what~ net was chosen. ~

for the theorem.


Theorem 3~ L21. If G is non discrete, then the ring L~(G) does
not have a multiplicative identity.


		Let e~ be a ~ net contracting to O~ G and
~L~(G) such thate is a multiplicative identity.

Now by the preceding Lemma 3, lim (e- e~e~t--o
U40
we have	lim i~e-e~:I   0
U~o
Since e~ is zero outside of V where U~V we


have that ~  ~ 0 arid therefore SV je~)~ ~  = 1
as  f f6(~I~~    mustbel.
	Therefore	e (~ I ~	= 1 for any neigh

bourhood V of zero.  But this implies there exists ~    such
that~(~)~ 0 and f Je(~))~~ = 1 which is a contradition
as the Haar measure is regular and G is not discrete. ~
but since
?roof

suppose ]e
   Let the group ring v(G) of G be defined by

(i)	v(G) = L1(G) if G is discrete

(ii)	v(G) =~et~~ whereXe ~  , f~ L~G) and
e is a unit for the convolution of function.

In the latter case if    ~  v(G)

define ~    X tuf1')
Banach Algebra over ~

		Let us examine some general algebraic properties
Algebra over C .  As is known if R is an integral

M is	a maximal ideal of R then RIM ~ to a field.
In the case where R is a Banach Algebra then
Consider the group ring of a group G.  It,

has maximal ideals M and v(G)/M z- C

		The above facts will be accepted without proof
are done in most modern Algebra classes.  If proofs or
information is desired please see  [2] chapter 1.

		Let~= ~M : M is a maximal ideal of v(G)3

We can fairly naturally define a class of functions from ~   C
by x(M) = coset representative of x in v(G)/M.  Therefore1 x(M)

is a map into ~.  For general rings R the following properties
hold.
Definition 4
Then v(G) is a
of Banach
domain and
RIM

therefore,
as they
further
(a)

(b)

(c)

(d)

(e)

(f)



that we shall
by )x+M	=

as x ~ x(M) +
-9-


If x = y+z, then x(M) = y(M) + z(M)

x	= y~ implies x(M) = y(M)s(M)

x	~ implies z(M) =Xz(M)

e(M) = 1 (where e is the identity of R)

x(M0) = 0 if and only if xEM0

M1~ M~ implies that for some Z6 R  x(M ) ~ x(M2 )
These properties are so algebraically obvious

not prove them.  If we define the norm of a coset
inf ~ zIl  then ~x(M)I = inf ~z   -~t)xII
~EXM

Now that we have explored the properties of a

general class of functions over G (i.e. - the integrable ones).

Let us digress a bit and consider a more specific set of 
functions.


Definition 5   ~ : G ~ C is called a character if  ~ (x~ = 1
~x~G and ~(x+y) = ~(x)  ~ (y).


  The set of all continuous characters is called

~	of G.

	Note the similarity of definition when V is a
topological vector space over ~ and V* the dual of V is the set
of continuous linear functionals on V.
Definition 6

the dual group


	There is an immediate way of determining all
- 10 -


group when that group is discrete. Let M be a

v(G) = e1(G) = L~(G).  Define M(g) by

M(g) = e~ (M)

		e~6~1(G), e9(~-- ~5(Kronecker delta)
Recalling that  ~th    9 ~ e~ we have

M(g+h) = ~ (M) = e~(M)  ~ = M(g)M(h) and

	~	I e     = 1,  i~rn) = 0 ~ = 1 and since

	=	M(g). M(-g), both M(g) and M(-g) must be of absolute
value le
characters in a
maximalideal of

	Let us now sketch how we see that there is a 1 - 1
correspondence between the set of maximal ideals and the set of
continuous characters in the discrete case.

	Naturally the maximal ideal space is contained in
the set of continous characters as G is discrete and therefore the
map M(g) must be continuous.  To see the other way 1et~ be a con-
tinuous character on G.  Define a map from L~(G) to~ by ~
9
(which is a way of representing any element of L1(G))goes to 
~x~%(~),

To show this is a ring homomorphism we need only show 
multiplicavity
as the other cases are trivial.

~(~x(gh)y(h)))<(g) =~ (~x(g-h)~(g) )y(h)

-	~ (~ x(g~h)/(g~h))~(h)y(h)
h9

-	~x(g)~(g) ~y(h)%(h).
b

	Q -~ I therefore it is not a zero map.  Denote its Kernel by M~.
It is easy to see that this sets up a one to one correspondence
- 11 -


between the two Bets.

	Since we have now disposed of the easy case let
us consider G where G is non-discrete.  In this case there is a

Blight difference as Lt ( G) is a maximal ideal of v( G).  Denote
L~(G) by M~ .  Then x(M~ ) = 0 for all x ~ L2 (G).  Therefore,
the correspondence is slightly different in that there is a one-
to-one correspondence between the characters and the other max-
imal ideals.  Let us proceed with the sketch of this.


Theorem 4 [2]  Let M be a maximal ideal of v(G), N~M~  and
let e (j ~L) be a ~ net contracting to the element ~  G then

lim e~+~ (M) = M(g), the limit exists, is unique and is
u~o
satisfied uniformly for all g ~ G, and ~ nets contracting to ~


M(g) is a character of G.



?roof	There exists z~L1(G) such that z(M)~ 0.  Recall
that   = the translate of z by g, then by Lemma 3 we have
(z9(M) - Z(M)e(~~I) (M) 4 )  z ~-z~ e9~~~i~O for ~
therefore e   (M) -

	This is uniformly for all g ~ G, and all ~ nets
e ~ ~    .  If we fix z then for all ~ nets

	M(g) = ++ZSMM


fixing the ~ net we see M(g) is independent with respect to z.
�


M(g) is continous a8,

~M(g') - M(g)~ =   (s~  - z~ )M~  ~
~M)


Ii	s~t-s9~I =
	Iz(M)j	z(M)

Furthermore Ie9~~ (M)\ ~ ~  =11 therefore,

		~	1 and M(o) = 1
M(g+h) = lim  (e   *e
		UV~OO    ~fU  h+'1 )(M) =

		lim  e~+~ (M) lim e   (M) = M(g) 'M(h)
	ujo	~
		as before i~M(g)II  = 1 andM is a character Ef'9 ~
therefore
as g'~ g.


		After we have this, we can analogously show a

correspondence between the set of continuous characters of G and
the maximal ideal space of v(G).  Therefore, we have P ~

We have given a characteri~ation of the dual group in terms of the
algebraic structure of the group ring.  Let us now consider how 
to

topologize the dual group.
A
Definition 7	lf f  ~ ( G) the fourier transform of f is f

where fA(~) = ~ .  Denote the set of all fA

by A (r).  Give r the weak topology determined by A (1').  The

reason behind this definition is that it will help establish a

duality between G and P

		Next we shall prove that with the topology that
~ ~i,~1' P - P  i e n
- 13 -


Theorem ~  ~6]

(a)	(x, ~) is a continuous function of G x

(b)	Let K, C be compact subsets of G and P , respectively.
Let ~ r be set of complex numbers ~ withil-zi 4
		     N(K,r) =  ~ :(x, ~ )~U~	for all x6K~
		     N(C,r) = ~ z:(x, X )EUr	forall ~6C3
		Then N(K,r) and N(C,r) are open	subsets of F and G
		respectively.
	(c)	The family of all translates of
		for the topology of P
	(d)	F'  is a LCA group.
all N(~<,r) is a base


Sketch of

(a)

Proof

We have the equality Af (~)(x,~) =
and therefore the continuity of ~ (~) ~uld

imply that of (x,~).  The continuity of (f~)(~ )
is proved in a standard manner.

(b)	One uses the fact that K is compact and (x~) is
continuous on G x r to construct a neighbourhood of
any point in N(K,r), and similarly for N(C,r).

(c)	If V is a neighbourhood of 0 ,f, ,   f~~i1(G)
then(\~~~:f~ (~) -~ (0)I4~cV.

One can assume f ~ E c~(G) as c~(~ is dense in L'(G).
u


Then if r < � IMax  Ii f~iJ1 and if ~ E N(K,r)
	where K ~ support ~	, i=1 ,  nA and K compact.
	fL(5) - f~ (oA ~	f,< l(x,i) - 1~	~ (x)1dx
	r ~ fl	< �
	Therefore, N(K,r)~ V, now 0 can be generalized to any point
in ('and therefore N(K,r) is a Base.

(d)	[~'+ N(K,r/2)]  - L~+ N(K,r/2)j ~ N(K,r)
therefore, the map (~, ~' )~ ~ - ~ is continuous and
we have F is a LCA group.


Theorem 6  L~  If G is Compact {' is discrete.  If G is discrete P
is compact.


Proof	If G is discrete then L1(G) has a unit ~ and
therefore its maximal ideal space is compact as ~h) 
=fG'e(x)h(~x)dx =
h(0) = 1 , 1 ~ c0( P ) if and only if P is compact.

	If G is compact and the Haar measure is normalized,
then

	f~(x)dx=~10  ~ 00   (*)


IfK= 0th~f~~(z)dx i8 ~ual to�~1dz =1

	If ~ ~ 0 then ~(x~)~ 1 for some x0e G and
{~(x)dx = ~ (x~) f~~x - x0 )dx =~(xc)i ~(x)dx.

	Therefore, (*) is proven. If f(x) = 1 for all X6 G

f~L'(G) as G is compact and fA(Q) = 1 and	0 for~~0,
- 15 -


therefore ~o3 is an open set in r' and P is discrete. 0


	As a

	(1)



(2)

	(3)
few examples we have:

if G = (~, +) with usual topology then

= (~R,+) with the usual topology

if G =~ the integers then r = (~,+) I(2~tIR) = T
and if G = (IR,+) I (2fr~~) = T, then r' =


	The above examples are easily verified, but to clarify
the method, let us work through number (2).

	If G = ~ then (l)~) =     some ~ and therefore (n~~) =
e     Therefore, the map h: F'  ~  T given by h(~ ) = e  is a
group isomorphism.  The two topologies coincide as well.  To see
this V(n,r) =~~: ~ -	~	} ~ n for a neighbourhood base

at zero. ~  V(n,r) if and only if	10(1  (2/h)arc sin (r/2),
therefore, the topologies coincide.

	~Iow that we have P is a topological group, we can of
course define a Haar measure on it and compute its group ring and
its dual group.

	Pontryagin showed that every LCA group is the dual group
of its dual group.  If we have G,~ ,~   (a series of dual 
groups))G  ~A
by the map o(: G~P~ defined by (~ ,~(x)) = (z,~) for all ~ in P
First of all o( is shown to be an algebraic isomorphism of G into 
P


then a homeomorphism, following which ~ (G) is shown to be dense
A                    A

and closed in ~ and therefore equal to U'  As a consequence of

this we see that every compact LCA group is the dual of a discrete
LCA group and vice versa.

	In conclusion, we have seen that ha~monic analysis,
namely the close study of v(G) (and therefore L1(G)) has given us
a characterisation of a topological concept, that of the dual 
group.



\bibliography{../dailyjournal/tunvyrs}
\bibliographystyle{plain}
\end{document}

