\documentclass[11pt]{report}
%%%%%%%%%%%%%%%%%%%%%%%%%%%%%%%%%%%%%%%%%%%%%%%%%%%%%%%%%%%%%%%%%%%%%%%%%%%%%%%%%%%%%%%%%%%%%%%%%%%%%%%%%%%%%%%%%%%%%%%%%%%%
%\usepackage[none,dcucite,abbr]{harvard}
\usepackage[T1]{fontenc}
\usepackage[latin1]{inputenc}
\usepackage{harvard}
\usepackage{graphicx}
\usepackage{amsmath,amsthm,amsfonts,amssymb}
\usepackage{uathesis}
\usepackage{lscape}
\usepackage{longtable}
\usepackage{pstricks,pst-node}
\usepackage[mathscr]{euscript}



\newtheorem{theorem}{Theorem}
\newtheorem{acknowledgement}[theorem]{Acknowledgement}
\newtheorem{algorithm}[theorem]{Algorithm}
\newtheorem{axiom}[theorem]{Axiom}
\newtheorem{case}[theorem]{Case}
\newtheorem{claim}[theorem]{Claim}
\newtheorem{conclusion}[theorem]{Conclusion}
\newtheorem{condition}[theorem]{Condition}
\newtheorem{conjecture}[theorem]{Conjecture}
\newtheorem{corollary}[theorem]{Corollary}
\newtheorem{criterion}[theorem]{Criterion}
\newtheorem{definition}[theorem]{Definition}
\newtheorem{example}[theorem]{Example}
\newtheorem{exercise}[theorem]{Exercise}
\newtheorem{lemma}[theorem]{Lemma}
\newtheorem{notation}[theorem]{Notation}
\newtheorem{problem}[theorem]{Problem}
\newtheorem{proposition}[theorem]{Proposition}
\newtheorem{remark}[theorem]{Remark}
\newtheorem{solution}[theorem]{Solution}
\newtheorem{summary}[theorem]{Summary}
%\newenvironment{proof}[1][Proof]{\textbf{#1.} }{\ \rule{0.5em}{0.5em}}
%\input{tcilatex}


%\newcommand{\iff}{\Longleftrightarrow}
\newcommand{\supp}{\mathrm{supp}}
\newcommand{\KG}{\ensuremath{KG}}
\newcommand{\integers}{\ensuremath{ \mathbb{Z}}}
\newcommand{\rationals}{\ensuremath{ \mathbb{Q}}}
\newcommand{\series}[3][x]{\ensuremath{#1_{#2}, \ldots, #1_{#3}}}
%\newcommand{\comdiag}[4]{%
%\begin{center}%
%\(%
%\begin{array}{ccc}%
%#1&\longrightarrow{}&#2\\%
%\downarrow&&\downarrow\\%
%#3&\longrightarrow{}&#4\\%
%\end{array}%
%\)%
%\end{center}}
\newcommand{\comdiag}[4]{%
%\begin{pspicture}(0,0)(2.5,5)%
\begin{psmatrix}%
$#1$&$#2$\\%
$#3$&$#4$\\%
%\psset{arrows=->}%
\ncline{1,1}{1,2}%
\ncline{1,2}{2,2}%
\ncline{2,2}{2,1}%
\ncline{2,1}{1,1}%
\end{psmatrix}%
%\end{pspicture}%
}
	



\begin{document}



\degree{\MSc}%

\dept{Mathematics}%
%

\field{Algebra}%
%
\permanentaddress{93 Fonda Green \\
Calgary, AB \\
Canada, T2A 5S4}%
%

\examiners{S. K. Sehgal, Gerald Gliff, A. Weiss, A. Al-Sallam}

\convocationseason{Fall}%
%

%BeginExpansion
%\frontpiece{THE UNIVERSITY OF ALBERTA



\title{The Unit Groups of Certain Group Rings}
\author{Brett G. Giles}
\admin
\doublespacing
\begin{abstract}
This thesis deals with the problem of describing the unit group of specific group rings over the integers. 
After a brief introduction to remind one of some of the properties of the group ring we start to discuss 
the unit groups. A few of the basic results as presented by Sehgal [6], Chapter 2, are shown. 
After this introduction we talk about specifics.

The first method to determine a unit group is then discussed. It is a general method, applicable to
any group.  However, in practice we see that it is unsuitable for any but a small number of groups.
In this section we talk in an expository manner as the proofs of the results are normally very
dependant on the particular group. The ones presented by this method later on are \(S_3, D_4, D_6\)
and \(A_4\).

Next, we present the method for groups of order $p^3$, where $p$ is an odd prime. These come from
the paper by Ritter and Sehgal [5]. In this paper they also present a method for determining the
unit group of a particular group of order $p^n$. This will not be treated here. I consider both
non-abelian groups of order $p^3$ and descriptions of the unit groups of both of their respective
group rings are presented. Later on in the paper I present the method as applied to the groups of
order 27.

The last theoretical results are on determining the unit group of the group ring over a group of
order $pq$ where $p\equiv 1(\mod q)$. These results come from a paper by Luthar [3]. No practical
examples are done of this method.

The next part deals with presenting actual groups and determining the unit structure of their
integral group ring. The first two, $S_3$ and $D_4$ are from previous authors. The first was done by
Hughes and Pearson [2], the second by C. Polcino-Milies [4]. The next, $D_6$, is new. The last one
is merely an expository account of Allen and Hobby's [1] rendering of \( \mathcal{U} ( \mathbb{Z}
\mathrm{A}_4)\). Our other concrete examples deal with the two non-abelian groups of order 27 as
presented in a paper by Ritter and Sehgal [5].  
\end{abstract}

\onehalfspacing	%everything from here on will be 1.5 spaced


\tableofcontents
\listoffigures
\listoftables

\bodyoftext 
\chapter{Preliminaries} 
\section{Generalities} 
Throughout this paper the term \emph{group ring} shall be taken as follows: The group ring $KG$ of
the group $G$ over the ring $K$ (which posesses an identity) is the ring of all formal sums
\[\alpha=\sum \lambda(g)g \] where \(g \in G\) and \( \lambda(g) \in K\) so that
 \(\supp(a)= \{g \mid \lambda(g) \neq 0\}\) is a finite set.

The ring structure is inherited from the structures of the group and the ring. The operations on 
\(KG\) are defined as follows:
\begin{enumerate}
\item \(\sum\lambda(g)g = \sum\mu(g)g \iff \forall  g \in G \lambda(g) = \mu(g).\)
\item \(\sum \lambda(g)g + \sum \mu(g)g = \sum(\lambda(g)+\mu(g))g.\)
\item \( \sum \lambda(g)g \centerdot \sum \mu(g)g = \sum \nu(g)g \)
\end{enumerate}
where \( \nu(g) = \sum \lambda(x)\mu(y) \) with the last sum being over all  \( (x,y) \in  
G\times  G \mathrm{with} \  x\centerdot y = g\).

In this paper, we are interested in a particular object that occurs in a group ring. This is the
\emph{unit group}. The unit group of a group ring \KG\ is the group of all elements invertible under
the multiplication of \KG. (It should be noted that the multiplicative identity in \KG\ is the
element \(le\), where \(1\) is the multiplicitive identity of \(K\) and \(e\) the group identity.)
Obviously, the unit group of \KG\ contains \( \pm G\). More extensive research becomes increasingly
difficult.

\section{Notation}

Throughout this paper the following will be assumed.
\begin{description}
\item[\integers{}:-] the ring of integers.
\item[\rationals:-] the ring of rational numbers.
\item[\( R_n \):-] the ring of \(n\) by \(n\) matrices with entries from \(R\).
\item[\(< c > \):-] The group generated by the element \(c\).  (There may be be more than one element 
for generating.)
\item[\(\Delta (G,N)\):-] \(< x-1 : x \in N> \mathrm{ in }\ \integers G\) where $N$ 
is normal in $G$. This also is \[\{\sum_{g \in G} \mu(g)g 
\in G : \sum_{x \in N} \mu(gx) = 0  \forall  g \in G\}\].
\end{description}

\section{General results}

In this section, I will present a few of the results from Sehgal [6] in order to give the reader a
feel as to what the unit group of a group ring may become at times.

\begin{proposition}

If $G$ is abelian of order $n$, then \[\mathscr{U}\integers{}G = \pm{}G\times F\] where $F$ is a
free abelian group of a determinable order. This order is dependent on the number of cyclic
subgroups of various orders in $G$.
\end{proposition}

The above theorem helps a great deal when dealing with groups, as it is often possible to get a
factor or sub-group of your group to fall in this particular category. In particular, this
proposition can be used to show that the unit group of \(\integers{}<x>\) where \(x^2=1\) is just
\(\pm{}<x>\) This fact is used later.

This proposition along with others can be used to prove the more powerful theorem that is stated now.


\begin{theorem}

If $G$ is a torsion group then \(\mathscr{U}\integers{}G = \pm{}G\) if and only if $G$ is one of
\begin{enumerate}
\item an abelian group with \(G^4 = \{1\}\) or
\item an abelian group with \(G^6 = \{1\}\) or
\item a Hamiltonian 2-group.
\end{enumerate}
\end{theorem}

This theorem completely characterizes torsion groups that have trivial
\(\mathscr{U}\integers{}G\). (A Hamiltonian 2-group is a group of the 
form \(E \times\)quaternions where \(E^2 = 1\)).

In conclusion, we note that there are many different areas one can explore when determining unit
groups.  These range from finding unit groups of particular integral group rings to determining when
torsion-free groups have trivial unit groups in their integral group ring.

\chapter{Theoretical considerations}
\section{Representation theory method}

This method enables us to give a concrete description of the unit groups of certain group rings
involving \integers.  The first two, $S_3$ and $D_4$ were done by Hughes and Pearson, and Polcino-Milies,
respectively. The third, which is $D_6$, was done by myself and involves a minor extension to the
method. The last one, $A_4$, which is included only in an expository way, was done by Allen and
Hobby. The general manner in which to apply this method is described below.

Consider the group G. We may use representation theory to determine its non-equivalent irreducible
representations. Call these $\theta{}_i$. These will be maps from \(\rationals{}G\) to matrices over 
\rationals.

If one takes these \(\theta{}_i\) that we have obtained, one may now define a map \(\theta:
\rationals{}G \rightarrow \rationals_{i1}\bigoplus\ldots\bigoplus \rationals_{in}\), by, if
\(a \in \rationals{}G\), then \(\theta(a) = (\theta_1(a),\ldots,\theta_n(a))\). 
Considering both sides of the mapping as vector spaces over \rationals, it is readily seen that
$\theta$ is a linear mapping. Let $A$ be the matrix of $\theta$. It will be noticed that
\(\theta\integers \subset \bigoplus\integers_n\). Next, by using the matrix $A$ and its inverse we
will be able to deduce a system of linear congruences that give us the restrictions needed for an
element of \(\bigoplus\integers_n\) to be in \(\integers{}G\). These, together with the fact that a
matrix with coefficients in \integers{} has to have an integral determinant in order to have an
integral inverse determine the proper group.

Naturally one sees that a significant problem with this method is the size of the matrix $A$
involved. It is a square matrix of size \(o(G)\). Another problem is that we have no guarantee that the
system of congruences will lead to a usable situation.

A further difficulty with the method is that the end result may turn out to be a direct product of
matrix rings.  In this case, it is just as difficult to determine properties from this description as
it was from the original.

Despite these difficulties, the method is sufficiently useful enough to apply it to a few groups of
small size.

\section{Second Method - Groups of order $p^3$}

In this section, we intend to study the unit groups of the integral group rings of groups of order
\(p^3\), where $p$ is an odd prime.  The first type, the commutative groups of order \(p^3\), are uninteresting
as their unit groups are just + G.

In dealing with the non-commutative groups of order \(p^3\), we note that there are two non-isomorphic
groups of that order. They are:
\[H=<a,b| a^{p^2}=e=b^p, b^{-1}ab=a^{p+1}> \mathrm{and}\]
\[G=<a,b,c| (a,b) = a^{-1}b^{-1}ab = c, ca=ac, cb=bc, a^p = e = b^p = c^p>.\]

Throughout this section we reserve the letters G and H to mean these two groups.

\subsection{Fibre product}

We must now define a concept that we will be using throughout this section --- that of the fibre
product.

To understand what a fibre product of rings is, consider the ring $R$ with the two ideals $I,J$ of $R$
such that $I\cap{}J = 0$.  Then we have the diagram
\begin{center}
\(
\begin{array}{ccc}
R&\longrightarrow{}&R/J\\
\downarrow&&\downarrow\\
R/I&\longrightarrow{}&R/(I+J)\\
\end{array}
\)
\end{center}

Then $R$ is the fibre product of $I$ and $J$ in the sense that
\[R\cong\{(\alpha,\beta)| \alpha{}\in{} R/I, \beta{}\in{} R/J, \hat{\alpha}=\tilde{\beta}\},\]


where $\hat{}$ is the map from $R/I$ to $R/(I+J)$ and $\tilde{}$ is the map from $R/J$ to $R/(I+J)$.

From the above we can deduce a fibre product of the unit groups as is given by the following diagram.
\begin{center}
\(
\begin{array}{ccc}
\mathscr{U}(R)&\longrightarrow{}&\mathscr{U}(R/J)\\
\downarrow&&\downarrow\\
\mathscr{U}(R/I)&\longrightarrow{}&\mathscr{U}(R/(I+J))\\
\end{array}
\)
\end{center}

\subsection{Use of Fibre Product}

In this section, we are going to apply this to the Group Ring
\(\integers{}X\) with $J = \Delta(X,N)$ as the kernel of the natural
homomorphism of $\integers{}X\longrightarrow{}\integers{}X/N$ with $N$ normal in $X$ and 
\(I = \hat{N}\integers{}G\) where \(N=\sum_{x\in{}N}x\). From here on in, we shall write $\hat{x}$ 
 for $<\hat{x}>$.

\subsubsection{Pseudo-diagonals of matrices.}
	
To present the proofs of this section, we will need to number certain matrices by their
pseudo-diagonals. If the $n\times{}n$ matrix $A=[a_{ij}]$ is considered, then the $j$-th
pseudo-diagonal is given by the elements \[a_{1,j+1}, a_{2,j+2}, \ldots, a_{n-1,j-1}, a_{nj}\] where
the second subscript is considered $modulo\ n$ and $j=0,1,2,3,...,n-1$.

Some of the matrices that we will be dealing with from here on will be numbered via their
pseudo-diagonals.  As an example the matrix \(B = [b_{i,j}]\), if numbered by pseudo-diagonals means
that the element $b_{i,j}$ is located on the $i$-th pseudo-diagonal at the $j$-th spot. When using
this indexing scheme, we have $0\leq{}i,j\leq{}n-1$.

For convenience, since we will be dealing with many diagonal-like matrices, we introduce the
following notation: \[A =  PDIAG_i(\series{0}{n-1})\] will represent the $n\times{}n$ matrix that has 
the elements \(x_0,\ldots,x_{n-1}\) on the $i$-th pseudo-diagonal and
zeroes elsewhere. If we are talking of the $0$-th pseudo-diagonal, we then mean the main diagonal and
refer to it as \[A =  DIAG(\series{0}{n-1})\] 

\subsection{Preliminary propositions}

Throughout this section we let $\omega$ denote a primitive $p$-th root of unity.

\begin{proposition}
Suppose \(\series{0}{p-1}\ \in\ \integers[\omega]\) then there exists \(t_i\ \in\
\integers[\omega]\) satisfying
\[\sum^{p-1}_{i=0} t_i\omega^{ij}=x_j, 0\leq{}j\leq{}p-1\] if and only if
\[\sum^{p-1}_{i=0}x_i\omega^{ki} \in{} p \integers[\omega]\ \forall\ 0\leq{}k\leq{p-1}\]
\end{proposition}

\begin{proof}
The system of equations is equivalent to
\[[\series[t]{0}{p-1}]W = [\series{0}{p-1}]\]
where $W= [a_{kl}]$, with \(a_{kl} = \omega^{(k-1)(l-1)}\). Now as $W$ is a character matrix, the
orthogonality relations of a primitive root of unity tell us that\[W^{-1} = \frac{1}{p}[a_{kl}^{-1}]\].

Our system is equivalent to \[[\series[t]{0}{p-1}] = [\series{0}{p-1}]W^{-1}.\]


Therefore, there exists a solution if and only if we have
\[\frac{1}{p}\sum_{i=0}^{p-1}\omega^{ik}x_i \in\ \integers[\omega]\] for all \(0\leq{}k\leq{}p-1\).
This, of course, is the same as saying \[\sum_{i=0}^{p-1}\omega^{ik}x_i \in\
p\integers[\omega]\forall{}\ 0\leq{}k\leq{}p-1\].
\end{proof}

\begin{proposition}

Let $A$ and $B$ be $p \times\ p$ matrices over $\rationals[\omega]$, with $A = DIAG(1, \omega, 
\omega^2, \ldots, \omega^{p-1})$ and $B$ as the matrix with ones
on pseudo-diagonal number one, (where numbering of pseudo-diagonals start at zero) and zeroes
elsewhere, that is $B = PDIAG_1(1, 1, \ldots, 1)$.

Then the $\integers[\omega]$ span of the matrices $\{A^iB^j, 0\leq{}i,j\leq{}p-1\}$ consists of all elements of the form $M= [x_{ij}]$,
where the numbering is via the pseudo-diagonals, and such that for each $j,k$ with  $0\leq{}i,j\leq{}p-1$
 we have\[\sum_{i=0}^{p-1}\ x_{ji}\omega^{ki}\  \in\ p\integers[\omega]\].
\end{proposition}

\begin{proof}

We see that for a fixed $j$, the matrices $\{A^iB^j, 0\leq{}i,j\leq{}p-1\}$, have non-zero entries only in the $j$-th
psuedo-diagonal. The vector (\series{0}{p-1}) in $\integers[\omega]$ is a diagonal in the span of $\{A^iB^j\}$ if and only
if there exists $t_i\ \in\ \integers[\omega]$ such that
\[\sum_{i=0}^{p-1}\ t_iA^i\ = \ DIAG(\series{0}{p-1})\]

This, we see, means that\[\sum_{i=0}^{p-1}\ t_i\omega^{ji}\ = \ x_j, 0\leq{}i,j\leq{}p-1 \]
and now applying the last proposition we now have our answer.
\end{proof}

\begin{proposition}

Let $o_1 \subset\ o_2$ be \integers-orders in a rational algebra. Then, if an element $\alpha\ \in\ o_1$
 has  an inverse in $o_2$,
then $\alpha$ already has an inverse in $o_1$.
\end{proposition}
\begin{proof}

As groups, we have that the indices behave in the following manner:
\[(0_2\ :\ \alpha{}o_1)\ =\ (\alpha{}o_2\ :\ \alpha{}o_1)\ \leq{}\  (o_2\ :\ o_1)\]
which implies that $\alpha{}o_1 \ = o_1$, and therefore shows us that $\alpha$ is a unit in $o_1$.
\end{proof}

Now we digress before continuing with our list of propositions.  Let us recall our group $H$ with the
two generators $a$ and $b$. Let us write $c = a^{-1}b{-1}ab = a^{p^{2}}$.
Then we have $H' = <c>$ of order $p$. Thus $\tilde{H} = 
H/<c>=<\tilde{a}> \times <\tilde{b}>$. As before, $\omega$ is a primitive $p$-th 
root of unity.  Then \[\rationals{}H \cong{} \rationals{}\tilde{H} \oplus 
\rationals{}(\omega)_p\] 
As well as this we have 
\[\rationals{}\tilde{H} \cong{} \rationals{}H/\Delta(H,<c>) \cong{} 
\rationals{}H\hat{c} \]and
\[\rationals(\omega)_{p} \cong{} \rationals{}H/\hat{c}\rationals{}H.\]

Clearly this gives us
\[\integers{}H / \Delta{}(H,<c>) \cong{} \integers{}\tilde{H} \] and the mapping
\[\integers{}H \rightarrow\integers\tilde{H} \oplus \integers[\omega{}]_{p} ,\]
with the projection onto the first component. Let us proceed with the calculation of the map with
respect to the second component. We easily see that
\[\tilde(c)\integers{}H +(1-c) \integers{}H = p\integers{}H + (1 - c)\integers{}H\]
as $c$ is the sum of $p$ elements all of which are units of \(\integers{}H\). (They are contained in $+ H$). 
Also, it is obvious that
\[\tilde{c}\integers{}H \cap (1-c) \integers{}H = 0 .\]

Thus, it is obvious that we have the fibre product
%\comdiag{\integers{}H}{\integers{}\tilde{H}}%
% {\integers{}H/\tilde{c}\integers{}H}{\integers{}\tilde{H}/p\integers{}\tilde{H}}
\comdiag{1}{2}{3}{4}
with all the maps being natural.


The $pXp$ matrices \(A =  PDIAG1(1,1,\ldots{},1,\omega{})\) and
$ B = DIAG(1,\omega{},\omega{}^2,\ldots,\omega{}^{p-1}$ obviously satisfy the
relations $A^p=\omega{}I, B^p = I, B^{-1}AB = A^{p^{2}+1}$.

The matrices $ \{B^jA^i | 0\leq{}i,j\leq{}p-1\}$ are linearly independant over 
$\integers{}[\omega{}]$ as is shown in the following: Assume 
\[\sum_{i,j}z_{ij}B^iA^j = 0\].
As $A^j$ has non-zero entries only  in the $j$-th pseudo-diagonal we have
\[\sum_{i}z{ij}B^iA^j = 0 \forall{} j, 0\leq{}j\leq{}p-1.\]
As $A$ is a non-singular matrix we have that
\[\sum_{i}z_{ij}B{i} = 0,\]
which immediately implies that $z_{ij} = 0 \forall 0\leq{}i,j\leq{}p-1$.

Let $T=\integers{}H/\hat{c}\integers{}H$, and $S_p$ be the
$\integers{}[\omega{}]-$span of the matrices $\{B^iA^j | 0\leq{}i,j\leq{}p-1\}$,
from above. The claim here is that $T \cong{} S_p$. Consider the map
\[\phi{}: \integers{}H \rightarrow{}S_p, \phi(a) = A, \phi(b) = B.\]
As $\hat{c} = (1+c+c^2+...+c^{p-1})$ is mapped by $\phi$ to 
$(1+\omega{}+\omega{}^2+...+\omega{}^{p-1})I$, which is zero, we have the induced map
$\phi{}_0: T \rightarrow{} S_p.$


Since $\phi{}(a^{pk}a^ib^j) = \omega{}^kA^iB^j$, we have that $\phi{}$ is onto $S_p$. 
Also we see that $\phi{}_0$ is $1\leftrightarrow{}1$, for if we tensor
both $T$ and $S_p$ with $\rationals{}$ we see that both of them have $\integers{}-$dimension
 of $p^n-p^{n-1}$

\begin{proposition}
A matrix $Z \in{} \integers{}[\omega{}]$ is in $S_p$ if and only if the matrix $X = Z'$ satisfies
\[\sum_{i=0}^{p-1} x_{ji}\omega{}^{ki} \in{} p\integers{}[\omega{}], 
\forall 0\leq{}j,k\leq{}p-1.\]
where $Z'$ is obtained from $Z$ by dividing all the entries below the main diagonal by $\omega{}$.
\end{proposition}
\begin{proof}

We make the observation that $A^i = PDIAG_i(1,1,\ldots,1,\omega,\ldots,\omega)$, 
where $\omega$ is repeated $i$ times. Thus, to compute $S_p$ it is enough to
find the span of $\{B^jA^i | 0\leq{}i,j\leq{}p-1\}$ separately for each $i$. 
Therefore, all we need do is to find all $\integers{}[\omega]$ vectors of the form 
$(z_{i0},\ldots,z_{ip-1})$ such that
\[\sum_{j=0}^{p-1} t_jB_jDIAG(1,1,\ldots,1,\omega,\ldots,\omega) = DIAG(z_{i0},\ldots,z_{ip-1})\]
which is equivalent to
\[\sum_{j=0}^{p-1} t_jB_j = DIAG(z_{i0},\ldots,z_{ip-i-1},\omega^{-1}z_{ip-i},\ldots,\omega^{-1}z_{ip-1})\]
and the result therefore follows from proposition 2.
\end{proof}

Now let us consider a fibre product of $\integers{}H$. We have the diagram










with all the maps natural, ~ being used to map T to $S_p$. Also,if we let the map from T to (~/p~)H be
called O1~ then let the map ~~:$S_p$ ('ZL/p'ZL)H be defined as ~ ~ Then the diagram above is
commutative.

Consider ~ If MESP we wish to write M as Xa~BiAi, a~~E'Zli[$\omega{}$], O~<i,j~p-1. Let M' be obtained from M
by dividing all the elements below the main diagonal by ~ Then the j-th pseudo-diagonal Xj,~,... ~
of M' is the same as the main diagonal of ~

We then have


W =


where W   [$\omega{}$1.,~], ~ = ~ O~<i,j~p-1.


	~ = (1/p)~ ~
k


Recalling that we have


M = ~,~BiAi

then from the above commutative diagram we have


~1(M) =
10


where ~ is obtained from ~ by takingw = 1 and going mod p.


In consideration of Proposition 3, we have the following theorem.

\subsection{Main result for type 1 groups of order $p^3$}

\begin{theorem}
a)

b)
(a,M) E 'ZLH X	I M' satisfies (*) and 02(a) = ~1(M)~.

(a,M) E ~ >( ~ M is a unit of ~ M' satisfies (*) and

02(a) = ~1(M)1.
\end{theorem}

In the theorem we have,
\begin{itemize}
\item{} M' is obtained from M by dividing every element below the main diagonal by '~, where a~ is a
primitive p-th root of unity.
\item{} The condition (*) is the one we have encountered many times already

p-i
~	E ~ 0<j,k~<p-1, ~=1.
i--O

where ~XjJ are numbered according to the pseudo-diagonals of M'.

\item{}  02: ~H   (71�/p'ZL)H is the natural map Mod p.

\item{} ~1(M) =x ~~biAi where ~ = (1/p)~wikxjk E ~[~] and   is 
obtained from a~, by
	i,j	k
putting ~ = 1 and going mod p.
\end{itemize}


\subsection{Groups of type 2 of order $p^3$}
	We now consider our second group of order 3 which we refer to as 
G. In order to
remind you of what it is, we include the definition of it one 
more time.


	G= <a,b,c I (a,b) = a-'b1ab = c, ca=ac, cb=bc, a~ = e = b~ = cP>.


		We note that the factor commutator group, G = G/<c> is elementary
abelian of order p2, G = <a> X <b>. This gives us the 
decomposition

~	\rationals(w)~.

In fact, we have


(~G  ~G/A(G,<c>)=\rationals Gc and
\rationals G/c\rationals G.


Clearly this gives us
\rationals G ~~i1 e FIXME ~ ~ by, if6511


~GIA(G,<c>)='ZLG and the mapping



with the projection onto the first component. Our next step Will 
be, as before, to compute the
the projection into the second component. We consider the fibre 
product diagram


	~G~'ZL1G



	'llaGk~G-~ ('ll /p~)G



where all the maps are the natural projections excepting the map 
on the bottom which is


O1:~G/c~G   (~Ip'ZL)G with
O1(~zciaibk) = ~zajbk, with Z E



Twisting of Group Rings
	In order to continue, we need to introduce the notion of twisted 
group rings at this
point. A twisted Group Ring is constructed in the same manner as 
an ordinary group ring
except that the definition of multiplication differs. In the 
twisted group ring, there is a
twisting factor that is used to commute elements.


	Let us take the twisted group ring of a ring R and a group G. 
This is then written as
R0G. The twisting may (and quite often is) be multiplication by a 
particular element of R.
Therefore, if a,b are in G and r is in R, we may define the 
multiplication of a and b in the
twisted group ring as ab  rba.


	Using the notation we have just introduced, it is easy to see 
that ~Gk~G is
isomorphic as a ring to the twisted group ring ~ [wi0 G with ba = 
wab. After this
identification, we see that the map Ol may be written as


O1(~aaibi) = ~ a~bi, ~
where we get a from a by substituting w = 1.


	Let us now define the map ~ from 'ZL[~]0G to 'ZL[wJ~ by


	a~ A = DIAG(l,w,...,~1), b~ B = PDIAG1(l,...,1).
We note that BA = wAB and that AP = I = B~.
12


	We conjecture that {AiBi~ is a linearly independent set over ~(w~ 
for O~i,j~p-1. This
can easily be seen as follows. Let ~ be in ~ for O~i,j~p-1. Then 
we see that

A~Bi = 0 implies~~~,~AiBi = 0,

as Bi has non-zero'elitries only in the j-th' diagonal. Since B 
is non-singular, we may remove
the Bi from the above result, which implies that   = 0 for all 
0~<i,j~p-1. Therefore, we now
have that if $S_p$  span{AiBi, O<ij~~p-l~, then
	~Gk'ZLG ~	$S_p$
From proposition 2 it follows that

p-i
	$S_p$ =	E ~[~J~I M satisfies~ ~ ~ p'ZL[~] for all O~ij~p-l~.
i=O


Let us consider the following fibre product diagram and extension 
to $S_p$.




(~p~)G
$S_p$
ifi

In the above, the map from $S_p$ to (~/p~)G is denoted by ~ and the 
map from ~G/~~G to
$S_p$ is denoted by ~o. Obviously, as before in a similar diagram, 
we define ~ to be the map
O1~61.


As before, given M in $S_p$ we wish to find ~1(M) E ~[w]0G. Let M = 
{x~,~} be
numbered by the pseudo-diagonals. We wish to find ~ [~] such that 
X~j a~~.AiBi = M. It is
necessary to find ~ such that

~a~~Ai = ~ 0~j<p-1.


Again as before we get this is equivalent to the matrix equation

~aoj,... ~ = [x~,0,... ~
where W = [w~~] numbered by columns and rows starting at zero and 
~ = ~ Again, by
following the previous procedure, we see that

p-i
(1/p)~ ~
From the above we have that    k=O
13


~'(M) =~1~a'bi and ~1(M) =~aija'~.
	1,J	1,]



Where a, ~is obtained from aiJ by substituting ~   1 and going 
mod p.


	The above constitutes the first part of the next theorem. The 
second part follows
directly from proposition 3 as $S_p$ is an order in ~fwJ~.



Main result for type 2 groups of order \(p^3\)
Theorem
a)

b)
{(a,M) E 71�G)<~[~~ I M' satisfies (*) and 02(a) =
{(a,M) E ~~GX~[w]~ M is a unit of ~ M satisfies (*) and

02(a) = ~1(M)}.
In the theorem we have,

(i)	02: ~G~('ZL/p'ZL)G is the natural map mod p;

(ii)	The condition (*) is the one we have encountered many times 
already


p-i
		E p'lL[w], O<j,k<p-1, ~
i=O

	where  ~ } are numbered according to the pseudo-diagonals of M.

(iii)	4)1(M) =~a~~aibi where aiJ = (1/p)~~kix~k E ~[w] and ~ is 
obtained from aiJ by
putting ~ 1 and going mod ~  k



Concluding remarks on groups of order \(p^3\)
	This concludes our section on groups of order p8 It should be 
noted that there are
striking similarities in the derivations for both types of 
groups. However, the proofs do need
to be presented separately due to the underlying differences. In 
a paper by Dr. S. Sehgal and
Dr. Jtirgen Ritter, the method presented here for type 1 groups 
of order \(p^3\) is extended to
similar type groups of order pfl.


	At the end of this paper, I will present examples of the methods 
shown in this section,
using groups of order 27.



C.	Third method - Groups of order pq
Generalities
	In this section, we shall study the unit group of groups of order 
pq, where p and q are
primes p 1 (mod q). We shall restrict our attention to the 
non-abelian group of this order. In
order to refresh your memory, we have
14



	G = <a,b I aP = bq,bab' = ai, j 4- 1 (mod p), jq   1 (mod p).

In this case, we will need to consider the set of normalized 
units of 'iliG given by the map


~(a) = 1 , ~(b) = b.


	The kernel of this homomorphism is obviously those elements of ~ 
for which the
coefficient of ak is 1 and the power of b is 0. Let us denote the 
kernel by ~. This set shall be
the group of normalized units of ~G. We note that any unit of ~G 
can be written as the
product of a normalized unit and a unit of 'ZL[b].


	Now since the units of ~[b] are known to us and we have the 
equation


	ba = aib

we only need to determine the set of normalized units.


	Let w be a primitive p-th root of unity. Then the field k = \rationals(~) 
is a cyclic extension
of \rationals of order p-i. Let k~ be the fixed field of the automorphism 
t:w   w~. Then k0 is of degree
(p-1)/q over \rationals. For any element a of k let


at, a(2)t, . . , ~(ql t

be the succesive applications of the automorphism t.


	Let R and R0 denote the rings of integers of k and k0, 
respectively. We note that R is
a free R0-module with the basis


	1, x,x2,
where

=	- 1.

is the prime	in R over the rational prime p. The corresponding 
prime in R0 is
Xo = (w-1) ~ .

As (wi-1)/(c~~-1) is a unit, we have that Xo = ~ as ideals.


Recall from number theory that in the above situation we have that
=~ RoIxoR0 = R/xR.

Therefore, in particular, we have that each element of R modulo x 
is congruent to a rational
integer.


	At this point, we present a lemma.
Lemma.

Suppose aER, with a-s (mod x), with 5 t ~. Then a can be written 
uniquely as
a = ~ +ap~1~P~1.

withXa~ = 5, and the a~E ~.
15


Proof.

Write


a = c0+c1w+c2w2+...+c~1w~-1

where the c~ are rational integers. Since a   S (mod x). we 
therefore have
	(t)	%+c1+...+%1=-s(mod p)


In full generality the first equation may be rewritten as

	a = (c0+m)+(c1+m)w+(c2+m)w2+...+(c~1+m)WP1

where m is a rational integer. The sum of the new coefficients is 
5 if and only if

=	5.

Considering (t) above, the relation gives us a unique value for m 
such that the equality does
hold.



The unit group as matrices over R.

	Recall that it is obvious that 'll'G   'lla[a,b] and, therefore, 
an element x of ~G can
be written as


x(a,b) = x0(a) + xi (a)b + ... + xq~1(a)b~~l

where x~(a) is an element of 'ZL[a]. Therefore, if x,y,z are in 
'ZLG, written as above, and

z   x~y, then we would have (upon recalling the definition of 
multiplication in a group ring)
the equations


z0(a) = x0(a)y0(a) + x1(a)yq~1(ai) +... + xq~1(a)y1(ai~')

z1(a) = x0(a)y1 (a) + x1(a)y0(ai) +... + xq~1(a)y2(ai~)





Zq~1 (a) = x0(a)y\rationals l(a) + x1(a)yq~2(ai) + ... + xq~1(a)y0(ai~~)


		Recalling that ~[a~ =~ 'ZLf XJ/(XP-1) we can therefore associate 
to an
element x(a,b) of 'ZLG the elements

	a0 = x0(~), a1 = x1(w),..., aq~1 =
with a~ in R.


	Consider the matrix
16
A=
a0, a1, . .  ,aq~1
	at	at	at
	 q-1'	 0'	' q-2



with entries in R. We sha call matrices with 1 is form matrices 
of type 1.


	From our previous calculations, we see that the obvious map 
x(a,b)~A is a
homomorphism from ~G into the matrices of type 1.


	Let us consider what happens when A is invertible in R. Denoting 
the first row of A.1
by ~ and we would then have the system of equations

~0a0 + ~1atq~1+...+flq~1alt~l}t=l
~a1 + ~1at0 + ~~1a~l)t-O





~0a~~1 + ~1atq~2 + ~
If we apply the automorphism t:~~w~ to these relations 
successively we see that


~1'
	~q~1' ~		,~qt2
	a
	a	a






is once again of the same type. Therefore, our conclusion is that 
the invertible matrices of the
type 1 form a group.


	Let us restrict the homomorphism from ~G to type 1 matrices down 
to ~, the group
of normalized units of 'lLG. Let us further restrict it to 
matrices of type 1 that satisfy the
following conditions:


(i)	a0=~l,a1=O,...,aq4=-O (mod x) or A-I (mod x).

(ii)	det A is a unit in R0.
17




	We conjecture that the homomorphism is actually an isomorphism. 
First, we show
that it is one-to-one. Let x(a,b) be in ~, with x(a,b) being 
mapped to the identity. Then


x0(w)=l, Xi(W)=O,...,Xqi(~))=O



Since x(a,b) is a normalized unit we also have


which means that

x0(1)=l, Xi(l)=O,...,Xqi(l)=O
x0(a)=1, x1(a)=O,...,xqi(a)=O.


That shows that the map is one-to-one.


	Now we turn our attention to the matter of onto. Let A be 
invertible of type 1 and
satisfy conditions (i) and (ii). Let B be the inverse of A. Let


	a0 = x0(w), a1 = x1(~),...,aq~i = Xq~i(w)


where the x~(X) are polynomials with rational integer 
coefficients of degree ~ p-i, the sum of
the coefficients of X~  1, or =0, corresponding to i=0, or i= 
1,2,...,q-1. Form the element


	x(a,b) = x0(a) + x1(a)b + ... + xq1(a)b~l
and the corresponding element for B


	y(a,b)   y0(a) + y1(a)b +...+ yq1(a)b(1l


which is	derived in a similiar manner.
Since AB = I we have


x1(~)ym(~) = 1 or 0
1+m=-i(mod p)


according as to i=0 or i=1,2,3,...,q-1.
Also we see that


x1(1)ym(1) = 1 or 0
1+m~i(mod p)


according as to i=0 or i=1,2,3,...,q-1. Therefore, x(a,b)y(a,b) = 
1. In the same manner as
M18


above we see that y(a,b)x(a,b) = 1.


	Therefore, we have proven that the group ~ of normalized units of 
'lLG is isomorphic
to the type 1 matrices in R that satisfy conditions (i) and (ii).

The unit group as matrices over R0
	In this section, we continue from where we left off at the end of 
the last section and
extend our description of ~.


	Let us put



-Xql+~1Xq2+...+~1.


Also let





Since we obviously have X-x=-X(mod x), we have that


Nk/k,(X-x) =- X~(mod Xo),



and, therefore, if we compare coefficients we will have
	(~)	x~ (mod Xo)


If one lets 6~ = 1 then let



P
lx

1 Xt	. (~t)ql



With some work, we can see that P~ is [p~~] ~ where the numbering 
is by rows and
columns and ~ = ~ where a~ is, of course, just a.


	Let E = PDIAG1(1,1,...,1) be a q X q matrix. It is quite obvious 
that

E-1 = ~  Eq-i. For a matrix M with entries in k, we shall denote 
by M' the
matrix obtained from M by applying the automorphism t to the 
entries of M. In consideration
of this, it is obvious that P' = EP, and (P~1)' = P-1E-'. 
Therefore if A has its entries in k, then
P-'AP has entries in k0 if and only if
19


A'  E A El.

This is equivalent to the matrix A being a type 1 matrix from the 
previous section. That is


a0, a1, . .  ,aq~1
		at	at
	atq1,	 0,	' q2
A






where the elements are in k.


It then follows that the map from A to X = P-i A P is an 
isomorphism of the ring of

matrices A of type 1 having entries in k with the ring of q >( q 
matrices X with entries in k0.


Let X   [x. .J O<i,j~q-1. Suppose that X with entries in R0 
satisfies the congruence
	(~)	x=-	[1	(mod ~~(=~~-l)).



Then the entries in the first row of the corresponding matrix A = 
P X P~l are


aEE = (1/~)[ x0(xA)(~q~i) + xl(X)(~q~2) +... + xq~i(x)(~o)l =
a1 = (1/~)[ xo(X)(~q~i)t + x1(x)(~q~2)t +... + xq~i(X)(~o)t )





	aqi = (1/~(q~l)t)[ x0(x)(~q~1)(~~l)t + x1(X)(~q2)((1l)t +...+ 
xq~1(X)(~0)(~l)t J = ~q~l/~(q~i)t.
where for O~i~<q-1,


x~(x) = x0,~ + x1,~x +...+ xq1,jX~l=~Xi(mod ~~+1),


This implies, if we consider the congruences involving the ~~'5 
and the congruences that one
can derive by successive applications of the automorphism t to 
them we have


	~= qxq-1 (mod x)
and


=	(x(i)t)q~l + ~(~(i)t)q~2 + . + ~q~l(~(i)t) + ~~-l (mod x)
y 20




	Since ~ and its conjugates via the isomorphism t are all 
associates of ~ this means
that the a~ are all elements of R. As well, (mod x) we will have 
the following holds true:


=	[l~(Xt/X)] [1~(X(2)t/X)] ~. [1-(X(\rationals1 t/x)J



Therefore, we have that
	a0 = (~/~	1 (mod x).


Since
	(~t/~)q - 1 = [(wi~1)/(w~1)]~~1	j~-1	0 (mod x),

therefore, (~t)q~~q 0 (mod ~~+l), and therefore, noticing that 
Xt-X is an associate of x, we have


0 (mod ~


giving us the relation
	a1 =	=- 0 (mod x).


Similarly, we can continue to show that a~=O (mod x) for 
2~~i~~q-1.


	Now let us do the converse. Suppose we have a matrix A of type 1 
with entries in R
that satisfy the conditions


c~~=-0 or 1 (mod x)

depending as i=0 or i=1,2,...,q-1. Let X = P-i A P. Number the 
matrix X as above, X=~x~,~]
0~i,j<q-1, the numbering according to columns and rows. In 
consideration of the above, we
note that


=	~  ~ ~~)(u)t(xi)(v)t
u,v


=	~ (~ ~ +~(xi)(v~u)t]) (u)t
U  V




= U~ (~X[(~qi~i/~)av(xi)(v}tJ ) (u)t







where Tr is the trace of k over k0. Since ~ is the different of 
the extension k over k0, we have
that ~ is in R0. As well, in view of the congruences involving 
the ~i and the a1, we would have
21


that if j>~i,


(Xi)(v)t) =Tr((Xq~i~i/~)aoXi) =Tr(Xq~i+ji/~) -1 or 0 (mod x0)


according to whether i --j or ~<] This means that the matrix X 
has entries in R0 satisfying
the conditions (~). Therefore, we have proven the following 
theorem.


THEOREM.
The group of normalized units N of ~G is isomorphic to the group 
of q X q matrices X in R0
that are invertible in R0 and satisfying the congruences


:3
	x= L	(mod ~0(--~~-l)).



	This concludes our section on the groups of order pq. The rest of 
the thesis deals with
concrete examples.
III.	Applications of Representation Method.


In this section, we shall use the first method described in this 
paper to describe the unit
groups of 83, D4 and D6, and, as well, as give a brief expository 
account of the method findings
for A4.



A. The Unit Group of ~S3.
	The first thing we do is consider a map deduced from a 
representation of 83 given by



0(12)  (1,4(011))

0(123) =(i,i(0:,31))


As one can see from the above this gives rise to a map
	0:~S3	~ e ~ e ~2

We define the map by linear extension using the convention that 
cycles multiply from the
right to the left (for example (1 2)(1 2 3) = (2 3)). Since we 
have defined 0 by linear extension
we see that this map is a homomorphism~


	Let B1 = {e, (1 2), (2 3), (1 3), (1 2 3), (1 3 2)} be a basis of 
~S3 and let OL = (OLi,
a2,...,a6) be an element of \rationals S3 with respect to the basis B1. 
Similarly let B2 be the canonical
basis for \rationals~\rationals~\rationals2 with X = (x1,x2,...,x6) being an element of that 
space with respect to B2.
	Then we may consider x	0a	aA where,






A





and upon further calculation we see that
111001
	1 -1	1	-1	0	-1
	1 -1	-1	0	-1	1
	1 -1	0	1	1	'~
	1  1	0	-i	1	-~
	1  1	-1	1	-1	0
22
23
	111	11	1
	1 -1	-1	-1	1  1
		2  2	-2	o -2	0
	A1=1/6	o  0	-2	2 -2	2
		o -2	0	22	-2
		2 -2	2	oo	2
	As A is invertible, it is readily seen that 0 is an isomorphism. 
Moreover from the
elements of A it is readily seen that

0~s3c'ZLe~~~.


		Furthermore, if we consider A-1, then for X~E~S3, we have
O-1XE~S3< =>


x1+x2+2x3
x1 - x2+2x3
x1 - x2 - 2x3 - 2x4
	xl	- x2	+2x4+2x5

x1+x2 - 2x3 - 2x2+2x5
x1+x2    +2x4 - 2x5 -
	+2x6=-0(mod 6)

-	2x5 - 2x6=-0(mod 6)
+2x6=-0(mod 6)

-0(mod 6)

	-0(mod 6)
2x6=-0(mod 6)


After simple row reduction we see that this reduces to the 
following set of three equations.
	x1+x2	+2x4+4x5+4x6=-0(mod 6)
	4x2	+4x5+2x5=-0(mod 6)
		4x~ + 2x4+ 4x5 + 2x6 0(mod 6)


The second and third reduce respectively to:

x2=-x6 - x5(mod 3) and
x4+x5=-x3+x5(mod 3)

Inspection of the	first equation shows us that
x1+x2=-0(mod 2)

If we also consider our first equation as an equation modulo 3 
and combine it with the result
of the second equation we get

x1=-x4+x6(mod 3)


	Therefore, the final result is as follows:


x1+x2=-0(mod 2)

x2=-x5 - x5(mod 3)

x1=x3+x5=-x4+x6(mod 3)
24



		Keeping all this in mind, we next consider the projection operator
~ Then we can see that



f	(x3 x4 I+=+(d6)
~x5 ~6)x3x5~x4x6mo~


Let us call this space Y.

		Let X=(x1,...,x6)EO~S3, with x6-=x8+x5-x4(mod 3). Let ~=x8x6-x4x5.
Consider
	(X3-x4)(x3+X5) = x3(x3+X5-x4)-x4x5  x3x6-x4x5(mod 3)	=-~(mod 3)


	In consideration of our row reduced equations it follows that X-1 
exits and is in O~S3
<=> x3x6-x4x5=~=+1 x1--tl,x2 = ~i.


	We see by composition of maps, that ~O is a homomorphism of ~S3 
into Y, and
therefore induces a homomorphisim of the unit group of ~S3 into 
the unit group of Y. We
will now show that this induced homomorphism is one to one and 
onto, proving that it is an
isomorphism.

	Let



Z= ( x3:6)E


Also let 6=x3x6-x4x5= t 1 and if X1,X2 E {-1,O,1} are defined by 
the congruences


x2 =- x6-x5 (mod 3) and
x1 =-x3+x5 (mod 3)

then neither x1 nor x2 is zero and all the above conditions are 
satisified. Therefore a=O-'X is a
unit in ~S3 with ~Oa=U. Therefore, by the above we have that ~O 
is one to one and onto
therefore we have


Theorem 1.

The unit grou~ of ~S3 is isomorphic to:
	(:d) E ~(~2) I a+c	b+d (mod 3)
q 25


B. The unit~ of 'ZliD4
	In this section, we will determine the group of units of ~D4, 
where D4 is the dihedral
group of order 8. This group is determined by the generators a,b 
together with the relations


a4=b2=baba= 1.

Before proceeding with the construction, we give a few 
definitions. The homomorphism

~	with ~(g)  1 for all gEG is called the augmentation function. 
Denote by V(~G) the

normal subgroup of the units u E 'ZLG such that ~(u) = 1. If u is 
in V(~G), it is called a

normalized unit. Finally an automorphism 0 of ~G is said to be 
normalized if ~0(g) = 1 for all

g in G.


	Now, as before, we determine our map from ~D4 to a direct sum of 
matrix rings over
~. In this case, the map will be





given by



		1, 1,-i, -1, (0;,~1))



		1,-i, 1,-i, (0;,1o))


		In the same manner as before, consider D4 as the basis of
G~D4 and use the canonical basis for the right hand side of the
above mapping. Then the map 0 can be represented by A where A
is the following matrix.


111001
	-1	-1	0	-1	1	0
	1	1	-1	0	0	-1
	-1	-1	0	1	-1	0
	1	-1	0		A=1	0
	-1	1	-1	0	0	1
	1	-1	0	-1	-1	0
	-1	1	1	0	0	-1
11

11

11

11

1 -1

1 -1

1 -1

1 -1



		Continuing our calculations we find that A-i is 1/8 times the
following matrix.
26
11111111
	2 0	-2	0	0 ~'	~
	o -2	0	2	2  0	-2	0
	o  2	0	-2	2  0	-2	0
	2  0	-2	0	0  2	0	-2


In the same manner as before, we see that if



/x5x6\
X = (x1,x2,x3,x4, 11)
x7x5/

with ~ then X belongs to O(~D4) <-->
	x1+x2+x3+x4+2x5	+2x8=0(mod8)









xl	- x2 x3+x4+2x5

-	2x8=0(mod 8)

	Applying row reduction in the same manner as the last example we 
get
	i)	x1+x2+x3+x4+2x5	+2x8=-0(mod 8)
	ii)	x2+x3	+2x~=-0(mod 4)
	iii)       x3 - x4 - x5 - x6		- x7+x8-=0(mod 4)
	iv)	x4+x5	+x7   =-0(mod2)
	v)	xl	+x8=-0(mod 2)
	vi)	x6+x7	-=0(mod 2)

	In addition, if X is to belong to O(U(~D4)), then x= + 1 for 
i=1,2,3,4 and we must
have x5x8-x6x7= t 1.


	Consider ~EGL(2,~), with



		xl, x1\
kx7, x8J
A27



which satisfy equations v) and vi). Then there exists ~ i= 
1,2,3,4 with
X--(X1,X2,X3,X4,%EO(~(~D4))<=> one of a, b, or C hold.


a)	x5=-1(mod 2); x5+x6+x7-x8=-O(mod 4); x5+x8=-2(mod 4)

b)	x8=-1(mod 2); x5+x6+x7-x8=2(mod 4); x5+x8=-O(mod 4)

c)	x8=-O(mod 2); x5+x5=-O(mod 4)


	Let us do the calculations that show this Consider; if X is in 
O~'ZLD4, then, as noted
before, we must have x1,x2,x3 and x4  t 1. Let ~ = t 1. Then 
there are only certain
combinations of the above that satisfy equations i) through vi). 
In particular, it should be
noted that either all of xi to x4 are either of the same sign or 
there are two of one sign and
two of the other. To see this, consider equations i) and v). If 
three of x1 to x4 were positive 1
and the fourth negative 1, then we would have that x5 + x8 1 (mod 
4). This is an obvious
contradiction to equation v)


	Therefore, let us consider the cases separately. If x1 = x2 = x3 
= x4 = ~, then equation i)
implies that
Equation iii) tells us that
Finally, equation ii) gives us that
2x5 + 2x8 4(mod 8) which implies that
x6+x8-=O(mod 4).



-x5-x6-x7+x8=-O(mod 4) which implies that
x5+x6+x7-x8=-O(mod 4).
x8=-1(mod 2).


It can be readily seen that the above is condition a).


Let us now consider x1=x2= -x3-x4=~. Here we see that equation i) 
implies
x5+x5=-O(mod 4).

Equation ii) implies

x8=-O(mod 2).

We see that we now have condition c). In addition, however, we 
can see that equation iii)
implies that
x5+x6+x7-x8=-O(mod 4).


Now suppose that x1 = -x2 = x3 = -x4 = ~. Then, we will have 
equation i) implying

x5+x8-=O(mod 4).

Equation ii) gives us

x8=-O(mod 2).
28


Again, we have condition c). This time, though, Equation iii) 
shows us
x5+x6+x6-x8=-2(mod 4).



	Finally the last combination to consider is x1  -x2 = -x3 = x4 = 
~. In this case we get
equation i) showing that
Equation ii) implies that


Equation iii) implies that


This finally is condition b).
x5~x8=-O(mod 4).
x8=-1(mod 2).


x5+x6+x7-x8=-2(mod 4).

	To show the other direction of the if and only if above is quite 
simple given the above
calculations. What one needs to do is to simply choose the 
particular x~ (i=1,2,3,4) as is given
above. These will then satisfy all the equations.


	Let us denote by ~ those matrices of GL(2,~) that satisfy 
equations v) and vi) above
and any one of a), b) or c). It is obvious by the linearity of 
the constraints that ~ is a
subgroup of GL(2,'ZL). For any element XE~ we can see by the 
above computations that there
are exactly two elements of O~(~D4) with X as the last member.


	Let 6--Il as above.

If a) holds, then ~=(6,6,6,6,X)EO(~('ZLD4)).


If b) holds, then ~ = (6,-6,-6,6,X)EO(~('ll�D4)).


If c) holds, there are two cases to consider. If we have the 
first, which is
x5+x6+x7-x5=-O(mod 4)

holding, then

= (6,6,-6,-6,X)EO(~(~D4)).

Otherwise, we have
x5+x6+x7-x5=-2(mod 4)

holding which gives us
= (6,-6,6,-6,X)EO(~('lliD4)).


	Now, we are in a position to describe the unit group of ~D4. If 
we choose aE~(~D4)
such that 0(a) = (x1,x2,x3,x4,X), then we can observe that ~(a) 
-~xi. From this and the preceding
information it is easy to see that since for any element X in ~ 
there are exactly two elements
in 0~(~D4) we have the theorem.


Theorem.
29


+ 11 >(~.




C. Units of ~D6
	In this section, we will show how a minor extension to the method 
can be used to
describe groups of higher orders. At this point, we will 
determine the unit group of 'ZLD6
where D6 is the dihedral group of order 12 given by the 
generators a,b together with the
relations


{ a2=b6=ab2ab2-e


Now if we applied the method of this section blindly we would be 
dealing with matrices of
order 12. This, to say the least, is inelegant. In addition to 
this the description of ~ ('ZLD6)
would include a direct product of two matrix groups. This does 
not give us a very satisfying
description, as determining properties of this type of product is 
not very easy.


	Instead, what we do in this section is to consider D6   C2 X 53, 
where C2 is the cyclic
group of order 2,  { + 1,-i} under multiplication. S3 is, as 
before, the symmetric group on 3
elements. Keeping this in mind, we have ~D6   ('ZLC2)S3, where 
~C2 is the group ring of the
group C2 over the ring ~, and the whole thing is the group ring 
of the group S3 over the ring
'ZLC2. The isomorphism of the above group rings is a well known 
theorem (See Sehgal~).


	At this point, let us denote \rationals C2 by ~. Then what we intend to do 
is to apply the
method used previously, replacing \rationals by ~ and 'iLby ~C2.This 
brings us to the point where
we may now define the map 0 as


where the mapping is defined by



0(1 2)=(1, -1,


and


0(1 2 3)-- (1, 1, (O;,~))



	At this point, one must keep in mind that the elements of the 
above vectors are in ~
not in (I~ as before.


	Now, let us consider the two modules that we have. The first ~S3 
is obviously a free
module over ~ with basis the elements of 53 The second is again a 
free module over ~ being
30


a direct sum of matrix rings over ~. We will use the standard 
basis for this module.


	Considering this, we may look at 0 as a module homomorphism 
between two free
modules. We may therefore represent it as a matrix A, in this 
case, a 6 by 6 matrix. Obviously
this matrix is going to have the same entries as the one we 
obtained when looking at ~S3,
with the distinction that these elements will be in ~ and not 
just in ~. For the sake of

convenience I will rewrite A and A1.

-





A--







and
111001
	1 -1	1	-1	0	-1
	1 -1	-1	0	-1	1
	1 -1	0	1	1	0
	1  1	0	-1	1	-1
	1  1	-1	1	-1	0


A1=1/6
111111
	1 -1	-1	-1	1	1
	2  2	-2	0	-2	0
	0 0	-2	2	-2	2
	0 -2	0	2	2	-2
	2 -2	2	0	0	2
As before, since A is invertible we see that 0 is an isomorphism. 
Also, as before, we are led to a
set of six congruences that describe when an element of 
(~c2)~(~C2)e(~c2)2 mapped by 0~'
is in ('ZLC2)S3. It is not neccessary to re-write the original 
equations. The result after row
reduction are the equations:

x1+ x2=-0(mod 2)

x2 x6-x5(mod 3)

x1=-x3+x5--x4+x6(mod 3)


	At the risk of being repititious, we note that the above 
congruences are in ~C2.
Namely that we are considering modulo the ideals generated by 2 
or 3 in 'ZLC2 in the above
equations.


	Let ~ denote the projection map of ~ onto ~2 Then we have that
4)	0((~C2)S3) =

31


(x3x4) : x3+x5=-x5+x6(mod 3)
x5 x

Let us call the above set ~.


	Let us now pause for a moment to consider some of the properties 
of 71'C2, where e is
the identity of C2 and ~ is the other element with ~2=e. The 
units of the group ring ~C2 are
easy to determine as they are only + C2. Therefore, for a matrix 
to be in the units of ('ZLC2)2
the determinant must be one of t e or +-~.


	Let us continue. If X= (x1,...,x6) is in 0(~C2)S3, then, if we 
let 6 represent any one of
+ e, t ~, then as was calculated before, we have that ~ exists 
and is in 0(~C2)S3 if and only
if


x3x6-x4x5=6, x1=6, x2= t6.


The mapping 4)0 is a ring homomorphism of ('lliC2)S3 into ~ and 
thus induces a homorphism
from ~((~C2)S3~~(Y). I conjecture that this is an isomorphism 
onto. To see this let



/x3x4\
II
6=x3x6~x4x5~x5xJ
Then	te 6r = t~ and ifwe choose x1,x2 which are in {e, 0, -e, ~, 
-~} to satisfy

x2=-x6 - x5(mod 3)
x1=-x3+x5(mod 5),

it follows from this that neither x1 nor x2 is 0 and that the 
above conditions are all satisfied.
Thus a=0~1X is a unit in (~C2)S3 with 4)0a=~. Furthermore, we can 
see that from the
preceding conditions that 4)0 is one-to-one.


	Therefore, remembering that ~D6=(~C2)S3 and that ~ we have
the following theorem.


THEOREM.

	i	(  ab\~~C).    -
~(~D4)=    cd)~2 a+c=b+d (mod




D.	The Unit Group of 'lLA4,(expository)

	In this section, we will present the charachterization of ~(~A4) 
as presented by Allen
and Hobby[1]. We will not include the proofs, as it is felt that 
nothing new is to be gained
from rehashing the method.


	As is known A4 has 4 irreducible representations, call them \rationals i = 
1,2,3,4. Also A4 is
generated by the two elements a=(1 2)(3 4) and b=(1 2 3). The 
representations 0~ i=1,2,3 are
�32



easily described as follows. 0~(a)=1 and 0~(b)=~~-1. The fourth, 
04~ is given as follows
	[~101~	-1 1 0 1
	04(a)=	401	and 04(b)=	-100
	L-iioJ	L -1 0 1 J


	Let ~ be defined by 0(r) = (01(r),02(r),03(r),04(r)). Then we may 
use
this map as in the preceding sections to determine the unit 
group. The characterization
arrived at by Allen and Hobby is as follows.


THEOREM
~('zLA4)={ t1}X{XESL(3,~): such that X satisfy the below 
conditions 1), 2),


1.

2.
3.
Every column sum of X is congruent to 1(mod 4).

No row contains all odd elements, and

one pseudo-trace is congruent to -1(mod 4) while the other two 
are congruent to 0(mod

4).

	There are three pseudo-traces on a three by three matrix, they 
are the sums of the
elements on the pseudo-diagonals that have been discussed 
previously.
IV. Groups of order 27


In this section, we Will use the method presented earlier in the 
paper to determine the unit
group of the integral group rings of the two non-commutative 
groups of order 27.


	I wish to remind you of the structure of these two groups.

G = <a,b(a,b)   c, ca = ac, cb = bc, a3 = b~ = c~ = 1>, and
H = <a,b I a9 = b3 = 1, b-1ab = a4>.

As is true in general, we have that G = <a>X<b>, and H = <a>X<b> 
are both
elementary abelian 3-groups. Also, it is well known that ~ =  G 
and ~'ZLH = + H. The
object of this section is to give a concrete description of both 
~'ZLG and ~



A.	First group of order 27.
	Let us consider G first. Referring back to our general proof we 
would let
A=DIAG(1,~,,w2) and B = PDIAG1. Our fibre product diagram would 
become
	I	I
~ (~3'ZL)G


5~ffi
In the above the map from $S_p$ to (~/3~)G is denoted by ~i and the 
map from 'ZLG/c'ZLG to

$S_p$ is denoted by ~o. These maps are defined in the same way as 
they were in general.

$S_p$ecializing the condition (*) from the general theorem we see 
that the matrix


		[xo,oxi,0x2,o1

M = Ix2~ixo,ixi,iI E

Lxl2x22x().2j
belongs to our $S_p$ if and only if for each i, O~i<2 the conditions


+ ~ + x~,2 E
		+	~ a' + x~,0a'2 E 3~[a'J
x~,0 + x~,1 a'2  +  x~,2a' E 3~[a']


hold. To find ~1(M), we need a~,~E'lL[a'] such that M = 
~a~,~AiBi. This will give us the matrix
equations







33
34
	Fao,jl	~1	1	1~	[ ~i0 1
	L:MJ]	�	L11	W~2    W~2]	L xA%2J

and this is equivalent to


(**)
a~,j  � (x~,o + xj,i + Xj,2)

a1,j =1/3 (x~,o + w2x~,1 + wx~,2)

a2j  � (x~,o + a'xj,1 + w2x~,2)

	From our previous work we know that ~1(M) = ~ Also, we know that 
the units
of ~G are pairs (a,M), with aE~~G and M in $S_p$ with ~1(M)   ~2(a). 
However, since we
know that ~ = t G, we need matrices M such that
	~1(M) =~a~~aibi	O2(+~ambn)


for some m,n. If we put Ir = a)-1 then we get

1  For two values of i and all j, a~,~=-O (mod ir).

2.	Considering the third value for i, either a~,0 = t 1, 
a~1-=a~2=-O (mod ir) or
a~,1 = +1 a,0=-a~,2=-O (mod ir) or
a~,2   +1 a,0=-a~,1=-O (mod Ir).


	This proves the following theorem.


THEOREM.
{ME~~[wJ3 I M satisfies 1. and 2. where ~ are given by (**)~


	From the above it is clear that the matrices in ~~[wJ3 which are 
congruent to
I (mod Ira) are contained in ~ and therefore, 'U~G is a 
congruence subgroup in


Second group of order 27
	Now, let us describe ~ If we have a matrix



x	=  =
[x00x10x201

1x21x01x111
Lx12x22x02J
A

satisfying (*) then the corresponding matrix in $S_p$ is
		[ xo 0 ~l,O ~2,o1
	=	L ~:x2112 CI,x:;,12  x421J
35


If we write A  ~a~~BiAi, then ~1(Z) = + h hEH if and only if the 
matrix X satisfies 1. and 2.
from above. Then we have the the following theorem.


THEOREM.
~ZE~~[c )J3IZ' satisfies 1. and 2. where ~ are given by (**)}.


Again it is easily seen that ~ is a congruence subgroup in 
SL(3,~[~J).
36


References

[1] P. J. Allen and C Hobby, A characterization of units in 
~[A4J, J Algebra 66 (1980),
534-543.
[2]	I. Hughes and K. R. Pearson, The group of units of the 
integral group ring ~S3, Canad.
Math. Bull. 15 (1972), 529-534.
[3] I. S. Luthar, Units in integral group rings, To appear.
[4]	C. Polcino-Milies, The units of the integral group ring ~D4, 
Bol. Soc. Brasil. Mat.
4(1972),85-92.
[5] J. Ritter and S. K. Sehgal, Integral group rings of some 
p-groups, To appear.
[6) 5. K. Sehgal, Topics in Group Rings, Dekker, New York, 1978. 
(Chapter 2)
THE UNIVERSITY OF ALBERTA

RELEASE FORM
NAME OF AUTHOR	Brett Gordon Giles
TITLE OF THESIS	The Unit Groups of Certain Group Rings

DEGREE FOR WHICH THESIS WAS PRESENTED   Master of Science
YEAR THIS DEGREE GRANTED	May 1981

		Permission is hereby granted to THE UNIVERSITY OF ALBERTA LIBRARY 
to
reproduce single copies of this thesis and to lend or sell such 
copies for private, scholarly
or scientific research purposes only.

		The author reserves other publication rights, and neither the 
thesis nor extensive
extracts from it may be printed or otherwise reproduced without 
the author's written
permission.

(SIGNED)


PERMANENT ADDRESS:










DATED...... 19
\end{document}