%!TEX root = /Users/gilesb/UofC/thesis/phd-thesis/phd-thesis.tex



\begin{tabbing}
Symbol~~~~~\= \ \ \ \ \ \ \ \ \ \ \ \ \ \ \ \ \ \ \ \ \ \ \ \ \ \ \ \ \ \ \ \ \ \ \ \  \parbox{5in}{Definition}\\

\addsymbol \mbox{$\rst{f}$}: {$\rst{f}$ is the restriction of $f$, see Definition~\ref{def:restriction_category}.}
\addsymbol \mbox{$\union$}: {Set union.}
\addsymbol \mbox{$\intersection$}: {Set intersection.}
\addsymbol \mbox{$\emptyset$}: {The empty set.}
\addsymbol \mbox{$\catdomain f$}: {The domain object of the map $f$, see Definition~\ref{def:category}.}
\addsymbol \mbox{$\catcodomain f$}: {The codomain object of the map $f$, see Definition~\ref{def:category}.}
\addsymbol \mbox{$\X,\Y, \A,\D,\R$}: {Categories, see Definition~\ref{def:category}.}
\addsymbol \mbox{$\in$}: {Element of, in sets.}
\addsymbol \mbox{$\to$}: {Separate domain and range of a map.}
\addsymbol \mbox{$\X(A,B)$}: {Hom-Set in $\X$. See Notation~\ref{notn:hom-set}.}
\addsymbol \mbox{$\{x | $condition on $x\}$}: {Define a set via a condition.}
\addsymbol \mbox{$\exists x.$}: {This means ``there exists an $x$ such that''.}
\addsymbol \mbox{\nat}: {The set of natural numbers, i.e., $\{0,1,2,\ldots\}$.}
\addsymbol \mbox{$[a_{ij}]$}: {The matrix whose $i,j$ element is $a_{ij}$.}
\addsymbol \mbox{$\sec{f}$}: {A right inverse of $f$.}
\addsymbol \mbox{$\retraction{f}$}: {A left inverse of $f$.}
\addsymbol \mbox{$\spl{E}{\B}$}: {The Karoubi envelope. See  Definition~\ref{def:split_category}.}
\addsymbol \mbox{$\langle f,g\rangle$}: {The product map of $f,g$. See Definition~\ref{def:categorical_product}.}
\addsymbol \mbox{$A\times B$}: {The product of $A,B$. See Definition~\ref{def:categorical_product}.}
\addsymbol \mbox{$[f,g]$}: {The coproduct map of $f,g$. See Definition~\ref{def:categorical_coproduct}.}
\addsymbol \mbox{$A+B$}: {The coproduct of $A,B$. See Definition~\ref{def:categorical_coproduct}.}
\addsymbol \mbox{$A\biproduct B$}: {The biproduct of $A,B$. See Definition~\ref{def:categorical_biproduct}.}
\addsymbol \mbox{$\alpha: F\natto G$}: {$\alpha$ is a natural transformation.}
\addsymbol \mbox{$F\cong G$}: {The natural transformation $F\natto G$ is an isomorphism.}
\addsymbol \mbox{$|S|$}: {The cardinality of the set $S$.}
\addsymbol \mbox{$F\leftadjoint G$}: {$F$ is the left adjoint of $G$.}
\addsymbol \mbox{$\oplus,\otimes,\odot$}: {Tensors of categories.}
\addsymbol \mbox{$\undef$}: {Signifies that a function is undefined at some value.}
\addsymbol \mbox{$\le$}: {An ordering relation.}
\addsymbol \mbox{$\open(A)$}: {Restriction idempotents of $A$.}
\addsymbol \mbox{$f\compatible g$}: {$f$ is compatible with $g$. See  Definition~\ref{def:compatible_maps}.}
\addsymbol \mbox{$f\join g$}: {The join of $f,g$. See Definition~\ref{def:joins}.}
\addsymbol \mbox{$\setminus$}: {Minus operation on sets.}
\addsymbol \mbox{$\meet$}: {Meets of maps, see Definition~\ref{def:meet_in_a_restriction_category}.}
\addsymbol \mbox{$\inv{f}$}: {Partial inverse of $f$, see Definition~\ref{def:partial_inverse_etc}.}
\addsymbol \mbox{$A\retract B$}: {$A$ is a retract of $B$.}
\addsymbol \mbox{$\rg{f}$}: {$\rg{f}$ is the range of $f$, see Definition~\ref{def:range_category}.}
\addsymbol \mbox{$\top$}: {Restriction terminal object. See  Definition~\ref{def:restriction_terminal_object}.}
\addsymbol \mbox{$!$}: {The unique map to the restriction terminal object.}
\addsymbol \mbox{$\Mstab$}: {A collection of monic maps.}
\addsymbol \mbox{$\forall X,$}: {For all $X$.}
\addsymbol \mbox{$\dgr{f}$}: {Apply the functor $\dagger$ to $f$.}
\addsymbol \mbox{$\implies$}: {Implies.}
\addsymbol \mbox{\complex}: {The field of complex numbers.}
\addsymbol \mbox{$f\xrightarrow{\cong}g$}: {$f$ is isomorphic to $g$.}
\addsymbol \mbox{\Xt}: {The completion of the discrete inverse category \X.}
\addsymbol \mbox{$f\xequiv{h}g$}: {$f$ is equivalent to $g$. See  Definition~\ref{def:xequivalence}.}
\addsymbol \mbox{$(f,C)$}: {Equivalence class of maps $f:A\to B\*C$.}
\addsymbol \mbox{$f:A\to\xtdmn{B}{C}$}: {Alternate way to write $(f,C):A \to B$.}
\addsymbol \mbox{$\definedas$}: {Defines the left hand side as the right hand side.}
\addsymbol \mbox{$\why$}: {The unique map from the initial object.}
\addsymbol \mbox{$\perp$}: {Disjointness relationship. See  Definition~\ref{def:disjointness_relation}.}
\addsymbol \mbox{$\ocdperp$}: {Open disjointness. See Definition~\ref{def:disjointness_in_open_x}.}
\addsymbol \mbox{$\djoin$}: {Disjoint join. See Definition~\ref{def:disjoint_join}.}
\addsymbol \mbox{$\disjointunion$}: {The disjoint union of sets.}
\addsymbol \mbox{$f\tjdown g$}: {Partial operation on $f,g$. See Definition~\ref{def:up_triangle_and_down_triangle}.}
\addsymbol \mbox{$f\tjup g$}: {Partial operation on $f,g$. See Definition~\ref{def:up_triangle_and_down_triangle}.}
%\addsymbol \mbox{$\tperp$}: {Disjointness relation defined by disjointness tensor.}
%\addsymbol \mbox{$\tjoin$}: {Disjoint join defined by disjoint sum tensor.}
\addsymbol \mbox{$\iff$}: {If and only if.}
% .
% .
% .
% ALWAYS KEEP THE FOLLOWING LINE
\end{tabbing}

%%% Local Variables:
%%% mode: latex
%%% TeX-master: "phd-thesis"
%%% End:
