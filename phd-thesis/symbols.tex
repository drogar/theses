%!TEX root = /Users/gilesb/UofC/thesis/phd-thesis/phd-thesis.tex



\begin{tabbing}
Symbol~~~~~\= \ \ \ \ \ \ \ \ \ \ \ \ \ \ \ \ \ \ \ \ \ \ \ \ \ \ \ \ \ \ \ \ \ \ \ \  \parbox{5in}{Definition}\\

\addsymbol \mbox{U of C}: {University of Calgary}
\addsymbol \mbox{\nat}: {The set of natural numbers, i.e., $\{0,1,2,\ldots\}$.}
\addsymbol \mbox{\integers}: {The ring of integers, i.e., $\{0,\pm1,\pm2,\ldots\}$.}
\addsymbol \mbox{\complex}: {The field of complex numbers.}
\addsymbol \mbox{$\uparrow$}: {Signifies that a function is undefined at some value.}
\addsymbol \mbox{$\conjugate{a}$}: {The complex conjugate of $a$ in \complex.}
\addsymbol \mbox{$\conjugate{A}$}: {The conjugate transpose of the matrix $A$.}
\addsymbol \mbox{$A\retract B$}: {$A$ is a retract of $B$.}
\addsymbol \mbox{$\rst{f}$}: {$rst{f}$ is the restriction of $f$, See Definition~\ref{def:restriction_category}.}
\addsymbol \mbox{$\rg{f}$}: {$rg{f}$ is the range of $f$, See Definition~\ref{def:range_category}.}
% .
% .
% .
% ALWAYS KEEP THE FOLLOWING LINE
\end{tabbing}

%%% Local Variables:
%%% mode: latex
%%% TeX-master: "phd-thesis"
%%% End:
