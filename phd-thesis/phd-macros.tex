%!TEX root = /Users/gilesb/UofC/thesis/phd-thesis/phd-thesis.tex
%----------------------------------------------------------------------
% a hack: \m is similar to \vcenter, but works better.
% \mp{0.2}{x}: raise the box x so that 20% of it is below the
% centerline. Unlike \vcenter, don't change the horizontal spacing
% Originator: Peter Selinger

\newlength{\localh}
\newlength{\locald}
\newbox\mybox
\def\mp#1#2{\setbox\mybox\hbox{#2}\localh\ht\mybox\locald\dp\mybox\addtolength{\localh}{-\locald}\raisebox{-#1\localh}{\box\mybox}}
\def\m#1{\scalebox{0.8}{\mp{0.5}{#1}}}

%----------------------------------------------------------------------
%  Boxes for flowcharts

\def\branchbox#1#2#3#4{ % draw diamond around current object and set "a0"
                        % "a1" and "au" positions
    \save
    []!U+/u#3/*{}="u"="#1u";
    []!R+/r#4/*{}="r";
    []!D+/d#3/*{}="d";
    []!L+/l#4/*{}="l";
    "d";"l"**{}?(#2)="#10";  % see note 3h for usage of ? and **
    "d";"r"**{}?(#2)="#11";
    \ar@{-}"u";"r"
    \ar@{-}"d";"r"
    \ar@{-}"u";"l"
    \ar@{-}"d";"l"
    \restore
}


\def\procbox#1#2{  % draw a box shaped like a procedure call around
                   % the object #1. #2=name of procedure
                   % usage:   \procbox{[ll].[].[ul]}{Proc1}
  \save#1="box";
  "box"!C*+<2ex,0ex>[F]\frm{}*{{\bf #2}};
  "box"!LU*[.]{}="lu";
  "box"!LD*[.]{}="ld";
  "box"!RU*[.]{}="ru";
  "box"!RD*[.]{}="rd";
  \ar@{-}"lu";"ld"
  \ar@{-}"ru";"rd"
  \restore
}

%----------------------------------------------------------------------
% Ad hoc quantum circuits using TikZ

\newenvironment{tikzqcircuit}{%
  \begingroup%
  \def\dot##1{\fill (##1) circle (.15);}%
  \def\triangle##1{\fill (##1)+(.3,0) --
    +(-.15,.25) -- +(-.15,-.25) -- cycle;}%
  \def\triangleadj##1{\fill (##1)+(-.3,0) --
    +(.15,.25) -- +(.15,-.25) -- cycle;}%
  \def\notgate##1{\filldraw[fill=white,thick] (##1) circle (.3);
                  \draw[thick] (##1)+(.3,0) -- +(-.3,0);
                  \draw[thick] (##1)+(0,.3) -- +(0,-.3);
                }%
  \def\controlled{\gencontrolled{\dot}}%
  \def\gencontrolled##1##2##3##4{\foreach\x in {##4} {
                    \draw[thick] (##3 |- 0,\x) -- (##3);
                  }
                  ##2{##3};
                  \foreach\x in {##4} {
                    ##1{##3 |- 0,\x};
                  }}%
  \def\tqgate##1##2{\biggate{##1}{##2}{##2}}%
  \def\biggate##1##2##3{\filldraw[fill=white,thick] (##2)+(-.45,-.4)
    rectangle ($(##3)+(.45,.4)$); \draw ($.5*(##2)+.5*(##3)$) node
    {\small ##1};}%
  \def\widebiggate##1##2##3##4{\filldraw[fill=white,thick] (##2)+(-1*##4,-.4)
    rectangle ($(##3)+(##4,.4)$); \draw ($.5*(##2)+.5*(##3)$) node
    {\small ##1};}%
  \def\widegate##1##2##3{\widebiggate{##1}{##3}{##3}{##2}}
  \def\gridx##1##2##3{\foreach\x in {##3} {
        \draw[thick] (##1,\x) -- (##2,\x);
      }}%
  \def\grid{\gridx{0}}%
  \def\cross##1{\draw (##1)+(-.3,-.3) -- +(.3,.3);
             \draw (##1)+(.3,-.3) -- +(-.3,.3);}%
  \def\bar##1{\fill (##1)+(-.05,-.4) rectangle ($(##1)+(.05,.4)$);}%
  \def\init##1##2{\bar{##2}; \leftlabel{##1}{##2};}
  \def\term##1##2{\bar{##2}; \rightlabel{##1}{##2};}
  \def\leftlabel##1##2{\draw (##2) node[left] {##1};}%
  \def\rightlabel##1##2{\draw (##2) node[right] {##1};}%
  \def\wirelabel##1##2{\draw (##2)+(0,-.15) node[above] {\footnotesize ##1};}%
  \def\multiwire##1{\draw (##1)+(-.1,-.3) -- +(.1,.3);}%
  \begin{tikzpicture}}
  {\end{tikzpicture}\endgroup}

%--------------------------------------------------------------------------------------
% General formatting macros
\newcommand{\prepprooflist}{\vspace{12pt} \hspace{-12pt}}

\newcommand{\itembf}[1]{\item{\textbf{#1}}}
\newcommand{\itemem}[1]{\item{\emph{#1}}}
\newcommand{\itemtt}[1]{\item{\texttt{#1}}}

\newcommand{\uts}{\hspace{0.1pt}}

\newcommand{\refitem}[2]{\ref{#1}[\ref{#2}]}

%------------------------------------------------------------------------------
% Term algebras

\newcommand{\unif}[2]{\ensuremath{\mathcal{U}({#1})\hspace{-0.3em}\downarrow\hspace{-0.3em}{#2}}\xspace}
\newcommand{\uniftus}{\unif{t,u}{\sigma}}
\newcommand{\ta}{\ensuremath{T_\Sigma}}
\newcommand{\tavar}{\ensuremath{\ta(X)}}

%------------------------------------------------------------------------------
% Combinatory algebras

\newcommand{\distinguished}[1]{\ensuremath{\text{\large #1}}\xspace}
\newcommand{\Kc}{\distinguished{K}}
\newcommand{\Sc}{\distinguished{S}}
\newcommand{\Ic}{\distinguished{I}}

\newcommand{\Bc}{\distinguished{B}}
\newcommand{\Cc}{\distinguished{C}}
\newcommand{\Wc}{\distinguished{W}}

\newcommand{\Dc}{\distinguished{D}}
\newcommand{\dc}{\ensuremath{\delta}\xspace}
\newcommand{\Fc}{\distinguished{F}}

%-----------------------------------------------------------------------------
% Axiom number and referencing
\newcommand{\axiom}[2]{\ensuremath{[\text{\bfseries {#1}.{#2}}]}\xspace}

\newcommand{\cataxiom}[1]{\axiom{C}{#1}}
\newcommand{\cataltaxiom}[1]{\axiom{C'}{#1}}

\newcommand{\rstaxiom}[1]{\axiom{R}{#1}}

\newcommand{\rgaxiom}[1]{\axiom{RR}{#1}}

\newcommand{\catone}{\cataxiom{1}}
\newcommand{\cattwo}{\cataxiom{2}}

\newcommand{\cataltone}{\cataltaxiom{1}}
\newcommand{\catalttwo}{\cataltaxiom{2}}
\newcommand{\cataltthree}{\cataltaxiom{3}}
\newcommand{\cataltfour}{\cataltaxiom{4}}
\newcommand{\cataltfive}{\cataltaxiom{5}}

\newcommand{\rone}{\rstaxiom{1}}
\newcommand{\rtwo}{\rstaxiom{2}}
\newcommand{\rthree}{\rstaxiom{3}}
\newcommand{\rfour}{\rstaxiom{4}}

\newcommand{\rrone}{\rgaxiom{1}}
\newcommand{\rrtwo}{\rgaxiom{2}}
\newcommand{\rrthree}{\rgaxiom{3}}
\newcommand{\rrfour}{\rgaxiom{4}}

%------------------------------------------------------------------------------
% General math notation macros
\newcommand{\nat}{\ensuremath{\mathbb{N}}\xspace}
\newcommand{\N}{\nat}
\newcommand{\complex}{\ensuremath{\mathbb{C}}\xspace}
\newcommand{\integers}{\ensuremath{\mathbb{Z}}\xspace}
\newcommand{\Z}{\integers}
                  % Complex
\newcommand{\D}{\ensuremath{\mathbb{D}}\xspace}                   % Dyadic
\newcommand{\Zs}{\ensuremath{\Z[\sqrt{2}]}\xspace}
\newcommand{\Zw}{\ensuremath{\Z[\omega]}\xspace}
\newcommand{\Ds}{\ensuremath{\D[\sqrt{2}]}\xspace}
\newcommand{\Dw}{\ensuremath{\D[\omega]}\xspace}
\newcommand{\Zb}{\ensuremath{\Z_2}\xspace}
\newcommand{\Zbw}{\ensuremath{\Zb[\omega]}\xspace}
\newcommand{\Zss}{\ensuremath{\Zb[\sqrt{2}]}\xspace}

\newcommand{\inv}[1]{\ensuremath{{#1}^{(-1)}}}

\newcommand{\mbf}{\ensuremath{\mathbf{f}}}
\newcommand{\mbg}{\ensuremath{\mathbf{g}}}
\newcommand{\mbh}{\ensuremath{\mathbf{h}}}

\newcommand{\nothing}{\phi}
\newcommand{\meetspl}{\ensuremath{\meet_{\text{K}}}\xspace}


\newcommand{\conjugate}[1]{{#1}^{*}}

\newcommand{\norm}[1]{\|#1\|}
\newcommand{\weight}[1]{\norm{#1}_{\rm weight}}
\newcommand{\snorm}[1]{\|#1\|}
\newcommand{\sweight}[1]{\snorm{#1}_{\rm weight}}

\DeclareMathOperator*{\definedas}{:\!=}

\DeclareMathOperator*{\dom}{dom}
\DeclareMathOperator*{\rng}{range}

\newcommand{\longsquiggly}[1]{\xymatrix{{}\ar@{~>}[r]^{#1}&{}}}
%-----------------------------------------------------------------------------
% Regular expressions, MA algo


\newcommand{\cC}{{\mathcal C}}
\newcommand{\cL}{{\mathcal L}}
\newcommand{\cK}{{\mathcal K}}
\newcommand{\emptyseq}{\varepsilon}
\newcommand{\sem}[1]{[\![#1]\!]}
\newcommand{\tuple}[1]{(#1)}
\newcommand{\seqeq}{\equiv}

\newcommand{\sS}{\mathscr{S}}
\newcommand{\sW}{{\mathscr{S}^\omega}}
\newcommand{\sC}{\mathscr{C}}
\newcommand{\sCp}{\mathscr{C}\,'}
\newcommand{\sHp}{\mathscr{H}\,'}
\newcommand{\sH}{\mathscr{H}}
\newcommand{\sm}{\setminus}
\newcommand{\seq}{\subseteq}
\newcommand{\pp}{{++}}

%------------------------------------------------------------------------------
% Special math notation



\let\*\otimes
\let\+\oplus
\let\<\langle
\let\>\rangle
\let\.\cdot

\newcommand{\biproduct}{\boxplus}
\newcommand{\blank}{\ensuremath{\text{\textvisiblespace}}}
\newcommand{\vc}[1]{\ensuremath{\mathbf{#1}}}

%------------------------------------------------------------------------------
%  Matrices


\newcommand{\xmatrix}[4]{{\renewcommand{\arraystretch}{#1}\arraycolsep=#2ex\left(\begin{array}{#3}#4\end{array}\right)}}       % resizable matrix
\newcommand{\zmatrix}{\xmatrix{.8}{.8}}         % matrix


\newcommand{\matindex}[1]{{\scriptsize#1}}% Matrix index


\newcommand{\bigI}{\rule[-1.5ex]{0mm}{4ex}~I~}



\newcommand{\qsmatgen}[5]{\ensuremath{#5 \begin{array} ( {c|c} ) #5 #1 & #5 #2 \\ \hline #5 #3& #5 #4\end{array}}}

\newcommand{\qsmatss}[4]{\qsmatgen{#1}{#2}{#3}{#4}{\scriptstyle}}
\newcommand{\qsmat}[4]{\qsmatgen{#1}{#2}{#3}{#4}{}}
% ----------------------------------------------------------------------
% Various macros to assist with TikZ drawing

\newcommand\umatrix[4]{\ensuremath{\begin{bmatrix}#1&#2\\#3&#4\end{bmatrix}}}
\newcommand*\leftrightfrom[3]{edge [<-] node[left] {$\scriptstyle #1$} %
                          node[right] {$\scriptstyle #2$} %
                          (#3)}
\newcommand*\leftrightto[3]{edge [->] node[left] {$\scriptstyle #1$} %
                          node[right] {$\scriptstyle #2$} %
                          (#3)}
\newcommand\starportfrom[3]{edge [<-] node[xshift=.5mm, yshift=-.5mm, above left] %
                    {$\scriptstyle #1$} %
                    node[xshift=-.5mm, yshift=.5mm, below right] %
                    {$\scriptstyle #2$}%
                    (#3)}
%------------------------------------------------------------------------------
% Types notation


\newcommand{\Qs}{\ensuremath{Q_{()}}\xspace}
\newcommand{\Bs}{\ensuremath{B_{()}}\xspace}

\newcommand{\type}[1]{\ensuremath{\mathbf{{#1}}}\xspace}
\newcommand{\bit}{\type{bit}}
\newcommand{\qubit}{\type{qubit}}
\newcommand{\qubits}{\qubit{s}\xspace}
\newcommand{\bits}{\bit{s}\xspace}

\newcommand{\qbit}{\type{qubit}}
\newcommand{\qbits}{\type{qubits}}

\newcommand{\interp}[1]{[\![#1]\!]}
%------------------------------------------------------------------------------
% General computing notation

\newcommand{\BigO}[1]{\ensuremath{\mathscr{O}({#1})}}

%------------------------------------------------------------------------------
% Quantum computing notation
\newcommand{\Had}{\text{Hadamard}}
\newcommand{\nottr}{\text{Not}}
\newcommand{\cnot}{Controlled{-}\nottr}

\newcommand{\lqpl}{L-QPL}

%\newcommand{\ket}[1]{\ensuremath{|#1\rangle}}
\newcommand{\kz}{\ket{0}}
\newcommand{\ko}{\ket{1}}
% gates as matrices

\newcommand{\mgate}[4]{\ensuremath{\begin{pmatrix}#1 & #2 \\ #3 & #4\end{pmatrix}}}

\newcommand{\notgate}{\mgate{0}{1}{1}{0}}
\newcommand{\hadgate}{\ensuremath{\frac{1}{\sqrt{2}}\mgate{1}{1}{1}{-1}}}
\newcommand{\vgate}{\mgate{1}{0}{0}{i}}
\newcommand{\ygate}{\mgate{0}{-i}{i}{0}}
\newcommand{\zgate}{\mgate{1}{0}{0}{-1}}
% Additional gates for drawing

\newcommand{\nowiregate}[1]{*{\xy *+<.6em>{#1};p\save+LU;+RU **\dir{-}\restore\save+RU;+RD **\dir{-}\restore\save+RD;+LD **\dir{-}\restore\POS+LD;+LU **\dir{-}\endxy}}

%------------------------------------------------------------------------------------
% General notation for categories
\newcommand{\catdomain}[1]{\partial_0({#1})}
\newcommand{\catcodomain}[1]{\partial_1({#1})}

\newcommand{\categorysection}[1]{{#1}^\diamond}
\newcommand{\retraction}[1]{{#1}_\diamond}

\newcommand{\categoryns}[4]{%
\begin{description}%
\item{\textbf{Objects: }}{#1}%
\item{\textbf{Maps: }}{#2}%
\item{\textbf{Identity: }}{#3}%
\item{\textbf{Composition: }}{#4}%
\end{description}%
}
\newcommand{\category}[4]{\categoryns{{#1};}{{#2};}{{#3};}{{#4}.}}


\newcommand{\natto}{\Rightarrow}
\newcommand{\specialcat}[1]{\textsc{#1}\xspace}
\newcommand{\ltrcatbb}[1]{\ensuremath{\mathbb{#1}}\xspace}

\newcommand{\retract}{\ensuremath{\triangleleft}}
\newcommand{\retractmaps}[2]{\ensuremath{\triangleleft_{#1}^{#2}}}


\newcommand{\A}{\ltrcatbb{A}}
\newcommand{\B}{\ltrcatbb{B}}
%\newcommand{\C}{\ltrcatbb{C}}
\newcommand{\cP}{\ltrcatbb{P}}
\newcommand{\R}{\ltrcatbb{R}}
\newcommand{\T}{\ltrcatbb{T}}
\newcommand{\X}{\ltrcatbb{X}}
\newcommand{\Y}{\ltrcatbb{Y}}

\newcommand{\cD}{\ltrcatbb{D}}
\newcommand{\cS}{\ltrcatbb{S}}

\newcommand{\Lat}{\ltrcatbb{L}}
\newcommand{\Mstab}{\ensuremath{\mathcal{M}}\xspace}
%\newcommand{\inj}{\specialcat{Inj}}
\newcommand{\pinj}{\specialcat{Pinj}}

\newcommand{\sets}{\specialcat{Sets}}
\newcommand{\finsets}{\ensuremath{\sets_f}}
\newcommand{\ord}{\specialcat{Ord}}
\newcommand{\cat}{\specialcat{Cat}}
\newcommand{\Par}{\specialcat{Par}}
\newcommand{\Ab}{\specialcat{Ab}}

\newcommand{\topcat}{\specialcat{Top}}
\newcommand{\topcatp}{\ensuremath{\topcat_p}\xspace}

\newcommand{\DSum}{\specialcat{DSum}}
\newcommand{\DJoin}{\specialcat{DJoin}}
\newcommand{\stabLat}{\specialcat{stabLat}}

\newcommand{\CFrob}{\ensuremath{\specialcat{CFrob}(\X)}\xspace}
\newcommand{\rel}{\specialcat{Rel}}

\newcommand{\fdh}{\specialcat{FdHilb}}

\newcommand{\spl}[2]{\ensuremath{\text{K}_{#1}(#2)}}

\newcommand{\leftadjoint}{\vdash}
\newcommand{\rightadjoint}{\dashv}

\newcommand{\zeroob}{\ensuremath{\mathbf{0}}}


\newcommand{\cpm}[1]{\specialcat{CPM(#1)}}
\newcommand{\bcpm}[1]{\ensuremath{\cpm{#1}^{\+}}}

%------------------------------------------------------------------------------
% Restriction categories

\newcommand{\compatible}{\smile}

\newcommand{\rcategoryequivns}[5]{%
\begin{description}%
\item{\textbf{Objects: }}{#1}%
\item{\textbf{Equivalence Classes of Maps: }}{#2}%
\item{\textbf{Identity: }}{#3}%
\item{\textbf{Composition: }}{#4}%
\item{\textbf{Restriction: }}{#5}%
\end{description}%
%
}

\newcommand{\rcategoryequiv}[5]{\rcategoryequivns{{#1};}{{#2};}{{#3};}{{#4};}{{#5}.}}

\newcommand{\rcategoryns}[5]{%
\begin{description}%
\item{\textbf{Objects: }}{#1}%
\item{\textbf{Maps: }}{#2}%
\item{\textbf{Identity: }}{#3}%
\item{\textbf{Composition: }}{#4}%
\item{\textbf{Restriction: }}{#5}%
\end{description}%
%
}


\newcommand{\rcategory}[5]{\rcategoryns{{#1};}{{#2};}{{#3};}{{#4};}{{#5}.}}

\newcommand{\restr}[1]{\overline{#1}}
\newcommand{\rst}[1]{\restr{#1}}

\newcommand{\rg}[1]{\hat{#1}}
\newcommand{\wrg}[1]{\widehat{#1}}

\newcommand{\why}{\ensuremath{\mathbf{?}}}
%--------------------------------------------------------------------------------------
% Notation for inverse categories


\newcommand{\uleft}[1]{\ensuremath{u_{#1}^l}}
\newcommand{\uright}[1]{\ensuremath{u_{#1}^r}}
\newcommand{\usl}{\uleft{\*}}
\newcommand{\usr}{\uright{\*}}

\newcommand{\exchange}[1]{\ensuremath{e\!x_{#1}}}
\newcommand{\excs}{\exchange{\*}}

\newcommand{\idelta}{\ensuremath{\inv{\Delta}}}
\newcommand{\Invc}[1]{\ensuremath{Inv(#1)}\xspace}
\newcommand{\Invf}{\ensuremath{\mathbf{INV}}\xspace}
\newcommand{\Inv}[1]{\ensuremath{\mathbf{INV}({#1})}\xspace}

\newcommand{\Xt}{\ensuremath{\widetilde{\X}}\xspace}


\newcommand{\hypXt}{\texorpdfstring{\Xt}{Xt}}
\newcommand{\hypX}{\texorpdfstring{\X}{X}}

\newcommand{\xtdmn}[2]{\ensuremath{{#1}_{|{#2}}}}
\newcommand{\xequiv}[1]{\ensuremath{\overset{\scriptscriptstyle #1}{\simeq}}}


\newcommand{\wtc}{\ensuremath{\widetilde{(\_)}}\xspace}
\newcommand{\wtf}{\ensuremath{\mathbf{T}}\xspace}
\newcommand{\wt}[1]{\ensuremath{\mathbf{T}({#1})}\xspace}

\newcommand{\dmap}[1]{\ensuremath{{#1}^{\Delta}_{\nabla}}\xspace}
%---------------------------------------------------------------------
% Notation for inverse sum categories

\newcommand{\imat}[1]{\ensuremath{\specialcat{iMat}({#1})}\xspace}
\newcommand{\imatx}{\imat{\X}}


\newcommand{\cp}[1]{\amalg_{#1}}
\newcommand{\cpa}{\cp{1}}
\newcommand{\cpb}{\cp{2}}

\newcommand{\icp}[1]{\inv{\cp{#1}}}
\newcommand{\icpa}{\icp{1}}
\newcommand{\icpb}{\icp{2}}

\newcommand{\scp}[1]{\cp{#1}^{*}}
\newcommand{\scpa}{\scp{1}}
\newcommand{\scpb}{\scp{2}}


\newcommand{\perpvv}[2]{{}_{{}_{#1}}{\perp_{{}_{#2}}}}
\newcommand{\perpaa}{\perpvv{A}{A}}
\newcommand{\perpab}{\perpvv{A}{B}}
\newcommand{\perpba}{\perpvv{B}{A}}
\newcommand{\perpcb}{\perpvv{C}{B}}
\newcommand{\tperp}{\perpvv{}{\oplus}}

\newcommand{\compat}{\smile}
\newcommand{\join}{\vee}
\newcommand{\meet}{\cap}
\newcommand{\intersection}{\cap}
\newcommand{\union}{\cup}
\newcommand{\disjointunion}{\uplus}

\newcommand{\undef}{\uparrow}

\newcommand{\djoin}{\sqcup}
\newcommand{\djoinbig}{\bigsqcup}
\DeclareMathOperator{\diag}{\scriptstyle{diag}}
\newcommand{\altjoin}{\,\square\,}
%\newcommand{\djoin}{{\,{{\sqcup}\hspace{-0.474em}{\mid}}\,\,}}
\newcommand{\disjoint}{\perp}
\newcommand{\invdisjoint}{\disjoint^0}
\newcommand{\invjoin}{\join^0}
\newcommand{\tjoin}{\djoin_{{_{\oplus}}}}

\DeclareMathOperator{\tjdown}{\triangledown}
\DeclareMathOperator{\tjup}{\vartriangle}
\DeclareMathOperator{\gtjdown}{\overline{\triangledown}}
\DeclareMathOperator{\gtjup}{\underline{\vartriangle}}

\newcommand{\ocdperp}{\,\underline{\perp}\,}
\newcommand{\eocdperp}{\,\underline{\underline{\perp}}\,}
\newcommand{\ocdperpsub}[1]{\,\underline{\perp}_{{}_{#1}}\,}
\newcommand{\cdperp}{\perp}
\newcommand{\invsum}[2]{{{#1}+{#2}}} %\bowtie_{{}_{{#1},{#2}}}}
\newcommand{\invinvsum}[2]{{{#1}\hat{+}{#2}}}


\newcommand{\open}[1]{\ensuremath{\mathcal{O}({#1})}}

\newcommand{\upl}{\uleft{\+}}
\newcommand{\upr}{\uright{\+}}

\newcommand{\com}[1]{\ensuremath{c_{#1}}}
\newcommand{\comp}{\com{\+}}
\newcommand{\assoc}[1]{\ensuremath{a_{#1}}}
\newcommand{\assocp}{\assoc{\+}}


\newcommand{\ea}{e_1}
\newcommand{\eb}{e_2}
\newcommand{\xa}{x_1}
\newcommand{\xb}{x_2}

\DeclareMathOperator{\rstp}{\rst{+}}
\DeclareMathOperator{\rgp}{\rg{+}}

\DeclareMathOperator{\mat}{Mat}
\DeclareMathOperator{\ob}{Ob}

%-------------------------------------------------
% sub/super scripting and decorating

\newcommand{\dual}[1]{\ensuremath{{#1}^{*}}}

\newcommand{\dgr}[1]{\ensuremath{{#1}^{\dagger}}}

\newcommand{\jay}{\ensuremath{j}\xspace}
\newcommand{\kay}{\ensuremath{k}\xspace}

\newcommand{\parity}[1]{\overline{#1}\/}
\newcommand{\residue}[1]{\overline{#1}\/}

\newcommand{\level}[1]{_{[#1]}}


\newcommand{\s}[1]{\{#1\}}
\newcommand{\da}{^{\dagger}}

\newcommand{\inverse}{^{-1}}
%----------------------------------------------------------------------
% Alphabetic labels
{\makeatletter
\gdef\alphalabels{\def\theenumi{\@alph\c@enumi}\def\labelenumi{(\theenumi)}}}

% ----------------------------------------------------------------------
% Theorem like environments

\theoremstyle{plain}
\newtheorem{theorem}{Theorem}[section]
\newtheorem{lemma}[theorem]{Lemma}
\newtheorem{proposition}[theorem]{Proposition}
\newtheorem{conjecture}[theorem]{Conjecture}
\newtheorem{corollary}[theorem]{Corollary}

\theoremstyle{definition}
\newtheorem{definition}[theorem]{Definition}
\newtheorem{remark}[theorem]{Remark}
\newtheorem{convention}[theorem]{Convention}
\newtheorem{example}[theorem]{Example}
\newtheorem{notation}[theorem]{Notation}
\newtheorem{notethm}[theorem]{Note}

%\theoremstyle{definition}
\newtheorem*{sltheorem}{Theorem}
\newtheorem*{sldefinition}{Definition}
\newtheorem*{slproposition}{Proposition}
\newtheorem*{slnotation}{Notation}
\newtheorem*{slexample}{Example}
\newtheorem*{slexercise}{Exercise}
\newtheorem*{sllemma}{Lemma}


%--------------------------------------------------------------------------------
% Macros for Turing categories

\newcommand{\name}[1]{\ensuremath{{}^\ulcorner\!\!#1^\urcorner}}
\newcommand{\iname}[1]{\name{#1}}   %{\ensuremath{{}_\llcorner#1_{\!\lrcorner}}}
\newcommand{\code}[1]{\ensuremath{#1_{\bullet}}}

\newcommand{\tur}[2]{\ensuremath{\tau_{{#1},{#2}}}}
\newcommand{\txy}{\tur{X}{Y}}
\newcommand{\multiapp}[2]{\ensuremath{{#1}^{(#2)}}}
\newcommand{\multibullet}[1]{\multiapp{\bullet}{#1}}
\newcommand{\imultiapp}[2]{\ensuremath{{#1}^{[#2]}}}
\newcommand{\imultibullet}[1]{\imultiapp{\bullet}{#1}}


\newcommand{\compa}{\ensuremath{\specialcat{Comp}(A)}\xspace}
\newcommand{\compn}{\ensuremath{\specialcat{Comp}(\N)}\xspace}
%-------------------------------------------------------------------------------
% Code environments for Verbatim package


%----------------------------------------------------------------------------------
% Code environments and definitions for listings

%-----------------------------------------------------------------------------------
% environments


%-----------------------------------------------------------------------------------
% Extra macros for describing the LQPL language



%-----------------------------------------------------------------------------------
% Macros for including code

%------------------------------------------------------------------------------------
% Formatting macros

%------------------------------------------------------------------------------------
%  Notation for describing syntax of language (lqpl primarily)

%%% Local Variables:
%%% mode: latex
%%% TeX-master: "phd-thesis"
%%% End:
