%!TEX root = /Users/gilesb/UofC/thesis/phd-thesis/phd-thesis.tex
%--------------------------------------------------------------------------------------
% General formatting macros

\newcommand{\bproofenum}{\vspace{12pt} \hspace{-12pt} \renewcommand{\theenumi}{(\roman{enumi})}\renewcommand{\labelenumi}{\theenumi}\begin{enumerate}}
\newcommand{\eproofenum}{\end{enumerate}\renewcommand{\theenumi}{\arabic{enumi}}\renewcommand{\labelenumi}{\theenumi.}}

\newcommand{\itembf}[1]{\item{\textbf{#1}}}
\newcommand{\itemem}[1]{\item{\emph{#1}}}
\newcommand{\itemtt}[1]{\item{\texttt{#1}}}

\newcommand{\uts}{\hspace{0.1pt}}

%------------------------------------------------------------------------------
% Term algebras

\newcommand{\unif}[2]{\ensuremath{\mathcal{U}({#1})\hspace{-0.3em}\downarrow\hspace{-0.3em}{#2}}\xspace}
\newcommand{\uniftus}{\unif{t,u}{\sigma}}
\newcommand{\ta}{\ensuremath{T_\Sigma}}
\newcommand{\tavar}{\ensuremath{\ta(X)}}



%------------------------------------------------------------------------------
% General math notation macros
\newcommand{\nat}{\ensuremath{\mathbb{N}}\xspace}
\newcommand{\complex}{\ensuremath{\mathbb{C}}\xspace}
\newcommand{\integers}{\ensuremath{\mathbb{Z}}\xspace}
\newcommand{\inv}[1]{\ensuremath{{#1}^{(-1)}}}

\newcommand{\mbf}{\ensuremath{\mathbf{f}}}
\newcommand{\mbg}{\ensuremath{\mathbf{g}}}
\newcommand{\mbh}{\ensuremath{\mathbf{h}}}

\newcommand{\nothing}{\phi}
\newcommand{\capspl}{\ensuremath{\cap_{\text{K}}}\xspace}

\DeclareMathOperator*{\dom}{dom}
\DeclareMathOperator*{\rng}{range}

%------------------------------------------------------------------------------
% Special math notation


\let\*\otimes
\let\+\oplus
\let\<\langle
\let\>\rangle

\newcommand{\blank}{\ensuremath{\text{\textvisiblespace}}}
\newcommand{\vc}[1]{\ensuremath{\mathbf{#1}}}

%------------------------------------------------------------------------------
% Types notation


\newcommand{\Qs}{\ensuremath{Q_{()}}\xspace}
\newcommand{\Bs}{\ensuremath{B_{()}}\xspace}

\newcommand{\type}[1]{\ensuremath{\mathbf{{#1}}}\xspace}
\newcommand{\bit}{\type{bit}}
\newcommand{\qubit}{\type{qubit}}
\newcommand{\qubits}{\qubit{s}\xspace}
\newcommand{\bits}{\bit{s}\xspace}

%------------------------------------------------------------------------------
% General computing notation

\newcommand{\BigO}[1]{\ensuremath{\mathscr{O}({#1})}}

%------------------------------------------------------------------------------
% Quantum computing notation
\newcommand{\Had}{\text{Hadamard}}
\newcommand{\nottr}{\text{Not}}
\newcommand{\cnot}{Controlled{-}\nottr}

\newcommand{\lqpl}{L-QPL}

% Additional gates

\newcommand{\nowiregate}[1]{*{\xy *+<.6em>{#1};p\save+LU;+RU **\dir{-}\restore\save+RU;+RD **\dir{-}\restore\save+RD;+LD **\dir{-}\restore\POS+LD;+LU **\dir{-}\endxy}}

%------------------------------------------------------------------------------------
% General notation for categories
\newcommand{\category}[4]{%
\begin{description}%
\item{\textbf{Objects: }}{#1}%
\item{\textbf{Maps: }}{#2}%
\item{\textbf{Identity: }}{#3}%
\item{\textbf{Composition: }}{#4}%
\end{description}%
}

\newcommand{\specialcat}[1]{\textsc{#1}\xspace}
\newcommand{\ltrcatbb}[1]{\ensuremath{\mathbb{#1}}\xspace}

\newcommand{\C}{\ltrcatbb{C}}
\newcommand{\X}{\ltrcatbb{X}}
\newcommand{\Y}{\ltrcatbb{Y}}
\newcommand{\Z}{\ltrcatbb{Z}}

\newcommand{\sets}{\specialcat{Sets}}
\newcommand{\Par}{\specialcat{Par}}

\newcommand{\spl}[2]{\ensuremath{\text{K}_{#1}(#2)}}

%------------------------------------------------------------------------------
% Restriction categories

\newcommand{\rcategoryequiv}[5]{%
\begin{description}%
\item{\textbf{Objects: }}{#1}%
\item{\textbf{Equivalence Classes of Maps: }}{#2}%
\item{\textbf{Identity: }}{#3}%
\item{\textbf{Composition: }}{#4}%
\item{\textbf{Restriction: }}{#5}%
\end{description}%
%
}


\newcommand{\rone}{[\emph{\bfseries R.1}]\xspace}
\newcommand{\rtwo}{[\emph{\bfseries R.2}]\xspace}
\newcommand{\rthree}{[\emph{\bfseries R.3}]\xspace}
\newcommand{\rfour}{[\emph{\bfseries R.4}]\xspace}

\newcommand{\restr}[1]{\overline{#1}}
\newcommand{\rst}[1]{\restr{#1}}

\newcommand{\Mstab}{\ensuremath{\mathcal{M}}\xspace}

%--------------------------------------------------------------------------------------
% Notation for inverse categories


%-------------------------------------------------
% subscripting

\newcommand{\jay}{\ensuremath{j}\xspace}
\newcommand{\kay}{\ensuremath{k}\xspace}


% ----------------------------------------------------------------------
% Theorem like environments

\theoremstyle{plain}
\newtheorem{theorem}{Theorem}[section]
\newtheorem{lemma}[theorem]{Lemma}
\newtheorem{proposition}[theorem]{Proposition}
\newtheorem{conjecture}[theorem]{Conjecture}
\newtheorem{corollary}[theorem]{Corollary}

\theoremstyle{definition}
\newtheorem{definition}[theorem]{Definition}
\newtheorem{remark}[theorem]{Remark}
\newtheorem{convention}[theorem]{Convention}
\newtheorem{example}[theorem]{Example}
\newtheorem{notation}[theorem]{Notation}
\newtheorem{notethm}[theorem]{Note}

%\theoremstyle{definition}
\newtheorem*{sltheorem}{Theorem}
\newtheorem*{sldefinition}{Definition}
\newtheorem*{slproposition}{Proposition}
\newtheorem*{slnotation}{Notation}
\newtheorem*{slexample}{Example}
\newtheorem*{slexercise}{Exercise}
\newtheorem*{sllemma}{Lemma}

% reference formats -
\labelformat{chapter}{chapter~#1}
\labelformat{section}{section~#1}
\labelformat{subsection}{sub-section~#1}
\labelformat{subsubsection}{sub-sub-section~#1}
\labelformat{equation}{equation~(#1)}
\labelformat{table}{table~#1}
\labelformat{figure}{figure~#1}

%-------------------------------------------------------------------------------
% Code environments for Verbatim package


%----------------------------------------------------------------------------------
% Code environments and definitions for listings

%-----------------------------------------------------------------------------------
% Extra macros for describing the LQPL language



%-----------------------------------------------------------------------------------
% Macros for including code 

%------------------------------------------------------------------------------------
% Formatting macros

%------------------------------------------------------------------------------------
%  Notation for describing syntax of language (lqpl primarily)


