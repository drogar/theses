%!TEX root = /Users/gilesb/UofC/thesis/phd-thesis/phd-thesis.tex
%--------------------------------------------------------------------------------------
% General formatting macros


\newcommand{\itembf}[1]{\item{\textbf{#1}}}
\newcommand{\itemem}[1]{\item{\emph{#1}}}
\newcommand{\itemtt}[1]{\item{\texttt{#1}}}

%------------------------------------------------------------------------------
% Term algebras

\newcommand{\unif}[2]{\ensuremath{\mathcal{U}({#1})\hspace{-0.3em}\downarrow\hspace{-0.3em}{#2}}\xspace}
\newcommand{\uniftus}{\unif{t,u}{\sigma}}
\newcommand{\ta}{\ensuremath{T_\Sigma}}
\newcommand{\tavar}{\ensuremath{\ta(X)}}


%------------------------------------------------------------------------------
% Restriction categories

%------------------------------------------------------------------------------
% General math notation macros
\newcommand{\nat}{\ensuremath{\mathbb{N}}\xspace}
\newcommand{\complex}{\ensuremath{\mathbb{C}}\xspace}
\newcommand{\integers}{\ensuremath{\mathbb{Z}}\xspace}



%------------------------------------------------------------------------------
% Types and special math notation



%------------------------------------------------------------------------------------
% General notation for categories



% Additional gates


%-------------------------------------------------
% subscripting

\newcommand{\jay}{\ensuremath{j}\xspace}
\newcommand{\kay}{\ensuremath{k}\xspace}


% ----------------------------------------------------------------------
% Theorem like environments

\theoremstyle{plain}
\newtheorem{theorem}{Theorem}[section]
\newtheorem{lemma}[theorem]{Lemma}
\newtheorem{proposition}[theorem]{Proposition}
\newtheorem{conjecture}[theorem]{Conjecture}
\newtheorem{corollary}[theorem]{Corollary}

\theoremstyle{definition}
\newtheorem{definition}[theorem]{Definition}
\newtheorem{remark}[theorem]{Remark}
\newtheorem{convention}[theorem]{Convention}
\newtheorem{example}[theorem]{Example}
\newtheorem{notation}[theorem]{Notation}
\newtheorem{notethm}[theorem]{Note}

%\theoremstyle{definition}
\newtheorem*{sltheorem}{Theorem}
\newtheorem*{sldefinition}{Definition}
\newtheorem*{slproposition}{Proposition}
\newtheorem*{slnotation}{Notation}
\newtheorem*{slexample}{Example}
\newtheorem*{slexercise}{Exercise}
\newtheorem*{sllemma}{Lemma}

% reference formats -
\labelformat{chapter}{chapter~#1}
\labelformat{section}{section~#1}
\labelformat{subsection}{sub-section~#1}
\labelformat{subsubsection}{sub-sub-section~#1}
\labelformat{equation}{equation~(#1)}
\labelformat{table}{table~#1}
\labelformat{figure}{figure~#1}

%-------------------------------------------------------------------------------
% Code environments for Verbatim package


%----------------------------------------------------------------------------------
% Code environments and definitions for listings

%-----------------------------------------------------------------------------------
% Extra macros for describing the LQPL language



%-----------------------------------------------------------------------------------
% Macros for including code 

%------------------------------------------------------------------------------------
% Formatting macros

%------------------------------------------------------------------------------------
%  Notation for describing syntax of language (lqpl primarily)


