%!TEX root = /Users/gilesb/UofC/thesis/phd-thesis/phd-thesis.tex
%--------------------------------------------------------------------------------------
% General formatting macros

\newcommand{\quest}[1]{\fbox{\fbox{\parbox{5in}{\footnotesize \textbf{ (Question): }#1}}}}
\newcommand{\note}[1]{\par\fbox{\fbox{\parbox{5in}{\footnotesize \textbf{(Internal Note): }#1}}}}

\newcommand{\bproofenum}{\vspace{12pt} \hspace{-12pt} \renewcommand{\theenumi}{(\roman{enumi})}\renewcommand{\labelenumi}{\theenumi}\begin{enumerate}}
\newcommand{\eproofenum}{\end{enumerate}\renewcommand{\theenumi}{\arabic{enumi}}\renewcommand{\labelenumi}{\theenumi.}}
\newcommand{\bproofitemize}{ \vspace{12pt} \hspace{-12pt} \begin{itemize}}
\newcommand{\eproofitemize}{\end{itemize}}
\newcommand{\bproofdesc}{ \vspace{12pt} \hspace{-12pt} \begin{description}}
\newcommand{\eproofdesc}{\end{description}}



%------------------------------------------------------------------------------
% Restriction categories
\newcommand{\blank}{\ensuremath{\text{\textvisiblespace}}}
\newcommand{\rone}{[\emph{\bfseries R.1}]\xspace}
\newcommand{\rtwo}{[\emph{\bfseries R.2}]\xspace}
\newcommand{\rthree}{[\emph{\bfseries R.3}]\xspace}
\newcommand{\rfour}{[\emph{\bfseries R.4}]\xspace}
\newcommand{\restr}[1]{\overline{#1}}
\newcommand{\rst}[1]{\restr{#1}}
\newcommand{\rcategory}[5]{%
\begin{description}%
\item{\textbf{Objects: }}{#1}%
\item{\textbf{Maps: }}{#2}%
\item{\textbf{Identity: }}{#3}%
\item{\textbf{Composition: }}{#4}%
\item{\textbf{Restriction: }}{#5}%
\end{description}%
%
}

\newcommand{\rcategoryequiv}[5]{%
\begin{description}%
\item{\textbf{Objects: }}{#1}%
\item{\textbf{Equivalence Classes of Maps: }}{#2}%
\item{\textbf{Identity: }}{#3}%
\item{\textbf{Composition: }}{#4}%
\item{\textbf{Restriction: }}{#5}%
\end{description}%
%
}


%------------------------------------------------------------------------------
% General math notation macros
\newcommand{\nothing}{\phi}
\newcommand{\nin}{\ensuremath{\in\hspace{-0.8em}/\hspace{0.2em}}}

\newcommand{\BigO}[1]{\ensuremath{\mathscr{O}({#1})}}
\DeclareMathOperator*{\dom}{dom}
\DeclareMathOperator*{\rng}{range}

\newcommand{\union}{\ensuremath{\bigcup}}
\newcommand{\disjunion}{\ensuremath{\sqcup}}
\newcommand{\intersect}{\ensuremath{\bigcap}}
\newcommand{\logor}{\ensuremath{\lor}}
\newcommand{\logand}{\ensuremath{\land}}
\newcommand{\lognand}{\ensuremath{\barwedge}}
\newcommand{\natmap}{\ensuremath{\Rightarrow}}
\newcommand{\fctrmap}{\ensuremath{\to}}
\newcommand{\fnctrmap}{\fctrmap}
\newcommand{\fmap}{\ensuremath{\to}}
\newcommand{\produces}{\ensuremath{\to}}
\newcommand{\ladjoint}{\ensuremath{\dashv}}
\newcommand{\pproj}{\ensuremath{\preceq_p}}

\newcommand{\tr}[1]{\ensuremath{\text{tr}(#1)}}

\newcommand{\colvec}[1]{\ensuremath{
\begin{pmatrix}#1\end{pmatrix}}}

\newcommand{\tbtmatrix}[4]{\ensuremath{
\begin{pmatrix}
#1&#2\\
#3&#4
\end{pmatrix}
}}

\newcommand{\quadmatrix}[4]{\ensuremath{
\left(
\begin{array}{c|c}
#1&#2\\
\hline
#3&#4
\end{array}
\right)
}}


%------------------------------------------------------------------------------
% Types and special math notation

\newcommand{\typecontext}{\ensuremath{\mathbb{T}}\xspace}

\newcommand{\type}[1]{\ensuremath{\mathbf{{#1}}}\xspace}
\newcommand{\bit}{\type{bit}}
\newcommand{\qubit}{\type{qubit}}
\newcommand{\qubits}{\qubit{s}\xspace}
\newcommand{\bits}{\bit{s}\xspace}

\newcommand{\Qs}{\ensuremath{Q_{()}}\xspace}
\newcommand{\Bs}{\ensuremath{B_{()}}\xspace}

\newcommand{\mbf}{\ensuremath{\mathbf{f}}}
\newcommand{\mbg}{\ensuremath{\mathbf{g}}}
\newcommand{\mbh}{\ensuremath{\mathbf{h}}}


\newcommand{\cp}[1]{\amalg_{#1}}
\newcommand{\cpa}{\cp{0}}
\newcommand{\cpb}{\cp{1}}
\let\*\otimes
\let\+\oplus
\let\<\langle
\let\>\rangle
\newcommand{\uts}{\hspace{0.1pt}}


\newcommand{\spl}[2]{\ensuremath{\text{K}_{#1}(#2)}}
\newcommand{\capspl}{\ensuremath{\cap_{\text{K}}}\xspace}
\newcommand{\inv}[1]{\ensuremath{{#1}^{(-1)}}}
\newcommand{\dual}[1]{\ensuremath{{#1}^{*}}}
\newcommand{\dgr}[1]{\ensuremath{{#1}^{\dagger}}}

\newcommand{\uleft}[1]{\ensuremath{u_{#1}^l}}
\newcommand{\uright}[1]{\ensuremath{u_{#1}^r}}
\newcommand{\usl}{\uleft{\*}}
\newcommand{\usr}{\uright{\*}}
\newcommand{\upl}{\uleft{+}}
\newcommand{\upr}{\uright{+}}
\newcommand{\nm}{\ensuremath{n\times{}m}}
\newcommand{\mn}{\ensuremath{m\times{}n}}


%--------------------------------------------------------------------------------------
% Notation for inverse categories
\newcommand{\Xt}{\ensuremath{\widetilde{\X}}\xspace}


\newcommand{\hypXt}{\texorpdfstring{\Xt}{Xt}}
\newcommand{\hypX}{\texorpdfstring{\X}{X}}

\newcommand{\xtdmn}[2]{\ensuremath{{#1}_{|{#2}}}}


\newcommand{\xequiv}[1]{\ensuremath{\overset{\scriptscriptstyle #1}{\simeq}}}


%------------------------------------------------------------------------------------
% General notation for categories

\newcommand{\specialcat}[1]{\textsc{#1}\xspace}
\newcommand{\ltrcat}[1]{\ensuremath{\mathfrak{#1}}\xspace}
\newcommand{\ltrcatbb}[1]{\ensuremath{\mathbb{#1}}\xspace}
\newcommand{\B}{\ltrcatbb{B}}
\newcommand{\C}{\ltrcatbb{C}}
\newcommand{\D}{\ltrcatbb{D}}
\newcommand{\E}{\ltrcatbb{E}}
\newcommand{\Lat}{\ltrcatbb{L}}
\newcommand{\X}{\ltrcatbb{X}}
\newcommand{\Y}{\ltrcatbb{Y}}
\newcommand{\Z}{\ltrcatbb{Z}}


\newcommand{\Hil}{\ensuremath{\mathcal{H}}\xspace}

\newcommand{\sets}{\specialcat{Sets}}
\newcommand{\rel}{\specialcat{Rel}}
\newcommand{\fdh}{\specialcat{FdHilb}}
\newcommand{\mon}{\specialcat{Mon}}
\newcommand{\ring}{\specialcat{Rng}}
\newcommand{\Par}{\specialcat{Par}}
\newcommand{\cring}{\specialcat{CRng}}
\newcommand{\cat}{\specialcat{Cat}}
\newcommand{\poset}{\specialcat{Poset}}
\newcommand{\preorder}{\specialcat{Preorder}}
\newcommand{\mcat}{\ensuremath{\mathcal{M}\text{\cat}}\xspace}

\newcommand{\Mstab}{\ensuremath{\mathcal{M}}\xspace}

\newcommand{\obj}[1]{\ensuremath{#1_{obj}}}
\newcommand{\bottom}[1]{\perp_{#1}}
\newcommand{\finpower}{\mathscr{P}_{fin}}

\newcommand{\category}[4]{%
\begin{description}%
\item{\textbf{Objects: }}{#1}%
\item{\textbf{Maps: }}{#2}%
\item{\textbf{Identity: }}{#3}%
\item{\textbf{Composition: }}{#4}%
 \end{description}%
%
}

\newcommand{\pfcategory}[3]{%
\begin{description}%
\item{\textbf{Well-Defined: }}{#1}%
\item{\textbf{Identities: }}{#2}%
\item{\textbf{Associativity: }}{#3}%
\end{description}}



% Additional gates

\newcommand{\nowiregate}[1]{*{\xy *+<.6em>{#1};p\save+LU;+RU **\dir{-}\restore\save+RU;+RD **\dir{-}\restore\save+RD;+LD **\dir{-}\restore\POS+LD;+LU **\dir{-}\endxy}}

%-------------------------------------------------
% subscripting

\newcommand{\jay}{\ensuremath{j}\xspace}
\newcommand{\kay}{\ensuremath{k}\xspace}

%!TEX root = /Users/gilesb/UofC/thesis/phd-thesis/phd-thesis.tex
%\floatstyle{boxed}
%\newfloat{QuantumCircuit}{htp}{qci}[chapter]
%\floatname{QuantumCircuit}{Circuit}

%\setlength{\textheight}{8.5in}
%\setlength{\textwidth}{6.5in}
%\setlength{\oddsidemargin}{-.3in}
%\setlength{\evensidemargin}{-.3in}
% macros for processing of the literate files


% ----------------------------------------------------------------------
% Theorem like environments

\theoremstyle{plain}
\newtheorem{theorem}{Theorem}[section]
\newtheorem{lemma}[theorem]{Lemma}
\newtheorem{proposition}[theorem]{Proposition}
\newtheorem{conjecture}[theorem]{Conjecture}
\newtheorem{corollary}[theorem]{Corollary}

\theoremstyle{definition}
\newtheorem{definition}[theorem]{Definition}
\newtheorem{remark}[theorem]{Remark}
\newtheorem{convention}[theorem]{Convention}
\newtheorem{example}[theorem]{Example}
\newtheorem{notation}[theorem]{Notation}
\newtheorem{notethm}[theorem]{Note}

%\theoremstyle{definition}
\newtheorem*{sltheorem}{Theorem}
\newtheorem*{sldefinition}{Definition}
\newtheorem*{slproposition}{Proposition}
\newtheorem*{slnotation}{Notation}
\newtheorem*{slexample}{Example}
\newtheorem*{slexercise}{Exercise}
\newtheorem*{sllemma}{Lemma}

% reference formats -
\labelformat{chapter}{chapter~#1}
\labelformat{section}{section~#1}
\labelformat{subsection}{sub-section~#1}
\labelformat{subsubsection}{sub-sub-section~#1}
\labelformat{equation}{equation~(#1)}
\labelformat{table}{table~#1}
\labelformat{figure}{figure~#1}

%-------------------------------------------------------------------------------
% Code environments for Verbatim package

\DefineVerbatimEnvironment%
 {happycodefirst}{Verbatim}{gobble=2,numbers=left,firstnumber=1,fontsize=\footnotesize}
\DefineVerbatimEnvironment%
 {qplcodethesis}{Verbatim}{commandchars=\\\{\},numbers=left,firstnumber=1,fontsize=\footnotesize}
\DefineVerbatimEnvironment%
 {bnf}{Verbatim}{commandchars=\\\{\},fontfamily=courier,fontsize=\footnotesize}
\DefineVerbatimEnvironment%
 {happycodecont}{Verbatim}{gobble=2,numbers=left,firstnumber=last,fontsize=\footnotesize}
\DefineVerbatimEnvironment%
{code}{Verbatim}{numbers=left,fontsize=\footnotesize,firstnumber=1}
\newcommand{\CodeResetNumbers}{\RecustomVerbatimEnvironment%
{code}{Verbatim}{numbers=left,fontsize=\footnotesize,firstnumber=1}}
\newcommand{\CodeContinueNumbers}{\RecustomVerbatimEnvironment%
{code}{Verbatim}{numbers=left,fontsize=\footnotesize,firstnumber=last}}
\renewcommand{\C}{\ensuremath{\mathbb{C}}}

%----------------------------------------------------------------------------------
% Code environments and definitions for listings


\lstdefinestyle{hskl}{language=Haskell,
%  basicstyle=\small, % swap this and the following line for prop. font
  basicstyle=\small,
%  keywordstyle=\underbar, % these look ugly
  identifierstyle=\itshape,
  commentstyle=\ttfamily,
%  commentstyle=\underbar
  flexiblecolumns=false,
  basewidth={0.5em,0.45em},
  morekeywords={Map},
  % The following replace compound charcters like ->
  % Something is missing - someone kindly sent me an email which
  % I've lost - but it's obvious how to add more.
  literate={-}{{$-$}}1 {+}{{$+$}}1 {/}{{$/$}}1
     {*}{{$\times$}}1 {=}{{$=$}}1
     {\%}{{$\%$}}1 {=}{{$=$}}1
           {>}{{$>$}}1 {<}{{$<$}}1
           {>>}{{$\gg$}}2 {<<}{{$\ll$}}2
           {=>}{{$\Rightarrow$}}2 {>=}{{$\geq$}}2 {<-}{{$\leftarrow$}}2
           {<=}{{$\leq$}}2
           {<==}{{$\Longleftarrow$}}3
           {=/=}{{$\neq$}}3
}
\lstdefinestyle{origlinqpl}{language=lqpl,basicstyle=\footnotesize\ttfamily,%
  numbers=left,%
        numberstyle=\tiny,%
        numbersep=6pt }
\lstdefinestyle{linqpl}{language=lqpl,basicstyle=\footnotesize,%
  numbers=left,%
        numberstyle=\tiny,%
        numbersep=6pt,%
  escapechar=`,%
  literate={-}{{$-$}}1 {+}{{$+$}}1 {/}{{$/$}}1%
     {*}{{$\times$}}1 {=}{{$=$}}1%
     {\%}{{$\%$}}1 {=}{{$=$}}1%
           {>}{{$>$}}1 {<}{{$<$}}1%
           {>>}{{$\gg$}}2 {<<}{{$\ll$}}2%
           {=>}{{$\Rightarrow$}}2 {>=}{{$\geq$}}2%
           {=<}{{$\leq$}}2%
           {<=}{{$\Leftarrow$}}3%
           {=/=}{{$\neq$}}3 }
\lstdefinestyle{inlinqpl}{language=lqpl,basicstyle=\footnotesize\ttfamily,%
  escapechar=`}

%-----------------------------------------------------------------------------------
% Extra macros for describing the LQPL language

\newcommand{\lbl}{{\cdptr}}
\newcommand{\cd}{\ensuremath{\mathcal{C}}}
\newcommand{\lblcd}{{\cd}}
%\newcommand{\lblcd}{\ensuremath{@\lbl}}
\newcommand{\cdptr}{\ensuremath{\triangleright\cd}}
%\newcommand{\ 2 do}[1]{\begin{quote}2 do: \textbf{#1}\end{quote}}
\newcommand{\lqpl}{L-QPL}
\newcommand{\linearqpl}{Linear QPL}

\newcommand{\inlqpl}[1]{\protect{\lstinline[style=inlinqpl]!#1!}}
\newcommand{\inlhskl}[1]{\lstinline[style=hskl]!#1!}


\newcommand{\Had}{\text{Hadamard}}
\newcommand{\nottr}{\text{Not}}
\newcommand{\Cnot}{controlled{-}\nottr}

\newcommand{\stacknode}[1]{\ensuremath{\mathrm{\texttt{#1}}}}
\newcommand{\Int}{\stacknode{Int}}
\newcommand{\Datatype}{\stacknode{datatype}}
\newcommand{\Bool}{\stacknode{Bool}}



%-----------------------------------------------------------------------------------
% Macros for including code 
\newcommand{\incsubsec}[1]{\subsubsection{#1}}
\newcommand{\incsubsubsec}[1]{\paragraph{#1}}
\newcommand{\qplcode}[1]{\textbf{#1}}


%------------------------------------------------------------------------------------
% Formatting macros
\newcommand{\marginnote}[1]{\marginpar{\tiny #1}}

%------------------------------------------------------------------------------------
%  Notation for describing syntax of language (lqpl primarily)
\newcommand{\syntacticset}[1]{\ensuremath{\mathbf{#1}}}

\newcommand{\n}{\syntacticset{N}}
\newcommand{\T}{\syntacticset{T}}
\newcommand{\Q}{\syntacticset{Q}}
\newcommand{\R}{\syntacticset{R}}
\newcommand{\Data}{\syntacticset{Data}}
\newcommand{\Qloc}{\syntacticset{Qloc}}
\newcommand{\Cloc}{\syntacticset{Cloc}}
\newcommand{\Loc}{\syntacticset{Loc}}
\newcommand{\Aexp}{\syntacticset{Aexp}}
\newcommand{\Bexp}{\syntacticset{Bexp}}
\newcommand{\Cexp}{\syntacticset{Cexp}}
\newcommand{\Stm}{\syntacticset{Stm}}
\newcommand{\eval}[3]{\ensuremath{\langle{#1},{#2}\rangle\to{#3}}}
\newcommand{\true}{\ensuremath{\mathbold{true}}}
\newcommand{\false}{\ensuremath{\mathbold{false}}}


\newcommand{\qcons}[1]{\texttt{#1}}
\newcommand{\qtype}[1]{\texttt{\bf{#1}}}
\newcommand{\bms}{\specialcat{Bqsm}}
\newcommand{\lbms}{\specialcat{lBqsm}}
\newcommand{\cms}{\specialcat{Cqsm}}
\newcommand{\ms}{\specialcat{QSM}}

\newcommand{\qsp}{\ensuremath{\to}}

\newcommand{\qsnodeij}[4]{\ensuremath{#1\{{#2}\qsp {#3}\}^{#4}}}

\newcommand{\qsbit}[4]{\ensuremath{#1\{0\qsp {#2};1\qsp {#3}\}^{#4}}}



\newcommand{\qsqubit}[6]{\ensuremath{#1\{00\qsp {#2};01\qsp {#3};10\qsp {#4};11\qsp {#5}\}^{#6}}}

\newcommand{\qsbitUp}[4]{\ensuremath{#1\left\{\begin{array}{l}%
                  0\qsp {#2}\\%
                        1\qsp {#3}%
                        \end{array}%
               \right\}^{#4}}}


\newcommand{\qsqubitUp}[6]{\ensuremath{#1\left\{\begin{array}{ll}%
                  00\qsp {#2};&01\qsp {#3}\\%
                        10\qsp {#4};&11\qsp {#5}%
                        \end{array}%
               \right\}^{#6}}}
\newcommand{\qsqubitAllUp}[6]{\ensuremath{#1\left\{\begin{array}{l}%
                  00\qsp {#2}\\%
                        01\qsp {#3}\\%
                        10\qsp {#4}\\%
                        11\qsp {#5}%
                        \end{array}%
               \right\}^{#6}}}



\newcommand{\eentail}{\ensuremath{\Vdash\!\!\dashv}}
\newcommand{\ientail}{\ensuremath{\Vdash}}
\newcommand{\qentail}{\ensuremath{\vdash}}
\newcommand{\qcontext}{\ensuremath{\, |\, }}
\newcommand{\qcontrolled}[2]{\ensuremath{{#1}\Leftarrow{#2}}}
\newcommand{\qvec}[1]{\ensuremath{\widetilde{#1}}}
\newcommand{\qop}[2]{\ensuremath{{#1}\cdot{#2}}}
\newcommand{\qopseq}[2]{\ensuremath{{#1};{#2}}}

\newcommand{\qresultsind}[3]{\ensuremath{{#2}\overset{#1}{\leadsto}{#3}}}
\newcommand{\qresultsin}[2]{\qresultsind{}{#1}{#2}}
\newcommand{\qresultsindUp}[3]{\ensuremath{\begin{array}{l}%
             {#2}\overset{#1}{\leadsto}\\%
  \qquad{#3}}%
          \end{array}}
\newcommand{\qresultsinUp}[2]{\qresultsindUp{}{#1}{#2}}
\newcommand{\stackthree}[3]{\ensuremath{\begin{array}{l}{#1}\\{#2}\\{#3}\end{array}}}
\newcommand{\stackthreec}[3]{\ensuremath{\begin{array}{c}{#1}\\{#2}\\{#3}\end{array}}}
\newcommand{\stacktwo}[2]{\ensuremath{\begin{array}{l}{#1}\\{#2}\end{array}}}
\newcommand{\stacktwoc}[2]{\ensuremath{\begin{array}{c}{#1}\\{#2}\end{array}}}
\newcommand{\qmeasnob}[3]{{\begin{singlespace}\stackthree{\text{meas }{#1}:}{\quad \ket{0}=>#2}{\quad \ket{1}=>#3}\end{singlespace}}}
\newcommand{\qmeas}[3]{{\begin{singlespace}\left\{\stackthree{\text{meas }{#1}:}{\ \ket{0}=>#2}{\ \ket{1}=>#3}\right\}\end{singlespace}}}
\newcommand{\qcompose}[2]{\ensuremath{{#1};{#2}}}
\newcommand{\qtensor}[2]{\ensuremath{{#1};;{#2}}}
\newcommand{\qins}{\ensuremath{\mathscr{I}}}
\newcommand{\qmod}{\ensuremath{\mathscr{M}}}
\newcommand{\qcase}[4]{{\begin{singlespace}\left\{\stacktwo{\text{case }{#1}:}{\{{\ {#2}=>{#3}}\}_{#4}}\right\}\end{singlespace}}}
\newcommand{\quse}[2]{{\left\{\text{use }{#1}:\ \{{#2}\}\right\}}}
\newcommand{\tcls}{\ensuremath{\tau_C}}
\newcommand{\semins}[1]{\texttt{#1}}
\newcommand{\qifelse}[6]{{\begin{singlespace}\left\{\stackthree{\text{if }{#1}=>{#2}}{{\{\ {#3}=>{#4}}\}_{#5}}{\text{else }=>{#6}}\right\}\end{singlespace}}}


\newcommand{\qsmins}[1]{\texttt{#1}}

\newcommand{\qsminsparm}[1]{\ensuremath{#1}}
\newcommand{\qsminswithp}[2]{\texttt{#1}\qsminsparm{\ #2}}
\newcommand{\dmpelemqc}{\text{\texttt{Qc}}}
\newcommand{\qsmbool}[1]{\text{\texttt{#1}}}
\newcommand{\qsmfalse}{\qsmbool{False}}
\newcommand{\qsmtrue}{\qsmbool{True}}
\newcommand{\qstackMod}[1]{\texttt{#1}}

\newcommand{\terminalio}[1]{\texttt{#1}}


\newcommand{\interpsem}[1]{\ensuremath{\left\llbracket {#1} \right\rrbracket}}

\newcommand{\Inflist}{\text{Inflist}}

\newcommand{\trspace}{\ensuremath{\qquad\qquad\qquad\qquad\qquad}}
\newcommand{\trbigspace}{\ensuremath{\qquad\qquad\qquad\qquad\qquad\qquad\qquad}}

\newcommand{\trspacefour}{\ensuremath{\qquad\qquad\qquad\qquad}}
\newcommand{\trspacethree}{\ensuremath{\qquad\qquad\qquad}}
\newcommand{\trspacetwo}{\ensuremath{\qquad\qquad}}

\newcommand{\ilsep}{\ensuremath{\blacktriangleright}}

\newcommand{\lqplmodifier}[1]{\text{\texttt{#1}}}
\newcommand{\IdOnly}{\lqplmodifier{IdOnly}}
\newcommand{\Left}{\lqplmodifier{Left}}
\newcommand{\Right}{\lqplmodifier{Right}}



% import and redefinition of general macros

\newcommand{\inflist}[1]{\ensuremath{\mathbb{IL}({#1})}}
\newcommand{\incsec}[1]{\subsection{#1}}
