%!TEX root = /Users/gilesb/UofC/thesis/phd-thesis/phd-thesis.tex
\chapter{Inverse categories} % (fold)
\label{cha:inverse_categories}

This chapter will introduce inverse categories. We first give
a few results about inverse categories and then proceed to show an inverse category which has
restriction products is a restriction pre-order.

Given this, the chapter then turns to focus on adding ``products'' to an inverse category, by which
we mean a structure that behaves in a product-like manner in the inverse category, which we call
\emph{inverse products}. These will be defined below in
Sub-Section~\ref{sub:inverse_product_definition}: Inverse products are given by a natural structure
on a tensor product which includes a diagonal but lacks projections. The diagonal map is required to
give a natural Frobenius structure on each object.

\section{Inverse Categories}
\label{sec:inverse_categories}

\begin{definition}\label{def:inverse_category}
  A restriction category in which every map is a partial
  isomorphism is called an \emph{inverse category}.
\end{definition}

\begin{lemma}
  \label{lem:inverse_idempotents_are_restriction_idempotents}
  In an inverse category, all idempotents are restriction idempotents.
\end{lemma}
\begin{proof}
  Given an idempotent $e$,
  \[
    \rst{e} = e\inv{e} = e e \inv{e} = e \rst{e} = \rst{e e} e = \rst{e} e = e.
  \]
\end{proof}

\begin{lemma}\label{lem:inverse_categories_are_range_categories}
  An inverse category \X is a range category, where $\rg{f} = \inv{f}f = \rst{\inv{f}}$.
\end{lemma}
\begin{proof}
  \prepprooflist
  \setlist[enumerate,1]{leftmargin=1.5cm}
  \begin{enumerate}
    \item[\rrone] $\restr{\rg{f}} = \rst{\rst{\inv{f}}} = \rst{\inv{f}} = \rg{f}$;
    \item[\rrtwo] $f \rg{f} = f \rst{\inv{f}} = f \inv{f} f = \rst{f} f = f$;
    \item[\rrthree] $\wrg{f\rst{g}} = \rst{\inv{(f\rst{g})}} = \rst{\inv{\rst{g}} \inv{f}} =
      \rst{\rst{g} \inv{f}} =
      \rst{g} \rst{\inv{f}} = \rst{\inv{f}} \rst{g} =\rg{f} \rst{g}$;
    \item[\rrfour]  $\wrg{\rg{f}g} = \rst{\inv{(\rst{\inv{f}} g)}} =
      \rst{\inv{g}\inv{\rst{\inv{f}}}} = \rst{\inv{g} \rst{\inv{f}}} =
      \rst{\inv{g} \inv{f}} = \rst{\inv{(f g)}} = \wrg{f g}$
  \end{enumerate}
\end{proof}

The property of being an inverse category is
preserved by splitting.

\begin{lemma}\label{lem:the_idempotent_splitting_of_an_inverse_category_is_an_inverse_category}
  When \X is an inverse category, $\spl{E}{X}$ is an inverse category.
\end{lemma}
\begin{proof}
  The inverse of $f:(A,e_1)\to(B,e_2)$   in \spl{E}{\X} is $e_2\inv{f}e_1$ as
  \[
    \llbracket f \inv{f} \rrbracket = e_1 f e_2 e_2 \inv{f} e_1
    = e_1 e_1 f e_2 \inv{f} e_1
    = e_1 f  \inv{f} e_1
    = e_1 e_1 \restr{f} e_1
    = e_1 \restr{f}
    = \llbracket\restr{f}\rrbracket
  \]
  and
  \begin{multline*}
    \llbracket \inv{f} f\rrbracket=
    e_2 \inv{f} e_1 e_1 f e_2
    = e_2 \inv{f} e_1 f e_2 e_2
    = e_2 \inv{f} f  e_2\\
    = e_2 e_2 \restr{\inv{f}}  e_2
    = e_2 \restr{\inv{f}}
    = \llbracket\restr{\inv{f}}\rrbracket.
  \end{multline*}

\end{proof}

\begin{example}[\pinj is an inverse category]\label{ex:pinj_is_an_inverse_category}
  For any map $f$, $\inv{f} = \{(y,x) | (x,y) \in
  f\}$. Note that $\inv{f}$ is a map in $\pinj$ due to the two dual conditions on maps as given in
  Example~\ref{ex:category_pinj}.
\end{example}
\begin{example}[\Par is not an inverse category]\label{ex:par_is_not_an_inverse_category}
  \Par, while it is a restriction category, is not an inverse category. For example, let
  $A=\{1,2\},\ B=\{1\}$ and $f=\{(1,1),(2,1)\}$ in \Par. The restriction of $f$ is $\rst{f} =
  \{(1,1),(2,2)\} = 1_A$. There is no partial function $g:B\to A$ such that $f g = 1_A$.
\end{example}
\begin{example}\label{ex:partial_isos_are_inverse_category}
  Generally, let \R be a restriction category, and \X the sub-category of \R having the same objects
  as \R and only the partial isomorphisms as maps. Then, \X is an inverse category.
\end{example}
\begin{example}\label{ex:groupoid_is_inverse_category}
  A groupoid, which is a category in which every map is an isomorphism, is an inverse category. Note
  that all maps in the groupoid are total, hence the partial isomorphisms are all isomorphisms.
\end{example}
\begin{example}\label{ex:partial_map_monics_is_inverse_category}
  Given a category $\cP$, create a partial map category as in
  Section~\ref{sub:partial_map_categories}, where the stable system of monics are all isomorphisms
  in $\cP$. Then the partial isomorphisms are the maps of the form
  $\xymatrix @R-15pt @C-15pt{&A'\ar[dl]_{m} \ar[dr]^{m'}\\A&&B}$, with inverse
  $\xymatrix @R-15pt @C-15pt{&A'\ar[dl]_{m'} \ar[dr]^{m}\\B&&A}$. The composition is
  $\xymatrix @R-15pt @C-15pt{&A\ar[dl]_{m^{-1}m} \ar[dr]^{m^{-1}m}\\A&&A}$ which is the
  identity. Taking just the partial isomorphisms gives us an inverse category.
\end{example}

\section{Inverse categories with restriction products} % (fold)
\label{sec:inverse_categories_with_restriction_products}
We start by showing that an inverse category with restriction products is a restriction pre-order
and thus, is a very restrictive notion.
\begin{proposition}\label{prop:an_inverse_category_with_products_is_a_restriction_preorder}
  Given an inverse category \X, if it has restriction products, it is a restriction pre-order as in
  Definition~\ref{def:compatible_maps}. That is,
  \[
    \xymatrix {
      A  \ar@<1ex>[r]^{f} \ar@<-1ex>[r]_{g} &B
    }
    \implies f \compatible g.
  \]
\end{proposition}
\begin{proof}
  Notice that $ \inv{\pi_1}  = \Delta \pi_1 \inv{\pi_1} =\Delta \restr{\pi_1} = \Delta$.
  This gives $\restr{\inv{\pi_1}} = 1$ and therefore $\pi_1$ (and similarly, $\pi_0$) is an
  isomorphism.

  Starting with the product map $\<f,g\>$,
  \[
    \infer={\restr{f}g = \restr{g}f}
    {\infer={\restr{f}g\Delta = \restr{g}f\Delta}
    {\infer={\restr{f}g\inv{\pi_1} = \restr{g}f\inv{\pi_0}}
    {\infer={\<f,g\>\pi_1 \inv{\pi_1} = \<f,g\>\pi_0 \inv{\pi_0}}
    {\<f,g\> = \<f,g\>}}}}
  \]
  which shows that $f$ and $g$ are compatible.
\end{proof}

\begin{corollary}
  \X\ is a inverse category with restriction products if and only if Total($\spl{r}{\X}$) is a meet pre-order.
\end{corollary}

\begin{proof}
  Total(\X), the subcategory of total maps on \X, has products and therefore every pair of parallel
  maps is compatible. As total compatible maps are equal, there is at most
  one map between any two objects. Hence, Total(\X) is a pre-order with the meet being the product.

  Similarly, from \cite{cockett2002:restcategories1} and \cite{cockettlack2004:restcategories3},
  Total($\spl{r}{\X}$) is an inverse category and has products and is therefore also a meet
  pre-order. This shows the ``only if'' side of the corollary.

  For the other direction, if Total($\spl{r}{\X}$) is a meet pre-order, define the product as the
  meet of the maps and the terminal object as the supremum of all maps.
\end{proof}

\begin{corollary}
  Every inverse category with restriction products is a full subcategory of a partial map category
  of a meet semi-lattice.
\end{corollary}


% section inverse_categories_with_restriction_products (end)
\section{Inverse products} % (fold)
\label{sec:inverse_products}

\subsection{Inverse product definition} % (fold)
\label{sub:inverse_product_definition}

%\begin{definition}\label{def:inverse_product_tensor}
%  Given a restriction category \X, a tensor $\*$ is called an \emph{inverse product tensor}
%   when:
% \begin{itemize}
%   \item $\*$ is a restriction functor, $\_ \* \_ : \X \times \X \to \X$.
%   \item $\*$ is a symmetric monoidal tensor satisfying the standard symmetric monoidal equations and
%     coherence diagrams hold (see, e.g., \cite{maclan97:categorieswrkmath}) and has the following
%     natural isomorphisms:
%     \begin{align*}
%       1 &: \boldsymbol{1}\to \X \\
%       \usl &: 1 \* A \xrightarrow{\cong} A
%       &\usr &: A \* 1 \xrightarrow{\cong} A\\
%       a_{\*} &: (A \* B) \* C \xrightarrow{\cong} A \* (B \* C)
%       &c_{\*} &: A \* B \xrightarrow{\cong} B \* A.
%     \end{align*}
%   \end{itemize}
%   Note that since all the coherence maps are isomorphisms, they are total.
% \end{definition}

\begin{definition}\label{def:inverse_product}
  An \emph{inverse product} on an inverse category \X is given by a symmetric tensor product, based
  on a restriction  bi-functor, $\_\*\_:\X\times\X \to \X$. The tensor makes \X a  symmetric
  monoidal category and there is a natural  ``semi-Frobenius'' diagonal map, $\Delta$ which is canonical.   If an
  inverse category has inverse products, it is called  a \emph{discrete inverse category}.


  The diagonal map $\Delta_A:A \to A\*A$ must be total and create a co-semigroup. It
  must satisfy the following diagrams:
 % in Figures~\ref{fig:inverse_product_cocommutativity},
 %  \ref{fig:inverse_product_coassociativity}, \ref{fig:inverse_product_exchange}, and
 %  \ref{fig:inverse_product_frobenius}.
  \[
    \xymatrix @!0 @C=90pt @R=35pt{
      A \ar[dr]_{\Delta} \ar[r]^{\Delta} &
      A \* A \ar[d]^{c_{\*}}\\
      & A \* A
    }\qquad
    \xymatrix @!0 @C=90pt @R=35pt{
      A \* A \ar[d]^{c_{\*}} \ar[r]^{\inv{\Delta}}  & A\\
      A \* A  \ar[ur]_{\inv{\Delta}}
    }
  \]
\begin{center}Cocommutative and Commutative\end{center}
%\caption*{Cocommutative}\label{fig:inverse_product_cocommutativity}

  \[
    \xymatrix @C=20pt @R=30pt{
      A \ar[rr]^{\Delta} \ar[d]_{\Delta} & &
      A \* A \ar[d]^{1\*\Delta}\\
      A\*A \ar[dr]_{\Delta \* 1}& &
      A \* ( A \* A) \\
      &   (A \* A) \* A \ar[ur]_{a_{\*}}
    }\qquad
    \xymatrix @C=20pt @R=30pt{
      A\*(A\*A) \ar[rr]^{1\*\idelta} \ar[d]_{\inv{a_\*}} && A\*A \ar[d]^{\idelta}\\
      (A\*A)\*A \ar[dr]_{\idelta\*1}&&A\\
     & A\*A \ar[ur]_{\idelta}
    }
  \]
\begin{center}Coassociative and Associative\end{center}
%\caption*{Coassociative and Associative}\label{fig:inverse_product_coassociativity}
If we define the map:
  \[
    \excs =  a_{\*}(1\*\inv{a_{\*}})(1\*(c_{\*}\*1))(1\* a_{\*})\inv{a_{\*}}): (A\*B)\*(C\*D) \to (A\*C) \* (B\*D).
  \]

  \[
    \xymatrix @C=25pt @R=35pt{
      A \* B \ar[dr]_{\Delta}
      \ar[rr]^{\Delta \* \Delta} && (A \* A) \* (B \* B) \ar[dl]^{\excs}\\
      &(A \* B) \* (A \* B)
    }
  \]
\begin{center}$\Delta$ is canonical\end{center}
%\caption{Exchange}\label{fig:inverse_product_exchange}

  \[
    \xymatrix @C=40pt @R=25pt{
      A \* A \ar[dd]_{(1\*\Delta) \inv{a_{\*}}} \ar[dr]^{\inv{\Delta}}
      \ar[rr]^{(\Delta \* 1) a_{\*}} & &
      A \* (A \* A) \ar[dd]^{1 \* \inv{\Delta}}\\
      & A \ar[dr]^{\Delta} & \\
      (A \* A) \* A \ar[rr]_{\inv{\Delta} \* 1} & &
      A \* A
    }
  \]
\begin{center}Frobenius\end{center}
%\caption{Frobenius}\label{fig:inverse_product_frobenius}


  Thus, $\Delta$ is a cocommutative, coassociative map which together with $\inv{\Delta}$ forms a
  special semi-Frobenius algebra. We use the prefix ``semi-'' as we are not requiring the unit laws
  of a Frobenius algebra as per Definition~\ref{def:frobeniusalgebra}.
\end{definition}

\begin{remark}
  Note also, cocommutativity implies commutativity, i.e., that $c_{\*}\inv{\Delta} = \inv{\Delta}$.
  One can see this as:
  \begin{align*}
    \Delta(c_{\*}\inv{\Delta})
      &= (\Delta c_{\*})\inv{\Delta} = \Delta\inv{\Delta} = \rst{\Delta} \text{ and}\\
    (c_{\*}\inv{\Delta})\Delta
      & = (c_{\*}\inv{\Delta})(\Delta c_{\*}) = \rst{c_{\*}\inv{\Delta}}.
  \end{align*}
  This means that both $\inv{\Delta}$ and $c_{\*}\inv{\Delta}$ are partial inverses for $\Delta$
  and are therefore equal.

  Similarly, coassociativity implies associativity as one can show that
  $(1\*\inv{\Delta})\inv{\Delta} =  \inv{a_{\*}}(\inv{\Delta}\* 1)\inv{\Delta}$ as
  \begin{align*}
    \Delta(1\*\Delta)\inv{a_{\*}}(\inv{\Delta}\* 1)\inv{\Delta} &=
    \Delta(\Delta\*1)a_{\*}\inv{a_{\*}}(\inv{\Delta}\* 1)\inv{\Delta} \\
    & =\Delta(\Delta\*1)(\inv{\Delta}\* 1)\inv{\Delta} \\
    & =\Delta 1\inv{\Delta} = 1.
  \end{align*}
\end{remark}

\begin{example}[\pinj is a discrete inverse category]\label{ex:pinj_is_a_discrete_inverse_category}
  In the inverse category \pinj (see Example~\ref{ex:pinj_is_an_inverse_category}), suppose we add
  the tensor given by the Cartesian product of sets. In detail, this means:
  \begin{align*}
    A \* B &= \{(a,b)| a\in A, b\in B\}\\
    f \* g &= \{((a,c),(b,d)) | (a,b) \in f, (c,d) \in g\}\\
    1 & = \{*\}\text{, a single element set.}
  \end{align*}
  The symmetric monoid isomorphisms are:
    \begin{align*}
      \usl &: \{(*,a)\} \mapsto \{a\}
      &\usr &: \{(a,*)\} \mapsto \{a\}\\
      a_{\*} &: \{((a,b),c)\} \mapsto \{(a,(b,c))\}
      &c_{\*} &: \{(a,b)\} \mapsto \{(b,a)\}
    \end{align*}

  Define $\Delta_A = \{(a,(a,a)) | a\in A\}$. Then \pinj is a discrete inverse category with the
  inverse product of $\*$. The required properties of cocommutativity, coassociativity and
  exchange are immediately obvious. To show the Frobenius rule for $\Delta$, first note that
  $\inv{\Delta}$ is defined only on the elements of $A\*A$ which agree in the first and second
  co-ordinate. We show the upper triangle of the Frobenius diagram in
  detail. Equation~\ref{eq:delta_inverse_delta} shows the result of applying $\Delta$ followed by
  $\inv{\Delta}$.
  \begin{equation}
    \Delta(\inv{\Delta}(A\*A)) = \Delta(\{a | (a,a) \in A\*A\})
    = \{(a,a) | (a,a) \in A\* A\}.\label{eq:delta_inverse_delta}
  \end{equation}
  Applying $(\Delta \* 1)a_{\*}$ to $A\*A$ is shown in Equation~\ref{eq:delta_tensor_one}.
  \begin{equation}
    a_{\*}(\Delta\*1(A\*A)) = a_{\*}(\{((a,a),a') | (a,a') \in A\*A\} = \{(a,(a,a')) | (a,a') \in
    A\* A\}.\label{eq:delta_tensor_one}
  \end{equation}
  Finally, applying $1 \* \inv{\Delta}$ to the result of Equation~\ref{eq:delta_tensor_one} gives us
  Equation~\ref{eq:one_tensor_delta_inverse}.
  \begin{equation}
    (1\*\inv{\Delta})(\{(a,(a,a')) | (a,a') \in  A\* A\} = \{(a,a) | (a,a) \in A\* A\} \label{eq:one_tensor_delta_inverse}.
  \end{equation}
  Thus, we have $\inv{\Delta}\Delta = (\Delta \* 1) a_{\*} (1\*\inv{\Delta})$ and the Frobenius
  condition is satisfied.
\end{example}

\begin{example}[\topcatp does not give a discrete inverse category]\label{ex:topcat_does_not_give_a_discrete_inverse_category}
  Recalling \topcatp from Example~\ref{ex:restriction_category_top}, we know that the partial
  isomorphisms of \topcatp form an inverse category --- \Inv{\topcatp}. Additionally, \topcatp has a
  product, given by the standard Cartesian product. This product does work as a tensor in
  \Inv{\topcatp}, but $\Delta$  is not a map in \Inv{\topcat} and hence it is not a discrete inverse
  category.
\end{example}

Inverse products are extra structure on an inverse category, rather than a property. An example to
demonstrate this is given next.

\begin{example}[Inverse products are additional structure]
  \label{example:invprodisstructure}
\end{example}
Any discrete category (i.e., a category with only the identity arrows) is a trivial inverse
category. To create an inverse product on a discrete category, add a commutative, associative,
idempotent multiplication, with a unit.

Let $\D$ be the discrete category with four objects $a,b,c$ and $d$. Then, define
two different inverse product tensors, $\*$ and $\odot$, with $d$ the unit of each as shown in
Table~\ref{tab:two_different_inverse_products}.

\begin{table}[!htbp]
  \begin{center}
  \begin{tabular}{|l||c|c|c|c|}
    \hline
    $\*$&a&b&c&d\\ \hline \hline
    a&a&a&a&a\\ \hline
    b&a&b&\textbf{b}&b\\ \hline
    c&a&\textbf{b}&c&c \\ \hline
    d&a&b&c&d \\ \hline
  \end{tabular}
  \qquad
  \begin{tabular}{|l||c|c|c|c|} \hline
    $\odot$&a&b&c&d\\ \hline \hline
    a&a&a&a&a\\ \hline
    b&a&b&\textbf{a}&b\\ \hline
    c&a&\textbf{a}&c&c \\ \hline
    d&a&b&c&d \\ \hline
  \end{tabular}
  \end{center}
  \caption{Two different inverse products on the same category.}
  \label{tab:two_different_inverse_products}
\end{table}

As $\D$ is discrete, $\Delta$ is forced to be the identity. One can check easily that each of the
the conditions for being an inverse product are satisfied by $\*$ and by $\odot$ with the trivial diagonal.

% subsection inverse_products (end)
\subsection{Diagrammatic Language} % (fold)
\label{sub:diagrammatic_language}

While it is certainly possible to prove results about inverse products using direct algebraic
manipulation, it is much more understandable to use circuits or string diagrams. See
\cite{selinger11:graphical} for a comparison of various graphical languages for monoidal categories.
As shown in
\cite{street-ross-1991-GTC-I}, diagrammatic reasoning is equivalent to reasoning algebraically for
symmetric monoidal categories.

In the diagrams, we will use the following representations:
\begin{itemize}
  \item $\Delta$ will be represented by an upward pointing triangle: \begin{tikzpicture}
      \path node [delta] at (0,0) {}; \end{tikzpicture}.
  \item $\inv{\Delta}$ by a downward triangle: \begin{tikzpicture}
      \path node [nabla] at (0,0) {}; \end{tikzpicture}.
  \item Maps by a rectangle with the name of the map inside: \begin{tikzpicture}
      \path node [map] at (0,0) {$\scriptstyle f$}; \end{tikzpicture}.
  \item Use of the tensor: \begin{tikzpicture}
      \node [style=tensor] (t1) at (0,0) {$\scriptstyle \*$}; \end{tikzpicture}.
  \item Unit introduction (often referred to as an $\eta$ map): \begin{tikzpicture}
      \path node [eta] at (0,0) {}; \end{tikzpicture}.
  \item Unit removal (often referred to as an $\epsilon$ map): \begin{tikzpicture}
      \path node [epsilon] at (0,0) {}; \end{tikzpicture}.
\end{itemize}
String diagrams in this thesis are to be read from top to bottom.

The axioms of Definition~\ref{def:inverse_product} then become:\\
\begin{tikzpicture}
  \begin{pgfonlayer}{nodelayer}
    \node [style=nothing] (0) at (-3.75, 5.5) {};
    \node [style=delta] (1) at (-3.75, 5) {};
    \node [style=nothing] (2) at (-4, 4.25) {};
    \node [style=nothing] (3) at (-3.5, 4.25) {};
    \node [style=nothing] (4) at (-3, 5) {$=$};
    \node [style=nothing] (5) at (-2.25, 5.5) {};
    \node [style=delta] (6) at (-2.25, 5) {};
    \node [style=nothing] (7) at (-2.5, 4.25) {};
    \node [style=nothing] (8) at (-2, 4.25) {};
    \node [style=nothing] (9) at (-3, 4) {Cocommutativity};
    \end{pgfonlayer}
    \begin{pgfonlayer}{edgelayer}
      \draw [] (0) to (1);
      \draw [] (1) to[out=305,in=90] (2);
      \draw [] (1) to[out=235,in=90] (3);
      \draw [] (5) to (6);
      \draw [] (6) to[out=235,in=90] (7);
      \draw [] (6) to[out=305,in=90] (8);
      \end{pgfonlayer}
\end{tikzpicture}
\hspace{15pt}
\begin{tikzpicture}
  \begin{pgfonlayer}{nodelayer}
    \node [style=nothing] (0) at (-3.75, 5.5) {};
    \node [style=delta] (1) at (-3.75, 5) {};
    \node [style=delta] (2) at (-4, 4.5) {};
    \node [style=nothing] (3a) at (-4.25, 4) {};
    \node [style=nothing] (3b) at (-3.75, 4) {};
    \node [style=nothing] (3c) at (-3.5, 4) {};
    \node [style=nothing] (4) at (-3, 5) {$=$};
    \node [style=nothing] (5) at (-2.25, 5.5) {};
    \node [style=delta] (6) at (-2.25, 5) {};
    \node [style=delta] (8) at (-2, 4.5) {};
    \node [style=nothing] (7a) at (-2.5, 4) {};
    \node [style=nothing] (7b) at (-2.25, 4) {};
    \node [style=nothing] (7c) at (-1.75, 4) {};
    \node [style=nothing] (9) at (-3, 3.5) {Coassociativity};
    \end{pgfonlayer}
    \begin{pgfonlayer}{edgelayer}
      \draw [] (0) to (1);
      \draw [] (1) to[out=305,in=90] (3c);
      \draw [] (1) to[out=235,in=90] (2);
      \draw [] (2) to[out=305,in=90] (3b);
      \draw [] (2) to[out=235,in=90] (3a);
      \draw [] (5) to (6);
      \draw [] (6) to[out=235,in=90] (7a);
      \draw [] (6) to[out=305,in=90] (8);
      \draw [] (8) to[out=305,in=90] (7c);
      \draw [] (8) to[out=235,in=90] (7b);
      \end{pgfonlayer}
\end{tikzpicture}
\hspace{15pt}
\begin{tikzpicture}
  \begin{pgfonlayer}{nodelayer}
    \node [style=nothing] (s0) at (-2.5, 3.75) {};
    \node [style=delta] (d0) at (-2.5, 3.25) {};
    \node [style=nothing] (e01) at (-3, 2.5) {};
    \node [style=nothing] (e02) at (-2, 2.5) {};
    \node [style=nothing] (7) at (-1.75, 3.25) {$=$};
    \node [style=nothing] (s2) at (-.625, 4.25) {};
    \node [style=tensor] (t1) at (-.625, 3.875) {$\scriptstyle \*$};
    \node [style=delta] (d1) at (-1, 3.25) {};
    \node [style=delta] (d2) at (-0.25, 3.25) {};
    \node [style=tensor] (t2) at (-1.125, 2.5) {$\scriptstyle \*$};
    \node [style=nothing] (e2) at (-1.125, 2) {};
    \node [style=tensor] (t3) at (-.125, 2.5) {$\scriptstyle \*$};
    \node [style=nothing] (e3) at (-0.125, 2) {};
    \node [style=nothing] (16) at (-1.5, 1.5) {Exchange};
    \end{pgfonlayer}
    \begin{pgfonlayer}{edgelayer}
      \draw [] (s0) to (d0);
      \draw [] (d0) to[out=235,in=90] (e01);
      \draw [] (d0) to[out=305,in=90] (e02);
      \draw [] (s2) to (t1);
      \draw [] (t1) to[out=235,in=90] (d1);
      \draw [] (t1) to[out=305,in=90] (d2);
      \draw [] (d1) to[out=235,in=90] (t2);
      \draw [] (d1) to[out=305,in=135] (t3);
      \draw [] (d2) to[out=235,in=45] (t2);
      \draw [] (d2) to[out=305,in=90] (t3);
      \draw [] (t2) to (e2);
      \draw [] (t3) to (e3);
      \end{pgfonlayer}
\end{tikzpicture}
\hspace{15pt}
\begin{tikzpicture}
  \begin{pgfonlayer}{nodelayer}
    \node [style=nothing] (0) at (-2.75, 3.75) {};
    \node [style=nothing] (1) at (-2.25, 3.75) {};
    \node [style=delta] (2) at (-2.75, 3.25) {};
    \node [style=nothing] (3) at (-2.75, 2) {};
    \node [style=nothing] (4) at (-2.25, 2) {};
    \node [style=nothing] (5) at (-0.5, 3) {$=$};
    \node [style=nothing] (6) at (0, 3.75) {};
    \node [style=nothing] (7) at (0.5, 3.75) {};
    \node [style=delta] (8) at (-1, 2.5) {};
    \node [style=delta] (9) at (0.5, 3.25) {};
    \node [style=nothing] (10) at (0, 2) {};
    \node [style=nothing] (11) at (0.5, 2) {};
    \node [style=nabla] (12) at (-2.25, 2.5) {};
    \node [style=nabla] (13) at (-1, 3.25) {};
    \node [style=nabla] (14) at (0, 2.5) {};
    \node [style=nothing] (15) at (-1.75, 3) {$=$};
    \node [style=nothing] (16) at (-1.25, 3.75) {};
    \node [style=nothing] (17) at (-0.75, 3.75) {};
    \node [style=nothing] (18) at (-1.25, 2) {};
    \node [style=nothing] (19) at (-0.75, 2) {};
    \node [style=nothing] (20) at (-1, 1.75) {Frobenius};
    \end{pgfonlayer}
    \begin{pgfonlayer}{edgelayer}
      \draw [] (0) to (2);
      \draw [] (2) to (3);
      \draw [] (7) to (9);
      \draw [] (9) to (11);
      \draw (1) to (12);
      \draw (2) to (12);
      \draw (12) to (4);
      \draw (13) to (8);
      \draw (16) to (13);
      \draw (17) to (13);
      \draw (8) to (18);
      \draw (8) to (19);
      \draw (14) to (10);
      \draw (9) to (14);
      \draw (6) to (14);
      \end{pgfonlayer}
\end{tikzpicture}
% end sub:diagrammatic_language


\subsection{Properties of discrete inverse categories} % (fold)
\label{sub:discrete_inverse_categories}

We now present some properties of discrete inverse categories.

\begin{lemma}\label{lem:properties_of_delta_and_tensor_in_a_discrete_inverse_category}
  In a discrete inverse category \X with the inverse product $\*$ and $\Delta$, where
  $e=\rst{e}$ is a restriction idempotent and $f,g,h$ are arrows in \X, the following are true:
  \begin{enumerate}[{(}i{)}]
    \item{}$e=\Delta (e\* 1) \inv{\Delta}$.\label{le:eisde1}
    \item{}$e\Delta (f \* g) = \Delta (e f \* g) $ (and $= \Delta (f \* e g) $ and
      $ = \Delta (e f \* e g)$.)\label{le:deltaefg}
    \item{}$ (f \* g e) \inv{\Delta} =(f \* g) \inv{\Delta} e $ (and $= (f e\* g) \inv{\Delta}$ and
      $ = (f e\* g e)\inv{\Delta}$.)\label{le:efginvdelta}
    \item{}$\restr{\Delta (f \* g) \inv{\Delta}} =
       \Delta(1\* g \inv{f})\inv{\Delta}$. \label{le:restfg}
    \item{} If $\Delta (h \* g) \inv{\Delta} = \restr{\Delta (h \* g) \inv{\Delta}}$ then
      $(\Delta (h \* g) \inv{\Delta}) h = \Delta (h \* g) \inv{\Delta}$.\label{le:hge}
    \item{}$\Delta (f\*1) = \Delta (g\*1) \implies f = g$.\label{le:dfgisfg}
    \item{}$(f\*1) = (g\*1) \implies f = g$.\label{le:fgisfg}
  \end{enumerate}
\end{lemma}
\begin{proof}
  \prepprooflist
  \begin{enumerate}[{(}i{)}]
    \item[\ref{le:eisde1}]
      \[
      \raisebox{30pt}{
      \begin{tikzpicture}
        \node at (0,1) (start) {};
        \node at (0,.5) [map] (e) {$\scriptstyle e$};
        \node at (0,0) (end) {};
        \draw (start) to (e);
        \draw (e) to (end);
      \end{tikzpicture}
      }
      \ \raisebox{45pt}{$=$}\
      \begin{tikzpicture}
        \node at (0,3) (start) {};
        \node at (0,2.5) [delta] (d1) {};
        \node at (0,2) [nabla] (n1) {};
        \node at (0,1.5) [delta] (d2) {};
        \node at (0,1) [nabla] (n2) {};
        \node at (0,.5) [map] (e) {$\scriptstyle e$};
        \node at (0,0) (end) {};
        \draw [] (start) to (d1);
        \draw [] (d1) to[out=235,in=125] (n1);
        \draw [] (d1) to[out=305,in=55] (n1);
        \draw [] (n1) to (d2);
        \draw [] (d2) to[out=235,in=125] (n2);
        \draw [] (d2) to[out=305,in=55] (n2);
        \draw [] (n2) to (e);
        \draw (e) to (end);
      \end{tikzpicture}
      \ \raisebox{45pt}{$=$}\
      \begin{tikzpicture}
        \node at (0,3) (start) {};
        \node at (0,2.5) [delta] (d1) {};
        \node at (-.25,2) [delta] (d2) {};
        \node at (.25,1.5) [nabla] (n1) {};
        \node at (0,1) [nabla] (n2) {};
        \node at (0,.5) [map] (e) {$\scriptstyle e$};
        \node at (0,0) (end) {};
        \draw [] (start) to (d1);
        \draw [] (d1) to[out=235,in=90] (d2);
        \draw [] (d1) to[out=305,in=55] (n1);
        \draw [] (d2) to[out=305,in=125] (n1);
        \draw [] (d2) to[out=235,in=125] (n2);
        \draw [] (n1) to[out=270,in=55] (n2);
        \draw [] (n2) to (e);
        \draw (e) to (end);
      \end{tikzpicture}
      \ \raisebox{45pt}{$=$}\
      \begin{tikzpicture}
        \node at (0,3) (start) {};
        \node at (0,2.5) [delta] (d1) {};
        \node at (-.25,2) [delta] (d2) {};
        \node at (-.5,1.5) [map] (e1) {$\scriptstyle e$};
        \node at (0,1.5) [map] (e2) {$\scriptstyle e$};
        \node at (.5,1.5) [map] (e3) {$\scriptstyle e$};
        \node at (.25,1) [nabla] (n1) {};
        \node at (0,.5) [nabla] (n2) {};
        \node at (0,0) (end) {};
        \draw [] (start) to (d1);
        \draw [] (d1) to[out=235,in=90] (d2);
        \draw [] (d1) to[out=305,in=90] (e3);
        \draw [] (d2) to[out=305,in=90] (e2);
        \draw [] (d2) to[out=235,in=90] (e1);
        \draw (e1) to[out=270,in=125] (n2);
        \draw (e2) to[out=270,in=125] (n1);
        \draw (e3) to[out=270,in=55] (n1);
        \draw [] (n1) to[out=270,in=55] (n2);
        \draw [] (n2) to (end);
      \end{tikzpicture}
      \ \raisebox{45pt}{$=$}\
      \begin{tikzpicture}
        \node at (0,3) (start) {};
        \node at (0,2.5) [delta] (d1) {};
        \node at (-.25,2) [map] (e1) {$\scriptstyle e$};
        \node at (-.25,1.5) [delta] (d2) {};
        \node at (.25,1) [nabla] (n1) {};
        \node at (.25,.5) [map] (e2) {$\scriptstyle e$};
        \node at (0,0) [nabla] (n2) {};
        \node at (0,-.5) (end) {};
        \draw [] (start) to (d1);
        \draw [] (d1) to[out=235,in=90] (e1);
        \draw (e1) to (d2);
        \draw [] (d1) to[out=305,in=55] (n1);
        \draw [] (d2) to[out=305,in=125] (n1);
        \draw [] (d2) to[out=235,in=125] (n2);
        \draw (n1) to (e2);
        \draw [] (e2) to[out=270,in=55] (n2);
        \draw [] (n2) to (end);
      \end{tikzpicture}
      \ \raisebox{45pt}{$=$}\
      \begin{tikzpicture}
        \node at (0,3) (start) {};
        \node at (0,2.5) [delta] (d1) {};
        \node at (-.25,2) [map] (e1) {$\scriptstyle e$};
        \node at (0,1.5) [nabla] (n1) {};
        \node at (0,1) [delta] (d2) {};
        \node at (.25,.5) [map] (e2) {$\scriptstyle e$};
        \node at (0,0) [nabla] (n2) {};
        \node at (0,-.5) (end) {};
        \draw [] (start) to (d1);
        \draw [] (d1) to[out=235,in=90] (e1);
        \draw [] (d1) to[out=305,in=55] (n1);
        \draw (e1) to[out=270,in=125] (n1);
        \draw [] (n1) to (d2);
        \draw [] (d2) to[out=305,in=90] (e2);
        \draw [] (d2) to[out=235,in=125] (n2);
        \draw (e2) to[out=270,in=55] (n2);
        \draw [] (n2) to (end);
      \end{tikzpicture}
      \ \raisebox{45pt}{$=$}\
      \begin{tikzpicture}
        \node at (0,3) (start) {};
        \node at (0,2.5) [delta] (d1) {};
        \node at (-.25,2) [map] (e1) {$\scriptstyle e$};
        \node at (0,1.5) [nabla] (n1) {};
        \node at (0,1) [delta] (d2) {};
        \node at (-.25,.5) [map] (e2) {$\scriptstyle e$};
        \node at (0,0) [nabla] (n2) {};
        \node at (0,-.5) (end) {};
        \draw [] (start) to (d1);
        \draw [] (d1) to[out=235,in=90] (e1);
        \draw [] (d1) to[out=305,in=55] (n1);
        \draw (e1) to[out=270,in=125] (n1);
        \draw [] (n1) to (d2);
        \draw [] (d2) to[out=235,in=90] (e2);
        \draw [] (d2) to[out=305,in=55] (n2);
        \draw (e2) to[out=270,in=125] (n2);
        \draw [] (n2) to (end);
      \end{tikzpicture}
      \ \raisebox{45pt}{$=$}\
      \]
      \[
      \begin{tikzpicture}
        \node at (0,3) (start) {};
        \node at (0,2.5) [delta] (d1) {};
        \node at (-.25,2) [map] (e1) {$\scriptstyle e$};
        \node at (-.25,1.5) [delta] (d2) {};
        \node at (.25,1) [nabla] (n1) {};
        \node at (-.25,.5) [map] (e2) {$\scriptstyle e$};
        \node at (0,0) [nabla] (n2) {};
        \node at (0,-.5) (end) {};
        \draw [] (start) to (d1);
        \draw [] (d1) to[out=235,in=90] (e1);
        \draw (e1) to (d2);
        \draw [] (d1) to[out=305,in=55] (n1);
        \draw [] (d2) to[out=305,in=125] (n1);
        \draw [] (d2) to[out=235,in=90] (e2);
        \draw (n1) to[out=270,in=55] (n2);
        \draw [] (e2) to[out=270,in=125] (n2);
        \draw [] (n2) to (end);
      \end{tikzpicture}
      \ \raisebox{45pt}{$=$}\
      \begin{tikzpicture}
        \node at (0,2.5) (start) {};
        \node at (0,2) [delta] (d1) {};
        \node at (-.25,1.5) [delta] (d2) {};
        \node at (-.5,1) [map] (e2) {$\scriptstyle e$};
        \node at (0,1) [map] (e1) {$\scriptstyle e$};
        \node at (.25,.5) [nabla] (n1) {};
        \node at (-.25,0) [map] (e3) {$\scriptstyle e$};
        \node at (0,-0.5) [nabla] (n2) {};
        \node at (0,-1) (end) {};
        \draw [] (start) to (d1);
        \draw [] (d1) to[out=235,in=90] (d2);
        \draw [] (d1) to[out=305,in=55] (n1);
        \draw [] (d2) to[out=305,in=90] (e1);
        \draw [] (d2) to[out=235,in=90] (e2);
        \draw (e1) to[out=270,in=125] (n1);
        \draw (e2) to (e3);
        \draw (n1) to[out=270,in=55] (n2);
        \draw [] (e3) to[out=270,in=125] (n2);
        \draw [] (n2) to (end);
      \end{tikzpicture}
      \ \raisebox{45pt}{$=$}
      \begin{tikzpicture}
        \node at (0,3) (start) {};
        \node at (0,2.5) [delta] (d1) {};
        \node at (-.25,2) [map] (e1) {$\scriptstyle e$};
        \node at (-.25,1.5) [delta] (d2) {};
        \node at (.25,1) [nabla] (n1) {};
        \node at (0,.5) [nabla] (n2) {};
        \node at (0,0) (end) {};
        \draw [] (start) to (d1);
        \draw [] (d1) to[out=235,in=90] (e1);
        \draw [] (d1) to[out=305,in=55] (n1);
        \draw (e1) to[out=270,in=90] (d2);
        \draw [] (d2) to[out=235,in=125] (n2);
        \draw [] (d2) to[out=305,in=125] (n1);
        \draw (n1) to[out=270,in=55] (n2);
        \draw [] (n2) to (end);
      \end{tikzpicture}
      \ \raisebox{45pt}{$=$}
      \begin{tikzpicture}
        \node at (0,3) (start) {};
        \node at (0,2.5) [delta] (d1) {};
        \node at (-.25,2) [map] (e1) {$\scriptstyle e$};
        \node at (0,1.5) [nabla] (n1) {};
        \node at (0,1) [delta] (d2) {};
        \node at (0,.5) [nabla] (n2) {};
        \node at (0,0) (end) {};
        \draw [] (start) to (d1);
        \draw [] (d1) to[out=235,in=90] (e1);
        \draw [] (d1) to[out=305,in=55] (n1);
        \draw (e1) to[out=270,in=125] (n1);
        \draw [] (n1) to (d2);
        \draw [] (d2) to[out=235,in=125] (n2);
        \draw [] (d2) to[out=305,in=55] (n2);
        \draw [] (n2) to (end);
      \end{tikzpicture}
      \ \raisebox{45pt}{$=$}\
      \raisebox{15pt}{
        \begin{tikzpicture}
        \node at (0,3) (start) {};
        \node at (0,2.5) [delta] (d1) {};
        \node at (-.25,2) [map] (e1) {$\scriptstyle e$};
        \node at (0,1.5) [nabla] (n1) {};
        \node at (0,1) (end) {};
        \draw [] (start) to (d1);
        \draw [] (d1) to[out=235,in=90] (e1);
        \draw [] (d1) to[out=305,in=55] (n1);
        \draw (e1) to[out=270,in=125] (n1);
        \draw [] (n1) to (end);
      \end{tikzpicture}
      }
      \,\raisebox{45pt}{.}
      \]
    \item[\ref{le:deltaefg}]This equality uses the previous equality, the commutativity
      of restriction idempotents (\rtwo) and the identity $\Delta\rst{\inv{\Delta}} = \Delta$.
      \[
      \raisebox{10pt}{
      \begin{tikzpicture}
        \node at (0,2) (start) {};
        \node at (0,1.5) [map] (e) {$\scriptstyle e$};
        \node at (0,1) [delta] (d2) {};
        \node at (-.25,.5) [map] (f) {$\scriptstyle f$};
        \node at (.25,.5) [map] (g) {$\scriptstyle g$};
        \node at (-.25,0) (end1) {};
        \node at (.25,0) (end2) {};
        \draw (start) to (e);
        \draw (e) to (d2);
        \draw (d2) to[out=235,in=90] (f);
        \draw (d2) to[out=305,in=90] (g);
        \draw (f) to (end1);
        \draw (g) to (end2);
      \end{tikzpicture}
      }
      \ \raisebox{45pt}{$=$}\
        \begin{tikzpicture}
        \node at (0,3) (start) {};
        \node at (0,2.5) [delta] (d1) {};
        \node at (-.25,2) [map] (e1) {$\scriptstyle e$};
        \node at (0,1.5) [nabla] (n1) {};
        \node at (0,1) [delta] (d2) {};
        \node at (-.25,.5) [map] (f) {$\scriptstyle f$};
        \node at (.25,.5) [map] (g) {$\scriptstyle g$};
        \node at (-.25,0) (end1) {};
        \node at (.25,0) (end2) {};
        \draw [] (start) to (d1);
        \draw [] (d1) to[out=235,in=90] (e1);
        \draw [] (d1) to[out=305,in=55] (n1);
        \draw (e1) to[out=270,in=125] (n1);
        \draw [] (n1) to (d2);
        \draw (d2) to[out=235,in=90] (f);
        \draw (d2) to[out=305,in=90] (g);
        \draw (f) to (end1);
        \draw (g) to (end2);
      \end{tikzpicture}
      \ \raisebox{45pt}{$=$}\
        \begin{tikzpicture}
        \node at (0,3) (start) {};
        \node at (0,2.5) [delta] (d1) {};
        \node at (-.25,2) [map] (e1) {$\scriptstyle e$};
        \node at (0,1.25) [map] (inverse-delta-r) {$\scriptstyle \rst{\inv{\Delta}}$};
        \node at (-.25,.5) [map] (f) {$\scriptstyle f$};
        \node at (.25,.5) [map] (g) {$\scriptstyle g$};
        \node at (-.25,0) (end1) {};
        \node at (.25,0) (end2) {};
        \draw [] (start) to (d1);
        \draw [] (d1) to[out=235,in=90] (e1);
        \draw [] (d1) to[out=305,in=55] (inverse-delta-r);
        \draw (e1) to[out=270,in=125] (inverse-delta-r);
        \draw (inverse-delta-r) to[out=235,in=90] (f);
        \draw (inverse-delta-r) to[out=305,in=90] (g);
        \draw (f) to (end1);
        \draw (g) to (end2);
      \end{tikzpicture}
      \ \raisebox{45pt}{$=$}\
        \begin{tikzpicture}
        \node at (0,3) (start) {};
        \node at (0,2.5) [delta] (d1) {};
        \node at (0,2) [map] (inverse-delta-r) {$\scriptstyle \rst{\inv{\Delta}}$};
        \node at (-.25,1.25) [map] (e1) {$\scriptstyle e$};
        \node at (-.25,.5) [map] (f) {$\scriptstyle f$};
        \node at (.25,.5) [map] (g) {$\scriptstyle g$};
        \node at (-.25,0) (end1) {};
        \node at (.25,0) (end2) {};
        \draw [] (start) to (d1);
        \draw [] (d1) to[out=235,in=125] (inverse-delta-r);
        \draw [] (d1) to[out=305,in=55] (inverse-delta-r);
        \draw (inverse-delta-r) to[out=235,in=90] (e1);
        \draw (e1) to (f);
        \draw (inverse-delta-r) to[out=305,in=90] (g);
        \draw (f) to (end1);
        \draw (g) to (end2);
      \end{tikzpicture}
      \ \raisebox{45pt}{$=$}\
      \raisebox{20pt}{
        \begin{tikzpicture}
        \node at (0,1.5) (start) {};
        \node at (0,1) [delta] (d1) {};
        \node at (-.25,.5) [map] (f_e) {$\scriptstyle e f$};
        \node at (.25,.5) [map] (g) {$\scriptstyle g$};
        \node at (-.25,0) (end1) {};
        \node at (.25,0) (end2) {};
        \draw [] (start) to (d1);
        \draw [] (d1) to[out=235,in=90] (f_e);
        \draw [] (d1) to[out=305,in=90] (g);
        \draw (f_e) to (end1);
        \draw (g) to (end2);
      \end{tikzpicture}
      }
      \,\raisebox{45pt}{.}
      \]
      The second equality ($e\Delta(f\*g) = \Delta(f\*e g)$) follows by cocommutativity. The third
      equality,  ($e\Delta(f\*g) = \Delta(e f\*e g)$) follows by naturality of $\Delta$.

    \item[\ref{le:efginvdelta}] As in \ref{le:deltaefg}, details are only given for the
      first equality. This proof is obtained by reversing the diagrams of \ref{le:deltaefg}.
      \[
      \raisebox{10pt}{
      \begin{tikzpicture}
        \node at (0,0) (start) {};
        \node at (0,.5) [map] (e) {$\scriptstyle e$};
        \node at (0,1) [nabla] (d2) {};
        \node at (-.25,1.5) [map] (f) {$\scriptstyle f$};
        \node at (.25,1.5) [map] (g) {$\scriptstyle g$};
        \node at (-.25,2) (end1) {};
        \node at (.25,2) (end2) {};
        \draw (start) to (e);
        \draw (e) to (d2);
        \draw (d2) to[out=125,in=270] (f);
        \draw (d2) to[out=55,in=270] (g);
        \draw (f) to (end1);
        \draw (g) to (end2);
      \end{tikzpicture}
      }
      \ \raisebox{45pt}{$=$}\
        \begin{tikzpicture}
        \node at (0,0) (start) {};
        \node at (0,.5) [nabla] (d1) {};
        \node at (-.25,1) [map] (e1) {$\scriptstyle e$};
        \node at (0,1.5) [delta] (n1) {};
        \node at (0,2) [nabla] (d2) {};
        \node at (-.25,2.5) [map] (f) {$\scriptstyle f$};
        \node at (.25,2.5) [map] (g) {$\scriptstyle g$};
        \node at (-.25,3) (end1) {};
        \node at (.25,3) (end2) {};
        \draw [] (start) to (d1);
        \draw [] (d1) to[out=125,in=270] (e1);
        \draw [] (d1) to[out=55,in=305] (n1);
        \draw (e1) to[out=90,in=235] (n1);
        \draw [] (n1) to (d2);
        \draw (d2) to[out=125,in=270] (f);
        \draw (d2) to[out=55,in=270] (g);
        \draw (f) to (end1);
        \draw (g) to (end2);
      \end{tikzpicture}
      \ \raisebox{45pt}{$=$}\
        \begin{tikzpicture}
        \node at (0,0) (start) {};
        \node at (0,.5) [nabla] (d1) {};
        \node at (-.25,1) [map] (e1) {$\scriptstyle e$};
        \node at (0,1.75) [map] (inverse-delta-r) {$\scriptstyle \rst{\inv{\Delta}}$};
        \node at (-.25,2.5) [map] (f) {$\scriptstyle f$};
        \node at (.25,2.5) [map] (g) {$\scriptstyle g$};
        \node at (-.25,3) (end1) {};
        \node at (.25,3) (end2) {};
        \draw [] (start) to (d1);
        \draw [] (d1) to[out=125,in=270] (e1);
        \draw [] (d1) to[out=55,in=305] (inverse-delta-r);
        \draw (e1) to[out=90,in=235] (inverse-delta-r);
        \draw (inverse-delta-r) to[out=125,in=270] (f);
        \draw (inverse-delta-r) to[out=55,in=270] (g);
        \draw (f) to (end1);
        \draw (g) to (end2);
      \end{tikzpicture}
      \ \raisebox{45pt}{$=$}\
      \raisebox{20pt}{
        \begin{tikzpicture}
        \node at (0,0) (start) {};
        \node at (0,.5) [nabla] (d1) {};
        \node at (-.25,1) [map] (f e) {$\scriptstyle f e$};
        \node at (.25,1) [map] (g) {$\scriptstyle g$};
        \node at (-.25,1.5) (end1) {};
        \node at (.25,1.5) (end2) {};
        \draw [] (start) to (d1);
        \draw [] (d1) to[out=125,in=270] (f e);
        \draw [] (d1) to[out=55,in=270] (g);
        \draw (f e) to (end1);
        \draw (g) to (end2);
      \end{tikzpicture}
      }
      \,\raisebox{45pt}{.}
      \]
      The other equalities follow for the same reasons as in \ref{le:deltaefg}.

    \item[\ref{le:restfg}]Here, we start by using the fact all maps have a partial inverse,
      therefore we have:
      \[
        \restr{\Delta (f \* g) \inv{\Delta} } =\Delta (f \* g) \inv{\Delta} \Delta (\inv{f} \*
        \inv{g}) \inv{\Delta}.
     \]
     Now, we proceed with showing the rest of the equality via diagrams.

     \[
        \begin{tikzpicture}
        \node at (0,3.5) (start) {};
        \node at (0,3) [delta] (d1) {};
        \node at (-.25,2.5) [map] (f) {$\scriptstyle f$};
        \node at (.25,2.5) [map] (g) {$\scriptstyle g$};
        \node at (0, 2) [nabla] (n1) {};
        \node at (0,1.5) [delta] (d2) {};
        \node at (-.5,1) [map] (f-inverse) {$\scriptstyle \inv{f}$};
        \node at (.5,1) [map] (g-inverse) {$\scriptstyle \inv{g}$};
        \node at (0, .5) [nabla] (n2) {};
        \node at (0,0) (end) {};
        \draw [] (start) to (d1);
        \draw [] (d1) to[out=235,in=90] (f);
        \draw [] (d1) to[out=305,in=90] (g);
        \draw (f) to[out=270,in=125] (n1);
        \draw (g) to[out=270,in=55] (n1);
        \draw (n1) to (d2);
        \draw [] (d2) to[out=235,in=90] (f-inverse);
        \draw [] (d2) to[out=305,in=90] (g-inverse);
        \draw (f-inverse) to[out=270,in=125] (n2);
        \draw (g-inverse) to[out=270,in=55] (n2);
        \draw (n2) to (end);
      \end{tikzpicture}
      \ \raisebox{45pt}{$=$}\
      \begin{tikzpicture}
        \node at (0,3.5) (start) {};
        \node at (0,3) [delta] (d1) {};
        \node at (.25,2.5) [map] (f) {$\scriptstyle f$};
        \node at (-.25,2.5) [map] (g) {$\scriptstyle g$};
        \node at (0, 2) [nabla] (n1) {};
        \node at (0,1.5) [delta] (d2) {};
        \node at (.5,1) [map] (f-inverse) {$\scriptstyle \inv{f}$};
        \node at (-.5,1) [map] (g-inverse) {$\scriptstyle \inv{g}$};
        \node at (0, .5) [nabla] (n2) {};
        \node at (0,0) (end) {};
        \draw [] (start) to (d1);
        \draw [] (d1) to[out=235,in=90] (g);
        \draw [] (d1) to[out=305,in=90] (f);
        \draw (g) to[out=270,in=125] (n1);
        \draw (f) to[out=270,in=55] (n1);
        \draw (n1) to (d2);
        \draw [] (d2) to[out=235,in=90] (g-inverse);
        \draw [] (d2) to[out=305,in=90] (f-inverse);
        \draw (g-inverse) to[out=270,in=125] (n2);
        \draw (f-inverse) to[out=270,in=55] (n2);
        \draw (n2) to (end);
      \end{tikzpicture}
      \ \raisebox{45pt}{$=$}\
      \begin{tikzpicture}
        \node at (0,3.5) (start) {};
        \node at (0,3) [delta] (d1) {};
        \node at (.25,2.5) [map] (f) {$\scriptstyle f$};
        \node at (-.25,2.5) [map] (g) {$\scriptstyle g$};
        \node at (-.25,2) [delta] (d2) {};
        \node at (.25, 1.5) [nabla] (n1) {};
        \node at (.5,1) [map] (f-inverse) {$\scriptstyle \inv{f}$};
        \node at (-.5,1) [map] (g-inverse) {$\scriptstyle \inv{g}$};
        \node at (0, .5) [nabla] (n2) {};
        \node at (0,0) (end) {};
        \draw [] (start) to (d1);
        \draw [] (d1) to[out=235,in=90] (g);
        \draw [] (d1) to[out=305,in=90] (f);
        \draw (g) to (d2);
        \draw (d2) to[out=305,in=125] (n1);
        \draw (f) to[out=270,in=55] (n1);
        \draw (n1) to (f-inverse);
        \draw [] (d2) to[out=235,in=90] (g-inverse);
        \draw (g-inverse) to[out=270,in=125] (n2);
        \draw (f-inverse) to[out=270,in=55] (n2);
        \draw (n2) to (end);
      \end{tikzpicture}
      \ \raisebox{45pt}{$=$}\
      \begin{tikzpicture}
        \node at (0,3.5) (start) {};
        \node at (0,3) [delta] (d1) {};
        \node at (-.25,2.5) [delta] (d2) {};
        \node at (-.5,2) [map] (rest-g) {$\scriptstyle \rst{g}$};
        \node at (0,2) [map] (g) {$\scriptstyle g$};
        \node at (.75,1.5) [map] (rest-f) {$\scriptstyle \rst{f}$};
        \node at (0.1,1.5) [map] (f-inverse) {$\scriptstyle \inv{f}$};
        \node at (.25, 1) [nabla] (n1) {};
        \node at (0, .5) [nabla] (n2) {};
        \node at (0,0) (end) {};
        \draw [] (start) to (d1);
        \draw [] (d1) to[out=235,in=90] (d2);
        \draw [] (d1) to[out=305,in=90] (rest-f);
        \draw (d2) to[out=235,in=90] (rest-g);
        \draw (d2) to[out=305,in=90] (g);
        \draw (rest-g) to[out=270,in=125] (n2);
v        \draw (g) to (f-inverse);
        \draw (f-inverse) to[out=270,in=125] (n1);
        \draw (rest-f) to[out=270,in=55] (n1);
        \draw (n1) to[out=270,in=55] (n2);
        \draw (n2) to (end);
      \end{tikzpicture}
      \ \raisebox{45pt}{$=$}\
     \]
     \[
      \begin{tikzpicture}
        \node at (0,3.5) (start) {};
        \node at (0,3) [delta] (d1) {};
        \node at (-.35,2.5) [delta] (d2) {};
        \node at (0,1.75) [map] (g-f-inverse) {$\scriptstyle g \inv{f}$};
        \node at (.35, 1) [nabla] (n1) {};
        \node at (0, .5) [nabla] (n2) {};
        \node at (0,0) (end) {};
        \draw [] (start) to (d1);
        \draw [] (d1) to[out=235,in=90] (d2);
        \draw [] (d1) to[out=305,in=55] (n1);
        \draw (d2) to[out=235,in=125] (n2);
        \draw (d2) to[out=305,in=90] (g-f-inverse);
        \draw (g-f-inverse) to[out=270,in=125] (n1);
        \draw (n1) to[out=270,in=55] (n2);
        \draw (n2) to (end);
      \end{tikzpicture}
      \ \raisebox{45pt}{$=$}\
      \begin{tikzpicture}
        \node at (0,3.5) (start) {};
        \node at (0,3) [delta] (d1) {};
        \node at (-.35,2.5) [delta] (d2) {};
        \node at (.35, 2) [nabla] (n1) {};
        \node at (-.35,1.25) [map] (g-f-inverse) {$\scriptstyle g \inv{f}$};
        \node at (0, .5) [nabla] (n2) {};
        \node at (0,0) (end) {};
        \draw [] (start) to (d1);
        \draw [] (d1) to[out=235,in=90] (d2);
        \draw [] (d1) to[out=305,in=55] (n1);
        \draw (d2) to[out=305,in=125] (n1);
        \draw (d2) to[out=235,in=90] (g-f-inverse);
        \draw (g-f-inverse) to[out=270,in=125] (n2);
        \draw (n1) to[out=270,in=55] (n2);
        \draw (n2) to (end);
      \end{tikzpicture}
      \ \raisebox{45pt}{$=$}\
        \begin{tikzpicture}
        \node at (0,3) (start) {};
        \node at (0,2.5) [delta] (d1) {};
        \node at (0, 2) [nabla] (n1) {};
        \node at (0,1.5) [delta] (d2) {};
        \node at (-.5,1) [map] (g-f-inverse) {$\scriptstyle g\inv{f}$};
        \node at (0, .5) [nabla] (n2) {};
        \node at (0,0) (end) {};
        \draw [] (start) to (d1);
        \draw [] (d1) to[out=235,in=125] (n1);
        \draw [] (d1) to[out=305,in=55] (n1);
        \draw (n1) to (d2);
        \draw [] (d2) to[out=235,in=90] (g-f-inverse);
        \draw [] (d2) to[out=305,in=55] (n2);
        \draw (g-f-inverse) to[out=270,in=125] (n2);
        \draw (n2) to (end);
      \end{tikzpicture}
      \ \raisebox{45pt}{$=$}\
      \raisebox{15pt}{
       \begin{tikzpicture}
        \node at (0,2) (start) {};
        \node at (0,1.5) [delta] (d2) {};
        \node at (.5,1) [map] (g-f-inverse) {$\scriptstyle g\inv{f}$};
        \node at (0, .5) [nabla] (n2) {};
        \node at (0,0) (end) {};
        \draw [] (start) to (d2);
        \draw [] (d2) to[out=305,in=90] (g-f-inverse);
        \draw [] (d2) to[out=235,in=125] (n2);
        \draw (g-f-inverse) to[out=270,in=55] (n2);
        \draw (n2) to (end);
      \end{tikzpicture}
      }
      \,\raisebox{45pt}{.}
     \]
    \item[\ref{le:hge}]Beginning with the assumption that $\Delta (h \* g)\inv{\Delta}$ equals its
      restriction and by item \ref{le:restfg}, we have:
      \[
        \begin{tikzpicture}
        \node at (0,3.5) (start) {};
        \node at (0,3) [delta] (d1) {};
        \node at (-.25,2.5) [map] (h1) {$\scriptstyle h$};
        \node at (.25,2.5) [map] (g) {$\scriptstyle g$};
        \node at (0, 2) [nabla] (n1) {};
        \node at (0,1.5) [map] (h) {$\scriptstyle h$};
        \node at (0,1) (end) {};
        \draw [] (start) to (d1);
        \draw [] (d1) to[out=235,in=90] (h1);
        \draw [] (d1) to[out=305,in=90] (g);
        \draw (h1) to[out=270,in=125] (n1);
        \draw (g) to[out=270,in=55] (n1);
        \draw (n1) to (h);
        \draw (h) to (end);
      \end{tikzpicture}
      \ \raisebox{45pt}{$=$}\
        \begin{tikzpicture}
        \node at (0,3.5) (start) {};
        \node at (0,3) [delta] (d1) {};
        \node at (.5,2.5) [map] (g-h-inverse) {$\scriptstyle g\inv{h}$};
        \node at (0, 2) [nabla] (n1) {};
        \node at (0,1.5) [map] (h) {$\scriptstyle h$};
        \node at (0,1) (end) {};
        \draw [] (start) to (d1);
        \draw [] (d1) to[out=235,in=125] (n1);
        \draw [] (d1) to[out=305,in=90] (g-h-inverse);
        \draw (g-h-inverse) to[out=270,in=55] (n1);
        \draw (n1) to (h);
        \draw (h) to (end);
      \end{tikzpicture}
      \ \raisebox{45pt}{$=$}\
        \begin{tikzpicture}
        \node at (0,3.5) (start) {};
        \node at (0,3) [delta] (d1) {};
        \node at (.5,2.25) [map] (g-h-inverse) {$\scriptstyle g\inv{h}$};
        \node at (-.25,1.5) [map] (h1) {$\scriptstyle h$};
        \node at (.25,1.5) [map] (h2) {$\scriptstyle h$};
        \node at (0, 1) [nabla] (n1) {};
        \node at (0,.5) (end) {};
        \draw [] (start) to (d1);
        \draw [] (d1) to[out=235,in=90] (h1);
        \draw [] (d1) to[out=305,in=90] (g-h-inverse);
        \draw (g-h-inverse) to[out=270,in=90] (h2);
        \draw (h1) to[out=270,in=125] (n1);
        \draw (h2) to[out=270,in=55] (n1);
        \draw (n1) to (end);
      \end{tikzpicture}
      \ \raisebox{45pt}{$=$}\
        \begin{tikzpicture}
        \node at (0,3.5) (start) {};
        \node at (0,3) [delta] (d1) {};
        \node at (.5,2.25) [map] (g-h-inverse) {$\scriptstyle g\rst{\inv{h}}$};
        \node at (-.25,1.5) [map] (h1) {$\scriptstyle h$};
        \node at (0, 1) [nabla] (n1) {};
        \node at (0,.5) (end) {};
        \draw [] (start) to (d1);
        \draw [] (d1) to[out=235,in=90] (h1);
        \draw [] (d1) to[out=305,in=90] (g-h-inverse);
        \draw (h1) to[out=270,in=125] (n1);
        \draw (g-h-inverse) to[out=270,in=55] (n1);
        \draw (n1) to (end);
      \end{tikzpicture}
      \ \raisebox{45pt}{$=$}\
        \begin{tikzpicture}
        \node at (0,3.5) (start) {};
        \node at (0,3) [delta] (d1) {};
        \node at (.25,2.25) [map] (g) {$\scriptstyle g$};
        \node at (-.5,1.5) [map] (h1) {$\scriptstyle h\rst{\inv{h}}$};
        \node at (0, 1) [nabla] (n1) {};
        \node at (0,.5) (end) {};
        \draw [] (start) to (d1);
        \draw [] (d1) to[out=235,in=90] (h1);
        \draw [] (d1) to[out=305,in=90] (g);
        \draw (h1) to[out=270,in=125] (n1);
        \draw (g) to[out=270,in=55] (n1);
        \draw (n1) to (end);
      \end{tikzpicture}
      \ \raisebox{45pt}{$=$}\
      \raisebox{15pt}{
        \begin{tikzpicture}
        \node at (0,3.5) (start) {};
        \node at (0,3) [delta] (d1) {};
        \node at (.25,2.5) [map] (g) {$\scriptstyle g$};
        \node at (-.25,2.5) [map] (h1) {$\scriptstyle h$};
        \node at (0, 2) [nabla] (n1) {};
        \node at (0,1.5) (end) {};
        \draw [] (start) to (d1);
        \draw [] (d1) to[out=235,in=90] (h1);
        \draw [] (d1) to[out=305,in=90] (g);
        \draw (h1) to[out=270,in=125] (n1);
        \draw (g) to[out=270,in=55] (n1);
        \draw (n1) to (end);
      \end{tikzpicture}
      }
      \ \raisebox{45pt}{$=$}\
      \raisebox{15pt}{
        \begin{tikzpicture}
        \node at (0,3.5) (start) {};
        \node at (0,3) [delta] (d1) {};
        \node at (-.25,2.5) [map] (g) {$\scriptstyle g$};
        \node at (.25,2.5) [map] (h1) {$\scriptstyle h$};
        \node at (0, 2) [nabla] (n1) {};
        \node at (0,1.5) (end) {};
        \draw [] (start) to (d1);
        \draw [] (d1) to[out=305,in=90] (h1);
        \draw [] (d1) to[out=235,in=90] (g);
        \draw (h1) to[out=270,in=55] (n1);
        \draw (g) to[out=270,in=125] (n1);
        \draw (n1) to (end);
      \end{tikzpicture}
      }
      \,\raisebox{45pt}{.}
      \]

    \item[\ref{le:dfgisfg}] Our assumption is that:
      \[
        \begin{tikzpicture}
        \node at (0,1.5) (start) {};
        \node at (0,1) [delta] (d1) {};
        \node at (-.25,.5) [map] (g) {$\scriptstyle g$};
        \node at (-.25, 0) (end1) {};
        \node at (.25,0) (end2) {};
        \draw [] (start) to (d1);
        \draw [] (d1) to[out=305,in=90] (end2);
        \draw [] (d1) to[out=235,in=90] (g);
        \draw (g) to (end1);
      \end{tikzpicture}
      \ \raisebox{20pt}{$=$}\
        \begin{tikzpicture}
        \node at (0,1.5) (start) {};
        \node at (0,1) [delta] (d1) {};
        \node at (-.25,.5) [map] (f) {$\scriptstyle f$};
        \node at (-.25, 0) (end1) {};
        \node at (.25,0) (end2) {};
        \draw [] (start) to (d1);
        \draw [] (d1) to[out=305,in=90] (end2);
        \draw [] (d1) to[out=235,in=90] (f);
        \draw (f) to (end1);
      \end{tikzpicture}
      \raisebox{20pt}{ and by cocommutativity, }
        \begin{tikzpicture}
        \node at (0,1.5) (start) {};
        \node at (0,1) [delta] (d1) {};
        \node at (.25,.5) [map] (g) {$\scriptstyle g$};
        \node at (.25, 0) (end1) {};
        \node at (-.25,0) (end2) {};
        \draw [] (start) to (d1);
        \draw [] (d1) to[out=235,in=90] (end2);
        \draw [] (d1) to[out=305,in=90] (g);
        \draw (g) to (end1);
      \end{tikzpicture}
      \ \raisebox{20pt}{$=$}\
        \begin{tikzpicture}
        \node at (0,1.5) (start) {};
        \node at (0,1) [delta] (d1) {};
        \node at (.25,.5) [map] (f) {$\scriptstyle f$};
        \node at (.25, 0) (end1) {};
        \node at (-.25,0) (end2) {};
        \draw [] (start) to (d1);
        \draw [] (d1) to[out=235,in=90] (end2);
        \draw [] (d1) to[out=305,in=90] (f);
        \draw (f) to (end1);
      \end{tikzpicture}
      \,\raisebox{20pt}{.}
      \]
      Hence,
      \[
      \ \raisebox{45pt}{$f=$}\
      \raisebox{15pt}{
        \begin{tikzpicture}
        \node at (0,3.5) (start) {};
        \node at (0,3) [delta] (d1) {};
        \node at (-.25,2.5) [map] (f1) {$\scriptstyle f$};
        \node at (.25,2.5) [map] (f2) {$\scriptstyle f$};
        \node at (0, 2) [nabla] (n1) {};
        \node at (0,1.5) (end) {};
        \draw [] (start) to (d1);
        \draw [] (d1) to[out=305,in=90] (f2);
        \draw [] (d1) to[out=235,in=90] (f1);
        \draw (f2) to[out=270,in=55] (n1);
        \draw (f1) to[out=270,in=125] (n1);
        \draw (n1) to (end);
      \end{tikzpicture}
      }
      \ \raisebox{45pt}{$=$}\
        \begin{tikzpicture}
        \node at (0,3.5) (start) {};
        \node at (0,3) [delta] (d1) {};
        \node at (-.25,2.5) [map] (f1) {$\scriptstyle f$};
        \node at (.25,2) [map] (f2) {$\scriptstyle f$};
        \node at (0, 1.5) [nabla] (n1) {};
        \node at (0,1) (end) {};
        \draw [] (start) to (d1);
        \draw [] (d1) to[out=305,in=90] (f2);
        \draw [] (d1) to[out=235,in=90] (f1);
        \draw (f2) to[out=270,in=55] (n1);
        \draw (f1) to[out=270,in=125] (n1);
        \draw (n1) to (end);
      \end{tikzpicture}
      \ \raisebox{45pt}{$=$}\
        \begin{tikzpicture}
        \node at (0,3.5) (start) {};
        \node at (0,3) [delta] (d1) {};
        \node at (-.25,2.5) [map] (g1) {$\scriptstyle g$};
        \node at (.25,2) [map] (f2) {$\scriptstyle f$};
        \node at (0, 1.5) [nabla] (n1) {};
        \node at (0,1) (end) {};
        \draw [] (start) to (d1);
        \draw [] (d1) to[out=305,in=90] (f2);
        \draw [] (d1) to[out=235,in=90] (g1);
        \draw (f2) to[out=270,in=55] (n1);
        \draw (g1) to[out=270,in=125] (n1);
        \draw (n1) to (end);
      \end{tikzpicture}
      \ \raisebox{45pt}{$=$}\
        \begin{tikzpicture}
        \node at (0,3.5) (start) {};
        \node at (0,3) [delta] (d1) {};
        \node at (-.25,2) [map] (g1) {$\scriptstyle g$};
        \node at (.25,2.5) [map] (f2) {$\scriptstyle f$};
        \node at (0, 1.5) [nabla] (n1) {};
        \node at (0,1) (end) {};
        \draw [] (start) to (d1);
        \draw [] (d1) to[out=305,in=90] (f2);
        \draw [] (d1) to[out=235,in=90] (g1);
        \draw (f2) to[out=270,in=55] (n1);
        \draw (g1) to[out=270,in=125] (n1);
        \draw (n1) to (end);
      \end{tikzpicture}
      \ \raisebox{45pt}{$=$}\
        \begin{tikzpicture}
        \node at (0,3.5) (start) {};
        \node at (0,3) [delta] (d1) {};
        \node at (-.25,2) [map] (g1) {$\scriptstyle g$};
        \node at (.25,2.5) [map] (g2) {$\scriptstyle g$};
        \node at (0, 1.5) [nabla] (n1) {};
        \node at (0,1) (end) {};
        \draw [] (start) to (d1);
        \draw [] (d1) to[out=305,in=90] (g2);
        \draw [] (d1) to[out=235,in=90] (g1);
        \draw (g2) to[out=270,in=55] (n1);
        \draw (g1) to[out=270,in=125] (n1);
        \draw (n1) to (end);
      \end{tikzpicture}
      \ \raisebox{45pt}{$= g.$}\
      \]
    \item[\ref{le:fgisfg}] Use the same diagrammatic argument as in item \ref{le:dfgisfg}.
  \end{enumerate}
\end{proof}

\begin{proposition}\label{prop:discrete_inverse_category_has_meets}
  A discrete inverse category has meets, where $f\meet g =\Delta (f\* g) \inv{\Delta}$.
\end{proposition}
\begin{proof}
  $f\meet g \le f$:
  \[
  \raisebox{45pt}{$f\meet g =$}
      \raisebox{15pt}{
        \begin{tikzpicture}
        \node at (0,3.5) (start) {};
        \node at (0,3) [delta] (d1) {};
        \node at (-.25,2.5) [map] (f) {$\scriptstyle f$};
        \node at (.25,2.5) [map] (g) {$\scriptstyle g$};
        \node at (0, 2) [nabla] (n1) {};
        \node at (0,1.5) (end) {};
        \draw [] (start) to (d1);
        \draw [] (d1) to[out=305,in=90] (g);
        \draw [] (d1) to[out=235,in=90] (f);
        \draw (g) to[out=270,in=55] (n1);
        \draw (f) to[out=270,in=125] (n1);
        \draw (n1) to (end);
      \end{tikzpicture}
      }
      \raisebox{45pt}{$=$}
      \raisebox{15pt}{
        \begin{tikzpicture}
        \node at (0,3.5) (start) {};
        \node at (0,3) [delta] (d1) {};
        \node at (-.5,2.5) [map] (f) {$\scriptstyle f\rst{\inv{f}}$};
        \node at (.25,2.5) [map] (g) {$\scriptstyle g$};
        \node at (0, 2) [nabla] (n1) {};
        \node at (0,1.5) (end) {};
        \draw [] (start) to (d1);
        \draw [] (d1) to[out=305,in=90] (g);
        \draw [] (d1) to[out=235,in=90] (f);
        \draw (g) to[out=270,in=55] (n1);
        \draw (f) to[out=270,in=125] (n1);
        \draw (n1) to (end);
      \end{tikzpicture}
      }
      \raisebox{45pt}{$=$}
      \raisebox{15pt}{
        \begin{tikzpicture}
        \node at (0,3.5) (start) {};
        \node at (0,3) [delta] (d1) {};
        \node at (-.25,2.5) [map] (f) {$\scriptstyle f$};
        \node at (.5,2.5) [map] (g) {$\scriptstyle g\rst{\inv{f}}$};
        \node at (0, 2) [nabla] (n1) {};
        \node at (0,1.5) (end) {};
        \draw [] (start) to (d1);
        \draw [] (d1) to[out=305,in=90] (g);
        \draw [] (d1) to[out=235,in=90] (f);
        \draw (g) to[out=270,in=55] (n1);
        \draw (f) to[out=270,in=125] (n1);
        \draw (n1) to (end);
      \end{tikzpicture}
      }
      \raisebox{45pt}{$=$}
        \begin{tikzpicture}
        \node at (0,3.5) (start) {};
        \node at (0,3) [delta] (d1) {};
        \node at (-.25,2) [map] (f) {$\scriptstyle f$};
        \node at (.5,2.5) [map] (g) {$\scriptstyle g\inv{f}$};
        \node at (.25,2) [map] (f2) {$\scriptstyle f$};
        \node at (0, 1.5) [nabla] (n1) {};
        \node at (0,1) (end) {};
        \draw [] (start) to (d1);
        \draw [] (d1) to[out=305,in=90] (g);
        \draw [] (d1) to[out=235,in=90] (f);
        \draw (g) to (f2);
        \draw (f2) to[out=270,in=55] (n1);
        \draw (f) to[out=270,in=125] (n1);
        \draw (n1) to (end);
      \end{tikzpicture}
      \ \raisebox{45pt}{$=$}
        \begin{tikzpicture}
        \node at (0,3.5) (start) {};
        \node at (0,3) [delta] (d1) {};
        \node at (.5,2.5) [map] (g) {$\scriptstyle g\inv{f}$};
        \node at (0, 2) [nabla] (n1) {};
        \node at (0,1.5) [map] (f) {$\scriptstyle f$};
        \node at (0,1) (end) {};
        \draw [] (start) to (d1);
        \draw [] (d1) to[out=305,in=90] (g);
        \draw [] (d1) to[out=235,in=125] (n1);
        \draw (g) to[out=270,in=55] (n1);
        \draw (n1) to (f);
        \draw (f) to (end);
      \end{tikzpicture}
      \ \raisebox{45pt}{$= \rst{f\meet g}f.$}
  \]

  $f\meet f = f$:
  \begin{equation*}
    f\meet f = \Delta(f\* f) \inv{\Delta} =f \Delta \inv{\Delta} = f\Delta.
  \end{equation*}

  $h(f\meet g) = h f \meet h g$:
  \begin{align*}
    h(f\meet g) &= h \Delta(f\* g) \inv{\Delta}& \text{Definition of }\meet\\
    &= \Delta(h \* h) (f \* g) \inv{\Delta} &\Delta\text{ natural}\\
    &= \Delta(h f\* h g) \inv{\Delta} &\text{compose maps}\\
    &= h f \meet h g&\text{Definition of }\meet.
  \end{align*}
\end{proof}

% subsection discrete_inverse_categories (end)


\subsection{The inverse subcategory of a discrete restriction category } % (fold)
\label{sub:the_inverse_subcategory_of_a_discrete_restriction_category}

Given a discrete restriction category, one can pick out the maps which are partial isomorphisms.
Using results from Sub-Section~\ref{sub:discrete_inverse_categories} and from
Sub-Section~\ref{sub:discrete_restriction_categories}, we will show that these maps form a
subcategory which is a discrete inverse category.

\begin{proposition}\label{lem:inv_x_is_a_discrete_inverse_category}
  Given \X is a discrete restriction category, the partial isomorphisms of \X, together with the objects
  of \X form a sub-restriction category which is a discrete inverse category. For the restriction
  category $\X$, we denote this sub-category by \Invc{\X}.
\end{proposition}
\begin{proof}
  As shown in Lemma~\ref{lem:rcs_partial_monic_section_inverse_properties}, partial isomorphisms
  are closed under composition. The identity maps are in \Invc{\X} and restrictions of
  partial isomorphisms are also partial isomorphisms.

  The product on the discrete restriction category \X becomes the tensor product of the restriction
  category \Invc{\X}. Table~\ref{tab:structural_maps_for_the_tensor_in_invx} shows how each of the
  elements of the tensor are defined. Note that the last definition makes explicit use of the fact
  we are in a discrete restriction category and hence the $\Delta$ of \X possesses a partial
  inverse.

  \begin{table}[!htbp]
    \begin{center}
      \begin{tabular}{|ccc|}
        \hline
        \X & \Invc{\X} & Inverse map\\
        \hline\hline
        $\scriptstyle A\times B$ & $\scriptstyle A\* B$ &\\
        \hline
        $\scriptstyle \top$ & $\scriptstyle 1$ &\\
        \hline
        $\scriptstyle \pi_1:\top\times A \to A$ & $\scriptstyle \usl:1\* A \to A$ & $\scriptstyle \<!,1\>$\\
        \hline
        $\scriptstyle \pi_0:A\times\top \to A$ & $\scriptstyle \usr:A\*1 \to A$& $\scriptstyle \<1,!\>$\\
        \hline
        ${\scriptstyle a_{\X} = \<\pi_0 \pi_0,\<\pi_0 \pi_1,\pi_1\>\>:(A\times B)\times C \to A\times(B\times C)}$
          & $\scriptstyle a_{\*}:(A\*B)\*C \to A\*(B\*C)$
          & $\scriptstyle \<\<\pi_0, \pi_1 \pi_0\>,\pi_1 \pi_1\>$\\
        \hline
        $\scriptstyle c_{\X}=\< \pi_1,\pi_0\>:A\times B \to B\times A$ & $\scriptstyle c_{\*}:A\*B \to B \* A$ & $\scriptstyle \< \pi_1,\pi_0\>$\\
        \hline
        $\scriptstyle \Delta_{\X}:A\to A\times A$ & $\scriptstyle \Delta:A \to A\* A$ & $\scriptstyle  \inv{\Delta_{\X}} $\\
        \hline
      \end{tabular}

    \end{center}
    \caption{Structural maps for the tensor in \Invc{\X}}
    \label{tab:structural_maps_for_the_tensor_in_invx}
  \end{table}

  The monoid coherence diagrams follow directly from the characteristics of the product in
  \X. Similarly, $\Delta$ is total as it is total in $\X$. It remains to show cocommutativity,
  coassociativity and the Frobenius condition.

  Cocommutativity requires $\Delta c_{\*} = c_{\*}$. We have
  \[
     \Delta_{\X} \< \pi_1,\pi_0\> = \<\Delta_{\X}\pi_1, \Delta_{\X}\pi_0\> = \<1,1\> =
     \Delta_{\X},
  \]
  giving us the required cocommutativity.

  Coassociativity requires $\Delta (1 \* \Delta) = \Delta (\Delta \* 1) a_{\*}$. Expressing this
  in \X, it is the requirement that
  \[
    \Delta_{\X} (1 \times \Delta_{\X}) =
      \Delta_{\X}(\Delta_{\X} \times 1) a_{\X}.
  \]
  Recalling that $f\times g \pi_0 = \pi_0 f$ and $f\times g \pi_1 = \pi_1 g$, we have:
  \begin{align*}
    \Delta_{\X}(\Delta_{\X} \times 1) a_{\X} &=\Delta_{\X}(\Delta_{\X} \times 1)
    \<\pi_0\pi_0,\<\pi_0\pi_1,\pi_1\>\>\\
    &= \<\Delta_{\X}(\Delta_{\X} \times 1)\pi_0\pi_0,\<\Delta_{\X}(\Delta_{\X} \times 1)\pi_0\pi_1,\Delta_{\X}(\Delta_{\X} \times 1)\pi_1\>\>\\
    &= \<\Delta_{\X}\pi_0\Delta_{\X}\pi_0,\<\Delta_{\X}\pi_0\Delta_{\X}\pi_1,\Delta_{\X}\pi_1 1\>\>\\
    &= \<1, \<1,1\>\> = \Delta_{\X}(1\times\Delta_{\X})
  \end{align*}
  and shows that we have coassociativity.

  The semi-Frobenius requirement is two-fold:
  \begin{align}
    \inv{\Delta} \Delta &= (\Delta \*1) a_{\*}(1\*\inv{\Delta}) \label{eq:frobenius_righths_need_in_invx}\\
    \inv{\Delta} \Delta &= (1 \* \Delta) \inv{a_{\*}}(\inv{\Delta}\* 1), \label{eq:frobenius_lefths_need_in_invx}.
  \end{align}
  In \X, these become:
  \begin{align}
    \inv{\Delta_{\X}} \Delta_{\X}
      &= (\Delta_{\X} \times 1) \<\pi_0 \pi_0,\<\pi_0 \pi_1,\pi_1\>\>(1\times\inv{\Delta_{\X}})
      \label{eq:frobenius_righths_expressed_in_x}\\
    \inv{\Delta_{\X}}\Delta_{\X}
      &= (1 \times \Delta_{\X}) \<\<\pi_0, \pi_1 \pi_0\>,\pi_1 \pi_1\>(\inv{\Delta_{\X}}\times 1).
      \label{eq:frobenius_lefths_expressed_in_x}
  \end{align}
  We will give the details of the proof for Equation~\ref{eq:frobenius_righths_expressed_in_x}.
  Equation~\ref{eq:frobenius_lefths_expressed_in_x} may be proven similarly.

  Note first that $\Delta(1 \times !)$ (and $\Delta(!\times 1)$) is the identity. Second, we see
  that maps to a product of objects may be expressed as a pairing --- i.e.  if
  $f:A \to B \times B$, then $f = \<f(1\times !), f(!\times 1)\>$.

  Using this we see that the left hand side of Equation~\ref{eq:frobenius_righths_expressed_in_x}
  may be computed as follows:
  \begin{align*}
    \inv{\Delta_{\X}} \Delta_{\X}
      & = \<\inv{\Delta_{\X}} \Delta_{\X}(1\times !), \inv{\Delta_{\X}} \Delta_{\X} (! \times 1)\>\\
    &= \<\inv{\Delta_{\X}}, \inv{\Delta_{\X}} \>
  \end{align*}
  Similarly, removing the associativity maps, the right hand side of the same equation becomes:
  \begin{align*}
    (\Delta_{\X} \times 1) (1\times\inv{\Delta_{\X}}) &
      = \<(\Delta_{\X} \times 1) (1\times\inv{\Delta_{\X}}) (1\times !),
      (\Delta_{\X} \times 1) (1\times\inv{\Delta_{\X}}) (! \times 1 )\> \\
    &= \<(\Delta_{\X} \times 1) (1\times\inv{\Delta_{\X}}) (1\times ! ), \inv{\Delta_{\X}}\> \\
    &= \<(\Delta_{\X} \times 1) (1\times\inv{\Delta_{\X}}) (1 \times \Delta_{\X})(1\times !\times !), \inv{\Delta_{\X}}\> \\
    &= \<(\Delta_{\X} \times 1) (1\times\rst{\inv{\Delta_{\X}}}) (1\times !\times !), \inv{\Delta_{\X}}\> \\
    &= \<(\Delta_{\X} \times 1) \rst{1\times\inv{\Delta_{\X}}} (1\times !\times !), \inv{\Delta_{\X}}\> \\
    &= \<\rst{(\Delta_{\X} \times 1) (1\times\inv{\Delta_{\X}})}
      (\Delta_{\X} \times 1)(1\times !\times !), \inv{\Delta_{\X}}\> \\ %rfour
    &= \<\rst{(\Delta_{\X} \times 1) (1\times\inv{\Delta_{\X}})} (1\times !), \inv{\Delta_{\X}}\> \\
      &= \<\rst{(\Delta_{\X} \times 1) (1\times\inv{\Delta_{\X}})(! \times 1)} (1\times !),
      \inv{\Delta_{\X}}\> \\ % add total to right of rst
    &= \<\rst{\inv{\Delta_{\X}}} (1\times !), \inv{\Delta_{\X}}\> \\
    &= \<\inv{\Delta_{\X}}\Delta_{\X}(1\times !), \inv{\Delta_{\X}}\> \\
    &= \<\inv{\Delta_{\X}}, \inv{\Delta_{\X}}\>
  \end{align*}
  and therefore we see that the first equation for the Frobenius condition is satisfied. Thus,
  $Inv(\X)$ is a discrete inverse category.
\end{proof}
% subsection the_inverse_subcategory_of_a_graphic_cartesian_restriction_category (end)

% section inverse_products (end)


\section{The category of Commutative Frobenius Algebras} % (fold)
\label{sec:the_category_of_commutative_frobenius_algebras}
Dagger categories generalize the category of Hilbert spaces which is often used to model quantum
computation. These were introduced in \cite{abramsky04:catsemquantprot} as \emph{strongly compact
closed categories}, an additional structure on compact closed categories.

Before introducing dagger categories, we define compact closed
categories.



\begin{definition}\label{def:compactclosedcat}
A \emph{compact closed category} \cD{} is a symmetric monoidal category with tensor $\*$ where each
object $A$ has a dual $A^{*}$. Additionally, there must exist families of maps $\eta_{A}: I \to
A^{*} \* A$ (the \emph{unit}) and $\epsilon_{A}: A\*A^{*}\to I$ (the \emph{counit}) such that
\[
  \xymatrix@C+15pt{
    A \ar[r]^{u_{A}} \ar@{=}[d]  & A\*I \ar[r]^(.4){1\*\eta_{A}}
        & A\* (A^{*}\*A) \ar[d]^{a_{A,A^{*},A}} \\
    A & I\* A \ar[l]^{u_{A}^{-1}} & (A\* A^{*})\*A \ar[l]^(.6){\epsilon_{A}\*1}
    }\text{ and }
  \xymatrix@C+15pt{
    A^{*} \ar[r]^{u_{A^*}} \ar@{=}[d]  & I\*A^* \ar[r]^(.4){\eta_{A}\*1}
        & (A^{*}\* A)\*A^{*} \ar[d]^{a_{A^{*},A,A^{*}}^{-1}} \\
    A^* & A^*\*I \ar[l]^{u_{A^*}^{-1}} & A^{*}\*(A\*A^{*}) \ar[l]^(.6){1\*\epsilon_{A}}
    }
  \]
commute.
\end{definition}

Given a map $f:A\to B$ in a compact closed category,  define the map $f^{*}:B^{*} \to A^{*}$ as
\[
  \xymatrix@C+10pt{
    B^{*}\ar[r]^{u_{B^{*}}} \ar[d]_{f^{*}}& I\*B^{*} \ar[r]^{\eta_{A}\*1}
      & A^{*}\*A\*B^{*}\ar[d]^{1\*f\*1}\\
    A^{*}&    A^{*}\*I\ar[l]^{u_{A^{*}}^{-1}}  &   A^{*}\*B\*B^{*}\ar[l]^{1\*\epsilon_{B}}.
  }
\]


%!TEX root = /Users/gilesb/UofC/thesis/phd-thesis/phd-thesis.tex
\subsection{Dagger categories}\label{ssec:daggercategories}

Although dagger categories were introduced in the context of compact closed categories, the concept
of a dagger is definable independently. This was first done in \cite{selinger05:dagger}.

\begin{definition}\label{def:daggercat}
  A \emph{dagger} on a category $D$ is a functor $\dagger:\dual{\cD}\to \cD$, which is  involutive,
  that is, $\dgr{\dgr{f}} = f$ and which is the identity on objects. A \emph{dagger category} is a
  category that has a dagger.
\end{definition}

Typically, the dagger is written as a superscript on the morphism. So, if $f:A\to B$ is a map in
\cD, then $\dgr{f}:B\to A$ is a map in \cD{} and is called the \emph{adjoint} of $f$. A map where
$f^{-1} = \dgr{f}$ is called \emph{unitary}. A map $f:A\to A$ with $f=\dgr{f}$ is called
\emph{self-adjoint} or \emph{Hermitian}.

\begin{definition}\label{def:daggersmc}
  A \emph{dagger symmetric monoidal category} is a symmetric monoidal category \cD{} with a dagger
  operator such that:
  \begin{enumerate}[{(}i{)}]
    \item For all maps $f:A\to B$ and $g:C\to D$, $\dgr{(f\*g)} = \dgr{f}\*\dgr{g}:B\*D \to A\* C$;\label{defitem:dagger_smc_one}
    \item The monoid structure isomorphisms $a_{A,B,C}:(A\*B)\* C\to A\*(B\*C)$, $u^l_{A}:I\*A\to
      A$, $u^r_{A}:A\*I \to A$ and  $c_{A,B}:A\*B \to B\*A$ are unitary.\label{defitem:dagger_smc_two}
  \end{enumerate}
\end{definition}


\begin{definition}\label{def:daggercompact}
  A \emph{dagger compact closed category} \cD{} is a dagger symmetric monoidal category
  that is compact closed where the diagram
  \[
    \xymatrix @C+20pt @R+10pt{
      I \ar[r]^{\epsilon^{\dagger}_{A}} \ar[dr]_{\eta_{A}} &A\*A^{*}\ar[d]^{c_{A,A^{*}}}\\
      &A^{*}\* A
    }
  \]
  commutes for all  objects $A$ in \cD.
\end{definition}

\begin{lemma}\label{lemma:daggerbiproducts}
If \cD{} is a dagger category with biproducts, with injections $in_{1},in_{2}$ and projections
$p_{1},p_{2}$, then the following are equivalent:
\begin{enumerate}[{(}i{)}]
  \item $\dgr{p_{i}} = in_{i}, i=1,2$, \label{ldpdgrpisq}
  \item $\dgr{(f\biproduct g)} = \dgr{f}\biproduct \dgr{g}$ and $\dgr{\Delta} = \nabla$,\label{ldpddeltisnab}
  \item $\dgr{\<f,g\>} = [\dgr{f},\dgr{g}]$,\label{ldpdcopisprod}
  \item The map $[\dgr{p_{1}},\dgr{p_{2}}]: \dgr{A} \biproduct \dgr{B} \to \dgr{(A\biproduct B)}$ is
    the identity map.\label{ldpcommute}
%the below diagram commutes:
%  \[
%    \xymatrix @C+20pt @R+10pt{
%      \dgr{A} \biproduct \dgr{B} \ar[d]_{id} \ar[dr]^{[\dgr{p_{1}},\dgr{p_{2}}]}\\
%      A\biproduct B\ar[r]_{id}&\dgr{(A\biproduct B)}.
%    }
%  \]
\end{enumerate}
\end{lemma}
\begin{proof}
  \begin{description}
    \item[\ref{ldpdgrpisq}$\implies$\ref{ldpddeltisnab}] To show $\dgr{\Delta} = \nabla$,
    draw the product cone for $\Delta$,
    \[
      \xymatrix {
        &A \ar[d]^{\Delta} \ar[dr]^{id} \ar[dl]_{id}\\
        A
         & A\biproduct A \ar[l]^{p_{1}}  \ar[r]_{p_{2}}
         & A
      }
    \]
    and apply the dagger functor to it. As $\dgr{p_{i}} = in_{i}$, and $\dagger$ is identity on
    objects, this is now a coproduct diagram and therefore $\dgr{\Delta} = \nabla$.

    For $\dgr{(f\biproduct g)} = \dgr{f}\biproduct\dgr{g}$, start with the diagram defining
    $f\biproduct g$ as a product of the arrows:
    \[
      \xymatrix{
        A\ar[d]_{f}  & A\biproduct B \ar[l]_{p_{1}} \ar[r]^{p_{2}} \ar[d]^{f\biproduct g}&A \ar[d]^{g}\\
        C & C\biproduct D \ar[l]^{p_{1}} \ar[r]_{p_{2}}  & D.
      }
    \]
    Then, apply the dagger functor to this diagram. This is now the diagram defining the
    coproduct of maps and therefore $\dgr{(f\biproduct g)} = \dgr{f}\biproduct\dgr{g}$.
    \item[\ref{ldpddeltisnab}$\implies$\ref{ldpdcopisprod}] The calculation showing this is
      \begin{eqnarray*}
        &[\dgr{f},\dgr{g}] & = \nabla; (\dgr{f}\biproduct \dgr{g})\\
        & &=\dgr{\Delta}; (\dgr{f}\biproduct \dgr{g})\\
        & &=\dgr{\Delta}; \dgr{(f\biproduct g)}\\
        & & = \dgr{((f\biproduct g);\Delta)}\\
        & & = \dgr{\<f,g\>}.
      \end{eqnarray*}
    \item[\ref{ldpdcopisprod}$\implies$\ref{ldpcommute}]
      Under the assumption,
      \[
        [\dgr{p_{1}},\dgr{p_{2}}] = \dgr{\<p_{1},p_{2}\>}=\dgr{id}=id.
      \]
    \item[\ref{ldpcommute}$\implies$\ref{ldpdgrpisq}] As $[in_{1},in_{2}]:\dgr{A} \biproduct \dgr{B}
      \to \dgr{A} \biproduct \dgr{B} = id = [\dgr{p_{1}},\dgr{p_{2}}]$, we immediately have
      $\dgr{p_{1}} = in_{1}$ and $\dgr{p_{2}} = in_{2}$.
%
%Using the injections and under
%    the assumption, the following diagram commutes:
%      \[
%        \xymatrix @C+20pt @R+10pt{
%          \dgr{A} \biproduct \dgr{B} \ar[d]_{id} \ar[dr]^{[\dgr{p_{1}},\dgr{p_{2}}]}\ar[r]^{[in_{1},in_{2}]}
%            & \dgr{A} \biproduct \dgr{B} \ar[d]^{id}\\
%          A\biproduct B\ar[r]_{id}&\dgr{(A\biproduct B)}
%        }
%      \]
%      and therefore,
  \end{description}
\end{proof}

\begin{definition} \label{def:biproductdaggerccc}
  A \emph{biproduct dagger compact closed category} is a dagger compact closed category with
  biproducts where the conditions of lemma \ref{lemma:daggerbiproducts} hold.
\end{definition}
\subsection{Examples of dagger categories}

\begin{example}[\fdh]\label{ex:fdhilbert_is_dagger_category}
The category of finite dimensional Hilbert spaces is the motivating example for
the creation of the dagger and is, in fact, a biproduct dagger compact closed category. The
biproduct is the direct sum of Hilbert spaces and the tensor for compact closure is the standard
tensor of Hilbert spaces. The dual $H^{*}$ of a space $H$ is the space of all continuous linear
functions from $H$ to the base field. The dagger is defined via the adjoint as being the unique map
$\dgr{f}:B\to A$ such that $\<f a|b\> = \<a | \dgr{f} b\>$ for all $a\in A, b\in B$.
\end{example}

\begin{example}[\rel]\label{ex:rel_is_dagger_category}
The category \rel of sets and relations has the tensor $S\*T \definedas S\times T$ and the biproduct
$S\biproduct T \definedas S\disjointunion T$. This is compact closed under $A^{*} \definedas A$ and
the dagger is the relational converse. That is, if the relation
$R=\{(s,t)|s\in S, t\in T\}:S\to T$, then $\dgr{R}=R^*=\{(t,s)|(s,t)\in R\}$.
\end{example}

\begin{example}[Inverse categories]\label{ex:inverse_category_is_dagger_category}
An inverse category \X is also a dagger category when the dagger is defined as the partial inverse.
The unitary maps are the total maps. When the inverse category \X is also a
symmetric monoidal category where the monoid $\*$ is actually a restriction bi-functor, then \X is
a dagger symmetric monoidal category.

Requirement \ref{defitem:dagger_smc_one} of Definition~\ref{def:daggersmc}  is fulfilled, as
\[
  (f\*g) \inv{(f\*g)} = \rst{f\*g}=\rst{f} \*\rst{g} =
   f\inv{f} \* g \inv{g} = (f\*g) (\inv{f} \* \inv{g})
\]
and since the partial inverse of $f\*g$ is unique, $\inv{(f\*g)} = \inv{f} \* \inv{g}$.
Requirement \ref{defitem:dagger_smc_two} is that the structure isomorphisms are unitary. This is, of
course, true as each of them are isomorphisms, hence total and therefore unitary.
\end{example}
%%% Local Variables:
%%% mode: latex
%%% TeX-master: "../../phd-thesis"
%%% End:

\section{Frobenius Algebras} % (fold)
\label{sec:frobenius_algebras}
In their most general setting, Frobenius algebras are defined as a finite dimensional algebra
over a field together with a non-degenerate pairing operation. We will continue with the definitions
that make this precise.

\subsection{Frobenius algebra definitions} % (fold)
\label{sub:frobenius_algebra_definitions}


\begin{definition}\label{def:frobeniusalgebra}
  Given a symmetric monoidal category \cD, a \emph{Frobenius algebra} is an object $X$ of \cD and
  four maps, $\nabla :X\*X \to X$, $e: I \to X$, $\Delta: X\to X\* X$ and $\epsilon:X\to I$, with
  the conditions that $(X,\nabla,e)$ forms a commutative monoid, $(X,\Delta, \epsilon)$ forms a
  commutative comonoid and the diagram
  \[
    \xymatrix{
      X\*X \ar[rr]^{X\*\Delta} \ar[dd]_{\Delta \* X} \ar[dr]^{\nabla}
        && X\*X\*X \ar[dd]^{\nabla\*X}\\
      & X\ar[dr]^{\Delta}\\
      X\*X\*X\ar[rr]_{X\*\nabla}  && X\*X
    }
  \]
  commutes. The Frobenius algebra is \emph{special} when $\Delta \nabla = 1_{X}$ and
  \emph{commutative} when $\Delta c_{X,X} = \Delta$.
\end{definition}
\begin{definition}\label{def:daggerfrob}
  A Frobenius algebra in a dagger symmetric monoidal category where $\Delta = \dgr{\nabla}$ and
  $\epsilon=\dgr{u}$ is a $\dagger$\emph{-Frobenius algebra}.
\end{definition}
For an example of a $\dagger$-Frobenius algebra, consider a finite dimensional Hilbert space $H$
with an orthonormal basis $\{\ket{\phi_{i}}\}$ and define $\Delta:H\to H\*H: \ket{\phi_{i}}\mapsto
\ket{\phi_{i}} \* \ket{\phi_{i}}$ and $\epsilon : H\to \complex : \ket{\phi_{i}} \mapsto 1$. Then $(H,
\nabla=\dgr{\Delta}, u=\dgr{\epsilon}, \Delta, \epsilon)$ forms a commutative special
$\dagger$-Frobenius algebra.

% subsection frobenius_algebra_definitions (end)

\subsection{Bases and Frobenius Algebras} % (fold)
\label{sub:bases_and_frobenius_algebras}
In \cite{coeckeetal08:ortho}, Coecke et. al. provide an algebraic description of orthogonal bases in
finite dimensional Hilbert spaces. Additionally,  an orthonormal basis for such a space is
a special commutative $\dagger$-Frobenius algebra. To show the other direction, given a commutative
$\dagger$-Frobenius algebra, $(H,\nabla,u)$ and for each element $\alpha\in H$, define the right
action of $\alpha$ as $R_{\alpha}:=(id\*\alpha)\, \nabla:H\to H$. Note the use of the fact that
elements $\alpha\in H$ can be considered as linear maps $\alpha:\complex \to H:1\mapsto \ket{\alpha}$.
The dagger of a right action is also a right action, $\dgr{R_{\alpha}} = R_{\alpha'}$ where
$\alpha'= u\, \nabla\, (id\* \dgr{\alpha})$, which is a consequence of the Frobenius identities.

The $(\_)'$ construction is actually an involution:
\begin{eqnarray*}
  &(\alpha')' &= u \nabla (id \* \dgr{\alpha'}) \\
  && = u \nabla (id \* \dgr{(u \nabla (id \* \dgr{\alpha}))}\\
  && = u \nabla (id \* ( (id \* \alpha) \Delta \epsilon))\\
  && = (u \* \alpha) (\nabla \* id) (id \* \Delta) (id \*  \epsilon)\\
  && = (u \* \alpha) (id \* \Delta) (\nabla \* id) (id \*  \epsilon)\\
  && = (u \* \alpha)  (id \*  \epsilon)\\
  && = \alpha
\end{eqnarray*}

\begin{lemma}\label{lemma:cstaralgebra}
  Any $\dagger$-Frobenius algebra in \fdh is a $C^{*}$-algebra.
\end{lemma}
\begin{proof}
  The endomorphism monoid of \fdh(H,H) is a $C^{*}$-algebra. From the proceeding, we have
  \[
    H \cong \fdh(\complex,H) \cong R_{[\fdh(\complex,H)]}\subseteq\fdh(H,H).
  \]
  This inherits the algebra structure from \fdh(H,H). Furthermore, since any finite dimensional
  involution-closed sub-algebra of a $C^{*}$-algebra is also a $C^{*}$-algebra, this shows the
  $\dagger$-Frobenius algebra is a $C^{*}$-algebra.
\end{proof}

Using the fact that the involution preserving homomorphisms from a finite dimensional commutative
$C^{*}$-algebra to $\complex$ form a basis for the dual of the underlying vector space, write these
homomorphisms as $\dgr{\phi_{i}}:H \to \complex$. Then their adjoints, $\phi_{i}:\complex\to H$ will form a
basis for the space $H$. These are the copyable elements in $H$.

This, together with continued applications of the Frobenius rules and linear algebra allow the
authors to prove the following Theorem.
\begin{theorem}
  Every commutative $\dagger$-Frobenius algebra in \fdh determines an orthogonal basis consisting
  of its copyable elements. Conversely, every orthogonal basis $\{\ket{\phi_{i}}\}_{i}$ determines
  a commutative $\dagger$-Frobenius algebra via \[\Delta:H\to H\*H: \ket{\phi_{i}}\mapsto
  \ket{\phi_{i}} \* \ket{\phi_{i}}\qquad\epsilon : H\to \complex : \ket{\phi_{i}} \mapsto 1\] and these
  constructions are inverse to each other.
\end{theorem}

% subsection bases_and_frobenius_algebras (end)

\subsection{Quantum and classical data}\label{sec:quantumclassical}
In \cite{coecke08structures}, Coecke et.al. build on the results of \cite{coeckeetal08:ortho}
to start from a $\dagger$-symmetric monoidal category and construct the minimal machinery needed to
model quantum and classical computations. For the rest of this section, $\cD$ will be assumed to be
such a category, with $\*$ the monoid tensor and $I$ the unit of the monoid.

\begin{definition}\label{def:compact_structure}
  A compact structure on an object $A$ in the category $\cD$ is given by the object $A$, an object
  $A^{*}$ called its \emph{dual} and the maps $\eta:I \to A^{*}\* A$, $\epsilon: A\* A^{*} \to I$
  such that the diagrams
  \[
    \xymatrix@C+20pt{
      A^{*} \ar[dr]^{id} \ar[d]_{\eta\*A^{*}} \\
      A^{*} \*A\*A^{*}  \ar[r]_(.6){A^{*} \*\epsilon} & A^{*}
    }
    \text{ and }
    \xymatrix@C+20pt{
      A \ar[r]^(.4){A\*\eta} \ar[dr]_{id} & A\* A^{*}\* A \ar[d]^{\epsilon\*A}\\
      & A
    }
  \]
  commute.
\end{definition}

\begin{definition}\label{def:quantumstructure}
  A \emph{quantum structure} is an object $A$ and map $\eta:I\to A\*A$ such that
  $(A,A,\eta,\dgr{\eta})$ form a compact structure.
\end{definition}
Note that $A$ is self-dual in definition \ref{def:quantumstructure}.

This allows the creation of the category $\cD_{q}$ which has as objects quantum structures and maps
are the maps in $\cD$ between the objects in the quantum structures.

In the category $\cD_{q}$, it is now possible to define the upper and lower $*$ operations on maps,
such that $(f_{*})^{*}= (f^{*})_{*} = \dgr{f}$:
\begin{eqnarray*}
&f^{*} &:= (\eta_{A}\*1) (1 \* f\*1) (1\*\dgr{\eta}_{B}),\\
&f_{*} &:= (\eta_{B}\*1) (1 \* \dgr{f}\*1) (1\*\dgr{\eta}_{A}).
\end{eqnarray*}

Next, define a classical structure on \cD.
\begin{definition}\label{def:classicalstructure}
  A \emph{classical structure} in \cD{} is an object $X$ together with two maps, $\Delta :X \to X\* X$,
  $\epsilon:X\to I$ such that $(X,\dgr{\Delta},\dgr{\epsilon},\Delta,\epsilon)$ forms a special
  Frobenius algebra.
\end{definition}

As above, this allows us to define $\cD_{c}$, the category whose objects are the classical
structures of $\cD$. The maps in $\cD_{c}$ are given by the maps in $\cD$ between the
objects of the classical structure.

Note that a classical structure will induce a quantum structure, setting $\eta_{X}$ to be
$\dgr{\epsilon_{X}}\, \Delta_{X}$.


Later on, in \ref{sec:the_category_of_commutative_frobenius_algebras}, we will show that commutative
special Frobenius algebras possess a specialized inverse category structure.
% subsection quantum_and_classical_data (end)


% section frobenius_algebras (end)

%%% Local Variables:
%%% mode: latex
%%% TeX-master: "../../phd-thesis"
%%% End:


\subsection{\CFrob is an inverse category}\label{ssec:cfrob_x_is_an_inverse_category}
\begin{example}[Commutative separable Frobenius algebras\cite{kock04}]\label{example:commfrob}
  Let \X be a symmetric monoidal category and form \CFrob as follows: \paragraph{\textbf{Objects:}}
  Commutative separable Frobenius algebras, a quintuple $(A,\nabla,\eta,\Delta,\epsilon)$ where
  $A$ is an object of \X with the following maps:
  $\nabla :A\*A \to A$, $\eta:I\to A$, $\Delta : A \to A\*A$, $\epsilon : A \to I$ which are natural
  maps in \X, with $(A,\nabla,\eta)$ a monoid and $(A,\Delta,\epsilon)$ a comonoid. Additionally,
  these satisfy
  \[
    \xymatrix @C=10pt @R=20pt{
      A \* A \ar[dd]_{1\*\Delta} \ar[dr]^{\nabla}
        \ar[rr]^{\Delta \* 1} & &
        A \* (A \* A) \ar[dd]^{1 \* \nabla}\\
      & A \ar[dr]^{\Delta} & \\
      (A \* A) \* A \ar[rr]_{\nabla \* 1} & &
        A \* A\\
      &*!<0pt,-25pt>{\text{\textbf{Frobenius}}}
    }
  \]
  together with the additional property that $\Delta \nabla = 1$ (separable).

  \paragraph{\textbf{Maps:}} The maps of \X between the objects of \X which preserve multiplication ($\nabla$)
  and comultiplication ($\Delta$) but do not necessarily preserve the units.
  This means a map $f$ must satisfy the following commuting diagrams:
  \[
    \xymatrix@C+25pt{
      A \ar[d]_{\Delta} \ar[r]^{f} & B \ar[d]^{\Delta}\\
      A\*A \ar[r]_{f\*f} & B\* B
    }
    \text{ and }
    \xymatrix@C+25pt{
      A\*A \ar[d]_{\nabla} \ar[r]^{f\*f}& B\*B \ar[d]^{\nabla}\\
      A \ar[r]_{f} & B.
    }
  \]
\end{example}

\begin{lemma}\label{lem:cfrobx_is_an_inverse_category}
  When \X is a symmetric monoidal category, \CFrob is an inverse category.
\end{lemma}
\begin{proof}
  We need to show that \CFrob has restrictions and that each map has a partial inverse. We do
  this by exhibiting the partial inverse of a map.
  For $f:X \to Y$, define $\inv{f}$ as
  \[
    Y \xrightarrow{1\*\eta} Y\*X \xrightarrow{1\*\Delta}
      Y\*X\*X \xrightarrow{1\*f\*1} Y\*Y\*X \xrightarrow{\nabla\*1}
      Y\*X \xrightarrow{\epsilon\*1}X.
  \]
  As a string diagram, this looks like:
  \[
  \begin{tikzpicture}
    \path node at (-.5,3) (start) {}
    node at (0,2.5) [eta] (eta1) {}
    node at (0,2) [delta] (d) {}
    node at (-.25,1.5) [map] (f) {$\scriptstyle f$}
    node at (-.5,1) [nabla] (n1) {}
    node at (-.5,.5) [epsilon] (e1) {}
    node at (0,0) (end) {};
    \draw [] (start) to[out=270,in=125] (n1);
    \draw [] (eta1) to (d);
    \draw [] (d) to[out=305,in=90] (end);
    \draw [] (d) to[out=235,in=90] (f);
    \draw [] (f) to[out=270,in=55] (n1);
    \draw [-] (n1) to (e1);
  \end{tikzpicture}
  \ \raisebox{25pt}{\text{.}}
  \]

  In the following proofs, we also use the following two identities from \cite{kock04}:
  \begin{align}
    (1\*\eta)\nabla &= 1,\\
    \Delta(1\*\epsilon) &= 1.
  \end{align}
  Diagrammatically, this is:
  \[
    \begin{tikzpicture}
    \path   node at (.5,1) (start) {}
    node at (0,1) [eta] (eta1) {}
    node at (.25,.5) [nabla] (n1) {}
    node at (.25,0) (end) {};
    \draw [] (eta1) to[out=270,in=125] (n1);
    \draw [] (start) to[out=270,in=55] (n1);
    \draw [] (n1)   to (end);
  \end{tikzpicture}
  \ \raisebox{15pt}{\text{= }}
  \begin{tikzpicture}
    \path node at (0,1) (start) {}
    node at (0,0) (end) {};
    \draw [-] (start) to (end);
  \end{tikzpicture}
  \ \raisebox{15pt}{\text{=}}
  \begin{tikzpicture}
    \path node at (0,1) (start) {}
    node at (0,.5) [delta] (d1) {}
    node at (-.25,0) (end) {}
    node at (.25,0) [epsilon] (e1) {};
    \draw [] (start) to (d1);
    \draw [] (d1) to[out=305,in=90] (e1);
    \draw [] (d1) to[out=235,in=90] (end);
  \end{tikzpicture}
  \ \raisebox{15pt}{.}
  \]
  Note that when combined with the Frobenius identities, this allows transforms of the following
  types:
  \[
  \begin{tikzpicture}
    \path node at (0,1.5) (s1) {}
    node at (.75,1.5) (s2) {}
    node at (0,1) [delta] (d1) {}
    node at (.5,.5) [nabla] (n1) {}
    node at (0,0) (end) {}
    node at (.5,0) [epsilon] (e1) {};
    \draw [] (s1) to (d1);
    \draw [] (s2) to[out=270,in=55] (n1);
    \draw [] (d1) to[out=235,in=90] (end);
    \draw [] (d1) to[out=305,in=125] (n1);
    \draw [] (n1) to (e1);
  \end{tikzpicture}
  \raisebox{15pt}{$=$}
  \begin{tikzpicture}
    \path node at (0,1.5) (s1) {}
    node at (.5,1.5) (s2) {}
    node at (.25,1) [nabla] (n1) {}
    node at (.25,.5) [delta] (d1) {}
    node at (0,0) (end) {}
    node at (.5,0) [epsilon] (e1) {};
    \draw [] (s1) to[out=270,in=125] (n1);
    \draw [] (s2) to[out=270,in=55] (n1);
    \draw [] (d1) to[out=235,in=90] (end);
    \draw [] (d1) to[out=305,in=90] (e1);
    \draw [] (n1) to (d1);
  \end{tikzpicture}
  \raisebox{15pt}{$=$}
  \begin{tikzpicture}
    \path node at (0,1.5) (s1) {}
    node at (.5,1.5)  (s2) {}
    node at (.25,1) [nabla] (n1) {}
    node at (.25,0.5) (end) {};
    \draw [] (s1) to[out=270,in=125] (n1);
    \draw [] (s2) to[out=270,in=55] (n1);
    \draw [] (n1) to (end);
  \end{tikzpicture}
  \raisebox{15pt}{ and }
  \begin{tikzpicture}
    \path node at (0,1.5)  [eta](s1) {}
    node at (.75,1.5) (s2) {}
    node at (0,1) [delta] (d1) {}
    node at (.5,.5) [nabla] (n1) {}
    node at (0,0) (end) {}
    node at (.5,0) (e1) {};
    \draw [] (s1) to (d1);
    \draw [] (s2) to[out=270,in=55] (n1);
    \draw [] (d1) to[out=235,in=90] (end);
    \draw [] (d1) to[out=305,in=125] (n1);
    \draw [] (n1) to (e1);
  \end{tikzpicture}
  \raisebox{15pt}{$=$}
  \begin{tikzpicture}
    \path node at (0,1.5)  [eta] (s1) {}
    node at (.5,1.5) (s2) {}
    node at (.25,1) [nabla] (n1) {}
    node at (.25,.5) [delta] (d1) {}
    node at (0,0) (end) {}
    node at (.5,0)  (e1) {};
    \draw [] (s1) to[out=270,in=125] (n1);
    \draw [] (s2) to[out=270,in=55] (n1);
    \draw [] (d1) to[out=235,in=90] (end);
    \draw [] (d1) to[out=305,in=90] (e1);
    \draw [] (n1) to (d1);
  \end{tikzpicture}
  \raisebox{15pt}{$=$}
  \begin{tikzpicture}
    \path node at (.25,1.5) (s1) {}
    node at (.25,1) [delta] (d1) {}
    node at (0,0.5) (end) {}
    node at (.5,0.5) (end1) {};
    \draw [] (s1) to (d1);
    \draw [] (d1) to[out=255,in=90] (end);
    \draw [] (d1) to[out=305,in=90] (end1);
  \end{tikzpicture}
  \raisebox{15pt}{.}
  \]

  First, we must show that $\inv{f}$ is a map in the category, i.e., that $\Delta (\inv{f} \*
  \inv{f}) = \inv{f} \Delta$ and $(\inv{f} \* \inv{f})\nabla = \nabla \inv{f}$. We show this for
  $\Delta$ using string diagrams, starting from $\Delta(\inv{f} \*
  \inv{f})$. The proof for the preservation of $\nabla$ proceeds in a similar manner.
  \[
  \begin{tikzpicture}
    \path node at (0,3) (start) {}
    node at (-.75,2.5) [eta] (eta1) {}
    node at (0,2.5) [delta] (d0) {}
    node at (.75,2.5) [eta] (eta2) {}
    node at (-.75,2) [delta] (d) {}
    node at (.75,2) [delta] (d2) {}
    node at (-.5,1.5) [map] (f) {$\scriptstyle f$}
    node at (.5,1.5) [map] (f2) {$\scriptstyle f$}
    node at (-.25,1) [nabla] (n1) {}
    node at (.25,1) [nabla] (n2) {}
    node at (-.25,.5) [epsilon] (e1) {}
    node at (.25,.5) [epsilon] (e2) {}
    node at (-.75,0) (end) {}
    node at (.75,0) (end2) {};
    \draw [] (start) to (d0);
    \draw [] (eta1) to (d);
    \draw [] (eta2) to (d2);
    \draw (d0) to[out=235,in=55] (n1);
    \draw (d0) to[out=305,in=125] (n2);
    \draw [] (d) to[out=235,in=90] (end);
    \draw [] (d) to[out=305,in=90] (f);
    \draw [] (f) to[out=270,in=125] (n1);
    \draw [-] (n1) to (e1);
    \draw [] (d2) to[out=305,in=90] (end2);
    \draw [] (d2) to[out=235,in=90] (f2);
    \draw [] (f2) to[out=270,in=55] (n2);
    \draw [-] (n2) to (e2);
  \end{tikzpicture}
  \raisebox{45pt}{$=$}
  \begin{tikzpicture}
    \path node at (0,3.5) (start) {}
    node at (-.75,2.5) [eta] (eta1) {}
    node at (.75,3) [eta] (eta2) {}
    node at (-.75,2) [delta] (d) {}
    node at (.75,2.5) [delta] (d2) {}
    node at (-.5,1.5) [map] (f) {$\scriptstyle f$}
    node at (.5,2) [map] (f2) {$\scriptstyle f$}
    node at (-.25,1) [nabla] (n1) {}
    node at (.25,1.5) [nabla] (n2) {}
    node at (-.25,.5) [epsilon] (e1) {}
    node at (-.75,0) (end) {}
    node at (.75,0) (end2) {};
    \draw [] (start) to[out=270,in=125] (n2);
    \draw [] (eta1) to (d);
    \draw [] (eta2) to (d2);
    \draw [] (d) to[out=235,in=90] (end);
    \draw [] (d) to[out=305,in=90] (f);
    \draw [] (f) to[out=270,in=125] (n1);
    \draw [-] (n1) to (e1);
    \draw [] (d2) to[out=305,in=90] (end2);
    \draw [] (d2) to[out=235,in=90] (f2);
    \draw [] (f2) to[out=270,in=55] (n2);
    \draw [-] (n2) to[out=270,in=55] (n1);
  \end{tikzpicture}
  \raisebox{45pt}{$=$}
  \begin{tikzpicture}
    \path node at (-0.25,3.5) (start) {}
    node at (-.75,3) [eta] (eta1) {}
    node at (.75,3) [eta] (eta2) {}
    node at (-.75,2.5) [delta] (d) {}
    node at (.75,2.5) [delta] (d2) {}
    node at (-.5,2) [map] (f) {$\scriptstyle f$}
    node at (.5,2) [map] (f2) {$\scriptstyle f$}
    node at (0,1.5) [nabla] (n1) {}
    node at (0.1,1) [nabla] (n2) {}
    node at (0.1,.5) [epsilon] (e1) {}
    node at (-.75,0) (end) {}
    node at (.5,0) (end2) {};
    \draw [] (start) to[out=270,in=55] (n2);
    \draw [] (eta1) to (d);
    \draw [] (eta2) to (d2);
    \draw [] (d) to[out=235,in=90] (end);
    \draw [] (d) to[out=305,in=90] (f);
    \draw [] (f) to[out=270,in=125] (n1);
    \draw [-] (n1) to[out=270,in=125] (n2);
    \draw [] (d2) to[out=305,in=90] (end2);
    \draw [] (d2) to[out=235,in=90] (f2);
    \draw [] (f2) to[out=270,in=55] (n1);
    \draw [-] (n2) to (e1);
  \end{tikzpicture}
  \raisebox{45pt}{$=$}
  \begin{tikzpicture}
    \path node at (.75,3.5) (start) {}
    node at (-.25,3) [eta] (eta1) {}
    node at (.25,3) [eta] (eta2) {}
    node at (-.25,2.5) [delta] (d) {}
    node at (.25,2.5) [delta] (d2) {}
    node at (0,2) [nabla] (n1) {}
    node at (0,1.5) [map] (f) {$\scriptstyle f$}
    node at (0.25,1) [nabla] (n2) {}
    node at (0.25,.5) [epsilon] (e1) {}
    node at (-.5,0) (end) {}
    node at (.75,0) (end2) {};
    \draw [] (start) to[out=270,in=55] (n2);
    \draw [] (eta1) to (d);
    \draw [] (eta2) to (d2);
    \draw [] (d) to[out=235,in=90] (end);
    \draw [] (d) to[out=305,in=125] (n1);
    \draw [] (d2) to[out=305,in=90] (end2);
    \draw [] (d2) to[out=235,in=55] (n1);
    \draw [-] (n1) to[out=270,in=90] (f);
    \draw [] (f) to[out=270,in=125] (n2);
    \draw [-] (n2) to (e1);
  \end{tikzpicture}
  \raisebox{45pt}{$=$}
  \begin{tikzpicture}
    \path node at (.75,3.5) (start) {}
    node at (.25,3) [eta] (eta2) {}
    node at (.25,2.5) [delta] (d2) {}
    node at (-0.25,2) [delta] (d) {}
    node at (0,1.5) [map] (f) {$\scriptstyle f$}
    node at (0.25,1) [nabla] (n2) {}
    node at (0.25,.5) [epsilon] (e1) {}
    node at (-.5,0) (end) {}
    node at (.5,0) (end2) {};
    \draw [] (start) to[out=270,in=55] (n2);
    \draw [] (eta2) to (d2);
    \draw [] (d2) to[out=305,in=90] (end2);
    \draw [] (d2) to[out=235,in=90] (d);
    \draw [] (d) to[out=235,in=90] (end);
    \draw [] (d) to[out=305,in=90] (f);
    \draw [] (f) to[out=270,in=125] (n2);
    \draw [-] (n2) to (e1);
  \end{tikzpicture}
  \ \raisebox{45pt}{$=$}
  \begin{tikzpicture}
    \path node at (.75,3) (start) {}
    node at (0,2.5) [eta] (eta2) {}
    node at (0,2) [delta] (d2) {}
    node at (-0.25,1.5) [delta] (d) {}
    node at (.25,1.5) [map] (f) {$\scriptstyle f$}
    node at (0.5,1) [nabla] (n2) {}
    node at (0.5,.5) [epsilon] (e1) {}
    node at (-.5,0) (end) {}
    node at (0,0) (exit) {};
    \draw [] (start) to[out=270,in=55] (n2);
    \draw [] (eta2) to (d2);
    \draw [] (d2) to[out=235,in=90] (d);
    \draw [] (d2) to[out=305,in=90] (f);
    \draw [] (d) to[out=235,in=90] (end);
    \draw [] (d) to[out=305,in=90] (exit);
    \draw [] (f) to[out=270,in=125] (n2);
    \draw [-] (n2) to (e1);
  \end{tikzpicture}
  \raisebox{45pt}{$=\inv{f}\Delta$.}
  \]
  Thus, $\inv{f}$ is a map in the category whenever $f$ is.

  If $\inv{f}$ is truly a partial inverse, we may then define $\rst{f} = f \inv{f}$.
  Using Theorem 2.20 from \cite{cockett2002:restcategories1}, we need only show:
  \begin{align}
    \inv{(\inv{f})} &= f\label{eq:finvinv_is_f}\\
    f\inv{f}f &= f\label{eq:ffinvf_is_f}\\
    f\inv{f}g\inv{g} &=g\inv{g} f\inv{f}.\label{eq:ffinv_commutes_gginv}
  \end{align}
  Proof of Equation~\ref{eq:finvinv_is_f}: $\inv{(\inv{f})} =$
  \[
  \begin{tikzpicture}
    \path node at (-.75,4) (start) {}
    node at (0,3.5) [eta] (eta2) {}
    node at (0,3) [delta] (d2) {}
    node at (0,2.5) [eta] (eta1) {}
    node at (0,2) [delta] (d1) {}
    node at (-.25,1.5) [map] (f) {$\scriptstyle f$}
    node at (-.5,1) [nabla] (n1) {}
    node at (-.5,.5) [epsilon] (e1) {}
    node at (-.5,0) [nabla] (n2) {}
    node at (-.5,-.5) [epsilon] (e2) {}
    node at (.25,-1) (end) {};
    \draw [] (start) to[out=270,in=125] (n2);
    \draw [] (eta2) to (d2);
    \draw [] (d2) to[out=235,in=125] (n1);
    \draw [] (d2) to[out=305,in=90] (end);
    \draw [] (eta1) to (d1);
    \draw [] (d1) to[out=305,in=55] (n2);
    \draw [] (d1) to[out=235,in=90] (f);
    \draw [] (f) to[out=270,in=55] (n1);
    \draw [-] (n1) to (e1);
    \draw [-] (n2) to (e2);
  \end{tikzpicture}
  \ \raisebox{70pt}{\text{=}}
  \begin{tikzpicture}
    \path node at (.25,4) (start) {}
    node at (-.5,3.5) [eta] (eta2) {}
    node at (-.5,3) [delta] (d2) {}
    node at (0,2.5) [eta] (eta1) {}
    node at (0,2) [delta] (d1) {}
    node at (-.25,1.5) [map] (f) {$\scriptstyle f$}
    node at (-.5,1) [nabla] (n1) {}
    node at (-.5,.5) [epsilon] (e1) {}
    node at (0,0) [nabla] (n2) {}
    node at (0,-.5) [epsilon] (e2) {}
    node at (-.5,-1) (end) {};
    \draw [] (start) to[out=270,in=55] (n2);
    \draw [] (eta2) to (d2);
    \draw [] (d2) to[out=305,in=125] (n1);
    \draw [] (d2) to[out=235,in=90] (end);
    \draw [] (eta1) to (d1);
    \draw [] (d1) to[out=305,in=125] (n2);
    \draw [] (d1) to[out=235,in=90] (f);
    \draw [] (f) to[out=270,in=55] (n1);
    \draw [-] (n1) to (e1);
    \draw [-] (n2) to (e2);
  \end{tikzpicture}
  \ \raisebox{70pt}{\text{= }}
  \begin{tikzpicture}
    \path node at (.25,4) (start) {}
    node at (-.5,1) [eta] (eta2) {}
    node at (-.5,.5) [delta] (d2) {}
    node at (0,3.5) [eta] (eta1) {}
    node at (0,3) [delta] (d1) {}
    node at (-.25,1.5) [map] (f) {$\scriptstyle f$}
    node at (-.25,0) [nabla] (n1) {}
    node at (-.25,-.5) [epsilon] (e1) {}
    node at (.25,2.5) [nabla] (n2) {}
    node at (.25,2) [epsilon] (e2) {}
    node at (-.5,-1) (end) {};
    \draw [] (start) to[out=270,in=55] (n2);
    \draw [] (eta2) to (d2);
    \draw [] (d2) to[out=305,in=125] (n1);
    \draw [] (d2) to[out=235,in=90] (end);
    \draw [] (eta1) to (d1);
    \draw [] (d1) to[out=305,in=125] (n2);
    \draw [] (d1) to[out=235,in=90] (f);
    \draw [] (f) to[out=270,in=55] (n1);
    \draw [-] (n1) to (e1);
    \draw [-] (n2) to (e2);
  \end{tikzpicture}
  \ \raisebox{70pt}{\text{= }}
  \begin{tikzpicture}
    \path node at (.5,4) (start) {}
    node at (0,3.5) [eta] (eta1) {}
    node at (.25,3) [nabla] (n1) {}
    node at (.25,2.5) [delta] (d1) {}
    node at (.5,2) [epsilon] (e1) {}
    node at (0,1.5) [map] (f) {$\scriptstyle f$}
    node at (-.5,1) [eta] (eta2) {}
    node at (-.25,.5) [nabla] (n2) {}
    node at (-.25,0) [delta] (d2) {}
    node at (0,-.5) [epsilon] (e2) {}
    node at (-.5,-1) (end) {};
    \draw [] (start) to[out=270,in=55] (n1);
    \draw [] (eta1) to[out=270,in=125] (n1);
    \draw [] (n1) to (d1);
    \draw [] (d1) to[out=305,in=90] (e1);
    \draw [] (d1) to[out=235,in=90] (f);
    \draw [] (f) to[out=270,in=55] (n2);
    \draw [] (eta2) to[out=270,in=125] (n2);
    \draw [-] (n2) to (d2);
    \draw [] (d2) to[out=305,in=125] (e2);
    \draw [] (d2) to[out=235,in=90] (end);
  \end{tikzpicture}
  \ \raisebox{70pt}{\text{=}}
  \begin{tikzpicture}
    \path node at (.5,4) (start) {}
    node at (0,1.5) [map] (f) {$\scriptstyle f$}
    node at (-.5,-1) (end) {};
    \draw [] (start) to[out=270,in=90] (f);
    \draw [] (f) to[out=270,in=90] (end);
  \end{tikzpicture}
  \ \raisebox{70pt}{\text{= }$f$.}
  \]
  Proof of Equation~\ref{eq:ffinvf_is_f}: $f \inv{f} f =$
  \[
  \begin{tikzpicture}
    \path node at (-.5,3) (start) {}
    node at (0,2.5) [eta] (eta1) {}
    node at (0,2) [delta] (d) {}
    node at (-.75,1.5) [map] (f1) {$\scriptstyle f$}
    node at (-.25,1.5) [map] (f2) {$\scriptstyle f$}
    node at (.25,1.5) [map] (f3) {$\scriptstyle f$}
    node at (-.5,1) [nabla] (n1) {}
    node at (-.5,.5) [epsilon] (e1) {}
    node at (0,0) (end) {};
    \draw [] (start) to[out=270,in=90] (f1);
    \draw [] (f1) to [out=270,in=125] (n1);
    \draw [] (eta1) to (d);
    \draw [] (d) to[out=305,in=90] (f3);
    \draw [] (f3) to[out=270,in=90] (end);
    \draw [] (d) to[out=235,in=90] (f2);
    \draw [] (f2) to[out=270,in=55] (n1);
    \draw [-] (n1) to (e1);
  \end{tikzpicture}
  \ \raisebox{40pt}{$=$ }
  \begin{tikzpicture}
    \path node at (-.5,3) (start) {}
    node at (0,2.5) [eta] (eta1) {}
    node at (0,2) [delta] (d) {}
    node at (-.5,1.5) [nabla] (n1) {}
    node at (-.5,1) [map] (f2) {$\scriptstyle f$}
    node at (.25,1) [map] (f3) {$\scriptstyle f$}
    node at (-.5,.5) [epsilon] (e1) {}
    node at (0,0) (end) {};
    \draw [] (start) to [out=270,in=125] (n1);
    \draw (n1) to (f2);
    \draw (f2) to (e1);
    \draw [] (eta1) to (d);
    \draw [] (d) to[out=305,in=90] (f3);
    \draw [] (f3) to[out=270,in=90] (end);
    \draw [] (d) to[out=235,in=55] (n1);
  \end{tikzpicture}
  \ \raisebox{40pt}{$=$ }
  \begin{tikzpicture}
    \path node at (-.5,3) (start) {}
    node at (0,2.5) [eta] (eta1) {}
    node at (-.25,2) [nabla] (n1) {}
    node at (-.25,1.5) [delta] (d) {}
    node at (-.5,1) [map] (f2) {$\scriptstyle f$}
    node at (0,1) [map] (f3) {$\scriptstyle f$}
    node at (-.5,.5) [epsilon] (e1) {}
    node at (0,0) (end) {};
    \draw [] (start) to[out=270,in=125] (n1);
    \draw [] (eta1) to[out=270,in=55] (n1);
    \draw [] (n1) to (d);
    \draw [] (d) to[out=305,in=90] (f3);
    \draw [] (d) to[out=235,in=90] (f2);
    \draw (f2) to (e1);
    \draw [] (f3) to[out=270,in=90] (end);
  \end{tikzpicture}
  \ \raisebox{40pt}{\text{= }}
  \begin{tikzpicture}
    \path node at (-.5,3) (start) {}
    node at (-.25,1.5) [map] (f3) {$\scriptstyle f$}
    node at (-.25,1) [delta] (d) {}
    node at (-.5,.5) [epsilon] (e1) {}
    node at (0,0) (end) {};
    \draw [] (start) to[out=270,in=90] (f3);
    \draw [] (f3) to (d);
    \draw [] (d) to[out=305,in=90] (end);
    \draw [] (d) to[out=235,in=90] (e1);
  \end{tikzpicture}
  \ \raisebox{40pt}{\text{= }}
  \begin{tikzpicture}
    \path node at (-.5,3) (start) {}
    node at (-.25,1.5) [map] (f3) {$\scriptstyle f$}
    node at (0,0) (end) {};
    \draw [] (start) to[out=270,in=90] (f3);
    \draw [] (f3) to[out=270,in=90] (end);
  \end{tikzpicture}
  \ \raisebox{40pt}{$= f$.}
  \]
  Proof of Equation~\ref{eq:ffinv_commutes_gginv}:  $f\inv{f}g\inv{g} =$

  \[
  \begin{tikzpicture}
    \path node at (-.5,3) (start) {}
    node at (0,2.5) [eta] (eta1) {}
    node at (0,2) [delta] (d1) {}
    node at (1,2.5) [eta] (eta2) {}
    node at (1,2) [delta] (d2) {}
    node at (-.75,1.5) [map] (f1) {$\scriptstyle f$}
    node at (-.25,1.5) [map] (f2) {$\scriptstyle f$}
    node at (.25,1.5) [map] (g1) {$\scriptstyle g$}
    node at (.75,1.5) [map] (g2) {$\scriptstyle g$}
    node at (-.5,1) [nabla] (n1) {}
    node at (-.5,.5) [epsilon] (e1) {}
    node at (.5,1) [nabla] (n2) {}
    node at (.5,.5) [epsilon] (e2) {}
    node at (.75,0) (end) {};
    \draw [] (start) to[out=270,in=90] (f1);
    \draw [] (eta1) to (d1);
    \draw [] (eta2) to (d2);
    \draw [] (d1) to[out=235,in=90] (f2);
    \draw [] (d1) to[out=305,in=90] (g1);
    \draw [] (d2) to[out=235,in=90] (g2);
    \draw [] (d2) to[out=305,in=90] (end);
    \draw [] (f1) to [out=270,in=125] (n1);
    \draw [] (f2) to[out=270,in=55] (n1);
    \draw [] (g1) to[out=270,in=125] (n2);
    \draw [] (g2) to[out=270,in=55] (n2);
    \draw [-] (n1) to (e1);
    \draw [-] (n2) to (e2);
  \end{tikzpicture}
  \ \raisebox{40pt}{$=$ }
  \begin{tikzpicture}
    \path node at (-.5,3) (start) {}
    node at (0,2.5) [eta] (eta1) {}
    node at (0,2) [delta] (d1) {}
    node at (1,2.5) [eta] (eta2) {}
    node at (1,2) [delta] (d2) {}
    node at (-.5,1.5) [nabla] (n1) {}
    node at (.5,1.5) [nabla] (n2) {}
    node at (-.5,1) [map] (f1) {$\scriptstyle f$}
    node at (.5,1) [map] (g1) {$\scriptstyle g$}
    node at (-.5,.5) [epsilon] (e1) {}
    node at (.5,.5) [epsilon] (e2) {}
    node at (.75,0) (end) {};
    \draw [] (start) to[out=270,in=125] (n1);
    \draw [] (eta1) to (d1);
    \draw [] (eta2) to (d2);
    \draw [] (d1) to[out=235,in=55] (n1);
    \draw [] (d1) to[out=305,in=125] (n2);
    \draw [] (d2) to[out=235,in=55] (n2);
    \draw [] (d2) to[out=305,in=90] (end);
    \draw [-] (n1) to (f1);
    \draw [-] (n2) to (g1);
    \draw [] (f1) to (e1);
    \draw [] (g1) to (e2);
  \end{tikzpicture}
  \ \raisebox{40pt}{$=$ }
  \begin{tikzpicture}
    \path node at (-.25,3) (start) {}
    node at (0,2.5) [eta] (eta1) {}
    node at (-.25,2) [nabla] (n1) {}
    node at (-.25,1.5) [delta] (d1) {}
    node at (.5,1.5) [eta] (eta2) {}
    node at (-.5,1) [map] (f1) {$\scriptstyle f$}
    node at (.25,1) [nabla] (n2) {}
    node at (-.5,.5) [epsilon] (e1) {}
    node at (.25,.5) [delta] (d2) {}
    node at (0,0) [map] (g1) {$\scriptstyle g$}
    node at (0,-.5) [epsilon] (e2) {}
    node at (.25,-1) (end) {};
    \draw [] (start) to[out=270,in=125] (n1);
    \draw [] (eta1) to[out=270,in=55] (n1);
    \draw [] (n1) to (d1);
    \draw [] (eta2) to[out=270,in=55] (n2);
    \draw [] (d1) to[out=235,in=90] (f1);
    \draw [] (d1) to[out=305,in=125] (n2);
    \draw [] (f1) to (e1);
    \draw [] (n2) to (d2);
    \draw [] (d2) to[out=235,in=90] (g1);
    \draw [] (d2) to[out=305,in=90] (end);
    \draw [] (g1) to (e2);
  \end{tikzpicture}
  \ \raisebox{40pt}{$=$ }
  \begin{tikzpicture}
    \path node at (-.25,2.5) (start) {}
    node at (-.25,1.5) [delta] (d1) {}
    node at (-.5,.5) [map] (f1) {$\scriptstyle f$}
    node at (-.5,0) [epsilon] (e1) {}
    node at (.25,1) [delta] (d2) {}
    node at (0,.5) [map] (g1) {$\scriptstyle g$}
    node at (0,0) [epsilon] (e2) {}
    node at (.25,-.5) (end) {};
    \draw [] (start) to (d1);
    \draw [] (d1) to[out=235,in=90] (f1);
    \draw [] (d1) to[out=305,in=90] (d2);
    \draw [] (f1) to (e1);
    \draw [] (d2) to[out=235,in=90] (g1);
    \draw [] (d2) to[out=305,in=90] (end);
    \draw [] (g1) to (e2);
  \end{tikzpicture}
  \ \raisebox{40pt}{$=$ }
  \begin{tikzpicture}
    \path node at (-.25,2.5) (start) {}
    node at (.25,2) [delta] (d2) {}
    node at (-.25,1.5) [delta] (d1) {}
    node at (-.5,.5) [map] (f1) {$\scriptstyle f$}
    node at (-.5,0) [epsilon] (e1) {}
    node at (0,.5) [map] (g1) {$\scriptstyle g$}
    node at (0,0) [epsilon] (e2) {}
    node at (.25,-.5) (end) {};
    \draw [] (start) to[out=270,in=90] (d2);
    \draw [] (d2) to[out=235,in=90] (d1);
    \draw [] (d2) to[out=305,in=90] (end);
    \draw [] (d1) to[out=235,in=90] (f1);
    \draw [] (d1) to[out=305,in=90] (g1);
    \draw [] (f1) to (e1);
    \draw [] (g1) to (e2);
  \end{tikzpicture}
  \ \raisebox{40pt}{$=$ }
  \begin{tikzpicture}
    \path node at (-.25,2.5) (start) {}
    node at (.25,2) [delta] (d2) {}
    node at (-.25,1.5) [delta] (d1) {}
    node at (-.5,1) [map] (g1) {$\scriptstyle g$}
    node at (-.5,.5) [epsilon] (e2) {}
    node at (0,1) [map] (f1) {$\scriptstyle f$}
    node at (0,.5) [epsilon] (e1) {}
    node at (.25,0) (end) {};
    \draw [] (start) to[out=270,in=90] (d2);
    \draw [] (d2) to[out=235,in=90] (d1);
    \draw [] (d2) to[out=305,in=90] (end);
    \draw [] (d1) to[out=235,in=90] (g1);
    \draw [] (d1) to[out=305,in=90] (f1);
    \draw [] (f1) to (e1);
    \draw [] (g1) to (e2);
  \end{tikzpicture}
  \ \raisebox{40pt}{$= g\inv{g}f\inv{f}$ }
\]
where the last step is accomplished by reversing all the previous diagrammatic steps.
Hence, \CFrob is an inverse category.
\end{proof}

\begin{theorem}\label{thm:cfrob_is_a_discrete_inverse_category}
  When \X is a symmetric monoidal category, \CFrob is a discrete inverse category.
\end{theorem}
\begin{proof}
  Lemma~\ref{lem:cfrobx_is_an_inverse_category} shows \CFrob is an inverse category. We
  need to show the conditions of Definition~\ref{def:inverse_product} are met.

  First, we see that the tensor of $\X$ is a tensor in \CFrob. $A\*B$ is an object in \CFrob
  with $\Delta_{A\*B} = (\Delta_A\*\Delta_B)(1\*c_{\*}\*1)$,
  $\nabla_{A\*B} =  (1\*c_{\*}\*1)(\nabla_A\*\nabla_B)$,
  $\eta_{A\*B} = \Delta_I(\eta_A \* \eta_B)$, and
  $\epsilon_{A\*B} =  (\epsilon_A\*\epsilon_B)\nabla_I$.

  The map $\Delta : A \to A\*A$ is a map in \CFrob. To show it preserves $\Delta$, we need to
  show $\Delta_A \Delta_{A\*A} = \Delta_A (\Delta_A \* \Delta_A)$:
  \[
  \raisebox{20pt}{$\Delta_A \Delta_{A\*A} =$}
  \begin{tikzpicture}
    \path node at (0,1.5) (start) {}
    node at (0,1) [delta] (d0) {}
    node at (-.25,.5) [delta] (d1) {}
    node at (.25,.5) [delta] (d2) {}
    node at (-.35,0) (e1) {}
    node at (-.15,0) (e2) {}
    node at (.15,0) (e3) {}
    node at (.35,0) (e4) {};
    \draw [] (start) to[out=270,in=90] (d0);
    \draw [] (d0) to[out=235,in=90] (d1);
    \draw [] (d0) to[out=305,in=90] (d2);
    \draw [] (d1) to[out=235,in=90] (e1);
    \draw [] (d1) to[out=305,in=90] (e3);
    \draw [] (d2) to[out=235,in=90] (e2);
    \draw [] (d2) to[out=305,in=90] (e4);
  \end{tikzpicture}
  \raisebox{20pt}{$=$}
  \begin{tikzpicture}
    \path node at (0,2) (start) {}
    node at (-.25,1.5) [delta] (d0) {}
    node at (0,1) [delta] (d1) {}
    node at (.25,.5) [delta] (d2) {}
    node at (-.35,0) (e1) {}
    node at (-.15,0) (e2) {}
    node at (.15,0) (e3) {}
    node at (.35,0) (e4) {};
    \draw [] (start) to[out=270,in=90] (d0);
    \draw [] (d0) to[out=235,in=90] (e1);
    \draw [] (d0) to[out=305,in=90] (d1);
    \draw [] (d1) to[out=235,in=90] (e3);
    \draw [] (d1) to[out=305,in=90] (d2);
    \draw [] (d2) to[out=235,in=90] (e2);
    \draw [] (d2) to[out=305,in=90] (e4);
  \end{tikzpicture}
  \raisebox{20pt}{$=$}
  \begin{tikzpicture}
    \path node at (0,2) (start) {}
    node at (-.25,1.5) [delta] (d0) {}
    node at (.25,1) [delta] (d1) {}
    node at (0,.5) [delta] (d2) {}
    node at (-.35,0) (e1) {}
    node at (-.15,0) (e2) {}
    node at (.15,0) (e3) {}
    node at (.35,0) (e4) {};
    \draw [] (start) to[out=270,in=90] (d0);
    \draw [] (d0) to[out=235,in=90] (e1);
    \draw [] (d0) to[out=305,in=90] (d1);
    \draw [] (d1) to[out=235,in=90] (d2);
    \draw [] (d1) to[out=305,in=90] (e4);
    \draw [] (d2) to[out=235,in=90] (e3);
    \draw [] (d2) to[out=305,in=90] (e2);
  \end{tikzpicture}
  \raisebox{20pt}{$=$}
  \begin{tikzpicture}
    \path node at (0,2) (start) {}
    node at (-.25,1.5) [delta] (d0) {}
    node at (.25,1) [delta] (d1) {}
    node at (0,.5) [delta] (d2) {}
    node at (-.35,0) (e1) {}
    node at (-.15,0) (e2) {}
    node at (.15,0) (e3) {}
    node at (.35,0) (e4) {};
    \draw [] (start) to[out=270,in=90] (d0);
    \draw [] (d0) to[out=235,in=90] (e1);
    \draw [] (d0) to[out=305,in=90] (d1);
    \draw [] (d1) to[out=235,in=90] (d2);
    \draw [] (d1) to[out=305,in=90] (e4);
    \draw [] (d2) to[out=235,in=90] (e2);
    \draw [] (d2) to[out=305,in=90] (e3);
  \end{tikzpicture}
  \raisebox{20pt}{$=\Delta_A (\Delta_A \* \Delta_A).$}
  \]
  Note that in the last step, we simply reverse the various associativity steps used previously.

  To show that $\Delta$ preserves the $\nabla$, we must show that
  $(\Delta_A\*\Delta_A)\nabla_{A\*A} = \nabla_A \Delta_A$. Starting with $(\Delta_A\*\Delta_A)\nabla_{A\*A} =$
  \[
  \begin{tikzpicture}
    \path node at (0,1.5) (s1) {}
    node at (.5,1.5) (s2) {}
    node at (0,1) [delta] (d0) {}
    node at (.5,1) [delta] (d1) {}
    node at (0,.5) [nabla] (n0) {}
    node at (.5,.5) [nabla] (n1) {}
    node at (0,0) (e0) {}
    node at (.5,0) (e1) {};
    \draw [] (s1) to[out=270,in=90] (d0);
    \draw [] (s2) to[out=270,in=90] (d1);
    \draw [] (d0) to[out=235,in=125] (n0);
    \draw [] (d0) to[out=305,in=125] (n1);
    \draw [] (d1) to[out=235,in=55] (n0);
    \draw [] (d1) to[out=305,in=55] (n1);
    \draw [] (n0) to[out=270,in=90] (e0);
    \draw [] (n1) to[out=270,in=90] (e1);
  \end{tikzpicture}
  \raisebox{20pt}{$=$}
  \begin{tikzpicture}
    \path node at (0,2.5) (s1) {}
    node at (.5,2.5) (s2) {}
    node at (.5,2) [delta] (d1) {}
    node at (0,1.5) [nabla] (n0) {}
    node at (0,1) [delta] (d0) {}
    node at (0,.5) [nabla] (n1) {}
    node at (0,0) (e0) {}
    node at (.5,0) (e1) {};
    \draw [] (s1) to[out=270,in=125] (n0);
    \draw [] (s2) to[out=270,in=90] (d1);
    \draw [] (d1) to[out=235,in=55] (n0);
    \draw [] (d1) to[out=305,in=55] (n1);
    \draw [] (n0) to[out=270,in=90] (d0);
    \draw [] (d0) to[out=235,in=125] (n1);
    \draw [] (d0) to[out=305,in=90] (e1);
    \draw [] (n1) to[out=270,in=90] (e0);
  \end{tikzpicture}
  \raisebox{20pt}{$=$}
  \begin{tikzpicture}
    \path node at (0,2.5) (s1) {}
    node at (.25,2.5) (s2) {}
    node at (.25,2) [nabla] (n0) {}
    node at (.25,1.5) [delta] (d1) {}
    node at (0,1) [delta] (d0) {}
    node at (0,.5) [nabla] (n1) {}
    node at (0,0) (e0) {}
    node at (.5,0) (e1) {};
    \draw [] (s1) to[out=270,in=125] (n0);
    \draw [] (s2) to[out=270,in=55] (n0);
    \draw [] (n0) to[out=270,in=90] (d1);
    \draw [] (d1) to[out=235,in=90] (d0);
    \draw [] (d1) to[out=305,in=55] (n1);
    \draw [] (d0) to[out=235,in=125] (n1);
    \draw [] (d0) to[out=305,in=90] (e1);
    \draw [] (n1) to[out=270,in=90] (e0);
  \end{tikzpicture}
  \raisebox{20pt}{$=$}
  \begin{tikzpicture}
    \path node at (0,2.5) (s1) {}
    node at (.25,2.5) (s2) {}
    node at (.25,2) [nabla] (n0) {}
    node at (.25,1.5) [delta] (d1) {}
    node at (0,1) [delta] (d0) {}
    node at (0,.5) [nabla] (n1) {}
    node at (0,0) (e0) {}
    node at (.5,0) (e1) {};
    \draw [] (s1) to[out=270,in=125] (n0);
    \draw [] (s2) to[out=270,in=55] (n0);
    \draw [] (n0) to[out=270,in=90] (d1);
    \draw [] (d1) to[out=235,in=90] (d0);
    \draw [] (d1) to[out=305,in=90] (e1);
    \draw [] (d0) to[out=235,in=125] (n1);
    \draw [] (d0) to[out=305,in=55] (n1);
    \draw [] (n1) to[out=270,in=90] (e0);
  \end{tikzpicture}
  \raisebox{20pt}{$=$}
  \begin{tikzpicture}
    \path node at (0,2.5) (s1) {}
    node at (.5,2.5) (s2) {}
    node at (.25,2) [nabla] (n0) {}
    node at (.25,1.5) [delta] (d1) {}
    node at (0,1) (e0) {}
    node at (.5,1) (e1) {};
    \draw [] (s1) to[out=270,in=125] (n0);
    \draw [] (s2) to[out=270,in=55] (n0);
    \draw [] (n0) to[out=270,in=90] (d1);
    \draw [] (d1) to[out=235,in=90] (e0);
    \draw [] (d1) to[out=305,in=90] (e1);
  \end{tikzpicture}
  \raisebox{20pt}{$= \nabla_A \Delta_A$.}
  \]
  Note that the proof uses the ``special'' property in a non-trivial way.

  Thus, we have a $\Delta$ in \CFrob. As $\nabla = \inv{\Delta}$, the Frobenius requirement for
  the inverse product is immediately fulfilled. Commutativity, cocommutativity, associativity,
  coassociativity and the exchange rule all follow from the properties of the commutative Frobenius
  algebras and therefore \CFrob is a discrete inverse category.
\end{proof}

% section the_category_of_commutative_frobenius_algebras (end)

%%% Local Variables:
%%% mode: latex
%%% TeX-master: "../../phd-thesis"
%%% End:



% chapter inverse_categories (end)
%%% Local Variables:
%%% mode: latex
%%% TeX-master: "../phd-thesis"
%%% End:
