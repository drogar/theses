%!TEX root = /Users/gilesb/UofC/thesis/phd-thesis/phd-thesis.tex
\chapter{Inverse categories and products} % (fold)
\label{cha:inverse_categories}

This chapter will introduce inverse categories. We first give
a few results about inverse categories and then proceed to show an inverse category which has
restriction products is a restriction preorder.

Given this fact, the chapter then focuses on adding product-like structures to an inverse category,
which we call \emph{inverse products}. These will be defined below in
Subsection~\ref{sub:inverse_product_definition}. Inverse products are given by a natural structure
on a tensor product which includes a diagonal but lacks projections. The diagonal map is required to
give a natural Frobenius structure on each object.

\section{Inverse categories}
\label{sec:inverse_categories}

\begin{definition}\label{def:inverse_category}
  A restriction category in which every map is a partial
  isomorphism is called an \emph{inverse category}.
\end{definition}

\begin{lemma}
  \label{lem:inverse_idempotents_are_restriction_idempotents}
  In an inverse category, all idempotents are restriction idempotents.
\end{lemma}
\begin{proof}
  Given an idempotent $e$,
  \[
    \rst{e} = e\inv{e} = e e \inv{e} = e \rst{e} = \rst{e e} e = \rst{e} e = e.
  \]
\end{proof}

\begin{lemma}\label{lem:inverse_categories_are_range_categories}
  An inverse category \X is a range category (Definition~\ref{def:range_category}), where
  $\rg{f} = \inv{f}f = \rst{\inv{f}}$.
\end{lemma}
\begin{proof}
  \prepprooflist
  \setlist[enumerate,1]{leftmargin=1.5cm}
  \begin{enumerate}
    \item[\rrone] $\restr{\rg{f}} = \rst{\rst{\inv{f}}} = \rst{\inv{f}} = \rg{f}$;
    \item[\rrtwo] $f \rg{f} = f \rst{\inv{f}} = f \inv{f} f = \rst{f} f = f$;
    \item[\rrthree] $\wrg{f\rst{g}} = \rst{\inv{(f\rst{g})}} = \rst{\inv{\rst{g}} \inv{f}} =
      \rst{\rst{g} \inv{f}} =
      \rst{g} \rst{\inv{f}} = \rst{\inv{f}} \rst{g} =\rg{f} \rst{g}$;
    \item[\rrfour]  $\wrg{\rg{f}g} = \rst{\inv{(\rst{\inv{f}} g)}} =
      \rst{\inv{g}\inv{\rst{\inv{f}}}} = \rst{\inv{g} \rst{\inv{f}}} =
      \rst{\inv{g} \inv{f}} = \rst{\inv{(f g)}} = \wrg{f g}$.
  \end{enumerate}
\end{proof}

For an inverse category \X, as $\inv{()}$ is an involution, the range $(\rg{\_})$ is in fact more
than just a range, it is a restriction in $\dual{\X}$.

The property of being an inverse category is preserved by splitting.

\begin{lemma}\label{lem:the_idempotent_splitting_of_an_inverse_category_is_an_inverse_category}
  When \X is an inverse category, $\spl{E}{X}$ is an inverse category.
\end{lemma}
\begin{proof}
  First recall that in the split category, for  $f:(A,e_1)\to(B,e_2)$, we have
  $f = e_1 f e_2 = e_1 f = f e_2$ and  by Proposition~\ref{prop:spleisarestrictioncat}  the
  restriction of  $f:(A,e_1)\to(B,e_2)$ is $e_1\rst{f}$. Then, the inverse of
  $f:(A,e_1)\to(B,e_2)$   in \spl{E}{\X}  is $\inv{f}$ as
  \[
    \inv{f} = \inv{e_1 f e_2} = \inv{e_2} \inv{f} \inv{e_1} = e_2 \inv{f} e_1.
  \]
  Note the last equality uses
  Lemma~\ref{lem:inverse_idempotents_are_restriction_idempotents}. Additionally, we have
  \[
    f \inv{f}
    = e_1 f \inv{f} = e_1 \rst{f}
    \text{ and }
    \inv{f} f =
    e_2 \inv{f} f
    = e_2 \restr{\inv{f}}.
  \]

\end{proof}

\begin{example}[\pinj is an inverse category]\label{ex:pinj_is_an_inverse_category}
  For any map $f$, $\inv{f} = \{(y,x) | (x,y) \in
  f\}$. Note that $\inv{f}$ is a map in $\pinj$ due to the two dual conditions on maps as given in
  Example~\ref{ex:category_pinj}.
\end{example}
\begin{example}[\Par is not an inverse category]\label{ex:par_is_not_an_inverse_category}
  \Par, while it is a restriction category, is not an inverse category. For example, let
  $A=\{1,2\},\ B=\{1\}$ and $f=\{(1,1),(2,1)\}$ in \Par. The restriction of $f$ is $\rst{f} =
  \{(1,1),(2,2)\} = 1_A$. There is no partial function $g:B\to A$ such that $f g = 1_A$.
\end{example}
\begin{example}\label{ex:partial_isos_are_inverse_category}
  Generally, let \R be a restriction category, and $\Inv{\R}$ the subcategory of \R having the same objects
  as \R but only the partial isomorphisms as maps. Then, $\Inv{\R}$ is an inverse category.
\end{example}
\begin{example}\label{ex:groupoid_is_inverse_category}
  A groupoid, which is a category in which every map is an isomorphism, is an inverse category. As
  all maps in the groupoid are total, the partial isomorphisms are all  isomorphisms.

  As well, we note that for $\X$ any inverse category, $Total(\X)$ is a groupoid.
\end{example}
\begin{example}\label{ex:partial_map_monics_is_inverse_category}
  Given a category $\cP$, create a partial map category as in
  Section~\ref{sub:partial_map_categories}, where the stable system of monics, $\Mstab$, contains of
  all  isomorphisms in $\cP$. Then the partial isomorphisms are the maps of the form
  \raisebox{12pt}{$\xymatrix @R-15pt @C-15pt{&A'\ar[dl]_{m} \ar[dr]^{m'}\\A&&B}$}, where $m'\in\Mstab$. Its inverse is
  \raisebox{12pt}{$\xymatrix @R-15pt @C-15pt{&A'\ar[dl]_{m'} \ar[dr]^{m}\\B&&A}$}. The composition is
   \raisebox{12pt}{$\xymatrix @R-15pt @C-15pt{&A\ar[dl]_{m^{-1}m} \ar[dr]^{m^{-1}m}\\A&&A}$}, which is the
  identity when $m$ is an isomorphism. The partial isomorphisms will have either $m$ or $m'$ in
  $\Mstab$, but may not be an isomorphism. Taking just the partial isomorphisms gives us an inverse
  category.
\end{example}

\begin{example}\label{ex:inverse-semigroups-are-inverse-categories}
  A semigroup\cite{petrich1984:inverse-semigroups} is a set with an associative binary operation. A
  semigroup need not have an identity. An inverse semigroup is a semigroup where each element $x$
  has an associated element $x^*$ such that:
  \[
    x = x x^* x\ \text{ and }\ x^* = x^* x x^*.
  \]

  In much the same way that a group may be viewed as a one object category, an inverse semigroup is
  a one object inverse category, where the elements are the maps. For any map $x$, we have $\inv{x}
  = x^*$.
\end{example}

\begin{example}[Equivalence relations]\label{ex:equivalence-relations-are-inverse-category}
  Equivalence relations of finite sets are representable as a pair of surjective functions onto
  another set. That is,
  \[
    \raisebox{-15pt}{$A \longsquiggly{E_{[f,g]}} B :=$ }
    \xymatrix{
      &C\\
      A\ar@{->>}[ur]^f && B\ar@{->>}[ul]_{g}
    }
    \raisebox{-15pt}{$\iff$}
    \xymatrix{
      C\\
      A+B \ar@{->>}[u]^{[f,g]}
    }
    % \raisebox{-15pt}{$\iff$}
    % \xymatrix{
    %   E_{[f,g]} \ar@<0.5ex>[d] \ar@<-0.5ex>[d]\\
    %   A+B
    %   }
  \]

  Define the category $\specialcat{EqR}$ with objects being finite sets and maps being equivalence classes of the
  relations $E_{[f,g]}:A\leadsto B$. The equivalence classes are given by the following:
  \[
   \xymatrix{
      &C \ar@{<->}[dd]^{b}\\
      A\ar@{->>}[ur]^f \ar@{->>}[dr]_{f'} && B\ar@{->>}[ul]_{g} \ar@{->>}[dl]^{g'}\\
      &C'
    } \raisebox{-38pt}{$\iff E_{[f,g]} \cong E_{[f',g']}$.}
  \]
  That is, whenever there is a bijection $b:C\leftrightarrow C'$ such that the above diagram
  commutes, then the relations are equivalent.

  The identity is given by $E_{[1,1]}$ and composition is by  pushout. The restriction of a relation
  is given by restricting it to the first element only, i.e., $\rst{E_{[f,g]}} = E_{[f,f]}$. This is
  an inverse category with the partial inverse of a relation given  by swapping the maps $f,g$. That
  is, $\inv{E_{[f,g]}} = E_{[g,f]}$. The total maps are those where $f$ is a bijection.
\end{example}

\begin{example}[Partial isometries]\label{ex:partial-isometries-are-inverse-cats}
  In a finite dimensional Hilbert space, endomaps which are self-adjoint and idempotent are called
  \emph{projectors}. Endomaps in \fdh also form a lattice where the meet of two maps, $f\wedge g$
  is given by $\lim_{n\to\infty} (f g)^n$.

   If we are given a map $f:H_1\to H_2$ between finite dimensional Hilbert spaces,
  then when $f f^{*}:H_1\to H_1$ is a projector, then $f$ is called a \emph{partial isometry}
  \cite{hines2010structure}. In general, partial isometries are not closed under
  composition. However, if composition of $f:H_1 \to H_2$ and $g:H_2 \to H_3$ is defined as:
  \[
      f g = f (f^* f \wedge g g^*) g,
  \]
  then Hilbert spaces with partial isometries form an inverse category. This is shown in Theorem~9.5.4
  of \cite{hines2010structure}. The construction of the composition for this inverse category is an
  application of a general construction for inverse semigroups given by Lawson\cite{lawson1998}.
\end{example}
\section{Inverse categories with restriction products} % (fold)
\label{sec:inverse_categories_with_restriction_products}
We start by showing that an inverse category with restriction products is a restriction preorder
and thus, is a very restrictive notion.
\begin{proposition}\label{prop:an_inverse_category_with_products_is_a_restriction_preorder}
  Given an inverse category \X, if it has restriction products, it is a restriction preorder as in
  Definition~\ref{def:compatible_maps}. That is,
  \[
    \xymatrix {
      A  \ar@<1ex>[r]^{f} \ar@<-1ex>[r]_{g} &B
    }
    \implies f \compatible g.
  \]
\end{proposition}
\begin{proof}
  Notice that for $\pi_1:A\times A \to A$, $\inv{\pi_1}  = \Delta \pi_1 \inv{\pi_1} =\Delta \restr{\pi_1} = \Delta$.
  This gives $\restr{\inv{\pi_1}} = 1$ and therefore $\pi_1$ (and similarly, $\pi_0$) is an
  isomorphism.

  Starting with the product map $\<f,g\>$,
  \[
    \infer={\restr{f}g = \restr{g}f}
    {\infer={\restr{f}g\Delta = \restr{g}f\Delta}
    {\infer={\restr{f}g\inv{\pi_1} = \restr{g}f\inv{\pi_0}}
    {\infer={\<f,g\>\pi_1 \inv{\pi_1} = \<f,g\>\pi_0 \inv{\pi_0}}
    {\<f,g\> = \<f,g\>}}}}
  \]
  which shows that $f$ and $g$ are compatible.
\end{proof}

\begin{corollary}
  \X\ is an inverse category with restriction products iff Total($\spl{}{\X}$) is a meet preorder.
\end{corollary}

\begin{proof}
  Total(\X), the subcategory of total maps on \X, has products and therefore every pair of parallel
  maps is compatible. As total compatible maps are equal, there is at most
  one map between any two objects. Hence, Total(\X) is a preorder with the meet being the product.

  By Lemma~\ref{lem:the_idempotent_splitting_of_an_inverse_category_is_an_inverse_category} and
  Lemma~\ref{lem:inverse_idempotents_are_restriction_idempotents}, $\spl{}{\X}$, the split of \X
  over all idempotents, is an inverse category.

  Similarly, from \cite{cockett2002:restcategories1} and \cite{cockettlack2004:restcategories3},
  Total($\spl{}{\X}$) has products and is therefore also a meet  preorder. This shows the ``only
  if'' side of the corollary.

  For the other direction, if Total($\spl{}{\X}$) is a meet preorder, then the product is the
  meet of the maps and the terminal object is the supremum of all maps.
\end{proof}

\begin{corollary}
  Every inverse category with restriction products is a full subcategory of a partial map category
  of a meet semi-lattice.
\end{corollary}


% section inverse_categories_with_restriction_products (end)
\section{Inverse products} % (fold)
\label{sec:inverse_products}

\subsection{Inverse product definition} % (fold)
\label{sub:inverse_product_definition}

%\begin{definition}\label{def:inverse_product_tensor}
%  Given a restriction category \X, a tensor $\*$ is called an \emph{inverse product tensor}
%   when:
% \begin{itemize}
%   \item $\*$ is a restriction functor, $\_ \* \_ : \X \times \X \to \X$.
%   \item $\*$ is a symmetric monoidal tensor satisfying the standard symmetric monoidal equations and
%     coherence diagrams hold (see, e.g., \cite{maclan97:categorieswrkmath}) and has the following
%     natural isomorphisms:
%     \begin{align*}
%       1 &: \boldsymbol{1}\to \X \\
%       \usl &: 1 \* A \xrightarrow{\cong} A
%       &\usr &: A \* 1 \xrightarrow{\cong} A\\
%       a_{\*} &: (A \* B) \* C \xrightarrow{\cong} A \* (B \* C)
%       &c_{\*} &: A \* B \xrightarrow{\cong} B \* A.
%     \end{align*}
%   \end{itemize}
%   Note that since all the coherence maps are isomorphisms, they are total.
% \end{definition}

\begin{definition}\label{def:inverse_product}
  An \emph{inverse product} on an inverse category \X is given by a symmetric tensor product, based
  on a restriction  bi-functor, $\_\*\_:\X\times\X \to \X$. Recall the structural maps of the tensor
  are the following  natural isomorphisms:
  \begin{align*}
    1 &: \boldsymbol{1}\to \X \\
    \usl &: 1 \* A \xrightarrow{\cong} A
    &\usr &: A \* 1 \xrightarrow{\cong} A\\
    a_{\*} &: (A \* B) \* C \xrightarrow{\cong} A \* (B \* C)
    &c_{\*} &: A \* B \xrightarrow{\cong} B \* A.
  \end{align*}
  The tensor makes \X a  symmetric  monoidal category as per Definition \ref{symmetricmonoidalcat}
  and there is a natural  diagonal map $\Delta$, which is canonical. If an inverse category has
  inverse products, it is called  a \emph{discrete inverse category}.

  The diagonal map $\Delta_A:A \to A\*A$ must be total and create a cosemigroup. It
  must satisfy Diagrams~\ref{dia:delta-is-cocommutative}, \ref{dia:delta-is-coassociative},
  \ref{dia:delta-is-semi-frobenius} and \ref{dia:delta-is-canonical} below.
 % in Figures~\ref{fig:inverse_product_cocommutativity},
 %  \ref{fig:inverse_product_coassociativity}, \ref{fig:inverse_product_exchange}, and
 %  \ref{fig:inverse_product_frobenius}.
  \begin{equation}
    \xymatrix @!0 @C=90pt @R=35pt{
      A \ar[dr]_{\Delta} \ar[r]^{\Delta} &
      A \* A \ar[d]^{c_{\*}}\\
      & A \* A
    }\qquad
    \xymatrix @!0 @C=90pt @R=35pt{
      A \* A \ar[d]^{c_{\*}} \ar[r]^{\inv{\Delta}}  & A\\
      A \* A  \ar[ur]_{\inv{\Delta}}
    }\label{dia:delta-is-cocommutative}
  \end{equation}
  \begin{center}Cocommutative and Commutative;\end{center}
%\caption*{Cocommutative}\label{fig:inverse_product_cocommutativity}

  \begin{equation}
    \xymatrix @C=20pt @R=30pt{
      A \ar[rr]^{\Delta} \ar[d]_{\Delta} & &
      A \* A \ar[d]^{1\*\Delta}\\
      A\*A \ar[dr]_{\Delta \* 1}& &
      A \* ( A \* A) \\
      &   (A \* A) \* A \ar[ur]_{a_{\*}}
    }\ \quad
    \xymatrix @C=20pt @R=30pt{
      A\*(A\*A) \ar[rr]^{1\*\idelta} \ar[d]_{\inv{a_\*}} && A\*A \ar[d]^{\idelta}\\
      (A\*A)\*A \ar[dr]_{\idelta\*1}&&A\\
     & A\*A \ar[ur]_{\idelta}
    }\label{dia:delta-is-coassociative}
  \end{equation}
  \begin{center}Coassociative and Associative;\end{center}
%\caption*{Coassociative and Associative}\label{fig:inverse_product_coassociativity}

%\caption{Exchange}\label{fig:inverse_product_exchange}

  \begin{equation}
    \xymatrix @C=40pt @R=25pt{
      A \* A \ar[dd]_{(1\*\Delta) \inv{a_{\*}}} \ar[dr]^{\inv{\Delta}}
      \ar[rr]^{(\Delta \* 1) a_{\*}} & &
      A \* (A \* A) \ar[dd]^{1 \* \inv{\Delta}}\\
      & A \ar[dr]^{\Delta} & \\
      (A \* A) \* A \ar[rr]_{\inv{\Delta} \* 1} & &
      A \* A
    }\label{dia:delta-is-semi-frobenius}
  \end{equation}
  \begin{center}$\Delta$ is Semi-Frobenius.\end{center}
%\caption{Frobenius}\label{fig:inverse_product_frobenius}
If we define the map:
  \[
    \excs =  a_{\*}(1\*\inv{a_{\*}})(1\*(c_{\*}\*1))(1\* a_{\*})\inv{a_{\*}}): (A\*B)\*(C\*D) \to (A\*C) \* (B\*D)
  \]
then we have:
  \begin{equation}
    \xymatrix @C=25pt @R=35pt{
      A \* B \ar[dr]_{\Delta}
      \ar[rr]^{\Delta \* \Delta} && (A \* A) \* (B \* B) \ar[dl]^{\excs}\\
      &(A \* B) \* (A \* B)
    }\label{dia:delta-is-canonical}
  \end{equation}
  \begin{center}$\Delta$ is canonical.\end{center}

  Thus, $\Delta$ is a cocommutative, coassociative map which together with $\inv{\Delta}$ forms a
  special semi-Frobenius algebra. We use the prefix ``semi-'' since we are not requiring the unit laws
  of a Frobenius algebra as presented in Definition~\ref{def:frobeniusalgebra}.
\end{definition}

Note also, cocommutativity implies commutativity, i.e., that $c_{\*}\inv{\Delta} = \inv{\Delta}$.
One can see this as:
\begin{align*}
  \Delta(c_{\*}\inv{\Delta})
  &= (\Delta c_{\*})\inv{\Delta} = \Delta\inv{\Delta} = \rst{\Delta} \text{ and}\\
  (c_{\*}\inv{\Delta})\Delta
  & = (c_{\*}\inv{\Delta})(\Delta c_{\*}) = \rst{c_{\*}\inv{\Delta}}.
\end{align*}
This means that both $\inv{\Delta}$ and $c_{\*}\inv{\Delta}$ are partial inverses for $\Delta$
and are therefore equal.

Similarly, coassociativity implies associativity, in that
$(1\*\inv{\Delta})\inv{\Delta} =  \inv{a_{\*}}(\inv{\Delta}\* 1)\inv{\Delta}$ as
\begin{align*}
  \Delta(1\*\Delta)\inv{a_{\*}}(\inv{\Delta}\* 1)\inv{\Delta} &=
  \Delta(\Delta\*1)a_{\*}\inv{a_{\*}}(\inv{\Delta}\* 1)\inv{\Delta} \\
  & =\Delta(\Delta\*1)(\inv{\Delta}\* 1)\inv{\Delta} \\
  & =\Delta 1\inv{\Delta} = 1.
\end{align*}

\begin{example}[\pinj is a discrete inverse category]\label{ex:pinj_is_a_discrete_inverse_category}
  In the inverse category \pinj (see Examples~\ref{ex:category_pinj} and
  \ref{ex:pinj_is_an_inverse_category}), we saw in Example~\ref{ex:pinj-has-a-smt} that Cartesian
  product is a symmetric tensor.

  Define $\Delta_A = \{(a,(a,a)) | a\in A\}$. Then \pinj is a discrete inverse category with the
  inverse product of $\*$. The required properties of cocommutativity, coassociativity and
  exchange are immediately obvious. To show the Frobenius rule for $\Delta$, first note that
  $\inv{\Delta}$ is defined only on the elements of $A\*A$ which agree in the first and second
  coordinate. We show the upper triangle of the Frobenius diagram in
  detail. Equation~\ref{eq:delta_inverse_delta} shows the result of applying $\inv{\Delta}$ followed by
  $\Delta$.
  \begin{equation}
    \Delta(\inv{\Delta}(A\*A)) = \Delta(\{a | (a,a) \in A\*A\})
    = \{(a,a) | (a,a) \in A\* A\}.\label{eq:delta_inverse_delta}
  \end{equation}
  Applying $(\Delta \* 1)a_{\*}$ to $A\*A$ is shown in Equation~\ref{eq:delta_tensor_one}.
  \begin{equation}
    a_{\*}(\Delta\*1(A\*A)) = a_{\*}(\{((a,a),a') | (a,a') \in A\*A\} = \{(a,(a,a')) | (a,a') \in
    A\* A\}.\label{eq:delta_tensor_one}
  \end{equation}
  Finally, applying $1 \* \inv{\Delta}$ to the result of Equation~\ref{eq:delta_tensor_one} gives us
  Equation~\ref{eq:one_tensor_delta_inverse}.
  \begin{equation}
    (1\*\inv{\Delta})(\{(a,(a,a')) | (a,a') \in  A\* A\} = \{(a,a) | (a,a) \in A\* A\} \label{eq:one_tensor_delta_inverse}.
  \end{equation}
  Thus, we have $\inv{\Delta}\Delta = (\Delta \* 1) a_{\*} (1\*\inv{\Delta})$ and the Frobenius
  condition is satisfied.
\end{example}

\begin{example}[\topcatp does not give a discrete inverse category]\label{ex:topcat_does_not_give_a_discrete_inverse_category}
  Recalling \topcatp from Example~\ref{ex:restriction_category_top}, we know that the partial
  isomorphisms of \topcatp form an inverse category --- \Inv{\topcatp}. Additionally, \topcatp has a
  product, given by the standard Cartesian product. However, as noted in
  Example~\ref{ex:topcatp-not-discrete}, \topcatp is not a discrete Cartesian restriction category.

  The product of \topcatp does work as a tensor in \Inv{\topcatp}, but $\Delta$  is not a map in
  \Inv{\topcat} and hence we do not have a discrete inverse  category.
\end{example}

Inverse products are extra structure on an inverse category, rather than a property. An example to
demonstrate this is given next.

\begin{example}[Inverse products are additional structure]
  \label{example:invprodisstructure}
\end{example}
Any discrete category (i.e., a category with only the identity arrows) is a trivial inverse
category. To create an inverse product on a discrete category, add a commutative, associative,
idempotent multiplication, with a unit.

Let $\D$ be the discrete category with four objects $a,b,c$ and $d$. Then, define
two different inverse product tensors, $\*$ and $\odot$, with $d$ the unit of each as shown in
Table~\ref{tab:two_different_inverse_products}.

\begin{table}[!htbp]
  \begin{center}
  \begin{tabular}{|l||c|c|c|c|}
    \hline
    $\*$&a&b&c&d\\ \hline \hline
    a&a&a&a&a\\ \hline
    b&a&b&\textbf{b}&b\\ \hline
    c&a&\textbf{b}&c&c \\ \hline
    d&a&b&c&d \\ \hline
  \end{tabular}
  \qquad
  \begin{tabular}{|l||c|c|c|c|} \hline
    $\odot$&a&b&c&d\\ \hline \hline
    a&a&a&a&a\\ \hline
    b&a&b&\textbf{a}&b\\ \hline
    c&a&\textbf{a}&c&c \\ \hline
    d&a&b&c&d \\ \hline
  \end{tabular}
  \end{center}
  \caption{Two different inverse products on the same category.}
  \label{tab:two_different_inverse_products}
\end{table}

As $\D$ is discrete, $\Delta$ is forced to be the identity. One can check easily that each of the
conditions for being an inverse product are satisfied by $\*$ and by $\odot$ with the trivial diagonal.

% subsection inverse_products (end)
\subsection{Diagrammatic Language} % (fold)
\label{sub:diagrammatic_language}

While it is certainly possible to prove results about inverse products using direct algebraic
manipulation, it is much more understandable to use circuits or string diagrams. See
\cite{selinger11:graphical} for a comparison of various graphical languages for monoidal categories.
As shown in
\cite{street-ross-1991-GTC-I}, diagrammatic reasoning is equivalent to reasoning algebraically for
symmetric monoidal categories.

In the diagrams, we will use the following representations:
\begin{itemize}
  \item $\Delta$ will be represented by an upward pointing triangle: \begin{tikzpicture}
      \path node [delta] at (0,0) {}; \end{tikzpicture}.
  \item $\inv{\Delta}$ by a downward triangle: \begin{tikzpicture}
      \path node [nabla] at (0,0) {}; \end{tikzpicture}.
  \item Maps by a rectangle with the name of the map inside: \begin{tikzpicture}
      \path node [map] at (0,0) {$\scriptstyle f$}; \end{tikzpicture}.
  \item Use of the tensor: \begin{tikzpicture}
      \node [style=tensor] (t1) at (0,0) {$\scriptstyle \*$}; \end{tikzpicture}.
  \item Unit introduction (often referred to as an $\eta$ map): \begin{tikzpicture}
      \path node [eta] at (0,0) {}; \end{tikzpicture}.
  \item Unit removal (often referred to as an $\epsilon$ map): \begin{tikzpicture}
      \path node [epsilon] at (0,0) {}; \end{tikzpicture}.
\end{itemize}
String diagrams in this thesis are to be read from top to bottom. Note that unit introduction and
unit removal maps are not required in a discrete inverse category. However, when they are present
they will be represented diagrammatically as above.

The axioms of Definition~\ref{def:inverse_product}, as string diagrams, become:\\
\begin{tikzpicture}
  \begin{pgfonlayer}{nodelayer}
    \node [style=nothing] (0) at (-3.75, 5.5) {};
    \node [style=delta] (1) at (-3.75, 5) {};
    \node [style=nothing] (2) at (-4, 4.25) {};
    \node [style=nothing] (3) at (-3.5, 4.25) {};
    \node [style=nothing] (4) at (-3, 5) {$=$};
    \node [style=nothing] (5) at (-2.25, 5.5) {};
    \node [style=delta] (6) at (-2.25, 5) {};
    \node [style=nothing] (7) at (-2.5, 4.25) {};
    \node [style=nothing] (8) at (-2, 4.25) {};
    \node [style=nothing] (9) at (-3, 4) {Cocommutativity,};
    \end{pgfonlayer}
    \begin{pgfonlayer}{edgelayer}
      \draw [] (0) to (1);
      \draw [] (1) to[out=305,in=90] (2);
      \draw [] (1) to[out=235,in=90] (3);
      \draw [] (5) to (6);
      \draw [] (6) to[out=235,in=90] (7);
      \draw [] (6) to[out=305,in=90] (8);
      \end{pgfonlayer}
\end{tikzpicture}
\hspace{15pt}
\begin{tikzpicture}
  \begin{pgfonlayer}{nodelayer}
    \node [style=nothing] (0) at (-3.75, 5.5) {};
    \node [style=delta] (1) at (-3.75, 5) {};
    \node [style=delta] (2) at (-4, 4.5) {};
    \node [style=nothing] (3a) at (-4.25, 4) {};
    \node [style=nothing] (3b) at (-3.75, 4) {};
    \node [style=nothing] (3c) at (-3.5, 4) {};
    \node [style=nothing] (4) at (-3, 5) {$=$};
    \node [style=nothing] (5) at (-2.25, 5.5) {};
    \node [style=delta] (6) at (-2.25, 5) {};
    \node [style=delta] (8) at (-2, 4.5) {};
    \node [style=nothing] (7a) at (-2.5, 4) {};
    \node [style=nothing] (7b) at (-2.25, 4) {};
    \node [style=nothing] (7c) at (-1.75, 4) {};
    \node [style=nothing] (9) at (-3, 3.5) {coassociativity,};
    \end{pgfonlayer}
    \begin{pgfonlayer}{edgelayer}
      \draw [] (0) to (1);
      \draw [] (1) to[out=305,in=90] (3c);
      \draw [] (1) to[out=235,in=90] (2);
      \draw [] (2) to[out=305,in=90] (3b);
      \draw [] (2) to[out=235,in=90] (3a);
      \draw [] (5) to (6);
      \draw [] (6) to[out=235,in=90] (7a);
      \draw [] (6) to[out=305,in=90] (8);
      \draw [] (8) to[out=305,in=90] (7c);
      \draw [] (8) to[out=235,in=90] (7b);
      \end{pgfonlayer}
\end{tikzpicture}
\hspace{15pt}
\begin{tikzpicture}
  \begin{pgfonlayer}{nodelayer}
    \node [style=nothing] (s0) at (-2.5, 3.75) {};
    \node [style=delta] (d0) at (-2.5, 3.25) {};
    \node [style=nothing] (e01) at (-3, 2.5) {};
    \node [style=nothing] (e02) at (-2, 2.5) {};
    \node [style=nothing] (7) at (-1.75, 3.25) {$=$};
    \node [style=nothing] (s2) at (-.625, 4.25) {};
    \node [style=tensor] (t1) at (-.625, 3.875) {$\scriptstyle \*$};
    \node [style=delta] (d1) at (-1, 3.25) {};
    \node [style=delta] (d2) at (-0.25, 3.25) {};
    \node [style=tensor] (t2) at (-1.125, 2.5) {$\scriptstyle \*$};
    \node [style=nothing] (e2) at (-1.125, 2) {};
    \node [style=tensor] (t3) at (-.125, 2.5) {$\scriptstyle \*$};
    \node [style=nothing] (e3) at (-0.125, 2) {};
    \node [style=nothing] (16) at (-1.5, 1.5) {exchange,};
    \end{pgfonlayer}
    \begin{pgfonlayer}{edgelayer}
      \draw [] (s0) to (d0);
      \draw [] (d0) to[out=235,in=90] (e01);
      \draw [] (d0) to[out=305,in=90] (e02);
      \draw [] (s2) to (t1);
      \draw [] (t1) to[out=235,in=90] (d1);
      \draw [] (t1) to[out=305,in=90] (d2);
      \draw [] (d1) to[out=235,in=90] (t2);
      \draw [] (d1) to[out=305,in=135] (t3);
      \draw [] (d2) to[out=235,in=45] (t2);
      \draw [] (d2) to[out=305,in=90] (t3);
      \draw [] (t2) to (e2);
      \draw [] (t3) to (e3);
      \end{pgfonlayer}
\end{tikzpicture}
\hspace{15pt}
\begin{tikzpicture}
  \begin{pgfonlayer}{nodelayer}
    \node [style=nothing] (0) at (-2.75, 3.75) {};
    \node [style=nothing] (1) at (-2.25, 3.75) {};
    \node [style=delta] (2) at (-2.75, 3.25) {};
    \node [style=nothing] (3) at (-2.75, 2) {};
    \node [style=nothing] (4) at (-2.25, 2) {};
    \node [style=nothing] (5) at (-0.5, 3) {$=$};
    \node [style=nothing] (6) at (0, 3.75) {};
    \node [style=nothing] (7) at (0.5, 3.75) {};
    \node [style=delta] (8) at (-1, 2.5) {};
    \node [style=delta] (9) at (0.5, 3.25) {};
    \node [style=nothing] (10) at (0, 2) {};
    \node [style=nothing] (11) at (0.5, 2) {};
    \node [style=nabla] (12) at (-2.25, 2.5) {};
    \node [style=nabla] (13) at (-1, 3.25) {};
    \node [style=nabla] (14) at (0, 2.5) {};
    \node [style=nothing] (15) at (-1.75, 3) {$=$};
    \node [style=nothing] (16) at (-1.25, 3.75) {};
    \node [style=nothing] (17) at (-0.75, 3.75) {};
    \node [style=nothing] (18) at (-1.25, 2) {};
    \node [style=nothing] (19) at (-0.75, 2) {};
    \node [style=nothing] (20) at (-1, 1.75) {Frobenius.};
    \end{pgfonlayer}
    \begin{pgfonlayer}{edgelayer}
      \draw [] (0) to (2);
      \draw [] (2) to (3);
      \draw [] (7) to (9);
      \draw [] (9) to (11);
      \draw (1) to (12);
      \draw (2) to (12);
      \draw (12) to (4);
      \draw (13) to (8);
      \draw (16) to (13);
      \draw (17) to (13);
      \draw (8) to (18);
      \draw (8) to (19);
      \draw (14) to (10);
      \draw (9) to (14);
      \draw (6) to (14);
      \end{pgfonlayer}
\end{tikzpicture}

The diagram for commutativity is obtained by flipping the diagram of cocommutativity
vertically. Similarly, the diagram for associativity is obtained by flipping the diagram for
coassociativity vertically.
% end sub:diagrammatic_language


\subsection{Properties of discrete inverse categories} % (fold)
\label{sub:discrete_inverse_categories}

We now present some properties of discrete inverse categories.

\begin{lemma}\label{lem:properties_of_delta_and_tensor_in_a_discrete_inverse_category}
  In a discrete inverse category \X with the inverse product $\*$ and $\Delta$, where
  $e=\rst{e}$ is a restriction idempotent and $f,g,h$ are arrows in \X, the following are true:
  \begin{enumerate}[{(}i{)}]
    \item{}$e=\Delta (e\* 1) \inv{\Delta}$;\label{le:eisde1}
    \item{}$e\Delta (f \* g) = \Delta (e f \* g) $ (and $= \Delta (f \* e g) $ and
      $ = \Delta (e f \* e g)$);\label{le:deltaefg}
    \item{}$ (f \* g e) \inv{\Delta} =(f \* g) \inv{\Delta} e $ (and $= (f e\* g) \inv{\Delta}$ and
      $ = (f e\* g e)\inv{\Delta}$);\label{le:efginvdelta}
    \item{}$\restr{\Delta (f \* g) \inv{\Delta}} =
       \Delta(1\* g \inv{f})\inv{\Delta}$; \label{le:restfg}
    \item{} If $\Delta (h \* g) \inv{\Delta} = \restr{\Delta (h \* g) \inv{\Delta}}$ then
      $(\Delta (h \* g) \inv{\Delta}) h = \Delta (h \* g) \inv{\Delta}$;\label{le:hge}
    \item{}$\Delta (f\*1) = \Delta (g\*1) \implies f = g$;\label{le:dfgisfg}
    \item{}$(f\*1) = (g\*1) \implies f = g$.\label{le:fgisfg}
  \end{enumerate}
\end{lemma}
\begin{proof}
  \prepprooflist
  \begin{enumerate}[{(}i{)}]
    \item[\ref{le:eisde1}]
      \[
  \begin{matrix}
      \begin{tikzpicture}
        \node at (0,1) (start) {};
        \node at (0,.5) [map] (e) {$\scriptstyle e$};
        \node at (0,0) (end) {};
        \draw (start) to (e);
        \draw (e) to (end);
      \end{tikzpicture}
  \end{matrix}
  =
  \begin{matrix}
      \begin{tikzpicture}
        \node at (0,3) (start) {};
        \node at (0,2.5) [delta] (d1) {};
        \node at (0,2) [nabla] (n1) {};
        \node at (0,1.5) [delta] (d2) {};
        \node at (0,1) [nabla] (n2) {};
        \node at (0,.5) [map] (e) {$\scriptstyle e$};
        \node at (0,0) (end) {};
        \draw [] (start) to (d1);
        \draw [] (d1) to[out=235,in=125] (n1);
        \draw [] (d1) to[out=305,in=55] (n1);
        \draw [] (n1) to (d2);
        \draw [] (d2) to[out=235,in=125] (n2);
        \draw [] (d2) to[out=305,in=55] (n2);
        \draw [] (n2) to (e);
        \draw (e) to (end);
      \end{tikzpicture}
  \end{matrix}
  =
  \begin{matrix}
      \begin{tikzpicture}
        \node at (0,3) (start) {};
        \node at (0,2.5) [delta] (d1) {};
        \node at (-.25,2) [delta] (d2) {};
        \node at (.25,1.5) [nabla] (n1) {};
        \node at (0,1) [nabla] (n2) {};
        \node at (0,.5) [map] (e) {$\scriptstyle e$};
        \node at (0,0) (end) {};
        \draw [] (start) to (d1);
        \draw [] (d1) to[out=235,in=90] (d2);
        \draw [] (d1) to[out=305,in=55] (n1);
        \draw [] (d2) to[out=305,in=125] (n1);
        \draw [] (d2) to[out=235,in=125] (n2);
        \draw [] (n1) to[out=270,in=55] (n2);
        \draw [] (n2) to (e);
        \draw (e) to (end);
      \end{tikzpicture}
  \end{matrix}
  =
  \begin{matrix}
      \begin{tikzpicture}
        \node at (0,3) (start) {};
        \node at (0,2.5) [delta] (d1) {};
        \node at (-.25,2) [delta] (d2) {};
        \node at (-.5,1.5) [map] (e1) {$\scriptstyle e$};
        \node at (0,1.5) [map] (e2) {$\scriptstyle e$};
        \node at (.5,1.5) [map] (e3) {$\scriptstyle e$};
        \node at (.25,1) [nabla] (n1) {};
        \node at (0,.5) [nabla] (n2) {};
        \node at (0,0) (end) {};
        \draw [] (start) to (d1);
        \draw [] (d1) to[out=235,in=90] (d2);
        \draw [] (d1) to[out=305,in=90] (e3);
        \draw [] (d2) to[out=305,in=90] (e2);
        \draw [] (d2) to[out=235,in=90] (e1);
        \draw (e1) to[out=270,in=125] (n2);
        \draw (e2) to[out=270,in=125] (n1);
        \draw (e3) to[out=270,in=55] (n1);
        \draw [] (n1) to[out=270,in=55] (n2);
        \draw [] (n2) to (end);
      \end{tikzpicture}
  \end{matrix}
  =
  \begin{matrix}
      \begin{tikzpicture}
        \node at (0,3) (start) {};
        \node at (0,2.5) [delta] (d1) {};
        \node at (-.25,2) [map] (e1) {$\scriptstyle e$};
        \node at (-.25,1.5) [delta] (d2) {};
        \node at (.25,1) [nabla] (n1) {};
        \node at (.25,.5) [map] (e2) {$\scriptstyle e$};
        \node at (0,0) [nabla] (n2) {};
        \node at (0,-.5) (end) {};
        \draw [] (start) to (d1);
        \draw [] (d1) to[out=235,in=90] (e1);
        \draw (e1) to (d2);
        \draw [] (d1) to[out=305,in=55] (n1);
        \draw [] (d2) to[out=305,in=125] (n1);
        \draw [] (d2) to[out=235,in=125] (n2);
        \draw (n1) to (e2);
        \draw [] (e2) to[out=270,in=55] (n2);
        \draw [] (n2) to (end);
      \end{tikzpicture}
  \end{matrix}
  =
  \begin{matrix}
      \begin{tikzpicture}
        \node at (0,3) (start) {};
        \node at (0,2.5) [delta] (d1) {};
        \node at (-.25,2) [map] (e1) {$\scriptstyle e$};
        \node at (0,1.5) [nabla] (n1) {};
        \node at (0,1) [delta] (d2) {};
        \node at (.25,.5) [map] (e2) {$\scriptstyle e$};
        \node at (0,0) [nabla] (n2) {};
        \node at (0,-.5) (end) {};
        \draw [] (start) to (d1);
        \draw [] (d1) to[out=235,in=90] (e1);
        \draw [] (d1) to[out=305,in=55] (n1);
        \draw (e1) to[out=270,in=125] (n1);
        \draw [] (n1) to (d2);
        \draw [] (d2) to[out=305,in=90] (e2);
        \draw [] (d2) to[out=235,in=125] (n2);
        \draw (e2) to[out=270,in=55] (n2);
        \draw [] (n2) to (end);
      \end{tikzpicture}
  \end{matrix}
  =
  \begin{matrix}
      \begin{tikzpicture}
        \node at (0,3) (start) {};
        \node at (0,2.5) [delta] (d1) {};
        \node at (-.25,2) [map] (e1) {$\scriptstyle e$};
        \node at (0,1.5) [nabla] (n1) {};
        \node at (0,1) [delta] (d2) {};
        \node at (-.25,.5) [map] (e2) {$\scriptstyle e$};
        \node at (0,0) [nabla] (n2) {};
        \node at (0,-.5) (end) {};
        \draw [] (start) to (d1);
        \draw [] (d1) to[out=235,in=90] (e1);
        \draw [] (d1) to[out=305,in=55] (n1);
        \draw (e1) to[out=270,in=125] (n1);
        \draw [] (n1) to (d2);
        \draw [] (d2) to[out=235,in=90] (e2);
        \draw [] (d2) to[out=305,in=55] (n2);
        \draw (e2) to[out=270,in=125] (n2);
        \draw [] (n2) to (end);
      \end{tikzpicture}
  \end{matrix}
  =
      \]
      \[
  \begin{matrix}
      \begin{tikzpicture}
        \node at (0,3) (start) {};
        \node at (0,2.5) [delta] (d1) {};
        \node at (-.25,2) [map] (e1) {$\scriptstyle e$};
        \node at (-.25,1.5) [delta] (d2) {};
        \node at (.25,1) [nabla] (n1) {};
        \node at (-.25,.5) [map] (e2) {$\scriptstyle e$};
        \node at (0,0) [nabla] (n2) {};
        \node at (0,-.5) (end) {};
        \draw [] (start) to (d1);
        \draw [] (d1) to[out=235,in=90] (e1);
        \draw (e1) to (d2);
        \draw [] (d1) to[out=305,in=55] (n1);
        \draw [] (d2) to[out=305,in=125] (n1);
        \draw [] (d2) to[out=235,in=90] (e2);
        \draw (n1) to[out=270,in=55] (n2);
        \draw [] (e2) to[out=270,in=125] (n2);
        \draw [] (n2) to (end);
      \end{tikzpicture}
  \end{matrix}
  =
  \begin{matrix}
      \begin{tikzpicture}
        \node at (0,2.5) (start) {};
        \node at (0,2) [delta] (d1) {};
        \node at (-.25,1.5) [delta] (d2) {};
        \node at (-.5,1) [map] (e2) {$\scriptstyle e$};
        \node at (0,1) [map] (e1) {$\scriptstyle e$};
        \node at (.25,.5) [nabla] (n1) {};
        \node at (-.25,0) [map] (e3) {$\scriptstyle e$};
        \node at (0,-0.5) [nabla] (n2) {};
        \node at (0,-1) (end) {};
        \draw [] (start) to (d1);
        \draw [] (d1) to[out=235,in=90] (d2);
        \draw [] (d1) to[out=305,in=55] (n1);
        \draw [] (d2) to[out=305,in=90] (e1);
        \draw [] (d2) to[out=235,in=90] (e2);
        \draw (e1) to[out=270,in=125] (n1);
        \draw (e2) to (e3);
        \draw (n1) to[out=270,in=55] (n2);
        \draw [] (e3) to[out=270,in=125] (n2);
        \draw [] (n2) to (end);
      \end{tikzpicture}
  \end{matrix}
  =
  \begin{matrix}
      \begin{tikzpicture}
        \node at (0,3) (start) {};
        \node at (0,2.5) [delta] (d1) {};
        \node at (-.25,2) [map] (e1) {$\scriptstyle e$};
        \node at (-.25,1.5) [delta] (d2) {};
        \node at (.25,1) [nabla] (n1) {};
        \node at (0,.5) [nabla] (n2) {};
        \node at (0,0) (end) {};
        \draw [] (start) to (d1);
        \draw [] (d1) to[out=235,in=90] (e1);
        \draw [] (d1) to[out=305,in=55] (n1);
        \draw (e1) to[out=270,in=90] (d2);
        \draw [] (d2) to[out=235,in=125] (n2);
        \draw [] (d2) to[out=305,in=125] (n1);
        \draw (n1) to[out=270,in=55] (n2);
        \draw [] (n2) to (end);
      \end{tikzpicture}
  \end{matrix}
  =
  \begin{matrix}
      \begin{tikzpicture}
        \node at (0,3) (start) {};
        \node at (0,2.5) [delta] (d1) {};
        \node at (-.25,2) [map] (e1) {$\scriptstyle e$};
        \node at (0,1.5) [nabla] (n1) {};
        \node at (0,1) [delta] (d2) {};
        \node at (0,.5) [nabla] (n2) {};
        \node at (0,0) (end) {};
        \draw [] (start) to (d1);
        \draw [] (d1) to[out=235,in=90] (e1);
        \draw [] (d1) to[out=305,in=55] (n1);
        \draw (e1) to[out=270,in=125] (n1);
        \draw [] (n1) to (d2);
        \draw [] (d2) to[out=235,in=125] (n2);
        \draw [] (d2) to[out=305,in=55] (n2);
        \draw [] (n2) to (end);
      \end{tikzpicture}
  \end{matrix}
  =
  \begin{matrix}
        \begin{tikzpicture}
        \node at (0,3) (start) {};
        \node at (0,2.5) [delta] (d1) {};
        \node at (-.25,2) [map] (e1) {$\scriptstyle e$};
        \node at (0,1.5) [nabla] (n1) {};
        \node at (0,1) (end) {};
        \draw [] (start) to (d1);
        \draw [] (d1) to[out=235,in=90] (e1);
        \draw [] (d1) to[out=305,in=55] (n1);
        \draw (e1) to[out=270,in=125] (n1);
        \draw [] (n1) to (end);
      \end{tikzpicture}
  \end{matrix}.
      \]
    \item[\ref{le:deltaefg}]This equality uses the previous equality, the commutativity
      of restriction idempotents (\rtwo) and the identity $\Delta\rst{\inv{\Delta}} = \Delta$.
      \[
  \begin{matrix}
      \begin{tikzpicture}
        \node at (0,2) (start) {};
        \node at (0,1.5) [map] (e) {$\scriptstyle e$};
        \node at (0,1) [delta] (d2) {};
        \node at (-.25,.5) [map] (f) {$\scriptstyle f$};
        \node at (.25,.5) [map] (g) {$\scriptstyle g$};
        \node at (-.25,0) (end1) {};
        \node at (.25,0) (end2) {};
        \draw (start) to (e);
        \draw (e) to (d2);
        \draw (d2) to[out=235,in=90] (f);
        \draw (d2) to[out=305,in=90] (g);
        \draw (f) to (end1);
        \draw (g) to (end2);
      \end{tikzpicture}
  \end{matrix}
  =
  \begin{matrix}
        \begin{tikzpicture}
        \node at (0,3) (start) {};
        \node at (0,2.5) [delta] (d1) {};
        \node at (-.25,2) [map] (e1) {$\scriptstyle e$};
        \node at (0,1.5) [nabla] (n1) {};
        \node at (0,1) [delta] (d2) {};
        \node at (-.25,.5) [map] (f) {$\scriptstyle f$};
        \node at (.25,.5) [map] (g) {$\scriptstyle g$};
        \node at (-.25,0) (end1) {};
        \node at (.25,0) (end2) {};
        \draw [] (start) to (d1);
        \draw [] (d1) to[out=235,in=90] (e1);
        \draw [] (d1) to[out=305,in=55] (n1);
        \draw (e1) to[out=270,in=125] (n1);
        \draw [] (n1) to (d2);
        \draw (d2) to[out=235,in=90] (f);
        \draw (d2) to[out=305,in=90] (g);
        \draw (f) to (end1);
        \draw (g) to (end2);
      \end{tikzpicture}
  \end{matrix}
  =
  \begin{matrix}
        \begin{tikzpicture}
        \node at (0,3) (start) {};
        \node at (0,2.5) [delta] (d1) {};
        \node at (-.25,2) [map] (e1) {$\scriptstyle e$};
        \node at (0,1.25) [map] (inverse-delta-r) {$\scriptstyle \rst{\inv{\Delta}}$};
        \node at (-.25,.5) [map] (f) {$\scriptstyle f$};
        \node at (.25,.5) [map] (g) {$\scriptstyle g$};
        \node at (-.25,0) (end1) {};
        \node at (.25,0) (end2) {};
        \draw [] (start) to (d1);
        \draw [] (d1) to[out=235,in=90] (e1);
        \draw [] (d1) to[out=305,in=55] (inverse-delta-r);
        \draw (e1) to[out=270,in=125] (inverse-delta-r);
        \draw (inverse-delta-r) to[out=235,in=90] (f);
        \draw (inverse-delta-r) to[out=305,in=90] (g);
        \draw (f) to (end1);
        \draw (g) to (end2);
      \end{tikzpicture}
  \end{matrix}
  =
  \begin{matrix}
        \begin{tikzpicture}
        \node at (0,3) (start) {};
        \node at (0,2.5) [delta] (d1) {};
        \node at (0,2) [map] (inverse-delta-r) {$\scriptstyle \rst{\inv{\Delta}}$};
        \node at (-.25,1.25) [map] (e1) {$\scriptstyle e$};
        \node at (-.25,.5) [map] (f) {$\scriptstyle f$};
        \node at (.25,.5) [map] (g) {$\scriptstyle g$};
        \node at (-.25,0) (end1) {};
        \node at (.25,0) (end2) {};
        \draw [] (start) to (d1);
        \draw [] (d1) to[out=235,in=125] (inverse-delta-r);
        \draw [] (d1) to[out=305,in=55] (inverse-delta-r);
        \draw (inverse-delta-r) to[out=235,in=90] (e1);
        \draw (e1) to (f);
        \draw (inverse-delta-r) to[out=305,in=90] (g);
        \draw (f) to (end1);
        \draw (g) to (end2);
      \end{tikzpicture}
  \end{matrix}
  =
  \begin{matrix}
        \begin{tikzpicture}
        \node at (0,1.5) (start) {};
        \node at (0,1) [delta] (d1) {};
        \node at (-.25,.5) [map] (f_e) {$\scriptstyle e f$};
        \node at (.25,.5) [map] (g) {$\scriptstyle g$};
        \node at (-.25,0) (end1) {};
        \node at (.25,0) (end2) {};
        \draw [] (start) to (d1);
        \draw [] (d1) to[out=235,in=90] (f_e);
        \draw [] (d1) to[out=305,in=90] (g);
        \draw (f_e) to (end1);
        \draw (g) to (end2);
      \end{tikzpicture}
  \end{matrix} .
      \]
      The second equality ($e\Delta(f\*g) = \Delta(f\*e g)$) follows by cocommutativity. The third
      equality,  ($e\Delta(f\*g) = \Delta(e f\*e g)$) follows by naturality of $\Delta$.

    \item[\ref{le:efginvdelta}] As in \ref{le:deltaefg}, details are only given for the
      first equality. This proof is obtained by reversing the diagrams of \ref{le:deltaefg}.
      \[
  \begin{matrix}
      \begin{tikzpicture}
        \node at (0,0) (start) {};
        \node at (0,.5) [map] (e) {$\scriptstyle e$};
        \node at (0,1) [nabla] (d2) {};
        \node at (-.25,1.5) [map] (f) {$\scriptstyle f$};
        \node at (.25,1.5) [map] (g) {$\scriptstyle g$};
        \node at (-.25,2) (end1) {};
        \node at (.25,2) (end2) {};
        \draw (start) to (e);
        \draw (e) to (d2);
        \draw (d2) to[out=125,in=270] (f);
        \draw (d2) to[out=55,in=270] (g);
        \draw (f) to (end1);
        \draw (g) to (end2);
      \end{tikzpicture}
  \end{matrix}
  =
  \begin{matrix}
        \begin{tikzpicture}
        \node at (0,0) (start) {};
        \node at (0,.5) [nabla] (d1) {};
        \node at (-.25,1) [map] (e1) {$\scriptstyle e$};
        \node at (0,1.5) [delta] (n1) {};
        \node at (0,2) [nabla] (d2) {};
        \node at (-.25,2.5) [map] (f) {$\scriptstyle f$};
        \node at (.25,2.5) [map] (g) {$\scriptstyle g$};
        \node at (-.25,3) (end1) {};
        \node at (.25,3) (end2) {};
        \draw [] (start) to (d1);
        \draw [] (d1) to[out=125,in=270] (e1);
        \draw [] (d1) to[out=55,in=305] (n1);
        \draw (e1) to[out=90,in=235] (n1);
        \draw [] (n1) to (d2);
        \draw (d2) to[out=125,in=270] (f);
        \draw (d2) to[out=55,in=270] (g);
        \draw (f) to (end1);
        \draw (g) to (end2);
      \end{tikzpicture}
  \end{matrix}
  =
  \begin{matrix}
        \begin{tikzpicture}
        \node at (0,0) (start) {};
        \node at (0,.5) [nabla] (d1) {};
        \node at (-.25,1) [map] (e1) {$\scriptstyle e$};
        \node at (0,1.75) [map] (inverse-delta-r) {$\scriptstyle \rst{\inv{\Delta}}$};
        \node at (-.25,2.5) [map] (f) {$\scriptstyle f$};
        \node at (.25,2.5) [map] (g) {$\scriptstyle g$};
        \node at (-.25,3) (end1) {};
        \node at (.25,3) (end2) {};
        \draw [] (start) to (d1);
        \draw [] (d1) to[out=125,in=270] (e1);
        \draw [] (d1) to[out=55,in=305] (inverse-delta-r);
        \draw (e1) to[out=90,in=235] (inverse-delta-r);
        \draw (inverse-delta-r) to[out=125,in=270] (f);
        \draw (inverse-delta-r) to[out=55,in=270] (g);
        \draw (f) to (end1);
        \draw (g) to (end2);
      \end{tikzpicture}
  \end{matrix}
  =
  \begin{matrix}
        \begin{tikzpicture}
        \node at (0,0) (start) {};
        \node at (0,.5) [nabla] (d1) {};
        \node at (-.25,1) [map] (f e) {$\scriptstyle f e$};
        \node at (.25,1) [map] (g) {$\scriptstyle g$};
        \node at (-.25,1.5) (end1) {};
        \node at (.25,1.5) (end2) {};
        \draw [] (start) to (d1);
        \draw [] (d1) to[out=125,in=270] (f e);
        \draw [] (d1) to[out=55,in=270] (g);
        \draw (f e) to (end1);
        \draw (g) to (end2);
      \end{tikzpicture}
  \end{matrix} .
      \]
      The other equalities follow for the same reasons as in \ref{le:deltaefg}.

    \item[\ref{le:restfg}]Here, we start by using the fact that all maps have a partial inverse,
      therefore we have:
      \[
        \restr{\Delta (f \* g) \inv{\Delta} } =\Delta (f \* g) \inv{\Delta} \Delta (\inv{f} \*
        \inv{g}) \inv{\Delta}.
     \]
     Now, we proceed with showing the rest of the equality via diagrams.

     \[
  \begin{matrix}
        \begin{tikzpicture}
        \node at (0,3.5) (start) {};
        \node at (0,3) [delta] (d1) {};
        \node at (-.25,2.5) [map] (f) {$\scriptstyle f$};
        \node at (.25,2.5) [map] (g) {$\scriptstyle g$};
        \node at (0, 2) [nabla] (n1) {};
        \node at (0,1.5) [delta] (d2) {};
        \node at (-.5,1) [map] (f-inverse) {$\scriptstyle \inv{f}$};
        \node at (.5,1) [map] (g-inverse) {$\scriptstyle \inv{g}$};
        \node at (0, .5) [nabla] (n2) {};
        \node at (0,0) (end) {};
        \draw [] (start) to (d1);
        \draw [] (d1) to[out=235,in=90] (f);
        \draw [] (d1) to[out=305,in=90] (g);
        \draw (f) to[out=270,in=125] (n1);
        \draw (g) to[out=270,in=55] (n1);
        \draw (n1) to (d2);
        \draw [] (d2) to[out=235,in=90] (f-inverse);
        \draw [] (d2) to[out=305,in=90] (g-inverse);
        \draw (f-inverse) to[out=270,in=125] (n2);
        \draw (g-inverse) to[out=270,in=55] (n2);
        \draw (n2) to (end);
      \end{tikzpicture}
  \end{matrix}
  =
  \begin{matrix}
      \begin{tikzpicture}
        \node at (0,3.5) (start) {};
        \node at (0,3) [delta] (d1) {};
        \node at (.25,2.5) [map] (f) {$\scriptstyle f$};
        \node at (-.25,2.5) [map] (g) {$\scriptstyle g$};
        \node at (0, 2) [nabla] (n1) {};
        \node at (0,1.5) [delta] (d2) {};
        \node at (.5,1) [map] (f-inverse) {$\scriptstyle \inv{f}$};
        \node at (-.5,1) [map] (g-inverse) {$\scriptstyle \inv{g}$};
        \node at (0, .5) [nabla] (n2) {};
        \node at (0,0) (end) {};
        \draw [] (start) to (d1);
        \draw [] (d1) to[out=235,in=90] (g);
        \draw [] (d1) to[out=305,in=90] (f);
        \draw (g) to[out=270,in=125] (n1);
        \draw (f) to[out=270,in=55] (n1);
        \draw (n1) to (d2);
        \draw [] (d2) to[out=235,in=90] (g-inverse);
        \draw [] (d2) to[out=305,in=90] (f-inverse);
        \draw (g-inverse) to[out=270,in=125] (n2);
        \draw (f-inverse) to[out=270,in=55] (n2);
        \draw (n2) to (end);
      \end{tikzpicture}
  \end{matrix}
  =
  \begin{matrix}
      \begin{tikzpicture}
        \node at (0,3.5) (start) {};
        \node at (0,3) [delta] (d1) {};
        \node at (.25,2.5) [map] (f) {$\scriptstyle f$};
        \node at (-.25,2.5) [map] (g) {$\scriptstyle g$};
        \node at (-.25,2) [delta] (d2) {};
        \node at (.25, 1.5) [nabla] (n1) {};
        \node at (.5,1) [map] (f-inverse) {$\scriptstyle \inv{f}$};
        \node at (-.5,1) [map] (g-inverse) {$\scriptstyle \inv{g}$};
        \node at (0, .5) [nabla] (n2) {};
        \node at (0,0) (end) {};
        \draw [] (start) to (d1);
        \draw [] (d1) to[out=235,in=90] (g);
        \draw [] (d1) to[out=305,in=90] (f);
        \draw (g) to (d2);
        \draw (d2) to[out=305,in=125] (n1);
        \draw (f) to[out=270,in=55] (n1);
        \draw (n1) to (f-inverse);
        \draw [] (d2) to[out=235,in=90] (g-inverse);
        \draw (g-inverse) to[out=270,in=125] (n2);
        \draw (f-inverse) to[out=270,in=55] (n2);
        \draw (n2) to (end);
      \end{tikzpicture}
  \end{matrix}
  =
  \begin{matrix}
      \begin{tikzpicture}
        \node at (0,3.5) (start) {};
        \node at (0,3) [delta] (d1) {};
        \node at (-.25,2.5) [delta] (d2) {};
        \node at (-.5,2) [map] (rest-g) {$\scriptstyle \rst{g}$};
        \node at (0,2) [map] (g) {$\scriptstyle g$};
        \node at (.75,1.5) [map] (rest-f) {$\scriptstyle \rst{f}$};
        \node at (0.1,1.5) [map] (f-inverse) {$\scriptstyle \inv{f}$};
        \node at (.25, 1) [nabla] (n1) {};
        \node at (0, .5) [nabla] (n2) {};
        \node at (0,0) (end) {};
        \draw [] (start) to (d1);
        \draw [] (d1) to[out=235,in=90] (d2);
        \draw [] (d1) to[out=305,in=90] (rest-f);
        \draw (d2) to[out=235,in=90] (rest-g);
        \draw (d2) to[out=305,in=90] (g);
        \draw (rest-g) to[out=270,in=125] (n2);
v        \draw (g) to (f-inverse);
        \draw (f-inverse) to[out=270,in=125] (n1);
        \draw (rest-f) to[out=270,in=55] (n1);
        \draw (n1) to[out=270,in=55] (n2);
        \draw (n2) to (end);
      \end{tikzpicture}
  \end{matrix}
  =
     \]
     \[
  \begin{matrix}
      \begin{tikzpicture}
        \node at (0,3.5) (start) {};
        \node at (0,3) [delta] (d1) {};
        \node at (-.35,2.5) [delta] (d2) {};
        \node at (0,1.75) [map] (g-f-inverse) {$\scriptstyle g \inv{f}$};
        \node at (.35, 1) [nabla] (n1) {};
        \node at (0, .5) [nabla] (n2) {};
        \node at (0,0) (end) {};
        \draw [] (start) to (d1);
        \draw [] (d1) to[out=235,in=90] (d2);
        \draw [] (d1) to[out=305,in=55] (n1);
        \draw (d2) to[out=235,in=125] (n2);
        \draw (d2) to[out=305,in=90] (g-f-inverse);
        \draw (g-f-inverse) to[out=270,in=125] (n1);
        \draw (n1) to[out=270,in=55] (n2);
        \draw (n2) to (end);
      \end{tikzpicture}
  \end{matrix}
  =
  \begin{matrix}
      \begin{tikzpicture}
        \node at (0,3.5) (start) {};
        \node at (0,3) [delta] (d1) {};
        \node at (-.35,2.5) [delta] (d2) {};
        \node at (.35, 2) [nabla] (n1) {};
        \node at (-.35,1.25) [map] (g-f-inverse) {$\scriptstyle g \inv{f}$};
        \node at (0, .5) [nabla] (n2) {};
        \node at (0,0) (end) {};
        \draw [] (start) to (d1);
        \draw [] (d1) to[out=235,in=90] (d2);
        \draw [] (d1) to[out=305,in=55] (n1);
        \draw (d2) to[out=305,in=125] (n1);
        \draw (d2) to[out=235,in=90] (g-f-inverse);
        \draw (g-f-inverse) to[out=270,in=125] (n2);
        \draw (n1) to[out=270,in=55] (n2);
        \draw (n2) to (end);
      \end{tikzpicture}
  \end{matrix}
  =
  \begin{matrix}
        \begin{tikzpicture}
        \node at (0,3) (start) {};
        \node at (0,2.5) [delta] (d1) {};
        \node at (0, 2) [nabla] (n1) {};
        \node at (0,1.5) [delta] (d2) {};
        \node at (-.5,1) [map] (g-f-inverse) {$\scriptstyle g\inv{f}$};
        \node at (0, .5) [nabla] (n2) {};
        \node at (0,0) (end) {};
        \draw [] (start) to (d1);
        \draw [] (d1) to[out=235,in=125] (n1);
        \draw [] (d1) to[out=305,in=55] (n1);
        \draw (n1) to (d2);
        \draw [] (d2) to[out=235,in=90] (g-f-inverse);
        \draw [] (d2) to[out=305,in=55] (n2);
        \draw (g-f-inverse) to[out=270,in=125] (n2);
        \draw (n2) to (end);
      \end{tikzpicture}
  \end{matrix}
  =
  \begin{matrix}
       \begin{tikzpicture}
        \node at (0,2) (start) {};
        \node at (0,1.5) [delta] (d2) {};
        \node at (.5,1) [map] (g-f-inverse) {$\scriptstyle g\inv{f}$};
        \node at (0, .5) [nabla] (n2) {};
        \node at (0,0) (end) {};
        \draw [] (start) to (d2);
        \draw [] (d2) to[out=305,in=90] (g-f-inverse);
        \draw [] (d2) to[out=235,in=125] (n2);
        \draw (g-f-inverse) to[out=270,in=55] (n2);
        \draw (n2) to (end);
      \end{tikzpicture}
  \end{matrix}.
     \]
    \item[\ref{le:hge}]Beginning with the assumption that $\Delta (h \* g)\inv{\Delta}$ equals its
      restriction and by item \ref{le:restfg}, we have:
      \[
  \begin{matrix}
        \begin{tikzpicture}
        \node at (0,3.5) (start) {};
        \node at (0,3) [delta] (d1) {};
        \node at (-.25,2.5) [map] (h1) {$\scriptstyle h$};
        \node at (.25,2.5) [map] (g) {$\scriptstyle g$};
        \node at (0, 2) [nabla] (n1) {};
        \node at (0,1.5) [map] (h) {$\scriptstyle h$};
        \node at (0,1) (end) {};
        \draw [] (start) to (d1);
        \draw [] (d1) to[out=235,in=90] (h1);
        \draw [] (d1) to[out=305,in=90] (g);
        \draw (h1) to[out=270,in=125] (n1);
        \draw (g) to[out=270,in=55] (n1);
        \draw (n1) to (h);
        \draw (h) to (end);
      \end{tikzpicture}
  \end{matrix}
  =
  \begin{matrix}
        \begin{tikzpicture}
        \node at (0,3.5) (start) {};
        \node at (0,3) [delta] (d1) {};
        \node at (.5,2.5) [map] (g-h-inverse) {$\scriptstyle g\inv{h}$};
        \node at (0, 2) [nabla] (n1) {};
        \node at (0,1.5) [map] (h) {$\scriptstyle h$};
        \node at (0,1) (end) {};
        \draw [] (start) to (d1);
        \draw [] (d1) to[out=235,in=125] (n1);
        \draw [] (d1) to[out=305,in=90] (g-h-inverse);
        \draw (g-h-inverse) to[out=270,in=55] (n1);
        \draw (n1) to (h);
        \draw (h) to (end);
      \end{tikzpicture}
  \end{matrix}
  =
  \begin{matrix}
        \begin{tikzpicture}
        \node at (0,3.5) (start) {};
        \node at (0,3) [delta] (d1) {};
        \node at (.5,2.25) [map] (g-h-inverse) {$\scriptstyle g\inv{h}$};
        \node at (-.25,1.5) [map] (h1) {$\scriptstyle h$};
        \node at (.25,1.5) [map] (h2) {$\scriptstyle h$};
        \node at (0, 1) [nabla] (n1) {};
        \node at (0,.5) (end) {};
        \draw [] (start) to (d1);
        \draw [] (d1) to[out=235,in=90] (h1);
        \draw [] (d1) to[out=305,in=90] (g-h-inverse);
        \draw (g-h-inverse) to[out=270,in=90] (h2);
        \draw (h1) to[out=270,in=125] (n1);
        \draw (h2) to[out=270,in=55] (n1);
        \draw (n1) to (end);
      \end{tikzpicture}
  \end{matrix}
  =
  \begin{matrix}
        \begin{tikzpicture}
        \node at (0,3.5) (start) {};
        \node at (0,3) [delta] (d1) {};
        \node at (.5,2.25) [map] (g-h-inverse) {$\scriptstyle g\rst{\inv{h}}$};
        \node at (-.25,1.5) [map] (h1) {$\scriptstyle h$};
        \node at (0, 1) [nabla] (n1) {};
        \node at (0,.5) (end) {};
        \draw [] (start) to (d1);
        \draw [] (d1) to[out=235,in=90] (h1);
        \draw [] (d1) to[out=305,in=90] (g-h-inverse);
        \draw (h1) to[out=270,in=125] (n1);
        \draw (g-h-inverse) to[out=270,in=55] (n1);
        \draw (n1) to (end);
      \end{tikzpicture}
  \end{matrix}
  =
  \begin{matrix}
        \begin{tikzpicture}
        \node at (0,3.5) (start) {};
        \node at (0,3) [delta] (d1) {};
        \node at (.25,2.25) [map] (g) {$\scriptstyle g$};
        \node at (-.5,1.5) [map] (h1) {$\scriptstyle h\rst{\inv{h}}$};
        \node at (0, 1) [nabla] (n1) {};
        \node at (0,.5) (end) {};
        \draw [] (start) to (d1);
        \draw [] (d1) to[out=235,in=90] (h1);
        \draw [] (d1) to[out=305,in=90] (g);
        \draw (h1) to[out=270,in=125] (n1);
        \draw (g) to[out=270,in=55] (n1);
        \draw (n1) to (end);
      \end{tikzpicture}
  \end{matrix}
  =
  \begin{matrix}
        \begin{tikzpicture}
        \node at (0,3.5) (start) {};
        \node at (0,3) [delta] (d1) {};
        \node at (.25,2.5) [map] (g) {$\scriptstyle g$};
        \node at (-.25,2.5) [map] (h1) {$\scriptstyle h$};
        \node at (0, 2) [nabla] (n1) {};
        \node at (0,1.5) (end) {};
        \draw [] (start) to (d1);
        \draw [] (d1) to[out=235,in=90] (h1);
        \draw [] (d1) to[out=305,in=90] (g);
        \draw (h1) to[out=270,in=125] (n1);
        \draw (g) to[out=270,in=55] (n1);
        \draw (n1) to (end);
      \end{tikzpicture}
  \end{matrix}
  =
  \begin{matrix}
        \begin{tikzpicture}
        \node at (0,3.5) (start) {};
        \node at (0,3) [delta] (d1) {};
        \node at (-.25,2.5) [map] (g) {$\scriptstyle g$};
        \node at (.25,2.5) [map] (h1) {$\scriptstyle h$};
        \node at (0, 2) [nabla] (n1) {};
        \node at (0,1.5) (end) {};
        \draw [] (start) to (d1);
        \draw [] (d1) to[out=305,in=90] (h1);
        \draw [] (d1) to[out=235,in=90] (g);
        \draw (h1) to[out=270,in=55] (n1);
        \draw (g) to[out=270,in=125] (n1);
        \draw (n1) to (end);
      \end{tikzpicture}
    \end{matrix}.
      \]

    \item[\ref{le:dfgisfg}] Our assumption is that:
      \[
  \begin{matrix}
        \begin{tikzpicture}
        \node at (0,1.5) (start) {};
        \node at (0,1) [delta] (d1) {};
        \node at (-.25,.5) [map] (g) {$\scriptstyle g$};
        \node at (-.25, 0) (end1) {};
        \node at (.25,0) (end2) {};
        \draw [] (start) to (d1);
        \draw [] (d1) to[out=305,in=90] (end2);
        \draw [] (d1) to[out=235,in=90] (g);
        \draw (g) to (end1);
      \end{tikzpicture}
  \end{matrix}
  =
  \begin{matrix}
        \begin{tikzpicture}
        \node at (0,1.5) (start) {};
        \node at (0,1) [delta] (d1) {};
        \node at (-.25,.5) [map] (f) {$\scriptstyle f$};
        \node at (-.25, 0) (end1) {};
        \node at (.25,0) (end2) {};
        \draw [] (start) to (d1);
        \draw [] (d1) to[out=305,in=90] (end2);
        \draw [] (d1) to[out=235,in=90] (f);
        \draw (f) to (end1);
      \end{tikzpicture}
  \end{matrix}
      \text{ and by cocommutativity, }
  \begin{matrix}
        \begin{tikzpicture}
        \node at (0,1.5) (start) {};
        \node at (0,1) [delta] (d1) {};
        \node at (.25,.5) [map] (g) {$\scriptstyle g$};
        \node at (.25, 0) (end1) {};
        \node at (-.25,0) (end2) {};
        \draw [] (start) to (d1);
        \draw [] (d1) to[out=235,in=90] (end2);
        \draw [] (d1) to[out=305,in=90] (g);
        \draw (g) to (end1);
      \end{tikzpicture}
  \end{matrix}
  =
  \begin{matrix}
        \begin{tikzpicture}
        \node at (0,1.5) (start) {};
        \node at (0,1) [delta] (d1) {};
        \node at (.25,.5) [map] (f) {$\scriptstyle f$};
        \node at (.25, 0) (end1) {};
        \node at (-.25,0) (end2) {};
        \draw [] (start) to (d1);
        \draw [] (d1) to[out=235,in=90] (end2);
        \draw [] (d1) to[out=305,in=90] (f);
        \draw (f) to (end1);
      \end{tikzpicture}
  \end{matrix}.
      \]
      Hence,
      \[
  \begin{matrix}
        \begin{tikzpicture}
        \node at (0,3.5) (start) {};
        \node at (0,3) [delta] (d1) {};
        \node at (-.25,2.5) [map] (f1) {$\scriptstyle f$};
        \node at (.25,2.5) [map] (f2) {$\scriptstyle f$};
        \node at (0, 2) [nabla] (n1) {};
        \node at (0,1.5) (end) {};
        \draw [] (start) to (d1);
        \draw [] (d1) to[out=305,in=90] (f2);
        \draw [] (d1) to[out=235,in=90] (f1);
        \draw (f2) to[out=270,in=55] (n1);
        \draw (f1) to[out=270,in=125] (n1);
        \draw (n1) to (end);
      \end{tikzpicture}
  \end{matrix}
  =
  \begin{matrix}
        \begin{tikzpicture}
        \node at (0,3.5) (start) {};
        \node at (0,3) [delta] (d1) {};
        \node at (-.25,2.5) [map] (f1) {$\scriptstyle f$};
        \node at (.25,2) [map] (f2) {$\scriptstyle f$};
        \node at (0, 1.5) [nabla] (n1) {};
        \node at (0,1) (end) {};
        \draw [] (start) to (d1);
        \draw [] (d1) to[out=305,in=90] (f2);
        \draw [] (d1) to[out=235,in=90] (f1);
        \draw (f2) to[out=270,in=55] (n1);
        \draw (f1) to[out=270,in=125] (n1);
        \draw (n1) to (end);
      \end{tikzpicture}
  \end{matrix}
  =
  \begin{matrix}
        \begin{tikzpicture}
        \node at (0,3.5) (start) {};
        \node at (0,3) [delta] (d1) {};
        \node at (-.25,2.5) [map] (g1) {$\scriptstyle g$};
        \node at (.25,2) [map] (f2) {$\scriptstyle f$};
        \node at (0, 1.5) [nabla] (n1) {};
        \node at (0,1) (end) {};
        \draw [] (start) to (d1);
        \draw [] (d1) to[out=305,in=90] (f2);
        \draw [] (d1) to[out=235,in=90] (g1);
        \draw (f2) to[out=270,in=55] (n1);
        \draw (g1) to[out=270,in=125] (n1);
        \draw (n1) to (end);
      \end{tikzpicture}
  \end{matrix}
  =
  \begin{matrix}
        \begin{tikzpicture}
        \node at (0,3.5) (start) {};
        \node at (0,3) [delta] (d1) {};
        \node at (-.25,2) [map] (g1) {$\scriptstyle g$};
        \node at (.25,2.5) [map] (f2) {$\scriptstyle f$};
        \node at (0, 1.5) [nabla] (n1) {};
        \node at (0,1) (end) {};
        \draw [] (start) to (d1);
        \draw [] (d1) to[out=305,in=90] (f2);
        \draw [] (d1) to[out=235,in=90] (g1);
        \draw (f2) to[out=270,in=55] (n1);
        \draw (g1) to[out=270,in=125] (n1);
        \draw (n1) to (end);
      \end{tikzpicture}
  \end{matrix}
  =
  \begin{matrix}
        \begin{tikzpicture}
        \node at (0,3.5) (start) {};
        \node at (0,3) [delta] (d1) {};
        \node at (-.25,2) [map] (g1) {$\scriptstyle g$};
        \node at (.25,2.5) [map] (g2) {$\scriptstyle g$};
        \node at (0, 1.5) [nabla] (n1) {};
        \node at (0,1) (end) {};
        \draw [] (start) to (d1);
        \draw [] (d1) to[out=305,in=90] (g2);
        \draw [] (d1) to[out=235,in=90] (g1);
        \draw (g2) to[out=270,in=55] (n1);
        \draw (g1) to[out=270,in=125] (n1);
        \draw (n1) to (end);
      \end{tikzpicture}
  \end{matrix}
  = g.
      \]
    \item[\ref{le:fgisfg}] Use the same diagrammatic argument as in item \ref{le:dfgisfg}.
  \end{enumerate}
\end{proof}

\begin{proposition}\label{prop:discrete_inverse_category_has_meets}
  A discrete inverse category has meets, where $f\meet g =\Delta (f\* g) \inv{\Delta}$.
\end{proposition}
\begin{proof}
  $f\meet g \le f$:
  \[
     f\meet g =
  \begin{matrix}
        \begin{tikzpicture}
        \node at (0,3.5) (start) {};
        \node at (0,3) [delta] (d1) {};
        \node at (-.25,2.5) [map] (f) {$\scriptstyle f$};
        \node at (.25,2.5) [map] (g) {$\scriptstyle g$};
        \node at (0, 2) [nabla] (n1) {};
        \node at (0,1.5) (end) {};
        \draw [] (start) to (d1);
        \draw [] (d1) to[out=305,in=90] (g);
        \draw [] (d1) to[out=235,in=90] (f);
        \draw (g) to[out=270,in=55] (n1);
        \draw (f) to[out=270,in=125] (n1);
        \draw (n1) to (end);
      \end{tikzpicture}
  \end{matrix}
  =
  \begin{matrix}
        \begin{tikzpicture}
        \node at (0,3.5) (start) {};
        \node at (0,3) [delta] (d1) {};
        \node at (-.5,2.5) [map] (f) {$\scriptstyle f\rst{\inv{f}}$};
        \node at (.25,2.5) [map] (g) {$\scriptstyle g$};
        \node at (0, 2) [nabla] (n1) {};
        \node at (0,1.5) (end) {};
        \draw [] (start) to (d1);
        \draw [] (d1) to[out=305,in=90] (g);
        \draw [] (d1) to[out=235,in=90] (f);
        \draw (g) to[out=270,in=55] (n1);
        \draw (f) to[out=270,in=125] (n1);
        \draw (n1) to (end);
      \end{tikzpicture}
  \end{matrix}
  =
  \begin{matrix}
        \begin{tikzpicture}
        \node at (0,3.5) (start) {};
        \node at (0,3) [delta] (d1) {};
        \node at (-.25,2.5) [map] (f) {$\scriptstyle f$};
        \node at (.5,2.5) [map] (g) {$\scriptstyle g\rst{\inv{f}}$};
        \node at (0, 2) [nabla] (n1) {};
        \node at (0,1.5) (end) {};
        \draw [] (start) to (d1);
        \draw [] (d1) to[out=305,in=90] (g);
        \draw [] (d1) to[out=235,in=90] (f);
        \draw (g) to[out=270,in=55] (n1);
        \draw (f) to[out=270,in=125] (n1);
        \draw (n1) to (end);
      \end{tikzpicture}
  \end{matrix}
  =
  \begin{matrix}
        \begin{tikzpicture}
        \node at (0,3.5) (start) {};
        \node at (0,3) [delta] (d1) {};
        \node at (-.25,2) [map] (f) {$\scriptstyle f$};
        \node at (.5,2.5) [map] (g) {$\scriptstyle g\inv{f}$};
        \node at (.25,2) [map] (f2) {$\scriptstyle f$};
        \node at (0, 1.5) [nabla] (n1) {};
        \node at (0,1) (end) {};
        \draw [] (start) to (d1);
        \draw [] (d1) to[out=305,in=90] (g);
        \draw [] (d1) to[out=235,in=90] (f);
        \draw (g) to (f2);
        \draw (f2) to[out=270,in=55] (n1);
        \draw (f) to[out=270,in=125] (n1);
        \draw (n1) to (end);
      \end{tikzpicture}
  \end{matrix}
  =
  \begin{matrix}
        \begin{tikzpicture}
        \node at (0,3.5) (start) {};
        \node at (0,3) [delta] (d1) {};
        \node at (.5,2.5) [map] (g) {$\scriptstyle g\inv{f}$};
        \node at (0, 2) [nabla] (n1) {};
        \node at (0,1.5) [map] (f) {$\scriptstyle f$};
        \node at (0,1) (end) {};
        \draw [] (start) to (d1);
        \draw [] (d1) to[out=305,in=90] (g);
        \draw [] (d1) to[out=235,in=125] (n1);
        \draw (g) to[out=270,in=55] (n1);
        \draw (n1) to (f);
        \draw (f) to (end);
      \end{tikzpicture}
  \end{matrix}
  = \rst{f\meet g}f.
  \]

  $f\meet f = f$:
  \begin{equation*}
    f\meet f = \Delta(f\* f) \inv{\Delta} =f \Delta \inv{\Delta} = f.
  \end{equation*}

  $h(f\meet g) = h f \meet h g$:
  \begin{align*}
    h(f\meet g) &= h \Delta(f\* g) \inv{\Delta}& \text{Definition of }\meet\\
    &= \Delta(h \* h) (f \* g) \inv{\Delta} &\Delta\text{ natural}\\
    &= \Delta(h f\* h g) \inv{\Delta} &\text{compose maps}\\
    &= h f \meet h g&\text{Definition of }\meet.
  \end{align*}
\end{proof}

% subsection discrete_inverse_categories (end)


\subsection{The inverse subcategory of a discrete Cartesian restriction category } % (fold)
\label{sub:the_inverse_subcategory_of_a_discrete_restriction_category}

Given a discrete Cartesian restriction category, one can pick out the maps which are partial isomorphisms.
Using results from Subsection~\ref{sub:discrete_inverse_categories} and from
Section~\ref{sub:discrete_restriction_categories}, we will show that these maps form a
subcategory which is a discrete inverse category.

\begin{proposition}\label{lem:inv_x_is_a_discrete_inverse_category}
  Given \X is a discrete Cartesian restriction category, the partial isomorphisms of \X, together
  with the objects of \X form a sub-restriction category which is a discrete inverse category. For
  the restriction category $\X$, we denote this subcategory by \Inv{\X}.
\end{proposition}
\begin{proof}
  As shown in Lemma~\ref{lem:rcs_partial_monic_section_inverse_properties}, partial isomorphisms
  are closed under composition. The identity maps are in \Inv{\X} and restrictions of
  partial isomorphisms are also partial isomorphisms.

  The product on the discrete Cartesian restriction category \X becomes the tensor product of the
  restriction category \Inv{\X}. Table~\ref{tab:structural_maps_for_the_tensor_in_invx} shows how
  each of the elements of the tensor are defined. Note that the last definition makes explicit use
  of the fact we are in a discrete Cartesian restriction category and hence the $\Delta$ of \X
  possesses a partial inverse.

  \begin{table}[!htbp]
    \begin{center}
      \begin{tabular}{|ccc|}
        \hline
        \X & \Inv{\X} & Inverse map\\
        \hline\hline
        $\scriptstyle A\times B$ & $\scriptstyle A\* B$ &\\[6pt]
        \hline
        $\scriptstyle \top$ & $\scriptstyle 1$ &\\[6pt]
        \hline
        $\scriptstyle \pi_1:\top\times A \to A$ & $\scriptstyle \usl:1\* A \to A$ & $\scriptstyle \<!,1\>$\\[6pt]
        \hline
        $\scriptstyle \pi_0:A\times\top \to A$ & $\scriptstyle \usr:A\*1 \to A$& $\scriptstyle \<1,!\>$\\[6pt]
        \hline
        ${\scriptstyle a_{\X} = \<\pi_0 \pi_0,\<\pi_0 \pi_1,\pi_1\>\>:(A\times B)\times C \to A\times(B\times C)}$
          & $\scriptstyle a_{\*}:(A\*B)\*C \to A\*(B\*C)$
          & $\scriptstyle \<\<\pi_0, \pi_1 \pi_0\>,\pi_1 \pi_1\>$\\[6pt]
        \hline
        $\scriptstyle c_{\X}=\< \pi_1,\pi_0\>:A\times B \to B\times A$ & $\scriptstyle c_{\*}:A\*B \to B \* A$ & $\scriptstyle \< \pi_1,\pi_0\>$\\[6pt]
        \hline
        $\scriptstyle \Delta_{\X}:A\to A\times A$ & $\scriptstyle \Delta:A \to A\* A$ & $\scriptstyle  \inv{\Delta_{\X}} $\\[6pt]
        \hline
      \end{tabular}

    \end{center}
    \caption{Structural maps for the tensor in \Inv{\X}}
    \label{tab:structural_maps_for_the_tensor_in_invx}
  \end{table}

  The monoid coherence diagrams follow directly from the characteristics of the product in
  \X. Similarly, $\Delta$ is total as it is total in $\X$. It remains to show cocommutativity,
  coassociativity and the Frobenius condition.

  Cocommutativity requires $\Delta c_{\*} = c_{\*}$. We have
  \[
     \Delta_{\X} \< \pi_1,\pi_0\> = \<\Delta_{\X}\pi_1, \Delta_{\X}\pi_0\> = \<1,1\> =
     \Delta_{\X},
  \]
  giving us the required cocommutativity.

  Coassociativity requires $\Delta (1 \* \Delta) = \Delta (\Delta \* 1) a_{\*}$. Expressing this
  in \X, it is the requirement that
  \[
    \Delta_{\X} (1 \times \Delta_{\X}) =
      \Delta_{\X}(\Delta_{\X} \times 1) a_{\X}.
  \]
  Recalling that $f\times g \pi_0 = \pi_0 f$ and $f\times g \pi_1 = \pi_1 g$, we have:
  \begin{align*}
    \Delta_{\X}(\Delta_{\X} \times 1) a_{\X} &=\Delta_{\X}(\Delta_{\X} \times 1)
    \<\pi_0\pi_0,\<\pi_0\pi_1,\pi_1\>\>\\
    &= \<\Delta_{\X}(\Delta_{\X} \times 1)\pi_0\pi_0,\<\Delta_{\X}(\Delta_{\X} \times 1)\pi_0\pi_1,\Delta_{\X}(\Delta_{\X} \times 1)\pi_1\>\>\\
    &= \<\Delta_{\X}\pi_0\Delta_{\X}\pi_0,\<\Delta_{\X}\pi_0\Delta_{\X}\pi_1,\Delta_{\X}\pi_1 1\>\>\\
    &= \<1, \<1,1\>\> = \Delta_{\X}(1\times\Delta_{\X})
  \end{align*}
  and shows that we have coassociativity.

  The semi-Frobenius requirement is two-fold:
  \begin{align}
    \inv{\Delta} \Delta &= (\Delta \*1) a_{\*}(1\*\inv{\Delta}), \label{eq:frobenius_righths_need_in_invx}\\
    \inv{\Delta} \Delta &= (1 \* \Delta) \inv{a_{\*}}(\inv{\Delta}\* 1) \label{eq:frobenius_lefths_need_in_invx}.
  \end{align}
  In \X, these become:
  \begin{align}
    \inv{\Delta_{\X}} \Delta_{\X}
      &= (\Delta_{\X} \times 1) \<\pi_0 \pi_0,\<\pi_0 \pi_1,\pi_1\>\>(1\times\inv{\Delta_{\X}}),
      \label{eq:frobenius_righths_expressed_in_x}\\
    \inv{\Delta_{\X}}\Delta_{\X}
      &= (1 \times \Delta_{\X}) \<\<\pi_0, \pi_1 \pi_0\>,\pi_1 \pi_1\>(\inv{\Delta_{\X}}\times 1).
      \label{eq:frobenius_lefths_expressed_in_x}
  \end{align}
  We will give the details of the proof for Equation~\ref{eq:frobenius_righths_expressed_in_x}.
  Proving Equation~\ref{eq:frobenius_lefths_expressed_in_x} is similar.

  Note first that $\Delta(1 \times !)$ (and $\Delta(!\times 1)$) is the identity. Second, we see
  that maps to a product of objects may be expressed as a pairing --- i.e.  if
  $f:A \to B \times B$, then $f = \<f(1\times !), f(!\times 1)\>$.

  Using this we see that the left hand side of Equation~\ref{eq:frobenius_righths_expressed_in_x}
  may be computed as follows:
  \[
    \inv{\Delta_{\X}} \Delta_{\X}
       = \<\inv{\Delta_{\X}} \Delta_{\X}(1\times !), \inv{\Delta_{\X}} \Delta_{\X} (! \times 1)\>
    = \<\inv{\Delta_{\X}}, \inv{\Delta_{\X}} \>.
  \]
  Similarly, removing the associativity maps, the right hand side of the same equation becomes:
% LHS below works as
%$  (1\times \inv{\Delta_{\X}})(!\times 1) = (! \times 1 \times 1 )( \inv{\Delta_{\X}})$
% And then $(\Delta_{\X} \times 1)(! \times 1 \times 1 ) = 1 \times 1 = 1$

  \begin{align*}
    (\Delta_{\X} \times 1) (1\times\inv{\Delta_{\X}}) &
      = \<(\Delta_{\X} \times 1) (1\times\inv{\Delta_{\X}}) (1\times !),
      (\Delta_{\X} \times 1) (1\times\inv{\Delta_{\X}}) (! \times 1 )\> \\
    &= \<(\Delta_{\X} \times 1) (1\times\inv{\Delta_{\X}}) (1\times ! ), \inv{\Delta_{\X}}\> \\
    &= \<(\Delta_{\X} \times 1) (1\times\inv{\Delta_{\X}}) (1 \times \Delta_{\X})(1\times !\times !), \inv{\Delta_{\X}}\> \\
    &= \<(\Delta_{\X} \times 1) (1\times\rst{\inv{\Delta_{\X}}}) (1\times !\times !), \inv{\Delta_{\X}}\> \\
    &= \<(\Delta_{\X} \times 1) \rst{1\times\inv{\Delta_{\X}}} (1\times !\times !), \inv{\Delta_{\X}}\> \\
    &= \<\rst{(\Delta_{\X} \times 1) (1\times\inv{\Delta_{\X}})}
      (\Delta_{\X} \times 1)(1\times !\times !), \inv{\Delta_{\X}}\> \\ %rfour
    &= \<\rst{(\Delta_{\X} \times 1) (1\times\inv{\Delta_{\X}})} (1\times !), \inv{\Delta_{\X}}\> \\
      &= \<\rst{(\Delta_{\X} \times 1) (1\times\inv{\Delta_{\X}})(! \times 1)} (1\times !),
      \inv{\Delta_{\X}}\> \\ % add total to right of rst
    &= \<\rst{\inv{\Delta_{\X}}} (1\times !), \inv{\Delta_{\X}}\> \\
    &= \<\inv{\Delta_{\X}}\Delta_{\X}(1\times !), \inv{\Delta_{\X}}\> = \<\inv{\Delta_{\X}}, \inv{\Delta_{\X}}\>
  \end{align*}
  and therefore we see that the first equation for the Frobenius condition is satisfied. Thus,
  $\Inv{\X}$ is a discrete inverse category.
\end{proof}
% subsection the_inverse_subcategory_of_a_graphic_cartesian_restriction_category (end)

% section inverse_products (end)
\section{The ``slice'' construction on a discrete inverse category}
\label{sec:slice_construction_on_discrete_inverse_category}

% In an inverse category, we will be interested in a specific class of maps, which will be used in
% future chapters to connect discrete inverse categories to discrete Cartesian restriction
% categories.
Throughout this section, we will assume $\X$ is a discrete inverse category.

In a discrete inverse category, suppose we are given a map $h:A\*B \to A\*C$. We define
$\dmap{h}:A\*B \to A\*C$ as the composite $(\Delta\*1)(1\*h)(\inv{\Delta}\*1)$. We want to consider
those maps where $h = \dmap{h}$. In our graphical language, this means
\[
    h=
  \begin{matrix}
        \begin{tikzpicture}
        \node at (0,3.5) (start) {};
        \node at (.6,3.5) (start2) {};
        \node at (0,3) [delta] (d1) {};
        \node at (.35,2.5) [map] (h) {$\scriptstyle h$};
        \node at (0, 2) [nabla] (n1) {};
        \node at (0,1.5) (end) {};
        \node at (.6,1.5) (end2) {};
        \draw [] (start) to (d1);
        \draw [] (start2) to[out=270,in=55] (h);
        \draw [] (d1) to[out=305,in=125] (h);
        \draw [] (d1) to[out=235,in=125] (n1);
        \draw (h) to[out=235,in=55] (n1);
        \draw (h) to[out=305,in=90] (end2);
        \draw (n1) to (end);
      \end{tikzpicture}
  \end{matrix}
  .
\]

Maps of this form satisfy a variety of closure properties, as shown in the lemmas below.

\begin{lemma}\label{lem:delta_nabla_is_idempotent}
  For any map $h$ in a discrete inverse category, $\dmap{\dmap{h}} = \dmap{h}$.
\end{lemma}
\begin{proof}
  \[
    \dmap{\dmap{h}}=
  \begin{matrix}
    \begin{tikzpicture}
      \node at (0,3) (start) {};
      \node at (.6,3) (start2) {};
      \node at (0,2.5) [delta] (d1) {};
      \node at (.25,2) [delta] (d2) {};
      \node at (.5,1.5) [map] (h) {$\scriptstyle \,h\,$};
      \node at (.25, 1) [nabla] (n2) {};
      \node at (0, .5) [nabla] (n1) {};
      \node at (0,0) (end) {};
      \node at (.6,0) (end2) {};
      \draw [] (start) to (d1);
      \draw [] (start2) to[out=270,in=55] (h);
      \draw [] (d1) to[out=305,in=90] (d2);
      \draw [] (d1) to[out=235,in=125] (n1);
      \draw [] (d2) to[out=305,in=125] (h);
      \draw [] (d2) to[out=235,in=125] (n2);
      \draw (h) to[out=235,in=55] (n2);
      \draw (h) to[out=305,in=90] (end2);
      \draw (n2) to[out=270,in=55] (n1);
      \draw (n1) to (end);
    \end{tikzpicture}
  \end{matrix}
  =
  \begin{matrix}
    \begin{tikzpicture}
      \node at (0,3) (start) {};
      \node at (.6,3) (start2) {};
      \node at (0,2.5) [delta] (d1) {};
      \node at (-.25,2) [delta] (d2) {};
      \node at (.35,1.5) [map] (h) {$\scriptstyle \,h\,$};
      \node at (-.25, 1) [nabla] (n2) {};
      \node at (0, .5) [nabla] (n1) {};
      \node at (0,0) (end) {};
      \node at (.6,0) (end2) {};
      \draw [] (start) to (d1);
      \draw [] (start2) to[out=270,in=55] (h);
      \draw [] (d1) to[out=235,in=90] (d2);
      \draw [] (d1) to[out=305,in=125] (h);
      \draw [] (d2) to[out=305,in=55] (n2);
      \draw [] (d2) to[out=235,in=125] (n2);
      \draw (h) to[out=235,in=55] (n1);
      \draw (h) to[out=305,in=90] (end2);
      \draw (n2) to[out=270,in=125] (n1);
      \draw (n1) to (end);
    \end{tikzpicture}
  \end{matrix}
  =
  \begin{matrix}
      \begin{tikzpicture}
        \node at (0,3.5) (start) {};
        \node at (.6,3.5) (start2) {};
        \node at (0,3) [delta] (d1) {};
        \node at (.35,2.5) [map] (h) {$\scriptstyle h$};
        \node at (0, 2) [nabla] (n1) {};
        \node at (0,1.5) (end) {};
        \node at (.6,1.5) (end2) {};
        \draw [] (start) to (d1);
        \draw [] (start2) to[out=270,in=55] (h);
        \draw [] (d1) to[out=305,in=125] (h);
        \draw [] (d1) to[out=235,in=125] (n1);
        \draw (h) to[out=235,in=55] (n1);
        \draw (h) to[out=305,in=90] (end2);
        \draw (n1) to (end);
      \end{tikzpicture}
  \end{matrix} = \dmap{h}.
  \]
\end{proof}

The self-closure aspect of maps where $h=\dmap{h}$ allows us to show all restriction idempotents
have this property:
\begin{corollary}\label{cor:restriction-idempotents-have-dmap-property}
  When $e=\rst{e}:A\*Y \to A\*Y$, then $e = \dmap{e}$.
\end{corollary}
\begin{proof}

Using
Lemma~\ref{lem:properties_of_delta_and_tensor_in_a_discrete_inverse_category} and the exchange rule,
when $e = \rst{e}: A\*Y \to A\*Y$, we have
\begin{equation}
  e =
  \begin{matrix}
    \begin{tikzpicture}
        \node at (0,3) (start) {};
        \node at (0,2.5) [delta] (d1) {};
        \node at (-.25,2) [map] (e1) {$\scriptstyle e$};
        \node at (0,1.5) [nabla] (n1) {};
        \node at (0,1) (end) {};
        \draw [] (start) to (d1);
        \draw [] (d1) to[out=235,in=90] (e1);
        \draw [] (d1) to[out=305,in=55] (n1);
        \draw (e1) to[out=270,in=125] (n1);
        \draw [] (n1) to (end);
      \end{tikzpicture}
  \end{matrix}
  =
  \begin{matrix}
    \begin{tikzpicture}
        \node at (0,3) (start) {};
        \node at (0.5,3) (start1) {};
        \node at (0,2.5) [delta] (d1) {};
        \node at (0.5,2.5) [delta] (d2) {};
        \node at (-.25,2) [map] (e1) {$\scriptstyle e$};
        \node at (0,1.5) [nabla] (n1) {};
        \node at (0.5,1.5) [nabla] (n2) {};
        \node at (0,1) (end) {};
        \node at (0.5,1) (end1) {};
        \draw [] (start) to (d1);
        \draw [] (start1) to (d2);
        \draw [] (d1) to[out=235,in=110] (e1);
        \draw [] (d1) to[out=305,in=55] (n1);
        \draw [] (d2) to[out=235,in=70] (e1);
        \draw [] (d2) to[out=305,in=55] (n2);
        \draw (e1) to[out=250,in=125] (n1);
        \draw (e1) to[out=290,in=125] (n2);
        \draw [] (n1) to (end);
        \draw [] (n2) to (end1);
      \end{tikzpicture}
  \end{matrix}
  =
  \begin{matrix}
    \begin{tikzpicture}
        \node at (0,3) (start) {};
        \node at (0.5,3) (start1) {};
        \node at (0,2.5) [delta] (d1) {};
        \node at (0.5,2.5) [delta] (d2) {};
        \node at (.25,2) [map] (e1) {$\scriptstyle e$};
        \node at (0,1.5) [nabla] (n1) {};
        \node at (0.5,1.5) [nabla] (n2) {};
        \node at (0,1) (end) {};
        \node at (0.5,1) (end1) {};
        \draw [] (start) to (d1);
        \draw [] (start1) to (d2);
        \draw [] (d1) to[out=305,in=110] (e1);
        \draw [] (d1) to[out=235,in=125] (n1);
        \draw [] (d2) to[out=235,in=70] (e1);
        \draw [] (d2) to[out=305,in=55] (n2);
        \draw (e1) to[out=250,in=55] (n1);
        \draw (e1) to[out=290,in=125] (n2);
        \draw [] (n1) to (end);
        \draw [] (n2) to (end1);
      \end{tikzpicture}
  \end{matrix}
  .
  \label{eq:idempotents-with-double-delta-nabla}
\end{equation}

But then the same graphical argument  as shown in Lemma~\ref{lem:delta_nabla_is_idempotent}  applied
to the right hand term of Equation~\ref{eq:idempotents-with-double-delta-nabla} gives $e = \dmap{e}$.
\end{proof}

There are a variety of other closure properties:
\begin{lemma}\label{lem:delta_nabla_maps_are_closed}
  In a discrete inverse category \X with the object $A$ define $\dmap{A}$ as the set of
  maps $h:A\*Y\to A\* Z$ where $h=\dmap{h}$. Then $\dmap{A}$ has the following properties:
  \begin{enumerate}[{(}i{)}]
  \item It is closed under partial inverses;\label{lemitem:delta_nabla_1}
  \item it is closed under composition;\label{lemitem:delta_nabla_2}
  \item it contains all maps of the form $1\*k$ where $k:Y\to Z$;\label{lemitem:delta_nabla_3}
  \item it contains $\Delta:A\*B \to A\*B\*A\*B$;\label{lemitem:delta_nabla_4}
  \end{enumerate}
\end{lemma}
\begin{proof}
  For~\ref{lemitem:delta_nabla_1}, if $h = \dmap{h}$, then
  \begin{multline*}
    \inv{h} = \inv{((\Delta\*1)(1\*h)(\inv{\Delta}\*1))} =\\
      \inv{(\inv{\Delta}\*1)}\inv{(1\*h)}\inv{(\Delta\*1)} =
        (\Delta\*1)(1\*\inv{h})(\inv{\Delta}\*1).
  \end{multline*}

  To show ~\ref{lemitem:delta_nabla_2}, we compose $h$ and $g$:
  \[
    h g=
  \begin{matrix}
        \begin{tikzpicture}
          \node at (0.25,4) (s1) {};
          \node at (.5,4) (s2) {};
          \node at (0.25,3.5) [delta] (d2) {};
          \node at (.65,3) [map] (h) {$\scriptstyle \ h\ $};
          \node at (0.25, 2.5) [nabla] (n2) {};
          \node at (0.25,2) [delta] (d3) {};
          \node at (.65,1.5) [map] (g) {$\scriptstyle \ g\ $};
          \node at (0.25, 1) [nabla] (n3) {};
          \node at (0.25,0.5) (end1) {};
          \node at (.5,0.5) (end2) {};
          \draw [] (s1) to (d2);
          \draw [] (s2) to[out=270,in=55] (h);
          \draw [] (d2) to[out=305,in=125] (h);
          \draw [] (d2) to[out=235,in=125] (n2);
          \draw (h) to[out=235,in=55] (n2);
          \draw (h) to[out=305,in=55] (g);
          \draw (n2) to (d3);
          \draw [] (d3) to[out=305,in=125] (g);
          \draw [] (d3) to[out=235,in=125] (n3);
          \draw (g) to[out=235,in=55] (n3);
          \draw (g) to[out=305,in=90] (end2);
          \draw (n3) to (end1);
        \end{tikzpicture}
  \end{matrix}
  =
  \begin{matrix}
        \begin{tikzpicture}
          \node at (0.25,4.5) (s1) {};
          \node at (.5,4.5) (s2) {};
          \node at (0.25,4) [delta] (d1) {};
          \node at (.65,3.5) [map] (h) {$\scriptstyle \ h\ $};
          \node at (0,3) [delta] (d2) {};
          \node at (0.25, 2.5) [nabla] (n2) {};
          \node at (.65,2) [map] (g) {$\scriptstyle \ g\ $};
          \node at (0.25, 1.5) [nabla] (n1) {};
          \node at (0.25,1) (end1) {};
          \node at (.5,1) (end2) {};
          \draw [] (s1) to (d1);
          \draw (d1) to[out=235,in=90] (d2);
          \draw (d1) to[out=305,in=125] (h);
          \draw [] (s2) to[out=270,in=55] (h);
          \draw [] (d2) to[out=305,in=125] (n2);
          \draw [] (d2) to[out=235,in=125] (n1);
          \draw (h) to[out=235,in=55] (n2);
          \draw (h) to[out=305,in=55] (g);
          \draw (n2) to[out=270,in=125] (g);
          \draw (g) to[out=235,in=55] (n1);
          \draw (g) to[out=305,in=90] (end2);
          \draw (n1) to (end1);
        \end{tikzpicture}
  \end{matrix}
  =
  \begin{matrix}
        \begin{tikzpicture}
          \node at (0.25,4.5) (s1) {};
          \node at (.5,4.5) (s2) {};
          \node at (0.25,4) [delta] (d1) {};
          \node at (0,3.5) [delta] (d2) {};
          \node at (.65,3) [map] (h) {$\scriptstyle \ h\ $};
          \node at (0.25, 2.5) [nabla] (n2) {};
          \node at (.65,2) [map] (g) {$\scriptstyle \ g\ $};
          \node at (0.25, 1.5) [nabla] (n1) {};
          \node at (0.25,1) (end1) {};
          \node at (.5,1) (end2) {};
          \draw [] (s1) to (d1);
          \draw (d1) to[out=235,in=90] (d2);
          \draw (d1) to[out=305,in=125] (h);
          \draw [] (s2) to[out=270,in=55] (h);
          \draw [] (d2) to[out=305,in=125] (n2);
          \draw [] (d2) to[out=235,in=125] (n1);
          \draw (h) to[out=235,in=55] (n2);
          \draw (h) to[out=305,in=55] (g);
          \draw (n2) to[out=270,in=125] (g);
          \draw (g) to[out=235,in=55] (n1);
          \draw (g) to[out=305,in=90] (end2);
          \draw (n1) to (end1);
        \end{tikzpicture}
  \end{matrix}
  =
  \begin{matrix}
        \begin{tikzpicture}
          \node at (0.25,4.5) (s1) {};
          \node at (.5,4.5) (s2) {};
          \node at (0.25,4) [delta] (d1) {};
          \node at (0.5,3.5) [delta] (d2) {};
          \node at (.85,3) [map] (h) {$\scriptstyle \ h\ $};
          \node at (0.5, 2.5) [nabla] (n2) {};
          \node at (.65,2) [map] (g) {$\scriptstyle \ g\ $};
          \node at (0.25, 1.5) [nabla] (n1) {};
          \node at (0.25,1) (end1) {};
          \node at (.5,1) (end2) {};
          \draw [] (s1) to (d1);
          \draw (d1) to[out=235,in=125] (n1);
          \draw (d1) to[out=305,in=90] (d2);
          \draw [] (s2) to[out=270,in=55] (h);
          \draw [] (d2) to[out=305,in=125] (h);
          \draw [] (d2) to[out=235,in=125] (n2);
          \draw (h) to[out=235,in=55] (n2);
          \draw (h) to[out=305,in=55] (g);
          \draw (n2) to[out=270,in=125] (g);
          \draw (g) to[out=235,in=55] (n1);
          \draw (g) to[out=305,in=90] (end2);
          \draw (n1) to (end1);
        \end{tikzpicture}
  \end{matrix}
  =
  \begin{matrix}
        \begin{tikzpicture}
          \node at (0.25,4) (s1) {};
          \node at (.5,4) (s2) {};
          \node at (0.25,3.5) [delta] (d1) {};
          \node at (.65,3) [map] (h) {$\scriptstyle \ h\ $};
          \node at (.65,2.5) [map] (g) {$\scriptstyle \ g\ $};
          \node at (0.25, 2) [nabla] (n1) {};
          \node at (0.25,1.5) (end1) {};
          \node at (.5,1.5) (end2) {};
          \draw [] (s1) to (d1);
          \draw (d1) to[out=235,in=125] (n1);
          \draw (d1) to[out=305,in=125] (h);
          \draw [] (s2) to[out=270,in=55] (h);
          \draw (h) to[out=235,in=125] (g);
          \draw (h) to[out=305,in=55] (g);
          \draw (g) to[out=235,in=55] (n1);
          \draw (g) to[out=305,in=90] (end2);
          \draw (n1) to (end1);
        \end{tikzpicture}
  \end{matrix}
  .
   \]

   For~\ref{lemitem:delta_nabla_3}, this follows immediately from
  \[
  \begin{matrix}
    \begin{tikzpicture}
      \node at (0,2) (s1) {};
      \node at (.5,2) (s2) {};
      \node at (0,1.5) [delta] (d1) {};
      \node at (.5,1) [map] (h) {$\scriptstyle h$};
      \node at (0,.5) [nabla] (n1) {};
      \node at (0,0) (end1) {};
      \node at (.5,0) (end2) {};
      \draw [] (s1) to (d1);
      \draw [] (s2) to (h);
      \draw [] (d1) to[out=305,in=55] (n1);
      \draw [] (d1) to[out=235,in=125] (n1);
      \draw (h) to (end2);
      \draw (n1) to (end1);
    \end{tikzpicture}
  \end{matrix}
  =
  \begin{matrix}
    \begin{tikzpicture}
      \node at (0.25,2) (s1) {};
      \node at (.5,2) (s2) {};
      \node at (.5,1) [map] (h) {$\scriptstyle h$};
       \node at (0.25,0) (end1) {};
      \node at (.5,0) (end2) {};
      \draw [] (s1) to (end1);
      \draw [] (s2) to (h);
      \draw (h) to (end2);
    \end{tikzpicture}
  \end{matrix}= 1\* h.
  \]

 To show~\ref{lemitem:delta_nabla_4}, recalling the exchange rule, we have
\[
  \dmap{\Delta} =
  \begin{matrix}
  \begin{tikzpicture}
    \node [style=nothing] (s1) at (-1, 3) {};
    \node [style=nothing] (s2) at (-0.25, 3) {};
    \node [style=delta] (d1) at (-1,2.5) {};
    \node [style=tensor] (t1) at (-0.5,2) {$\scriptstyle \*$};
    \node [style=delta] (d2) at (-0.5,1.5) {};
    \node [style=tensor] (t2) at (-0.75,1) {$\scriptstyle \*$};
    \node [style=nabla] (n1) at (-1,0.5) {};
    \node [style=tensor] (t3) at (-0.25,0.5) {$\scriptstyle \*$};
    \node [style=nothing] (e1) at (-1,0) {};
    \node [style=nothing] (e2) at (-0.25,0) {};
    \draw (s1) to (d1);
    \draw (s2) to[out=270,in=55] (t1);
    \draw (d1) to[out=235,in=125] (n1);
    \draw (d1) to[out=305,in=125] (t1);
    \draw (t1) to (d2);
    \draw (d2) to[out=235,in=90] (t2);
    \draw (d2) to[out=305,in=55] (t3);
    \draw (t2) to[out=235,in=55] (n1);
    \draw (t2) to[out=305,in=125] (t3);
    \draw (n1) to (e1);
    \draw (t3) to (e2);
  \end{tikzpicture}
  \end{matrix}
  =
  \begin{matrix}
  \begin{tikzpicture}
    \node [style=nothing] (s1) at (-1, 2.5) {};
    \node [style=nothing] (s2) at (-0.25, 2.5) {};
    \node [style=delta] (d1) at (-1,2) {};
    \node [style=delta] (d2) at (-.75,1.25) {};
    \node [style=delta] (d3) at (-0.25,1.25) {};
    \node [style=nabla] (n1) at (-1,0.5) {};
    \node [style=tensor] (t3) at (-0.25,0.5) {$\scriptstyle \*$};
    \node [style=nothing] (e1) at (-1,0) {};
    \node [style=nothing] (e2) at (-0.25,0) {};
    \draw (s1) to (d1);
    \draw (s2) to (d3);
    \draw (d1) to[out=235,in=125] (n1);
    \draw (d1) to[out=305,in=90] (d2);
    \draw (d2) to[out=235,in=55] (n1);
    \draw (d2) to[out=305,in=90] (t3);
    \draw (d3) to[out=235,in=125] (t3);
    \draw (d3) to[out=305,in=55] (t3);
    \draw (n1) to (e1);
    \draw (t3) to (e2);
  \end{tikzpicture}
  \end{matrix}
  =
  \begin{matrix}
  \begin{tikzpicture}
    \node [style=nothing] (s1) at (-1, 2.5) {};
    \node [style=nothing] (s2) at (-0.25, 2.5) {};
    \node [style=delta] (d1) at (-1,2) {};
    \node [style=delta] (d2) at (-1.25,1.25) {};
    \node [style=delta] (d3) at (-0.25,1.75) {};
    \node [style=nabla] (n1) at (-1,0.5) {};
    \node [style=tensor] (t3) at (-0.25,0.5) {$\scriptstyle \*$};
    \node [style=nothing] (e1) at (-1,0) {};
    \node [style=nothing] (e2) at (-0.25,0) {};
    \draw (s1) to (d1);
    \draw (s2) to (d3);
    \draw (d1) to[out=235,in=90] (d2);
    \draw (d1) to[out=305,in=90] (t3);
    \draw (d2) to[out=235,in=125] (n1);
    \draw (d2) to[out=305,in=55] (n1);
    \draw (d3) to[out=235,in=125] (t3);
    \draw (d3) to[out=305,in=55] (t3);
    \draw (n1) to (e1);
    \draw (t3) to (e2);
  \end{tikzpicture}
  \end{matrix}
  =
  \begin{matrix}
  \begin{tikzpicture}
    \node [style=nothing] (s1) at (-1, 2.5) {};
    \node [style=nothing] (s2) at (-0.25, 2.5) {};
    \node [style=delta] (d1) at (-1,2) {};
    \node [style=delta] (d3) at (-0.25,2) {};
    \node [style=tensor] (t3) at (-0.25,1) {$\scriptstyle \*$};
    \node [style=nothing] (e1) at (-1,0.5) {};
    \node [style=nothing] (e2) at (-0.25,0.5) {};
    \draw (s1) to (d1);
    \draw (s2) to (d3);
    \draw (d1) to[out=235,in=90] (e1);
    \draw (d1) to[out=305,in=90] (t3);
    \draw (d3) to[out=235,in=125] (t3);
    \draw (d3) to[out=305,in=55] (t3);
    \draw (t3) to (e2);
  \end{tikzpicture}
  \end{matrix} =\Delta.
\]
\end{proof}

Because of these closure rules, rather than stating $h = \dmap{h}$ we may equivalently say $h\in
\dmap{A}$ when $h=\dmap{h}:A\*X \to A\*Y$.

From Lemma~\ref{lem:delta_nabla_maps_are_closed}, we see that we will be able to form a category
based on maps $h$ such that $h=\dmap{h}$. Of course, this category is dependent upon the choice of
the object $A$, hence we will label it $\X[A]$, as it is reminiscent of the simple slice category of
Example~\ref{ex:simple-slice-adjoint} over an object $A$ for an ordinary Cartesian category. We make
this precise in the following proposition:

\begin{proposition}\label{prop:slice-a-is-a-discrete-inverse-category}
  Given a discrete inverse category $\X$, define $\X[A]$ as  the restriction category:
  \rcategory{The objects of $\X$}{A map $h=\dmap{h}:A\*X\to A\*Y$ in \X is a map from $X$ to $Y$ in
    $\X[A]$}{$1\*1$ in \X}{Composition in \X}{$\rst{h}$ in $\X[A]$ is given by $\rst{h}$ in $\X$}
  Then, $\X[A]$ is a discrete inverse category.
\end{proposition}
\begin{proof}
  Given Lemma~\ref{lem:delta_nabla_maps_are_closed}, we see immediately that $\X[A]$ is a
  category. By Corollary~\ref{cor:restriction-idempotents-have-dmap-property}, we know $\rst{h} = \dmap{\rst{h}}$.
  We must show  that $\X[A]$ has a tensor and a Frobenius $\Delta$.
  % We must show the restriction is in $\X[A]$, i.e.,
  % $h=\dmap{h}$ then $\rst{h} = \dmap{\rst{h}}$ and that $\X[A]$ has a tensor and a Frobenius $\Delta$.

  % Recalling that $\rst{(1\*h)} = 1\* \rst{h}$,
  % \begin{multline*}
  %   (\Delta\*1) \rst{(1\*h)} (\inv{\Delta}\*1) = \rst{(\Delta\*1)(1\*h)}(\Delta\*1)(\inv{\Delta}\*1)
  %   = \\
  %   \rst{(\Delta\*1)(1\*h)} \le \rst{(\Delta\*1)(1\*h)(\inv{\Delta}\*1)} = \rst{h}.
  % \end{multline*}
  % Note this means there is some $k$ such that
  % $\rst{k} = (\Delta\*1) \rst{(1\*h)}(\inv{\Delta}\*1)$. Next, computing $\rst{k}h$ we have:
  % \begin{align*}
  %   \rst{k}h &= (\Delta\*1) \rst{(1\*h)} (\inv{\Delta}\*1) (\Delta\*1) (1\*h) (\inv{\Delta}\*1) \\
  %   & = (\Delta\*1) \rst{(1\*h)}\, \rst{(\inv{\Delta}\*1)} (1\*h) (\inv{\Delta}\*1) \\
  %   & = (\Delta\*1) \rst{(\inv{\Delta}\*1)}\,  \rst{(1\*h)}(1\*h) (\inv{\Delta}\*1) \\
  %   & = (\Delta\*1) \wrg{(\Delta\*1)}  (1\*h) (\inv{\Delta}\*1) \\
  %   & = (\Delta\*1) (1\*h) (\inv{\Delta}\*1) \\
  %   & = h.
  % \end{align*}
  % But by Lemma~\ref{lem:restriction_cats_are_partial_order_enriched}, if $\rst{k} h = h$, we have
  % $\rst{h} \le \rst{k}$. As we already have $\rst{k}\le\rst{h}$ this means they are equal and the
  % restriction is in $\X[A]$, thus, $\X[A]$ is a restriction category.

  The tensor of objects $X\*Y$ in $\X[A]$ is the element $A\* X\*Y$ in $\X$. For two maps $h,g$
  in $\X[A]$, $h\*g$ is given by the $\X$ map
  $(\Delta\*1\*1)(1\*c_{\*}\*1)(h\*g)(1\*c_{\*}\*1)(\inv{\Delta}\*1\*1)$. This is a map in
  $\X[A]$:
  \[
    \dmap{(h\*g)}=
  \begin{matrix}
        \begin{tikzpicture}
          \node at (0,3.5) (s1) {};
          \node at (.5,3.5) (s2) {};
          \node at (.75,3.5) (s3) {};
          \node at (0,3) [delta] (d1) {};
          \node at (.25,2.5) [delta] (d2) {};
          \node at (.25,1.75) [map] (h) {$\scriptstyle \,h\,$};
          \node at (.75,1.75) [map] (g) {$\scriptstyle \,g\,$};
          \node at (0.25, 1) [nabla] (n2) {};
          \node at (0, .5) [nabla] (n1) {};
          \node at (0,0) (end1) {};
          \node at (.5,0) (end2) {};
          \node at (.75,0) (end3) {};
          \draw [] (s1) to (d1);
          \draw [] (s2) to[out=270,in=55] (h);
          \draw [] (s3) to[out=270,in=55] (g);
          \draw [] (d1) to[out=305,in=90] (d2);
          \draw [] (d1) to[out=235,in=125] (n1);
          \draw [] (d2) to[out=305,in=125] (g);
          \draw [] (d2) to[out=235,in=125] (h);
          \draw (h) to[out=235,in=125] (n2);
          \draw (h) to[out=305,in=90] (end2);
          \draw (g) to[out=235,in=55] (n2);
          \draw (g) to[out=305,in=90] (end3);
          \draw (n2) to[out=270,in=55] (n1);
          \draw (n1) to (end1);
        \end{tikzpicture}
  \end{matrix}
  =
  \begin{matrix}
        \begin{tikzpicture}
          \node at (0,3.5) (s1) {};
          \node at (.5,3.5) (s2) {};
          \node at (.75,3.5) (s3) {};
          \node at (0,3) [delta] (d1) {};
          \node at (-.25,2.5) [delta] (d2) {};
          \node at (.25,1.75) [map] (h) {$\scriptstyle \,h\,$};
          \node at (.75,1.75) [map] (g) {$\scriptstyle \,g\,$};
          \node at (-0.25, 1) [nabla] (n2) {};
          \node at (0, .5) [nabla] (n1) {};
          \node at (0,0) (end1) {};
          \node at (.5,0) (end2) {};
          \node at (.75,0) (end3) {};
          \draw [] (s1) to (d1);
          \draw [] (s2) to[out=270,in=55] (h);
          \draw [] (s3) to[out=270,in=55] (g);
          \draw [] (d1) to[out=235,in=90] (d2);
          \draw [] (d1) to[out=305,in=125] (g);
          \draw [] (d2) to[out=305,in=125] (h);
          \draw [] (d2) to[out=235,in=125] (n2);
          \draw (h) to[out=235,in=55] (n2);
          \draw (h) to[out=305,in=90] (end2);
          \draw (g) to[out=235,in=55] (n1);
          \draw (g) to[out=305,in=90] (end3);
          \draw (n2) to[out=270,in=125] (n1);
          \draw (n1) to (end1);
        \end{tikzpicture}
  \end{matrix}
  =
  \begin{matrix}
        \begin{tikzpicture}
          \node at (0,2.5) (s1) {};
          \node at (.5,2.5) (s2) {};
          \node at (.75,2.5) (s3) {};
          \node at (0,2) [delta] (d1) {};
          \node at (.25,1.25) [map] (h) {$\scriptstyle \,h\,$};
          \node at (.75,1.25) [map] (g) {$\scriptstyle \,g\,$};
          \node at (0, .5) [nabla] (n1) {};
          \node at (0,0) (end1) {};
          \node at (.5,0) (end2) {};
          \node at (.75,0) (end3) {};
          \draw [] (s1) to (d1);
          \draw [] (s2) to[out=270,in=55] (h);
          \draw [] (s3) to[out=270,in=55] (g);
          \draw [] (d1) to[out=235,in=125] (h);
          \draw [] (d1) to[out=305,in=125] (g);
          \draw (h) to[out=235,in=125] (n1);
          \draw (h) to[out=305,in=90] (end2);
          \draw (g) to[out=235,in=55] (n1);
          \draw (g) to[out=305,in=90] (end3);
          \draw (n1) to (end1);
        \end{tikzpicture}
  \end{matrix}
= h\*g.
  \]

  The $\Delta$ in $\X[A]$ is given by the map $1\*\Delta$ in \X. The various identities required of
  $\Delta$ hold in $\X[A]$ as they hold in $\X$, therefore $\X[A]$ is a discrete inverse category.
\end{proof}

We note that there are functors between $\X$ and $\X[A]$, given by:
\[
  G:\X\to\X[A];\quad G:B\mapsto B;\quad G:f \mapsto 1\*f
\]
and
\[
  F:\X[A]\to\X; \quad F:B \mapsto A\*B; \quad F: f \mapsto f.
\]
However, these do not form an adjoint pair as the relation is
\[
  \infer{\X[A](X,Y)}{\X(A\*X,A\*Y)}\quad \raisebox{10pt}{that is,}\ \  \infer{\X[A](X,G(Y))}{\X(F(X),F(Y))}
\]
rather than the required
\[
  \infer{\X[A](X,G(Y))}{\X(F(X),Y)}.
\]

\begin{example}
In \pinj, note that the slice over the object $\{*\}$ just gives us a category where the maps are in
bijective correspondence with the maps of \pinj, as $\{*\}$ is the identity for the inverse product.
\end{example}

\subsection{The interpretation of the slice construction in resource theory}
\label{subsec:interpretation-of-the-slice}
Although resource theory is generally not in the scope of this thesis, the slice construction of
this section may be examined from the viewpoint of resource theory as  given in Coecke, Fritz and
Spekkens\cite{coecke2014mathematical}. Coecke, Fritz and Spekkens describe a mathematical theory of
resources inspired, in part, by the pragmatic approach exemplified in chemistry --- understanding
something means being able to make use of it. Specifically in chemistry, this means understanding
chemicals as resources.  The way to model this mathematically is with a symmetric monoidal category:

\begin{definition}[\cite{coecke2014mathematical}, Definition 2.1]\label{def:resource-theory}
  A \emph{resource theory} is a symmetric monoidal category $(\R, \*)$, where the objects
  $\objects{\R}$ represent the actual resources and the morphisms $f:A\to B$ represent the
  transformation of resource $A$ into resource $B$ which can be done with no cost.
\end{definition}

In this definition, composition of maps then represent sequential composition of transformation, the
operation $\*$ represents parallel composition of transformation and the identity $I$ of $\*$
denotes an empty resource.

An important question in such a theory is ``Given a resource $A$, is it possible to transform it to
a resource $B$?'' Of course, answering such a question will depend upon the details of the category
and the difficulty of answering such a question may vary greatly. This may be reduced to a question
of resource convertibility:

\begin{definition}[\cite{coecke2014mathematical}, Definition 4.1]\label{def:resource-convertibility}
  A \emph{theory of resource convertibility} $(R,+,\succeq,0)$ is a set $R$ with a binary operation
  $+$, element $0\in R$ such that when $a\simeq b \definedas a \succeq b$ and $b \succeq a$, then
  \[
    a+(b+c) \simeq (a+b)+c \qquad a+b\simeq b+a \qquad a+0 \simeq 0 +a
  \]
  and
  \[
     a\succeq b,\ c\succeq d \implies a+c \succeq b+d.
  \]
\end{definition}

When $\R$ is a resource theory, then if we set $R \definedas \objects{\R}$, $+\definedas \*$, $a
\succeq b \definedas \R(a,b) \neq \emptyset$ and $0\definedas I$, then $(R,+,\succeq,0)$ is a theory
of resource convertibility.

From this, it is possible to define what is meant by a catalyst:
\begin{definition}[\cite{coecke2014mathematical}, Definition 4.8]\label{def:catalyst}
  Given $R$ is a theory of resource convertibility, then a resource $a\in R$ is a \emph{catalyst}
  for $b,c\in R$ when $b\nsucceq c$ but $a+b \succeq a+c$.
\end{definition}

Lifting this definition to a resource theory says the object $A$ is a catalyst for objects $B,C$
when there is a map $f: A\*B \to A\*C$, but no map $g:B\to C$. We might call such maps $f$,
\emph{catalytic transformations} and objects $A$, \emph{catalytic objects}.

Here, we begin to see the connection to the slice category in a discrete inverse category. As a
discrete inverse category is also a symmetric monoidal category, it is therefore also a resource
theory and hence may generate a theory of resource convertibility.

However, given $\X$ is a discrete inverse category, the slice category $\X[A]$ does not correspond
directly to the catalytic transformations as defined above for two reasons:
\begin{enumerate}[{(}i{)}]
  \item For two objects $B,C$ in $\X[A]$ with $f:B\to C$ in $\X[A]$, there is no guarantee that we
    do \emph{not} have a map  $g:B\to C$ in \X.
  \item There may be maps $f:A\+B \to A\+C$ in \X, which are not in $\X[A]$ since they do not
    preserve $A$.
\end{enumerate}

Continuing with the inspiration of chemistry, we note that $\X[A]$ may perhaps be considered as an
alternate way of describing  ``the catalytic transformations'' for $A$. In chemistry, one of the
requirements is that the catalyst is not changed by the transformation in any way. The theory of
resource convertibility from Definition~\ref{def:resource-convertibility} handles this by
considering the resource as an indivisible item. In a resource theory, one way to specify that the
catalytic object has not changed is to identify a subcategory of ``reversible'' transformations and
then to require the catalytic transformation be in $\X[A]$.

% chapter inverse_categories (end)
%%% Local Variables:
%%% mode: latex
%%% TeX-master: "../phd-thesis"
%%% End:
