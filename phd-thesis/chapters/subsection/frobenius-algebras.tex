\subsection{Frobenius Algebras} % (fold)
\label{sub:frobenius_algebras}
Frobenius algebras were originally defined as a finite dimensional algebra over a field together with a
non-degenerate pairing operation. Here we present the general categorical definition:

\begin{definition}\label{def:frobeniusalgebra}
  Given a symmetric monoidal category \cD, a \emph{Frobenius algebra} is an object $X$ of \cD and
  four maps, $\nabla :X\*X \to X$, $\eta: I \to X$, $\Delta: X\to X\* X$ and $\epsilon:X\to I$, with
  the conditions that $(X,\nabla,\eta)$ forms a commutative monoid, $(X,\Delta, \epsilon)$ forms a
  commutative comonoid and the diagrams
  \[
    \xymatrix@C-10pt{
      X\*X \ar[rr]^{1\*\Delta} \ar[dd]_{\Delta \* 1} \ar[dr]^{\nabla}
        && X\*X\*X \ar[dd]^{\nabla\*1}\\
      & X\ar[dr]^{\Delta}\\
      X\*X\*X\ar[rr]_{1\*\nabla}  && X\*X
    }\quad
    \xymatrix{
      A \ar[r]^{\Delta} \ar@{=}[ddr] & A\*A \ar[d]^{\epsilon\*1} \\
      & I\*A \ar[d]^{\usl}\\
      & A
    }\quad
    \xymatrix{
      A \ar[r]^{\inv{\usr}} \ar@{=}[ddr] & A\*I \ar[d]^{1\*\eta}\\
      & A\*A \ar[d]^{\nabla} \\
      & A
    }
  \]
  all commute. The Frobenius algebra is \emph{special} when $\Delta \nabla = 1_{X}$ and
  \emph{commutative} when $\Delta c_{X,X} = \Delta$. Note that special is sometimes referred to as
  \emph{separable}.
\end{definition}
\begin{definition}\label{def:daggerfrob}
  A Frobenius algebra in a dagger symmetric monoidal category where $\Delta = \dgr{\nabla}$ and
  $\epsilon=\dgr{\eta}$ is a $\dagger$\emph{-Frobenius algebra}.
\end{definition}
For an example of a $\dagger$-Frobenius algebra, consider a finite dimensional Hilbert space $H$
with an orthonormal basis $\{\ket{\phi_{i}}\}$ and define $\Delta:H\to H\*H: \ket{\phi_{i}}\mapsto
\ket{\phi_{i}} \* \ket{\phi_{i}}$ and $\epsilon : H\to \complex : \ket{\phi_{i}} \mapsto 1$. Then $(H,
\nabla=\dgr{\Delta}, \eta=\dgr{\epsilon}, \Delta, \epsilon)$ forms a commutative special
$\dagger$-Frobenius algebra.


In \cite{coeckeetal08:ortho}, Coecke et. al. give a correspondence between Frobenius algebras and
orthogonal bases in finite dimensional Hilbert spaces. An orthonormal basis for such a space
determines, as above, a special commutative $\dagger$-Frobenius algebra. To show the other
direction, given a commutative $\dagger$-Frobenius algebra, $(H,\nabla,\eta)$, for each element
$\alpha\in H$ define the right action of $\alpha$ as $R_{\alpha}:=(id\*\alpha)\, \nabla:H\to
H$. Note the use of the fact that elements $\alpha\in H$ can be considered as linear maps
$\alpha:\complex \to H:1\mapsto \ket{\alpha}$. The dagger of a right action is also a right action,
$\dgr{R_{\alpha}} = R_{\alpha'}$ where $\alpha'= \eta\, \nabla\, (id\* \dgr{\alpha})$, which is a
consequence of the Frobenius identities.

The $(\_)'$ construction is actually an involution:
\begin{eqnarray*}
  &(\alpha')' &= \eta \nabla (id \* \dgr{\alpha'}) \\
  && = u \nabla (id \* \dgr{(\eta \nabla (id \* \dgr{\alpha}))}\\
  && = u \nabla (id \* ( (id \* \alpha) \Delta \epsilon))\\
  && = (u \* \alpha) (\nabla \* id) (id \* \Delta) (id \*  \epsilon)\\
  && = (u \* \alpha) (id \* \Delta) (\nabla \* id) (id \*  \epsilon)\\
  && = (u \* \alpha)  (id \*  \epsilon)\\
  && = \alpha.
\end{eqnarray*}

\begin{lemma}\label{lemma:cstaralgebra}
  Any $\dagger$-Frobenius algebra in \fdh is a $C^{*}$-algebra.
\end{lemma}
\begin{proof}
  The endomorphism monoid of \fdh(H,H) is a $C^{*}$-algebra. From the proceeding, we have
  \[
    H \cong \fdh(\complex,H) \cong R_{[\fdh(\complex,H)]}\subseteq\fdh(H,H).
  \]
  This inherits the algebra structure from \fdh(H,H). Furthermore, since any finite dimensional
  involution-closed sub-algebra of a $C^{*}$-algebra is also a $C^{*}$-algebra, this shows the
  $\dagger$-Frobenius algebra is a $C^{*}$-algebra.
\end{proof}

Using the fact that the involution preserving homomorphisms from a finite dimensional commutative
$C^{*}$-algebra to $\complex$ form a basis for the dual of the underlying vector space, write these
homomorphisms as $\dgr{\phi_{i}}:H \to \complex$. Then their adjoints, $\phi_{i}:\complex\to H$ will form a
basis for the space $H$. These are the copyable elements in $H$.

This, together with continued applications of the Frobenius rules and linear algebra allow the
authors to prove the following Theorem.
\begin{theorem}
  Every commutative $\dagger$-Frobenius algebra in \fdh determines an orthogonal basis consisting
  of its copyable elements. Conversely, every orthogonal basis $\{\ket{\phi_{i}}\}_{i}$ determines
  a commutative $\dagger$-Frobenius algebra via \[\Delta:H\to H\*H: \ket{\phi_{i}}\mapsto
  \ket{\phi_{i}} \* \ket{\phi_{i}}\qquad\epsilon : H\to \complex : \ket{\phi_{i}} \mapsto 1\] and these
  constructions are inverse to each other.
\end{theorem}

% In \cite{coecke08structures}, Coecke et.al. build on the results of \cite{coeckeetal08:ortho}
% to start from a $\dagger$-symmetric monoidal category and construct the minimal machinery needed to
% model quantum and classical computations. For the rest of this section, $\cD$ will be assumed to be
% such a category, with $\*$ the monoid tensor and $I$ the unit of the monoid.

% \begin{definition}\label{def:compact_structure}
%   A compact structure on an object $A$ in the category $\cD$ is given by the object $A$, an object
%   $A^{*}$ called its \emph{dual} and the maps $\eta:I \to A^{*}\* A$, $\epsilon: A\* A^{*} \to I$
%   such that the diagrams
%   \[
%     \xymatrix@C+20pt{
%       A^{*} \ar[dr]^{id} \ar[d]_{\eta\*A^{*}} \\
%       A^{*} \*A\*A^{*}  \ar[r]_(.6){A^{*} \*\epsilon} & A^{*}
%     }
%     \text{ and }
%     \xymatrix@C+20pt{
%       A \ar[r]^(.4){A\*\eta} \ar[dr]_{id} & A\* A^{*}\* A \ar[d]^{\epsilon\*A}\\
%       & A
%     }
%   \]
%   commute.
% \end{definition}

% \begin{definition}\label{def:quantumstructure}
%   A \emph{quantum structure} is an object $A$ and map $\eta:I\to A\*A$ such that
%   $(A,A,\eta,\dgr{\eta})$ form a compact structure.
% \end{definition}
% Note that $A$ is self-dual in definition \ref{def:quantumstructure}.

% This allows the creation of the category $\cD_{q}$ which has as objects quantum structures and maps
% are the maps in $\cD$ between the objects in the quantum structures.

% In the category $\cD_{q}$, it is now possible to define the upper and lower $*$ operations on maps,
% such that $(f_{*})^{*}= (f^{*})_{*} = \dgr{f}$:
% \begin{eqnarray*}
% &f^{*} &:= (\eta_{A}\*1) (1 \* f\*1) (1\*\dgr{\eta}_{B}),\\
% &f_{*} &:= (\eta_{B}\*1) (1 \* \dgr{f}\*1) (1\*\dgr{\eta}_{A}).
% \end{eqnarray*}

% Next, define a classical structure on \cD.
% \begin{definition}\label{def:classicalstructure}
%   A \emph{classical structure} in \cD{} is an object $X$ together with two maps, $\Delta :X \to X\* X$,
%   $\epsilon:X\to I$ such that $(X,\dgr{\Delta},\dgr{\epsilon},\Delta,\epsilon)$ forms a special
%   Frobenius algebra.
% \end{definition}

% As above, this allows us to define $\cD_{c}$, the category whose objects are the classical
% structures of $\cD$. The maps in $\cD_{c}$ are given by the maps in $\cD$ between the
% objects of the classical structure.

% Note that a classical structure will induce a quantum structure, setting $\eta_{X}$ to be
% $\dgr{\epsilon_{X}}\, \Delta_{X}$.

% subsection frobenius_algebras (end)

%%% Local Variables:
%%% mode: latex
%%% TeX-master: "../../phd-thesis"
%%% End:
