\subsection{Frobenius Algebras} % (fold)
\label{sub:frobenius_algebras}
Frobenius algebras were originally defined as a finite dimensional algebra over a field together with a
non-degenerate pairing operation. Here we present the general categorical definition:

\begin{definition}\label{def:frobeniusalgebra}
  Given a symmetric monoidal category \cD, a \emph{Frobenius algebra} is an object $X$ of \cD and
  four maps, $\nabla :X\*X \to X$, $\eta: I \to X$, $\Delta: X\to X\* X$ and $\epsilon:X\to I$, with
  the conditions that $(X,\nabla,\eta)$ forms a commutative monoid, $(X,\Delta, \epsilon)$ forms a
  commutative comonoid and the diagrams
  \[
    \xymatrix@C-10pt{
      X\*X \ar[rr]^{1\*\Delta} \ar[dd]_{\Delta \* 1} \ar[dr]^{\nabla}
        && X\*X\*X \ar[dd]^{\nabla\*1}\\
      & X\ar[dr]^{\Delta}\\
      X\*X\*X\ar[rr]_{1\*\nabla}  && X\*X
    }\qquad
    \xymatrix@C+15pt{
      X \ar[r]^{\Delta} \ar@{=}[ddr] & X\*X \ar[d]^{\epsilon\*1} \\
      & I\*X \ar[d]^{\usl}\\
      & X
    }\qquad
    \xymatrix@C+15pt{
      X \ar[r]^{\inv{\usr}} \ar@{=}[ddr] & X\*I \ar[d]^{1\*\eta}\\
      & X\*X \ar[d]^{\nabla} \\
      & X
    }
  \]
  all commute. The Frobenius algebra is \emph{special} when $\Delta \nabla = 1_{X}$ and
  \emph{commutative} when $\Delta c_{X,X} = \Delta$. Note that special is sometimes referred to as
  \emph{separable}.
\end{definition}
\begin{definition}\label{def:daggerfrob}
  A Frobenius algebra in a dagger symmetric monoidal category where $\Delta = \dgr{\nabla}$ and
  $\epsilon=\dgr{\eta}$ is a $\dagger$\emph{-Frobenius algebra}.
\end{definition}
For an example of a $\dagger$-Frobenius algebra, consider a finite dimensional Hilbert space $H$
with an orthonormal basis $\{\ket{\phi_{i}}\}$ and define $\Delta:H\to H\*H: \ket{\phi_{i}}\mapsto
\ket{\phi_{i}} \* \ket{\phi_{i}}$ and $\epsilon : H\to \complex : \ket{\phi_{i}} \mapsto 1$. Then $(H,
\nabla=\dgr{\Delta}, \eta=\dgr{\epsilon}, \Delta, \epsilon)$ forms a commutative special
$\dagger$-Frobenius algebra.


In \cite{coeckeetal08:ortho}, Coecke, Pavlovi\'c and Vicary give a correspondence between Frobenius
algebras and orthogonal bases in finite dimensional Hilbert spaces. An orthonormal basis for such a
space determines, as above, a special commutative $\dagger$-Frobenius algebra. To show the other
direction, given a commutative $\dagger$-Frobenius algebra, $(H,\nabla,\eta)$, for each element
$\alpha\in H$ define the right action of $\alpha$ as $R_{\alpha}:=(id\*\alpha)\, \nabla:H\to
H$. Note the use of the fact that elements $\alpha\in H$ can be considered as linear maps
$\alpha:\complex \to H:1\mapsto \ket{\alpha}$. The dagger of a right action is also a right action,
$\dgr{R_{\alpha}} = R_{\alpha'}$ where $\alpha'= \eta\, \nabla\, (id\* \dgr{\alpha})$, which is a
consequence of the Frobenius identities.

The $(\_)'$ construction is actually an involution:
\begin{eqnarray*}
  &(\alpha')' &= \eta \nabla (id \* \dgr{\alpha'}) \\
  && = u \nabla (id \* \dgr{(\eta \nabla (id \* \dgr{\alpha}))}\\
  && = u \nabla (id \* ( (id \* \alpha) \Delta \epsilon))\\
  && = (u \* \alpha) (\nabla \* id) (id \* \Delta) (id \*  \epsilon)\\
  && = (u \* \alpha) (id \* \Delta) (\nabla \* id) (id \*  \epsilon)\\
  && = (u \* \alpha)  (id \*  \epsilon)\\
  && = \alpha.
\end{eqnarray*}

\begin{lemma}\label{lemma:cstaralgebra}
  Any $\dagger$-Frobenius algebra in \fdh is a $C^{*}$-algebra.
\end{lemma}
\begin{proof}
  The endomorphism monoid of \fdh(H,H) is a $C^{*}$-algebra. From the proceeding, we have
  \[
    H \cong \fdh(\complex,H) \cong R_{[\fdh(\complex,H)]}\subseteq\fdh(H,H).
  \]
  This inherits the algebra structure from \fdh(H,H). Furthermore, since any finite dimensional
  involution-closed sub-algebra of a $C^{*}$-algebra is also a $C^{*}$-algebra, this shows the
  $\dagger$-Frobenius algebra is a $C^{*}$-algebra.
\end{proof}

Using the fact that the involution preserving homomorphisms from a finite dimensional commutative
$C^{*}$-algebra to $\complex$ form a basis for the dual of the underlying vector space, write these
homomorphisms as $\dgr{\phi_{i}}:H \to \complex$. Then their adjoints, $\phi_{i}:\complex\to H$ will form a
basis for the space $H$. These are the copyable elements in $H$.

This, together with continued applications of the Frobenius rules and linear algebra allow Coecke,
Pavlovi\'c and Vicary to prove the following Theorem.
\begin{theorem}[Coecke, Pavolvi\'c, Vicary]\label{thm:cpv5.1}
  Every commutative $\dagger$-Frobenius algebra in \fdh determines an orthogonal basis consisting
  of its copyable elements. Conversely, every orthogonal basis $\{\ket{\phi_{i}}\}_{i}$ determines
  a commutative $\dagger$-Frobenius algebra via \[\Delta:H\to H\*H: \ket{\phi_{i}}\mapsto
  \ket{\phi_{i}} \* \ket{\phi_{i}}\qquad\epsilon : H\to \complex : \ket{\phi_{i}} \mapsto 1\] and these
  constructions are inverse to each other.
\end{theorem}

An interesting aspect of Theorem~\ref{thm:cpv5.1} is that it uses only algebraic structures to
determine a basis as the ``copyable elements''. In a quantum computation, the choice of basis
determines the copyable elements.

\begin{remark}
  Vicary, in \cite{vicary2011categorical}, further explores $\dagger$-Frobenius algebras to define
  \emph{involution monoids} on $\dagger$-monoidal categories with duals. This led to an alternate
  proof of Lemma~\ref{lemma:cstaralgebra} based on monoids. From this, Vicary shows that
  the category of commutative $\dagger$-monoids in \fdh, with monoid morphisms as maps, is equivalent
  to $\dual{\finsets}$ -- the dual of the category of finite sets. Of course, this is another way to
  move to the classical world from the quantum world.  Vicary goes on in
  \cite{mclarty1992elementary} to define a finite quantum Boolean topos and finite Boolean topos and
  show that the category of classical structures in a finite quantum Boolean topos is equivalent to
  a finite Boolean topos and in fact, every finite Boolean topos arises in this way.
\end{remark}
% subsection frobenius_algebras (end)

%%% Local Variables:
%%% mode: latex
%%% TeX-master: "../../phd-thesis"
%%% End:
