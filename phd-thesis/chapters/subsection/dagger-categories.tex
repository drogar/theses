%!TEX root = /Users/gilesb/UofC/thesis/phd-thesis/phd-thesis.tex
\subsection{Dagger categories}\label{ssec:daggercategories}

Although dagger categories were introduced in the context of compact closed categories, the concept
of a dagger is definable independently. This was first done in \cite{selinger05:dagger}.

\begin{definition}\label{def:daggercat}
  A \emph{dagger} on a category $D$ is a functor $\dagger:\cD^{op}\to \cD$, which is  involutive,
  that is, $\dgr{\dgr{f}} = f$ which is the identity on objects. A \emph{dagger category} is a
  category that has a dagger.
\end{definition}

Typically, the dagger is written as a superscript on the morphism. So, if $f:A\to B$ is a map in
\cD, then $\dgr{f}:B\to A$ is a map in \cD{} and is called the \emph{adjoint} of $f$. A map where
$f^{-1} = \dgr{f}$ is called \emph{unitary}. A map $f:A\to A$ with $f=\dgr{f}$ is called
\emph{self-adjoint} or \emph{Hermitian}.

\begin{definition}\label{def:daggersmc}
  A \emph{dagger symmetric monoidal category} is a symmetric monoidal category \cD{} with a dagger
  operator such that:
  \begin{enumerate}[{(}i{)}]
    \item For all maps $f:A\to B$ and $g:C\to D$, $\dgr{(f\*g)} = \dgr{f}\*\dgr{g}:B\*D \to A\* C$;\label{defitem:dagger_smc_one}
    \item The monoid structure isomorphisms $a_{A,B,C}:(A\*B)\* C\to A\*(B\*C)$, $u^l_{A}:I\*A\to
      A$, $u^r_{A}:A\*I \to A$ and  $c_{A,B}:A\*B \to B\*A$ are unitary.\label{defitem:dagger_smc_two}
  \end{enumerate}
\end{definition}


\begin{definition}\label{def:daggercompact}
  A \emph{dagger compact closed category} \cD{} is a dagger symmetric monoidal category
  that is compact closed where the diagram
  \[
    \xymatrix @C+20pt @R+10pt{
      I \ar[r]^{\epsilon^{\dagger}_{A}} \ar[dr]_{\eta_{A}} &A\*A^{*}\ar[d]^{c_{A,A^{*}}}\\
      &A^{*}\* A
    }
  \]
  commutes for all  objects $A$ in \cD.
\end{definition}

\begin{lemma}\label{lemma:daggerbiproducts}
If \cD{} is a dagger category with biproducts, with injections $in_{1},in_{2}$ and projections
$p_{1},p_{2}$, then the following are equivalent:
\begin{enumerate}[{(}i{)}]
  \item $\dgr{p_{i}} = in_{i}, i=1,2$, \label{ldpdgrpisq}
  \item $\dgr{(f\biproduct g)} = \dgr{f}\biproduct \dgr{g}$ and $\dgr{\Delta} = \nabla$,\label{ldpddeltisnab}
  \item $\dgr{\<f,g\>} = [\dgr{f},\dgr{g}]$,\label{ldpdcopisprod}
  \item The map $[\dgr{p_{1}},\dgr{p_{2}}]: \dgr{A} \biproduct \dgr{B} \to \dgr{(A\biproduct B)}$ is
    the identity map.\label{ldpcommute}
%the below diagram commutes:
%  \[
%    \xymatrix @C+20pt @R+10pt{
%      \dgr{A} \biproduct \dgr{B} \ar[d]_{id} \ar[dr]^{[\dgr{p_{1}},\dgr{p_{2}}]}\\
%      A\biproduct B\ar[r]_{id}&\dgr{(A\biproduct B)}.
%    }
%  \]
\end{enumerate}
\end{lemma}
\begin{proof}
  \begin{description}
    \item[\ref{ldpdgrpisq}$\implies$\ref{ldpddeltisnab}] To show $\dgr{\Delta} = \nabla$,
    draw the product cone for $\Delta$,
    \[
      \xymatrix {
        &A \ar[d]^{\Delta} \ar[dr]^{id} \ar[dl]_{id}\\
        A
         & A\biproduct A \ar[l]^{p_{1}}  \ar[r]_{p_{2}}
         & A
      }
    \]
    and apply the dagger functor to it. As $\dgr{p_{i}} = in_{i}$, and $\dagger$ is identity on
    objects, this is now a coproduct diagram and therefore $\dgr{\Delta} = \nabla$.

    For $\dgr{(f\biproduct g)} = \dgr{f}\biproduct\dgr{g}$, start with the diagram defining
    $f\biproduct g$ as a product of the arrows:
    \[
      \xymatrix{
        A\ar[d]_{f}  & A\biproduct B \ar[l]_{p_{1}} \ar[r]^{p_{2}} \ar[d]^{f\biproduct g}&A \ar[d]^{g}\\
        C & C\biproduct D \ar[l]^{p_{1}} \ar[r]_{p_{2}}  & D.
      }
    \]
    Then, apply the dagger functor to this diagram. This is now the diagram defining the
    co-product of maps and therefore $\dgr{(f\biproduct g)} = \dgr{f}\biproduct\dgr{g}$.
    \item[\ref{ldpddeltisnab}$\implies$\ref{ldpdcopisprod}] The calculation showing this is
      \begin{eqnarray*}
        &[\dgr{f},\dgr{g}] & = \nabla; (\dgr{f}\biproduct \dgr{g})\\
        & &=\dgr{\Delta}; (\dgr{f}\biproduct \dgr{g})\\
        & &=\dgr{\Delta}; \dgr{(f\biproduct g)}\\
        & & = \dgr{((f\biproduct g);\Delta)}\\
        & & = \dgr{\<f,g\>}
      \end{eqnarray*}
    \item[\ref{ldpdcopisprod}$\implies$\ref{ldpcommute}]
      Under the assumption,
      \[
        [\dgr{p_{1}},\dgr{p_{2}}] = \dgr{\<p_{1},p_{2}\>}=\dgr{id}=id.
      \]
    \item[\ref{ldpcommute}$\implies$\ref{ldpdgrpisq}] As $[in_{1},in_{2}]:\dgr{A} \biproduct \dgr{B}
      \to \dgr{A} \biproduct \dgr{B} = id = [\dgr{p_{1}},\dgr{p_{2}}]$, we immediately have
      $\dgr{p_{1}} = in_{1}$ and $\dgr{p_{2}} = in_{2}$.
%
%Using the injections and under
%    the assumption, the following diagram commutes:
%      \[
%        \xymatrix @C+20pt @R+10pt{
%          \dgr{A} \biproduct \dgr{B} \ar[d]_{id} \ar[dr]^{[\dgr{p_{1}},\dgr{p_{2}}]}\ar[r]^{[in_{1},in_{2}]}
%            & \dgr{A} \biproduct \dgr{B} \ar[d]^{id}\\
%          A\biproduct B\ar[r]_{id}&\dgr{(A\biproduct B)}
%        }
%      \]
%      and therefore,
  \end{description}
\end{proof}

\begin{definition} \label{def:biproductdaggerccc}
  A \emph{biproduct dagger compact closed category} is a dagger compact closed category with
  biproducts where the conditions of lemma \ref{lemma:daggerbiproducts} hold.
\end{definition}
\subsection{Examples of dagger categories}

\begin{example}[\fdh]\label{ex:fdhilbert_is_dagger_category}
The category of finite dimensional Hilbert spaces is the motivating example for
the creation of the dagger and is, in fact, a biproduct dagger compact closed category. The
biproduct is the direct sum of Hilbert spaces and the tensor for compact closure is the standard
tensor of Hilbert spaces. The dual $H^{*}$ of a space $H$ is the space of all continuous linear
functions from $H$ to the base field. The dagger is defined via the adjoint as being the unique map
$\dgr{f}:B\to A$ such that $\<f a|b\> = \<a | \dgr{f} b\>$ for all $a\in A, b\in B$.
\end{example}

\begin{example}[\rel]\label{ex:rel_is_dagger_category}
The category \rel of sets and relations has the tensor $S\*T = S\times T$, the
Cartesian product and the biproduct $S\biproduct T = S+T$, the disjoint union. This is compact
closed under $A^{*} = A$ and the dagger is the ${}^*$ operation, the relational converse. That is,
if the relation $R=\{(s,t)|s\in S, t\in T\}:S\to T$, then $\dgr{R}=R^*=\{(t,s)|(s,t)\in R\}$.
\end{example}

\begin{example}[Inverse categories]\label{ex:inverse_category_is_dagger_category}
An inverse category \X is also a dagger category when the dagger is defined as the partial inverse.
The unitary maps are the total maps. When the inverse category \X is also a
symmetric monoidal category where the monoid $\*$ is actually a restriction bi-functor, then \X is
a dagger symmetric monoidal category.

Requirement \ref{defitem:dagger_smc_one} of Definition~\ref{def:daggersmc}  is fulfilled, as
\[
  (f\*g) \inv{(f\*g)} = \rst{f\*g}=\rst{f} \*\rst{g} =
   f\inv{f} \* g \inv{g} = (f\*g) (\inv{f} \* \inv{g})
\]
and since the partial inverse of $f\*g$ is unique, $\inv{(f\*g)} = \inv{f} \* \inv{g}$.
Requirement \ref{defitem:dagger_smc_two} is that the structure isomorphisms are unitary. This is, of
course, true as each of them are isomorphisms, hence total and therefore unitary.
\end{example}
%%% Local Variables:
%%% mode: latex
%%% TeX-master: "../../phd-thesis"
%%% End:
