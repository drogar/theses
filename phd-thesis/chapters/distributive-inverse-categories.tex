%!TEX root = /Users/gilesb/UofC/thesis/phd-thesis/phd-thesis.tex

\chapter{Distributive inverse categories} % (fold)
\label{chap:distributive_inverse_categories}
We now consider inverse categories with both a disjoint sum and inverse product, where the inverse
product distributes overs the disjoint sum in a specific way. This chapter will show that the
Cartesian Completion of such a category is a distributive restriction category where the product
distributes over the coproduct.

\section{Distributive restriction categories} % (fold)
\label{sec:distributive_restriction_categories}

\begin{definition}\label{def:distributive_category}
  A Cartesian category $\B$ with coproducts is called a
  \emph{distributive}\cite{cockett1993-introduction-distributive} category when
  \[
    (A\times B) + (A\times C) \cong A \times (B+ C).
  \]
\end{definition}

\begin{definition}\label{def:distributive_restriction_category}
  A Cartesian restriction category $\D$ with a restriction zero and coproducts is called a
  \emph{distributive restriction category}\cite{cockettlack2004:restcategories3}
  when there is an isomorphism $\rho$ such that
  \[
    A\times (B+C) \xrightarrow{\rho} (A\times B) +(A\times C).
  \]
\end{definition}

If $\D$ is a distributive restriction category, then Total($\D$) is a distributive
category as in Definition~\ref{def:distributive_category}.

% subsection distributive_restriction_categories (end)

\section{Distributive inverse categories} % (fold)
\label{sec:distributive_inverse_categories}
% \note{
% The following two \emph{should} be equivalent. Either
% \begin{itemize}
%   \item $\*$ preserves sums
%   \item $\*$ preserves disjoint
% \end{itemize}
%
% Break down the map into disjoint items, if equivalent on pieces, then they are
% equivalent.
% }
\begin{definition}\label{def:distributive_inverse_category}
  A \emph{distributive inverse category} \D consists of the following:
  \begin{itemize}
    \item \D is an inverse category;
    \item \D has an inverse product with tensor $\*$, per Definition~\ref{def:inverse_product};
    \item \D has a disjoint sum tensor, $\+$, per Definition~\ref{def:disjoint_sum_tensor} and
    \item There is a family of isomorphisms, $d$, such that
    \begin{equation}
      \xymatrix{
      && A\*B\\
      A \* (B \+ C) \ar[rr]^{d} \ar[urr]^{1\*\icpa} \ar[drr]_{1\*\icpb}
        && (A\*B) \+ (A \* C) \ar[u]_{\icpa} \ar[d]^{\icpb}\\
      && A\*C,
      }\label{eq:inverse_distributive_laws}
    \end{equation}
    commutes in \D for any choices of objects $A, B, C$.
  \end{itemize}
\end{definition}

\begin{example}[\pinj is a distributive inverse category]
  The following defines the isomorphism $d$ of Diagram~\ref{eq:inverse_distributive_laws}:
  \begin{equation}
    d((a,(x,n)) = \begin{cases}
      ((a,x),1) & n = 1\\
      ((a,x),2) & n = 2.
      \end{cases}\label{eq:pinj_distributive_isomporphism}
    \end{equation}
\end{example}
Note that as we are operating in an inverse category, we also have the inverse of
diagram~\ref{eq:inverse_distributive_laws} available to us. That is,
\begin{equation}
  \xymatrix{
  A\*B \ar[d]_{\cpa} \ar[drr]^{1\*\cpa}\\
  (A\*B) \+ (A \* C)  \ar[rr]^{\inv{d}}  &&A \* (B \+ C) \\
  A\*C\ar[u]^{\cpb} \ar[urr]_{1\*\cpb}
  }\label{eq:inverse_distributive_laws_reversed}
\end{equation}
is also a commuting diagram in \D.

\begin{definition}\label{def:join_of_perp_rst_maps}
  Suppose \X is an inverse category with a disjoint sum tensor $\+$ and a restriction zero. Then
  for maps $f:A \to B$ and $g:A \to C$ with $\rst{f} \perp \rst{g}$, define the map $[f,g]:A \to
  B\+C$ as $(f \cpa) \djoin (g \cpb)$. This is well defined as $\wrg{\cpa} \perp \wrg{\cpb}$ and
  therefore by \axiom{Dis}{7}, $f\cpa \perp g\cpb$.
\end{definition}

\begin{lemma}
  Suppose \X is an inverse category \X\ with:
  \begin{itemize}
    \item a disjoint sum tensor $\+$,
    \item a restriction zero, and
    \item an inverse  product $\*$ which distributes over disjoint joins, (that is,
      $f\*(g\djoin h) = (f\*g) \djoin (f\* h)$).
  \end{itemize}
  Then, \X is an inverse distributive  category.
\end{lemma}
\begin{proof}
  By assumption, we have the first three items of
  Definition~\ref{def:distributive_inverse_category}. Therefore, we need to construct an
  isomorphism $d$ such that diagram~\ref{eq:inverse_distributive_laws} commutes. We claim that the
  map $d = [1\* \icpa, 1\* \icpb]$ does this.

  First, note that the typing of $d$ is correct. By Definition~\ref{def:join_of_perp_rst_maps},
  \[
    d = ((1 \* \icpa) \cpa )\djoin ((1\*\icpb) \cpb) : A \* (B \+ C) \to (A\*B)\+(A\*C)
  \]
  as
  \begin{align*}
      A \* (B \+ C) & \xrightarrow{(1\*\icpa)} A\* B \xrightarrow{\icpa} (A\*B)\+(A\*C),\\
      A \* (B \+ C) & \xrightarrow{(1\*\icpb)} A\* C \xrightarrow{\icpb} (A\*B)\+(A\*C).
  \end{align*}

  Next, we need to show $d$ is an isomorphism. We will do this by showing both $\rst{d} = 1$ and
  $\rst{\inv{d}} = 1$.
  As a consequence of Lemma~\ref{lem:join_is_associative_and_commutative_monoid}, we know
  the inverse of $d$ is
  \[
    \inv{((1 \* \icpa) \cpa )}\djoin \inv{((1\*\icpb) \cpb)} =
      ( \icpa(1 \* \cpa) ) \djoin ( \icpb(1\*\cpb)).
  \]

  Having $\*$ distribute over the disjoint sum means that for any maps $f,h,k$ with $h\perp k$, we
  have $f \* (h\djoin k) = (f\*h) \djoin (f\*k)$. We use this in the calculation of the restriction
  of $d$:
  \begin{align*}
    \rst{((1 \* \icpa) \cpa )\djoin ((1\*\icpb) \cpb)}
      &= \rst{((1 \* \icpa) \cpa )}\djoin \rst{((1\*\icpb) \cpb)} \\
      &= (1\* \rst{\icpa}) \djoin (1\* \rst{\icpb}) \\
      &= (1\* (\rst{\icpa} \djoin \rst{\icpb}) \\
      &= 1\* ((1\+0) \djoin (0\+1) )\\
      &=1 \* 1 = 1.
  \end{align*}
  The calculation for $\rst{\inv{d}}$ also shows it is $1$:
  \begin{align*}
    \rst{( \icpa(1 \* \cpa) ) \djoin ( \icpb(1\*\cpb))}
      &= \rst{(\icpa(1 \* \cpa) )}\djoin \rst{(\icpb(1\*\cpb))} \\
      &= \rst{(\icpa\rst{(1 \* \cpa)} )}\djoin \rst{(\icpb\rst{(1\*\cpb)})} \\
      &= \rst{(\icpa)}\djoin \rst{(\icpb)}\\
      &= (1\+0) \djoin (0\+1) \\
      &= 1.
  \end{align*}

  Hence, $[1\* \icpa, 1\* \icpb]$ is an isomorphism.
  Finally, we must show that diagram~\ref{eq:inverse_distributive_laws} commutes:
  \begin{align*}
    d \icpa & = \left( ((1 \* \icpa) \cpa )\djoin ((1\*\icpb) \cpb) \right)\icpa \\
     & = \left(((1 \* \icpa) \cpa )\icpa \right) \djoin \left( ((1\*\icpb) \cpb) \icpa\right) \\
     & = \left((1 \* \icpa) 1 \right) \djoin \left( (1\*\icpb) 0\right) \\
     & = (1 \* \icpa) \djoin 0\\
     & = 1 \* \icpa
  \end{align*}
  and
  \begin{align*}
    d \icpb & = \left( ((1 \* \icpa) \cpa )\djoin ((1\*\icpb) \cpb) \right)\icpb \\
     & = \left(((1 \* \icpa) \cpa )\icpb \right) \djoin \left( ((1\*\icpb) \cpb) \icpb\right) \\
     & = 0 \djoin (1 \* \icpb)\\
     & = 1 \* \icpb.
  \end{align*}
  This shows the fourth condition is satisfied and $\X$ is a distributive inverse category.
\end{proof}

We have seen that a second tensor distributing over the disjoint joins implies that we have an
inverse distributive category. We now show the converse is true.
\begin{lemma}\label{lem:distributive_means_distribute_over_triangles}
  Given an inverse distributive category \X, then $h\*(f\tjdown g) = (h\* f) \tjdown (h\* g)$
  whenever $f\tjdown g$ exists and $h\*(f\tjup g) = (h\* f) \tjup (h\* g)$ whenever
  $f\tjup g$ exists.
\end{lemma}
\begin{proof}
  Let $h:A \to C$, $f:B \to D$ and $g: B \to E$. Consider the following diagram:
  \[
    \xymatrix@R+20pt@C+10pt{
      & && B\*E\\
      A\*C \ar[rr]^{h\*(f\tjdown g)} \ar[drrr]_{h\*g} \ar[urrr]^{h\*f}
        && B \* (D \+ E) \ar[r]^{\cong} \ar[ur]_{1\*\icpa} \ar[dr]^{1\*\icpb}
        & (B\*D) \+ (B \* E) \ar[d]^{\icpb} \ar[u]_{\icpa}\\
      & && B\*D.
    }
  \]
  The two leftmost  triangles commute by the diagram for $f\tjdown g$. The right hand triangles
  commute as per Definition~\ref{def:distributive_inverse_category}. By the uniqueness of the
  $\tjdown$ operation we see  $h\*(f\tjdown g) = (h\* f)\tjdown(h\* g)$,

  The argument for showing $h\*(f\tjup g) = (h\* f)\tjup(h\* g)$ follows the same pattern.
  %TODO: Is the isomorphism important? Do we need to do something with it?
\end{proof}
\begin{lemma}\label{lem:distributive_means_distribute_over_join}
  Given an inverse distributive category \X, then $\*$ distributes over the disjoint join.
\end{lemma}
\begin{proof}
  First recall the definition of $f\djoin g = (\rst{f}\tjdown\rst{g})(f\tjup g)$. In order
  to show $h\*(f\djoin g) = (h\*f)\djoin (h\*g)$, we need to show that
  \begin{equation}
    h\*(\rst{f}\tjdown\rst{g})(f\tjup g) = (\rst{h\*f}\tjdown\rst{h\*g})(h\*f\tjup h\*g).
    \label{eq:distribute_over_left_right}
  \end{equation}

  Since $h\*(\rst{f}\tjdown\rst{g})(f\tjup g) =
  (\rst{h}\*(\rst{f}\tjdown\rst{g}))(h\*(f\tjup g))$, Equation~\ref{eq:distribute_over_left_right}
  follows directly from  Lemma~\ref{lem:distributive_means_distribute_over_triangles} and the fact
  that $\*$ is a  restriction functor.
\end{proof}

\begin{corollary}\label{cor:inverse_product_and_disjointness}
  Suppose we have an inverse distributive category \X. Then,
  \begin{enumerate}[{(}i{)}]
    \item if $f \perp g$, then $h\*f \perp h\*  g$ for any $h$,
    \item  if  $f\perp g : A\to B$ and $h \perp k: C\to D$, then $(f\*h)\perp (g\*k)$.
   \end{enumerate}
\end{corollary}
\begin{proof}
  \prepprooflist
  \begin{enumerate}[{(}i{)}]
    \item As $f\perp g$, we have $f\tjup g$ and $f\tjdown g$. By
      Lemma~\ref{lem:distributive_means_distribute_over_triangles}, both $h\*f \tjup h\*g$ and
      $h\*f \tjdown h\* g$ exist and therefore $h\*f \perp h\*g$.
    \item By the previous item, we have that $((f\djoin g)\*h) \perp ((f\djoin g)\* k)$. Then, by
      \axiom{DJ}{1} and \axiom{Dis}{3} we have $(f\*h)\perp (g\*k)$.
  \end{enumerate}
\end{proof}
% subsection distributive_inverse_categories (end)

\section{Discrete inverse categories with disjoint sums} % (fold)
\label{sec:discrete_inverse_categories_with_disjoint_sums}

We now consider the case where we have a discrete inverse category with inverse product
 $\*$ and a disjoint sum $\+$, where the $\*$ tensor preserves the disjoint join.

A map in \Xt is related to a map in \X in the following way:
\[
  \infer{A\xrightarrow{f} B\*C \text{ in }\X.}{A \xrightarrow{\ (f,C)\ } B \text{ in } \Xt}
\]

Our goal is to show that a disjoint sum in a distributive inverse category becomes a coproduct
in \Xt.

\begin{lemma}\label{lem:x_tilde_has_a_restriction_zero}
  Given $\X$ is a distributive inverse category, then $\Xt$, the discrete Cartesian restriction
  category created from $\X$, has a restriction zero.
\end{lemma}
\begin{proof}
  Recall from Theorem~\ref{thm:discrete_inverse_categories_are_equivalent_to_discrete_restriction_categories}
  that $\X$ is equivalent as a category to $\Xt$ under the identity on objects
  functor
  \[
    {\raisebox{-18pt}{$\wtf:\X \to \Xt;$}} \quad
    \xymatrix{A\ar[d]^{f}\\B}\ \raisebox{-18pt}{$\mapsto$}\, \xymatrix{A\ar[d]^{(f\inv{\usr},1)}\\B}
    \,\raisebox{-18pt}{.}
  \]

  In $\X$, we know $0$ is a terminal and initial object, with maps
  $A \xrightarrow{t_A} 0 $ and $0 \xrightarrow{z_A} A$, where  $\rst{0_{A,A}} = 0_{A,A} = t_A z_A$.

  First we note that $0$ is both initial and terminal in $\Xt$, with the terminal maps being
  $\wtf(t_A)$ and initial maps being $\wtf(z_A)$.

  As was also shown in
  Theorem~\ref{thm:discrete_inverse_categories_are_equivalent_to_discrete_restriction_categories},
$\wtf$ is a restriction functor, so in $\Xt$ we have
  \[
    0_{A,A} = \wtf(t_A)\wtf(z_A) = \wtf(t_A z_A) = \wtf(0_{A,A}) = \wtf(\rst{0_{A,A}}) =
    \rst{\wtf(0_{A,A})} = \rst{0_{A,A}}.
  \]
  Hence, $0_{A,A}$ is a restriction zero in $\Xt$.
\end{proof}

\begin{lemma}\label{lem:maps_from_disjoint_sums_have_disjoint_equivalences}
  In a distributive inverse category \X, the following hold:
  \begin{enumerate}[{(}i{)}]
    \item Given $f : A \to Y\*C$, we can construct $f' : A \to Y \* (C\+D)$  for some object $D$ such
      that $f \xequiv{} f'$.\label{lemitem:construct_f_prime}
    \item Given $f : A \to Y \* C$, $g : A\to Y \* D$, then the $f' : A \to Y \* (C\+D)$,
      $g' : B \to Y \* (C\+D)$ as constructed in  \ref{lemitem:construct_f_prime} satisfies $\icpa f'
      \perp \icpb g'$.\label{lemitem:constructed_are_disjoint}
  \end{enumerate}
\end{lemma}
\begin{proof}
  \prepprooflist
  \begin{enumerate}[{(}i{)}]
  \item Set $f' = f(1\*\cpa)$.
    To show $f\xequiv{} f'$, we must first show their restriction is the same:
    \[
      \rst{f(1\*\cpa)} = \rst{f\rst{(1\*\cpa)}} = \rst{f 1} =  \rst{f}.
    \]
    The mediating map between $f$ and $f'$ is, of course, $1\*\cpa$:
    \[
    \xymatrix @C=20pt @R=20pt{
      &  Y \* C \ar@{.>}[dd]^{1\*\cpa}\\
      A \ar[ur]^f \ar[dr]_{f'} \\
      & Y\*(C\+ D).
    }
    \]
  By the same reasoning we may also create $f':A\to Y\*(D\+C)$ by setting $f' = f(1\*\cpb)$.
  \item First note we have $\icpa f',\,\icpb g': A\+B \to Y\*(C\+D)$. In order to show  $\icpa f'
    \perp \icpb g'$, we will proceed by showing their restrictions and ranges are disjoint.
    As $\rst{\inv{\cpa}} \perp \rst{\inv{\cpb}}$ and $\rst{\inv{\cpa}f'} \le \rst{\inv{\cpa}}$ and
    $\rst{\inv{\cpb}g'} \le \rst{\inv{\cpb}}$, we immediately have
    $\rst{\inv{\cpa}f'} \perp \rst{\inv{\cpb}g'}$.

    For the ranges, we have
    \begin{align*}
      \wrg{\inv{\cpa} (f (1\*\cpa))} &=
      \rst{((1\*\icpa)\inv{f})\cpa}\\
      &= \rst{((1\*\icpa)\inv{f}}\\
      &\le \rst{(1\*\icpa)}
    \end{align*}
    and similarly
    \[
      \wrg{\inv{\cpb} g'} \le \rst{(1\*\icpb)}.
    \]

    Using Lemma~\ref{lem:disjointness_various}
    we know that $\rst{(\icpa)}\perp \rst{(\icpb)}$. From
    Corollary~\ref{cor:inverse_product_and_disjointness}
    we conclude that $\rst{(1\*\icpa)}\perp \rst{(1\*\icpb)}$ and giving us
    $\wrg{\inv{\cpa}f'} \perp \wrg{\inv{\cpb}g'}$ and therefore $\inv{\cpa}f' \perp \inv{\cpb}g'$.
  \end{enumerate}
\end{proof}
\begin{proposition}\label{thm:x_tilde_has_coproducts_if_x_is_inverse_distributive_category}
  Given \X is a distributive inverse category, then the category \Xt has coproducts.
\end{proposition}
\begin{proof}
  The tensor object $A\+B$ in \X will become the coproduct of $A, B$ in \Xt.

  The injection maps of the coproduct are $i_1 =(\cpa \inv{\usr}, 1)$ and
  $i_2 = (\cpb \inv{\usr}, 1)$.

  Consider the following diagram in \Xt:
  \[
    \xymatrix{
      A \ar[drrr]^{(f,C)} \ar[dr]_{i_1} \\
      & A\+B \ar@{.>}[rr]^h & & Y. \\
      B \ar[urrr]_{(g,D)} \ar[ur]^{i_2}
    }
  \]

  In \X, this comes from the diagram:
  \begin{equation}
    \xymatrix{
      A \ar[dr]_{\cpa} \ar[rr]^f && Y\*C\\
      & A\+B\\
      B \ar[ur]^{\cpb} \ar[rr]^g && Y\*D\\
    }\label{diag:pre-coproduct}
  \end{equation}
  where the extraneous unit isomorphisms are removed.

  This corresponds to the conditions of
  Lemma~\ref{lem:maps_from_disjoint_sums_have_disjoint_equivalences}. Hence by that lemma we may
  revise Diagram~\ref{diag:pre-coproduct} as
  \begin{equation}
    \xymatrix@R+20pt {
      A \ar[dr]^{\cpa} \ar[drrrrr]^{f'}\\
      &A\+B \ar[rrrr]^(.4){\icpa f' \djoin \icpb g'} &&&& Y\*(C\+D)\\
      B \ar[ur]^{\cpb} \ar[urrrrr]_{g'}\\
    }\label{diag:pre-coproduct-with-common-map}
  \end{equation}
  where $f'$ and $g'$ are respectively equivalent to $f,g$.

  Lifting Diagram~\ref{diag:pre-coproduct-with-common-map} to \X, we see this corresponds to the
  desired coproduct diagram, where  $h$ in \Xt is the map $(\icpa f' \djoin \icpb g', (C\+D))$.

  By construction, in \X, we have
  \[
    \cpa (\icpa f' \djoin \cpa\icpb g') = (\cpa\icpa f') \djoin (\cpa\icpb g') = f' \djoin 0 = f'
  \]
  and
  \[
    \cpb (\icpa f' \djoin \cpa\icpb g') = g'.
  \]
  Hence, in \Xt, we have  $(i_1 \usr, 1) h = f$ and $(i_2 \usr, 1) h = g$.

  All that remains to be shown is that $h$ is unique.

  Suppose there is another $(k,E)$ in \Xt such that it satisfies the coproduct properties, i.e.,
  that $i_1 (k,E) = (f',C\+D)$ and $i_2 (k,E) = (g',C\+D)$. In \X, $k:A\+B \to Y \* E$ and we have
  \begin{align*}
    \cpa k &\xequiv{} f'\quad\text{and}\\
    \cpb k &\xequiv{} g'.
  \end{align*}
  Since equivalence is a transitive relation, this means we have
  \begin{align*}
    f &\xequiv{q_1} \cpa k\quad\text{and}\\
    g &\xequiv{q_2} \cpb k,
  \end{align*}
  where the maps $q_1: Y\*C \to Y\*E$ and $q_2:Y\*D \to Y\*E$ fulfill the respective equivalence diagrams.

  Explicitly for $k, f$ and $q_1$, this gives us:
  \[
    \xymatrix @C=20pt @R=20pt{
      & Y \* C \ar@{.>}[dd]^{q_1}\\
      A \ar[ur]^{f} \ar[dr]_{\cpa k} \\
      & Y\*E.
    }
  \]

  Now, we turn out attention to showing that $k \xequiv{} h = \icpa f' \djoin \icpb g'$. Consider
  \begin{equation}
    \xymatrix @C=35pt @R=20pt{
      & Y \* (C\+D)\ar@{.>}[dd]^{t}\\
      A \ar[ur]^(.4){\icpa f' \djoin \icpb g'\ \ \,} \ar[dr]_{k} \\
      & Y\*E.
    }\label{diag:k-equiv-f-djoin-g}
  \end{equation}

  As \X is an inverse category, we know there is a map $t$ that makes this diagram commute,
  namely $t =  \inv{(\icpa f' \djoin \icpb g')}k$. However, we must show this is in $\dmap{Y}$.

  Next, recalling Definition~\ref{def:up_triangle_and_down_triangle} consider
  \begin{equation*}
    \xymatrix@C+0pt @R+20pt{
      Y\*C\ar[rr]^{1\*\cpa} \ar[drr]_{q_1} &&Y\*(C\+D)   \ar@{.>}[d]|-{q_1\tjup q_2}
        && Y\*D \ar[ll]_{1\*\cpb} \ar[dll]^{q_2}\\
      &&Y\*E.
      }%\label{diag:q1_q2_tjup}
  \end{equation*}
  The map $q_1\tjup q_2$ exists iff $\wrg{q_1} \perp \wrg{q_2}$. But, the map $t$ from above
  does make the diagram commute. This can be shown by
  \begin{align*}
    (1\*\cpa)& \inv{(\icpa f' \djoin \icpb g')}k\\
    &=  (1\*\cpa)\inv{(\icpa f(1\*\cpa) \djoin \icpb g(1\*\cpb))}k\\
    &=  (1\*\cpa)((1\*\icpa) \inv{f}\cpa \djoin (1\*\icpb) \inv{g}\cpb) k \\
    &=  ((1\*1) \inv{f}\cpa \djoin (1\*0) \inv{g}\cpb) k \\ % as Y\*0
                                % Y\*0 = 0
    &=  \inv{f} \cpa k\\
    &= q_1
  \end{align*}
  and similarly for $q_2$. Therefore,  $\wrg{q_1} \perp \wrg{q_2}$.

  This means we may form the map $q_1\tjup q_2:Y\*(C\+D) \to Y\*E$. But then the map
  $(q_1\tjup  q_2)$  makes Diagram~\ref{diag:k-equiv-f-djoin-g} commute. At the same time,
  $q_1,q_2\in\dmap{Y}$, but this means $q_1\tjup q_2 \in \dmap{Y}$ as
  \[
    \raisebox{40pt}{$q_1=$ }
    \begin{tikzpicture}
      \node at (0,2.75) (s) {};
      \node at (.75,2.75) (s2) {};
      \node at (0,2.25) [delta] (d1) {};
      \node at (.5,1.75) [map] (c) {$\scriptstyle 1\*\cpa$};
      \node at (.5,1) [map] (q) {$\scriptstyle q_1\tjup q_2$};
      \node at (0, .5) [nabla] (n1) {};
      \node at (0,0) (end) {};
      \node at (.75,0) (end2) {};
      \draw [] (s) to (d1);
      \draw [] (s2) to[out=270,in=55] (c);
      \draw [] (d1) to[out=305,in=125] (c);
      \draw [] (d1) to[out=235,in=125] (n1);
      \draw [] (c) to[out=305,in=55] (q);
      \draw [] (c) to[out=235,in=125] (q);
      \draw [] (q) to[out=235,in=55] (n1);
      \draw [] (q) to[out=305,in=90] (end2);
      \draw (n1) to (end);
    \end{tikzpicture}
    \,\raisebox{40pt}{$=$}
    \begin{tikzpicture}
      \node at (0,3.75) (s) {};
      \node at (.75,3.75) (s2) {};
      \node at (0,3.25) [delta] (d1) {};
      \node at (0.25,2.75) [delta] (d2) {};
      \node at (.75,2.25) [map] (c) {$\scriptstyle 1\*\cpa$};
      \node at (0.25, 1.75) [nabla] (n2) {};
      \node at (.5,1) [map] (q) {$\scriptstyle q_1\tjup q_2$};
      \node at (0, .5) [nabla] (n1) {};
      \node at (0,0) (end) {};
      \node at (.75,0) (end2) {};
      \draw [] (s) to (d1);
      \draw [] (s2) to[out=270,in=55] (c);
      \draw [] (d1) to[out=305,in=90] (d2);
      \draw [] (d1) to[out=235,in=125] (n1);
      \draw [] (d2) to[out=305,in=125] (c);
      \draw [] (d2) to[out=235,in=125] (n2);
      \draw [] (c) to[out=305,in=55] (q);
      \draw [] (c) to[out=235,in=55] (n2);
      \draw (n2) to[out=270,in=125] (q);
      \draw [] (q) to[out=235,in=55] (n1);
      \draw [] (q) to[out=305,in=90] (end2);
      \draw (n1) to (end);
    \end{tikzpicture}
    \,\raisebox{40pt}{$=$}
    \begin{tikzpicture}
      \node at (0,3.75) (s) {};
      \node at (.75,3.75) (s2) {};
      \node at (0,3.25) [delta] (d1) {};
      \node at (-0.25,2.75) [delta] (d2) {};
      \node at (.5,2.25) [map] (c) {$\scriptstyle 1\*\cpa$};
      \node at (0, 1.75) [nabla] (n2) {};
      \node at (.5,1) [map] (q) {$\scriptstyle q_1\tjup q_2$};
      \node at (0, .5) [nabla] (n1) {};
      \node at (0,0) (end) {};
      \node at (.75,0) (end2) {};
      \draw [] (s) to (d1);
      \draw [] (s2) to[out=270,in=55] (c);
      \draw [] (d1) to[out=235,in=90] (d2);
      \draw [] (d1) to[out=305,in=125] (c);
      \draw [] (d2) to[out=305,in=125] (n2);
      \draw [] (d2) to[out=235,in=125] (n1);
      \draw [] (c) to[out=305,in=55] (q);
      \draw [] (c) to[out=235,in=55] (n2);
      \draw (n2) to[out=270,in=125] (q);
      \draw [] (q) to[out=235,in=55] (n1);
      \draw [] (q) to[out=305,in=90] (end2);
      \draw (n1) to (end);
    \end{tikzpicture}
    \,\raisebox{40pt}{$=$}
    \begin{tikzpicture}
      \node at (0,3.75) (s) {};
      \node at (.75,3.75) (s2) {};
      \node at (0,3.25) [delta] (d1) {};
      \node at (.5,2.75) [map] (c) {$\scriptstyle 1\*\cpa$};
      \node at (-0.25,2.25) [delta] (d2) {};
      \node at (0, 1.75) [nabla] (n2) {};
      \node at (.5,1) [map] (q) {$\scriptstyle q_1\tjup q_2$};
      \node at (0, .5) [nabla] (n1) {};
      \node at (0,0) (end) {};
      \node at (.75,0) (end2) {};
      \draw [] (s) to (d1);
      \draw [] (s2) to[out=270,in=55] (c);
      \draw [] (d1) to[out=235,in=90] (d2);
      \draw [] (d1) to[out=305,in=125] (c);
      \draw [] (d2) to[out=305,in=125] (n2);
      \draw [] (d2) to[out=235,in=125] (n1);
      \draw [] (c) to[out=305,in=55] (q);
      \draw [] (c) to[out=235,in=55] (n2);
      \draw (n2) to[out=270,in=125] (q);
      \draw [] (q) to[out=235,in=55] (n1);
      \draw [] (q) to[out=305,in=90] (end2);
      \draw (n1) to (end);
    \end{tikzpicture}
    \,\raisebox{40pt}{$=$}
    \begin{tikzpicture}
      \node at (0,3.75) (s) {};
      \node at (.75,3.75) (s2) {};
      \node at (0,3.25) [delta] (d1) {};
      \node at (.5,2.75) [map] (c) {$\scriptstyle 1\*\cpa$};
      \node at (0, 2.25) [nabla] (n2) {};
      \node at (0,1.5) [delta] (d2) {};
      \node at (.5,1) [map] (q) {$\scriptstyle q_1\tjup q_2$};
      \node at (0, .5) [nabla] (n1) {};
      \node at (0,0) (end) {};
      \node at (.75,0) (end2) {};
      \draw [] (s) to (d1);
      \draw [] (s2) to[out=270,in=55] (c);
      \draw [] (d1) to[out=235,in=125] (n2);
      \draw [] (d1) to[out=305,in=125] (c);
      \draw [] (c) to[out=305,in=55] (q);
      \draw [] (c) to[out=235,in=55] (n2);
      \draw (n2) to[out=270,in=90] (d2);
      \draw [] (d2) to[out=305,in=125] (q);
      \draw [] (d2) to[out=235,in=125] (n1);
      \draw [] (q) to[out=235,in=55] (n1);
      \draw [] (q) to[out=305,in=90] (end2);
      \draw (n1) to (end);
    \end{tikzpicture}
    \,\raisebox{40pt}{$=$}
    \begin{tikzpicture}
      \node at (0,2.75) (s) {};
      \node at (.75,2.75) (s2) {};
      \node at (.5,2.25) [map] (c) {$\scriptstyle 1\*\cpa$};
      \node at (0,1.5) [delta] (d2) {};
      \node at (.5,1) [map] (q) {$\scriptstyle q_1\tjup q_2$};
      \node at (0, .5) [nabla] (n1) {};
      \node at (0,0) (end) {};
      \node at (.75,0) (end2) {};
      \draw [] (s) to[out=270,in=125] (c);
      \draw [] (s2) to[out=270,in=55] (c);
      \draw [] (c) to[out=305,in=55] (q);
      \draw [] (c) to[out=235,in=90] (d2);
      \draw [] (d2) to[out=305,in=125] (q);
      \draw [] (d2) to[out=235,in=125] (n1);
      \draw [] (q) to[out=235,in=55] (n1);
      \draw [] (q) to[out=305,in=90] (end2);
      \draw (n1) to (end);
    \end{tikzpicture}
    \,\raisebox{40pt}{$= (1\*\cpa)\dmap{(q_1\tjup q_2)}$.}
  \]
  Similarly, $q_2 = (1\*\cpb)\dmap{(q_1\tjup q_2)}$ giving us $q_1\tjup q_2 \in \dmap{Y}$ by the
  uniqueness of the $\tjup$ map. Therefore $q_1\tjup q_2$ provides an
  equivalence between $k$ and $\icpa f' \djoin \icpb g'$, meaning the coproduct is unique.
\end{proof}

\begin{corollary}\label{cor:xt_is_a_distributive_restriction_category}
  When \X is a distributive inverse category, $\Xt$ is a distributive restriction category.
\end{corollary}
\begin{proof}
  As $\Xt$ has restriction products by Lemma~\ref{lem:tensor_on_x_is_the_restriction_product_on_xt},
  restriction coproducts by
  Theorem~\ref{thm:x_tilde_has_coproducts_if_x_is_inverse_distributive_category} and the equations
  for distributivity follow directly from the distributivity of the base tensors, we see $\Xt$ is a
  distributive restriction category.
\end{proof}
% chapter disjoint_sum_categories (end)

%%% Local Variables:
%%% mode: latex
%%% TeX-master: "../phd-thesis"
%%% End:
