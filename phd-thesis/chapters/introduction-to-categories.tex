%!TEX root = /Users/gilesb/UofC/thesis/phd-thesis/phd-thesis.tex

\chapter{Introduction to Categories}\label{chap:introduction_to_categories}

This chapter introduces categories and fixes notation for them. More
details for category theory can be found from, e.g., \cite{maclan97:categorieswrkmath}, \cite{cockett2009:ctcs},
\cite{barr:ctcs} and \cite{various:nlab}.

\section{Definition of a category}
\label{sec:definition_of_a_category}


A category may be defined in a variety of equivalent ways. As much of our
work will involve the exploration of partial and reversible maps, their domains and codomains, we
choose a definition that highlights the algebraic nature of these.

\begin{definition}\label{def:category}
  A \emph{category} $\A$ is a directed graph consisting of objects $A_o$ and maps $A_m$. Each $f\in
  A_m$ has two associated objects in $A_o$, called the domain, $\catdomain f$, and codomain,
  $\catcodomain f$. When $\catdomain f$ is the object $X$  and $\catcodomain f$ is the object $Y$, we
  will write $f:X \to Y$. For $f, g \in A_m$, if $f:X\to Y$ and $g:Y \to Z$, there is a map called
  the \emph{composite} of $f$ and $g$, written $f g$,\footnote{Note that composition is written in
    diagrammatic order throughout this paper.} such that $f g:X \to Z$. For any $W \in A_o$ there is
  an \emph{identity} map $1_W:W \to W$. Additionally, these two axioms must hold:
  \begin{align*}
    \catone &\text{ for }f:X \to Y,\ 1_X f = f = f 1_Y,&\text{(Unit laws)}\\
    \cattwo &\text{ given }f:X \to Y,\ g:Y \to Z\text{ and }h: Z\to W\text{, then }f (g h) = (f g) h.&\text{( Associativity)}
  \end{align*}
\end{definition}

Throughout   this thesis, we will be working with \emph{small} categories, that is, those categories
whose collection of maps and collection of objects is, in fact, a set. We will give categories of
``all'' sets as an example, and the reader should take that to mean all sets contained within a
sufficiently large set.

\section{Properties of maps} % (fold)
\label{sub:properties_of_maps}
Many interesting properties of maps are generalizations of notions used for sets and functions. We
present a few of these in Table~\ref{tab:properties_of_maps_in_categories}, together with their
categorical definition. Throughout Table~\ref{tab:properties_of_maps_in_categories}, $e,f,g$ are
maps in a category $C$ with $e:A \to A$ and $f,g:A \to B$.
\begin{table}[!htbp]
  \begin{center}
    \begin{tabular}{|p{1in}p{1in}p{3.73in}|}
      \hline
      {\bf Sets} & {\bf Categorical Property} & {\bf Definition}\\
      \hline
      \hline
      Injective & Monic & $f$ is monic whenever $h f = k f$ means that $h = k$.\\
      \hline
      Surjective & Epic & The dual notion to monic, $g$ is epic whenever $g h = g k$ means that $h = k$.
      A map that is both monic and epic is called \emph{bijic}.\\
      \hline
      Left~Inverse & Section & $f$ is a section when there is a map $\categorysection{f}$ such that $f \categorysection{f} = 1_A$. $f$
      is also referred to as the \emph{left inverse} of $\categorysection{f}$.\\
      \hline
      Right Inverse & Retraction & $f$ is a retraction when there is a map $\retraction{f}$ such that $\retraction{f} f = 1_B$.
      $f$ is also referred to as the \emph{right inverse} of $\retraction{f}$. A map that is both a section and a
      retraction is called an \emph{isomorphism}.\\
      \hline
      Idempotent & Idempotent & An endomorphism $e$ is idempotent whenever $e e = e$.\\
      \hline
    \end{tabular}
  \end{center}
  \caption{Properties of Maps In Categories}
  \label{tab:properties_of_maps_in_categories}
\end{table}

There are number of basic properties of maps enumerated in this lemma:

\begin{lemma}\label{lem:categorical_properties_of_maps}
  In a category \B,
  \begin{enumerate}[{(}i{)}]
    \item If $f,g$ are monic, then $f g$ is monic.
    \item If $f g$ is monic, then $f$ is monic.
    \item $f$ being a section means it is monic.
    \item $f, g$ sections implies that $f g$ is a section.
    \item $f g$ a section means $f$ is a section.
    \item If $f:A \to B$ is both a section and a retraction, then $\categorysection{f} = \retraction{f}$, where $\categorysection{f}$ and $\retraction{f}$ are as
      defined in Table~\ref{tab:properties_of_maps_in_categories}.
    \item  $f$ is an isomorphism if and only if it is an epic section.
  \end{enumerate}
\end{lemma}
\begin{proof}
  \prepprooflist
  \begin{enumerate}[{(}i{)}]
    \item Suppose $h f g = k f g$. As $g$ is monic, $h f = k f$. As $f$ is monic, this gives us $h =
      k$ and therefore $f g$ is monic.
    \item See \cite{barr:ctcs}, chapter 2.
    \item Suppose $h f = k f$. Then $h f \categorysection{f} = k f \categorysection{f}$ giving us $h
      1 = k 1$ and therefore $h = k$  and $f$ is monic.
    \item We are given $f\categorysection{f} =1$ and $g\categorysection{g} =1 $. But then $fg
      \categorysection{g}\categorysection{f} = f \categorysection{f} = 1$ and $f g$ is a section.
    \item We are given there is some $h$ such that $(f g) h = 1$. This means $f (gh) =1$ and $f$ is
      a section.
    \item See \cite{cockett2009:ctcs}, Lemma 1.2.2.
    \item See \cite{cockett2009:ctcs}, Lemma 1.2.3.
  \end{enumerate}

\end{proof}

Note there are corresponding properties for epics and retractions, obtained by dualizing the
statements of Lemma~\ref{lem:categorical_properties_of_maps}.

Suppose $f:A \to B$ is a retraction with left inverse $\retraction{f}:B \to A$. Note that $f
\retraction{f}$ is idempotent as $f \retraction{f} f \retraction{f} = f 1_B \retraction{f} = f
\retraction{f}$. If we are given an idempotent $e$, we say $e$ is \emph{split} if there is a
retraction $f$ with $e = f \retraction{f}$.

In general, not all idempotents in a category will split. The following construction allows us to
create a category based on the original one in which all idempotents do split.

\begin{definition}\label{def:split_category}
  Given a category $\B$ and a set of idempotents $E$ of $\B$, we may create \emph{Split${}_E$($\B$)}.
  As this is called the Karoubi envelope, this is normally written as $\spl{E}{\B}$, and defined as:
  \category{$(A,e)$, where $A$ is an object of \B, $e:A\to A$ and $e\in E$.}
    {$f_{d,e}:(A,d)\to(B,e)$ is given by $f:A\to B$ in \B, where $f = d\uts f e$.}
    {The map $e_{e,e}$ for $(A,e)$.}
    {Inherited from \B.}
  When $E$ is the set of all idempotents in $\B$, we write $\spl{}{\B}$.
\end{definition}
This is the standard idempotent splitting construction, variously known as the Karoubi
envelope or Cauchy completion.

\begin{lemma}\label{lem:split_category_splits_and_has_category}
  Given a category $\B$, then it is a full sub-category of $\spl{}{\B}$ and all idempotents split
  in  $\spl{}{\B}$.
\end{lemma}
\begin{proof}
  We identify each object $A$ in $\B$ with the object $(A,1)$ in $\spl{}{\B}$. The only maps between
  $(A,1)$ and $(B,1)$ in $\spl{}{\B}$ are the maps between $A$ and $B$ in $\B$, hence we have a
  full sub-category.

  Suppose we have the map $d_{e,e}: (A,e) \to (A,e)$ with $d d = d$, i.e., it is idempotent in $C$
  and $\spl{}{\B}$. In $\spl{}{\B}$, we have the maps $d_{e,d}:(A,e) \to (A,d)$ and $d_{d,e}:(A,d) \to
  (A,e)$ where $d_{d,e} d_{e,d} = d_{d,d} = 1_{(A,d)}$ and $d_{e,d} d_{d,e} = d_{e,e}$. Hence,
  it is a splitting of the map  $d_{e,e}$.
\end{proof}
% subsection properties_of_maps (end)

\section{Functors and natural transformations} % (fold)
\label{ssub:functors_and_natural_transformations}

\begin{definition}\label{def:functor}
  A map $F:\X \to \Y$ between categories, as in Definition~\ref{def:category}, is called a
  \emph{functor}, provided it satisfies the following:
  \begin{itemize}
    \item[\axiom{F}{1}] $F(\catdomain(f)) = \catdomain(F(f))$ and $F(\catcodomain(f)) = \catcodomain(F(f))$;
    \item[\axiom{F}{2}] $F(f g) = F(f)F(g)$;
  \end{itemize}
\end{definition}

\begin{lemma}\label{lem:cat_is_a_category}
  Categories and functors form a category \cat.
\end{lemma}
\begin{proof}
  \prepprooflist
  \categoryns{Categories.}{Functors.}{The identity functor which takes an object to the same object
    and a map to the same map.}{Given $F:\A\to\B$, $G:\B\to\D$, define the functor  $F G:\A\to\D$ such that
    $F G(x) = G(F(x))$ which is clearly associative.}
  Note this would be a \emph{large} category, as its collection of objects would not be a set.
\end{proof}

\begin{definition}\label{def:faithful_functor}
  A functor $F:\X\to\Y$ is \emph{faithful} when for each pair of objects $A,B$ in $\X$, and map
  $g:FA \to FB$ in $\Y$, there is a map $f:A\to B$ in $\X$ such that $F f = g$.
\end{definition}

\begin{definition}\label{def:full_functor}
  A functor $F:\X\to\Y$ is \emph{full} when for parallel maps $f, f'$, if $Ff = Ff'$ then $f = f'$.
\end{definition}
We may also consider the notion of containment between categories:

\begin{definition}\label{def:subcategories}
  Given the categories $\B$ and $\D$, we say that \B is a \emph{sub-category} of \D when each object
  of \B is an object of \D and when each map of \B is a map of \D.
\end{definition}

When $\B$ is a subcategory of $\D$, the functor $J:\B \to \D$ which takes each object to itself in
$\D$ and each map to itself in $\D$ is called the inclusion functor. When $J$ is a full functor, we
say $\B$ is a full subcategory of $\D$.

Functors with two arguments, e.g., $F:\sets \times \sets \to \sets$ which satisfy \axiom{F}{1} and
\axiom{F}{2} for each argument independently are called \emph{bi-functors}.


We will often restrict ourselves to specific classes of functors which either \emph{preserve} or
\emph{reflect} certain characteristics of the domain category or codomain category. To be more
precise, we provide some definitions.

\begin{definition}\label{def:diagram_in_a_category}
  Given a category $\cS$, a \emph{diagram}
  in a category $\B$ of \emph{shape} $\cS$ is a functor $D:\cS \to \B$.
\end{definition}

\begin{definition}\label{def:property_of_a_diagram}
  A \emph{property} of a diagram $D$, written $P(D)$ is a logical relation expressed using the
  objects and maps of the diagram $D$.
\end{definition}

For example,   $P(f:A \to B) = \exists h : B \to A. h f = 1_A$ expresses that $f$ is a retraction.

\begin{definition}\label{def:functor_preserving_and_reflecting_a_property}
  A functor $F$ \emph{preserves} the property $P$ over maps $f_i$ and objects $A_j$ when:
  \[
     P(f_1,\ldots,f_n, A_1,\ldots,A_m) \implies P(F(f_1),\ldots,F(f_n), F(A_1),\ldots,F(A_m)).
  \]
  A functor $F$ \emph{reflects} the property $P$ over maps $f_i$ and objects $A_j$ when:
  \[
    P(F(f_1),\ldots,F(f_n), F(A_1),\ldots,F(A_m)) \implies P(f_1,\ldots,f_n, A_1,\ldots,A_m).
  \]
\end{definition}

For example, all functors preserve the properties of being an idempotent or a retraction or section,
but in general, not the property of being monic.

A functor $F:\B \to \D$ induces a map between hom-sets in $\B$ and hom-sets in $\D$. For
each object $A,B$ in $\B$ we have the map:
\[
  F_{AB} : \B(A,B) \to \D(F(A),F(B)).
\]

\begin{definition}\label{def:natural_transformation}
  Given functors $F,G:\X \to \Y$, a \emph{natural transformation} $\alpha:F \natto G$ is a collection
  of maps in $\Y$, $\alpha_X : F(X) \to G(X)$, indexed by the objects of $\X$ such that for all
  $f:X_1 \to X_2$ in $\X$ the following diagram in $\Y$ commutes:
  \[\xymatrix @R+10pt @C+10pt{
      F(X_1) \ar[r]^{F(f)} \ar[d]_{\alpha_{X_1}} & F(X_2) \ar[d]^{\alpha_{X_2}}\\
      G(X_1) \ar[r]_{G(f)} &  G(X_2)
    }
  \]
\end{definition}
% subsection functors_and_natural_transformations (end)

\section{Enrichment of categories} % (fold)
\label{sub:enrichement_of_categories}
\begin{notation}\label{notn:hom-set}
  If $\X$ is a category, then the maps from $A$ to $B$ in $\X$ are denoted $\X(A,B)$.
  If $\X(A,B)$ is a set for all objects in $\X$, we say $\X$ is \emph{enriched} in sets. More generally,
  categories may be enriched in any monoidal category. For example a category may be
  enriched in abelian groups, vector spaces, posets, categories or commutative monoids.
\end{notation}

Specific types of enrichment may force a structure on a category. Examples:
\begin{enumerate}
 \item If $\X$ is enriched in sets of cardinality of 0 or 1, then $\X$ is a pre-order.
 \item If $\X$ is enriched in pointed sets with the monoid of smash product, the $\X$ has zero morphisms.
 \item If $\X$ is enriched in abelian groups, then $\X$ is a preadditive category (has zero
   morphisms and finite products are the same as the coproduct).
\end{enumerate}

% subsection enrichment_of_categories (end)
\section{Examples of categories} % (fold)
\label{sub:examples_of_categories}
In this section, we will offer a few examples of categories.

\begin{example}
  A group $G$ may be considered as a one-object category $\mathbb{G}$, with object $\{*\}$. The elements of the
  group are the maps between $\{*\}$. As $G$ is a group, there is an identity, composition is given
  by the group multiplication and additionally, each map has an inverse. As $\mathbb{G}(\{*\},\{*\})
  = G$, this category is enriched in groups.
\end{example}

Four categories derived from sets are \sets, \Par, \rel and \pinj:

\begin{example}[\sets]\label{ex:category_sets}
In this category, the maps are given by all set functions.
\category{Sets}{Set functions}{The identity function}{Standard composition of functions}
\end{example}

\begin{example}[\Par]\label{ex:category_par}
For this example, the maps are all partial set functions.
\category{Sets}{Partial set functions}{The identity function}{Standard composition of functions}
\end{example}

\begin{example}[\rel]\label{ex:category_rel}
\rel is often of interest in quantum programming language semantics:
\category{All sets}{Relations: $R:X \to Y$}{$1_X = \{(x,x) | x \in X\}$}{$RS = \{(x,z) |\,
\exists y. (x,y) \in R$ and $(y,z)\in S\}$}
\end{example}

Note that \rel is enriched in posets, via set inclusion. \Par can be viewed as a sub-category of
\rel, with the same objects, but only allowing maps which are partial functions, i.e., deterministic
relations where if $(x,y), (x,y') \in f$, then $y = y'$. \Par is also enriched in posets, via the
same inclusion ordering as in \rel.

\begin{example}[\pinj]\label{ex:category_pinj}
Our final example based on sets is one that will be used throughout this thesis. The category \pinj consists of
the partial injective functions over sets. Similarly to \Par, it may be considered as a subcategory
of \rel. The maps $f,g$ (relations in \rel) in \pinj are defined as follows:
\begin{align}
   (x,y)\in f\text{ and }(x,y')\in f & \text{ implies } y = y',\label{eq:pinj_relation_is_a_function}\\
   (x,y)\in f\text{ and }(x',y)\in f & \text{ implies } x = x'.\label{eq:pinj_converse_relation_is_a_function}
\end{align}
\category{All sets}{Relations: $f:X \to Y$ which satisfy Equation~\ref{eq:pinj_relation_is_a_function} and
Equation~\ref{eq:pinj_converse_relation_is_a_function}}{$1_X = \{(x,x) | x \in X\}$}{$f g = \{(x,z) |\,
\exists y. (x,y) \in f$ and $(y,z)\in g\}$}
\end{example}

\begin{example}[\topcat]\label{ex:category_top}
  This is the category of topological spaces.
  \category{Topological spaces}{Continuous functions}{The identity function}{Function composition}
  As the composition of continuous maps is also continuous, this is a category.
\end{example}

Our next example shows maps in categories need not always be something normally thought of as a
function or relation.

\begin{example}[Matrix Category]\label{ex:matrix_category}
Given a rig $R$ (i.e., a ri\textbf{n}g minus \textbf{n}egatives, e.g., the natural numbers), one may
form the category \textsc{Mat}($R$). For example, the category of matrices of natural numbers is:
\category{\nat}{$[r_{i j}]: n \to m$ where $[r_{i j}]$ is an $n \times m$ matrix over \nat}{
$I_n$}{Matrix multiplication}
\end{example}

Our last example describes a construction on a category.

\begin{example}[Dual Category]\label{ex:dual_category}
Given a category $\B$, we may form the \emph{dual} of $\B$, written $\B^{op}$ as the following
category:
\category{The objects of $\B$}{$f^{op}:B\to A$ in $\B^{op}$ when $f:A\to B$ in $\B$.}{
The identity maps of $\B$}{If $f g = h$ in $\B$, $g^{op} f^{op} = h^{op}$}
\end{example}

% subsection examples_of_categories (end)

\section{Limits and colimits in categories} % (fold)
\label{sub:limits_and_colimits_in_categories}

We shall review only a few basic limits/colimits in categories, in order to set up notation and
terminology. First we discuss initial and terminal objects.

\begin{definition}\label{def:initial_object}
  An \emph{initial object} in a category $\B$ is an object which has exactly one map to each other
  object in the category. The dual notion is \emph{terminal object}. Every object in the category
  has exactly one map to the terminal object.
\end{definition}


\begin{lemma}\label{lem:initial_objects_are_unique}
  Suppose $I,J$ are initial objects in $\B$. Then there is a unique isomorphism $i:I \to J$.
\end{lemma}
\begin{proof}
  First, note that by definition there is only one map from $I$ to $I$ --- which must be the
  identity map. As $I$ is initial there is a map $i: I \to J$. As $J$ is initial there is a map
  $j:J \to I$. But this means $i j : I \to I = 1$ and $j i : J \to J = 1$ and hence $i$ is the
  unique isomorphism from $I$ to $J$.
\end{proof}

Dually, we have the corresponding result to Lemma~\ref{lem:initial_objects_are_unique} for terminal
objects --- they are also unique up to a unique isomorphism.

In categories, following the terminology in sets, we normally designate the initial object by $0$
and the terminal object by $1$.
A map \emph{from} the terminal object to another object in the category is often referred to as an
\emph{element}.

We now turn to products and co-products.

\begin{definition}\label{def:categorical_product}
  Let $A,B$ be objects of the category $\B$. Then the object $A \times B$ is a \emph{product} of
  $A$ and $B$ when:
  \begin{itemize}
    \item There exist maps $\pi_0, \pi_1$ with $\pi_0:A\times B \to A$, $\pi_1:A\times B \to B$;
    \item Given an object $C$ with maps $f:C\to A$ and $g:C \to B$ there is a unique map
    $\<f,g\>$ such that the following diagram commutes:
    \[
      \xymatrix@C+15pt@R+25pt{
        &&&A\\
        C\ar[urrr]^{f} \ar[drrr]_{g}\ar@{.>}[rr]^{\<f,g\>} & &A\times B \ar[ur]_{\pi_0}\ar[dr]^{\pi_1}\\
        &&&B
      }
    \]
  \end{itemize}

\end{definition}

A co-product is the dual of a product.

\begin{definition}\label{def:categorical_coproduct}
  Let $A,B$ be objects of the category $\B$. Then the object $A + B$ is a \emph{coproduct} of
  $A$ and $B$ when:
  \begin{itemize}
    \item There exist maps $\cpa, \cpb$ with $\cpa:A \to A + B$, $\cpb:B \to A+ B$;
    \item Given an object $C$ with maps $h:A\to C$ and $k:B \to C$ there is a unique map
    $[h,k]$ such that the following diagram commutes:
    \[
      \xymatrix@C+15pt@R+25pt{
        A \ar[drrr]^h \ar[dr]_{\cpa} \\
        &A +B \ar@{.>}[rr]^{[h,k]} && C\\
        B \ar[urrr]_k \ar[ur]^{\cpb}
      }
    \]
  \end{itemize}
\end{definition}

It is possible for an object to both a limit and a co-limit at the same time:
\begin{definition}\label{def:categorical_zero}
  Given a category $\B$, any object that is both a terminal and initial object is called a
  \emph{zero object}. This object is labelled $\zeroob$.
\end{definition}

Note that any category with a zero object has a special map, $\zeroob_{A,B}$
between any two objects $A,B$ of the category given by: $\zeroob_{A,B} : A \to \mathbf{0} \to B$.

\begin{definition}\label{def:categorical_biproduct}
  In a category \B, with products and coproducts, where:
  \[
    \cp{i}\Pi_j = \begin{cases}
      1 & i = j\\
      0 & i \neq j
      \end{cases}
  \]
  and given any two objects, $A,B$, $A\times B$ is the same as to $A+B$ , then $A\times B$ is
  referred to as the \emph{biproduct} and designated as $A\biproduct B$. A category \cD{} is said to
  have \emph{finite biproducts} when it has a zero object $\mathbf{0}$ and when each pair of objects
  $A,B$ have a biproduct $A\biproduct B$.

  Note the biproduct is often written  as $\+$. As we will be using $\+$ frequently in this thesis
  where it is not a biproduct, this alternate notation for $\biproduct$ will be used instead.
\end{definition}

Note that a category with finite biproducts is enriched in commutative monoids. If $f,g:A\to
B$, define $f+g:A\to B$ as $\<id_{A}, id_{A}\>\, (f\biproduct g)\, [id_{B},id_{B}]$. The unit for the
addition is $\zeroob_{A,B}$. Throughout this thesis, when working in a category with finite
biproducts, $\<id, id\>$ will be designated by $\Delta$ and $[id,id]$ will be designated by
$\nabla$.

% subsection limits_and_colimits_in_categories (end)


\section{Symmetric Monoidal Categories} % (fold)
\label{sub:categories_with_additional_structure}

\begin{definition}\label{symmetricmonoidalcat}
  A \emph{symmetric monoidal category}\cite{barr:ctcs,maclan97:categorieswrkmath} \cD{} is a
  category equipped with a monoid $\*$ (a bi-functor $\*:\cD \times \cD \to \cD$) together with
  four families of natural isomorphisms:  $a_{A,B,C}:A\*(B\*C) \to (A\*B)\*C$, $u^r_{A}:A\*I\to A$,
  $u^l_{A}:I\*A \to A$ and $c_{A,B}:A\*B \to B\* A$, which satisfy coherence diagrams and
  equations shown in Figures~\ref{fig:SMC_pentagon}, \ref{fig:SMC_unit}, \ref{fig:SMC_commutes},
  \ref{fig:SMC_unit_symmettry} and \ref{fig:SMC_associativity_symmetry}. The isomorphisms are
  referred to as the \emph{structure isomorphisms}  for the symmetric monoidal category. $I$ is the
  unit of the monoid. A symmetric monoidal category where each of $a_{A,B,C}$, $u^r_{A}$, $u^l_{A}$
  and $c_{A,B}$ are identity maps is called a \emph{strict symmetric monoidal category}.
\end{definition}

\begin{figure}[!htbp]
\[
  \xymatrix@C+25pt{
    A\*(B\*(C\*D) \ar[r]^{a_{A,B,(C\*D)}} \ar[d]_{1\*a_{B,C,D}}
      & (A\*B)\*(C\*D) \ar[r]^{a_{(A\*B),C,D}}
      & ((A\*B)\*C)\*D \ar[d]^{a_{A,B,C}\*1}\\
    A\*((B\*C)\*D) \ar[rr]_{a_{A,(B\*C),D}}
      && (A\*(B\*C))\*D
  }
\]
\caption{Pentagon diagram for associativity in an SMC.}\label{fig:SMC_pentagon}
\end{figure}
\begin{figure}[!htbp]
\[
  \xymatrix@C+5pt@R+10pt{
    A\*(I\*B) \ar[rr]^{a_{A,I,B}} \ar[dr]_{1\*u^l_B}
      && (A\*I)\*B \ar[dl]^{u^r_A \* 1}\\
      &A\*B
  }
\]
\[\text{ and } u^r_I = u^l_I: I\* I \to I\]
\caption{Unit diagram and equation in an SMC.}\label{fig:SMC_unit}
\end{figure}
\begin{figure}[!htbp]
\[
  \xymatrix@C+5pt@R+10pt{
    A\*B \ar[r]^{c_{A,B}} \ar@{=}[dr]
      & B\*A \ar[d]^{c_{B,A}}\\
      &A\*B
  }
\]
\caption{Symmetry in an SMC.}\label{fig:SMC_commutes}
\end{figure}
\begin{figure}[!htbp]
\[
  \xymatrix@C+5pt@R+10pt{
    A\*I \ar[rr]^{c_{A,I}} \ar[dr]_{u^r_A}
      && I\*A \ar[dl]^{u^l_A}\\
      &A
  }
\]
\caption{Unit symmetry in an SMC.}\label{fig:SMC_unit_symmettry}
\end{figure}
\begin{figure}[!htbp]
\[
  \xymatrix@C+15pt@R+10pt{
    (A\*B)\*C \ar[r]^{c_{(A\*B),C}} \ar[d]_{a^{-1}_{A,B,C}}
      & C\*(A\*B) \ar[d]^{a_{C,A,B}}\\
    A\*(B\*C) \ar[d]_{1\*c_{B,C}}
      & (C\*A)\*B \ar[d]^{c_{C,A}\*1}\\
    A\*(C\*B) \ar[r]^{a_{A,C,B}}
      & C\*(A\*B)\text{ ,}
  }\qquad
  \xymatrix@C+15pt@R+10pt{
    A\*(B\*C) \ar[r]^{c_{A,(B\*C)}} \ar[d]_{a_{A,B,C}}
      & (B\*C)\*A \ar[d]^{a^{-1}_{B,C,A}}\\
    (A\*B)\*C \ar[d]_{c_{A,B}\*1}
      & B\*(C\*A) \ar[d]^{1\*c_{C,A}}\\
    (B\*A)\*C \ar[r]^{a^{-1}_{B,A,C}}
      & B\*(A\*C)
  }
\]
\caption{Associativity symmetry in an SMC.}\label{fig:SMC_associativity_symmetry}
\end{figure}
The essence of the coherence diagrams is that any diagram composed solely of the structure
isomorphisms will commute.
% subsection symmetric monoidal categories (end)


%%% Local Variables:
%%% mode: latex
%%% TeX-master: "../phd-thesis"
%%% End:
