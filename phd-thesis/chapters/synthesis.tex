%!TEX root = /Users/gilesb/UofC/thesis/phd-thesis/phd-thesis.tex
\chapter{Synthesis of quantum operations} % (fold)
\label{cha:synthesis_of_quantum_operations}

\section{Introduction to synthesis} % (fold)
\label{sec:introduction_to_synthesis}
An important problem in quantum information theory is the decomposition of arbitrary unitary
operators into gates from some fixed universal set {\cite{neilsen2000:QuantumComputationAndInfo}}. Depending on the
operator to be decomposed, this may either be done exactly or to within some given accuracy
$\epsilon$; the former problem is known as {\em exact synthesis} and the latter as {\em approximate
synthesis} {\cite{Kliuchnikov-et-al}}.

% section introduction_to_synthesis (end)
\section{Exact synthesis of multi-qubit operators} % (fold)
\label{sec:exact_synthesis_of_multi_qubit_operators}
Here, we focus on the problem of exact synthesis for $n$-qubit operators, using the Clifford+$T$
universal gate set. Recall that the Clifford group on $n$ qubits is generated by the Hadamard gate
$H$, the phase gate $S$, the controlled-not gate, and the scalar $\omega=e^{i\pi/4}$ (one may allow
arbitrary unit scalars, but it is not convenient for our purposes to do so). It is well-known that
one obtains a universal gate set by adding the non-Clifford operator $T$ {\cite{neilsen2000:QuantumComputationAndInfo}}.

\begin{equation}\label{eqn-generators}
  \begin{array}{c}
  \displaystyle
  \omega = e^{i\pi/4},\quad
  H = \frac{1}{\sqrt{2}}\zmatrix{cc}{1&1\\1&-1},\quad 
  S = \zmatrix{cc}{1&0\\0&i},\\\\[-1.5ex]
  \displaystyle
  {\it CNOT} = \zmatrix{cccc}{1&0&0&0\\0&1&0&0\\0&0&0&1\\0&0&1&0},\quad
  T = \zmatrix{cc}{1&0\\0&e^{i\pi/4}}.
\end{array}
\end{equation}

In addition to the Clifford+$T$ group on $n$ qubits, as defined above, we also consider the
slightly larger group of Clifford+$T$ operators ``with ancillas''. We say that an $n$-qubit
operator $U$ is a Clifford+$T$ operator {\em with ancillas} if there exists $m\geq 0$ and a
Clifford+$T$ operator $U'$ on $n+m$ qubits, such that $U'(\ket{\phi}\otimes\ket{0})=
(U\ket{\phi})\otimes\ket{0}$ for all $n$-qubit states $\ket{\phi}$.

Kliuchnikov, Maslov, and Mosca {\cite{Kliuchnikov-et-al}} showed that a single-qubit operator $U$
is in the Clifford+$T$ group if and only if all of its matrix entries belong to the ring
$\Z[\frac1{\sqrt{2}},i]$. They also showed that the Clifford+$T$ groups ``with ancillas'' and
``without ancillas'' coincide for $n=1$, but not for $n\geq 2$. Moreover, Kliuchnikov et
al.~conjectured that for all $n$, an $n$-qubit operator $U$ is in the Clifford+$T$ group with
ancillas if and only if its matrix entries belong to $\Z[\frac1{\sqrt{2}},i]$. They also
conjectured that a single ancilla qubit is always sufficient in the representation of a
Clifford+$T$ operator with ancillas. This section of the thesis will prove these conjectures. In
particular, this yields an algorithm for exact Clifford+$T$ synthesis of $n$-qubit operators. We
also obtain a characterization of the Clifford+$T$ group on $n$ qubits without ancillas.

It is important to note that, unlike in the single-qubit case, the circuit synthesized here are not
in any sense canonical, and very far from optimal. Thus, the question of {\em efficient} synthesis
is not addressed here.

\subsection{Algebraic background} % (fold)
\label{sub:algebraic_background}
We first introduce some notation and terminology, following {\cite{Kliuchnikov-et-al}} where
possible. Recall that $\N$ is the set of natural numbers including 0, and $\Z$ is the ring of
\linebreak integers. We write $\Zb=\Z/2\Z$ for the ring of integers modulo 2. Let $\D$ be the ring
of {\em dyadic fractions}, defined as $\D = \Z[\frac12] = \s{\frac{a}{2^n}\mid a\in\Z, n\in\N}$.

Let $\omega = e^{i\pi/4} = (1+i)/\sqrt{2}$. Note that $\omega$ is an 8th root of unity satisfying
$\omega^2=i$ and $\omega^4=-1$. We will consider three different rings related to $\omega$:

\begin{definition}
  Consider the following rings. Note that the first two are subrings of the complex numbers, and
  the third one is not:
  \begin{itemize}
    \item $\Dw = \s{a\omega^3+b\omega^2+c\omega+d \mid a,b,c,d\in\D}$.
    \item $\Zw = \s{a\omega^3+b\omega^2+c\omega+d \mid a,b,c,d\in\Z}$.
    \item $\Zbw = \s{p\omega^3+q\omega^2+r\omega+s \mid p,q,r,s\in\Zb}$.
  \end{itemize}
  Note that the ring $\Zbw$ only has 16 elements. The laws of addition and multiplication are
  uniquely determined by the ring axioms and the property $\omega^4=1\ (\mod 2)$. We call the
  elements of $\Zbw$ {\em residues} (more precisely, residue classes of $\Zw$ modulo 2).
\end{definition}

\begin{remark}
  The ring $\Dw$ is the same as the ring $\Z[\frac1{\sqrt{2}},i]$ mentioned in the statement of
  Theorem~\ref{thm-main}. However, as already pointed out in {\cite{Kliuchnikov-et-al}}, the
  formulation in terms of $\omega$ is far more convenient algebraically.
\end{remark}

\begin{remark}
  The ring $\Zw$ is also called the {\em ring of algebraic integers} of $\Dw$. It has an intrinsic
  definition, i.e., one that is independent of the particular presentation of $\Dw$. Namely, a
  complex number is called an {\em algebraic integer} if it is the root of some polynomial with
  integer coefficients and leading coefficient 1. It follows that $\omega$, $i$, and $\sqrt{2}$ are
  algebraic integers, whereas, for example, $1/\sqrt{2}$ is not. The ring $\Zw$ then consists of
  precisely those elements of $\Dw$ that are algebraic integers.
\end{remark}

\subsubsection{Conjugate and norm} % (fold)
\label{ssub:conjugate_and_norm}
\begin{remark}[Complex conjugate and norm]
  Since $\Dw$ and $\Zw$ are subrings of the complex numbers, they inherit the usual notion of
  complex conjugation. We note that $\omega\da = -\omega^3$. This yields the following formula:
  \begin{equation}\label{eqn-adjoint}
    (a\omega^3+b\omega^2+c\omega+d)\da = -c\omega^3-b\omega^2-a\omega+d.
  \end{equation}
  Similarly, the sets $\Dw$ and $\Zw$ inherit the usual norm from the complex numbers. It is given
  by the following explicit formula, for $t=a\omega^3+b\omega^2+c\omega+d$:
  \begin{equation}\label{eqn-cplx-norm}
    \snorm{t}^2 = t\da t = (a^2+b^2+c^2+d^2) + (cd+bc+ab-da) \sqrt{2}.
  \end{equation}
\end{remark}

\begin{definition}[Weight]
  For $t\in\Dw$ or $t\in\Zw$, the {\em weight} of $t$ is denoted $\sweight{t}$, and is given by:
  \begin{equation}\label{eqn-weight}
    \sweight{t}^2 = a^2+b^2+c^2+d^2.
  \end{equation}
\end{definition}

Note that the square of the norm is valued in $\D[\sqrt{2}]$, whereas the square of the weight is
valued in $\D$. We also extend the definition of norm and weight to vectors in the obvious way: For
$u = (u_\jay)_\jay$, we define
\[
  \norm{u}^2 = \sum_\jay\norm{u_\jay}^2
  \quad\mbox{and}\quad
  \weight{u}^2 = \sum_\jay\weight{u_\jay}^2.
\]

\begin{lemma}\label{lem-ring-norm}
  Consider a vector $u\in\Dw^n$. If $\norm{u}^2$ is an integer, then $\weight{u}^2=\norm{u}^2$.
\end{lemma}

\begin{proof}
  Any $t\in\D[\sqrt{2}]$ can be uniquely written as $t=a+b\sqrt{2}$, where $a,b\in\D$. We can call
  $a$ the {\em dyadic part} of $t$. Now the claim is obvious, because $\weight{u}^2$ is exactly the
  dyadic part of $\norm{u}^2$.
\end{proof}

% subsubsection conjugate_and_norm (end)

\subsubsection{Denominator exponents} % (fold)
\label{ssub:denominator_exponents}
\begin{definition}
  Let $t\in\Dw$. A natural number $k\in\N$ is called a {\em denominator exponent} for $t$ if
  $\sqrt{2}^k t \in \Zw$. It is obvious that such $k$ always exists. The least such $k$ is called
  the {\em least denominator exponent} of $t$.

  More generally, we say that $k$ is a denominator exponent for a vector or matrix if it is a
  denominator exponent for all of its entries. The least denominator exponent for a vector or
  matrix is therefore the least $k$ that is a denominator exponent for all of its entries.
\end{definition}

\begin{remark}
  Our notion of least denominator exponent is almost the same as the ``smallest denominator
  exponent'' of {\cite{Kliuchnikov-et-al}}, except that we do not permit $k<0$.
\end{remark}

% subsubsection denominator_exponents (end)

\subsubsection{Residues} % (fold)
\label{ssub:residues}
\begin{remark}
  The ring $\Zbw$ is not a subring of the complex numbers; rather, it is a quotient of the ring
  $\Zw$. Indeed, consider the {\em parity function} $\parity{()}:\Z\to\Zb$, which is the unique
  ring homomorphism. It satisfies $\parity{a}=0$ if $a$ is even and $\parity{a}=1$ if $a$ is odd.
  The parity map induces a surjective ring homomorphism $\rho:\Zw\to\Zbw$, defined by
  \[ 
    \rho(a\omega^3+b\omega^2+c\omega+d) =
    \parity{a}\omega^3+\parity{b}\omega^2+\parity{c}\omega+\parity{d}.
  \]
  We call $\rho$ the {\em residue map}, and we call $\rho(t)$ the {\em residue} of $t$.
\end{remark}

\begin{convention}
  Since residues will be important for the constructions of this paper, we introduce a shortcut
  notation, writing each residue $p\omega^3+q\omega^2+r\omega+s$ as a string of binary digits
  $pqrs$.
\end{convention}

What makes residues useful for our purposes is that many important operations on $\Zw$ are
well-defined on residues. Here, we say that an operation $f:\Zw\to\Zw$ is {\em well-defined on
residues} if for all $t,s$, $\rho(t)=\rho(s)$ implies $\rho(f(t))=\rho(f(s))$.

For example, two operations that are obviously well-defined on residues are complex conjugation,
which takes the form $(pqrs)\da = rqps$ by (\ref{eqn-adjoint}), and multiplication by $\omega$,
which is just a cyclic shift $\omega(pqrs)=qrsp$. Table~\ref{tab-residue} shows two other important
operations on residues, namely multiplication by $\sqrt{2}$ and the squared norm.

\begin{table}
  \[ \begin{array}{c|c|c} \rho(t) & \rho(\sqrt{2}\,t) & \rho(t\da t)\\\hline
    0000 & 0000 & 0000 \\
    0001 & 1010 & 0001 \\
    0010 & 0101 & 0001 \\
    0011 & 1111 & 1010 \\

    0100 & 1010 & 0001 \\
    0101 & 0000 & 0000 \\
    0110 & 1111 & 1010 \\
    0111 & 0101 & 0001 \\
  \end{array}\qquad
  \begin{array}{c|c|c} \rho(t) & \rho(\sqrt{2}\,t) & \rho(t\da t)\\\hline
    1000 & 0101 & 0001 \\
    1001 & 1111 & 1010 \\
    1010 & 0000 & 0000 \\
    1011 & 1010 & 0001 \\

    1100 & 1111 & 1010 \\
    1101 & 0101 & 0001 \\
    1110 & 1010 & 0001 \\
    1111 & 0000 & 0000 \\
  \end{array}
  \]
  \caption{Some operations on residues}\label{tab-residue}
\end{table}

\begin{definition}[$k$-Residue]
  Let $t\in\Dw$ and let $k$ be a (not necessarily least) denominator exponent for $t$. The {\em
  $k$-residue of $t$}, in symbols $\rho_k(t)$, is defined to be
  \[ 
    \rho_k(t) = \rho(\sqrt{2}^k t).
  \]
\end{definition}

\begin{definition}[Reducibility]
  We say that a residue $x\in\Zbw$ is {\em reducible} if it is of the form $\sqrt{2}\,y$, for some
  $y\in\Zbw$. Moreover, we say that $x\in\Zbw$ is {\em twice reducible} if it is of the form $2y$,
  for some $y\in\Zbw$.
\end{definition}

\begin{lemma}\label{lem-reducible}
  For a residue $x$, the following are equivalent:
  \begin{enumerate}\alphalabels
    \item $x$ is reducible;
    \item $x\in\s{0000,0101,1010,1111}$;
    \item $\sqrt{2}\,x = 0000$;
    \item $x\da x=0000$.
  \end{enumerate}
  Moreover, $x$ is twice reducible iff $x=0000$.
\end{lemma}

\begin{proof}
  By inspection of Table~\ref{tab-residue}.
\end{proof}

\begin{lemma}\label{lem-reducible2}
  Let $t\in\Zw$. Then $t/2\in\Zw$ if and only if $\rho(t)$ is twice reducible, and
  $t/\sqrt{2}\in\Zw$ if and only if $\rho(t)$ is reducible.
\end{lemma}

\begin{proof}
  The first claim is trivial, as $\rho(t)=0000$ if and only if all components of $t$ are even. For
  the second claim, the left-to-right implication is also trivial: assume $t'=t/\sqrt{2}\in\Zw$.
  Then $\rho(t) = \rho(\sqrt{2}\,t')$, which is reducible by definition. Conversely, let $t\in\Zw$
  and assume that $\rho(t)$ is reducible. Then $\rho(t)\in\s{0000, 0101, 1010, 1111}$, and it can
  be seen from Table~\ref{tab-residue} that $\rho(\sqrt{2}\,t)=0000$. Therefore, $\sqrt{2}\,t$ is
  twice reducible by the first claim; hence $t$ is reducible.
\end{proof}

\begin{corollary}\label{cor-reducible3}
  Let $t\in\Dw$ and let $k>0$ be a denominator exponent for $t$. Then $k$ is the least denominator
  exponent for $t$ if and only if $\rho_k(t)$ is irreducible.
\end{corollary}

\begin{proof}
  Since $k$ is a denominator exponent for $t$, we have $\sqrt{2}^k t \in \Zw$. Moreover, $k$ is
  least if and only if $\sqrt{2}^{k-1} t\not\in\Zw$. By Lemma~\ref{lem-reducible2}, this is the
  case if and only if $\rho(\sqrt{2}^k t)=\rho_k(t)$ is irreducible.
\end{proof}

\begin{definition}
  The notions of residue, $k$-residue, reducibility, and twice-reducibility all extend in an
  obvious componentwise way to vectors and matrices. Thus, the residue $\rho(u)$ of a vector or
  matrix $u$ is obtained by taking the residue of each of its entries, and similar for
  $k$-residues. Also, we say that a vector or matrix is reducible if each of its entries is
  reducible, and similarly for twice-reducibility.
\end{definition}


\begin{example}\label{exa-k-residue}
  Consider the matrix
  \[ \small U \,{=}\, \frac{\small 1}{\small \sqrt{2}^3}\!\footnotesize\zmatrix{cccc}{
    -\omega^3+\omega -1          
			& \omega^2+\omega+1 
				& \omega^2 
					& -\omega\\
    \omega^2+\omega 
			& -\omega^3+\omega^2 
				& -\omega^2-1
					& \omega^3+\omega \\
    \omega^3+\omega^2
			& -\omega^3-1
				& 2\omega^2 
					&0 \\
    -1
			& \omega
				& 1 
					& -\omega^3+2\omega
    }.
  \]
  It has least denominator exponent $3$. Its $3$-, $4$-, and $5$-residues are:
  \[ 
    \begin{split}
    &
    \rho_3(U) = \zmatrix{cccc}{
      1011 & 0111 & 0100 & 0010\\
      0110 & 1100 & 0101 & 1010\\
      1100 & 1001 & 0000 & 0000\\
      0001 & 0010 & 0001 & 1000
    },
    \\&
    \rho_4(U) = \zmatrix{cccc}{
      1010 & 0101 & 1010 & 0101\\
      1111 & 1111 & 0000 & 0000\\
      1111 & 1111 & 0000 & 0000\\
      1010 & 0101 & 1010 & 0101
    },\quad
    \rho_5(U) = 0.\end{split}
  \]
\end{example}

% subsubsection residues (end)


% subsection algebraic_background (end)

\subsection{Decomposition into two-level matrices} % (fold)
\label{sub:decomposition_into_two_level_matrices}
Recall that a {\em two-level matrix} is an $n\times n$-matrix that acts non-trivially on at most
two vector components {\cite{neilsen2000:QuantumComputationAndInfo}}. If
\[
  U=\zmatrix{cc}{a&b\\c&d}
\]
is a $2\times 2$-matrix and $\jay\neq\ell$, we write $U\level{\jay,\ell}$ for the two-level
$n\times n$-matrix defined by
\[
  U\level{\jay,\ell}=\begin{blockarray}{cccccc}
    &\matindex{\cdots}& \matindex{\jay} &\matindex{\cdots}& \matindex{\ell} &\matindex{\cdots} \\
    \begin{block}{c(c|c|c|c|c)}
      \matindex{\vdots} & \bigI &&&& \\\cline{2-6}
      \matindex{\jay} & & a && b \\\cline{2-6}
      \matindex{\vdots} & && \bigI && \\\cline{2-6}
      \matindex{\ell} & & c && d &  \\\cline{2-6}
      \matindex{\vdots} &&&&& \bigI \\
    \end{block}
  \end{blockarray}\,,
\]
and we say that $U\level{\jay,\ell}$ is a two-level matrix {\em of type $U$}. Similarly, if $a$ is
a scalar, we write $a\level{\jay}$ for the one-level matrix
\[
  a\level{\jay} = \begin{blockarray}{cccc}
    &\matindex{\cdots}& \matindex{\jay} &\matindex{\cdots} \\
    \begin{block}{c(c|c|c)}
      \matindex{\vdots} & \bigI && \\\cline{2-4}
      \matindex{\jay} & & a & \\\cline{2-4}
      \matindex{\vdots} & && \bigI\\
    \end{block}
  \end{blockarray}\,,
\]
and we say that $a\level{\jay}$  is a one-level matrix {\em of type $a$}.

\begin{lemma}[Row operation]\label{lem-row}
  Let $u=(u_1,u_2)^T\in\Dw^2$ be a vector with denominator exponent $k>0$ and $k$-residue
  $\rho_k(u)=(x_1,x_2)$, such that $x_1\da x_1 = x_2\da x_2$. Then there exists a sequence of
  matrices $U_1,\ldots,U_h$, each of which is $H$ or $T$, such that $v=U_1\cdots U_hu$ has
  denominator exponent $k-1$, or equivalently, $\rho_k(v)$ is defined and reducible.
\end{lemma}

\begin{proof}
  It can be seen from Table~\ref{tab-residue} that $x_1\da x_1$ is either $0000$, $1010$, or $0001$.
  \begin{itemize}
  \item Case 1: $x_1\da x_1 = x_2\da x_2 = 0000$. In this case, $\rho_k(u)$ is already reducible,
    and there is nothing to show.
  \item Case 2: $x_1\da x_1 = x_2\da x_2 = 1010$. In this case, we know from
    Table~\ref{tab-residue} that $x_1,x_2\in\s{0011, 0110, 1100, 1001}$. In particular, $x_1$ is a
    cyclic permutation of $x_2$, say, $x_1=\omega^m x_2$. Let $v=HT^mu$. Then
    \[
      \begin{split}
        \rho_k(\sqrt{2}\,v) &=
        \rho_k(\zmatrix{cc}{1&1\\1&-1}\zmatrix{cc}{1&0\\0&\omega^m}\zmatrix{c}{u_1\\u_2})
        \\&= \rho_k\zmatrix{c}{u_1+\omega^m u_2\\ u_1-\omega^m u_2} \\&=
        \zmatrix{c}{x_1+\omega^m x_2\\x_1-\omega^m x_2} =
        \zmatrix{c}{0000\\0000}.
      \end{split}
    \]
    This shows that $\rho_k(\sqrt{2}\,v)$ is twice reducible; therefore, $\rho_k(v)$ is defined and
    reducible as claimed.
  \item Case 3: $x_1\da x_1 = x_2\da x_2 = 0001$. In this case, we know from Table~\ref{tab-residue}
    that $x_1,x_2\in\s{0001,0010,0100,1000}\cup\s{0111,1110,1101,1011}$. If both $x_1,x_2$ are in
    the first set, or both are in the second set, then $x_1$ and $x_2$ are cyclic permutations of
    each other, and we proceed as in case 2. The only remaining cases are that $x_1$ is a cyclic
    permutation of $0001$ and $x_2$ is a cyclic permutation of $0111$, or vice versa. But then
    there exists some $m$ such that $x_1+\omega^mx_2=1111$. Letting $u'=HT^mu$, we have
    \[
      \begin{split}
        \rho_k(\sqrt{2}\,u') &=
        \rho_k(\zmatrix{cc}{1&1\\1&-1}\zmatrix{cc}{1&0\\0&\omega^m}\zmatrix{c}{u_1\\u_2})
        \\&= \rho_k\zmatrix{c}{u_1+\omega^m u_2\\ u_1-\omega^m u_2} \\&=
        \zmatrix{c}{x_1+\omega^m x_2\\x_1-\omega^m x_2} =
        \zmatrix{c}{1111\\1111}.
      \end{split}
    \]
    Since this is reducible, $u'$ has denominator exponent $k$. Let $\rho_k(u')=(y_1,y_2)$. Because
    $\sqrt{2}\,y_1=\sqrt{2}\,y_2=1111$, we see from Table~\ref{tab-residue} that
    $y_1,y_2\in\s{0011,0110,1100,1001}$ and $y_1\da y_1=y_2\da y_2=1010$. Therefore, $u'$ satisfies
    the condition of case 2 above. Proceeding as in case 2, we find $m'$ such that $v=HT^{m'}
    u'=HT^{m'} HT^mu$ has denominator exponent $k-1$. This finishes the proof.\qedhere
  \end{itemize}
\end{proof}

\begin{lemma}[Column lemma]\label{lem-column}
  Consider a unit vector $u\in\Dw^n$, i.e., an $n$-dimensional column vector of norm 1 with entries
  from the ring $\Dw$. Then there exist a sequence $U_1,\ldots,U_h$ of one- and two-level unitary
  matrices of types $X$, $H$, $T$, and $\omega$ such that $U_1\cdots U_hu = e_1$, the first
  standard basis vector.
\end{lemma}

\begin{proof}
  The proof is by induction on $k$, the least denominator exponent of $u$. Let
  $u=(u_1,\ldots,u_n)^T$.
  \begin{itemize}
  \item Base case. Suppose $k=0$. Then $u\in\Zw^n$. Since by assumption $\norm{u}^2=1$, it follows 
    by Lemma~\ref{lem-ring-norm} that $\weight{u}^2=1$. Since $u_1,\ldots,u_n$ are elements of
    $\Zw$, their weights are non-negative integers. It follows that there is precisely one $\jay$
    with $\weight{u_\jay}=1$, and $\weight{u_\ell}=0$ for all $\ell\neq \jay$. Let
    $u'=X\level{1,\jay}u$ if $\jay\neq 1$, and $u'=u$ otherwise. Now $u'_1$ is of the form
    $\omega^{-m}$, for some $m\in\s{0,\ldots,7}$, and $u'_\ell=0$ for all $\ell\neq 1$. We have
    $\omega^m\level{1}u'=e_1$, as desired.
  \item Induction step. Suppose $k>0$. Let $v=\sqrt{2}^ku\in\Zw^n$, and let $x=\rho_k(u) =
    \rho(v)$. From $\norm{u}^2=1$, it follows that $\norm{v}^2 = v_1\da v_1+\ldots+v_n\da v_n = 2^k$.
    Taking residues of the last equation, we have
    \begin{equation}
      x_1\da x_1+\ldots+x_n\da x_n = 0000.
    \end{equation}
    It can be seen from Table~\ref{tab-residue} that each summand $x_\jay\da x_\jay$ is either
    $0000$, $0001$, or $1010$. Since their sum is $0000$, it follows that there is an even number
    of $\jay$ such that $x_\jay\da x_\jay=0001$, and an even number of $\jay$ such that $x_\jay\da
    x_\jay=1010$.

    We do an inner induction on the number of irreducible components of $x$. If $x$ is reducible,
    then $u$ has denominator exponent $k-1$ by Corollary~\ref{cor-reducible3}, and we can apply the
    outer induction hypothesis. Now suppose there is some $\jay$ such that $x_\jay$ is irreducible;
    then $x_\jay\da x_\jay\neq 0000$ by Lemma~\ref{lem-reducible}. Because of the evenness property
    noted above, there must exist some $\ell\neq \jay$ such that $x_\jay\da x_\jay=x_\ell\da
    x_\ell$. Applying Lemma~\ref{lem-row} to $u'=(u_\jay,u_\ell)^T$, we find a sequence $\vec U$ of
    row operations of types $H$ and $T$, making $\rho_k(\vec Uu')$ reducible. We can lift this to a
    two-level operation $\vec U\level{\jay,\ell}$ acting on $u$; thus $\rho_k(\vec
    U\level{\jay,\ell}u)$ has fewer irreducible components than $x=\rho_k(u)$, and the inner
    induction hypothesis applies.\qedhere
  \end{itemize}
\end{proof}

\begin{lemma}[Matrix decomposition]\label{lem-matrix-decomposition}
  Let $U$ be a unitary $n\times n$-matrix with entries in $\Dw$. Then there exists a sequence
  $U_1,\ldots,U_h$ of one- and two-level unitary matrices of types $X$, $H$, $T$, and $\omega$ such
  that $U=U_1\cdots U_h$.
\end{lemma}

\begin{proof}
  Equivalently, it suffices to show that there exist one- and two-level unitary matrices
  $V_1,\ldots,V_h$ of types $X$, $H$, $T$, and $\omega$ such that $V_h\cdots V_1U=I$. This is an
  easy consequence of the column lemma, exactly as in e.g. {\cite[Sec.~4.5.1]{neilsen2000:QuantumComputationAndInfo}}.
  Specifically, first use the column lemma to find suitable one- and two-level row operations
  $V_1,\ldots,V_{h_1}$ such that the leftmost column of $V_{h_1}\cdots V_1 U$ is $e_1$. Because
  $V_{h_1}\cdots V_1 U$ is unitary, it is of the form
  \[ 
    \zmatrix{c|c}{1&0\\\hline \rule{0mm}{2.1ex}0&U'}.
  \]
  Now recursively find row operations to reduce $U'$ to the identity matrix.
\end{proof}

\begin{example}\label{exa-decomp}
  We will decompose the matrix $U$ from Example~\ref{exa-k-residue}. We start with the first column
  $u$ of $U$:
	\[
    \begin{array}{c}
    \displaystyle
    u=\frac{1}{\sqrt{2}^3}\zmatrix{c}{
    -\omega^3+\omega -1\\
    \omega^2+\omega \\
    \omega^3+\omega^2\\
    -1
    }, \\\\[-1.5ex]
    \rho_3(u) = \zmatrix{c}{
    1011\\
    0110\\
    1100\\
    0001
    },\quad \rho_3(u_\jay\da u_\jay) = \zmatrix{c}{
    0001\\
    1010 \\
    1010\\
    0001
    }.
    \end{array}
  \]
	Rows 2 and 3 satisfy case 2 of Lemma~\ref{lem-row}. As they are not aligned, first apply
  $T_{[2,3]}^3$ and then $H_{[2,3]}$. Rows 1 and 4 satisfy case 3. Applying $H_{[1,4]} T_{[1,4]}^2$,
  the residues become $\rho_3(u'_1)=0011$ and $\rho_3(u'_4)=1001$, which requires applying $H_{[1,4]}
  T_{[1,4]}$. We now have
	\[
    \begin{array}{@{}c@{}}
      \displaystyle 
      H_{[1,4]} T_{[1,4]} H_{[1,4]} T_{[1,4]}^2 H_{[2,3]} T_{[2,3]}^3 u=
  	  v=\frac{1}{\footnotesize\sqrt{2}^2}\small\zmatrix{c}{
      0 \\
  		0 \\
      \omega^2{+}\omega  \\
      -\omega {+}1 
      }, \\\\[-1.5ex]
      \rho_2(v) = \zmatrix{c}{
      0000\\
      0000\\
      0110\\
      0011
      },\quad \rho_2(v_\jay\da v_\jay) = \zmatrix{c}{
      0000\\
      0000\\
      1010\\
      1010
      }.
    \end{array}
  \]
	Rows 3 and 4 satisfy case 2, while rows 1 and 2 are already reduced. We reduce rows 3 and 4 by
  applying $H_{[3,4]} T_{[3,4]}$. Continuing, the first column is completely reduced to $e_1$ by
  further applying $\omega_{[1]}^7 X_{[1,4]} H_{[3,4]} T_{[3,4]}^3$. The complete decomposition of
  $u$ is therefore given by
	\[
    \begin{split}
      W_1={}&\omega_{[1]}^7  X_{[1,4]} H_{[3,4]} T_{[3,4]}^3H_{[3,4]} T_{[3,4]}\\&
      H_{[1,4]} T_{[1,4]} H_{[1,4]} T_{[1,4]}^2 H_{[2,3]} T_{[2,3]}^3.
    \end{split}
    \]
	Applying this to the original matrix $U$, we have $W_1U=$
	\[
    \small \frac{\small 1}{\small\sqrt{2}^3}\footnotesize\zmatrix{cccc}{
    \sqrt{2}^3             & 0      & 0                           & 0\\
    0 & \omega^3{-}\omega^2{+}\omega{+}1  & {-}\omega^2{-}\omega{-}1          & \omega^2\\
    0 & 0                           & \omega^3{+}\omega^2{-}\omega{+}1  & \omega^3{+}\omega^2{-}\omega{-}1\\
    0 & \omega^3{+}\omega^2{+}\omega{+}1 & \omega^2                    & \omega^3{-}\omega^2{+}1
    }.
  \]
  Continuing with the rest of the columns, we find $W_2 = \omega_{[2]}^6 H_{[2,4]} T_{[2,4]}^3
  H_{[2,4]} T_{[2,4]}$, $W_3 = \omega_{[3]}^4 H_{[3,4]} T_{[3,4]}^3 H_{[3,4]}$, and
  $W_4=\omega_{[4]}^5$. We then have $U = W_1\da\, W_2\da\, W_3\da\, W_4\da$, or explicitly:
  \[
    \begin{split} 
      U ={}&  T_{[2,3]}^5 H_{[2,3]} T_{[1,4]}^6 H_{[1,4]} T_{[1,4]}^7 H_{[1,4]}\\&
    					T_{[3,4]}^7 H_{[3,4]} 
    					T_{[3,4]}^5 H_{[3,4]}
                                            X_{[1,4]} \omega_{[1]}\\&
      T_{[2,4]}^7 H_{[2,4]} T_{[2,4]}^5 H_{[2,4]} \omega_{[2]}^2
      H_{[3,4]} T_{[3,4]}^5 H_{[3,4]} \omega_{[3]}^4
      \omega_{[4]}^3.
    \end{split}
  \]
\end{example}


% subsection decomposition_into_two_level_matrices (end)

\subsection{Main result} % (fold)
\label{sub:main_result}
Consider the ring $\Z[\frac1{\sqrt{2}},i]$, consisting of complex numbers of the form
\[
  \frac{1}{2^n} (a + bi + c\sqrt{2} + di\sqrt{2}),
\]
where $n\in\N$ and $a,b,c,d\in\Z$. Our goal is to prove the following theorem, which was
conjectured by Kliuchnikov et al.~{\cite{Kliuchnikov-et-al}}:
\begin{theorem}\label{thm-main}
  Let $U$ be a unitary $2^n\times 2^n$ matrix. Then the following are equivalent:
\begin{enumerate}\alphalabels
  \item[(a)] $U$ can be exactly represented by a quantum circuit over the Clifford+$T$ gate set,
    possibly using some finite number of ancillas that are initialized and finalized in state $\ket0$.
  \item[(b)] The entries of $U$ belong to the ring $\Z[\frac1{\sqrt{2}},i]$.
\end{enumerate}
Moreover, in (a), a single ancilla is always sufficient.
\end{theorem}
\begin{proof}
  First note that, since all the elementary Clifford+$T$ gates, as shown in (\ref{eqn-generators}),
  take their matrix entries in $\Dw=\Z[\frac1{\sqrt{2}},i]$, the implication (a) $\implies$ (b) is
  trivial. For the converse, let $U$ be a unitary $2^n\times 2^n$ matrix with entries from $\Dw$.
  By Lemma~\ref{lem-matrix-decomposition}, $U$ can be decomposed into one- and two-level matrices
  of types $X$, $H$, $T$, and $\omega$. It is well-known that each such matrix can be further
  decomposed into controlled-not gates and multiply-controlled $X$, $H$, $T$, and $\omega$-gates,
  for example using Gray codes {\cite[Sec.~4.5.2]{neilsen2000:QuantumComputationAndInfo}}. But all of these gates have
  well-known exact representations in Clifford+$T$ with ancillas, see e.g. {\cite[Fig.~4(a) and
  Fig.~9]{AMMR12}} (and noting that a controlled-$\omega$ gate is the same as a $T$-gate). This
  finishes the proof of (b) $\implies$ (a).
  
  The final claim that needs to be proved is that a circuit for $U$ can always be found using at
  most one ancilla. It is already known that for $n>1$, an ancilla is sometimes necessary
  {\cite{Kliuchnikov-et-al}}. To show that a single ancilla is sufficient, in light of the above
  decomposition, it is enough to show that the following can be implemented with one ancilla:
  \begin{enumerate}\alphalabels
  \item a multiply-controlled $X$-gate;
  \item a multiply-controlled $H$-gate;
  \item a multiply-controlled $T$-gate.
  \end{enumerate}
    We first recall from {\cite[Fig.~4(a)]{AMMR12}} that a
    singly-controlled Hadamard gate can be decomposed into Clifford+$T$
    gates with no ancillas:
  \[ 
    \m{\begin{tikzqcircuit}[scale=0.5] 
      \grid{2}{0,1};
      \controlled{\tqgate{$H$}}{1,0}{1};
    \end{tikzqcircuit}}
  =\!\!\!
    \m{\begin{tikzqcircuit}[scale=0.5]
      \grid{11}{0,1};
      \tqgate{$S$}{1,0};
      \tqgate{$H$}{2.5,0};
      \tqgate{$T$}{4,0};
      \controlled{\notgate}{5.5,0}{1};
      \widegate{$T\da$}{.55}{7,0};
      \tqgate{$H$}{8.5,0};
      \widegate{$S\da$}{.55}{10,0};
    \end{tikzqcircuit}.}
  \]
  We also recall that an $n$-fold controlled $iX$-gate can be represented using $O(n)$ Clifford+$T$
  gates with no ancillas. Namely, for $n=1$, we have
  \[
    \m{\begin{tikzqcircuit}[scale=0.5]
        \grid{2}{1,2};
        \controlled{\widegate{$iX$}{.65}}{1,1}{2};
      \end{tikzqcircuit}}
      =\!\!\!
      \m{\begin{tikzqcircuit}[scale=0.5]
        \grid{3.5}{1,2};
        \tqgate{$S$}{1,2};
        \controlled{\notgate}{2.5,1}{2};
      \end{tikzqcircuit},}
  \]
  and for $n\geq 2$, we can use
  \[
      \m{\begin{tikzqcircuit}[scale=0.5]
      \grid{2}{1,2,3,4,5};
      \draw(0.4,2.7) node[anchor=center]{$\vdots$};
      \draw(0.4,4.7) node[anchor=center]{$\vdots$};
      \controlled{\widegate{$iX$}{.65}}{1,1}{2,3,4,5};
      \draw(1.6,2.7) node[anchor=center]{$\vdots$};
      \draw(1.6,4.7) node[anchor=center]{$\vdots$};
    \end{tikzqcircuit}}
    =\!\!\!
    \m{\begin{tikzqcircuit}[scale=0.5]
      \gridx{0.1}{13.5}{1,2,3,4,5};
      \tqgate{$H$}{1.1,1};
      \widegate{$T\da$}{.55}{2.5,1};
      \controlled{\notgate}{3.75,1}{2,3};
      \tqgate{$T$}{5,1};
      \controlled{\notgate}{6.25,1}{4,5};
      \widegate{$T\da$}{.55}{7.5,1};
      \controlled{\notgate}{8.75,1}{2,3};
      \tqgate{$T$}{10,1};
      \controlled{\notgate}{11.25,1}{4,5};
      \tqgate{$H$}{12.5,1};
      \draw(3.25,2.7) node[anchor=center]{$\vdots$};
      \draw(4.25,2.7) node[anchor=center]{$\vdots$};
      \draw(5.75,4.7) node[anchor=center]{$\vdots$};
      \draw(6.75,4.7) node[anchor=center]{$\vdots$};
      \draw(8.25,2.7) node[anchor=center]{$\vdots$};
      \draw(9.25,2.7) node[anchor=center]{$\vdots$};
      \draw(10.75,4.7) node[anchor=center]{$\vdots$};
      \draw(11.75,4.7) node[anchor=center]{$\vdots$};
    \end{tikzqcircuit},}
  \]
  with further decompositions of the multiply-controlled not-gates as in
  {\cite[Lem.~7.2]{Barenco-etal-1995}} and {\cite[Fig.~4.9]{neilsen2000:QuantumComputationAndInfo}}. 
  We then obtain the following representations for (a)--(c), using only one ancilla:
  \[
  (a)
    \m{\begin{tikzqcircuit}[scale=0.47]
      \grid{2}{0.5,2,3};
      \draw(0.4,2.7) node[anchor=center]{$\vdots$};
      \controlled{\tqgate{$X$}}{1,0.5}{2,3};
      \draw(1.6,2.7) node[anchor=center]{$\vdots$};
    \end{tikzqcircuit}}
    =\!\!
    \m{\begin{tikzqcircuit}[scale=0.47]
      \grid{8.5}{0,2,3};
      \gridx{1}{7.5}{1};
      \draw(1,2.7) node[anchor=center]{$\vdots$};
      \init{$0$}{1,1};
      \controlled{\widegate{$iX$}{.75}}{2.5,1}{2,3};
      \controlled{\tqgate{$X$}}{4.25,0}{1};
      \controlled{\widegate{$-iX$}{.75}}{6,1}{2,3};
      \term{$0$}{7.5,1};
      \draw(7.5,2.7) node[anchor=center]{$\vdots$};
    \end{tikzqcircuit}}
  \]\[
  (b)
    \m{\begin{tikzqcircuit}[scale=0.47]
      \grid{2}{0.5,2,3};
      \draw(0.4,2.7) node[anchor=center]{$\vdots$};
      \controlled{\tqgate{$H$}}{1,0.5}{2,3};
      \draw(1.6,2.7) node[anchor=center]{$\vdots$};
    \end{tikzqcircuit}}
    =\!\!
    \m{\begin{tikzqcircuit}[scale=0.47]
      \grid{8.5}{0,2,3};
      \gridx{1}{7.5}{1};
      \draw(1,2.7) node[anchor=center]{$\vdots$};
      \init{$0$}{1,1};
      \controlled{\widegate{$iX$}{.75}}{2.5,1}{2,3};
      \controlled{\tqgate{$H$}}{4.25,0}{1};
      \controlled{\widegate{$-iX$}{.75}}{6,1}{2,3};
      \term{$0$}{7.5,1};
      \draw(7.5,2.7) node[anchor=center]{$\vdots$};
    \end{tikzqcircuit}}
  \]\[
  (c)
    \m{\begin{tikzqcircuit}[scale=0.47]
      \grid{2}{0.5,2,3};
      \draw(0.4,2.7) node[anchor=center]{$\vdots$};
      \controlled{\tqgate{$T$}}{1,0.5}{2,3};
      \draw(1.6,2.7) node[anchor=center]{$\vdots$};
    \end{tikzqcircuit}}
    =\!\!\!
    \m{\begin{tikzqcircuit}[scale=0.47]
      \grid{8.5}{1,2,3};
      \gridx{1}{7.5}{0};
      \draw(1,2.7) node[anchor=center]{$\vdots$};
      \init{$0$}{1,0};
      \controlled{\widegate{$iX$}{.75}}{2.5,0}{1,2,3};
      \tqgate{$T$}{4.25,0}{1};
      \controlled{\widegate{$-iX$}{.75}}{6,0}{1,2,3};
      \term{$0$.}{7.5,0};
      \draw(7.5,2.7) node[anchor=center]{$\vdots$};
    \end{tikzqcircuit}}
  \]
\end{proof}

\begin{remark}
  The fact that one ancilla is always sufficient in Theorem~\ref{thm-main} is primarily of
  theoretical interest. In practice, one may assume that on most quantum computing architectures,
  ancillas are relatively cheap. Moreover, the use of additional ancillas can significantly reduce
  the size and depth of the generated circuits (see e.g.~\cite{Selinger-toffoli}).
\end{remark}

% subsection main_result (end)

\subsection{The no-ancilla case} % (fold)
\label{sub:the_no_ancilla_case}
\begin{lemma}\label{lem-det1}
  Under the hypotheses of Theorem~\ref{thm-main}, assume that $\det U=1$. Then $U$ can be exactly
  represented by a Clifford+$T$ circuit with no ancillas.
\end{lemma}

\begin{proof}
  This requires only minor modifications to the proof of Theorem~\ref{thm-main}. First observe that
  whenever an operator of the form $HT^m$ was used in the proof of Lemma~\ref{lem-row}, we can
  instead use $T^{-m}(iH)T^m$ without altering the rest of the argument. In the base case of
  Lemma~\ref{lem-column}, the operator $X\level{1,\jay}$ can be replaced by $iX\level{1,\jay}$.
  Also, in the base case of Lemma~\ref{lem-column}, whenever $n\geq 2$, the operator
  $\omega\level{1}$ can be replaced by $W\level{1,2}$, where
  \[ 
    W = \zmatrix{cc}{\omega&0\\0&\omega^{-1}}.
  \]
  Therefore, the decomposition of Lemma~\ref{lem-matrix-decomposition} can be performed so as to
  yield only two-level matrices of types
  \begin{equation}\label{eqn-det1}
    iX,\quad T^{-m}(iH)T^m,\quad \mbox{and}\ W,
  \end{equation}
  plus at most one one-level matrix of type $\omega^m$. But since all two-level matrices of types
  (\ref{eqn-det1}), as well as $U$ itself, have determinant 1, it follows that $\omega^m = 1$. We
  finish the proof by observing that the multiply-controlled operators of types (\ref{eqn-det1})
  possess ancilla-free Clifford+$T$ representations, with the latter two given by
  \[
    \m{\begin{tikzqcircuit}[scale=0.47]
        \grid{5}{1,2,3};
        \draw(1,2.7) node[anchor=center]{$\vdots$};
        \controlled{\widegate{$T^{-m}(iH)T^m$}{2}}{2.5,1}{2,3};
        \draw(4,2.7) node[anchor=center]{$\vdots$};
      \end{tikzqcircuit}}
    =\!\!
    \m{\begin{tikzqcircuit}[scale=0.47]
        \grid{14.9}{1,2,3};
        \draw(1,2.7) node[anchor=center]{$\vdots$};
        \widegate{$T^m$}{.55}{1,1};
        \tqgate{$S$}{2.5,1};
        \tqgate{$H$}{4,1};
        \tqgate{$T$}{5.5,1};
        \controlled{\widegate{$iX$}{.75}}{7.25,1}{2,3};
        \widegate{$T\da$}{.55}{9,1};
        \tqgate{$H$}{10.5,1};
        \widegate{$S\da$}{.55}{12,1};
        \widegate{$T^{-m}$}{.75}{13.75,1};
        \draw(13.9,2.7) node[anchor=center]{$\vdots$};
      \end{tikzqcircuit}}
  \]\[
  \m{\begin{tikzqcircuit}[scale=0.47]
      \grid{2}{1,2,3};
      \draw(0.4,2.7) node[anchor=center]{$\vdots$};
      \controlled{\widegate{$W$}{.6}}{1,1}{2,3};
      \draw(1.6,2.7) node[anchor=center]{$\vdots$};
    \end{tikzqcircuit}}
  =\!\!
  \m{\begin{tikzqcircuit}[scale=0.47]
      \grid{7.5}{1,2,3};
      \draw(0.4,2.7) node[anchor=center]{$\vdots$};
      \controlled{\widegate{$iX$}{.75}}{1.25,1}{2,3};
      \tqgate{$T$}{3,1};
      \controlled{\widegate{$-iX$}{.75}}{4.75,1}{2,3};
      \widegate{$T\da$}{.55}{6.5,1};
      \draw(7.1,2.7) node[anchor=center]{$\vdots$};
    \end{tikzqcircuit}}
  \]
\end{proof}

As a corollary, we obtain a characterization of the $n$-qubit Clifford+$T$ group (with no ancillas)
for all $n$:
  
\begin{corollary}\label{cor-noancilla}
  Let $U$ be a unitary $2^n\times 2^n$ matrix. Then the following are equivalent:
\begin{enumerate}\alphalabels
\item[(a)] $U$ can be exactly represented by a quantum
  circuit over the Clifford+$T$ gate set on $n$ qubits with no ancillas.
\item[(b)] The entries of $U$ belong to the ring
  $\Z[\frac1{\sqrt{2}},i]$, and:
  \begin{itemize}
    \item $\det U=1$, if $n\geq 4$;
    \item $\det U\in\s{-1,1}$, if $n=3$;
    \item $\det U\in\s{i,-1,-i,1}$, if $n=2$;
    \item $\det U\in\s{\omega,i,\omega^3,-1,\omega^5,-i,\omega^7,1}$,
      if $n\leq 1$.
    \end{itemize}
  \end{enumerate}
\end{corollary}

\begin{proof}
  For (a) $\implies$ (b), it suffices to note that each of the generators of the Clifford+$T$ group,
  regarded as an operation on $n$ qubits, satisfies the conditions in (b). For (b) $\implies$ (a), let
  us define for convenience $d_0=d_1=\omega$, $d_2=i$, $d_3=-1$, and $d_n=1$ for $n\geq 4$. First
  note that for all $n$, the Clifford+$T$ group on $n$ qubits (without ancillas) contains an
  element $D_n$ whose determinant is $d_n$, namely $D_n=I$ for $n\geq 4$, $D_3=T\otimes I\otimes
  I$, $D_2=T\otimes I$, $D_1=T$, and $D_0=\omega$. Now consider some $U$ satisfying (b). By
  assumption, $\det U=d_n^m$ for some $m$. Let $U'=UD_n^{-m}$, then $\det U'=1$. By
  Lemma~\ref{lem-det1}, $U'$, and therefore $U$, is in the Clifford+$T$ group with no ancillas.
\end{proof}

\begin{remark}
  Note that the last condition in Corollary~\ref{cor-noancilla}, namely that $\det U$ is a power of
  $\omega$ for $n\leq 1$, is of course redundant, as this already follows from $\det
  U\in\Z[\frac1{\sqrt{2}},i]$ and $|\det U|=1$. We stated the condition for consistency with the
  case $n\geq 2$.
\end{remark}

\begin{remark}
  The situation of Theorem~\ref{thm-main} and Corollary~\ref{cor-noancilla} is analogous to the
  case of classical reversible circuits. It is well-known that the not-gate, controlled-not gate,
  and Toffoli gate generate all classical reversible functions on $n\leq 3$ bits. For $n\geq 4$
  bits, they generate exactly those reversible boolean functions that define an {\em even
  permutation} of their inputs (or equivalently, those that have determinant 1 when viewed in
  matrix form) {\cite{Musset97}}; the addition of a single ancilla suffices to recover all boolean
  functions.
\end{remark}

% subsection the_no_ancilla_case (end)

\subsection{Complexity} % (fold)
\label{sub:complexity}
The proof of Theorem~\ref{thm-main} immediately yields an algorithm, albeit not a very efficient
one, for synthesizing a Clifford+$T$ circuit with ancillas from a given operator $U$. We estimate
the size of the generated circuits.

We first estimate the number of (one- and two-level) operations generated by the matrix
decomposition of Lemma~\ref{lem-matrix-decomposition}. The row operation from Lemma~\ref{lem-row}
requires only a constant number of operations. Reducing a single $n$-dimensional column from
denominator exponent $k$ to $k-1$, as in the induction step of Lemma~\ref{lem-column}, requires
$O(n)$ operations; therefore, the number of operations required to reduce the column completely is
$O(nk)$.

Now consider applying Lemma~\ref{lem-matrix-decomposition} to an $n\times n$-matrix with least
denominator exponent $k$. Reducing the first column requires $O(nk)$ operations, but unfortunately,
it may {\em increase} the least denominator exponent of the rest of the matrix, in the worst case,
to $3k$. Namely, each row operation of Lemma~\ref{lem-row} potentially increases the denominator
exponent by $2$, and any given row may be subject to up to $k$ row operations, resulting in a
worst-case increase of its denominator exponent from $k$ to $3k$ during the reduction of the first
column. It follows that reducing the second column requires up to $O(3(n-1)k)$ operations, reducing
the third column requires up to $O(9(n-2)k)$ operations, and so on. Using the identity
$\sum_{j=0}^{n-1}3^j(n-j) = (3^{n+1}-2n-3)/4$, this results in a total of $O(3^nk)$ one- and
two-level operations for Lemma~\ref{lem-matrix-decomposition}.

In the context of Theorem~\ref{thm-main}, we are dealing with $n$ qubits, i.e., a $2^n\times
2^n$-operator, which therefore decomposes into $O(3^{2^n}k)$ two-level operations. Using one
ancilla, each two-level operation can be decomposed into $O(n)$ Clifford+$T$ gates, resulting in a
total gate count of $O(3^{2^n}\!nk)$ elementary Clifford+$T$ gates.

% subsection complexity (end)

% section exact_synthesis_of_multi_qubit_operators (end)
% chapter synthesis_of_quantum_operations (end)
