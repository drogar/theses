%!TEX root = /Users/gilesb/UofC/thesis/phd-thesis/phd-thesis.tex
\chapter{Introduction}\label{chap:introduction}
\section{Summary}\label{sec:summary}
A ``quantum'' setting has a fundamental duality given by the ``dagger'' of dagger
categories\cite{selinger05:dagger,abramsky05:abstractscalars}. On the other hand, classical
computation is fundamentally asymmetric and has no duality. In passing from a quantum setting to a more classical setting, one
may want to keep this duality for as long as possible and, thus, consider the intermediate step of
passing to ``reversible'' computation --- which has an obvious self-duality given by the ability to
reverse the computation. It is reasonable to
wonder whether one can then pass from a reversible setting to a classical setting quite
independently from the underlying quantum setting. Such an abstract
passage would allow a direct translation into the reversible world of the classical notions of
computation, for example.

Of course, from a quantum setting, it is already possible to pass directly to a classical setting by
taking the homomorphisms between special coalgebras, where ``special'' means the coalgebra must be
the algebra part of a separable Frobenius algebra. That the coalgebra should be special in this manner
may be justified by regarding this as a two step process through reversible computation. However,
this leaves some gaps: How does one pass, in general, between a quantum setting to a reversible
setting and how does one obtain a classical setting from a reversible setting? This thesis answers
these questions.
\section{Background of reversible computation}
\label{sec:background}

In 1961, Landauer\cite{landauer1961irreversibility} examined irreversible computing
and showed that it must dissipate energy, i.e., produce heat at a specified minimal level. This is
due to applying a physically irreversible operation to non-random data, leading to an entropy
increase in the computer. While there are various objections to the connection between
\emph{logical} irreversibility and heat generation, summarized by Bennett\cite{bennett2003notes},
this led to an interest in exploring reversible computation, because of its potential energy advantage.

Bennett, in 1973,  \cite{bennett:1973reverse} showed how one can simulate an ordinary Turing machine
using a reversible Turing machine, based on reversible transitions. Since approximately 2000, there has
been an increased interest in reversible computing. Active areas include research in database
theory, specifically the view-update problem
\cite{Pierce06:lenses,Pierce05:combinators,Johnson08:reverse} and quantum computing. Recall that
quantum computation may be modelled by unitary transforms
\cite{neilsen2000:QuantumComputationAndInfo}, each of which is reversible, followed by an
irreversible measurement.

An important aspect of the treatment of reversible computing is the consideration of the partiality
inherit in programs, as it is possible for programs to never provide a result for certain
inputs. Some of the reversibility research referenced in this section considers partiality to a
greater or lesser degree, but none of them treat it as a central consideration.

Partiality was shown to have a purely algebraic description by Cockett and Lack in
\cite{cockett2002:restcategories1,cockettlack2003:restcategories2,cockettlack2004:restcategories3}. They
introduce a restriction  operator on maps, which associated to a map a partial identity on its
domain. In \cite{cockett2002:restcategories1}, they recalled the concept of \emph{inverse
  category}, a category equipped with a restriction operator in which all the maps have partial
inverses, i.e., are reversible. Categories with restriction operators are presented in
Chapter~\ref{chap:restriction_categories}, while inverse categories are explored in
Chapter~\ref{cha:inverse_categories}.

The semantics of reversible computing has been explored in a variety of ways, including by
developing various reversible programming languages. An early example of this is Janus
\cite{lutz1986janus}, an imperative language written as an experiment in producing a language that
did not erase information. However, it does not appear that any semantic underpinnings were
developed for this language. Additionally, there are special purpose reversible languages, such as
biXid \cite{bixid06}, a language developed explicitly to transform XML \cite{xml} from one data
schema to another. The main novelty of biXid is that a single program targets two schemas and will
transform in either direction.

Zuliani \cite{zuliani01:reversibility} provides a reversible language with an operational
semantics. He examines logical reversibility via comparing the probabilistic Guarded Command
Language ($pGCL$) \cite{MorganIver99} to the quantum Guarded Command Language ($qGCL$)
\cite{sanders:quantum}. Zuliani provides a method for transforming an irreversible $pGCL$ program
into a reversible one. This is accomplished via an application of expectation transform semantics to
the $pGCL$ program. Interestingly, in this work, partial programs are specifically excluded from the
definition of reversible programs. The initial definition of a reversible program is strict, i.e.,
the program is equivalent to $skip$ which does nothing. To alleviate this and allow us to extract
the output, Zuliani follows the example of \cite{bennett:1973reverse} and modifies the result so
that the output is copied before reversing the rest of the program.

In addition, a number of reversible calculi have been developed. Danos and Krivine
\cite{danos2004reversible} extend CCS (Calculus for Communicating Systems)
\cite{milner1980calculus,milner1989communication} to produces RCCS, which adds reversible
transitions to CCS. This is done by adding a syntax for backtracking, together with a labelling
which guides the backtracking. The interesting aspect of this work is the applicability to
concurrent programs.

Phillips and Uladowski \cite{phillips2006operational} take a different approach to creating a
reversible CCS from that of Danos and Krivine. Rather, their stated goal is to use a structural
approach, inspired by \cite{abramsky05:reversible}. The paper is only an initial step in this
process, primarily explaining how to turn dynamic rules (such as choice operators) into a series of
static rules that keep all the information of the input. For example (from the paper), in standard
CCS, we have the rule
\[
  \infer{X+Y \to X'}{X\to X'}.
\]
To preserve information and allow reversibility, this is replaced with
\[
  \infer{X+Y \to X' +Y}{X\to X'}.
\]

In \cite{abramsky05:reversible}, Abramsky considers linear logic as his computational model. This is
done by producing a Linear Combinatory Algebra \cite{abramsky02:GOI} from the involutive reversible
maps over a term algebra and showing these are bi-orthogonal automata. (An automata is considered
orthogonal if it is non-ambiguous and left-linear. It is bi-orthogonal when both the automata and
its converse are both orthogonal). While the paper does use reversible term rewritings as the
basis for computation, its emphasis is on how one can derive a linear combinatory algebra.
Linear combinatory algebras are themselves not reversible systems.

In \cite{mu06bidirectional} and \cite{muetal04:injreversible}, Mu et.al. introduce the language
\textbf{Inv}, a language that is composed only of partial injective functions. The language has an
operational semantics based on determinate relations and converses. They provide a variety of examples of the
language, including translations from XML to HTML and simple functions such as $wrap$, which wraps
its argument into a list. They continue by describing how non-injective functions may be converted to
injective ones in \textbf{Inv} via the addition of logging. In fact, they use this logging to argue
the language is equivalent in terms of power to the reversible Turing machine of
\cite{bennett:1973reverse}.

Some of the ideas of Mu et.al. are closely related to the theory developed in this thesis. For
example,  given two functions $f,g$ of \textbf{Inv}, constructing their union, $f\union g$ requires
that both $\dom f \intersection \dom g = \nothing$ and $\rng f \intersection \rng g =
\nothing$. This is an example of a disjoint join as introduced in Section~\ref{sec:disjoint_joins}.

In Di Pierro et.al. \cite{DiPierro200625}, the authors consider groupoids as their mathematical
model for reversible computations. This leads them to develop \textbf{rCL}, reversible combinatory
logic. This logic consists of a pair of terms, $\<M|H\>$ where $M$ is a term of standard combinatory
logic and $H$ is a history, with a specified syntax. In standard combinatory logic, the $k$ term is
irreversible in that it erases its second argument. In \textbf{rCL}, application of the $k$ term
copies the second argument into the history, preserving reversibility. Groupoids are a specific
example of inverse categories, which are total.

A recent reversible language is Theseus, \cite{james2014theseus}, by Sabry and James. Theseus is a
functional language which compiles to a graphical language
\cite{james2013isomorphic,james2012information} for reversible computation, based on
isomorphisms of finite sums and products of types. Their chosen isomorphisms include commutativity
and associativity for sums and products, units for product and distributivity of product over
sums. The basic graphical language is extended with recursive types and looping operators and
therefore introduces partiality due to the possibility of non-terminating loops. Their abstract
model for this language is a dagger symmetric traced bimonoidal category\cite{selinger05:dagger}.

Additionally, we note there are a number of quantum programming languages which, as noted, included
reversible operations. Our primary example is LQPL \cite{giles2007}, a compiled language based on the
semantics of \cite{selinger04:qpl}. The language includes a variety of reversible operations
(unitary transforms) as primitives, a linear type system and an operational semantics. More recently
Quipper \cite{green2013introduction,green2013quipper}, which focuses on methods to handle very large
circuits, is a language embedded in Haskell \cite{peyton2003:haskell98}. Quipper uses quantum
\emph{and} classical circuits as an underlying model. An interesting aspect of reversibility in
Quipper is the inclusion of an operator to compute the reverse of a given circuit.

In much of the research on reversibility, specific conditions are placed on some aspect of the
computational model or reversible language to ensure ``programs'' in this model are reversible.
The variety of models and languages obscures the fundamental commonality of reversibility.
By basing our theory on inverse categories, our treatment clarifies the relationship
between these various approaches.


\section{Objectives}
\label{sec:objectives}

This thesis proposes a categorical semantics for reversible computing. Based upon our review of
current research as noted in Section~\ref{sec:background}, reversibility still lacks a unifying
semantic model. Standard computability has Cartesian closed categories \cite{barr:ctcs} and Turing
categories \cite{cockett-hostra08-intro-to-turing}, while quantum computing has had much success
with dagger compact closed categories
\cite{selinger04:towardssemantics,selinger05:dagger,abramsky05:abstractscalars}.

We present inverse categories as an abstract semantics for partial reversible computation.
Inverse categories admit product-like and coproduct-like structures, respectively called
inverse product and disjoint sum. Inverse categories with an inverse product are called discrete
inverse categories, due to their possessing an inverse for comultiplication. Similarly, when
$\Delta$ in a Cartesian restriction category has an inverse, it will be called a discrete Cartesian
restriction category. Section~\ref{sec:the_restriction_category_hypxt} shows how the ``Cartesian
Completion'' of an inverse category can be constructed. This enables us to create a discrete
Cartesian restriction category from a discrete inverse category. It is then shown that we have an
equivalence between the category of discrete inverse categories and the category of discrete
Cartesian restriction categories.

The next step is to show how to add a disjoint sum to an inverse category and how the
Cartesian Completion results in a distributive restriction category when one starts with a
distributive inverse category.

An example of a discrete inverse category with disjoint sums is provided by the commutative
Frobenius algebras in any additive symmetric monoidal category. As Frobenius algebras are related to
bases in finite dimensional Hilbert spaces\cite{coeckeetal08:ortho}, this provides a connection
between inverse categories and quantum computing.

Finally, we develop the structure of inverse Turing categories and inverse partial combinatory
algebras, directly based on Turing category and partial combinatory algebras  from
\cite{cockett-hostra08-intro-to-turing,cockett2010:categories-and-computability}, using the main
result of this thesis. This places the connection between reversible and irreversible
computing on a more abstract footing.


\section{Outline} % (fold)
\label{sec:outline}

We assume a knowledge of basic algebra including definitions and properties of groups, rings,
fields, vector spaces and matrices. The reader may consult \cite{lang:algebra} if further details
are needed.

Chapter~\ref{chap:introduction_to_categories} introduces the various categorical concepts that will
be used throughout this thesis.

Chapter~\ref{chap:restriction_categories} describes restriction categories, an algebraic
formulation of partiality in categories. We discuss joins, meets and ranges in
restriction categories and their relation to partial map categories. We describe products in
restriction categories, and define discrete Cartesian restriction categories, which will be
important to the thesis. Various examples of restriction categories are given.

Chapter~\ref{cha:inverse_categories} introduces inverse categories and provides examples of them. We
show that inverse categories with a restriction product collapse to a restriction preorder, that
is, a restriction category in which all parallel maps agree wherever they are both defined. Then,
Section~\ref{sec:inverse_products} introduces the concept of inverse products and explores the
properties of the inverse product. Inverse categories with inverse products are called
\emph{discrete inverse categories}.

Chapter~\ref{chap:completing_a_discrete_inverse_category} then presents the ``Cartesian
Completion'' --- a construction of a discrete Cartesian restriction category from a discrete
inverse category. Section~\ref{sec:equivalence_classes_of_maps_in_hypx} presents the details of the
equivalence relation on  maps of a discrete inverse category needed in the construction, while
Section~\ref{sec:the_restriction_category_hypxt} and
Section~\ref{sec:the_category_hypxt_is_cartesian} prove the construction gives a Cartesian
restriction category. Section~\ref{sec:equivalence_of_categories} culminates in
Theorem~\ref{thm:discrete_inverse_categories_are_equivalent_to_discrete_restriction_categories}
giving an equivalence between the category of discrete inverse categories and the
category of discrete Cartesian restriction categories. We note here that this is not a 2-equivalence
of categories. We provide some simple examples of the construction.


Chapter~\ref{cha:disjointness_in_inverse_categories} begins the exploration of how to add a
coproduct-like construction to inverse categories. Paralleling the previous chapter, we show the
existence of a restriction coproduct implies that an inverse category must be a preorder, i.e.,
that all parallel maps are equal. Section~\ref{sec:disjointness_in_an_inverse_category} defines a
disjointness relation in an inverse category. We show that disjointness may be defined on all maps
or equivalently only on the restriction idempotents of the inverse category. This allows us to
define the disjoint join in Section~\ref{sec:disjoint_joins}.

Chapter~\ref{chap:disjoint_sum_tensors} introduces the disjoint sum, an object in an inverse
category with a disjoint join, which behaves like a coproduct. The disjoint sum has injection maps
which are subject to certain conditions. When an inverse category has all disjoint sums, it is
possible to define a symmetric monoidal tensor based on the disjoint sum. The remainder of the
chapter explores what constraints on a tensor will allow the creation of  a disjoint sum. We define
a disjoint sum tensor, a symmetric monoidal tensor in the inverse category with specific additional
constraints. A disjoint sum tensor allows us to define both a disjointness relation and a disjoint
join based on the tensor. Disjoint sum tensors do produce disjoint sums and conversely, the tensor
defined by disjoint sums is a disjoint sum tensor.

Chapter~\ref{cha:matrix_categories} introduces a matrix construction on inverse categories with
disjoint joins in order to add disjoint sums. We show that an inverse category \X with a disjoint join
is equivalent to its matrix category, i.e., $\X \cong \imat{\X}$. From this,
we produce an reflection between the category of inverse categories with disjoint joins and the
category of inverse categories with disjoint sums.

In Chapter~\ref{chap:distributive_inverse_categories}  a distributive inverse category is defined as
an inverse category where the inverse product distributes over the disjoint sum. The Cartesian
Completion of a distributive inverse category turns the disjoint sum into a coproduct and, in fact,
will create a distributive restriction category.

Chapter~\ref{chap:commutative_frobenius_algebras} discusses commutative Frobenius algebras. The
chapter starts with providing a background on dagger categories and frobenius algebras, showing how
the latter are equivalent to bases in a finite dimensional Hilbert space. The category \CFrob,
the category of commutative Frobenius algebras in a symmetric monoidal category $\X$, is
introduced. \CFrob is shown to be a discrete inverse category. Furthermore, when \X is an
additive tensor category with zero maps, \CFrob has disjoint sums.

In Chapter~\ref{chap:turing_categories}, Turing categories and partial combinatory algebras are
introduced as a way to formulate computability. The corresponding structures in inverse categories,
inverse Turing categories and inverse partial combinatory algebras, are then investigated. We show the
equivalence of these structures to the ones in discrete Cartesian restriction categories.

Chapter~\ref{cha:conclusions_and_future_work} starts with a summary of the contributions of this
thesis and concludes with a short section on potential areas of further exploration.



% section algebraic_setting (end)
%%% Local Variables:
%%% mode: latex
%%% TeX-master: "../phd-thesis"
%%% End:
