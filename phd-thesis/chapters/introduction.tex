%!TEX root = /Users/gilesb/UofC/thesis/phd-thesis/phd-thesis.tex
\chapter{Introduction}
\section{Background}
\label{sec:background}

\section{Objectives}
\label{sec:objectives}

\section{Contributions}
\label{sec:contributions}

The contribution of this thesis is the characterization of products and coproducts in inverse
categories. We show in Proposition~\ref{prop:an_inverse_category_with_products_is_a_restriction_preorder}
and Proposition~\ref{prop:inverse_category_with_coproducts_is_pre-order} that each of products and
coproducts impose a trivialization of the structure of the base category.

\section{Outline} % (fold)
\label{sec:outline}

We assume a knowledge of basic algebra including definitions and properties of groups, rings,
fields, vector spaces and matrices. The reader may consult \cite{lang:algebra} if further details
are needed.

Chapter~\ref{chap:abstract_computability} introduces the various concepts that will be used
throughout this thesis. The basics of category theory are reviewed in Section~\ref{sec:categories}
followed by an introduction to restriction categories in
Section~\ref{sec:restriction_categories}. Section~\ref{sec:daggercategories} and
Section~\ref{sec:frobenius_algebras}  introduce dagger categories and Frobenius
algebras respectively. These constructs are of particular interest in the study of quantum semantics
and will provide a source of examples for us.

Chapter~\ref{cha:inverse_categories} starts with showing that inverse categories with a restriction
product collapse into a restriction pre-order, that is, all parallel maps agree wherever they are
both defined. Section~\ref{sec:inverse_products} introduces the concept of inverse products and
explores the properties of inverse categories with these constructions. Inverse categories with
inverse products are labelled discrete inverse categories. A major example of discrete inverse
categories, commutative Frobenius algebras is then introduced in
Section~\ref{sec:the_category_of_commutative_frobenius_algebras}.
Section~\ref{sec:completing_a_discrete_inverse_category} then explores how to construct a Cartesian
restriction category based on a discrete inverse category, culminating in
Theorem~\ref{thm:discrete_inverse_categories_are_equivalent_to_discrete_restriction_categories}
giving an equivalence adjunction between discrete inverse categories and discrete Cartesian
restriction categories, which were introduced in SubSection~\ref{sub:discrete_restriction_categories}.

Having explored how to introduce a product-like construction in an inverse category,
Chapter~\ref{cha:disjointness_in_inverse_categories} begins the exploration of how to add a
coproduct-like construction. Paralleling the previous chapter, in
Section~\ref{sec:coproducts_in_restriction_categories} we show that restriction coproducts
impose an even greater amount of structure on an inverse category. The existence of a restriction
coproduct implies that an inverse category must be a pre-order, i.e., that all parallel maps are
equal. Section~\ref{sec:disjointness_in_an_inverse_category} starts with a quick review of joins in
restriction categories, followed by defining a disjointness relation in an inverse category. We show
that disjointness may be defined on all maps or equivalently only on the restriction idempotents of
the inverse category. This allows us to now define the disjoint join in
Section~\ref{sec:disjoint_joins}. We then conclude this chapter with
Section~\ref{sec:tensors_for_disjointness} showing how specific conditions on a symmetric monoidal
tensor in an inverse category allow us to define both a disjointness relation and a disjoint join
from that tensor.

Chapter~\ref{cha:inverse_sum_categories} builds upon the previous chapters to introduce
coproduct-like constructions into the inverse
category. Section~\ref{sec:disjointness_in_frobenius_algebras} starts by expanding the example of
commutative Frobenius algebras, showing there is a disjointness relation in them. In
Section~\ref{sec:inverse_sum_categories} we explore the relationship between disjoint joins and
inverse sums. This culminates in showing that an inverse category \X with a tensor generating disjoint
joins gives rise to a matrix category over \X and that \X is equivalent to its matrix
category. Finally, in Section~\ref{sec:completing_a_distributive_inverse_category} we show that the
construction introduced in Section~\ref{sec:completing_a_discrete_inverse_category} also lifts an
inverse sum up to a coproduct and in fact will create a distributive restriction category when the
inverse product distributes over the disjoint join.

Our conclusions and thoughts for potentially interesting areas to explore further are then given in
Chapter~\ref{cha:conclusions_and_future_work}.



% section algebraic_setting (end)
%%% Local Variables:
%%% mode: latex
%%% TeX-master: "../phd-thesis"
%%% End:
