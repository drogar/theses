%!TEX root = /Users/gilesb/UofC/thesis/phd-thesis/phd-thesis.tex
\chapter{Inverse categories} % (fold)
\label{cha:inverse_categories}

This chapter will focus on adding ``products'' to an inverse category.

We will show below an inverse category which has a
restriction product is a restriction pre-order. Given this, by ``product'', we mean a construction
that behaves in a product-like manner in an inverse category. We call these \emph{inverse products}, which
will be defined below in Sub-Section~\ref{sub:inverse_product_tensor}.
Inverse products are given by a tensor product which supports a diagonal but lacks projections. The
diagonal map is required to give a natural Frobenius structure to each object.

\section{Inverse categories with restriction products} % (fold)
\label{sec:inverse_categories_with_restriction_products}
We start by showing that an inverse category with restriction products is a restriction pre-order.

\begin{definition}\label{def:compatible_maps}
  Two parallel maps $f,g:A \to B$ in a restriction category are \emph{compatible}, written as $f
  \compatible g$, when $\rst{f} g = \rst{g} f$.
\end{definition}
\begin{definition}\label{def:restrictionpreorder}
  A restriction category \X is a \emph{restriction pre-order} when all parallel pairs of maps are
  compatible.
\end{definition}
\begin{proposition}\label{prop:an_inverse_category_with_products_is_a_restriction_preorder}
  Given an inverse category \X, if it has restriction products, it is a restriction pre-order. That
  is,
  \[
    \xymatrix {
      A  \ar@<1ex>[r]^{f} \ar@<-1ex>[r]_{g} &B
    }
    \implies f \compatible g.
  \]
\end{proposition}
\begin{proof}
  Notice,
  \begin{align*}
    \inv{\pi_1} & = \Delta \pi_1 \inv{\pi_1}\\
    &=\Delta \restr{\pi_1}\\
    &=\Delta.
  \end{align*}
  This gives $\restr{\inv{\pi_1}} = 1$ and therefore $\pi_1$ (and similarly, $\pi_0$) is an
  isomorphism.

  Starting with the product map $\<f,g\>$,
  \[
    \infer={\restr{f}g = \restr{g}f}
    {\infer={\restr{f}g\Delta = \restr{g}f\Delta}
    {\infer={\restr{f}g\inv{\pi_1} = \restr{g}f\inv{\pi_0}}
    {\infer={\<f,g\>\pi_1 \inv{\pi_1} = \<f,g\>\pi_0 \inv{\pi_0}}
    {\<f,g\> = \<f,g\>}}}}
  \]
  which shows that $f$ and $g$ are compatible.
\end{proof}

\begin{corollary}
  \X\ is a Cartesian inverse category if and only if Total($\spl{r}{\X}$) is a meet pre-order.
\end{corollary}

\begin{proof}
  Total(\X), the subcategory of total maps on \X, has products and therefore every pair of parallel
  maps is compatible. As total compatible maps are equal, there is at most
  one map between any two objects. Hence, Total(\X) is a pre-order with the meet being the product.

  Similarly, from \cite{cockett2002:restcategories1} and \cite{cockettlack2004:restcategories3},
  Total($\spl{r}{\X}$) is an inverse category and has products and is therefore also a meet
  pre-order. This shows the ``only if'' side of the corollary.

  For the other direction, if Total($\spl{r}{\X}$) is a meet pre-order, define the product as the
  meet of the maps and the terminal object as the supremum of all maps.
\end{proof}

\begin{corollary}
  Every Cartesian inverse category is a full subcategory of a partial map category of a meet
  semi-lattice.
\end{corollary}


% section inverse_categories_with_restriction_products (end)
\section{Inverse products} % (fold)
\label{sec:inverse_products}

\subsection{Inverse product tensor} % (fold)
\label{sub:inverse_product_tensor}

\begin{definition}\label{def:inverse_product_tensor}
  Given a restriction category \X, a tensor $\*$ is called an \emph{inverse product tensor}
  when:
\begin{itemize}
  \item $\*$ is a restriction functor, $\_ \* \_ : \X \times \X \to \X$.
  \item $\*$ is a symmetric monoidal tensor satisfying the standard symmetric monoidal equations and
    coherence diagrams hold (see, e.g., \cite{maclan97:categorieswrkmath}) and has the following
    natural isomorphisms:
    \begin{align*}
      1 &: \boldsymbol{1}\to \X \\
      \usl &: 1 \* A \xrightarrow{\cong} A
      &\usr &: A \* 1 \xrightarrow{\cong} A\\
      a_{\*} &: (A \* B) \* C \xrightarrow{\cong} A \* (B \* C)
      &c_{\*} &: A \* B \xrightarrow{\cong} B \* A.
    \end{align*}
  \end{itemize}
  Note that since all the coherence maps are isomorphisms, they are total.
\end{definition}

\begin{definition}\label{def:inverse_product}
  An \emph{inverse product} on an inverse category \X is given by an inverse product tensor $\*$
  together with a  natural ``Frobenius'' diagonal map, $\Delta$. Additionally, we define the
  map $\excs: (A\*B)\*(C\*D) \to (A\*C) \* (B\*D)$:
  \[
    \excs =  a_{\*}(1\*\inv{a_{\*}})(1\*(c_{\*}\*1))(1\* a_{\*})\inv{a_{\*}}).
  \]

  The diagonal map $\Delta_A:A \to A\*A$ must be total and satisfy the following:
  \[
    \xymatrix @!0 @C=90pt @R=35pt{
      A \ar[dr]_{\Delta} \ar[r]^{\Delta} &
      A \* A \ar[d]^{c_{\*}}\\
      & A \* A\\
      &*!<3pc,-15pt>{\text{\textbf{Co-commutative}}}
    }
  \]
  \[
    \xymatrix @C=30pt @R=30pt{
      A \ar[rr]^{\Delta} \ar[d]_{\Delta} & &
      A \* A \ar[d]^{1\*\Delta}\\
      A\*A \ar[dr]_{\Delta \* 1}& & A \* ( A \* A) \\
      &   (A \* A) \* A \ar[ur]_{a_{\*}}\\
      &*!<0pc,-35pt>{\text{\textbf{Co-associative}}}
    }
  \]
  \[
    \xymatrix @C=40pt @R=35pt{
      A \* B \ar[d]_{\Delta}
      \ar[rr]^{\Delta \* \Delta} & &
      (A \* A) \* (B \* B) \ar[d]^{\excs}\\
      (A \* B) \* (A \* B) \ar@{=}[rr] & &
      (A \* B) \* (A \* B)\\
      &*!<0pc,-25pt>{\text{\textbf{Exchange}}}
    }
  \]

  \[
    \xymatrix @C=40pt @R=25pt{
      A \* A \ar[dd]_{(1\*\Delta) \inv{a_{\*}}} \ar[dr]^{\inv{\Delta}}
      \ar[rr]^{(\Delta \* 1) a_{\*}} & &
      A \* (A \* A) \ar[dd]^{1 \* \inv{\Delta}}\\
      & A \ar[dr]^{\Delta} & \\
      (A \* A) \* A \ar[rr]_{\inv{\Delta} \* 1} & &
      A \* A\\
      &*!<0pc,-25pt>{\text{\textbf{Frobenius}}}
    }
  \]
  Thus, $\Delta$ is a co-commutative, co-associative map which together with $\inv{\Delta}$ forms a
  Frobenius algebra.
\end{definition}

\begin{remark}
  Note also, co-commutativity implies that $c_{\*}\inv{\Delta} = \inv{\Delta}$.
  One can see this as:
  \begin{align*}
    \Delta(c_{\*}\inv{\Delta})
      &= (\Delta c_{\*})\inv{\Delta} = \Delta\inv{\Delta} = \rst{\Delta} \text{ and}\\
    (c_{\*}\inv{\Delta})\Delta
      & = (c_{\*}\inv{\Delta})(\Delta c_{\*}) = \rst{c_{\*}\inv{\Delta}}.
  \end{align*}
  This means that both $\inv{\Delta}$ and $c_{\*}\inv{\Delta}$ are partial inverses for $\Delta$
  and are therefore equal.

  Similarly, one can show that $(\inv{\Delta}\* 1)\inv{\Delta} =
  a_{\*}(\inv{\Delta}\* 1)\inv{\Delta}$.
\end{remark}


Inverse products are extra structure on an inverse category, rather than a property. A concrete
category showing this is given in the following example.

\begin{example}[Showing that inverse product is additional structure]
  \label{example:invprodisstructure}
\end{example}
Any discrete category (i.e., a category with only the identity arrows) is a trivial inverse
category. To create an inverse product on a discrete category, add a commutative, associative,
idempotent multiplication, with a unit.

Let $\D$ be the discrete category of four objects and label them as $a,b,c$ and $d$. Then, define
two different inverse product tensors, $\*$ and $\odot$ as shown in
Table~\ref{tab:two_different_inverse_products}.

\begin{table}[h!]
  \begin{center}
  \begin{tabular}{|l||c|c|c|c|}
    \hline
    $\*$&a&b&c&d\\ \hline \hline
    a&a&a&a&a\\ \hline
    b&a&b&\textbf{b}&b\\ \hline
    c&a&\textbf{b}&c&c \\ \hline
    d&a&b&c&d \\ \hline
  \end{tabular}
  \qquad
  \begin{tabular}{|l||c|c|c|c|} \hline
    $\odot$&a&b&c&d\\ \hline \hline
    a&a&a&a&a\\ \hline
    b&a&b&\textbf{a}&b\\ \hline
    c&a&\textbf{a}&c&c \\ \hline
    d&a&b&c&d \\ \hline
  \end{tabular}
  \end{center}
  \caption{Two different inverse products on the same category.}
  \label{tab:two_different_inverse_products}
\end{table}

As $\D$ is discrete, $\Delta$ is forced to be the identity. By inspection, we can see each of
the conditions for $\Delta$ are satisfied by $\*$ and by $\odot$.

% subsection inverse_products (end)
\subsection{Diagrammatic Language} % (fold)
\label{sub:diagrammatic_language}

While it is certainly possible to prove results about inverse products using direct algebraic
manipulation, it is much more succinct to proceed with string diagrams. See
\cite{selinger11:graphical} for a comparison of various graphical languages for monoidal categories.
As shown in
\cite{street-ross-1991-GTC-I}, diagrammatic reasoning is equivalent to reasoning algebraically for
symmetric monoidal categories.

In the diagrams, we will use the following representations:
\begin{itemize}
  \item $\Delta$ will be represented by an upward pointing triangle: \begin{tikzpicture}
      \path node [delta] at (0,0) {}; \end{tikzpicture}.
  \item $\inv{\Delta}$ by a downward triangle: \begin{tikzpicture}
      \path node [nabla] at (0,0) {}; \end{tikzpicture}.
  \item maps by a rectangle with the map inside: \begin{tikzpicture}
      \path node [map] at (0,0) {$\scriptstyle f$}; \end{tikzpicture}.
  \item unit introduction (often referred to as an $\eta$ map): \begin{tikzpicture}
      \path node [eta] at (0,0) {}; \end{tikzpicture}.
  \item unit removal (often referred to as an $\epsilon$ map): \begin{tikzpicture}
      \path node [epsilon] at (0,0) {}; \end{tikzpicture}.
\end{itemize}
String diagrams in this thesis are to be read from top to bottom.

The axioms of Definition~\ref{def:inverse_product} then become:\\
\begin{tikzpicture}
  \begin{pgfonlayer}{nodelayer}
    \node [style=nothing] (0) at (-3.75, 5.5) {};
    \node [style=delta] (1) at (-3.75, 5) {};
    \node [style=nothing] (2) at (-4, 4.25) {};
    \node [style=nothing] (3) at (-3.5, 4.25) {};
    \node [style=nothing] (4) at (-3, 5) {$=$};
    \node [style=nothing] (5) at (-2.25, 5.5) {};
    \node [style=delta] (6) at (-2.25, 5) {};
    \node [style=nothing] (7) at (-2.5, 4.25) {};
    \node [style=nothing] (8) at (-2, 4.25) {};
    \node [style=nothing] (9) at (-3, 4) {Co-commutativity};
    \end{pgfonlayer}
    \begin{pgfonlayer}{edgelayer}
      \draw [] (0) to (1);
      \draw [] (1) to[out=305,in=90] (2);
      \draw [] (1) to[out=235,in=90] (3);
      \draw [] (5) to (6);
      \draw [] (6) to[out=235,in=90] (7);
      \draw [] (6) to[out=305,in=90] (8);
      \end{pgfonlayer}
\end{tikzpicture}
\hspace{15pt}
\begin{tikzpicture}
  \begin{pgfonlayer}{nodelayer}
    \node [style=nothing] (0) at (-3.75, 5.5) {};
    \node [style=delta] (1) at (-3.75, 5) {};
    \node [style=delta] (2) at (-4, 4.5) {};
    \node [style=nothing] (3a) at (-4.25, 4) {};
    \node [style=nothing] (3b) at (-3.75, 4) {};
    \node [style=nothing] (3c) at (-3.5, 4) {};
    \node [style=nothing] (4) at (-3, 5) {$=$};
    \node [style=nothing] (5) at (-2.25, 5.5) {};
    \node [style=delta] (6) at (-2.25, 5) {};
    \node [style=delta] (8) at (-2, 4.5) {};
    \node [style=nothing] (7a) at (-2.5, 4) {};
    \node [style=nothing] (7b) at (-2.25, 4) {};
    \node [style=nothing] (7c) at (-1.75, 4) {};
    \node [style=nothing] (9) at (-3, 3.5) {Co-associativity};
    \end{pgfonlayer}
    \begin{pgfonlayer}{edgelayer}
      \draw [] (0) to (1);
      \draw [] (1) to[out=305,in=90] (3c);
      \draw [] (1) to[out=235,in=90] (2);
      \draw [] (2) to[out=305,in=90] (3b);
      \draw [] (2) to[out=235,in=90] (3a);
      \draw [] (5) to (6);
      \draw [] (6) to[out=235,in=90] (7a);
      \draw [] (6) to[out=305,in=90] (8);
      \draw [] (8) to[out=305,in=90] (7c);
      \draw [] (8) to[out=235,in=90] (7b);
      \end{pgfonlayer}
\end{tikzpicture}
\hspace{15pt}
\begin{tikzpicture}
  \begin{pgfonlayer}{nodelayer}
    \node [style=nothing] (0) at (-2.55, 3.75) {};
    \node [style=nothing] (1) at (-2.45, 3.75) {};
    \node [style=delta] (2) at (-2.5, 3.25) {};
    \node [style=nothing] (3) at (-3, 2.5) {};
    \node [style=nothing] (4) at (-2.75, 2.5) {};
    \node [style=nothing] (5) at (-2.25, 2.5) {};
    \node [style=nothing] (6) at (-2, 2.5) {};
    \node [style=nothing] (7) at (-1.75, 3.25) {$=$};
    \node [style=nothing] (8) at (-.675, 3.75) {};
    \node [style=nothing] (9) at (-0.575, 3.75) {};
    \node [style=delta] (10) at (-1, 3.25) {};
    \node [style=delta] (11) at (-0.25, 3.25) {};
    \node [style=nothing] (12) at (-1.25, 2.5) {};
    \node [style=nothing] (13) at (-1, 2.5) {};
    \node [style=nothing] (14) at (-0.25, 2.5) {};
    \node [style=nothing] (15) at (0, 2.5) {};
    \node [style=nothing] (16) at (-1.5, 2) {Exchange};
    \end{pgfonlayer}
    \begin{pgfonlayer}{edgelayer}
      \draw [] (0) to (2);
      \draw [] (1) to (2);
      \draw [] (2) to (3);
      \draw [] (2) to (4);
      \draw [] (2) to (5);
      \draw [] (2) to (6);
      \draw [] (8) to[out=270,in=90] (10);
      \draw [] (9) to[out=270,in=90] (11);
      \draw [] (10) to (12);
      \draw [] (10) to (14);
      \draw [] (11) to (13);
      \draw [] (11) to (15);
      \end{pgfonlayer}
\end{tikzpicture}
\hspace{15pt}
\begin{tikzpicture}
  \begin{pgfonlayer}{nodelayer}
    \node [style=nothing] (0) at (-2.75, 3.75) {};
    \node [style=nothing] (1) at (-2.25, 3.75) {};
    \node [style=delta] (2) at (-2.75, 3.25) {};
    \node [style=nothing] (3) at (-2.75, 2) {};
    \node [style=nothing] (4) at (-2.25, 2) {};
    \node [style=nothing] (5) at (-0.5, 3) {$=$};
    \node [style=nothing] (6) at (0, 3.75) {};
    \node [style=nothing] (7) at (0.5, 3.75) {};
    \node [style=delta] (8) at (-1, 2.5) {};
    \node [style=delta] (9) at (0.5, 3.25) {};
    \node [style=nothing] (10) at (0, 2) {};
    \node [style=nothing] (11) at (0.5, 2) {};
    \node [style=nabla] (12) at (-2.25, 2.5) {};
    \node [style=nabla] (13) at (-1, 3.25) {};
    \node [style=nabla] (14) at (0, 2.5) {};
    \node [style=nothing] (15) at (-1.75, 3) {$=$};
    \node [style=nothing] (16) at (-1.25, 3.75) {};
    \node [style=nothing] (17) at (-0.75, 3.75) {};
    \node [style=nothing] (18) at (-1.25, 2) {};
    \node [style=nothing] (19) at (-0.75, 2) {};
    \node [style=nothing] (20) at (-1, 1.75) {Frobenius};
    \end{pgfonlayer}
    \begin{pgfonlayer}{edgelayer}
      \draw [] (0) to (2);
      \draw [] (2) to (3);
      \draw [] (7) to (9);
      \draw [] (9) to (11);
      \draw (1) to (12);
      \draw (2) to (12);
      \draw (12) to (4);
      \draw (13) to (8);
      \draw (16) to (13);
      \draw (17) to (13);
      \draw (8) to (18);
      \draw (8) to (19);
      \draw (14) to (10);
      \draw (9) to (14);
      \draw (6) to (14);
      \end{pgfonlayer}
\end{tikzpicture}
% end sub:diagrammatic_language


\subsection{Discrete inverse categories} % (fold)
\label{sub:discrete_inverse_categories}

\begin{definition}\label{def:discrete_inverse_category}
  If an inverse category has inverse products as in Definition~\ref{def:inverse_product}, we call it
  a \emph{discrete inverse category}.
\end{definition}

In this Sub-Section we will present some properties of discrete inverse categories. This  will
form the basis for our eventual goal, that of connecting discrete inverse categories to Cartesian
restriction categories. The connection between them will be given as a functor that lifts a discrete
inverse category to a discrete Cartesian restriction category.

\begin{lemma}\label{lem:properties_of_delta_and_tensor_in_a_discrete_inverse_category}
  In a discrete inverse category \X with the inverse product $\*$ and $\Delta$, where
  $e=\rst{e}$ is a restriction idempotent and $f,g,h$ are arrows in \X, the following are true:
  \begin{enumerate}[{(}i{)}]
    \item{}$e=\Delta (e\* 1) \inv{\Delta}$.\label{le:eisde1}
    \item{}$e\Delta (f \* g) = \Delta (e f \* g) $ (and $= \Delta (f \* e g) $ and
      $ = \Delta (e f \* e g)$.)\label{le:deltaefg}
    \item{}$ (f \* g e) \inv{\Delta} =(f \* g) \inv{\Delta} e $ (and $= (f e\* g) \inv{\Delta}$ and
      $ = (f e\* g e)\inv{\Delta}$.)\label{le:efginvdelta}
    \item{}$\restr{\Delta (f \* g) \inv{\Delta}} =
       \Delta(1\* g \inv{f})\inv{\Delta}$. \label{le:restfg}
    \item{} If $\Delta (h \* g) \inv{\Delta} = \restr{\Delta (h \* g) \inv{\Delta}}$ then
      $(\Delta (h \* g) \inv{\Delta}) h = \Delta (h \* g) \inv{\Delta}$.\label{le:hge}
    \item{}$\Delta (f\*1) = \Delta (g\*1) \implies f = g$.\label{le:dfgisfg}
    \item{}$(f\*1) = (g\*1) \implies f = g$.\label{le:fgisfg}
  \end{enumerate}
\end{lemma}
\begin{proof}
  \prepprooflist
  \begin{enumerate}[{(}i{)}]
    \item[\ref{le:eisde1}]
      \[
      \raisebox{30pt}{
      \begin{tikzpicture}
        \node at (0,1) (start) {};
        \node at (0,.5) [map] (e) {$\scriptstyle e$};
        \node at (0,0) (end) {};
        \draw (start) to (e);
        \draw (e) to (end);
      \end{tikzpicture}
      }
      \ \raisebox{45pt}{$=$}\
      \begin{tikzpicture}
        \node at (0,3) (start) {};
        \node at (0,2.5) [delta] (d1) {};
        \node at (0,2) [nabla] (n1) {};
        \node at (0,1.5) [delta] (d2) {};
        \node at (0,1) [nabla] (n2) {};
        \node at (0,.5) [map] (e) {$\scriptstyle e$};
        \node at (0,0) (end) {};
        \draw [] (start) to (d1);
        \draw [] (d1) to[out=235,in=125] (n1);
        \draw [] (d1) to[out=305,in=55] (n1);
        \draw [] (n1) to (d2);
        \draw [] (d2) to[out=235,in=125] (n2);
        \draw [] (d2) to[out=305,in=55] (n2);
        \draw [] (n2) to (e);
        \draw (e) to (end);
      \end{tikzpicture}
      \ \raisebox{45pt}{$=$}\
      \begin{tikzpicture}
        \node at (0,3) (start) {};
        \node at (0,2.5) [delta] (d1) {};
        \node at (-.25,2) [delta] (d2) {};
        \node at (.25,1.5) [nabla] (n1) {};
        \node at (0,1) [nabla] (n2) {};
        \node at (0,.5) [map] (e) {$\scriptstyle e$};
        \node at (0,0) (end) {};
        \draw [] (start) to (d1);
        \draw [] (d1) to[out=235,in=90] (d2);
        \draw [] (d1) to[out=305,in=55] (n1);
        \draw [] (d2) to[out=305,in=125] (n1);
        \draw [] (d2) to[out=235,in=125] (n2);
        \draw [] (n1) to[out=270,in=55] (n2);
        \draw [] (n2) to (e);
        \draw (e) to (end);
      \end{tikzpicture}
      \ \raisebox{45pt}{$=$}\
      \begin{tikzpicture}
        \node at (0,3) (start) {};
        \node at (0,2.5) [delta] (d1) {};
        \node at (-.25,2) [delta] (d2) {};
        \node at (-.5,1.5) [map] (e1) {$\scriptstyle e$};
        \node at (0,1.5) [map] (e2) {$\scriptstyle e$};
        \node at (.5,1.5) [map] (e3) {$\scriptstyle e$};
        \node at (.25,1) [nabla] (n1) {};
        \node at (0,.5) [nabla] (n2) {};
        \node at (0,0) (end) {};
        \draw [] (start) to (d1);
        \draw [] (d1) to[out=235,in=90] (d2);
        \draw [] (d1) to[out=305,in=90] (e3);
        \draw [] (d2) to[out=305,in=90] (e2);
        \draw [] (d2) to[out=235,in=90] (e1);
        \draw (e1) to[out=270,in=125] (n2);
        \draw (e2) to[out=270,in=125] (n1);
        \draw (e3) to[out=270,in=55] (n1);
        \draw [] (n1) to[out=270,in=55] (n2);
        \draw [] (n2) to (end);
      \end{tikzpicture}
      \ \raisebox{45pt}{$=$}\
      \begin{tikzpicture}
        \node at (0,3) (start) {};
        \node at (0,2.5) [delta] (d1) {};
        \node at (-.25,2) [map] (e1) {$\scriptstyle e$};
        \node at (-.25,1.5) [delta] (d2) {};
        \node at (.25,1) [nabla] (n1) {};
        \node at (.25,.5) [map] (e2) {$\scriptstyle e$};
        \node at (0,0) [nabla] (n2) {};
        \node at (0,-.5) (end) {};
        \draw [] (start) to (d1);
        \draw [] (d1) to[out=235,in=90] (e1);
        \draw (e1) to (d2);
        \draw [] (d1) to[out=305,in=55] (n1);
        \draw [] (d2) to[out=305,in=125] (n1);
        \draw [] (d2) to[out=235,in=125] (n2);
        \draw (n1) to (e2);
        \draw [] (e2) to[out=270,in=55] (n2);
        \draw [] (n2) to (end);
      \end{tikzpicture}
      \ \raisebox{45pt}{$=$}\
      \begin{tikzpicture}
        \node at (0,3) (start) {};
        \node at (0,2.5) [delta] (d1) {};
        \node at (-.25,2) [map] (e1) {$\scriptstyle e$};
        \node at (0,1.5) [nabla] (n1) {};
        \node at (0,1) [delta] (d2) {};
        \node at (.25,.5) [map] (e2) {$\scriptstyle e$};
        \node at (0,0) [nabla] (n2) {};
        \node at (0,-.5) (end) {};
        \draw [] (start) to (d1);
        \draw [] (d1) to[out=235,in=90] (e1);
        \draw [] (d1) to[out=305,in=55] (n1);
        \draw (e1) to[out=270,in=125] (n1);
        \draw [] (n1) to (d2);
        \draw [] (d2) to[out=305,in=90] (e2);
        \draw [] (d2) to[out=235,in=125] (n2);
        \draw (e2) to[out=270,in=55] (n2);
        \draw [] (n2) to (end);
      \end{tikzpicture}
      \ \raisebox{45pt}{$=$}\
      \begin{tikzpicture}
        \node at (0,3) (start) {};
        \node at (0,2.5) [delta] (d1) {};
        \node at (-.25,2) [map] (e1) {$\scriptstyle e$};
        \node at (0,1.5) [nabla] (n1) {};
        \node at (0,1) [delta] (d2) {};
        \node at (-.25,.5) [map] (e2) {$\scriptstyle e$};
        \node at (0,0) [nabla] (n2) {};
        \node at (0,-.5) (end) {};
        \draw [] (start) to (d1);
        \draw [] (d1) to[out=235,in=90] (e1);
        \draw [] (d1) to[out=305,in=55] (n1);
        \draw (e1) to[out=270,in=125] (n1);
        \draw [] (n1) to (d2);
        \draw [] (d2) to[out=235,in=90] (e2);
        \draw [] (d2) to[out=305,in=55] (n2);
        \draw (e2) to[out=270,in=125] (n2);
        \draw [] (n2) to (end);
      \end{tikzpicture}
      \ \raisebox{45pt}{$=$}\
      \]
\[
      \begin{tikzpicture}
        \node at (0,3) (start) {};
        \node at (0,2.5) [delta] (d1) {};
        \node at (-.25,2) [map] (e1) {$\scriptstyle e$};
        \node at (-.25,1.5) [delta] (d2) {};
        \node at (.25,1) [nabla] (n1) {};
        \node at (-.25,.5) [map] (e2) {$\scriptstyle e$};
        \node at (0,0) [nabla] (n2) {};
        \node at (0,-.5) (end) {};
        \draw [] (start) to (d1);
        \draw [] (d1) to[out=235,in=90] (e1);
        \draw (e1) to (d2);
        \draw [] (d1) to[out=305,in=55] (n1);
        \draw [] (d2) to[out=305,in=125] (n1);
        \draw [] (d2) to[out=235,in=90] (e2);
        \draw (n1) to[out=270,in=55] (n2);
        \draw [] (e2) to[out=270,in=125] (n2);
        \draw [] (n2) to (end);
      \end{tikzpicture}
      \ \raisebox{45pt}{$=$}\
      \begin{tikzpicture}
        \node at (0,2.5) (start) {};
        \node at (0,2) [delta] (d1) {};
        \node at (-.25,1.5) [delta] (d2) {};
        \node at (-.5,1) [map] (e2) {$\scriptstyle e$};
        \node at (0,1) [map] (e1) {$\scriptstyle e$};
        \node at (.25,.5) [nabla] (n1) {};
        \node at (-.25,0) [map] (e3) {$\scriptstyle e$};
        \node at (0,-0.5) [nabla] (n2) {};
        \node at (0,-1) (end) {};
        \draw [] (start) to (d1);
        \draw [] (d1) to[out=235,in=90] (d2);
        \draw [] (d1) to[out=305,in=55] (n1);
        \draw [] (d2) to[out=305,in=90] (e1);
        \draw [] (d2) to[out=235,in=90] (e2);
        \draw (e1) to[out=270,in=125] (n1);
        \draw (e2) to (e3);
        \draw (n1) to[out=270,in=55] (n2);
        \draw [] (e3) to[out=270,in=125] (n2);
        \draw [] (n2) to (end);
      \end{tikzpicture}
      \ \raisebox{45pt}{$=$}
      \begin{tikzpicture}
        \node at (0,3) (start) {};
        \node at (0,2.5) [delta] (d1) {};
        \node at (-.25,2) [map] (e1) {$\scriptstyle e$};
        \node at (-.25,1.5) [delta] (d2) {};
        \node at (.25,1) [nabla] (n1) {};
        \node at (0,.5) [nabla] (n2) {};
        \node at (0,0) (end) {};
        \draw [] (start) to (d1);
        \draw [] (d1) to[out=235,in=90] (e1);
        \draw [] (d1) to[out=305,in=55] (n1);
        \draw (e1) to[out=270,in=90] (d2);
        \draw [] (d2) to[out=235,in=125] (n2);
        \draw [] (d2) to[out=305,in=125] (n1);
        \draw (n1) to[out=270,in=55] (n2);
        \draw [] (n2) to (end);
      \end{tikzpicture}
      \ \raisebox{45pt}{$=$}
      \begin{tikzpicture}
        \node at (0,3) (start) {};
        \node at (0,2.5) [delta] (d1) {};
        \node at (-.25,2) [map] (e1) {$\scriptstyle e$};
        \node at (0,1.5) [nabla] (n1) {};
        \node at (0,1) [delta] (d2) {};
        \node at (0,.5) [nabla] (n2) {};
        \node at (0,0) (end) {};
        \draw [] (start) to (d1);
        \draw [] (d1) to[out=235,in=90] (e1);
        \draw [] (d1) to[out=305,in=55] (n1);
        \draw (e1) to[out=270,in=125] (n1);
        \draw [] (n1) to (d2);
        \draw [] (d2) to[out=235,in=125] (n2);
        \draw [] (d2) to[out=305,in=55] (n2);
        \draw [] (n2) to (end);
      \end{tikzpicture}
      \ \raisebox{45pt}{$=$}\
      \raisebox{15pt}{
        \begin{tikzpicture}
        \node at (0,3) (start) {};
        \node at (0,2.5) [delta] (d1) {};
        \node at (-.25,2) [map] (e1) {$\scriptstyle e$};
        \node at (0,1.5) [nabla] (n1) {};
        \node at (0,1) (end) {};
        \draw [] (start) to (d1);
        \draw [] (d1) to[out=235,in=90] (e1);
        \draw [] (d1) to[out=305,in=55] (n1);
        \draw (e1) to[out=270,in=125] (n1);
        \draw [] (n1) to (end);
      \end{tikzpicture}
      }
      \,\raisebox{45pt}{.}
\]
    \item[\ref{le:deltaefg}]This equality uses the previous equality, the commutativity
      of restriction idempotents (\rtwo) and the identity $\Delta\rst{\inv{\Delta}} = \Delta$.
      \[
      \raisebox{10pt}{
      \begin{tikzpicture}
        \node at (0,2) (start) {};
        \node at (0,1.5) [map] (e) {$\scriptstyle e$};
        \node at (0,1) [delta] (d2) {};
        \node at (-.25,.5) [map] (f) {$\scriptstyle f$};
        \node at (.25,.5) [map] (g) {$\scriptstyle g$};
        \node at (-.25,0) (end1) {};
        \node at (.25,0) (end2) {};
        \draw (start) to (e);
        \draw (e) to (d2);
        \draw (d2) to[out=235,in=90] (f);
        \draw (d2) to[out=305,in=90] (g);
        \draw (f) to (end1);
        \draw (g) to (end2);
      \end{tikzpicture}
      }
      \ \raisebox{45pt}{$=$}\
        \begin{tikzpicture}
        \node at (0,3) (start) {};
        \node at (0,2.5) [delta] (d1) {};
        \node at (-.25,2) [map] (e1) {$\scriptstyle e$};
        \node at (0,1.5) [nabla] (n1) {};
        \node at (0,1) [delta] (d2) {};
        \node at (-.25,.5) [map] (f) {$\scriptstyle f$};
        \node at (.25,.5) [map] (g) {$\scriptstyle g$};
        \node at (-.25,0) (end1) {};
        \node at (.25,0) (end2) {};
        \draw [] (start) to (d1);
        \draw [] (d1) to[out=235,in=90] (e1);
        \draw [] (d1) to[out=305,in=55] (n1);
        \draw (e1) to[out=270,in=125] (n1);
        \draw [] (n1) to (d2);
        \draw (d2) to[out=235,in=90] (f);
        \draw (d2) to[out=305,in=90] (g);
        \draw (f) to (end1);
        \draw (g) to (end2);
      \end{tikzpicture}
      \ \raisebox{45pt}{$=$}\
        \begin{tikzpicture}
        \node at (0,3) (start) {};
        \node at (0,2.5) [delta] (d1) {};
        \node at (-.25,2) [map] (e1) {$\scriptstyle e$};
        \node at (0,1.25) [map] (inverse-delta-r) {$\scriptstyle \rst{\inv{\Delta}}$};
        \node at (-.25,.5) [map] (f) {$\scriptstyle f$};
        \node at (.25,.5) [map] (g) {$\scriptstyle g$};
        \node at (-.25,0) (end1) {};
        \node at (.25,0) (end2) {};
        \draw [] (start) to (d1);
        \draw [] (d1) to[out=235,in=90] (e1);
        \draw [] (d1) to[out=305,in=55] (inverse-delta-r);
        \draw (e1) to[out=270,in=125] (inverse-delta-r);
        \draw (inverse-delta-r) to[out=235,in=90] (f);
        \draw (inverse-delta-r) to[out=305,in=90] (g);
        \draw (f) to (end1);
        \draw (g) to (end2);
      \end{tikzpicture}
      \ \raisebox{45pt}{$=$}\
        \begin{tikzpicture}
        \node at (0,3) (start) {};
        \node at (0,2.5) [delta] (d1) {};
        \node at (0,2) [map] (inverse-delta-r) {$\scriptstyle \rst{\inv{\Delta}}$};
        \node at (-.25,1.25) [map] (e1) {$\scriptstyle e$};
        \node at (-.25,.5) [map] (f) {$\scriptstyle f$};
        \node at (.25,.5) [map] (g) {$\scriptstyle g$};
        \node at (-.25,0) (end1) {};
        \node at (.25,0) (end2) {};
        \draw [] (start) to (d1);
        \draw [] (d1) to[out=235,in=125] (inverse-delta-r);
        \draw [] (d1) to[out=305,in=55] (inverse-delta-r);
        \draw (inverse-delta-r) to[out=235,in=90] (e1);
        \draw (e1) to (f);
        \draw (inverse-delta-r) to[out=305,in=90] (g);
        \draw (f) to (end1);
        \draw (g) to (end2);
      \end{tikzpicture}
      \ \raisebox{45pt}{$=$}\
      \raisebox{20pt}{
        \begin{tikzpicture}
        \node at (0,1.5) (start) {};
        \node at (0,1) [delta] (d1) {};
        \node at (-.25,.5) [map] (f_e) {$\scriptstyle f e$};
        \node at (.25,.5) [map] (g) {$\scriptstyle g$};
        \node at (-.25,0) (end1) {};
        \node at (.25,0) (end2) {};
        \draw [] (start) to (d1);
        \draw [] (d1) to[out=235,in=90] (f_e);
        \draw [] (d1) to[out=305,in=90] (g);
        \draw (f_e) to (end1);
        \draw (g) to (end2);
      \end{tikzpicture}
      }
      \,\raisebox{45pt}{.}
      \]
      The second equality ($e\Delta(f\*g) = \Delta(f\*e g)$) follows by co-commutativity. The third
      equality,  ($e\Delta(f\*g) = \Delta(e f\*e g)$) follows by naturality of $\Delta$.

    \item[\ref{le:efginvdelta}] As in \ref{le:deltaefg}, details are only given for the
      first equality. This proof is arrived at by reversing the diagrams of \ref{le:deltaefg}.
      \[
      \raisebox{10pt}{
      \begin{tikzpicture}
        \node at (0,0) (start) {};
        \node at (0,.5) [map] (e) {$\scriptstyle e$};
        \node at (0,1) [nabla] (d2) {};
        \node at (-.25,1.5) [map] (f) {$\scriptstyle f$};
        \node at (.25,1.5) [map] (g) {$\scriptstyle g$};
        \node at (-.25,2) (end1) {};
        \node at (.25,2) (end2) {};
        \draw (start) to (e);
        \draw (e) to (d2);
        \draw (d2) to[out=125,in=270] (f);
        \draw (d2) to[out=55,in=270] (g);
        \draw (f) to (end1);
        \draw (g) to (end2);
      \end{tikzpicture}
      }
      \ \raisebox{45pt}{$=$}\
        \begin{tikzpicture}
        \node at (0,0) (start) {};
        \node at (0,.5) [nabla] (d1) {};
        \node at (-.25,1) [map] (e1) {$\scriptstyle e$};
        \node at (0,1.5) [delta] (n1) {};
        \node at (0,2) [nabla] (d2) {};
        \node at (-.25,2.5) [map] (f) {$\scriptstyle f$};
        \node at (.25,2.5) [map] (g) {$\scriptstyle g$};
        \node at (-.25,3) (end1) {};
        \node at (.25,3) (end2) {};
        \draw [] (start) to (d1);
        \draw [] (d1) to[out=125,in=270] (e1);
        \draw [] (d1) to[out=55,in=305] (n1);
        \draw (e1) to[out=90,in=235] (n1);
        \draw [] (n1) to (d2);
        \draw (d2) to[out=125,in=270] (f);
        \draw (d2) to[out=55,in=270] (g);
        \draw (f) to (end1);
        \draw (g) to (end2);
      \end{tikzpicture}
      \ \raisebox{45pt}{$=$}\
        \begin{tikzpicture}
        \node at (0,0) (start) {};
        \node at (0,.5) [nabla] (d1) {};
        \node at (-.25,1) [map] (e1) {$\scriptstyle e$};
        \node at (0,1.75) [map] (inverse-delta-r) {$\scriptstyle \rst{\inv{\Delta}}$};
        \node at (-.25,2.5) [map] (f) {$\scriptstyle f$};
        \node at (.25,2.5) [map] (g) {$\scriptstyle g$};
        \node at (-.25,3) (end1) {};
        \node at (.25,3) (end2) {};
        \draw [] (start) to (d1);
        \draw [] (d1) to[out=125,in=270] (e1);
        \draw [] (d1) to[out=55,in=305] (inverse-delta-r);
        \draw (e1) to[out=90,in=235] (inverse-delta-r);
        \draw (inverse-delta-r) to[out=125,in=270] (f);
        \draw (inverse-delta-r) to[out=55,in=270] (g);
        \draw (f) to (end1);
        \draw (g) to (end2);
      \end{tikzpicture}
      \ \raisebox{45pt}{$=$}\
      \raisebox{20pt}{
        \begin{tikzpicture}
        \node at (0,0) (start) {};
        \node at (0,.5) [nabla] (d1) {};
        \node at (-.25,1) [map] (f e) {$\scriptstyle f e$};
        \node at (.25,1) [map] (g) {$\scriptstyle g$};
        \node at (-.25,1.5) (end1) {};
        \node at (.25,1.5) (end2) {};
        \draw [] (start) to (d1);
        \draw [] (d1) to[out=125,in=270] (f e);
        \draw [] (d1) to[out=55,in=270] (g);
        \draw (f e) to (end1);
        \draw (g) to (end2);
      \end{tikzpicture}
      }
      \,\raisebox{45pt}{.}
      \]
      The other equalities follow for the same reasons as in \ref{le:deltaefg}.

    \item[\ref{le:restfg}]Here, we start by using the fact all maps have a partial inverse,
      therefore we have:
      \[
        \restr{\Delta (f \* g) \inv{\Delta} } =\Delta (f \* g) \inv{\Delta} \Delta (\inv{f} \*
        \inv{g}) \inv{\Delta}.
     \]
     Now, we proceed with showing the rest of the equality via diagrams.

     \[
        \begin{tikzpicture}
        \node at (0,3.5) (start) {};
        \node at (0,3) [delta] (d1) {};
        \node at (-.25,2.5) [map] (f) {$\scriptstyle f$};
        \node at (.25,2.5) [map] (g) {$\scriptstyle g$};
        \node at (0, 2) [nabla] (n1) {};
        \node at (0,1.5) [delta] (d2) {};
        \node at (-.5,1) [map] (f-inverse) {$\scriptstyle \inv{f}$};
        \node at (.5,1) [map] (g-inverse) {$\scriptstyle \inv{g}$};
        \node at (0, .5) [nabla] (n2) {};
        \node at (0,0) (end) {};
        \draw [] (start) to (d1);
        \draw [] (d1) to[out=235,in=90] (f);
        \draw [] (d1) to[out=305,in=90] (g);
        \draw (f) to[out=270,in=125] (n1);
        \draw (g) to[out=270,in=55] (n1);
        \draw (n1) to (d2);
        \draw [] (d2) to[out=235,in=90] (f-inverse);
        \draw [] (d2) to[out=305,in=90] (g-inverse);
        \draw (f-inverse) to[out=270,in=125] (n2);
        \draw (g-inverse) to[out=270,in=55] (n2);
        \draw (n2) to (end);
      \end{tikzpicture}
      \ \raisebox{45pt}{$=$}\
      \begin{tikzpicture}
        \node at (0,3.5) (start) {};
        \node at (0,3) [delta] (d1) {};
        \node at (.25,2.5) [map] (f) {$\scriptstyle f$};
        \node at (-.25,2.5) [map] (g) {$\scriptstyle g$};
        \node at (0, 2) [nabla] (n1) {};
        \node at (0,1.5) [delta] (d2) {};
        \node at (.5,1) [map] (f-inverse) {$\scriptstyle \inv{f}$};
        \node at (-.5,1) [map] (g-inverse) {$\scriptstyle \inv{g}$};
        \node at (0, .5) [nabla] (n2) {};
        \node at (0,0) (end) {};
        \draw [] (start) to (d1);
        \draw [] (d1) to[out=235,in=90] (g);
        \draw [] (d1) to[out=305,in=90] (f);
        \draw (g) to[out=270,in=125] (n1);
        \draw (f) to[out=270,in=55] (n1);
        \draw (n1) to (d2);
        \draw [] (d2) to[out=235,in=90] (g-inverse);
        \draw [] (d2) to[out=305,in=90] (f-inverse);
        \draw (g-inverse) to[out=270,in=125] (n2);
        \draw (f-inverse) to[out=270,in=55] (n2);
        \draw (n2) to (end);
      \end{tikzpicture}
      \ \raisebox{45pt}{$=$}\
      \begin{tikzpicture}
        \node at (0,3.5) (start) {};
        \node at (0,3) [delta] (d1) {};
        \node at (.25,2.5) [map] (f) {$\scriptstyle f$};
        \node at (-.25,2.5) [map] (g) {$\scriptstyle g$};
        \node at (-.25,2) [delta] (d2) {};
        \node at (.25, 1.5) [nabla] (n1) {};
        \node at (.5,1) [map] (f-inverse) {$\scriptstyle \inv{f}$};
        \node at (-.5,1) [map] (g-inverse) {$\scriptstyle \inv{g}$};
        \node at (0, .5) [nabla] (n2) {};
        \node at (0,0) (end) {};
        \draw [] (start) to (d1);
        \draw [] (d1) to[out=235,in=90] (g);
        \draw [] (d1) to[out=305,in=90] (f);
        \draw (g) to (d2);
        \draw (d2) to[out=305,in=125] (n1);
        \draw (f) to[out=270,in=55] (n1);
        \draw (n1) to (f-inverse);
        \draw [] (d2) to[out=235,in=90] (g-inverse);
        \draw (g-inverse) to[out=270,in=125] (n2);
        \draw (f-inverse) to[out=270,in=55] (n2);
        \draw (n2) to (end);
      \end{tikzpicture}
      \ \raisebox{45pt}{$=$}\
      \begin{tikzpicture}
        \node at (0,3.5) (start) {};
        \node at (0,3) [delta] (d1) {};
        \node at (-.25,2.5) [delta] (d2) {};
        \node at (-.5,2) [map] (rest-g) {$\scriptstyle \rst{g}$};
        \node at (0,2) [map] (g) {$\scriptstyle g$};
        \node at (.75,1.5) [map] (rest-f) {$\scriptstyle \rst{f}$};
        \node at (0.1,1.5) [map] (f-inverse) {$\scriptstyle \inv{f}$};
        \node at (.25, 1) [nabla] (n1) {};
        \node at (0, .5) [nabla] (n2) {};
        \node at (0,0) (end) {};
        \draw [] (start) to (d1);
        \draw [] (d1) to[out=235,in=90] (d2);
        \draw [] (d1) to[out=305,in=90] (rest-f);
        \draw (d2) to[out=235,in=90] (rest-g);
        \draw (d2) to[out=305,in=90] (g);
        \draw (rest-g) to[out=270,in=125] (n2);
v        \draw (g) to (f-inverse);
        \draw (f-inverse) to[out=270,in=125] (n1);
        \draw (rest-f) to[out=270,in=55] (n1);
        \draw (n1) to[out=270,in=55] (n2);
        \draw (n2) to (end);
      \end{tikzpicture}
      \ \raisebox{45pt}{$=$}\
     \]
     \[
      \begin{tikzpicture}
        \node at (0,3.5) (start) {};
        \node at (0,3) [delta] (d1) {};
        \node at (-.35,2.5) [delta] (d2) {};
        \node at (0,1.75) [map] (g-f-inverse) {$\scriptstyle g \inv{f}$};
        \node at (.35, 1) [nabla] (n1) {};
        \node at (0, .5) [nabla] (n2) {};
        \node at (0,0) (end) {};
        \draw [] (start) to (d1);
        \draw [] (d1) to[out=235,in=90] (d2);
        \draw [] (d1) to[out=305,in=55] (n1);
        \draw (d2) to[out=235,in=125] (n2);
        \draw (d2) to[out=305,in=90] (g-f-inverse);
        \draw (g-f-inverse) to[out=270,in=125] (n1);
        \draw (n1) to[out=270,in=55] (n2);
        \draw (n2) to (end);
      \end{tikzpicture}
      \ \raisebox{45pt}{$=$}\
      \begin{tikzpicture}
        \node at (0,3.5) (start) {};
        \node at (0,3) [delta] (d1) {};
        \node at (-.35,2.5) [delta] (d2) {};
        \node at (.35, 2) [nabla] (n1) {};
        \node at (-.35,1.25) [map] (g-f-inverse) {$\scriptstyle g \inv{f}$};
        \node at (0, .5) [nabla] (n2) {};
        \node at (0,0) (end) {};
        \draw [] (start) to (d1);
        \draw [] (d1) to[out=235,in=90] (d2);
        \draw [] (d1) to[out=305,in=55] (n1);
        \draw (d2) to[out=305,in=125] (n1);
        \draw (d2) to[out=235,in=90] (g-f-inverse);
        \draw (g-f-inverse) to[out=270,in=125] (n2);
        \draw (n1) to[out=270,in=55] (n2);
        \draw (n2) to (end);
      \end{tikzpicture}
      \ \raisebox{45pt}{$=$}\
        \begin{tikzpicture}
        \node at (0,3) (start) {};
        \node at (0,2.5) [delta] (d1) {};
        \node at (0, 2) [nabla] (n1) {};
        \node at (0,1.5) [delta] (d2) {};
        \node at (-.5,1) [map] (g-f-inverse) {$\scriptstyle g\inv{f}$};
        \node at (0, .5) [nabla] (n2) {};
        \node at (0,0) (end) {};
        \draw [] (start) to (d1);
        \draw [] (d1) to[out=235,in=125] (n1);
        \draw [] (d1) to[out=305,in=55] (n1);
        \draw (n1) to (d2);
        \draw [] (d2) to[out=235,in=90] (g-f-inverse);
        \draw [] (d2) to[out=305,in=55] (n2);
        \draw (g-f-inverse) to[out=270,in=125] (n2);
        \draw (n2) to (end);
      \end{tikzpicture}
      \ \raisebox{45pt}{$=$}\
      \raisebox{15pt}{
       \begin{tikzpicture}
        \node at (0,2) (start) {};
        \node at (0,1.5) [delta] (d2) {};
        \node at (.5,1) [map] (g-f-inverse) {$\scriptstyle g\inv{f}$};
        \node at (0, .5) [nabla] (n2) {};
        \node at (0,0) (end) {};
        \draw [] (start) to (d2);
        \draw [] (d2) to[out=305,in=90] (g-f-inverse);
        \draw [] (d2) to[out=235,in=125] (n2);
        \draw (g-f-inverse) to[out=270,in=55] (n2);
        \draw (n2) to (end);
      \end{tikzpicture}
      }
      \,\raisebox{45pt}{.}
     \]
    \item[\ref{le:hge}]Beginning with the assumption that $\Delta (h \* g)\inv{\Delta}$ equals its
      restriction and by item \ref{le:restfg}, we have:
      \[
        \begin{tikzpicture}
        \node at (0,3.5) (start) {};
        \node at (0,3) [delta] (d1) {};
        \node at (-.25,2.5) [map] (h1) {$\scriptstyle h$};
        \node at (.25,2.5) [map] (g) {$\scriptstyle g$};
        \node at (0, 2) [nabla] (n1) {};
        \node at (0,1.5) [map] (h) {$\scriptstyle h$};
        \node at (0,1) (end) {};
        \draw [] (start) to (d1);
        \draw [] (d1) to[out=235,in=90] (h1);
        \draw [] (d1) to[out=305,in=90] (g);
        \draw (h1) to[out=270,in=125] (n1);
        \draw (g) to[out=270,in=55] (n1);
        \draw (n1) to (h);
        \draw (h) to (end);
      \end{tikzpicture}
      \ \raisebox{45pt}{$=$}\
        \begin{tikzpicture}
        \node at (0,3.5) (start) {};
        \node at (0,3) [delta] (d1) {};
        \node at (.5,2.5) [map] (g-h-inverse) {$\scriptstyle g\inv{h}$};
        \node at (0, 2) [nabla] (n1) {};
        \node at (0,1.5) [map] (h) {$\scriptstyle h$};
        \node at (0,1) (end) {};
        \draw [] (start) to (d1);
        \draw [] (d1) to[out=235,in=125] (n1);
        \draw [] (d1) to[out=305,in=90] (g-h-inverse);
        \draw (g-h-inverse) to[out=270,in=55] (n1);
        \draw (n1) to (h);
        \draw (h) to (end);
      \end{tikzpicture}
      \ \raisebox{45pt}{$=$}\
        \begin{tikzpicture}
        \node at (0,3.5) (start) {};
        \node at (0,3) [delta] (d1) {};
        \node at (.5,2.25) [map] (g-h-inverse) {$\scriptstyle g\inv{h}$};
        \node at (-.25,1.5) [map] (h1) {$\scriptstyle h$};
        \node at (.25,1.5) [map] (h2) {$\scriptstyle h$};
        \node at (0, 1) [nabla] (n1) {};
        \node at (0,.5) (end) {};
        \draw [] (start) to (d1);
        \draw [] (d1) to[out=235,in=90] (h1);
        \draw [] (d1) to[out=305,in=90] (g-h-inverse);
        \draw (g-h-inverse) to[out=270,in=90] (h2);
        \draw (h1) to[out=270,in=125] (n1);
        \draw (h2) to[out=270,in=55] (n1);
        \draw (n1) to (end);
      \end{tikzpicture}
      \ \raisebox{45pt}{$=$}\
        \begin{tikzpicture}
        \node at (0,3.5) (start) {};
        \node at (0,3) [delta] (d1) {};
        \node at (.5,2.25) [map] (g-h-inverse) {$\scriptstyle g\rst{\inv{h}}$};
        \node at (-.25,1.5) [map] (h1) {$\scriptstyle h$};
        \node at (0, 1) [nabla] (n1) {};
        \node at (0,.5) (end) {};
        \draw [] (start) to (d1);
        \draw [] (d1) to[out=235,in=90] (h1);
        \draw [] (d1) to[out=305,in=90] (g-h-inverse);
        \draw (h1) to[out=270,in=125] (n1);
        \draw (g-h-inverse) to[out=270,in=55] (n1);
        \draw (n1) to (end);
      \end{tikzpicture}
      \ \raisebox{45pt}{$=$}\
        \begin{tikzpicture}
        \node at (0,3.5) (start) {};
        \node at (0,3) [delta] (d1) {};
        \node at (.25,2.25) [map] (g) {$\scriptstyle g$};
        \node at (-.5,1.5) [map] (h1) {$\scriptstyle h\rst{\inv{h}}$};
        \node at (0, 1) [nabla] (n1) {};
        \node at (0,.5) (end) {};
        \draw [] (start) to (d1);
        \draw [] (d1) to[out=235,in=90] (h1);
        \draw [] (d1) to[out=305,in=90] (g);
        \draw (h1) to[out=270,in=125] (n1);
        \draw (g) to[out=270,in=55] (n1);
        \draw (n1) to (end);
      \end{tikzpicture}
      \ \raisebox{45pt}{$=$}\
      \raisebox{15pt}{
        \begin{tikzpicture}
        \node at (0,3.5) (start) {};
        \node at (0,3) [delta] (d1) {};
        \node at (.25,2.5) [map] (g) {$\scriptstyle g$};
        \node at (-.25,2.5) [map] (h1) {$\scriptstyle h$};
        \node at (0, 2) [nabla] (n1) {};
        \node at (0,1.5) (end) {};
        \draw [] (start) to (d1);
        \draw [] (d1) to[out=235,in=90] (h1);
        \draw [] (d1) to[out=305,in=90] (g);
        \draw (h1) to[out=270,in=125] (n1);
        \draw (g) to[out=270,in=55] (n1);
        \draw (n1) to (end);
      \end{tikzpicture}
      }
      \ \raisebox{45pt}{$=$}\
      \raisebox{15pt}{
        \begin{tikzpicture}
        \node at (0,3.5) (start) {};
        \node at (0,3) [delta] (d1) {};
        \node at (-.25,2.5) [map] (g) {$\scriptstyle g$};
        \node at (.25,2.5) [map] (h1) {$\scriptstyle h$};
        \node at (0, 2) [nabla] (n1) {};
        \node at (0,1.5) (end) {};
        \draw [] (start) to (d1);
        \draw [] (d1) to[out=305,in=90] (h1);
        \draw [] (d1) to[out=235,in=90] (g);
        \draw (h1) to[out=270,in=55] (n1);
        \draw (g) to[out=270,in=125] (n1);
        \draw (n1) to (end);
      \end{tikzpicture}
      }
      \,\raisebox{45pt}{.}
      \]

    \item[\ref{le:dfgisfg}] Our assumption is that:
      \[
        \begin{tikzpicture}
        \node at (0,1.5) (start) {};
        \node at (0,1) [delta] (d1) {};
        \node at (-.25,.5) [map] (g) {$\scriptstyle g$};
        \node at (-.25, 0) (end1) {};
        \node at (.25,0) (end2) {};
        \draw [] (start) to (d1);
        \draw [] (d1) to[out=305,in=90] (end2);
        \draw [] (d1) to[out=235,in=90] (g);
        \draw (g) to (end1);
      \end{tikzpicture}
      \ \raisebox{20pt}{$=$}\
        \begin{tikzpicture}
        \node at (0,1.5) (start) {};
        \node at (0,1) [delta] (d1) {};
        \node at (-.25,.5) [map] (f) {$\scriptstyle f$};
        \node at (-.25, 0) (end1) {};
        \node at (.25,0) (end2) {};
        \draw [] (start) to (d1);
        \draw [] (d1) to[out=305,in=90] (end2);
        \draw [] (d1) to[out=235,in=90] (f);
        \draw (f) to (end1);
      \end{tikzpicture}
      \raisebox{20pt}{ and by co-commutativity, }
        \begin{tikzpicture}
        \node at (0,1.5) (start) {};
        \node at (0,1) [delta] (d1) {};
        \node at (.25,.5) [map] (g) {$\scriptstyle g$};
        \node at (.25, 0) (end1) {};
        \node at (-.25,0) (end2) {};
        \draw [] (start) to (d1);
        \draw [] (d1) to[out=235,in=90] (end2);
        \draw [] (d1) to[out=305,in=90] (g);
        \draw (g) to (end1);
      \end{tikzpicture}
      \ \raisebox{20pt}{$=$}\
        \begin{tikzpicture}
        \node at (0,1.5) (start) {};
        \node at (0,1) [delta] (d1) {};
        \node at (.25,.5) [map] (f) {$\scriptstyle f$};
        \node at (.25, 0) (end1) {};
        \node at (-.25,0) (end2) {};
        \draw [] (start) to (d1);
        \draw [] (d1) to[out=235,in=90] (end2);
        \draw [] (d1) to[out=305,in=90] (f);
        \draw (f) to (end1);
      \end{tikzpicture}
      \,\raisebox{20pt}{.}
      \]
      Hence,
      \[
      \ \raisebox{45pt}{$f=$}\
      \raisebox{15pt}{
        \begin{tikzpicture}
        \node at (0,3.5) (start) {};
        \node at (0,3) [delta] (d1) {};
        \node at (-.25,2.5) [map] (f1) {$\scriptstyle f$};
        \node at (.25,2.5) [map] (f2) {$\scriptstyle f$};
        \node at (0, 2) [nabla] (n1) {};
        \node at (0,1.5) (end) {};
        \draw [] (start) to (d1);
        \draw [] (d1) to[out=305,in=90] (f2);
        \draw [] (d1) to[out=235,in=90] (f1);
        \draw (f2) to[out=270,in=55] (n1);
        \draw (f1) to[out=270,in=125] (n1);
        \draw (n1) to (end);
      \end{tikzpicture}
      }
      \ \raisebox{45pt}{$=$}\
        \begin{tikzpicture}
        \node at (0,3.5) (start) {};
        \node at (0,3) [delta] (d1) {};
        \node at (-.25,2.5) [map] (f1) {$\scriptstyle f$};
        \node at (.25,2) [map] (f2) {$\scriptstyle f$};
        \node at (0, 1.5) [nabla] (n1) {};
        \node at (0,1) (end) {};
        \draw [] (start) to (d1);
        \draw [] (d1) to[out=305,in=90] (f2);
        \draw [] (d1) to[out=235,in=90] (f1);
        \draw (f2) to[out=270,in=55] (n1);
        \draw (f1) to[out=270,in=125] (n1);
        \draw (n1) to (end);
      \end{tikzpicture}
      \ \raisebox{45pt}{$=$}\
        \begin{tikzpicture}
        \node at (0,3.5) (start) {};
        \node at (0,3) [delta] (d1) {};
        \node at (-.25,2.5) [map] (g1) {$\scriptstyle g$};
        \node at (.25,2) [map] (f2) {$\scriptstyle f$};
        \node at (0, 1.5) [nabla] (n1) {};
        \node at (0,1) (end) {};
        \draw [] (start) to (d1);
        \draw [] (d1) to[out=305,in=90] (f2);
        \draw [] (d1) to[out=235,in=90] (g1);
        \draw (f2) to[out=270,in=55] (n1);
        \draw (g1) to[out=270,in=125] (n1);
        \draw (n1) to (end);
      \end{tikzpicture}
      \ \raisebox{45pt}{$=$}\
        \begin{tikzpicture}
        \node at (0,3.5) (start) {};
        \node at (0,3) [delta] (d1) {};
        \node at (-.25,2) [map] (g1) {$\scriptstyle g$};
        \node at (.25,2.5) [map] (f2) {$\scriptstyle f$};
        \node at (0, 1.5) [nabla] (n1) {};
        \node at (0,1) (end) {};
        \draw [] (start) to (d1);
        \draw [] (d1) to[out=305,in=90] (f2);
        \draw [] (d1) to[out=235,in=90] (g1);
        \draw (f2) to[out=270,in=55] (n1);
        \draw (g1) to[out=270,in=125] (n1);
        \draw (n1) to (end);
      \end{tikzpicture}
      \ \raisebox{45pt}{$=$}\
        \begin{tikzpicture}
        \node at (0,3.5) (start) {};
        \node at (0,3) [delta] (d1) {};
        \node at (-.25,2) [map] (g1) {$\scriptstyle g$};
        \node at (.25,2.5) [map] (g2) {$\scriptstyle g$};
        \node at (0, 1.5) [nabla] (n1) {};
        \node at (0,1) (end) {};
        \draw [] (start) to (d1);
        \draw [] (d1) to[out=305,in=90] (g2);
        \draw [] (d1) to[out=235,in=90] (g1);
        \draw (g2) to[out=270,in=55] (n1);
        \draw (g1) to[out=270,in=125] (n1);
        \draw (n1) to (end);
      \end{tikzpicture}
      \ \raisebox{45pt}{$= g.$}\
      \]
    \item[\ref{le:fgisfg}] Use the same diagrammatic argument as in item \ref{le:dfgisfg}.
  \end{enumerate}
\end{proof}

\begin{proposition}\label{prop:discrete_inverse_category_has_meets}
  A discrete inverse category has meets, where $f\meet g =\Delta (f\* g) \inv{\Delta}$.
\end{proposition}
\begin{proof}
  $f\meet g \le f$:
  \[
  \raisebox{45pt}{$f\meet g =$}
      \raisebox{15pt}{
        \begin{tikzpicture}
        \node at (0,3.5) (start) {};
        \node at (0,3) [delta] (d1) {};
        \node at (-.25,2.5) [map] (f) {$\scriptstyle f$};
        \node at (.25,2.5) [map] (g) {$\scriptstyle g$};
        \node at (0, 2) [nabla] (n1) {};
        \node at (0,1.5) (end) {};
        \draw [] (start) to (d1);
        \draw [] (d1) to[out=305,in=90] (g);
        \draw [] (d1) to[out=235,in=90] (f);
        \draw (g) to[out=270,in=55] (n1);
        \draw (f) to[out=270,in=125] (n1);
        \draw (n1) to (end);
      \end{tikzpicture}
      }
      \raisebox{45pt}{$=$}
      \raisebox{15pt}{
        \begin{tikzpicture}
        \node at (0,3.5) (start) {};
        \node at (0,3) [delta] (d1) {};
        \node at (-.5,2.5) [map] (f) {$\scriptstyle f\rst{\inv{f}}$};
        \node at (.25,2.5) [map] (g) {$\scriptstyle g$};
        \node at (0, 2) [nabla] (n1) {};
        \node at (0,1.5) (end) {};
        \draw [] (start) to (d1);
        \draw [] (d1) to[out=305,in=90] (g);
        \draw [] (d1) to[out=235,in=90] (f);
        \draw (g) to[out=270,in=55] (n1);
        \draw (f) to[out=270,in=125] (n1);
        \draw (n1) to (end);
      \end{tikzpicture}
      }
      \raisebox{45pt}{$=$}
      \raisebox{15pt}{
        \begin{tikzpicture}
        \node at (0,3.5) (start) {};
        \node at (0,3) [delta] (d1) {};
        \node at (-.25,2.5) [map] (f) {$\scriptstyle f$};
        \node at (.5,2.5) [map] (g) {$\scriptstyle g\rst{\inv{f}}$};
        \node at (0, 2) [nabla] (n1) {};
        \node at (0,1.5) (end) {};
        \draw [] (start) to (d1);
        \draw [] (d1) to[out=305,in=90] (g);
        \draw [] (d1) to[out=235,in=90] (f);
        \draw (g) to[out=270,in=55] (n1);
        \draw (f) to[out=270,in=125] (n1);
        \draw (n1) to (end);
      \end{tikzpicture}
      }
      \raisebox{45pt}{$=$}
        \begin{tikzpicture}
        \node at (0,3.5) (start) {};
        \node at (0,3) [delta] (d1) {};
        \node at (-.25,2) [map] (f) {$\scriptstyle f$};
        \node at (.5,2.5) [map] (g) {$\scriptstyle g\inv{f}$};
        \node at (.25,2) [map] (f2) {$\scriptstyle f$};
        \node at (0, 1.5) [nabla] (n1) {};
        \node at (0,1) (end) {};
        \draw [] (start) to (d1);
        \draw [] (d1) to[out=305,in=90] (g);
        \draw [] (d1) to[out=235,in=90] (f);
        \draw (g) to (f2);
        \draw (f2) to[out=270,in=55] (n1);
        \draw (f) to[out=270,in=125] (n1);
        \draw (n1) to (end);
      \end{tikzpicture}
      \ \raisebox{45pt}{$=$}
        \begin{tikzpicture}
        \node at (0,3.5) (start) {};
        \node at (0,3) [delta] (d1) {};
        \node at (.5,2.5) [map] (g) {$\scriptstyle g\inv{f}$};
        \node at (0, 2) [nabla] (n1) {};
        \node at (0,1.5) [map] (f) {$\scriptstyle f$};
        \node at (0,1) (end) {};
        \draw [] (start) to (d1);
        \draw [] (d1) to[out=305,in=90] (g);
        \draw [] (d1) to[out=235,in=125] (n1);
        \draw (g) to[out=270,in=55] (n1);
        \draw (n1) to (f);
        \draw (f) to (end);
      \end{tikzpicture}
      \ \raisebox{45pt}{$= \rst{f\meet g}f.$}
  \]

  $f\meet f = f$:
  \begin{equation*}
    f\meet f = \Delta(f\* f) \inv{\Delta} =f \Delta \inv{\Delta} = f\Delta.
  \end{equation*}

  $h(f\meet g) = h f \meet h g$:
  \begin{align*}
    h(f\meet g) &= h \Delta(f\* g) \inv{\Delta}& \text{Definition of }\meet\\
    &= \Delta(h \* h) (f \* g) \inv{\Delta} &\Delta\text{ natural}\\
    &= \Delta(h f\* h g) \inv{\Delta} &\text{compose maps}\\
    &= h f \meet h g&\text{Definition of }\meet.
  \end{align*}
\end{proof}

\begin{example}[\pinj is a discrete inverse category]\label{ex:pinj_is_a_discrete_inverse_category}
  In the inverse category \pinj (see Example~\ref{ex:pinj_is_an_inverse_category}), suppose we add
  the tensor given by the Cartesian product of sets. In detail, this means:
  \begin{align*}
    A \* B &= \{(a,b)| a\in A, b\in B\}\\
    f \* g &= \{((a,c),(b,d)) | (a,b) \in f, (c,d) \in g\}\\
    1 & = \{*\}\text{, a single element set.}
  \end{align*}
  The symmetric monoid isomorphisms are:
    \begin{align*}
      \usl &: \{(*,a)\} \mapsto \{a\}
      &\usr &: \{(a,*)\} \mapsto \{a\}\\
      a_{\*} &: \{((a,b),c)\} \mapsto \{(a,(b,c))\}
      &c_{\*} &: \{(a,b)\} \mapsto \{(b,a)\}
    \end{align*}

  Define $\Delta_A = \{(a,(a,a)) | a\in A\}$. Then \pinj is a discrete inverse category with the
  inverse product of $\*$. The required properties of co-commutativity, co-associativity and
  exchange are immediately obvious. To show the Frobenius rule for $\Delta$, first note that
  $\inv{\Delta}$ is defined only on the elements of $A\*A$ which agree in the first and second
  co-ordinate. We show the upper triangle of the Frobenius diagram in
  detail. Equation~\ref{eq:delta_inverse_delta} shows the result of applying $\Delta$ followed by
  $\inv{\Delta}$.
  \begin{equation}
    \Delta(\inv{\Delta}(A\*A)) = \Delta(\{a | (a,a) \in A\*A\})
    = \{(a,a) | (a,a) \in A\* A\}.\label{eq:delta_inverse_delta}
  \end{equation}
  Applying $(\Delta \* 1)a_{\*}$ to $A\*A$ is shown in Equation~\ref{eq:delta_tensor_one}.
  \begin{equation}
    a_{\*}(\Delta\*1(A\*A)) = a_{\*}(\{((a,a),a') | (a,a') \in A\*A\} = \{(a,(a,a')) | (a,a') \in
    A\* A\}.\label{eq:delta_tensor_one}
  \end{equation}
  Finally, applying $1 \* \inv{\Delta}$ to the result of Equation~\ref{eq:delta_tensor_one} gives us
  Equation~\ref{eq:one_tensor_delta_inverse}.
  \begin{equation}
    (1\*\inv{\Delta})(\{(a,(a,a')) | (a,a') \in  A\* A\} = \{(a,a) | (a,a) \in A\* A\} \label{eq:one_tensor_delta_inverse}.
  \end{equation}
  Thus, we have $\inv{\Delta}\Delta = (\Delta \* 1) a_{\*} (1\*\inv{\Delta})$ and the Frobenius
  condition is satisfied.
\end{example}
% subsection discrete_inverse_categories (end)


\subsection{The inverse subcategory of a discrete restriction category } % (fold)
\label{sub:the_inverse_subcategory_of_a_discrete_restriction_category}

Given a discrete restriction category, one can pick out the maps which are partial isomorphisms.
Using results from Sub-Section~\ref{sub:discrete_inverse_categories} and from
Sub-Section~\ref{sub:discrete_restriction_categories},
this section will show that these maps form a restriction subcategory and in fact form a discrete
inverse category.

\begin{proposition}\label{lem:inv_x_is_a_discrete_inverse_category}
  Given \X is a discrete restriction category, the invertible maps of \X, together with the objects
  of \X form a sub-restriction category which is a discrete inverse category. For the restriction
  category $\X$, we denote this sub-category by \Invc{\X}.
\end{proposition}
\begin{proof}
  As shown in Lemma~\ref{lem:rcs_partial_monic_section_inverse_properties}, partial isomorphisms
  are closed under composition. The identity maps are in \Invc{\X} and restrictions of
  partial isomorphisms are also partial isomorphisms.

  The product on the discrete restriction category \X becomes the tensor product of the restriction
  category \Invc{\X}. Table~\ref{tab:structural_maps_for_the_tensor_in_invx} shows how each of the
  elements of the tensor are defined. Note that the last definition makes explicit use of the fact
  we are in a discrete restriction category and hence the $\Delta$ of \X possesses a partial
  inverse.

  \begin{table}[h!]
    \begin{center}
      \begin{tabular}{|ccc|}
        \hline
        \X & \Invc{\X} & Inverse map\\
        \hline\hline
        $\scriptstyle A\times B$ & $\scriptstyle A\* B$ &\\
        \hline
        $\scriptstyle \top$ & $\scriptstyle 1$ &\\
        \hline
        $\scriptstyle \pi_1:\top\times A \to A$ & $\scriptstyle \usl:1\* A \to A$ & $\scriptstyle \<!,1\>$\\
        \hline
        $\scriptstyle \pi_0:A\times\top \to A$ & $\scriptstyle \usr:A\*1 \to A$& $\scriptstyle \<1,!\>$\\
        \hline
        ${\scriptstyle \<\pi_0 \pi_0,\<\pi_0 \pi_1,\pi_1\>\>:(A\times B)\times C \to A\times(B\times C)}$
          & $\scriptstyle a_{\*}:(A\*B)\*C \to A\*(B\*C)$
          & $\scriptstyle \<\<\pi_0, \pi_1 \pi_0\>,\pi_1 \pi_1\>$\\
        \hline
        $\scriptstyle \< \pi_1,\pi_0\>:A\times B \to B\times A$ & $\scriptstyle c_{\*}:A\*B \to B \* A$ & $\scriptstyle \< \pi_1,\pi_0\>$\\
        \hline
        $\scriptstyle \Delta_{\X}:A\to A\times A$ & $\scriptstyle \Delta:A \to A\* A$ & $\scriptstyle  \inv{\Delta_{\X}} $\\
        \hline
      \end{tabular}

    \end{center}
    \caption{Structural maps for the tensor in \Invc{\X}}
    \label{tab:structural_maps_for_the_tensor_in_invx}
  \end{table}

  The monoid coherence diagrams follow directly from the characteristics of the product in
  \X. Similarly, $\Delta$ is total as it is total in $\X$. It remains to show co-commutativity,
  co-associativity and the Frobenius condition.

  Co-commutativity requires $\Delta c_{\*} = c_{\*}$. This means we need
  \[\Delta_{\X} \< \pi_1,\pi_0\> = \Delta_{\X}.\] Once again, this follows immediately from the
  definition of the restriction product.

  Co-associativity requires $\Delta (1 \* \Delta) = \Delta (\Delta \* 1) a_{\*}$. Expressing this
  in \X, we require
  \[
    \Delta_{\X} (1 \times \Delta_{\X}) =
      \Delta_{\X}(\Delta_{\X} \times 1) (\<\pi_0 \pi_0,\<\pi_0 \pi_1,\pi_1\>\>).
  \]
  Again each is equal based on the properties of the restriction product.

  The Frobenius requirement is two-fold:
  \begin{align}
    \inv{\Delta} \Delta &= (\Delta \*1) a_{\*}(1\*\inv{\Delta}) \label{eq:frobenius_righths_need_in_invx}\\
    \inv{\Delta} \Delta &= (1 \* \Delta) \inv{a_{\*}}(\inv{\Delta}\* 1), \label{eq:frobenius_lefths_need_in_invx}.
  \end{align}
  In \X, this becomes:
  \begin{align}
    \inv{\Delta_{\X}} \Delta_{\X}
      &= (\Delta_{\X} \times 1) \<\pi_0 \pi_0,\<\pi_0 \pi_1,\pi_1\>\>(1\times\inv{\Delta_{\X}})
      \label{eq:frobenius_righths_expressed_in_x}\\
    \inv{\Delta_{\X}}\Delta_{\X}
      &= (1 \times \Delta_{\X}) \<\<\pi_0, \pi_1 \pi_0\>,\pi_1 \pi_1\>(\inv{\Delta_{\X}}\times 1).
      \label{eq:frobenius_lefths_expressed_in_x}
  \end{align}
  We will detail the proof of Equation~\ref{eq:frobenius_righths_expressed_in_x}.
  Equation~\ref{eq:frobenius_lefths_expressed_in_x} may be proven similarly.

  Note first that $\Delta(1 \times !)$ (and $\Delta(!\times 1)$) is the identity. Second, we see
  that maps to a product of objects may be split into a product map --- i.e.  if
  $f:A \to B \times B$, then $f = \<f(1\times !), f(!\times 1)\>$.

  Using this we see that the left hand side of Equation~\eqref{eq:frobenius_righths_expressed_in_x}
  computes as follows:
  \begin{align*}
    \inv{\Delta_{\X}} \Delta_{\X}
      & = \<\inv{\Delta_{\X}} \Delta_{\X}(1\times !), \inv{\Delta_{\X}} \Delta_{\X} (! \times 1)\>\\
    &= \<\inv{\Delta_{\X}}, \inv{\Delta_{\X}} \>
  \end{align*}
  Similarly, removing the associativity maps, the right hand side of the same equation becomes:
  \begin{align*}
    (\Delta_{\X} \times 1) (1\times\inv{\Delta_{\X}}) &
      = \<(\Delta_{\X} \times 1) (1\times\inv{\Delta_{\X}}) (1\times !),
      (\Delta_{\X} \times 1) (1\times\inv{\Delta_{\X}}) (! \times 1 )\> \\
    &= \<(\Delta_{\X} \times 1) (1\times\inv{\Delta_{\X}}) (1\times ! ), \inv{\Delta_{\X}}\> \\
    &= \<(\Delta_{\X} \times 1) (1\times\inv{\Delta_{\X}}) (1 \times \Delta_{\X})(1\times !\times !), \inv{\Delta_{\X}}\> \\
    &= \<(\Delta_{\X} \times 1) (1\times\rst{\inv{\Delta_{\X}}}) (1\times !\times !), \inv{\Delta_{\X}}\> \\
    &= \<(\Delta_{\X} \times 1) \rst{1\times\inv{\Delta_{\X}}} (1\times !\times !), \inv{\Delta_{\X}}\> \\
    &= \<\rst{(\Delta_{\X} \times 1) (1\times\inv{\Delta_{\X}})}
      (\Delta_{\X} \times 1)(1\times !\times !), \inv{\Delta_{\X}}\> \\ %rfour
    &= \<\rst{(\Delta_{\X} \times 1) (1\times\inv{\Delta_{\X}})} (1\times !), \inv{\Delta_{\X}}\> \\
      &= \<\rst{(\Delta_{\X} \times 1) (1\times\inv{\Delta_{\X}})(! \times 1)} (1\times !),
      \inv{\Delta_{\X}}\> \\ % add total to right of rst
    &= \<\rst{\inv{\Delta_{\X}}} (1\times !), \inv{\Delta_{\X}}\> \\
    &= \<\inv{\Delta_{\X}}\Delta_{\X}(1\times !), \inv{\Delta_{\X}}\> \\
    &= \<\inv{\Delta_{\X}}, \inv{\Delta_{\X}}\>
  \end{align*}
  and therefore we see that the first equation for the Frobenius condition is satisfied. Thus,
  $Inv(\X)$ is a discrete inverse category.
\end{proof}
% subsection the_inverse_subcategory_of_a_graphic_cartesian_restriction_category (end)

% section inverse_products (end)


\section{The category of Commutative Frobenius Algebras} % (fold)
\label{sec:the_category_of_commutative_frobenius_algebras}
Dagger categories generalize the category of Hilbert spaces which is often used to model quantum
computation. These were introduced in \cite{abramsky04:catsemquantprot} as \emph{strongly compact
closed categories}, an additional structure on compact closed categories.

Before introducing dagger categories, we define compact closed
categories.



\begin{definition}\label{def:compactclosedcat}
A \emph{compact closed category} \cD{} is a symmetric monoidal category with tensor $\*$ where each
object $A$ has a dual $A^{*}$. Additionally, there must exist families of maps $\eta_{A}: I \to
A^{*} \* A$ (the \emph{unit}) and $\epsilon_{A}: A\*A^{*}\to I$ (the \emph{counit}) such that
\[
  \xymatrix@C+15pt{
    A \ar[r]^{u_{A}} \ar@{=}[d]  & A\*I \ar[r]^(.4){1\*\eta_{A}}
        & A\* (A^{*}\*A) \ar[d]^{a_{A,A^{*},A}} \\
    A & I\* A \ar[l]^{u_{A}^{-1}} & (A\* A^{*})\*A \ar[l]^(.6){\epsilon_{A}\*1}
    }\text{ and }
  \xymatrix@C+15pt{
    A^{*} \ar[r]^{u_{A^*}} \ar@{=}[d]  & I\*A^* \ar[r]^(.4){\eta_{A}\*1}
        & (A^{*}\* A)\*A^{*} \ar[d]^{a_{A^{*},A,A^{*}}^{-1}} \\
    A^* & A^*\*I \ar[l]^{u_{A^*}^{-1}} & A^{*}\*(A\*A^{*}) \ar[l]^(.6){1\*\epsilon_{A}}
    }
  \]
commute.
\end{definition}

Given a map $f:A\to B$ in a compact closed category,  define the map $f^{*}:B^{*} \to A^{*}$ as
\[
  \xymatrix@C+10pt{
    B^{*}\ar[r]^{u_{B^{*}}} \ar[d]_{f^{*}}& I\*B^{*} \ar[r]^{\eta_{A}\*1}
      & A^{*}\*A\*B^{*}\ar[d]^{1\*f\*1}\\
    A^{*}&    A^{*}\*I\ar[l]^{u_{A^{*}}^{-1}}  &   A^{*}\*B\*B^{*}\ar[l]^{1\*\epsilon_{B}}.
  }
\]


%!TEX root = /Users/gilesb/UofC/thesis/phd-thesis/phd-thesis.tex
\subsection{Dagger categories}\label{ssec:daggercategories}

Although dagger categories were introduced in the context of compact closed categories, the concept
of a dagger is definable independently. This was first done in \cite{selinger05:dagger}.

\begin{definition}\label{def:daggercat}
  A \emph{dagger} on a category $D$ is a functor $\dagger:\dual{\cD}\to \cD$, which is  involutive,
  that is, $\dgr{\dgr{f}} = f$ and which is the identity on objects. A \emph{dagger category} is a
  category that has a dagger.
\end{definition}

Typically, the dagger is written as a superscript on the morphism. So, if $f:A\to B$ is a map in
\cD, then $\dgr{f}:B\to A$ is a map in \cD{} and is called the \emph{adjoint} of $f$. A map where
$f^{-1} = \dgr{f}$ is called \emph{unitary}. A map $f:A\to A$ with $f=\dgr{f}$ is called
\emph{self-adjoint} or \emph{Hermitian}.

\begin{definition}\label{def:daggersmc}
  A \emph{dagger symmetric monoidal category} is a symmetric monoidal category \cD{} with a dagger
  operator such that:
  \begin{enumerate}[{(}i{)}]
    \item For all maps $f:A\to B$ and $g:C\to D$, $\dgr{(f\*g)} = \dgr{f}\*\dgr{g}:B\*D \to A\* C$;\label{defitem:dagger_smc_one}
    \item The monoid structure isomorphisms $a_{A,B,C}:(A\*B)\* C\to A\*(B\*C)$, $u^l_{A}:I\*A\to
      A$, $u^r_{A}:A\*I \to A$ and  $c_{A,B}:A\*B \to B\*A$ are unitary.\label{defitem:dagger_smc_two}
  \end{enumerate}
\end{definition}


\begin{definition}\label{def:daggercompact}
  A \emph{dagger compact closed category} \cD{} is a dagger symmetric monoidal category
  that is compact closed where the diagram
  \[
    \xymatrix @C+20pt @R+10pt{
      I \ar[r]^{\epsilon^{\dagger}_{A}} \ar[dr]_{\eta_{A}} &A\*A^{*}\ar[d]^{c_{A,A^{*}}}\\
      &A^{*}\* A
    }
  \]
  commutes for all  objects $A$ in \cD.
\end{definition}

\begin{lemma}\label{lemma:daggerbiproducts}
If \cD{} is a dagger category with biproducts, with injections $in_{1},in_{2}$ and projections
$p_{1},p_{2}$, then the following are equivalent:
\begin{enumerate}[{(}i{)}]
  \item $\dgr{p_{i}} = in_{i}, i=1,2$, \label{ldpdgrpisq}
  \item $\dgr{(f\biproduct g)} = \dgr{f}\biproduct \dgr{g}$ and $\dgr{\Delta} = \nabla$,\label{ldpddeltisnab}
  \item $\dgr{\<f,g\>} = [\dgr{f},\dgr{g}]$,\label{ldpdcopisprod}
  \item The map $[\dgr{p_{1}},\dgr{p_{2}}]: \dgr{A} \biproduct \dgr{B} \to \dgr{(A\biproduct B)}$ is
    the identity map.\label{ldpcommute}
%the below diagram commutes:
%  \[
%    \xymatrix @C+20pt @R+10pt{
%      \dgr{A} \biproduct \dgr{B} \ar[d]_{id} \ar[dr]^{[\dgr{p_{1}},\dgr{p_{2}}]}\\
%      A\biproduct B\ar[r]_{id}&\dgr{(A\biproduct B)}.
%    }
%  \]
\end{enumerate}
\end{lemma}
\begin{proof}
  \begin{description}
    \item[\ref{ldpdgrpisq}$\implies$\ref{ldpddeltisnab}] To show $\dgr{\Delta} = \nabla$,
    draw the product cone for $\Delta$,
    \[
      \xymatrix {
        &A \ar[d]^{\Delta} \ar[dr]^{id} \ar[dl]_{id}\\
        A
         & A\biproduct A \ar[l]^{p_{1}}  \ar[r]_{p_{2}}
         & A
      }
    \]
    and apply the dagger functor to it. As $\dgr{p_{i}} = in_{i}$, and $\dagger$ is identity on
    objects, this is now a coproduct diagram and therefore $\dgr{\Delta} = \nabla$.

    For $\dgr{(f\biproduct g)} = \dgr{f}\biproduct\dgr{g}$, start with the diagram defining
    $f\biproduct g$ as a product of the arrows:
    \[
      \xymatrix{
        A\ar[d]_{f}  & A\biproduct B \ar[l]_{p_{1}} \ar[r]^{p_{2}} \ar[d]^{f\biproduct g}&A \ar[d]^{g}\\
        C & C\biproduct D \ar[l]^{p_{1}} \ar[r]_{p_{2}}  & D.
      }
    \]
    Then, apply the dagger functor to this diagram. This is now the diagram defining the
    coproduct of maps and therefore $\dgr{(f\biproduct g)} = \dgr{f}\biproduct\dgr{g}$.
    \item[\ref{ldpddeltisnab}$\implies$\ref{ldpdcopisprod}] The calculation showing this is
      \begin{eqnarray*}
        &[\dgr{f},\dgr{g}] & = \nabla; (\dgr{f}\biproduct \dgr{g})\\
        & &=\dgr{\Delta}; (\dgr{f}\biproduct \dgr{g})\\
        & &=\dgr{\Delta}; \dgr{(f\biproduct g)}\\
        & & = \dgr{((f\biproduct g);\Delta)}\\
        & & = \dgr{\<f,g\>}.
      \end{eqnarray*}
    \item[\ref{ldpdcopisprod}$\implies$\ref{ldpcommute}]
      Under the assumption,
      \[
        [\dgr{p_{1}},\dgr{p_{2}}] = \dgr{\<p_{1},p_{2}\>}=\dgr{id}=id.
      \]
    \item[\ref{ldpcommute}$\implies$\ref{ldpdgrpisq}] As $[in_{1},in_{2}]:\dgr{A} \biproduct \dgr{B}
      \to \dgr{A} \biproduct \dgr{B} = id = [\dgr{p_{1}},\dgr{p_{2}}]$, we immediately have
      $\dgr{p_{1}} = in_{1}$ and $\dgr{p_{2}} = in_{2}$.
%
%Using the injections and under
%    the assumption, the following diagram commutes:
%      \[
%        \xymatrix @C+20pt @R+10pt{
%          \dgr{A} \biproduct \dgr{B} \ar[d]_{id} \ar[dr]^{[\dgr{p_{1}},\dgr{p_{2}}]}\ar[r]^{[in_{1},in_{2}]}
%            & \dgr{A} \biproduct \dgr{B} \ar[d]^{id}\\
%          A\biproduct B\ar[r]_{id}&\dgr{(A\biproduct B)}
%        }
%      \]
%      and therefore,
  \end{description}
\end{proof}

\begin{definition} \label{def:biproductdaggerccc}
  A \emph{biproduct dagger compact closed category} is a dagger compact closed category with
  biproducts where the conditions of lemma \ref{lemma:daggerbiproducts} hold.
\end{definition}
\subsection{Examples of dagger categories}

\begin{example}[\fdh]\label{ex:fdhilbert_is_dagger_category}
The category of finite dimensional Hilbert spaces is the motivating example for
the creation of the dagger and is, in fact, a biproduct dagger compact closed category. The
biproduct is the direct sum of Hilbert spaces and the tensor for compact closure is the standard
tensor of Hilbert spaces. The dual $H^{*}$ of a space $H$ is the space of all continuous linear
functions from $H$ to the base field. The dagger is defined via the adjoint as being the unique map
$\dgr{f}:B\to A$ such that $\<f a|b\> = \<a | \dgr{f} b\>$ for all $a\in A, b\in B$.
\end{example}

\begin{example}[\rel]\label{ex:rel_is_dagger_category}
The category \rel of sets and relations has the tensor $S\*T \definedas S\times T$ and the biproduct
$S\biproduct T \definedas S\disjointunion T$. This is compact closed under $A^{*} \definedas A$ and
the dagger is the relational converse. That is, if the relation
$R=\{(s,t)|s\in S, t\in T\}:S\to T$, then $\dgr{R}=R^*=\{(t,s)|(s,t)\in R\}$.
\end{example}

\begin{example}[Inverse categories]\label{ex:inverse_category_is_dagger_category}
An inverse category \X is also a dagger category when the dagger is defined as the partial inverse.
The unitary maps are the total maps. When the inverse category \X is also a
symmetric monoidal category where the monoid $\*$ is actually a restriction bi-functor, then \X is
a dagger symmetric monoidal category.

Requirement \ref{defitem:dagger_smc_one} of Definition~\ref{def:daggersmc}  is fulfilled, as
\[
  (f\*g) \inv{(f\*g)} = \rst{f\*g}=\rst{f} \*\rst{g} =
   f\inv{f} \* g \inv{g} = (f\*g) (\inv{f} \* \inv{g})
\]
and since the partial inverse of $f\*g$ is unique, $\inv{(f\*g)} = \inv{f} \* \inv{g}$.
Requirement \ref{defitem:dagger_smc_two} is that the structure isomorphisms are unitary. This is, of
course, true as each of them are isomorphisms, hence total and therefore unitary.
\end{example}
%%% Local Variables:
%%% mode: latex
%%% TeX-master: "../../phd-thesis"
%%% End:

\section{Frobenius Algebras} % (fold)
\label{sec:frobenius_algebras}
In their most general setting, Frobenius algebras are defined as a finite dimensional algebra
over a field together with a non-degenerate pairing operation. We will continue with the definitions
that make this precise.

\subsection{Frobenius algebra definitions} % (fold)
\label{sub:frobenius_algebra_definitions}


\begin{definition}\label{def:frobeniusalgebra}
  Given a symmetric monoidal category \cD, a \emph{Frobenius algebra} is an object $X$ of \cD and
  four maps, $\nabla :X\*X \to X$, $e: I \to X$, $\Delta: X\to X\* X$ and $\epsilon:X\to I$, with
  the conditions that $(X,\nabla,e)$ forms a commutative monoid, $(X,\Delta, \epsilon)$ forms a
  commutative comonoid and the diagram
  \[
    \xymatrix{
      X\*X \ar[rr]^{X\*\Delta} \ar[dd]_{\Delta \* X} \ar[dr]^{\nabla}
        && X\*X\*X \ar[dd]^{\nabla\*X}\\
      & X\ar[dr]^{\Delta}\\
      X\*X\*X\ar[rr]_{X\*\nabla}  && X\*X
    }
  \]
  commutes. The Frobenius algebra is \emph{special} when $\Delta \nabla = 1_{X}$ and
  \emph{commutative} when $\Delta c_{X,X} = \Delta$.
\end{definition}
\begin{definition}\label{def:daggerfrob}
  A Frobenius algebra in a dagger symmetric monoidal category where $\Delta = \dgr{\nabla}$ and
  $\epsilon=\dgr{u}$ is a $\dagger$\emph{-Frobenius algebra}.
\end{definition}
For an example of a $\dagger$-Frobenius algebra, consider a finite dimensional Hilbert space $H$
with an orthonormal basis $\{\ket{\phi_{i}}\}$ and define $\Delta:H\to H\*H: \ket{\phi_{i}}\mapsto
\ket{\phi_{i}} \* \ket{\phi_{i}}$ and $\epsilon : H\to \complex : \ket{\phi_{i}} \mapsto 1$. Then $(H,
\nabla=\dgr{\Delta}, u=\dgr{\epsilon}, \Delta, \epsilon)$ forms a commutative special
$\dagger$-Frobenius algebra.

% subsection frobenius_algebra_definitions (end)

\subsection{Bases and Frobenius Algebras} % (fold)
\label{sub:bases_and_frobenius_algebras}
In \cite{coeckeetal08:ortho}, Coecke et. al. provide an algebraic description of orthogonal bases in
finite dimensional Hilbert spaces. Additionally,  an orthonormal basis for such a space is
a special commutative $\dagger$-Frobenius algebra. To show the other direction, given a commutative
$\dagger$-Frobenius algebra, $(H,\nabla,u)$ and for each element $\alpha\in H$, define the right
action of $\alpha$ as $R_{\alpha}:=(id\*\alpha)\, \nabla:H\to H$. Note the use of the fact that
elements $\alpha\in H$ can be considered as linear maps $\alpha:\complex \to H:1\mapsto \ket{\alpha}$.
The dagger of a right action is also a right action, $\dgr{R_{\alpha}} = R_{\alpha'}$ where
$\alpha'= u\, \nabla\, (id\* \dgr{\alpha})$, which is a consequence of the Frobenius identities.

The $(\_)'$ construction is actually an involution:
\begin{eqnarray*}
  &(\alpha')' &= u \nabla (id \* \dgr{\alpha'}) \\
  && = u \nabla (id \* \dgr{(u \nabla (id \* \dgr{\alpha}))}\\
  && = u \nabla (id \* ( (id \* \alpha) \Delta \epsilon))\\
  && = (u \* \alpha) (\nabla \* id) (id \* \Delta) (id \*  \epsilon)\\
  && = (u \* \alpha) (id \* \Delta) (\nabla \* id) (id \*  \epsilon)\\
  && = (u \* \alpha)  (id \*  \epsilon)\\
  && = \alpha
\end{eqnarray*}

\begin{lemma}\label{lemma:cstaralgebra}
  Any $\dagger$-Frobenius algebra in \fdh is a $C^{*}$-algebra.
\end{lemma}
\begin{proof}
  The endomorphism monoid of \fdh(H,H) is a $C^{*}$-algebra. From the proceeding, we have
  \[
    H \cong \fdh(\complex,H) \cong R_{[\fdh(\complex,H)]}\subseteq\fdh(H,H).
  \]
  This inherits the algebra structure from \fdh(H,H). Furthermore, since any finite dimensional
  involution-closed sub-algebra of a $C^{*}$-algebra is also a $C^{*}$-algebra, this shows the
  $\dagger$-Frobenius algebra is a $C^{*}$-algebra.
\end{proof}

Using the fact that the involution preserving homomorphisms from a finite dimensional commutative
$C^{*}$-algebra to $\complex$ form a basis for the dual of the underlying vector space, write these
homomorphisms as $\dgr{\phi_{i}}:H \to \complex$. Then their adjoints, $\phi_{i}:\complex\to H$ will form a
basis for the space $H$. These are the copyable elements in $H$.

This, together with continued applications of the Frobenius rules and linear algebra allow the
authors to prove the following Theorem.
\begin{theorem}
  Every commutative $\dagger$-Frobenius algebra in \fdh determines an orthogonal basis consisting
  of its copyable elements. Conversely, every orthogonal basis $\{\ket{\phi_{i}}\}_{i}$ determines
  a commutative $\dagger$-Frobenius algebra via \[\Delta:H\to H\*H: \ket{\phi_{i}}\mapsto
  \ket{\phi_{i}} \* \ket{\phi_{i}}\qquad\epsilon : H\to \complex : \ket{\phi_{i}} \mapsto 1\] and these
  constructions are inverse to each other.
\end{theorem}

% subsection bases_and_frobenius_algebras (end)

\subsection{Quantum and classical data}\label{sec:quantumclassical}
In \cite{coecke08structures}, Coecke et.al. build on the results of \cite{coeckeetal08:ortho}
to start from a $\dagger$-symmetric monoidal category and construct the minimal machinery needed to
model quantum and classical computations. For the rest of this section, $\cD$ will be assumed to be
such a category, with $\*$ the monoid tensor and $I$ the unit of the monoid.

\begin{definition}\label{def:compact_structure}
  A compact structure on an object $A$ in the category $\cD$ is given by the object $A$, an object
  $A^{*}$ called its \emph{dual} and the maps $\eta:I \to A^{*}\* A$, $\epsilon: A\* A^{*} \to I$
  such that the diagrams
  \[
    \xymatrix@C+20pt{
      A^{*} \ar[dr]^{id} \ar[d]_{\eta\*A^{*}} \\
      A^{*} \*A\*A^{*}  \ar[r]_(.6){A^{*} \*\epsilon} & A^{*}
    }
    \text{ and }
    \xymatrix@C+20pt{
      A \ar[r]^(.4){A\*\eta} \ar[dr]_{id} & A\* A^{*}\* A \ar[d]^{\epsilon\*A}\\
      & A
    }
  \]
  commute.
\end{definition}

\begin{definition}\label{def:quantumstructure}
  A \emph{quantum structure} is an object $A$ and map $\eta:I\to A\*A$ such that
  $(A,A,\eta,\dgr{\eta})$ form a compact structure.
\end{definition}
Note that $A$ is self-dual in definition \ref{def:quantumstructure}.

This allows the creation of the category $\cD_{q}$ which has as objects quantum structures and maps
are the maps in $\cD$ between the objects in the quantum structures.

In the category $\cD_{q}$, it is now possible to define the upper and lower $*$ operations on maps,
such that $(f_{*})^{*}= (f^{*})_{*} = \dgr{f}$:
\begin{eqnarray*}
&f^{*} &:= (\eta_{A}\*1) (1 \* f\*1) (1\*\dgr{\eta}_{B}),\\
&f_{*} &:= (\eta_{B}\*1) (1 \* \dgr{f}\*1) (1\*\dgr{\eta}_{A}).
\end{eqnarray*}

Next, define a classical structure on \cD.
\begin{definition}\label{def:classicalstructure}
  A \emph{classical structure} in \cD{} is an object $X$ together with two maps, $\Delta :X \to X\* X$,
  $\epsilon:X\to I$ such that $(X,\dgr{\Delta},\dgr{\epsilon},\Delta,\epsilon)$ forms a special
  Frobenius algebra.
\end{definition}

As above, this allows us to define $\cD_{c}$, the category whose objects are the classical
structures of $\cD$. The maps in $\cD_{c}$ are given by the maps in $\cD$ between the
objects of the classical structure.

Note that a classical structure will induce a quantum structure, setting $\eta_{X}$ to be
$\dgr{\epsilon_{X}}\, \Delta_{X}$.


Later on, in \ref{sec:the_category_of_commutative_frobenius_algebras}, we will show that commutative
special Frobenius algebras possess a specialized inverse category structure.
% subsection quantum_and_classical_data (end)


% section frobenius_algebras (end)

%%% Local Variables:
%%% mode: latex
%%% TeX-master: "../../phd-thesis"
%%% End:


\subsection{\CFrob is an inverse category}\label{ssec:cfrob_x_is_an_inverse_category}
\begin{example}[Commutative separable Frobenius algebras\cite{kock04}]\label{example:commfrob}
  Let \X be a symmetric monoidal category and form \CFrob as follows: \paragraph{\textbf{Objects:}}
  Commutative separable Frobenius algebras, a quintuple $(A,\nabla,\eta,\Delta,\epsilon)$ where
  $A$ is an object of \X with the following maps:
  $\nabla :A\*A \to A$, $\eta:I\to A$, $\Delta : A \to A\*A$, $\epsilon : A \to I$ which are natural
  maps in \X, with $(A,\nabla,\eta)$ a monoid and $(A,\Delta,\epsilon)$ a comonoid. Additionally,
  these satisfy
  \[
    \xymatrix @C=10pt @R=20pt{
      A \* A \ar[dd]_{1\*\Delta} \ar[dr]^{\nabla}
        \ar[rr]^{\Delta \* 1} & &
        A \* (A \* A) \ar[dd]^{1 \* \nabla}\\
      & A \ar[dr]^{\Delta} & \\
      (A \* A) \* A \ar[rr]_{\nabla \* 1} & &
        A \* A\\
      &*!<0pt,-25pt>{\text{\textbf{Frobenius}}}
    }
  \]
  together with the additional property that $\Delta \nabla = 1$ (separable).

  \paragraph{\textbf{Maps:}} The maps of \X between the objects of \X which preserve multiplication ($\nabla$)
  and comultiplication ($\Delta$) but do not necessarily preserve the units.
  This means a map $f$ must satisfy the following commuting diagrams:
  \[
    \xymatrix@C+25pt{
      A \ar[d]_{\Delta} \ar[r]^{f} & B \ar[d]^{\Delta}\\
      A\*A \ar[r]_{f\*f} & B\* B
    }
    \text{ and }
    \xymatrix@C+25pt{
      A\*A \ar[d]_{\nabla} \ar[r]^{f\*f}& B\*B \ar[d]^{\nabla}\\
      A \ar[r]_{f} & B.
    }
  \]
\end{example}

\begin{lemma}\label{lem:cfrobx_is_an_inverse_category}
  When \X is a symmetric monoidal category, \CFrob is an inverse category.
\end{lemma}
\begin{proof}
  We need to show that \CFrob has restrictions and that each map has a partial inverse. We do
  this by exhibiting the partial inverse of a map.
  For $f:X \to Y$, define $\inv{f}$ as
  \[
    Y \xrightarrow{1\*\eta} Y\*X \xrightarrow{1\*\Delta}
      Y\*X\*X \xrightarrow{1\*f\*1} Y\*Y\*X \xrightarrow{\nabla\*1}
      Y\*X \xrightarrow{\epsilon\*1}X.
  \]
  As a string diagram, this looks like:
  \[
  \begin{tikzpicture}
    \path node at (-.5,3) (start) {}
    node at (0,2.5) [eta] (eta1) {}
    node at (0,2) [delta] (d) {}
    node at (-.25,1.5) [map] (f) {$\scriptstyle f$}
    node at (-.5,1) [nabla] (n1) {}
    node at (-.5,.5) [epsilon] (e1) {}
    node at (0,0) (end) {};
    \draw [] (start) to[out=270,in=125] (n1);
    \draw [] (eta1) to (d);
    \draw [] (d) to[out=305,in=90] (end);
    \draw [] (d) to[out=235,in=90] (f);
    \draw [] (f) to[out=270,in=55] (n1);
    \draw [-] (n1) to (e1);
  \end{tikzpicture}
  \ \raisebox{25pt}{\text{.}}
  \]

  In the following proofs, we also use the following two identities from \cite{kock04}:
  \begin{align}
    (1\*\eta)\nabla &= 1,\\
    \Delta(1\*\epsilon) &= 1.
  \end{align}
  Diagrammatically, this is:
  \[
    \begin{tikzpicture}
    \path   node at (.5,1) (start) {}
    node at (0,1) [eta] (eta1) {}
    node at (.25,.5) [nabla] (n1) {}
    node at (.25,0) (end) {};
    \draw [] (eta1) to[out=270,in=125] (n1);
    \draw [] (start) to[out=270,in=55] (n1);
    \draw [] (n1)   to (end);
  \end{tikzpicture}
  \ \raisebox{15pt}{\text{= }}
  \begin{tikzpicture}
    \path node at (0,1) (start) {}
    node at (0,0) (end) {};
    \draw [-] (start) to (end);
  \end{tikzpicture}
  \ \raisebox{15pt}{\text{=}}
  \begin{tikzpicture}
    \path node at (0,1) (start) {}
    node at (0,.5) [delta] (d1) {}
    node at (-.25,0) (end) {}
    node at (.25,0) [epsilon] (e1) {};
    \draw [] (start) to (d1);
    \draw [] (d1) to[out=305,in=90] (e1);
    \draw [] (d1) to[out=235,in=90] (end);
  \end{tikzpicture}
  \ \raisebox{15pt}{.}
  \]
  Note that when combined with the Frobenius identities, this allows transforms of the following
  types:
  \[
  \begin{tikzpicture}
    \path node at (0,1.5) (s1) {}
    node at (.75,1.5) (s2) {}
    node at (0,1) [delta] (d1) {}
    node at (.5,.5) [nabla] (n1) {}
    node at (0,0) (end) {}
    node at (.5,0) [epsilon] (e1) {};
    \draw [] (s1) to (d1);
    \draw [] (s2) to[out=270,in=55] (n1);
    \draw [] (d1) to[out=235,in=90] (end);
    \draw [] (d1) to[out=305,in=125] (n1);
    \draw [] (n1) to (e1);
  \end{tikzpicture}
  \raisebox{15pt}{$=$}
  \begin{tikzpicture}
    \path node at (0,1.5) (s1) {}
    node at (.5,1.5) (s2) {}
    node at (.25,1) [nabla] (n1) {}
    node at (.25,.5) [delta] (d1) {}
    node at (0,0) (end) {}
    node at (.5,0) [epsilon] (e1) {};
    \draw [] (s1) to[out=270,in=125] (n1);
    \draw [] (s2) to[out=270,in=55] (n1);
    \draw [] (d1) to[out=235,in=90] (end);
    \draw [] (d1) to[out=305,in=90] (e1);
    \draw [] (n1) to (d1);
  \end{tikzpicture}
  \raisebox{15pt}{$=$}
  \begin{tikzpicture}
    \path node at (0,1.5) (s1) {}
    node at (.5,1.5)  (s2) {}
    node at (.25,1) [nabla] (n1) {}
    node at (.25,0.5) (end) {};
    \draw [] (s1) to[out=270,in=125] (n1);
    \draw [] (s2) to[out=270,in=55] (n1);
    \draw [] (n1) to (end);
  \end{tikzpicture}
  \raisebox{15pt}{ and }
  \begin{tikzpicture}
    \path node at (0,1.5)  [eta](s1) {}
    node at (.75,1.5) (s2) {}
    node at (0,1) [delta] (d1) {}
    node at (.5,.5) [nabla] (n1) {}
    node at (0,0) (end) {}
    node at (.5,0) (e1) {};
    \draw [] (s1) to (d1);
    \draw [] (s2) to[out=270,in=55] (n1);
    \draw [] (d1) to[out=235,in=90] (end);
    \draw [] (d1) to[out=305,in=125] (n1);
    \draw [] (n1) to (e1);
  \end{tikzpicture}
  \raisebox{15pt}{$=$}
  \begin{tikzpicture}
    \path node at (0,1.5)  [eta] (s1) {}
    node at (.5,1.5) (s2) {}
    node at (.25,1) [nabla] (n1) {}
    node at (.25,.5) [delta] (d1) {}
    node at (0,0) (end) {}
    node at (.5,0)  (e1) {};
    \draw [] (s1) to[out=270,in=125] (n1);
    \draw [] (s2) to[out=270,in=55] (n1);
    \draw [] (d1) to[out=235,in=90] (end);
    \draw [] (d1) to[out=305,in=90] (e1);
    \draw [] (n1) to (d1);
  \end{tikzpicture}
  \raisebox{15pt}{$=$}
  \begin{tikzpicture}
    \path node at (.25,1.5) (s1) {}
    node at (.25,1) [delta] (d1) {}
    node at (0,0.5) (end) {}
    node at (.5,0.5) (end1) {};
    \draw [] (s1) to (d1);
    \draw [] (d1) to[out=255,in=90] (end);
    \draw [] (d1) to[out=305,in=90] (end1);
  \end{tikzpicture}
  \raisebox{15pt}{.}
  \]

  First, we must show that $\inv{f}$ is a map in the category, i.e., that $\Delta (\inv{f} \*
  \inv{f}) = \inv{f} \Delta$ and $(\inv{f} \* \inv{f})\nabla = \nabla \inv{f}$. We show this for
  $\Delta$ using string diagrams, starting from $\Delta(\inv{f} \*
  \inv{f})$. The proof for the preservation of $\nabla$ proceeds in a similar manner.
  \[
  \begin{tikzpicture}
    \path node at (0,3) (start) {}
    node at (-.75,2.5) [eta] (eta1) {}
    node at (0,2.5) [delta] (d0) {}
    node at (.75,2.5) [eta] (eta2) {}
    node at (-.75,2) [delta] (d) {}
    node at (.75,2) [delta] (d2) {}
    node at (-.5,1.5) [map] (f) {$\scriptstyle f$}
    node at (.5,1.5) [map] (f2) {$\scriptstyle f$}
    node at (-.25,1) [nabla] (n1) {}
    node at (.25,1) [nabla] (n2) {}
    node at (-.25,.5) [epsilon] (e1) {}
    node at (.25,.5) [epsilon] (e2) {}
    node at (-.75,0) (end) {}
    node at (.75,0) (end2) {};
    \draw [] (start) to (d0);
    \draw [] (eta1) to (d);
    \draw [] (eta2) to (d2);
    \draw (d0) to[out=235,in=55] (n1);
    \draw (d0) to[out=305,in=125] (n2);
    \draw [] (d) to[out=235,in=90] (end);
    \draw [] (d) to[out=305,in=90] (f);
    \draw [] (f) to[out=270,in=125] (n1);
    \draw [-] (n1) to (e1);
    \draw [] (d2) to[out=305,in=90] (end2);
    \draw [] (d2) to[out=235,in=90] (f2);
    \draw [] (f2) to[out=270,in=55] (n2);
    \draw [-] (n2) to (e2);
  \end{tikzpicture}
  \raisebox{45pt}{$=$}
  \begin{tikzpicture}
    \path node at (0,3.5) (start) {}
    node at (-.75,2.5) [eta] (eta1) {}
    node at (.75,3) [eta] (eta2) {}
    node at (-.75,2) [delta] (d) {}
    node at (.75,2.5) [delta] (d2) {}
    node at (-.5,1.5) [map] (f) {$\scriptstyle f$}
    node at (.5,2) [map] (f2) {$\scriptstyle f$}
    node at (-.25,1) [nabla] (n1) {}
    node at (.25,1.5) [nabla] (n2) {}
    node at (-.25,.5) [epsilon] (e1) {}
    node at (-.75,0) (end) {}
    node at (.75,0) (end2) {};
    \draw [] (start) to[out=270,in=125] (n2);
    \draw [] (eta1) to (d);
    \draw [] (eta2) to (d2);
    \draw [] (d) to[out=235,in=90] (end);
    \draw [] (d) to[out=305,in=90] (f);
    \draw [] (f) to[out=270,in=125] (n1);
    \draw [-] (n1) to (e1);
    \draw [] (d2) to[out=305,in=90] (end2);
    \draw [] (d2) to[out=235,in=90] (f2);
    \draw [] (f2) to[out=270,in=55] (n2);
    \draw [-] (n2) to[out=270,in=55] (n1);
  \end{tikzpicture}
  \raisebox{45pt}{$=$}
  \begin{tikzpicture}
    \path node at (-0.25,3.5) (start) {}
    node at (-.75,3) [eta] (eta1) {}
    node at (.75,3) [eta] (eta2) {}
    node at (-.75,2.5) [delta] (d) {}
    node at (.75,2.5) [delta] (d2) {}
    node at (-.5,2) [map] (f) {$\scriptstyle f$}
    node at (.5,2) [map] (f2) {$\scriptstyle f$}
    node at (0,1.5) [nabla] (n1) {}
    node at (0.1,1) [nabla] (n2) {}
    node at (0.1,.5) [epsilon] (e1) {}
    node at (-.75,0) (end) {}
    node at (.5,0) (end2) {};
    \draw [] (start) to[out=270,in=55] (n2);
    \draw [] (eta1) to (d);
    \draw [] (eta2) to (d2);
    \draw [] (d) to[out=235,in=90] (end);
    \draw [] (d) to[out=305,in=90] (f);
    \draw [] (f) to[out=270,in=125] (n1);
    \draw [-] (n1) to[out=270,in=125] (n2);
    \draw [] (d2) to[out=305,in=90] (end2);
    \draw [] (d2) to[out=235,in=90] (f2);
    \draw [] (f2) to[out=270,in=55] (n1);
    \draw [-] (n2) to (e1);
  \end{tikzpicture}
  \raisebox{45pt}{$=$}
  \begin{tikzpicture}
    \path node at (.75,3.5) (start) {}
    node at (-.25,3) [eta] (eta1) {}
    node at (.25,3) [eta] (eta2) {}
    node at (-.25,2.5) [delta] (d) {}
    node at (.25,2.5) [delta] (d2) {}
    node at (0,2) [nabla] (n1) {}
    node at (0,1.5) [map] (f) {$\scriptstyle f$}
    node at (0.25,1) [nabla] (n2) {}
    node at (0.25,.5) [epsilon] (e1) {}
    node at (-.5,0) (end) {}
    node at (.75,0) (end2) {};
    \draw [] (start) to[out=270,in=55] (n2);
    \draw [] (eta1) to (d);
    \draw [] (eta2) to (d2);
    \draw [] (d) to[out=235,in=90] (end);
    \draw [] (d) to[out=305,in=125] (n1);
    \draw [] (d2) to[out=305,in=90] (end2);
    \draw [] (d2) to[out=235,in=55] (n1);
    \draw [-] (n1) to[out=270,in=90] (f);
    \draw [] (f) to[out=270,in=125] (n2);
    \draw [-] (n2) to (e1);
  \end{tikzpicture}
  \raisebox{45pt}{$=$}
  \begin{tikzpicture}
    \path node at (.75,3.5) (start) {}
    node at (.25,3) [eta] (eta2) {}
    node at (.25,2.5) [delta] (d2) {}
    node at (-0.25,2) [delta] (d) {}
    node at (0,1.5) [map] (f) {$\scriptstyle f$}
    node at (0.25,1) [nabla] (n2) {}
    node at (0.25,.5) [epsilon] (e1) {}
    node at (-.5,0) (end) {}
    node at (.5,0) (end2) {};
    \draw [] (start) to[out=270,in=55] (n2);
    \draw [] (eta2) to (d2);
    \draw [] (d2) to[out=305,in=90] (end2);
    \draw [] (d2) to[out=235,in=90] (d);
    \draw [] (d) to[out=235,in=90] (end);
    \draw [] (d) to[out=305,in=90] (f);
    \draw [] (f) to[out=270,in=125] (n2);
    \draw [-] (n2) to (e1);
  \end{tikzpicture}
  \ \raisebox{45pt}{$=$}
  \begin{tikzpicture}
    \path node at (.75,3) (start) {}
    node at (0,2.5) [eta] (eta2) {}
    node at (0,2) [delta] (d2) {}
    node at (-0.25,1.5) [delta] (d) {}
    node at (.25,1.5) [map] (f) {$\scriptstyle f$}
    node at (0.5,1) [nabla] (n2) {}
    node at (0.5,.5) [epsilon] (e1) {}
    node at (-.5,0) (end) {}
    node at (0,0) (exit) {};
    \draw [] (start) to[out=270,in=55] (n2);
    \draw [] (eta2) to (d2);
    \draw [] (d2) to[out=235,in=90] (d);
    \draw [] (d2) to[out=305,in=90] (f);
    \draw [] (d) to[out=235,in=90] (end);
    \draw [] (d) to[out=305,in=90] (exit);
    \draw [] (f) to[out=270,in=125] (n2);
    \draw [-] (n2) to (e1);
  \end{tikzpicture}
  \raisebox{45pt}{$=\inv{f}\Delta$.}
  \]
  Thus, $\inv{f}$ is a map in the category whenever $f$ is.

  If $\inv{f}$ is truly a partial inverse, we may then define $\rst{f} = f \inv{f}$.
  Using Theorem 2.20 from \cite{cockett2002:restcategories1}, we need only show:
  \begin{align}
    \inv{(\inv{f})} &= f\label{eq:finvinv_is_f}\\
    f\inv{f}f &= f\label{eq:ffinvf_is_f}\\
    f\inv{f}g\inv{g} &=g\inv{g} f\inv{f}.\label{eq:ffinv_commutes_gginv}
  \end{align}
  Proof of Equation~\ref{eq:finvinv_is_f}: $\inv{(\inv{f})} =$
  \[
  \begin{tikzpicture}
    \path node at (-.75,4) (start) {}
    node at (0,3.5) [eta] (eta2) {}
    node at (0,3) [delta] (d2) {}
    node at (0,2.5) [eta] (eta1) {}
    node at (0,2) [delta] (d1) {}
    node at (-.25,1.5) [map] (f) {$\scriptstyle f$}
    node at (-.5,1) [nabla] (n1) {}
    node at (-.5,.5) [epsilon] (e1) {}
    node at (-.5,0) [nabla] (n2) {}
    node at (-.5,-.5) [epsilon] (e2) {}
    node at (.25,-1) (end) {};
    \draw [] (start) to[out=270,in=125] (n2);
    \draw [] (eta2) to (d2);
    \draw [] (d2) to[out=235,in=125] (n1);
    \draw [] (d2) to[out=305,in=90] (end);
    \draw [] (eta1) to (d1);
    \draw [] (d1) to[out=305,in=55] (n2);
    \draw [] (d1) to[out=235,in=90] (f);
    \draw [] (f) to[out=270,in=55] (n1);
    \draw [-] (n1) to (e1);
    \draw [-] (n2) to (e2);
  \end{tikzpicture}
  \ \raisebox{70pt}{\text{=}}
  \begin{tikzpicture}
    \path node at (.25,4) (start) {}
    node at (-.5,3.5) [eta] (eta2) {}
    node at (-.5,3) [delta] (d2) {}
    node at (0,2.5) [eta] (eta1) {}
    node at (0,2) [delta] (d1) {}
    node at (-.25,1.5) [map] (f) {$\scriptstyle f$}
    node at (-.5,1) [nabla] (n1) {}
    node at (-.5,.5) [epsilon] (e1) {}
    node at (0,0) [nabla] (n2) {}
    node at (0,-.5) [epsilon] (e2) {}
    node at (-.5,-1) (end) {};
    \draw [] (start) to[out=270,in=55] (n2);
    \draw [] (eta2) to (d2);
    \draw [] (d2) to[out=305,in=125] (n1);
    \draw [] (d2) to[out=235,in=90] (end);
    \draw [] (eta1) to (d1);
    \draw [] (d1) to[out=305,in=125] (n2);
    \draw [] (d1) to[out=235,in=90] (f);
    \draw [] (f) to[out=270,in=55] (n1);
    \draw [-] (n1) to (e1);
    \draw [-] (n2) to (e2);
  \end{tikzpicture}
  \ \raisebox{70pt}{\text{= }}
  \begin{tikzpicture}
    \path node at (.25,4) (start) {}
    node at (-.5,1) [eta] (eta2) {}
    node at (-.5,.5) [delta] (d2) {}
    node at (0,3.5) [eta] (eta1) {}
    node at (0,3) [delta] (d1) {}
    node at (-.25,1.5) [map] (f) {$\scriptstyle f$}
    node at (-.25,0) [nabla] (n1) {}
    node at (-.25,-.5) [epsilon] (e1) {}
    node at (.25,2.5) [nabla] (n2) {}
    node at (.25,2) [epsilon] (e2) {}
    node at (-.5,-1) (end) {};
    \draw [] (start) to[out=270,in=55] (n2);
    \draw [] (eta2) to (d2);
    \draw [] (d2) to[out=305,in=125] (n1);
    \draw [] (d2) to[out=235,in=90] (end);
    \draw [] (eta1) to (d1);
    \draw [] (d1) to[out=305,in=125] (n2);
    \draw [] (d1) to[out=235,in=90] (f);
    \draw [] (f) to[out=270,in=55] (n1);
    \draw [-] (n1) to (e1);
    \draw [-] (n2) to (e2);
  \end{tikzpicture}
  \ \raisebox{70pt}{\text{= }}
  \begin{tikzpicture}
    \path node at (.5,4) (start) {}
    node at (0,3.5) [eta] (eta1) {}
    node at (.25,3) [nabla] (n1) {}
    node at (.25,2.5) [delta] (d1) {}
    node at (.5,2) [epsilon] (e1) {}
    node at (0,1.5) [map] (f) {$\scriptstyle f$}
    node at (-.5,1) [eta] (eta2) {}
    node at (-.25,.5) [nabla] (n2) {}
    node at (-.25,0) [delta] (d2) {}
    node at (0,-.5) [epsilon] (e2) {}
    node at (-.5,-1) (end) {};
    \draw [] (start) to[out=270,in=55] (n1);
    \draw [] (eta1) to[out=270,in=125] (n1);
    \draw [] (n1) to (d1);
    \draw [] (d1) to[out=305,in=90] (e1);
    \draw [] (d1) to[out=235,in=90] (f);
    \draw [] (f) to[out=270,in=55] (n2);
    \draw [] (eta2) to[out=270,in=125] (n2);
    \draw [-] (n2) to (d2);
    \draw [] (d2) to[out=305,in=125] (e2);
    \draw [] (d2) to[out=235,in=90] (end);
  \end{tikzpicture}
  \ \raisebox{70pt}{\text{=}}
  \begin{tikzpicture}
    \path node at (.5,4) (start) {}
    node at (0,1.5) [map] (f) {$\scriptstyle f$}
    node at (-.5,-1) (end) {};
    \draw [] (start) to[out=270,in=90] (f);
    \draw [] (f) to[out=270,in=90] (end);
  \end{tikzpicture}
  \ \raisebox{70pt}{\text{= }$f$.}
  \]
  Proof of Equation~\ref{eq:ffinvf_is_f}: $f \inv{f} f =$
  \[
  \begin{tikzpicture}
    \path node at (-.5,3) (start) {}
    node at (0,2.5) [eta] (eta1) {}
    node at (0,2) [delta] (d) {}
    node at (-.75,1.5) [map] (f1) {$\scriptstyle f$}
    node at (-.25,1.5) [map] (f2) {$\scriptstyle f$}
    node at (.25,1.5) [map] (f3) {$\scriptstyle f$}
    node at (-.5,1) [nabla] (n1) {}
    node at (-.5,.5) [epsilon] (e1) {}
    node at (0,0) (end) {};
    \draw [] (start) to[out=270,in=90] (f1);
    \draw [] (f1) to [out=270,in=125] (n1);
    \draw [] (eta1) to (d);
    \draw [] (d) to[out=305,in=90] (f3);
    \draw [] (f3) to[out=270,in=90] (end);
    \draw [] (d) to[out=235,in=90] (f2);
    \draw [] (f2) to[out=270,in=55] (n1);
    \draw [-] (n1) to (e1);
  \end{tikzpicture}
  \ \raisebox{40pt}{$=$ }
  \begin{tikzpicture}
    \path node at (-.5,3) (start) {}
    node at (0,2.5) [eta] (eta1) {}
    node at (0,2) [delta] (d) {}
    node at (-.5,1.5) [nabla] (n1) {}
    node at (-.5,1) [map] (f2) {$\scriptstyle f$}
    node at (.25,1) [map] (f3) {$\scriptstyle f$}
    node at (-.5,.5) [epsilon] (e1) {}
    node at (0,0) (end) {};
    \draw [] (start) to [out=270,in=125] (n1);
    \draw (n1) to (f2);
    \draw (f2) to (e1);
    \draw [] (eta1) to (d);
    \draw [] (d) to[out=305,in=90] (f3);
    \draw [] (f3) to[out=270,in=90] (end);
    \draw [] (d) to[out=235,in=55] (n1);
  \end{tikzpicture}
  \ \raisebox{40pt}{$=$ }
  \begin{tikzpicture}
    \path node at (-.5,3) (start) {}
    node at (0,2.5) [eta] (eta1) {}
    node at (-.25,2) [nabla] (n1) {}
    node at (-.25,1.5) [delta] (d) {}
    node at (-.5,1) [map] (f2) {$\scriptstyle f$}
    node at (0,1) [map] (f3) {$\scriptstyle f$}
    node at (-.5,.5) [epsilon] (e1) {}
    node at (0,0) (end) {};
    \draw [] (start) to[out=270,in=125] (n1);
    \draw [] (eta1) to[out=270,in=55] (n1);
    \draw [] (n1) to (d);
    \draw [] (d) to[out=305,in=90] (f3);
    \draw [] (d) to[out=235,in=90] (f2);
    \draw (f2) to (e1);
    \draw [] (f3) to[out=270,in=90] (end);
  \end{tikzpicture}
  \ \raisebox{40pt}{\text{= }}
  \begin{tikzpicture}
    \path node at (-.5,3) (start) {}
    node at (-.25,1.5) [map] (f3) {$\scriptstyle f$}
    node at (-.25,1) [delta] (d) {}
    node at (-.5,.5) [epsilon] (e1) {}
    node at (0,0) (end) {};
    \draw [] (start) to[out=270,in=90] (f3);
    \draw [] (f3) to (d);
    \draw [] (d) to[out=305,in=90] (end);
    \draw [] (d) to[out=235,in=90] (e1);
  \end{tikzpicture}
  \ \raisebox{40pt}{\text{= }}
  \begin{tikzpicture}
    \path node at (-.5,3) (start) {}
    node at (-.25,1.5) [map] (f3) {$\scriptstyle f$}
    node at (0,0) (end) {};
    \draw [] (start) to[out=270,in=90] (f3);
    \draw [] (f3) to[out=270,in=90] (end);
  \end{tikzpicture}
  \ \raisebox{40pt}{$= f$.}
  \]
  Proof of Equation~\ref{eq:ffinv_commutes_gginv}:  $f\inv{f}g\inv{g} =$

  \[
  \begin{tikzpicture}
    \path node at (-.5,3) (start) {}
    node at (0,2.5) [eta] (eta1) {}
    node at (0,2) [delta] (d1) {}
    node at (1,2.5) [eta] (eta2) {}
    node at (1,2) [delta] (d2) {}
    node at (-.75,1.5) [map] (f1) {$\scriptstyle f$}
    node at (-.25,1.5) [map] (f2) {$\scriptstyle f$}
    node at (.25,1.5) [map] (g1) {$\scriptstyle g$}
    node at (.75,1.5) [map] (g2) {$\scriptstyle g$}
    node at (-.5,1) [nabla] (n1) {}
    node at (-.5,.5) [epsilon] (e1) {}
    node at (.5,1) [nabla] (n2) {}
    node at (.5,.5) [epsilon] (e2) {}
    node at (.75,0) (end) {};
    \draw [] (start) to[out=270,in=90] (f1);
    \draw [] (eta1) to (d1);
    \draw [] (eta2) to (d2);
    \draw [] (d1) to[out=235,in=90] (f2);
    \draw [] (d1) to[out=305,in=90] (g1);
    \draw [] (d2) to[out=235,in=90] (g2);
    \draw [] (d2) to[out=305,in=90] (end);
    \draw [] (f1) to [out=270,in=125] (n1);
    \draw [] (f2) to[out=270,in=55] (n1);
    \draw [] (g1) to[out=270,in=125] (n2);
    \draw [] (g2) to[out=270,in=55] (n2);
    \draw [-] (n1) to (e1);
    \draw [-] (n2) to (e2);
  \end{tikzpicture}
  \ \raisebox{40pt}{$=$ }
  \begin{tikzpicture}
    \path node at (-.5,3) (start) {}
    node at (0,2.5) [eta] (eta1) {}
    node at (0,2) [delta] (d1) {}
    node at (1,2.5) [eta] (eta2) {}
    node at (1,2) [delta] (d2) {}
    node at (-.5,1.5) [nabla] (n1) {}
    node at (.5,1.5) [nabla] (n2) {}
    node at (-.5,1) [map] (f1) {$\scriptstyle f$}
    node at (.5,1) [map] (g1) {$\scriptstyle g$}
    node at (-.5,.5) [epsilon] (e1) {}
    node at (.5,.5) [epsilon] (e2) {}
    node at (.75,0) (end) {};
    \draw [] (start) to[out=270,in=125] (n1);
    \draw [] (eta1) to (d1);
    \draw [] (eta2) to (d2);
    \draw [] (d1) to[out=235,in=55] (n1);
    \draw [] (d1) to[out=305,in=125] (n2);
    \draw [] (d2) to[out=235,in=55] (n2);
    \draw [] (d2) to[out=305,in=90] (end);
    \draw [-] (n1) to (f1);
    \draw [-] (n2) to (g1);
    \draw [] (f1) to (e1);
    \draw [] (g1) to (e2);
  \end{tikzpicture}
  \ \raisebox{40pt}{$=$ }
  \begin{tikzpicture}
    \path node at (-.25,3) (start) {}
    node at (0,2.5) [eta] (eta1) {}
    node at (-.25,2) [nabla] (n1) {}
    node at (-.25,1.5) [delta] (d1) {}
    node at (.5,1.5) [eta] (eta2) {}
    node at (-.5,1) [map] (f1) {$\scriptstyle f$}
    node at (.25,1) [nabla] (n2) {}
    node at (-.5,.5) [epsilon] (e1) {}
    node at (.25,.5) [delta] (d2) {}
    node at (0,0) [map] (g1) {$\scriptstyle g$}
    node at (0,-.5) [epsilon] (e2) {}
    node at (.25,-1) (end) {};
    \draw [] (start) to[out=270,in=125] (n1);
    \draw [] (eta1) to[out=270,in=55] (n1);
    \draw [] (n1) to (d1);
    \draw [] (eta2) to[out=270,in=55] (n2);
    \draw [] (d1) to[out=235,in=90] (f1);
    \draw [] (d1) to[out=305,in=125] (n2);
    \draw [] (f1) to (e1);
    \draw [] (n2) to (d2);
    \draw [] (d2) to[out=235,in=90] (g1);
    \draw [] (d2) to[out=305,in=90] (end);
    \draw [] (g1) to (e2);
  \end{tikzpicture}
  \ \raisebox{40pt}{$=$ }
  \begin{tikzpicture}
    \path node at (-.25,2.5) (start) {}
    node at (-.25,1.5) [delta] (d1) {}
    node at (-.5,.5) [map] (f1) {$\scriptstyle f$}
    node at (-.5,0) [epsilon] (e1) {}
    node at (.25,1) [delta] (d2) {}
    node at (0,.5) [map] (g1) {$\scriptstyle g$}
    node at (0,0) [epsilon] (e2) {}
    node at (.25,-.5) (end) {};
    \draw [] (start) to (d1);
    \draw [] (d1) to[out=235,in=90] (f1);
    \draw [] (d1) to[out=305,in=90] (d2);
    \draw [] (f1) to (e1);
    \draw [] (d2) to[out=235,in=90] (g1);
    \draw [] (d2) to[out=305,in=90] (end);
    \draw [] (g1) to (e2);
  \end{tikzpicture}
  \ \raisebox{40pt}{$=$ }
  \begin{tikzpicture}
    \path node at (-.25,2.5) (start) {}
    node at (.25,2) [delta] (d2) {}
    node at (-.25,1.5) [delta] (d1) {}
    node at (-.5,.5) [map] (f1) {$\scriptstyle f$}
    node at (-.5,0) [epsilon] (e1) {}
    node at (0,.5) [map] (g1) {$\scriptstyle g$}
    node at (0,0) [epsilon] (e2) {}
    node at (.25,-.5) (end) {};
    \draw [] (start) to[out=270,in=90] (d2);
    \draw [] (d2) to[out=235,in=90] (d1);
    \draw [] (d2) to[out=305,in=90] (end);
    \draw [] (d1) to[out=235,in=90] (f1);
    \draw [] (d1) to[out=305,in=90] (g1);
    \draw [] (f1) to (e1);
    \draw [] (g1) to (e2);
  \end{tikzpicture}
  \ \raisebox{40pt}{$=$ }
  \begin{tikzpicture}
    \path node at (-.25,2.5) (start) {}
    node at (.25,2) [delta] (d2) {}
    node at (-.25,1.5) [delta] (d1) {}
    node at (-.5,1) [map] (g1) {$\scriptstyle g$}
    node at (-.5,.5) [epsilon] (e2) {}
    node at (0,1) [map] (f1) {$\scriptstyle f$}
    node at (0,.5) [epsilon] (e1) {}
    node at (.25,0) (end) {};
    \draw [] (start) to[out=270,in=90] (d2);
    \draw [] (d2) to[out=235,in=90] (d1);
    \draw [] (d2) to[out=305,in=90] (end);
    \draw [] (d1) to[out=235,in=90] (g1);
    \draw [] (d1) to[out=305,in=90] (f1);
    \draw [] (f1) to (e1);
    \draw [] (g1) to (e2);
  \end{tikzpicture}
  \ \raisebox{40pt}{$= g\inv{g}f\inv{f}$ }
\]
where the last step is accomplished by reversing all the previous diagrammatic steps.
Hence, \CFrob is an inverse category.
\end{proof}

\begin{theorem}\label{thm:cfrob_is_a_discrete_inverse_category}
  When \X is a symmetric monoidal category, \CFrob is a discrete inverse category.
\end{theorem}
\begin{proof}
  Lemma~\ref{lem:cfrobx_is_an_inverse_category} shows \CFrob is an inverse category. We
  need to show the conditions of Definition~\ref{def:inverse_product} are met.

  First, we see that the tensor of $\X$ is a tensor in \CFrob. $A\*B$ is an object in \CFrob
  with $\Delta_{A\*B} = (\Delta_A\*\Delta_B)(1\*c_{\*}\*1)$,
  $\nabla_{A\*B} =  (1\*c_{\*}\*1)(\nabla_A\*\nabla_B)$,
  $\eta_{A\*B} = \Delta_I(\eta_A \* \eta_B)$, and
  $\epsilon_{A\*B} =  (\epsilon_A\*\epsilon_B)\nabla_I$.

  The map $\Delta : A \to A\*A$ is a map in \CFrob. To show it preserves $\Delta$, we need to
  show $\Delta_A \Delta_{A\*A} = \Delta_A (\Delta_A \* \Delta_A)$:
  \[
  \raisebox{20pt}{$\Delta_A \Delta_{A\*A} =$}
  \begin{tikzpicture}
    \path node at (0,1.5) (start) {}
    node at (0,1) [delta] (d0) {}
    node at (-.25,.5) [delta] (d1) {}
    node at (.25,.5) [delta] (d2) {}
    node at (-.35,0) (e1) {}
    node at (-.15,0) (e2) {}
    node at (.15,0) (e3) {}
    node at (.35,0) (e4) {};
    \draw [] (start) to[out=270,in=90] (d0);
    \draw [] (d0) to[out=235,in=90] (d1);
    \draw [] (d0) to[out=305,in=90] (d2);
    \draw [] (d1) to[out=235,in=90] (e1);
    \draw [] (d1) to[out=305,in=90] (e3);
    \draw [] (d2) to[out=235,in=90] (e2);
    \draw [] (d2) to[out=305,in=90] (e4);
  \end{tikzpicture}
  \raisebox{20pt}{$=$}
  \begin{tikzpicture}
    \path node at (0,2) (start) {}
    node at (-.25,1.5) [delta] (d0) {}
    node at (0,1) [delta] (d1) {}
    node at (.25,.5) [delta] (d2) {}
    node at (-.35,0) (e1) {}
    node at (-.15,0) (e2) {}
    node at (.15,0) (e3) {}
    node at (.35,0) (e4) {};
    \draw [] (start) to[out=270,in=90] (d0);
    \draw [] (d0) to[out=235,in=90] (e1);
    \draw [] (d0) to[out=305,in=90] (d1);
    \draw [] (d1) to[out=235,in=90] (e3);
    \draw [] (d1) to[out=305,in=90] (d2);
    \draw [] (d2) to[out=235,in=90] (e2);
    \draw [] (d2) to[out=305,in=90] (e4);
  \end{tikzpicture}
  \raisebox{20pt}{$=$}
  \begin{tikzpicture}
    \path node at (0,2) (start) {}
    node at (-.25,1.5) [delta] (d0) {}
    node at (.25,1) [delta] (d1) {}
    node at (0,.5) [delta] (d2) {}
    node at (-.35,0) (e1) {}
    node at (-.15,0) (e2) {}
    node at (.15,0) (e3) {}
    node at (.35,0) (e4) {};
    \draw [] (start) to[out=270,in=90] (d0);
    \draw [] (d0) to[out=235,in=90] (e1);
    \draw [] (d0) to[out=305,in=90] (d1);
    \draw [] (d1) to[out=235,in=90] (d2);
    \draw [] (d1) to[out=305,in=90] (e4);
    \draw [] (d2) to[out=235,in=90] (e3);
    \draw [] (d2) to[out=305,in=90] (e2);
  \end{tikzpicture}
  \raisebox{20pt}{$=$}
  \begin{tikzpicture}
    \path node at (0,2) (start) {}
    node at (-.25,1.5) [delta] (d0) {}
    node at (.25,1) [delta] (d1) {}
    node at (0,.5) [delta] (d2) {}
    node at (-.35,0) (e1) {}
    node at (-.15,0) (e2) {}
    node at (.15,0) (e3) {}
    node at (.35,0) (e4) {};
    \draw [] (start) to[out=270,in=90] (d0);
    \draw [] (d0) to[out=235,in=90] (e1);
    \draw [] (d0) to[out=305,in=90] (d1);
    \draw [] (d1) to[out=235,in=90] (d2);
    \draw [] (d1) to[out=305,in=90] (e4);
    \draw [] (d2) to[out=235,in=90] (e2);
    \draw [] (d2) to[out=305,in=90] (e3);
  \end{tikzpicture}
  \raisebox{20pt}{$=\Delta_A (\Delta_A \* \Delta_A).$}
  \]
  Note that in the last step, we simply reverse the various associativity steps used previously.

  To show that $\Delta$ preserves the $\nabla$, we must show that
  $(\Delta_A\*\Delta_A)\nabla_{A\*A} = \nabla_A \Delta_A$. Starting with $(\Delta_A\*\Delta_A)\nabla_{A\*A} =$
  \[
  \begin{tikzpicture}
    \path node at (0,1.5) (s1) {}
    node at (.5,1.5) (s2) {}
    node at (0,1) [delta] (d0) {}
    node at (.5,1) [delta] (d1) {}
    node at (0,.5) [nabla] (n0) {}
    node at (.5,.5) [nabla] (n1) {}
    node at (0,0) (e0) {}
    node at (.5,0) (e1) {};
    \draw [] (s1) to[out=270,in=90] (d0);
    \draw [] (s2) to[out=270,in=90] (d1);
    \draw [] (d0) to[out=235,in=125] (n0);
    \draw [] (d0) to[out=305,in=125] (n1);
    \draw [] (d1) to[out=235,in=55] (n0);
    \draw [] (d1) to[out=305,in=55] (n1);
    \draw [] (n0) to[out=270,in=90] (e0);
    \draw [] (n1) to[out=270,in=90] (e1);
  \end{tikzpicture}
  \raisebox{20pt}{$=$}
  \begin{tikzpicture}
    \path node at (0,2.5) (s1) {}
    node at (.5,2.5) (s2) {}
    node at (.5,2) [delta] (d1) {}
    node at (0,1.5) [nabla] (n0) {}
    node at (0,1) [delta] (d0) {}
    node at (0,.5) [nabla] (n1) {}
    node at (0,0) (e0) {}
    node at (.5,0) (e1) {};
    \draw [] (s1) to[out=270,in=125] (n0);
    \draw [] (s2) to[out=270,in=90] (d1);
    \draw [] (d1) to[out=235,in=55] (n0);
    \draw [] (d1) to[out=305,in=55] (n1);
    \draw [] (n0) to[out=270,in=90] (d0);
    \draw [] (d0) to[out=235,in=125] (n1);
    \draw [] (d0) to[out=305,in=90] (e1);
    \draw [] (n1) to[out=270,in=90] (e0);
  \end{tikzpicture}
  \raisebox{20pt}{$=$}
  \begin{tikzpicture}
    \path node at (0,2.5) (s1) {}
    node at (.25,2.5) (s2) {}
    node at (.25,2) [nabla] (n0) {}
    node at (.25,1.5) [delta] (d1) {}
    node at (0,1) [delta] (d0) {}
    node at (0,.5) [nabla] (n1) {}
    node at (0,0) (e0) {}
    node at (.5,0) (e1) {};
    \draw [] (s1) to[out=270,in=125] (n0);
    \draw [] (s2) to[out=270,in=55] (n0);
    \draw [] (n0) to[out=270,in=90] (d1);
    \draw [] (d1) to[out=235,in=90] (d0);
    \draw [] (d1) to[out=305,in=55] (n1);
    \draw [] (d0) to[out=235,in=125] (n1);
    \draw [] (d0) to[out=305,in=90] (e1);
    \draw [] (n1) to[out=270,in=90] (e0);
  \end{tikzpicture}
  \raisebox{20pt}{$=$}
  \begin{tikzpicture}
    \path node at (0,2.5) (s1) {}
    node at (.25,2.5) (s2) {}
    node at (.25,2) [nabla] (n0) {}
    node at (.25,1.5) [delta] (d1) {}
    node at (0,1) [delta] (d0) {}
    node at (0,.5) [nabla] (n1) {}
    node at (0,0) (e0) {}
    node at (.5,0) (e1) {};
    \draw [] (s1) to[out=270,in=125] (n0);
    \draw [] (s2) to[out=270,in=55] (n0);
    \draw [] (n0) to[out=270,in=90] (d1);
    \draw [] (d1) to[out=235,in=90] (d0);
    \draw [] (d1) to[out=305,in=90] (e1);
    \draw [] (d0) to[out=235,in=125] (n1);
    \draw [] (d0) to[out=305,in=55] (n1);
    \draw [] (n1) to[out=270,in=90] (e0);
  \end{tikzpicture}
  \raisebox{20pt}{$=$}
  \begin{tikzpicture}
    \path node at (0,2.5) (s1) {}
    node at (.5,2.5) (s2) {}
    node at (.25,2) [nabla] (n0) {}
    node at (.25,1.5) [delta] (d1) {}
    node at (0,1) (e0) {}
    node at (.5,1) (e1) {};
    \draw [] (s1) to[out=270,in=125] (n0);
    \draw [] (s2) to[out=270,in=55] (n0);
    \draw [] (n0) to[out=270,in=90] (d1);
    \draw [] (d1) to[out=235,in=90] (e0);
    \draw [] (d1) to[out=305,in=90] (e1);
  \end{tikzpicture}
  \raisebox{20pt}{$= \nabla_A \Delta_A$.}
  \]
  Note that the proof uses the ``special'' property in a non-trivial way.

  Thus, we have a $\Delta$ in \CFrob. As $\nabla = \inv{\Delta}$, the Frobenius requirement for
  the inverse product is immediately fulfilled. Commutativity, cocommutativity, associativity,
  coassociativity and the exchange rule all follow from the properties of the commutative Frobenius
  algebras and therefore \CFrob is a discrete inverse category.
\end{proof}

% section the_category_of_commutative_frobenius_algebras (end)

%%% Local Variables:
%%% mode: latex
%%% TeX-master: "../../phd-thesis"
%%% End:



\section{Completing a discrete inverse category} % (fold)
\label{sec:completing_a_discrete_inverse_category}

The purpose of this section is to prove that the category of discrete inverse categories is
equivalent to the the category of discrete restriction categories. In order to prove this, we show
how to construct a discrete restriction category, \Xt, from a discrete inverse category, \X.

\subsection{Equivalence classes of maps in \hypX} % (fold)
\label{sub:equivalence_classes_of_maps_in_hypx}


\begin{definition}\label{def:xequivalence}
  In a discrete inverse category \X as defined above, the map $f$ is equivalent to $f'$ in \X when
  $\restr{f} = \restr{f'}$ in \X and the below diagram commutes for some map $h$:
  \[
    \xymatrix @C=40pt @R=15pt{
      & & B \* C \ar@{.>}[dr]^{(\Delta\* 1) \, a_{\*}}\\
      && & B \* (B\* C) \ar@{.>}[dd]^{1\* h} \\
      A \ar[uurr]^f \ar[ddrr]_{f'}&&&\\
      && & B \* (B \* C'). \ar@{.>}[dl]^(.4){\ \inv{a_{\*}}\,(\inv{\Delta}\* 1)}\\
      && B\* C'
    }
  \]
\end{definition}

\begin{notation}\label{notn:xequivalence}
  When $f$ is equivalent to $g$ as in Definition~\ref{def:xequivalence} via the mediating map $h$,
  this is written as:
  \[
    f\xequiv{h}g.
  \]
\end{notation}

\begin{lemma}\label{lem:mediating_map_equivalence_is_symmetric_reflexive_and_transitive}
  Definition~\ref{def:xequivalence} gives a symmetric, reflexive equivalence class of maps in \X.
\end{lemma}
\begin{proof}
  \prepprooflist
  \begin{description}
    \itembf{Reflexivity: } Choose $h$ as the identity map.
    \itembf{Symmetry: } Suppose $f\xequiv{h}g$. Then, $\restr{f} = \restr{g}$ and $f k = g$ where
      \[
        k = (\Delta\* 1) \, a_{\*}\, (1\*h)\, \inv{a_{\*}}\,(\inv{\Delta}\* 1).
      \] Applying $\inv{k}$,
      which is
      \[
        (\Delta\* 1) \, a_{\*}\, (1\*\inv{h})\, \inv{a_{\*}}\,(\inv{\Delta}\* 1),
      \]
      we have
      \[
        g \inv{k} = f k \inv{k} = f \restr{k} = \restr{f k} f
        = \restr{g} f = \restr{f} f = f.
      \]

      Thus, $g\xequiv{\inv{h}} f$.

    \itembf{Transitivity: } Suppose $f\xequiv{h} f'$ and $f' \xequiv{k} f''$. Consider the
      compositions of the mediating portions of the equivalences:
      \[
        \ell = ((\Delta \* 1)  a_{\*}  (1 \* h ) \inv{a_{\*}} (\inv{\Delta}\* 1))
          ( (\Delta \* 1) a_{\*}  (1 \* k) \inv{a_{\*}} (\inv{\Delta}\* 1)).
      \]

      In string diagram form, this is:
      \[
      \ \raisebox{45pt}{$\ell=$}\
        \begin{tikzpicture}
          \node at (0.5,4) (s1) {};
          \node at (1,4) (s2) {};
          \node at (0.5,3.5) [delta] (d2) {};
          \node at (1,3) [map] (h) {$\scriptstyle \ h\ $};
          \node at (0.5, 2.5) [nabla] (n2) {};
          \node at (0.5,2) [delta] (d3) {};
          \node at (1,1.5) [map] (k) {$\scriptstyle \ k\ $};
          \node at (0.5, 1) [nabla] (n3) {};
          \node at (0.5,.5) (end1) {};
          \node at (1,.5) (end2) {};
          \draw [] (s1) to (d2);
          \draw [] (s2) to[out=270,in=55] (h);
          \draw [] (d2) to[out=305,in=125] (h);
          \draw [] (d2) to[out=235,in=125] (n2);
          \draw (h) to[out=235,in=55] (n2);
          \draw (h) to[out=305,in=55] (k);
          \draw (n2) to (d3);
          \draw [] (d3) to[out=305,in=125] (k);
          \draw [] (d3) to[out=235,in=125] (n3);
          \draw (k) to[out=235,in=55] (n3);
          \draw (k) to[out=305,in=90] (end2);
          \draw (n3) to (end1);
        \end{tikzpicture}
      \]
      By pasting the diagrams which give the above equivalences, we see that $f \ell = f''$.
      However, $\ell$ is not in the form of a mediating map as presented.

      The claim is that $\ell$ is the actual mediating map for $f$ and $f''$. That is, that we have
      $f(\Delta \* 1)a_{\*}(1 \* \ell)\inv{a_{\*}}(\inv{\Delta}\*1) = f''$. In the interest of some
      brevity, this is shown below with the associativity maps elided from the equations.

      We need to show that $(\Delta \* 1)(1 \* \ell)(\inv{\Delta}\*1) = \ell$. Diagrammatically, the
      left hand side $=$
      \[
        \begin{tikzpicture}
          \node at (0,4.5) (s1) {};
          \node at (.5,4.5) (s2) {};
          \node at (0,4) [delta] (d1) {};
          \node at (0.25,3.5) [delta] (d2) {};
          \node at (.65,3) [map] (h) {$\scriptstyle \ h\ $};
          \node at (0.25, 2.5) [nabla] (n2) {};
          \node at (0.25,2) [delta] (d3) {};
          \node at (.65,1.5) [map] (k) {$\scriptstyle \ k\ $};
          \node at (0.25, 1) [nabla] (n3) {};
          \node at (0, .5) [nabla] (n1) {};
          \node at (0,0) (end1) {};
          \node at (.5,0) (end2) {};
          \draw [] (s1) to (d1);
          \draw (d1) to[out=305,in=90] (d2);
          \draw (d1) to[out=235,in=125] (n1);
          \draw [] (s2) to[out=270,in=55] (h);
          \draw [] (d2) to[out=305,in=125] (h);
          \draw [] (d2) to[out=235,in=125] (n2);
          \draw (h) to[out=235,in=55] (n2);
          \draw (h) to[out=305,in=55] (k);
          \draw (n2) to (d3);
          \draw [] (d3) to[out=305,in=125] (k);
          \draw [] (d3) to[out=235,in=125] (n3);
          \draw (k) to[out=235,in=55] (n3);
          \draw (k) to[out=305,in=90] (end2);
          \draw (n3) to[out=270,in=55] (n1);
          \draw (n1) to (end1);
        \end{tikzpicture}
      \ \raisebox{65pt}{$=$}\
        \begin{tikzpicture}
          \node at (0.25,4.5) (s1) {};
          \node at (.5,4.5) (s2) {};
          \node at (0.25,4) [delta] (d1) {};
          \node at (0,3.5) [delta] (d2) {};
          \node at (.65,3) [map] (h) {$\scriptstyle \ h\ $};
          \node at (0.25, 2.5) [nabla] (n2) {};
          \node at (0.25,2) [delta] (d3) {};
          \node at (.65,1.5) [map] (k) {$\scriptstyle \ k\ $};
          \node at (0, 1) [nabla] (n3) {};
          \node at (0.25, .5) [nabla] (n1) {};
          \node at (0.25,0) (end1) {};
          \node at (.5,0) (end2) {};
          \draw [] (s1) to (d1);
          \draw (d1) to[out=235,in=90] (d2);
          \draw (d1) to[out=305,in=125] (h);
          \draw [] (s2) to[out=270,in=55] (h);
          \draw [] (d2) to[out=305,in=125] (n2);
          \draw [] (d2) to[out=235,in=125] (n3);
          \draw (h) to[out=235,in=55] (n2);
          \draw (h) to[out=305,in=55] (k);
          \draw (n2) to (d3);
          \draw [] (d3) to[out=305,in=125] (k);
          \draw [] (d3) to[out=235,in=55] (n3);
          \draw (k) to[out=235,in=55] (n1);
          \draw (k) to[out=305,in=90] (end2);
          \draw (n3) to[out=270,in=125] (n1);
          \draw (n1) to (end1);
        \end{tikzpicture}
      \ \raisebox{65pt}{$=$}\
        \begin{tikzpicture}
          \node at (0.25,4.5) (s1) {};
          \node at (.5,4.5) (s2) {};
          \node at (0.25,4) [delta] (d1) {};
          \node at (.65,3.5) [map] (h) {$\scriptstyle \ h\ $};
          \node at (0,3) [delta] (d2) {};
          \node at (0.25, 2.5) [nabla] (n2) {};
          \node at (0.25,2) [delta] (d3) {};
          \node at (0, 1.5) [nabla] (n3) {};
          \node at (.65,1) [map] (k) {$\scriptstyle \ k\ $};
          \node at (0.25, .5) [nabla] (n1) {};
          \node at (0.25,0) (end1) {};
          \node at (.5,0) (end2) {};
          \draw [] (s1) to (d1);
          \draw (d1) to[out=235,in=90] (d2);
          \draw (d1) to[out=305,in=125] (h);
          \draw [] (s2) to[out=270,in=55] (h);
          \draw [] (d2) to[out=305,in=125] (n2);
          \draw [] (d2) to[out=235,in=125] (n3);
          \draw (h) to[out=235,in=55] (n2);
          \draw (h) to[out=305,in=55] (k);
          \draw (n2) to (d3);
          \draw [] (d3) to[out=305,in=125] (k);
          \draw [] (d3) to[out=235,in=55] (n3);
          \draw (k) to[out=235,in=55] (n1);
          \draw (k) to[out=305,in=90] (end2);
          \draw (n3) to[out=270,in=125] (n1);
          \draw (n1) to (end1);
        \end{tikzpicture}
      \ \raisebox{65pt}{$=$}\
        \begin{tikzpicture}
          \node at (0.25,4.5) (s1) {};
          \node at (.5,4.5) (s2) {};
          \node at (0.25,4) [delta] (d1) {};
          \node at (.65,3.5) [map] (h) {$\scriptstyle \ h\ $};
          \node at (0.25, 3) [nabla] (n2) {};
          \node at (0.25,2.5) [delta] (d2) {};
          \node at (0.25, 2) [nabla] (n3) {};
          \node at (0.25,1.5) [delta] (d3) {};
          \node at (.65,1) [map] (k) {$\scriptstyle \ k\ $};
          \node at (0.25, .5) [nabla] (n1) {};
          \node at (0.25,0) (end1) {};
          \node at (.5,0) (end2) {};
          \draw [] (s1) to (d1);
          \draw (d1) to[out=235,in=125] (n2);
          \draw (d1) to[out=305,in=125] (h);
          \draw [] (s2) to[out=270,in=55] (h);
          \draw (h) to[out=235,in=55] (n2);
          \draw (h) to[out=305,in=55] (k);
          \draw [] (d2) to[out=305,in=55] (n3);
          \draw [] (d2) to[out=235,in=125] (n3);
          \draw (n2) to (d2);
          \draw (n3) to (d3);
          \draw [] (d3) to[out=305,in=125] (k);
          \draw [] (d3) to[out=235,in=125] (n1);
          \draw (k) to[out=235,in=55] (n1);
          \draw (k) to[out=305,in=90] (end2);
          \draw (n1) to (end1);
        \end{tikzpicture}
      \ \raisebox{65pt}{$=$}\
      \raisebox{15pt}{
        \begin{tikzpicture}
          \node at (0.5,4) (s1) {};
          \node at (1,4) (s2) {};
          \node at (0.5,3.5) [delta] (d2) {};
          \node at (1,3) [map] (h) {$\scriptstyle \ h\ $};
          \node at (0.5, 2.5) [nabla] (n2) {};
          \node at (0.5,2) [delta] (d3) {};
          \node at (1,1.5) [map] (k) {$\scriptstyle \ k\ $};
          \node at (0.5, 1) [nabla] (n3) {};
          \node at (0.5,.5) (end1) {};
          \node at (1,.5) (end2) {};
          \draw [] (s1) to (d2);
          \draw [] (s2) to[out=270,in=55] (h);
          \draw [] (d2) to[out=305,in=125] (h);
          \draw [] (d2) to[out=235,in=125] (n2);
          \draw (h) to[out=235,in=55] (n2);
          \draw (h) to[out=305,in=55] (k);
          \draw (n2) to (d3);
          \draw [] (d3) to[out=305,in=125] (k);
          \draw [] (d3) to[out=235,in=125] (n3);
          \draw (k) to[out=235,in=55] (n3);
          \draw (k) to[out=305,in=90] (end2);
          \draw (n3) to (end1);
        \end{tikzpicture}
        }
      \ \raisebox{65pt}{$=\ell.$}
      \]
    \end{description}
    Thus,  $f\xequiv{\ell}f''$.
\end{proof}

\begin{corollary}\label{cor:equivalence_simplified_diagram}
  If $\restr{f} = \restr{g}$ in \X, a discrete inverse category, and the diagram
  \[
    \xymatrix @C=40pt @R=15pt{
      & & B \* C \ar@{.>}[dd]^{1\*h}\\
      A \ar[urr]^f \ar[drr]_{g}\\
      && B\* C'
    }
  \]
  commutes for some $h$, then there is a $h'$ such that $f\xequiv{h'}g$.
\end{corollary}
\begin{proof}
  This follows immediately from
  \[
    \begin{tikzpicture}
      \node at (0,2) (s1) {};
      \node at (.5,2) (s2) {};
      \node at (0,1.5) [delta] (d1) {};
      \node at (.5,1) [map] (h) {$\scriptstyle h$};
      \node at (0,.5) [nabla] (n1) {};
      \node at (0,0) (end1) {};
      \node at (.5,0) (end2) {};
      \draw [] (s1) to (d1);
      \draw [] (s2) to (h);
      \draw [] (d1) to[out=305,in=55] (n1);
      \draw [] (d1) to[out=235,in=125] (n1);
      \draw (h) to (end2);
      \draw (n1) to (end1);
    \end{tikzpicture}
    \ \raisebox{28pt}{$=\ $}
    \begin{tikzpicture}
      \node at (0.25,2) (s1) {};
      \node at (.5,2) (s2) {};
      \node at (.5,1) [map] (h) {$\scriptstyle h$};
       \node at (0.25,0) (end1) {};
      \node at (.5,0) (end2) {};
      \draw [] (s1) to (end1);
      \draw [] (s2) to (h);
      \draw (h) to (end2);
    \end{tikzpicture}
    \ \raisebox{28pt}{$= 1\* h$.}
  \]
  Therefore we can set $h' = 1 \* h$.
\end{proof}
% end_fold sub:equivalence_classes_of_maps_in_hypx

\subsection{The restriction category \hypXt} % (fold)
\label{sub:the_restriction_category_hypxt}

\begin{definition}\label{def:xt}
  When \X is an inverse category, define \Xt\ as:
  \categoryns{objects as in \X;}
  {
    A map $(f,C):A\to B$ in \Xt is the equivalence class of the map $f:A\to B\*C$ as in
    Definition~\ref{def:xequivalence} with the following relationship between maps in \Xt and \X: %
    \[
      \infer{A\xrightarrow{f} B\*C \text{ in }\X}{A \xrightarrow{\ (f,C)\ } B \text{ in } \Xt};
    \]
  }
  {% identity
    by
    \[
      \infer{A\xrightarrow{\inv{u_{\*}^r}}A\* 1}
            {A \xrightarrow{(\inv{u_{\*}^r}, 1)} A};
    \]
  }
  {% composition
    given by
    \[
      \infer{
        \infer{A\xrightarrow{(f (g\*1) a_{\*},C' \* B')} C}
              {A\xrightarrow{f (g\*1) a_{\*}} C \* (C' \* B')}
            }
            {A \xrightarrow{\ (f,B')\ } B \xrightarrow{\ (g,C')\ } C}.
    \]
  }

\end{definition}

When considering an \Xt\ map $(f,C):A\to B$ in \X, we occasionally use the notation $f:A\to
\xtdmn{B}{C}$ ($\equiv f:A\to B\* C$).

\begin{lemma}\label{lem:xt_is_a_category}
  \Xt\ as defined above is a category.
\end{lemma}
\begin{proof}
  The maps are well defined, as shown in
  Lemma~\ref{lem:mediating_map_equivalence_is_symmetric_reflexive_and_transitive}. The existence of
  the identity map is due to the tensor $\*$ being defined on \X, an inverse category, hence
  $\inv{\usr}$ is defined.

  It remains to show the composition is associative and that $(\inv{\usr}, 1)$ acts as an identity
  in \Xt.

  \textbf{Associativity:}
  Consider
  \[
    A\xrightarrow{(f,B')}B\xrightarrow{(g,C')} C \xrightarrow{(h,D')}D.
  \]

  To show the associativity of this in \Xt, we need to show in \X that
  \[
    \restr{(f(g\*1)a_{\*}) (h\*1)a_{\*}} = \restr{f (((g(h\*1)a_{\*})\*1) a_{\*})}
  \]
  and that there exists a mediating map between the two of them.

  To see that the restrictions are equal, first note that by the functorality of $\*$, for any two
  maps $u$ and $v$, we have $u v \* 1 = (u\*1) (v\*1)$. Second, the naturality of $a_{\*}$ gives us
  that $a_{\*} (h \* 1) = ((h\*1)\*1) a_{\*}$. Thus,
  \begin{align*}
    \restr{f(g\*1)a_{\*} (h\*1)a_{\*}}
      & = \restr{f(g\*1)a_{\*} (h\*1)\restr{a_{\*}}}
    & \text{Lemma \ref{lem:restrictionvarious}} \\
    & = \restr{f(g\*1)a_{\*} (h\*1)} & \restr{a_{\*}}=1 \\
    & = \restr{f(g\*1) ((h\*1)\*1) a_{\*} } & a_{\*}\text{ natural} \\
    & = \restr{f(g\*1) ((h\*1)\*1) }
      &  \text{Lemma \ref{lem:restrictionvarious}}\\
    & = \restr{f(g\*1) ((h\*1)\*1) (a_{\*}\*1)}
      & \text{Lemma \ref{lem:restrictionvarious}}\\
    & = \restr{f((g (h\*1)a )\*1)} & \text{ see above}\\
    & = \restr{f((g (h\*1)a )\*1)a_{\*}} & a_{\*}.
  \end{align*}

  For the mediating map, see the diagram below, where the calculation is in \X. The path starting at
  the top left at $A$ and going right to \xtdmn{D}{D' \* (C' \* B')} is grouping parentheses to the
  left. Starting at $A$ and then going down to $\xtdmn{(\xtdmn{D}{D' \* C'})}{B'}$ followed by
  right to $ \xtdmn{D}{(D' \* C') \* B'}$ is grouping parentheses to the right. The
  commutativity of the diagram is shown by the commutativity of the internal portions, which all
  follow from the standard coherence diagrams for the tensor and naturality of association.

  \[
    \xymatrix @C=35pt @R=35pt{
      A \ar[d]_f \ar[r]^(.4){f (g\*1) a_{\*}} \ar[dr]^(.4){f(g\*1)} &
        \xtdmn{C}{C' \* B'} \ar[r]^{h\*1}
        & \xtdmn{(\xtdmn{D}{D'})}{C' \* B'} \ar[r]^{a_{\*}}
        & \xtdmn{D}{D' \* (C' \* B')}
        \ar@{.>}[dd]^{1\*\inv{a_{\*}}}\\
      \xtdmn{B}{B'} \ar[d]_{(g (h\*1) a_{\*})\*1} \ar[r]^{g\*1}
        &\xtdmn{(\xtdmn{C}{C'})}{B'} \ar[u]_{a_{\*}} \ar[r]^(.4){(h\*1)\*1}
        & \xtdmn{(\xtdmn{(\xtdmn{D}{D'})}{C'})}{B'}
        \ar[u]_{a_\*} \ar[dll]_{a_{\*}\*1}
      \\
      \xtdmn{(\xtdmn{D}{D' \* C'})}{B'}  \ar[rrr]_{a_{\*}}
        &&& \xtdmn{D}{(D' \* C') \* B'}. %\ar @/^14pt/ [uu]^{1\*a_{\*}}
    }
  \]
  From this, we can conclude
  \[
    (f(g\*1)a_{\*}) (h\*1)a_{\*} \xequiv{1\*\inv{a_{\*}}} f (((g(h\*1)a_{\*})\*1) a_{\*})
  \]
  which gives us that composition in \Xt is associative.

  \textbf{Identity:} This requires:
  \[
    (f,C) (\inv{\usr}, 1) = (f,C) = (\inv{\usr}, 1) (f,C)
  \]
  for all maps $A\xrightarrow{(f,C)}B$ in \Xt.
  By Lemma~\ref{lem:restrictionvarious} we have
  $\restr{f (\inv{\usr}\*1) a_{\*}} = \restr{f}$. Then, calculating in \X, we have a mediating map
  of $1 \* \usl$ as shown below.
  \[
    \xymatrix @C=50pt @R=45pt{
      A \ar[r]^f \ar[ddrrr]_f&
        B \* C \ar[r]^(.4){\inv{\usr}\*1}
        \ar@/_20pt/[rr]_{1 \* \inv{\usl}}
        \ar@{=}[ddrr]
        & (B \* 1) \* C \ar[r]^{a_{\*}}
        & B \* (1 \* C) \ar@{.>}[dd]^{1 \* \usl} \\
      \\
      &&& B \* C.
    }
  \]
  $\restr{\inv{\usr} (f \*1)  a_{\*}} = \restr{f}$ by the naturality of $\inv{\usr}$ and
  Lemma~\ref{lem:restrictionvarious}. The diagram
  \[
    \xymatrix @C=50pt @R=45pt{
      A \ar@{=}[dd] \ar[r]^{\inv{\usr}} \ar[dr]_{f}
        &      A \* 1 \ar[r]^{f \* 1}
        & (B \* C) \* 1 \ar[r]^{a_{\*}}
        & B \* (C \* 1)\ar@{.>}[dd]^{1 \* \usr} \\
      &B\*C \ar[ur]^{\inv{\usr}} \ar[urr]_{1\*\inv{\usr}} \ar@{=}[drr]\\
      A \ar[rrr]^f &&& B \* C
    }
  \]
  shows our mediating map is $1 \* \usr$.
\end{proof}


Define the restriction in \Xt\ as follows:
\[
  \infer{A\xrightarrow{\restr{f}  \inv{u_{\*}^r}} A\*1 \text{ in }\X.}
        {\infer{A\xrightarrow{\restr{(f,C)}}A}
               {A\xrightarrow{(f,C)}B}
        }
\]

\begin{lemma}\label{lem:xt_is_a_restriction_category}
  The category \Xt with restriction defined as above is a restriction category.
\end{lemma}
\begin{proof}
  Given the above definition the four restriction axioms must now be checked. For the remainder of
  this proof, all diagrams will be in $\X$.

  \rone ($\restr{f} f = f$). Calculating the restriction of the left hand side in
      \X, we have:
      \begin{align*}
        \restr{\rst{f}\inv{\usr} (f\*1) a_{\*}} & = \restr{\rst{f}\inv{\usr} (f\*1)}
          & \text{Lemma \ref{lem:restrictionvarious}}\\
        & = \restr{\rst{f}f \inv{\usr}}  & \inv{\usr} \text{ natural}\\
        & = \restr{f \inv{\usr}}  & \text{ \rone in }\X\\
        & = \restr{f } & \text{Lemma \ref{lem:restrictionvarious}}.
      \end{align*}

      Then, the following diagram
      \[
        \xymatrix @C=40pt @R=25pt{
          A \ar[r]^{\restr{f} \inv{\usr}}
          \ar @/_25pt/[ddrrr]_f  \ar[drr]^{\restr{f}f}
          &A \* 1 \ar[r]^{f\* 1}
          &(A \* B) \* 1 \ar[r]^{a_{\*}} \ar[ddr]^{\usr}
          &A \* (B \* 1) \ar@{.>}[dd]^{1 \* {\usr}}\\
          &&A\*B \ar[u]^{\inv{\usr}} \ar@{=} [dr]\\
          && &A \* B
        }
      \]
      shows $\rst{f}\inv{\usr} (f\*1) a_{\*} \xequiv{1 \* {\usr}} f$ in \X and therefore $\rst{f}f
      = f$ in \Xt.

     \rtwo ($\restr{g} \restr{f} = \restr{f} \restr{g}$). We must show
      \begin{equation}
        \rst{f}\inv{\usr} ((\restr{g}\inv\usr)\*1)) a_{\*} \xequiv{}
        \rst{g}\inv{\usr}((\restr{f}\inv\usr)\*1) a_{\*}.
        \label{eq:rtwo_in_xt}
      \end{equation}
      The restriction of the left hand
      side equals the restriction of the right hand side as seen below:
      \begin{align*}
        \restr{\rst{f}\inv{\usr} ((\restr{g}\inv\usr)\*1)) a_{\*}}
        & = \restr{\rst{f}(\restr{g}\inv\usr)\inv{\usr} a_{\*}} & \inv{\usr}   \text{ natural}\\
        & = \restr{\rst{g}\restr{f}\inv\usr\inv{\usr} a_{\*}} &  \text{\rtwo in }\X\\
        & = \restr{\rst{g}\inv{\usr}((\restr{f}\inv\usr)\*1) a_{\*}} & \inv{\usr}   \text{natural}.
      \end{align*}

      %\note{!!!!! Above calc works without being under restr, therefore have we not just shown
      %that $\restr{g}  \restr{f} = \restr{f}  \restr{g}$ ????
      %Note that the mediating map is $id$!!!!!!!!!}

      The below diagram commutes by the naturality of $\usr$ and the tensor coherence,
      \[
        \xymatrix @C=48pt @R=55pt{
          A \ar[r]^{\restr{g} \inv{\usr}}
            \ar[dr]_{\restr{f}\restr{g}}^{\restr{g}\restr{f}}
            \ar[d]_{\restr{f}\inv{\usr}}
            &A\*1 \ar[r]^(.4){(\rst{f}\inv{\usr})\*1}
            & (A \*1) \* 1
            \ar[r]^{a_{\*}}  \ar[dl]_{\usr \usr}
            & A \* (1\*1) \ar@{.>}[dd]_{1\*id}\\ %\ar @/^25pt/ @{=}[dd]
          A\*1 \ar[d]_{(\rst{g}\inv{\usr})\*1}
            &A \ar@/_25pt/[ur]_{\inv{\usr}\inv{\usr}} \\
             %\ar@/^25pt/[dl]^(.3){\inv{\usr}\inv{\usr}}\\
          (A\*1)\*1 \ar[rrr]_{a_\*} \ar[ur]^{\usr \usr}
            &&& A \* (1\*1)
        }
      \]
      which allows us to conclude $\rst{f} \rst{g} = \rst{g} \rst{f}$ in \Xt.

    \rthree ($\restr{\restr{f} g} = \restr{f} \restr{g}$ ). We must show
      \begin{equation}
        \restr{(\restr{f} \inv{\usr}) (g\* 1) a_{\*}} \inv{\usr} \xequiv{}
        (\restr{f} \inv{\usr})(\rst{g} \inv{\usr}\* 1) a_{\*}.
        \label{eq:rthree_in_xt}
      \end{equation}
      As above, the first step is
      to show that the restrictions of each side of Equation~\ref{eq:rthree_in_xt} are the same.
      Computing the restriction of the left
      hand side in \X:
      \begin{align*}
        \rst{\restr{(\restr{f} \inv{\usr}) (g\* 1) a_{\*}} \inv{\usr}}
        & = \rst{\restr{(\restr{f} \inv{\usr}) (g\* 1) a_{\*}}} & \text{Lemma \ref{lem:restrictionvarious}}\\
        & = \restr{(\restr{f} \inv{\usr}) (g\* 1) a_{\*}} &
          \text{Lemma \ref{lem:restrictionvarious}}\\
        & = \restr{\restr{f} g \inv{\usr} a_{\*}} & \inv{\usr} \text{ natural}\\
        & = \restr{\restr{f} g } & \text{Lemma \ref{lem:restrictionvarious}}\\
        & = \restr{f} \rst{g} & \text{\rthree in }\X.
      \end{align*}
      The restriction of the right hand side computes in \X as:
      \begin{align*}
        &\rst{(\restr{f} \inv{\usr})(\rst{g} \inv{\usr}\* 1) a_{\*}}\\
        & = \rst{(\restr{f} \inv{\usr}) (\rst{g} \inv{\usr}\* 1) } &  \text{Lemma \ref{lem:restrictionvarious}}\\
        & = \rst{\restr{f}  \rst{g} \inv{\usr}\inv{\usr} } &  \inv{\usr} \text{ natural}\\
        & = \rst{\restr{f}  \rst{g} } &\text{Lemma \ref{lem:restrictionvarious}}\\
        & = \restr{f} \rst{g} & \text{Lemma \ref{lem:restrictionvarious}}.
      \end{align*}

      Additionally, we see $\rst{\rst{f} g}$ in \Xt is expressed in \X as:
      \begin{align*}
        &\restr{(\restr{f} \inv{\usr})(g\* 1) a_{\*}} \inv{\usr} \\
        & = \rst{f} \inv{\usr} \rst{g\* 1} & \text{\rthree, \rfour} \\
        & = \rst{f} \rst{g} \inv{\usr} & \*
          \text{a restriction bi-functor, }\inv{\usr}\text{ natural.}
      \end{align*}

      The following diagram in \X follows the right hand side of Equation~\ref{eq:rthree_in_xt}
      with the top curved arrow and the left hand side of Equation~\ref{eq:rthree_in_xt} with the
      bottom curved arrow. Note that we are using that
      $\restr{(\restr{f} \inv{\usr}) (g\* 1) a_{\*}} = \restr{f}\rst{g}$ as shown above.
      \[
        \xymatrix @C=33pt @R=20pt{
          A \ar@/^45pt/[rrrrr]^{\restr{f} \inv{\usr}(\restr{g} \inv{\usr} \* 1)a_{\*}}
            \ar@/_65pt/[dddddrrrrr]_{\restr{f}\, \restr{g}\inv{\usr}}
            \ar[r]^(.7){\restr{f}}
            \ar[dr]_(.7){\restr{f}}
            &A \ar[r]^{\inv{\usr}}
            \ar@{=}[d]
            &A \* 1 \ar[r]^{\restr{g}\*1}
            \ar[dd]_{\usr}
            &A \* 1 \ar[r]^(.41){\inv{\usr}\* 1}
            \ar[ddd]_{\usr}
            &(A \* 1) \* 1 \ar[r]^{a_{\*}}
            \ar[dddd]_{\usr}
            & A \* (1\*1) \ar@{.>}[ddddd]^{1\*\usr}\\
          &A\ar@{=}[dr]\\
          &&A \ar[dr]_{\restr{g}}\\
          &&&A \ar[dr]_{\inv\usr}\\
          &&&&A\*1 \ar@{=}[dr]\\
          &&&&&A\*1.
        }
      \]
      Hence, in \X, $\restr{(\restr{f} \inv{\usr}) (g\* 1) a_{\*}} \inv{\usr} \xequiv{1\*\usr}
      (\restr{f} \inv{\usr}) (\rst{g} \inv{\usr}\* 1) a_{\*}$ and therefore $\restr{\restr{f} g} =
      \restr{f} \restr{g}$ in \Xt.

      \rfour ($f \restr{g} = \restr{f g} f$). We must show
      \begin{equation}
        f (\restr{g} \inv{\usr}\* 1) a_{\*} \xequiv{}
        \rst{f (g \* 1)} \inv{\usr} (f\* 1) a_{\*}.
        \label{eq:rfour_in_xt}
      \end{equation}
      The restriction of the left hand side of Equation~\ref{eq:rfour_in_xt} is:
      \begin{align*}
        \rst{f (\restr{g} \inv{\usr}\* 1) a_{\*}}
          & = \rst{f (\restr{g} \inv{\usr}\* 1)} & \text{Lemma \ref{lem:restrictionvarious}} \\
        & = \rst{f \rst{g} \inv{\usr}} \* \rst{f} & \* \text{ restriction functor}\\
        & = \rst{f \rst{g}} \* \rst{f} & \text{Lemma \ref{lem:restrictionvarious}} \\
        & = \rst{f (\rst{g} \* 1)}
      \end{align*}
      and the restriction of the right hand side of Equation~\ref{eq:rfour_in_xt}  is:
      \begin{align*}
        \rst{\rst{f (g \* 1)} \inv{\usr} (f\* 1) a_{\*}}
          & = \rst{\rst{f (g \* 1)} \inv{\usr} (f\* 1) } &  \text{Lemma \ref{lem:restrictionvarious}}\\
        & = \rst{\rst{f (g \* 1)} f \inv{\usr}  } & \inv{\usr} \text{ natural}\\
        & = \rst{f \rst{(g \* 1)}  \inv{\usr}  } & \text{ \rfour for }\X\\
        & = \rst{f (\rst{g} \* 1)  \inv{\usr}  } & \* \text{ a restriction functor}\\
        & = \rst{f (\rst{g} \* 1)    } & \text{Lemma \ref{lem:restrictionvarious}.}
      \end{align*}
      Computing the right hand side of Equation~\ref{eq:rfour_in_xt} in \X,
      \begin{align*}
        \rst{f(g\* 1)a_{\*}} \inv{\usr} (f\* 1) a_{\*}
          & = \rst{f(g\* 1)} f \inv{\usr} a_{\*} & \inv{\usr}\text{ natural,}\\
        & = f (\rst{g} \* 1) \inv{\usr} a_{\*} & \rfour.
      \end{align*}

      Thus,
      \[
        \xymatrix @C=40pt @R=35pt{
          A \ar[r]^f \ar[dr]_{f}
            & B\* C \ar[rr]^{\restr{g} \inv{\usr}\* 1}
            &
            & (B\* 1) \* C \ar[r]^{a_{\*}}
            & B \* (1 \* C)\ar@{.>}[d]^{1\*c_{\*}}\\
          & B\*C \ar[r]_{\rst{g}\*1}
            & B \* C \ar[r]_{\inv{\usr}}
            & (B \* C) \* 1 \ar[r]_{a_{\*}}
            & B \* (C\*1)
        }
      \]
  and hence, \Xt\ is a restriction category.
\end{proof}
% subsection the_restriction_category_hypxt (end)
\subsection{The category \hypXt is a discrete restriction category} % (fold)
\label{sub:the_category_hypxt_is_cartesian}



\begin{lemma}\label{lem:tensor_unit_of_x_is_terminal_object_of_xt}
  The unit of the inverse product in \X is the terminal object in \Xt.
\end{lemma}
\begin{proof}
  The unique map to the terminal object for any object $A$ in \Xt is the equivalence class of maps
  represented by $(\inv{\usl},A)$. For this to be a terminal object, the diagram
  \[
    \xymatrix @C=40pt @R=25pt{
      X \ar[r]^{\restr{(f,C)}} \ar[d]^{(f,C)} & X \ar[r]^{!_X}  &\top  \\
      Y \ar[urr]_{!_Y}
    }
  \]
  must commute for all choices of $f$. Translating this to \X, this is the same as requiring
  \[
    \xymatrix @C=40pt @R=25pt{
      X \ar[r]^{\restr{f}} \ar[d]^{f} & X \ar[r]^{\inv{\usr} }
      & X\*1 \ar[r]^{\inv{\usl}}  &1\*X\*1 \ar@{.>}[dl]_{1\*(\usr f)}  \\
      Y \*C\ar[rr]_{\inv{\usl}} & & 1 \*Y \* C
    }
  \]
  to commute, which is true by \rone and from the coherence diagrams for the inverse product tensor.
\end{proof}

Next,we show that the category \Xt\ has restriction products, given by the action of \wtc on
the $\*$ tensor in \X.

First, define total maps $\pi_0$, $\pi_1$ in \Xt by:
\begin{align}
  \pi_0:\qquad & A \* B \xrightarrow{(1,B)} A, \label{eq:defn:pia}\\
  \pi_1:\qquad & A \* B \xrightarrow{(c_{\*},A)} B. \label{eq:defn:pib}
\end{align}

\begin{definition}\label{def:product_map_in_xt}
  Given a discrete inverse category $\X$, suppose we are given the maps $ Z \xrightarrow{(f,C)} A$
  and $Z \xrightarrow{(g,C')} B$ in $\Xt$. Then define $\<(f,C),(g,C')\>$as
  \begin{equation}
    Z\xrightarrow{(\Delta  ( f \* g )  (1\* c_{\*} \* 1), C\* C')} A \* B\label{eq:defn:fg}
  \end{equation}
  where associativity is assumed as needed. Note that with the associativity maps, this is actually:
  \begin{equation}
    Z\xrightarrow{(\Delta  ( f \* g )  a_{\*} (1\*\inv{a_{\*}})
      (1\* (c_{\*} \* 1)) (1\*a_{\*}) \inv{a_{\*}}, C\* C')} A \* B.\label{eq:defn:fg2}
  \end{equation}
\end{definition}

\begin{lemma}\label{lem:tensor_on_x_is_the_restriction_product_on_xt}
  On \Xt, $\*$ is a restriction product with projections $\pi_0, \pi_1$ and the product of maps
  $f, g$ being $\<f,g\>$.
\end{lemma}
\begin{proof}
  From the definition above, as $1$ and $c_{\*}$ are isomorphisms, the maps $\pi_0, \pi_1$ are
  total.

  In order to show that $\rst{\<f,g\>} = \rst{f}\,\rst{g}$, first reduce the left hand side:
  \begin{align*}
    \rst{\<f,g\>}
      &=\rst{\Delta(f\*g)(1\*c_{\*}\*1)}\inv{\usr}&\text{in }\X, \text{ definition of restriction}\\
    &=\rst{\Delta(f\*g)}\inv{\usr} &\\
    &=\rst{\Delta\rst{(f\*g)}}\inv{\usr} &\text{from Lemma \ref{lem:restrictionvarious}}\\
    &=\rst{\Delta(\rst{f}\*\rst{g})}\inv{\usr} &\*\text{ is a restriction functor}\\
    &=\rst{\rst{f}\,\rst{g}\,\Delta(1\*1)}\inv{\usr}
      &\text{Lemma  \ref{lem:properties_of_delta_and_tensor_in_a_discrete_inverse_category}(\ref{le:deltaefg}) twice}\\
    &=\rst{\rst{f}\,\rst{g}}\inv{\usr} &\text{Lemma  \ref{lem:restrictionvarious}}\\
    &=\rst{f}\,\rst{g}\inv{\usr}  &\text{Lemma  \ref{lem:restrictionvarious}.}\\
  \end{align*}

  Then, the right hand side reduces as:
  \begin{align*}
    \rst{f} \rst{g}
    &= \rst{f}\inv{\usr}(\rst{g}\inv{\usr} \* 1) a_{\*} & \text{in \X by definitions}\\
    &= \rst{f} \rst{g}\inv{\usr}\inv{\usr} a_{\*} &  \inv{\usr}\text{ natural.}
  \end{align*}
  The restriction of the left hand side and the right hand side, in \X, is $\rst{\rst{f} \rst{g}}$.
  This is done by applying Lemma~\ref{lem:restrictionvarious} once on the left and
  thrice on the right.

  Thus, this shows $\rst{\<f,g\>}=\rst{f} \rst{g}$ in \Xt where the mediating map in \X is
  $1\*\usr$.

  Next, to show $\<f,g\> \pi_0 \le f$ (and $\<f,g\> \pi_1 \le g$), it is required to show
  $\rst{\<f,g\>\pi_{0}} f = \<f,g\> \pi_{0}$. Calculating the left side, we see:
  \begin{align*}
    \rst{\<f,g\>\pi_{0}} f &=\rst{\<f,g\>\rst{\pi_{0}}} f &\text{Lemma \ref{lem:restrictionvarious}}\\
    &=\rst{\<f,g\>} f &\pi_{0}\text{ is total}\\
    &=\rst{f}\,\rst{g}\, f&\text{ by above}\\
    &=\rst{g} \rst{f} f & \rtwo\\
    &=\rst{g} f& \rone.
  \end{align*}
  Now, turning to the right hand side:
  \begin{align*}
    \<f,g\>\pi_{0} &= \Delta(f\*g)(1\*c_{\*}\*1) 1 &\text{in \X, by definition.}
  \end{align*}
  To show these are equal in \Xt, we need to first show the restrictions are the same in \X and
  then show there is a mediating map between the images in \X. The restriction of $\rst{g} f$ is
  $\rst{f} \rst{g}$ immediately by \rthree and \rtwo. For the right hand side, calculate in \X:
  \begin{align*}
    \rst{\Delta(f\*g)(1\*c_{\*}\*1)}
      & = \rst{\Delta(f\*g)} & \text{Lemma  \ref{lem:restrictionvarious}}\\
    & = \Delta(f\*g)(\inv{f} \* \inv{g})\inv{\Delta} & \X \text{ is an inverse category}\\
    & = \Delta(\rst{f} \* \rst{g})\inv{\Delta} & \\
    & = \rst{f} \rst{g} \Delta\inv{\Delta}
      & \text{Lemma \ref{lem:properties_of_delta_and_tensor_in_a_discrete_inverse_category}(\ref{le:deltaefg}) twice}\\
    & = \rst{f} \rst{g}.
  \end{align*}

  The diagram below shows the required mediating map.
  \[
    \xymatrix @C=27pt @R=25pt{
      & & & A \* C \ar[dr]^{\Delta\*1}\\
      & & & & A \* A \* C \ar@{.>}[d]^{1\*\Delta}\\
      &Z\ar[uurr]^{f}& & & A \* A\* C \* A \* C\ar@{.>}[d]^{1\*1\*\inv{f}}\\
      Z\ar[ur]^{\rst{g}}\ar[dr]_{\Delta}&& & &  A \* A \*C \* Z \ar@{.>}[d]^{1 \* 1 \* 1 \* g}\\
      &Z\*Z\ar[dr]_{f\* g}& & & A\*A\*C\*B\*C'\ar@{.>}[d]^{1\*1\*c_{\*}\*1}\\
      &&A\*C\*B\* C'\ar[dr]_{1\*c_{\*}\*1}&&A\*A\*B\*C\*C'\ar[dl]^{\inv{\Delta}\*1\*1\*1}\\
      & & & A\*B\*C\*C'.
    }
  \]
\end{proof}

% subsection the_category_hypxt_is_cartesian (end)

At this point, we have shown that \Xt is a restriction category with restriction products. This
leads us to the following theorem:

\begin{theorem}\label{thm:xt_is_a_discrete_crc_when_x_is_an_inverse_category}
  For any inverse category \X, the category \Xt is a discrete restriction category.
\end{theorem}
\begin{proof}
  The fact that \Xt is a Cartesian restriction category is immediate from
  Lemmas~\ref{lem:xt_is_a_category}, \ref{lem:xt_is_a_restriction_category},
  \ref{lem:tensor_unit_of_x_is_terminal_object_of_xt} and
  \ref{lem:tensor_on_x_is_the_restriction_product_on_xt}.

  To show that it is discrete, we need only show that the map $(\Delta \inv{\usr},1)$ is in the
  same equivalence class as $\Xt$'s $\Delta(= \<1,1\> = \<(\inv{\usr},1),(\inv{\usr},1))$. As both
  $\Delta$ and $\inv{\usr}$ are total, the restriction of each side is the same, namely $1$. The
  diagram below uses Corollary \ref{cor:equivalence_simplified_diagram} and shows that the two maps
  are in the same equivalence class.
  \[
    \xymatrix @C=190pt @R=40pt{
      & A \* A \* 1 \ar@{.>}[d]^{\inv{\usr}}\\
      A \ar[ur]^{\Delta\inv{\usr}}
        \ar[r]_{\Delta(\inv{\usr}\*\inv{\usr})(1\*c_{\*}\*1)}& A\*A\*1\*1
    }
  \]
\end{proof}
\subsection{Equivalence of categories} % (fold)
\label{sub:equivalence_of_categories}

This section will show that the category of discrete inverse categories (maps being restriction
functors that preserve the inverse tensor) is equivalent to the category of discrete restriction
categories (maps being the restriction functors which preserve the product). In the following, $\X$
will always be a discrete inverse category, $\D$ and $\B$ will be discrete restriction categories.

We approach the equivalence proof by exhibiting the universal property for discrete inverse
categories for the functor $\Invf$ from discrete restriction categories to discrete inverse
categories. The functor $\Invf$ maps a discrete restriction category to its inverse subcategory and
maps functors between discrete restriction categories to a functor having the same action on the
partial inverses. That is, given $G:\B \to \D$, then:
\begin{align*}
  &\Inv{G}: \Inv{\B} \to \Inv{\D}\\
  &\Inv{G}(A) = G{A}&\text{(all objects of $\D$ are in $Inv(\D)$)}\\
  &\Inv{G}(f) = G(f)&\text{(restriction functors preserve partial inverse)}.
\end{align*}
We continue by showing the $\eta$ and $\varepsilon$ of the universal property are isomorphisms.
First, let $\eta:\X \to \Inv{\Xt}$ be an identity on objects functor. For maps $f$ in \X, $\eta(f)
= (f\inv{\usr},1)$.

Next, consider a functor $F:\X \to \Inv{\D}$ defined as follows:
\begin{description}
  \item{Objects:} $F^{\#}:A \mapsto F(A)$
  \item{Arrows:} $F^{\#}:(f,C) \mapsto F(f)\pi_0$
\end{description}

This allows us to write the diagram:
\begin{equation}
  \xymatrix @C=65pt @R=40pt{
    \X \ar[r]^{\eta} \ar[rd]_{F}& \Inv{\Xt} \ar[d]^{\Inv{F^{\#}}}\\
    &\Inv{\D}.
  }
  \label{dia:universal_property_of_inverse_categories}
\end{equation}
In order to show this is a universal diagram, we proceed with a series of lemmas building to the
result.

\begin{lemma}\label{lem:all_invertible_maps_in_xt_are_of_the_form_f_inv_usr}
  For any discrete inverse category $\X$, all invertible maps $(g,C):A\to B$ in $\Xt$ are in the
  equivalence class of $(f \inv{\usr},1)$ for some $f:A\to B$.
\end{lemma}
\begin{proof}
  As $(g,C)$ is invertible in \Xt, the map $\inv{(g,C)}: B \to A$ exists. The map $\inv{(g,C)}$ must be in
  the equivalence class of some map $k:B \to A \* D$. By construction, the map  $\rst{(g,C)}$ is
  in the equivalence class of the map $\rst{g}\inv{\usr}:A \to A\*1$ in \X. This means,
  diagramming in \X, there is an $n$ such that
  \[
    \xymatrix @C=45pt @R=25pt{
      B \ar[r]^{k} \ar[rrddd]_{\rst{k}\inv{\usr}}
        & A \*D \ar[r]^{f\*1}&B\*C\*D \ar[d]^{\Delta\*1}\\
      & &  B\*B\*C\*D \ar@{.>}[d]^{1\* n} \\
      & &  B\*B\*1 \ar[d]^{\inv{\Delta}\* 1}\\
      && B\* 1
    }
  \]
  commutes.

  Starting with $g:A\to B\*C$, construct the map $f$ in \X with the following diagram:
  \[
    \xymatrix @C=225pt @R=20pt{
      A \ar[r]^{g} \ar@{.>}[rdddddddd]_{f}& B \*C \ar[d]^{\Delta\*1}\\
      &B\*B\*C \ar[d]^{1\*\Delta\*1}\\
      &B\*B\*B\*C \ar[d]^{1\*1\*k\*1}\\
      &B\*B\*A\*D\*C \ar[d]^{1\*1\*g\*1\*1}\\
      &B\*B\*B\*C\*D\*C \ar[d]^{1\*\inv{\Delta}\*1\*c_{\*}}\\
      &B\*B\*C\*C\*D \ar[d]^{1\*1\*\inv{\Delta}\*1}\\
      &B\*B\*C\*D \ar[d]^{1\*n}\\
      &B\*B\*1 \ar[d]^{(\inv{\Delta}\* 1) \usl }\\
      &B.
    }
  \]
  By its construction, $f:A\to B$ in \X and $(f\inv{\usr},1)$ are in the same equivalence class as
  $(g,C)$.

\end{proof}

\begin{lemma}\label{lem:universal_diagram_is_a_commutative_diagram}
  Diagram~\ref{dia:universal_property_of_inverse_categories} above is a commutative diagram.

\end{lemma}
\begin{proof}
  Chasing maps around the diagram, we have:
  \[
    \xymatrix @C=35pt @R=40pt{
      f \ar@{|->}[rr]^{\eta} \ar@{|->}[rd]_{F}&& (f\inv{\usr},1) \ar@{|->}[d]^{\Inv{F^{\#}}}\\
      &F(f) \ar@{=}[r] & F(f\inv{\usr})\pi_0.
    }
  \]
  As $\eta$ is identity on the objects, Diagram~\ref{dia:universal_property_of_inverse_categories}
  commutes.
\end{proof}

\begin{lemma}\label{lem:inv_is_full_and_faithful}
  The functor $\Invf$ from the category of discrete restriction categories to the category of
  discrete inverse categories is full and faithful.
\end{lemma}
\begin{proof}
  To show fullness, we must show $\Invf$ is surjective on hom-sets. Given a functor between two
  categories in the image of $\Invf$, i.e., $G:\Inv{\B}\to \Inv{\D}$, construct a functor
  $H:\B\to\D$ as follows:
  \begin{description}
    \item{Action on objects:} $H(A) = G(A),$
    \item{Objects on maps:} $H(f) = G(\<f,1\>)\pi_0.$
  \end{description}
  $H$ is well defined as we know $\<f,1\>$ is an invertible map and therefore in the domain of $G$.
  To see $H$ is a functor:
  \begin{align*}
    H(1) &= G(\<1,1\>)\pi_0 = \Delta_{\D}\pi_0 = 1,\\
    H(f g) &= G(\<f g,1\>)\pi_0 = G(\<f,1\>)\pi_0G(\<g,1\>)\pi_0 = H(f)H(g).
  \end{align*}
  But on any invertible map, $H(f) = G(\<f,1\>)\pi_0 = \<G(f),1\>\pi_0 = G(f)$ and therefore
  $\Inv{H} = G$, so $\Invf$ is full.

  Next, assume we have $F,G:\B\to\D$ with $\Inv{F} = \Inv{G}$. Considering $F(f)$ and $F(g)$, we
  know $F(\<f,1\>) = G(\<f,1\>) $ as $\<f,1\>$ is invertible. Thus, as the functors preserve the
  product structure, we have
  \[
    F(f) = F(\<f,1\>)F(\pi_0)= G(\<f,1\>)G(\pi_0)= G(f).
  \]
  Thus, $\Invf$ is faithful.
\end{proof}

\begin{corollary}\label{cor:f_sharp_is_unique}
  The functor $F^{\#}$ in Diagram~\ref{dia:universal_property_of_inverse_categories} is unique.
\end{corollary}
\begin{proof}
  This follows immediately from Lemma~\ref{lem:inv_is_full_and_faithful}, $\Invf$ is faithful.
\end{proof}
\begin{corollary}\label{cor:eta_and_xt_are_universal_for_inv}
  The category \Xt and functor $\eta:\X\to \Inv{\Xt}$ is a universal pair for the functor $\Invf$.
\end{corollary}
\begin{proof}
  Immediate from Corollary~\ref{cor:f_sharp_is_unique} and Lemma~\ref{lem:universal_diagram_is_a_commutative_diagram}.
\end{proof}

\begin{lemma}\label{lem:the_functor_eta_is_an_isomorphism}
  The functor $\eta:\X \to \Inv{\Xt}$ is an isomorphism.
\end{lemma}
\begin{proof}
  As $\eta$ is an identity on objects functor, we need only show that it is full and faithful.
  Referring to Lemma~\ref{lem:all_invertible_maps_in_xt_are_of_the_form_f_inv_usr} above, we
  immediately see that $\eta$ is full. For faithful, if we assume $(f\inv{\usr},1)$ is equal in \Xt
  to $(g\inv{\usr},1)$. This means in \X, that $\rst{f} = \rst{g}$ and there is a $h$ such that
  \[
    \xymatrix @C=40pt @R=15pt{
      & & B \* 1 \ar@{.>}[dr]^{(\Delta\* 1) \, a_{\*}}\\
      && & B \* (B\* 1) \ar@{.>}[dd]^{1\* h} \\
      A \ar[uurr]^{f\inv{\usr}} \ar[ddrr]_{g\inv{\usr}}&&&\\
      && & B \* (B \* 1). \ar@{.>}[dl]^(.4){\ \inv{a_{\*}}\,(\inv{\Delta}\* 1)}\\
      && B\* 1
    }
  \]
  This simplifies out to $g = f \Delta (1\* h) \inv{\Delta}$. But by
  Lemma~\ref{lem:properties_of_delta_and_tensor_in_a_discrete_inverse_category}, part \ref{le:restfg},
  $\Delta (1\* h) \inv{\Delta} = \rst{\Delta (1\* h) \inv{\Delta}}$. Setting $\Delta (1\* h)
  \inv{\Delta}$ as $k$, we have $g = f \rst{k}$. This gives us:
  \[
    g = f \rst{k} = \rst{f k} f = \rst{f \rst{k}} f = \rst{g} f = \rst{f} f = f.
  \]
  This shows $\eta$ is faithful and hence an isomorphism between $\X$ and $\Inv{\Xt}$.
\end{proof}

\begin{theorem}\label{thm:discrete_inverse_categories_are_equivalent_to_discrete_restriction_categories}
  The category of discrete inverse categories (objects are discrete inverse categories, maps are
  inverse tensor preserving functors) is equivalent to the category of discrete restriction
  categories (objects are discrete restriction categories, maps are the Cartesian restriction
  functors).
\end{theorem}
\begin{proof}
  From the above lemmas, we have shown that we have an adjoint:
  \begin{equation}
    (\eta,\varepsilon):\wtf \vdash \Invf :D_{i c} \to D_{r c}. \label{eq:inv_and_wtf_are_adjoint_pair}
  \end{equation}
  By Lemma~\ref{lem:the_functor_eta_is_an_isomorphism} we know $\eta$ is an isomorphism. But this
  means the functor $\wtf$ is full and faithful, as shown in, e.g., Proposition 2.2.6 of
  \cite{cockett2009:ctcs}. From lemma \ref{lem:inv_is_full_and_faithful} we know that $\Invf$ is
  full and faithful. But again by the previous reference, this means $\varepsilon$ is an
  isomorphism. Thus, by Corollary \ref{cor:eta_and_xt_are_universal_for_inv} and Proposition 2.2.7
  of \cite{cockett2009:ctcs} we have the equivalence of the two categories.
\end{proof}
% subsection equivalence_of_categories (end)
\subsection{Examples of the \texorpdfstring{\wtc}{tilde} construction} % (fold)
\label{sub:examples_of_the_wtf_construction}

\begin{example}[Completing a finite discrete inverse category]
  \label{example:completing_a_finite_discrete_inverse_category}
\end{example}
Continuing from Example~\ref{example:invprodisstructure}, recall the discrete category of 4
elements with two different tensors. Completing these gives two different lattices. They are either
the straight line lattice, or the diamond semi-lattice. Below are the details of these constructions.

Recall $\D$ has four elements $a,b,c$ and $d$, and there are two possible inverse product tensors,
given in Table~\ref{tab:two_different_inverse_products}. (Repeated here for your convenience).
\begin{table*}[h!]
  \begin{center}
  \begin{tabular}{|l||c|c|c|c|}
    \hline
    $\*$&a&b&c&d\\ \hline \hline
    a&a&a&a&a\\ \hline
    b&a&b&\textbf{b}&b\\ \hline
    c&a&\textbf{b}&c&c \\ \hline
    d&a&b&c&d \\ \hline
  \end{tabular}
  \qquad
  \begin{tabular}{|l||c|c|c|c|} \hline
    $\odot$&a&b&c&d\\ \hline \hline
    a&a&a&a&a\\ \hline
    b&a&b&\textbf{a}&b\\ \hline
    c&a&\textbf{a}&c&c \\ \hline
    d&a&b&c&d \\ \hline
  \end{tabular}
  \end{center}
  \caption[]{Two different inverse products on the same category.} %[] option removes from LOT
\end{table*}


Define $\Delta$ as the identity map. Then, for the first tensor, $\*$ of
Table~\ref{tab:two_different_inverse_products}, $\widetilde{\D}$ has the following
maps
\begin{align*}
  %a \xrightarrow{(id,a)} a \ (\equiv a \xrightarrow{(id, b)} a \equiv a \xrightarrow{(id,c)} a
  %\equiv a \xrightarrow{(id,d)} a), \qquad a \xrightarrow{(id,a)} b, \qquad a \xrightarrow{(id, a)}
  %c , \qquad a \xrightarrow{(id, a)} d \\
  a \xrightarrow{(id,a)\ (\equiv (id, b) \equiv (id,c) \equiv (id,d))} a, \qquad a
    \xrightarrow{(id,a)} b, \qquad a \xrightarrow{(id, a)} c , \qquad a \xrightarrow{(id, a)} d \\
  %b \xrightarrow{(id,b)} b \ (\equiv b \xrightarrow{(id, c)} b \equiv b \xrightarrow{(id, d)} b ),
  %\qquad b \xrightarrow{(id,b)} c, \qquad b \xrightarrow{(id,b)} d\\
  b \xrightarrow{(id,b) \ (\equiv (id, c) \equiv (id, d))} b , \qquad b \xrightarrow{(id,b)} c,
    \qquad b \xrightarrow{(id,b)} d\\
  c \xrightarrow{(id, c) \ (\equiv (id, d))} c ,   \qquad c \xrightarrow{(id,c)} d\\
  d \xrightarrow{(id,d)} d
\end{align*}
resulting in the straight-line ($a \to b \to c \to d$) lattice. The tensor in \D becomes the meet
and hence is a categorical product in $\widetilde{\D}$. Note that the only partial inverses in
$\widetilde{\D}$ are the identity functions and that for all maps $f$, $\<f,1\> = id$.

With the second tensor, $\odot$ from Table~\ref{tab:two_different_inverse_products}, we have:
\begin{align*}
  a \xrightarrow{(id,a) \ (\equiv (id, b) \equiv (id,c) \equiv (id,d))} a, \qquad
    a \xrightarrow{(id,a)} b, \qquad a \xrightarrow{(id, a)} c , \qquad a \xrightarrow{(id, a)} d \\
  b \xrightarrow{(id,b) \ (\equiv (id, d))} b ,  \qquad b \xrightarrow{(id,b)} d\\
  c \xrightarrow{(id, c) \ (\equiv (id, d))} c,   \qquad c \xrightarrow{(id,c)} d\\
  d \xrightarrow{(id,d)} d
\end{align*}
resulting in the ``diamond'' lattice, \raisebox{19pt}{
$
  \xymatrix @R=3pt @C=8pt {
    & b \ar[dr]\\
    a \ar[ur] \ar[dr] & &d\\
    & c \ar[ur]
  }
$}. Once again, the tensor in \D becomes the meet.

\begin{example}[Lattice completion.]\label{example:lattice_completion}
  Suppose we have a set together with an idempotent, commutative, associative operation $\wedge$ on
  the set, giving us a lattice, \Lat. Further suppose the set is partially ordered via $\le$ with
  the order being compatible with $\wedge$.

  Then, we may create a pullback square for any $x' \le x,\ y' \le x$ with
  \[
    \xymatrix{
      &x\\
      x' \ar[ur]_{\le} & & y' \ar[ul]^{\le}\\
      &x'\wedge y'.  \ar[ul]_{\le} \ar[ur]^{\le}
    }
  \]

  Considering \Lat as a category, we see that all maps are monic and therefore, we may create a
  partial map category $\text{Par}(\Lat,\Mstab)$ where the stable system of monics are all the maps.

  Then $\widetilde{\text{Par}(\Lat,\Mstab)}$ becomes the completion of the lattice over $\wedge$.
\end{example}
\begin{example}[$\widetilde{\text{\pinj}}$ is \Par]\label{ex:tilde_pinj_is_par}
  Noting that the objects of both \pinj and \Par are sets, we simply need to show that any partial
  function is in the equivalence class of some $f$, a map in \pinj.

  Suppose we are given $g : A \to B = \{(a,b) | a \in A, b\in B\}$, a partial function in sets. Of
  course, if it is a partial injective function, then $g$ is in the equivalence class of $(g,\{*\})$
  and we are done.  If it is not injective, that means there are one or more elements of $B$ which
  appear in the left hand element of $g$ multiple times. Construct a new function $h$ as follows:
  \begin{equation}
     h \doteq \{(a,(b,a)) | (a,b) \in g\} \label{eq:new_partial_injective}
  \end{equation}
  By its definition, $h : A \to B \*A$ is injective, $(h,A): A \to B$ coincides with $g$ and therefore
  we see that using the \wtc construction on \pinj results in \Par.
\end{example}
% subsection examples_of_the_wtf_construction (end)


% chapter inverse_categories (end)
%%% Local Variables:
%%% mode: latex
%%% TeX-master: "../phd-thesis"
%%% End:
