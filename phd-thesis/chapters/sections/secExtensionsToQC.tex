%!TEX root = /Users/gilesb/UofC/thesis/phd-thesis/phd-thesis.tex
\section{Extensions to quantum circuits}\label{sec:extensionstoqc}
To facilitate the transition to the programming language \lqpl,
this section  introduces three extensions to quantum circuits. The
extensions are  \emph{renaming}, \emph{wire bending and  crossing},
and \emph{scoped control}.
 Each extension adds expressive power
to quantum circuits but does not change the semantic power. For
each of the extensions, examples of how to re-write the extension in
standard quantum circuit terminology will be provided.
\subsection{Renaming}\label{subsec:renaming}
Quantum circuits currently allow renaming to be an implicit part of any
gate. The circuit in \vref{qc:renaming} gives
 an operation to explicitly do this and its rewriting in standard
circuit notation.

\begin{figure}[htbp]
\centerline{%
\Qcircuit @C=1em @R=.7em {
  &\qw_y& \gate{x:=y}  & \qw& \qw_x &\equiv &  &\qw_y& \gate{I} & \qw& \qw_x
}}
\caption{Renaming of a \protect{\qubit} and its equivalent diagram}\label{qc:renaming}
\end{figure}

\subsection{Wire crossing}\label{subsec:extensionwirecrossing}
Crossing and bending of wires in a circuit diagram is added to allow
a simpler presentation of algorithms. The circuit in
\vref{qc:benditbaby} illustrates the concept of re-organizing and
bending of wires.

\begin{figure}[htbp]
\[
\begin{array}{c}
\Qcircuit @C=1em @R=.7em {
  &\qw& \multigate{1}{U_1}&\qw \ar @{-} [ddr]\\
  &\qw& \ghost{U_1}  & \qw & \qw & \qw & \qw & \qw & \qw \\
  &   &              &                & &\qw & \multigate{1}{U_3} &\qw \\
  &\qw& \multigate{1}{U_2}  & \qw \ar @{-} [dr] & &  \qw &\ghost{U_3} & \qw \\
  &\qw& \ghost{U_2}&\qw \ar @{-}  [ur]&&\qw &\qw &\qw
}
\end{array}
\]
\caption{Bending}\label{qc:benditbaby}
\end{figure}



\subsection{Scoped control}
This extension allows us to group different operations in a circuit and
show that all of them are controlled by a particular \qubit. This is
the same as attaching separate control wires to each of the gates in the
grouped operations. Measurements are not affected by control.
\Vref{qc:scopedcontrol} shows a scoped control box on the left which
includes a measure. The right hand side of the same figure
shows the circuit translated back to
standard circuit notation, with the measure not being affected by
the control.

\begin{figure}[htbp]
\[
\begin{array}{ccc}
\Qcircuit @C=1em @R=.7em {
  &\qw& \gate{H}\gategroup{1}{2}{4}{4}{1.25em}{-}  & \gate{Z}& \qw&\qw\\
  &\qw& \qswap  & \qw& \qw&\qw\\
  &\qw& \qswap \qwx  & \qw& \qw&\qw\\
  &\qw& \qw  & \measureD{M}& \cw&\cw\\
  &  &    &      &     &  \\
  &\qw&  \control \ar @{-} [-1,0]+<0ex,.1ex> \qw& \qw&\qw&\qw
}
& \qquad \raisebox{-3em}{$\equiv$} \qquad \qquad &
\Qcircuit @C=1em @R=.7em {
  & \gate{H} &\qw& \gate{Z}& \qw&\qw\\
  &\qw& \qswap  & \qw& \qw&\qw\\
  &\qw& \qswap \qwx  & \qw& \qw&\qw\\
  &\qw& \qw  & \qw&\measureD{M}& \cw\\
  &  &    &      &     &  \\
  &\ctrl{-5}&\ctrl{-3}&  \ctrl{-5}&\qw&\qw
}
\end{array}
\]
\caption{Scope of control}\label{qc:scopedcontrol}
\end{figure}

Scoping boxes correspond to procedures and blocks in \lqpl.

Naturally, both scoping and bending may be combined as in
\ref{qc:allexts}.


\begin{figure}[htbp]
\[
\begin{array}{c}
\Qcircuit @C=1em @R=.7em {
  &\qw& \multigate{1}{U_1}&\qw&\qw&\qw&\multigate{1}{U_3}\gategroup{1}{7}{3}{8}{.8em}{-}  & \gate{Z}& \qw&\qw&\qw\\
  &\qw& \ghost{U_1}  & \multigate{1}{U_2}&\qw \ar @{-}  [dddr]&&\ghost{U_3} &\measureD{M} &\controlo \cw \cwx[1] &\control \cw \cwx[1] &\cw\\
  &\qw& \qw  & \ghost{U_2}&\qw \ar @{-}  [ur]&& & &\nowiregate{U_4} &\gate{U_5} &\qw \\
  &  &    &      &  & & & &   &  \\
  & & & & & & \control \ar @{-} [-2,0]-<0ex,.9ex> \qw &\qw \ar @{-} @/_/ [uur] &
}
\end{array}
\]
\caption{Extensions sample}\label{qc:allexts}
\end{figure}

However, note that exchanging wires is not the same as
swap. Exchanging a pair of wires is not affected by
 control, but a swap is affected by control,
 as shown in \vref{qc:bendisnotswap}.

\begin{figure}[htbp]
\[
\Qcircuit @C=1em @R=1.5em {
&\qw&\qswap  &\qw \\
&\qw&\qswap \qwx &\qw\\
&\qw& \control \qwx \qw & \qw
 } \quad \raisebox{-1.9em}{$\equiv$} \quad
\Qcircuit @C=1em @R=1.5em {
&\qw&\qswap \ghost{U} {\POS"1,2"."2,2"."1,3"."2,3"!C*+<2ex>\frm{-}} &\qw &\raisebox{-4em}{$\neq$}&
&\qw \ar @{-} [dr] & {\hphantom{ }} \gategroup{1}{7}{2}{8}{2ex}{-} &\qw\\
&\qw&\qswap \qwx &\qw & &&\qw \ar @{-}[ur] &&\qw\\
&\qw& \control \ar @{-} [-1,0]-<0ex,1.1ex> \qw &\qw & &
&\qw&\control \ar @{-} [-1,0]-<0ex,1.1ex> \qw &\qw
 }\quad \raisebox{-1.9em}{$\equiv$} \quad
\Qcircuit @C=1em @R=1.5em {
&\qw \ar @{-} [dr] &  &\qw\\
&\qw \ar @{-}[ur] &&\qw\\
&\qw&\qw &\qw
}
\]
\caption{Swap in control vs. exchange in control}\label{qc:bendisnotswap}
\end{figure}

\subsection{Circuit identities}\label{subsec:qcidentites}
Circuits allow the writing of  the
same algorithm in multiple ways. This sub-section will list some
of the circuit identities that hold with the extended notation.

First, note that although a measure may appear inside a
control box, it is not affected by the control, as in
\vref{qc:measurenotaffectedbycontrol}. Conversely, a
measurement commutes with control of a circuit as in
\vref{qc:measurenotaffectedbycontrol}.

\begin{figure}[htbp]
\centerline{%
\Qcircuit @C=1em @R=1.5em {
&\qw&\measureD{M}\gategroup{1}{2}{1}{3}{2ex}{-} &\qw &\raisebox{-4em}{$\equiv$}& &\qw&\measureD{M}& \qw\\
&\qw& \control \ar @{-} [-1,0]-<0ex,2.35ex> \qw &\qw & &&\qw&\qw&\qw
}}
\caption{Measure is not affected by control}\label{qc:measurenotaffectedbycontrol}
\end{figure}


\begin{figure}[htbp]
\centerline{%
\Qcircuit @C=1em @R=1.5em {
&\qw & \measureD{M} & \control \cwx[1] \cw &\cw & & \raisebox{-4em}{$\equiv$}& &\qw &\ctrl{1}  & \measureD{M}& \cw\\
&\qw& \qw & \gate{C_1} & \qw & & & & \qw & \gate{C_1} &\qw &\qw
}}
\caption{Control  is not affected by measure}\label{qc:controlnotaffectedbymeasure}
\end{figure}

One of the notations introduced earlier was that of \emph{0-control}.
This type of control is the same as applying a $Not$ transform before
and after a \emph{1-control}, as shown in \vref{qc:zctrlisonecontrol}.


\begin{figure}[htbp]
\centerline{%
\Qcircuit @C=1em @R=1.5em {
&\qw & \gate{U} & \qw & & \raisebox{-4em}{$\equiv$}& &\qw &\qw & \gate{U} &\qw & \qw \\
&\qw& \ctrlo{-1} & \qw & & & & \qw & \targ & \ctrl{-1} &\targ  &\qw
}}
\caption{Zero control is syntactic sugar}\label{qc:zctrlisonecontrol}
\end{figure}


\Vref{qc:scopedctrleqparallelcontrol} shows
that scoped control of multiple transforms is the same as
controlling those transforms individually.
\begin{figure}[htbp]
\centerline{%
\Qcircuit @C=1em @R=1.5em {
&\qw&\gate{U_1}\gategroup{1}{2}{2}{3}{1ex}{-} &\qw &\raisebox{-4em}{$\equiv$}& &\qw&\gate{U_1}&\qw&\qw\\
&\qw&\gate{U_2}&\qw & &&\qw&\qw&\gate{U_2}&\qw\\
&\qw& \control \ar @{-} [-1,0]-<0ex,2.1ex> \qw &\qw & &&\qw&\ctrl{-2}&\ctrl{-1}&\qw
}}
\caption{Scoped control is parallel control}\label{qc:scopedctrleqparallelcontrol}
\end{figure}
\Ref{qc:scopedctrleqserialcontrol} similarly shows
that scoped control of multiple transforms of the same \qubit{} is
the same as controlling those transforms serially.
\begin{figure}[htbp]
\centerline{%
\Qcircuit @C=1em @R=1.5em {
&\qw&\gate{U_1}\gategroup{1}{2}{1}{4}{1ex}{-} &\gate{U_2}&\qw &\raisebox{-2em}{$\equiv$}& &\qw&\gate{U_1}&\gate{U_2}&\qw\\
&\qw& \control \ar @{-} [-1,0]-<0ex,2.1ex> \qw &\qw &\qw &&&\qw&\ctrl{-1}&\ctrl{-1}&\qw
}}
\caption{Scoped control is serial control}\label{qc:scopedctrleqserialcontrol}
\end{figure}

Multiple control commutes with scoping as shown in
\vrefrange{qc:scopecommutecontrolone}{qc:scopecommutecontroltwo}.
\begin{figure}[htbp]
\centerline{%
\Qcircuit @C=1em @R=1.5em {
&\qw&\gate{U_1}\gategroup{1}{2}{2}{3}{2ex}{-} &\qw &\raisebox{-4em}{$\equiv$}& &\qw&\gate{U_1}&\qw\\
&\qw&\ctrl{-1}&\qw & &&\qw&\ctrl{-1}&\qw\\
&\qw& \control \ar @{-} [-1,0]-<0ex,1.1ex> \qw &\qw & &&\qw&\ctrl{-1}&\qw
}}
\caption{Multiple control}\label{qc:scopecommutecontrolone}
\end{figure}
\begin{figure}[htbp]
\centerline{%
\Qcircuit @C=1em @R=1.5em {
&\qw&\ctrl{1}\gategroup{1}{2}{2}{3}{2ex}{-} &\qw &\raisebox{-4em}{$\equiv$}& &\qw&\control \ar @{-} [1,0]+<0ex,2.5ex> \qw&\qw\\
&\qw&\gate{U_1}&\qw & &&\qw&\gate{U_1}\gategroup{2}{7}{3}{8}{2ex}{-} &\qw\\
&\qw& \control \ar @{-} [-1,0]-<0ex,2.6ex> \qw &\qw & &&\qw&\ctrl{-1}&\qw
}}
\caption{Control scopes commute}\label{qc:scopecommutecontroltwo}
\end{figure}

