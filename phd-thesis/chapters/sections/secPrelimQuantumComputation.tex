%!TEX root = /Users/gilesb/UofC/thesis/phd-thesis/phd-thesis.tex
\section{Basic quantum computation}\label{sec:appBasicsOfQC}
\subsection{Quantum bits}\label{sec:appQuantumBits}
Quantum computation deals with operations on \qubit{}s. A \qubit{} is typically
represented in the literature on quantum computation as a complex
linear combination of \ket{0} and \ket{1}, respectively
identified with (1,0) and (0,1) in $\C^2$. Because of the
identification of the basis vectors, any \qubit{} can be identified
with a non-zero vector in $\C^2$.
In standard quantum computation, the important piece of
 information in a \qubit{} is its direction rather than
amplitude. In other words, given $q=\alpha\ket{0}+\beta\ket{1}$ and
$q'=\alpha'\ket{0}+\beta'\ket{1}$
where $\alpha = \gamma\alpha'$ and
$\beta = \gamma\beta'$, then $q$ and $q'$ represent the same
quantum state.



A \qubit{} that has either $\alpha$ or $\beta$ zero is said to be in
a \emph{classical state}. Any other combination of values is said to be a
\emph{superposition}.

\Vref{sec:QuantumCircuits} will introduce quantum circuits
which act on \qubits{}. This section will have some forward
references to circuits to illustrate points introduced here.

\subsection{Quantum entanglement}\label{sec:appQuantumEntanglement}
Consider what happens when working with a pair of \qubit{}s, $p$ and $q$.
This can be considered as the a vector in $\C^4$ and written as
\begin{equation}
\alpha_{00}\ket{00}+\alpha_{01}\ket{01}+\alpha_{10}\ket{10}+
\alpha_{11}\ket{11}.\label{eq:appTwoqubits}
\end{equation}
In the case where $p$ and $q$
are two independent \qubit{}s, with $p=\alpha\ket{0}+\beta\ket{1}$
and $q=\gamma\ket{0}+\delta\ket{1}$,
\begin{equation}
p \otimes q = \alpha\gamma\ket{00}+\alpha\delta\ket{01}+\beta\gamma\ket{10}+
\beta\delta\ket{11}
\end{equation}
where $p \otimes q$  is the standard tensor product of
$p$ and $q$ regarded as vectors. There are states of two \qubits{}
that cannot be written as a tensor product. As an example, the state
\begin{equation}
\frac{1}{\sqrt{2}}\ket{00} +
\frac{1}{\sqrt{2}}\ket{11}\label{eq:appEntangledqubits}
\end{equation}
is not  a tensor product of two distinct \qubit{}s.
In this case the two \qubit{}s are said to be \emph{entangled}.

\subsection{Quantum gates}\label{sec:appQuantumGates}
\emph{Quantum gates} operate on \qubit{}s. These
gates are conceptually similar to logic gates in the classical world.
 In the
classical world the only non-trivial single
\bit{} gate is the Not gate which
sends $0$ to $1$ and $1$ to $0$. However, there are
infinitely many non-trivial quantum
gates.

An $n{-}\qubit{}$ quantum gate is represented by
a $2^n \times 2^n$ matrix. A
necessary and sufficient condition for such a matrix to be
a quantum gate is that it is \emph{unitary}.

The entanglement of two \qubits{}, $p$ and $q$, is accomplished by applying a
\Had{} transformation to $p$ followed by
a \nottr{} applied to $q$ controlled by $p$.
The circuit in
\vref{qc:appEntangle} shows how to entangle two \qubit{}s that start with
an initial state of $\ket{00}$. See
\vref{fig:stackTeleportation} for how this can be done in \lqpl.

A list of some common gates, together with their usual
quantum circuit representation is given in the next section in
 \vref{tab:qgatesAndRep}.

\subsection{Measurement}\label{sec:appMeasurement}
The  other allowed
operation on a \qubit{} or group of \qubit{}s is
measurement. When a  \qubit{} is measured it
assumes only one of two possible values, either \ket{0} or \ket{1}. Given
\begin{equation}
q=\alpha\ket{0}+\beta\ket{1}\label{eq:appSinglequbit}
\end{equation}
where $|\alpha|^2+|\beta|^2 = 1$, then measuring $q$ will result in
\ket{0} with probability $|\alpha|^2$ and \ket{1} with
probability $|\beta|^2$.
Once a \qubit{} is measured, re-measuring will always produce the same
value.

In multi-\qubit{} systems the order of measurement does not matter.
If $p$ and $q$ are as in \vref{eq:appTwoqubits}, let us suppose measuring $p$
gives  \ket{0}. The measure will result in  that value with probability
$|\alpha_{00}|^2 + |\alpha_{01}|^2$, after which the system
collapses to the state:
\begin{equation}
\alpha_{00}\ket{00} + \alpha_{01}\ket{01}\label{eq:appCollapsedFirstBit}
\end{equation}

Measuring the second \qubit{}, $q$, will give \ket{0} with
probability $|\alpha_{00}|^2$ or \ket{1}
with probability $|\alpha_{01}|^2$.

Conversely, if  $q$ was measured first and gave us \ket{0}
(with a probability of $|\alpha_{00}|^2 + |\alpha_{10}|^2$)
and then $p$ was measured, $p$ will give us \ket{0} with probability
$|\alpha_{00}|^2$ or \ket{1} with probability $|\alpha_{10}|^2$.

Thus, when measuring both $p$ and $q$, the probability of getting \ket{0}
from both measures is $|\alpha_{00}|^2$, regardless of which \qubit{}
is measured first.

Considering states such as in \ref{eq:appEntangledqubits},
measuring either \qubit{} would actually force the other \qubit{} to the
same value. This type of entanglement is used in
many quantum algorithms such as
quantum teleportation.
\subsection{Mixed states}\label{sec:appMixedStates}
The notion of \emph{mixed states} refers to an outside observer's knowledge of
the state of a quantum system. Consider a $1$ \qubit{} system
\begin{equation}
\nu = \alpha\ket{0} + \beta\ket{1}.\label{eq:appOnequbitSystem}
\end{equation}

If $\nu$ is measured but the results of the measurement are not examined,
 the state of the system is either \ket{0} or
\ket{1} and is no longer in a superposition. This type of state
is written as:
\begin{equation}
\nu = |\alpha|^2\{\ket{0}\} + |\beta|^2\{\ket{1}\}.\label{eq:appMixedState}
\end{equation}

An external (to the state) observer knows that the state
of $\nu$ is as expressed in
\vref{eq:appMixedState}. Since the results of the measurement
were not examined, the exact state ($0$ or $1$) is unknown.
Instead, a probability is assigned  as expressed in the equation.
Thus, if the \qubit{} $\nu$ is measured and the results are not
examined, $\nu$ can be treated as a probabilistic \bit{} rather
than a  \qubit.
\subsection{Density matrix notation}\label{sec:appDensityMatrix}
The state of any quantum system of \qubit{}s may be
represented via a \emph{density matrix}. In this
notation, given a \qubit{} $\nu$,  the coefficients
of \ket{0} and \ket{1} form a column vector $u$. Then the
density matrix corresponding to $\nu$ is $uu^{*}$. If
 $\nu=\alpha\ket{0} + \beta\ket{1}$,
\begin{equation}
\begin{singlespace}
\nu = \begin{pmatrix}
\alpha \\
\beta
\end{pmatrix}
\begin{pmatrix}
\overline{\alpha} & \overline{\beta}
\end{pmatrix}
=
\begin{pmatrix}
\alpha\overline{\alpha} & \alpha\overline{\beta}\\
\beta\overline{\alpha} & \beta\overline{\beta}
\end{pmatrix}.
\end{singlespace}
\end{equation}

When working with mixed states  the density matrix of each
component of the mixed state is added. For example, the mixed state shown in
\vref{eq:appMixedState} would be represented by the
density matrix

\begin{equation}
\begin{singlespace}
|\alpha|^2\begin{pmatrix}
1&0 \\
0&0
\end{pmatrix}
+
|\beta|^2\begin{pmatrix}
0&0 \\
0&1
\end{pmatrix}
=
\begin{pmatrix}
|\alpha|^2&0 \\
0&|\beta|^2
\end{pmatrix}.\label{eq:appDensityOfMixedState}
\end{singlespace}
\end{equation}

Note that since the density matrix of mixed states is a linear combination of
other density matrices, it is possible to have two different mixed states
represented by the same density matrix.

 The advantage of this notation is
that it becomes much more compact for mixed state systems. Additionally,
scaling issues are handled by insisting the density matrix has a trace = 1.
During a general quantum computation, as we shall see,
the trace can actually fall below 1 indicating
that the computation is not everywhere total.

\subsubsection{Gates and density matrices}\label{app:subsubsecGatesAndDensityMatrices}
When considering  a \qubit{} $q$ as a column vector and
a unitary  transform $T$ as a matrix, the  result of applying the transform
$T$ to $q$ is the new vector $T q$. The
 density matrix of the original \qubit{} is given by $q\, q^{*}$, while
the density matrix of the transformed \qubit{} is
$(T q) ( T q)^{*}$, which equals $ T (q q^{*}) T^{*}$. Thus, when
a \qubit{} $q$ is represented by a density matrix $A$, the
formula for applying the transform $T$ to $q$ is $T A T^{*}$.

