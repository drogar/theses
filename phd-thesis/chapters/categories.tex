%!TEX root = /Users/gilesb/UofC/thesis/phd-thesis/phd-thesis.tex

\chapter{Category theory}\label{chap:category_theory}
\section{Restriction categories} % (fold)
\label{sec:restriction_categories}


Restriction categories were originally introduced in
 \cite{cockett2002:restcategories1} as an algebraic way to formulate
partial map categories:
\begin{definition}A \emph{restriction category}
is a category \X\ together with a \emph{restriction operator} on
maps:
\[
\infer{\restr{f}:A\to A}{f:A \to B}
\]
where $f$ is an map of \X\ and $A,B$ are objects of \X, such that the
following four \emph{restriction identities} hold, whenever the
compositions\footnote{Note that composition is
written in diagrammatic order throughout this paper.} are defined.
\begin{align*}
&[\text{\emph{\bfseries R.1}}]\ \restr{f} f = f & &
[\text{\emph{\textbf{R.2}}}]\ \restr{g}  \restr{f} = \restr{f}  \restr{g}\\
&[\text{\emph{\textbf{R.3}}}]\ \restr{\restr{f}  g} = \restr{f}   \restr{g} & &
[\text{\emph{\textbf{R.4}}}]\  f \restr{g} = \restr{f g} f
\end{align*}
\end{definition}

\begin{definition}
A \emph{restriction functor} is a functor which preserves the restriction. That is,
given a functor $F: \X \to \Y$ with \X\  and \Y\ restriction categories,
$F$ is a restriction functor if:
\[ F(\restr{f}) = \restr{F(f)}.\]
\end{definition}

Any map such that $r=\restr{r}$ is an idempotent, as $\rst{r}\rst{r} = \rst{\rst{r} r} = \rst{r}$,
 and is called a
\emph{restriction idempotent}. All maps $\restr{f}$ are restriction idempotents as
 $\rst{f}=\rst{\rst{f}}$. Below, we record some basic facts for restriction categories
 shown in
\cite{cockett2002:restcategories1} pp 4-5:
\begin{lemma}\label{lem:identities_involving_restriction}
In a restriction category \X,
\bproofenum
\item{}$\rst{f}$ is idempotent;
\item{} $\rst{f g} = \rst{f g} \, \rst{f}$;\label{lemsub:restriction_identities_two}
\item{} $\rst{f g} = \rst{f \rst{g}}$ ;
\item{} $\rst{\rst{f}} = \rst{f}$;
\item{} $\rst{f}\,\rst{g} = \rst{\rst{f}\,\rst{g}}$;
\item{} $f$ monic implies $\rst{f} = 1$;
\item{} $f = \rst{g} f \implies \rst{g}\,\rst{f} = \rst{f}$.
\eproofenum
\end{lemma}

A map $f:A\to B$ in a restriction category is said to be \emph{total} when
$\rst{f} = 1_A$. The total maps in a restriction category form a subcategory
$Total(\X) \subseteq \X$.

An example of a restriction category is \Par, the category with objects sets and arrows the partial
functions between sets. In \Par, the restriction of $f:A\to B$ is:
\[\rst{f}(x) =
\begin{cases}
  x&\text{if $f(x)$ is defined,}\\
  \uparrow&\text{if }f(x)\text{ is }\uparrow.
\end{cases}
\]
(The symbol $\uparrow$ means that the function is undefined at that element). In \Par, the
total maps correspond precisely to the functions that are defined on all elements of the domain.


\subsection{Enrichment and meets} % (fold)
\label{sub:enrichment_and_meets}


In any restriction category, there is a partial
order on each hom-set, given by $f \le g$ iff
$\restr{f} g = f$, where $f,g:A\to B$.
\begin{lemma}\label{lem:restriction_cats_are_partial_order_enriched}
 In a restriction category \X:
\bproofenum
\item  $\le$ as defined
above is a partial order on each hom-set;
\item $f \le g \implies \restr{f} \le \restr{g}$;
\item $f \le g \implies h f \le h g$;
\item $f \le g \implies f h \le g h$;
\item $f \le 1 \iff f = \restr{f}$.
\eproofenum
\end{lemma}
\begin{proof}
\bproofenum
\item With $f,g,h$ parallel maps in \X, each of the requirements for a
partial order is shown below:
\bd
\itembf{Reflexivity:} $\restr{f} f = f$ and therefore, $ f \le f$.
\itembf{Anti-Symmetry:} Given $\restr{f}g = f$ and $\restr{g}f = g$, it
follows:
\[f = \restr{f} f = \restr{\restr{f} g} f = \restr{f}\, \restr{g} f
  = \restr{g}\restr{f} f =  \restr{g} f = g.\]
\itembf{Transitivity:} Given $f \le g$ and $g\le h$,
\[\restr{f} h = \restr{\restr{f} g} h = \restr{f}\, \restr{g} h = \restr{f} g = f\]
showing that $f \le h$.
\ed
\item The premise is that $\restr{f} g = f$. From this,
$ \restr{f}\, \restr{g} = \restr{\restr{f} g} = \restr{f}$,
showing $\restr{f} \le \restr{g}$.
\item
$\restr{h f} hg = h \restr{f} g = h f$
and therefore $h f \le hg$.
\item $\restr{f} g = f$, this shows
$\restr{f h} g h = \restr{\restr{f} g h} g h
= \restr{f}\, \restr{g h} g h = \restr{f} g h = f h$
and therefore $f h \le g h$.
\item As $f \le 1$ means precisely $\restr{f}1 = f$.
\eproofenum
\end{proof}

Lemma \ref{lem:restriction_cats_are_partial_order_enriched} shows that restriction categories are enriched in
partial orders.

\begin{definition}
A restriction category has \emph{meets} if there is an operation
$\cap$ on parallel maps:
\[
\infer{A\xrightarrow{f\cap g} B}
  {A\overset{f}{\underset{g}{\rightrightarrows}}B}
\]
such that $f\cap g \le f, f\cap g \le g, f\cap f = f,
h (f\cap g) = h f \cap hg$.
\end{definition}

Meets were introduced in \cite{cockett-guo-hofstra-2012:range2}.
Here are some  basic results on meets:

\begin{lemma}\label{lem:properties_of_meets_in_restriction_categories}
In a restriction category \X with meets, where $f, g, h$ are maps in
\X, the following are true:
\bproofenum
\item $f\le g \text{ and } f \le h \iff f \le g\cap h$; \label{lemsub:properties_of_meets_one}
\item $f\cap g = g \cap f$;\label{lemsub:properties_of_meets_two}
\item $\restr{f\cap 1} = f \cap 1$;\label{lemsub:properties_of_meets_three}
\item $(f \cap g) \cap h = f \cap (g \cap h)$;
\item $r(f\cap g) = r f \cap g$ where $r=\rst{r}$ is a restriction idempotent;
\item $(f\cap g)r = f r \cap g$ where $r=\rst{r}$ is a restriction idempotent;
\item $\restr{f\cap g} \le \restr{f}$ (and therefore $\restr{f\cap g} \le \restr{g}$);
\item $ (f \cap 1) f = f \cap 1$;
\item $ e(e \cap 1) = e$ where $e$ is idempotent.
%\item $ e \cap e' = e e'$
\eproofenum
\end{lemma}
\begin{proof}
\bproofenum
\item $f\le g \text{ and } f \le h$ means precisely
$f = \restr{f} g \text{ and } f = \restr{f} h$.
Therefore,
\[\restr{f} (g\cap h) =  \restr{f} g \cap \restr{f} h =  f\cap f = f\]
and so $f \le g \cap h$.
Conversely, given $f \le g\cap h$, we have $f = \restr{f} (g\cap h) =
\restr{f} g \cap \restr{f} h \le \restr{f} g $. But $f \le \restr{f}
g$ means $f = \restr{f}\,\restr{f} g = \restr{f}g$ and therefore $f \le
g$. Similarly, $f \le h$.
\item From \ref{lemsub:properties_of_meets_one}, as by definition, $f\cap g \le g$ and $f \cap g \le f$.

\item $f\cap 1 = \restr{f\cap 1} (f \cap 1)=
(\restr{f \cap 1} f ) \cap (\restr{f \cap 1}) \le \restr{f \cap 1}$ from
which the result follows. %definition of $\le$, \rthree and \rone

\item By definition and transitivity, $(f\cap g)\cap h \le f, g, h$
therefore by \ref{lemsub:properties_of_meets_one} $(f \cap g) \cap h \le
 f \cap (g \cap h)$. Similarly, $f \cap (g \cap h) \le(f \cap g) \cap h$
giving the equality.
\item Given  $r f \cap g \le r f$, calculate:
\[
r f\cap g
= \restr{r f\cap g} r f
= \restr{r (r f\cap g)} f
= \restr{r r f\cap r g} f
= \restr{r (f\cap g)} f
= r \restr{f\cap g} f
= r (f\cap g).\]
\item Using the previous point with the restriction
idempotent $\restr{f r}$,
\[
f r \cap g
= f \restr{r} \cap g   %%r rest id
= \restr{f r }f \cap g  %% R.4
=\restr{f r}(f\cap g)   %Prev
= \restr{f r}\, \restr{f\cap g} f % meet <=
=  \restr{f\cap g}\, \restr{f r} f % R2
=  \restr{f\cap g} f \restr{r}  %R4
= (f\cap g) r.\] % meet, r rest id
\item For the first claim,
\[\restr{f\cap g}\, \restr{f}
=\restr{\restr{f}(f\cap g)}
=\restr{(\restr{f}f)\cap g}
=\restr{f\cap g}.\]
The second claim then follows by \ref{lemsub:properties_of_meets_two}
\item Given $ f \cap 1 \leq f$:
\[ f \cap 1 \leq f \iff  \restr{f \cap 1} f = f \cap 1
\iff  (f \cap 1) f = f \cap 1\]
where the last step is by item \ref{lemsub:properties_of_meets_three} of this lemma.
\item As $e$ is idempotent, $e (e\cap 1) = (e e \cap e) = e$.

\eproofenum
\end{proof}
% subsection enrichment_and_meets (end)
\subsection{Partial monics, sections and isomorphisms} % (fold)
\label{sub:restricted_monics_sections_and_partial_isomorphisms}

Partial isomorphisms play a central role in this paper and below we develop
some their basic properties.

\begin{definition} A map $f$ in a restriction category \X is said:
\bi
\item To be a \emph{partial isomorphism} when there is a
 \emph{partial inverse}, written $\inv{f}$ with $f\inv{f} =\restr{f}$
and $\inv{f}f = \restr{\inv{f}}$;
\item To be a \emph{partial monic} if $h f = k f \implies
h \restr{f} = k \restr{f}$;
\item To be a \emph{partial section} if there exists an  $h$ such that $f h = \restr{f}$;
\item To be a \emph{restriction monic} if it is a section $s$ with a retraction
$r$ such that $r s = \restr{r s}$.
\ei
\end{definition}

\begin{lemma}\label{lem:rcs_partial_monic_section_inverse_properties}
In a restriction category:
\bproofenum
\item $f,\ g$ partial monic implies $f g$ is partial monic;
\item $f$ a partial section implies $f$ is partial monic;
\item $f,\ g$ partial sections implies $f g$ is a partial section;
\item The partial inverse of $f$, when it exists, is unique;
\item If $f,\ g$ have partial inverses and $f\,g$ exists, then $f\,g$ has a partial inverse;
\item A restriction monic $s$ is a partial isomorphism.
\eproofenum
\end{lemma}
\begin{proof}
\bproofenum
\item Suppose $h f g = k f g$. As $g$ is partial monic,
$h f \restr{g} = k f \restr{g}$. Therefore:
\begin{align*}
h \restr{f g} f &= k \restr{f g} f &\rfour\\
h \restr{f g}\,\restr{f} &= k \restr{f g}\, \restr{f} & f\text{
partial monic}\\
h \restr{fg}&= k \restr{fg} & \text{Lemma
\ref{lem:identities_involving_restriction}, \ref{lemsub:restriction_identities_two}}
\end{align*}
\item Suppose $g f = k f$. Then, $g\restr{f} = g f h = k f h = k \restr{f}$.
\item We have $f h = \restr{f}$ and $g h' = \restr{g}$. Therefore,
\begin{align*}
f g h' h &= f \restr{g} h & g \text{ partial section}\\
&= \restr{f g} f h & \rfour\\
&= \restr{f g}\, \restr{f} & f \text{ partial section}\\
&= \restr{f}\, \restr{f g} & \rtwo\\
&= \restr{\restr{f}f g} & \rthree\\
&= \restr{f g} & \rone
\end{align*}
\item Suppose both $\inv{f}$ and $f^*$ are partial inverses of $f$. Then,
\begin{multline*}
\inv{f}
= \restr{\inv{f}}\inv{f} %R.1
=\inv{f}f\inv{f}  %Assumption, inverse is \inv{f}
= \inv{f} \restr{f}   %Assumption, inverse is \inv{f}
= \inv{f} f f^*   %Assumption, inverse is f^*
= \inv{f} f \restr{f^*} f^*  \\ %R.1
= \restr{\inv{f}}\restr{f^*} f^*   %Assumption, inverse is \inv{f}
= \restr{f^*}\restr{\inv{f}} f^* %R.2
= f^* f \restr{\inv{f}}  f^* %Assumption, inverse is f^*
= f^* f \inv{f} f f^* %Assumption, inverse is \inv{f}
= f^* f f^* %Assumption, inverse is \inv{f} and R.1
= f^* %Assumption, inverse is f^*
\end{multline*}
\item For $f:A\to B,\ g:B\to C$ with partial inverses $\inv{f}$ and $\inv{g}$ respectively, the
partial inverse of $f g$ is $\inv{g} \inv{f}$. Calculating $f g \inv{g} \inv{f}$
using all the
restriction identities:
\[f g \inv{g} \inv{f} = f \restr{g} \inv{f} = \restr{f g} f \inv{f} =
\restr{f g}\, \restr{f} =
\restr{f}\, \restr{f g} = \restr{\restr{f} f g} = \restr{f g}.\]
The calculation of $\inv{g} \inv{f} f g = \restr{\inv{g} \inv{f}}$ is similar.
\item The partial inverse of $s$ is $\restr{r s}\,r$. First, note
that $\restr{\restr{r s}\,r}
= \restr{r s}\,\restr{r}
= \restr{r}\, \restr{r s}
= \restr{\restr{r}\,r s}
= \restr{r s}$.
Then, it follows that $(\restr{r s}\,r) s
= r s\,= \restr{r s}
= \restr{\restr{r s}r} $ and
$s (\restr{r s}\,r)
= s r \restr{s} %sr = 1 as r is the retraction of the section s
= \restr{s}$.
\eproofenum
\end{proof}


A restriction category in which every map is a partial
isomorphism is called an \emph{inverse category}.

An interesting property of inverse categories:

\begin{lemma}\label{lem:inverse_idempotents_are_restriction_idempotents}
  In an inverse category, all idempotents are restriction idempotents.
\end{lemma}
\begin{proof}
Given an idempotent $e$,
\[\rst{e} = e\inv{e} = e e \inv{e} = e \rst{e} = \rst{e e} e = \rst{e} e = e.\]
\end{proof}
% subsection restricted_monics_sections_and_partial_isomorphisms (end)
\subsection{Split restriction categories} % (fold)
\label{sub:split_restriction_categories}

The split restriction category,
$\spl{E}{\X}$  is defined as:
\category{$(A,e)$, where $A$ is an object of \X,
$e:A\to A$ and $e\in E$.}
{$f:(A,d)\to(B,e)$ is given by $f:A\to B$ in \X, where $f = d\uts f e$.}
{The map $e$ for $(A,e)$.}
{inherited from \X.}
This is the standard idempotent splitting construction, also known as the Karoubi
envelope.

Note that for $f:(A,d)\to(B,e)$, by definition, in \X we have $f=d\uts f e$, giving
\[d\uts f = d(d\uts f e) = d\uts d\uts f e = d\uts f e =f\ \text{ and }\  f e = (d\uts f e)e = d\uts f e\uts e = d\uts f e = f.\]
When \X is a restriction category, there is an immediate candidate
for a restriction in
$\spl{E}{\X}$. If $f\in\spl{E}{\X}$
is $e_1 f e_2$ in $\X$, then define $\restr{f}$ as given by $e_1 \restr{f}$ in \X. Note that
for $f:(A,d)\to(B,e)$, in \X we have:
\[d\uts\restr{f} = \restr{d\uts f} d = \restr{f} d.\]
\begin{proposition}\label{prop:spleisarestrictioncat}
If \X is a restriction category and $E$ is a set of idempotents, then
the restriction as defined above makes $\spl{E}{\X}$ a restriction category.
\end{proposition}
\begin{proof}
The restriction takes $f:(A,e_1)\to (B,e_2)$ to an endomorphism of $(A,e_1)$. The restriction is in
$\spl{E}{\X}$ as
\[e_1 (e_1\restr{f}) e_1 = e_1 \restr{f} e_1
= \restr{e_1 f} e_1 e_1
= \restr{e_1 f} e_1
= e_1 \restr{ f}.\]

Checking the 4 restriction axioms:
\begin{align*}
&[\text{{\bfseries R.1}}]\  \llbracket\restr{f}f\rrbracket = e_1 \restr{f} f
= e_1 f = \llbracket f\rrbracket\\
&[\text{{\textbf{R.2}}}]\ \llbracket\restr{g}\restr{f}\rrbracket =
e_1\restr{g}  e_1\restr{f}
= e_1 e_1\restr{g}  \restr{f} = e_1 e_1\restr{f}  \restr{g}
= e_1\restr{f}  e_1\restr{g}  = \llbracket\restr{f}\restr{g}\rrbracket\\
&[\text{{\textbf{R.3}}}]\ \llbracket\restr{\restr{f} g } \equiv
e_1 \restr{e_1 \restr{f}  g}
=  \restr{e_1 e_1 \restr{f} g} e_1
=  \restr{e_1 \restr{f} g} e_1
=  e_1 \restr{\restr{f} g}
= e_1 \restr{f}\restr{g}
= e_{1}e_{1}\restr{f}\restr{g}
= e_1 \restr{f}e_1\restr{g}
= \llbracket\restr{f}\, \restr{g}\rrbracket\\
&[\text{{\textbf{R.4}}}]\  \llbracket f \restr{g} \rrbracket =
 e_1f e_2 \restr{g}
= \restr{e_1 f e_2 g} e_1 f e_2
= \restr{e_1 e_1 f e_2 g} e_1 e_1 f e_2
= e_1 \restr{ e_1 f e_2 g} e_1 f e_2
= e_1 \restr{f g} e_1 f e_2
= \llbracket\restr{f g} f\rrbracket
\end{align*}
\end{proof}

Given this, provided all identity maps are in $E$, $\spl{E}{\X}$ is a
restriction category with $\X$ as a full sub-restriction category, via
the embedding defined by taking an object $A$ in \X to
the object $(A,1)$ in $\spl{E}{\X}$.  Furthermore, the property of being
an inverse category is preserved by splitting.

\begin{lemma}\label{lem:the_idempotent_splitting_of_an_inverse_category_is_an_inverse_category}
When \X is an inverse category, $\spl{E}{X}$ is an inverse category.
\end{lemma}
\begin{proof}
The inverse
of $f:(A,e_1)\to(B,e_2)$
in \spl{E}{\X} is $e_2\inv{f}e_1$ as
\[\llbracket f \inv{f} \rrbracket = e_1 f e_2 e_2 \inv{f} e_1
= e_1 e_1 f e_2 \inv{f} e_1
= e_1 f  \inv{f} e_1
= e_1 e_1 \restr{f} e_1
= e_1 \restr{f}
= \llbracket\restr{f}\rrbracket\]
and
\begin{multline*}
  \llbracket \inv{f} f\rrbracket=
e_2 \inv{f} e_1 e_1 f e_2
= e_2 \inv{f} e_1 f e_2 e_2
= e_2 \inv{f} f  e_2\\
= e_2 e_2 \restr{\inv{f}}  e_2
= e_2 \restr{\inv{f}}
= \llbracket\restr{\inv{f}}\rrbracket
\end{multline*}

\end{proof}

\begin{proposition}\label{pro:in_rc_x_with_meets_split_x_is_cong_to_split_r_x}
In a restriction category \X, with meets, let $R$ be the set of restriction idempotents.
Then, $\spl{}{\X} \cong \spl{R}{\X}$
(where \spl{}{\X} is the split of \X over all idempotents). Furthermore,
$\spl{R}{\X}$ has meets.
\end{proposition}
\begin{proof}
The proof below first shows the equivalence of the two categories, then addresses the claim that
$\spl{R}{\X}$ has meets.

For equivalence, we require two functors,
\[U:\spl{R}{\X}\to\spl{}{\X}\text{ and }V:\spl{}{\X}\to\spl{R}{\X},\]
with:
\begin{align}
  U V \cong I_{\spl{R}{\X}}\\
  V U \cong I_{\spl{}{\X}}.
\end{align}


$U$ is the standard inclusion functor. $V$ will take the object $(A,e)$ to
$(A,e\cap 1)$ and the map $f:(A,e_1)\to (B,e_2)$ to $(e_1\cap 1)f $.

$V$ is a functor as:
\bd
\itembf{Well Defined:} If  $f:(A,e_1) \to (B,e_2)$, then
$(e_1\cap 1) f $ is a map in \X from $A$ to $B$ and
$ (e_1\cap 1)(e_1\cap 1) f  (e_2 \cap 1) =
(e_1\cap 1) (f  e_2 \cap f ) = (e_1\cap 1) (f \cap f)= (e_1\cap 1) f$, therefore,
$V(f):V((A,e_1)) \to V((B,e_2))$.
\itembf{Identities:} $V(e) = (e\cap 1 ) e = e \cap 1$ by
lemma \ref{lem:properties_of_meets_in_restriction_categories}.
\itembf{Composition:} $V(f) V(g)
= (e_1\cap 1 ) f (e_2 \cap 1) g
= (e_1\cap 1 ) f e_2 (e_2 \cap 1) g
= (e_1\cap 1 ) f  (e_2 \cap e_{2}) g
= (e_1\cap 1 ) f e_2 g
= (e_1\cap 1 ) f g
= V(f g)$.
\ed

Recalling from Lemma \ref{lem:properties_of_meets_in_restriction_categories}, $(e\cap 1)$ is a restriction
 idempotent. Using this fact, the commutativity of restriction idempotents
and the general idempotent identities from  Lemma \ref{lem:properties_of_meets_in_restriction_categories}, the
composite functor $U V$ is the identity on $\spl{r}{\X}$ as when $e$ is a
restriction idempotent,
$e = e (e\cap 1) = (e\cap 1) e = (e\cap 1)$.

For the other direction,  note that for a particular idempotent $e:A\to A$,
this gives the maps $e:(A,e)\to(A,e\cap 1)$ and
$e\cap 1 : (A,e\cap 1) \to (A,e)$, again by Lemma \ref{lem:properties_of_meets_in_restriction_categories}.
These maps give the natural isomorphism between $I$ and $V U$ as
\[\xymatrix{
(A,e)\ar[r]^e \ar[dr]_{e} &(A,e\cap 1)\ar[d]^{e\cap 1}\\
&(A,e) }\qquad
\text{ and  }\qquad
\xymatrix{
(A,e\cap 1)\ar[r]^{e\cap 1} \ar[dr]_{e\cap 1} &(A,e)\ar[d]^{e}\\
&(A,e\cap 1) }
\]
both commute. Therefore, $U V = I$ and $V U \cong I$, giving
an equivalence of the categories.

For the rest of this proof, the bolded functions, e.g., $\mbf$ are in $\spl{R}{\X}$.
Italic functions, e.g., $f$ are in \X.

To show that $\spl{R}{\X}$ has meets,  designate the meet in $\spl{R}{\X}$ as \capspl
and define $\mbf \capspl \mbg$ as the map given by the \X map $f \cap g$, where
$\mbf,\mbg:(A,d)\to(B,e)$ in $\spl{R}{\X}$ and $f,g:A\to B$ in \X . This is
a map in $\spl{R}{\X}$ as
$d(f \cap g)e = (d\uts f \cap d\uts g) e = (f \cap g) e = (f e \cap g) = f\cap g$
 where the penultimate equality is by Lemma \ref{lem:properties_of_meets_in_restriction_categories}. By definition
 $\restr{\mbf \capspl \mbg }$ is $d\restr{f\cap g}$.

It is necessary to show \capspl satisfies the four meet properties.
\bi
\item{$\mbf\capspl \mbg \le \mbf$: } We need to show
$\rst{\mbf \capspl \mbg} \mbf =  \mbf \capspl \mbg$.  Calculating now in \X:
\begin{align*}
d \rst{f \cap g} f&= \rst{d(f\cap g)} d f\\
& = \rst{df \cap dg} df \\
& = \rst{f \cap g} f \\
& = f \cap g
\end{align*}
which is the definition of $\mbf \capspl \mbg$.
\item{$\mbf\capspl \mbg \le \mbg$: } Similarly and once again calculating in \X,
\begin{align*}
d \rst{f \cap g} g&= \rst{d(f\cap g)} d g\\
& = \rst{df \cap dg} dg \\
& = \rst{f \cap g} g \\
& = f \cap g
\end{align*}
which is the definition of $\mbf \capspl \mbg$.
\item{$\mbf\capspl \mbf = \mbf$: } From the definition, this is $f \cap f = f$ which
is just $ \mbf$.
\item{$\mbh(\mbf\capspl \mbg) = \mbh\mbf \capspl \mbh\mbg$: }
From the definition, this is given in \X by $ h (f \cap g) =
h f \cap h g$ which in $\spl{R}{\X}$ is $\mbh\mbf \capspl \mbh\mbg$.
\ei
\end{proof}
% subsection split_restriction_categories (end)



\subsection{Partial Map Categories} % (fold)
\label{sub:partial_map_categories}

In \cite{cockett2002:restcategories1}, it is shown that
 split restriction categories are equivalent to
\emph{partial map categories}. The main
definitions and results related to partial map categories are given below.

\begin{definition}
A collection $\Mstab$ of monics is \emph{a stable system of monics}
when it includes all isomorphisms, is closed under composition and is
pullback stable.
\end{definition}

\emph{Stable} in this definition means that if $m:A\to B$ is in
\Mstab, then for arbitrary $b$ with codomain $B$, the
pullback
\[\xymatrix{
A'\ar[r]^a \ar[d]_{m'} &A\ar[d]^{m}\\
B' \ar[r]_{b} & B}
\]
exists and $m' \in \Mstab$. A category that has a stable system of monics
is referred to as an \Mstab-category.

\begin{lemma}
If $nm \in \Mstab$, a stable system of monics, and $m$ is monic, then $n \in
\Mstab$.
\end{lemma}
\begin{proof}
The commutative square
\[\xymatrix{
A\ar[d]_n \ar[r]^{1} &A\ar[d]^{nm}\\
A' \ar[r]_{m} & B}
\] is a pullback.
\end{proof}

Given a category \C and a stable system of monics, the
 \emph{partial map category}, $\text{Par}(\C,\Mstab)$ is:
\rcategoryequiv{$A\in\C$}
{$(m,f):A\to B$  with $m:A' \to A$ is
 in \Mstab and $f:A' \to B$ is a map in \C. i.e.,
$\xymatrix @R-15pt @C-15pt{&A'\ar[dl]_{m} \ar[dr]^{f}\\A&&B}$.}
{$1_A,1_A:A \to A$}
{via a pullback, $(m,f)(m',g) = (m'' m, f' g)$ where
 \[\xymatrix @C-15pt @R-15pt{&&A''\ar[dl]_{m''}\ar[dr]^{f'}\\
 &A'\ar[dl]_{m}\ar[dr]_{f}&\text{{\tiny (pb)}}&B'\ar[dl]^{m'}\ar[dr]^{g}\\
A&&B&&C}\]}
{$\restr{(m,f)} = (m,m)$}

For the maps, $(m,f) \sim (m',f')$ when there is an isomorphism $\gamma : A'' \to A'$ such that $\gamma m' = m$
and $\gamma f' = f$.

In \cite{cockettlack2003:restcategories2}, it is shown that:
\begin{theorem}[Cockett-Lack]
Every restriction category is a full subcategory of a partial map category.
\end{theorem}
% subsection partial_map_categories (end)
\subsection{Restriction products and Cartesian restriction categories} % (fold)
\label{sub:restriction_products_and_cartesian_restriction_categories}


Restriction categories have analogues of products and terminal objects.


\begin{definition}
In a restriction category \X\, a \emph{restriction product}  of two objects $X, Y$ is an object
$X\times Y$ equipped with \emph{total} projections $\pi_0:X\times Y\to X,
\pi_1:X\times Y\to Y $ where:
\begin{quote}
$\forall f:Z\to X, g: Z\to Y,\quad \exists$ a unique
$\<f,g\>:Z \to X\times Y$ such that
\bi
\item $\<f,g\> \pi_0 \le f$,
\item $\<f,g\> \pi_1 \le g$ and
\item $\restr{\<f,g\>} = \restr{f}\, \restr{g} ( = \restr{g}\, \restr{f})$.
\ei
\end{quote}
\end{definition}

\begin{definition}
In a restriction category \X\, a \emph{restriction terminal object}
is an object $\top$ such that $\forall X$, there is a
unique total map $!_X : X \to \top$ and the diagram
\[
\xymatrix @C=40pt @R=25pt{
X \ar[r]^{\restr{f}} \ar[d]^{f} & X \ar[r]^{!_X}  &\top  \\
Y \ar[urr]_{!_Y}}
\]
commutes. That is,  $f\, !_Y = \restr{f}\, !_X$. Note this implies
that a restriction terminal object is unique up to a unique isomorphism.
\end{definition}

\begin{definition}
A restriction category \X\ is \emph{Cartesian} if it has all restriction products
and a restriction terminal object.
\end{definition}

\begin{definition}
An object $A$ in a Cartesian restriction category is \emph{discrete}
when the diagonal map,
\[\Delta:A \to A \times A\]
is a partial isomorphism.

A Cartesian restriction category is \emph{discrete} when every object is
discrete.
\end{definition}

\begin{theorem}\label{thm:a_crc_is_discrete_iff_it_has_meets}
A Cartesian restriction category \X is discrete if and only if it has meets.
\end{theorem}
\begin{proof}
If \X has meets, then
\[
\Delta(\pi_0 \cap \pi_1) = \Delta\pi_0 \cap \Delta\pi_1 = 1\cap 1 = 1
\]
and as $\<\pi_0,\pi_1\>$ is identity,
\begin{align*}
\restr{\pi_0 \cap \pi_1} &= \restr{\pi_0 \cap \pi_1} \<\pi_0, \pi_1\> \\
&=\<\rst{\pi_0 \cap \pi_1}\pi_0, \rst{\pi_0 \cap \pi_1}\pi_1\>\\
&=\<\pi_0 \cap \pi_1,\pi_0 \cap \pi_1\>\\
&=(\pi_0 \cap \pi_1 )\Delta
\end{align*}
and therefore, $\pi_0 \cap \pi_1$ is $\inv{\Delta}$.

For the other direction, set $f\cap g = \<f,g\>\inv{\Delta}$.
By the definition of the restriction product:
\[
f \cap g =  \<f,g\>\inv{\Delta} =\<f,g\>\inv{\Delta} \Delta \pi_0 =
\<f,g\>\restr{\inv{\Delta}}\pi_0 \le \<f,g\>\pi_0 \le f
\]
Similarly, substituting $\pi_1$ for $\pi_0$ above, this gives
 $f \cap g \le g$.
For the left distributive law,
\[
h(f \cap g) = h \<f,g\>\inv{\Delta} =  \<h f,h g\>\inv{\Delta} = h f \cap h g
\]
and finally an intersection of a map with itself is
\[f\cap f = \<f,f\> \inv{\Delta} = (f \Delta) \inv{\Delta} = f \restr{\Delta} = f \]
as $\Delta$ is total. This shows that $\cap$ as defined above is a meet for the
Cartesian restriction category \X.

\end{proof}

We shall refer to a Cartesian restriction category in which every object is
discrete as simply a discrete restriction category.
% subsection restriction_products_and_cartesian_restriction_categories (end)

\subsection{Graphic Categories} % (fold)
\label{sub:graphic_categories}


In a Cartesian restriction category, a map $A\xrightarrow{f}B$ is
called \emph{graphic} when the maps
\[
A\xrightarrow{\<f,1\>}B\times A\qquad \text{and}\qquad
A\xrightarrow{\<\rst{f},1\>}A\times A
\]
have partial inverses.
 A Cartesian restriction category is \emph{graphic} when all of its maps
are graphic.
\begin{lemma}\label{lem:graphic_maps_are_closed_in_a_cartesian_restriction_category}
In a Cartesian restriction category:
\bproofenum
\item Graphic maps are closed under composition;
\item Graphic maps are closed under the restriction;
\item An object is discrete if and only if its identity map is graphic.
\eproofenum
\end{lemma}
\begin{proof}
\bproofenum
\item To show closure, it is necessary to show that $\<f g,1\>$ has a partial inverse.
By Lemma \ref{lem:rcs_partial_monic_section_inverse_properties}, the uniqueness of the partial inverse gives
\[\inv{(\<f,1\> ; \<g,1\>\times 1)} = \inv{\<g,1\>} \times 1 ; \inv{\<f,1\>} .\]
 By the definition of the restriction product, $\rst{\<f g,1\>} = \rst{f g}$. Additionally,
a straightforward calculation shows that $\rst{\<f,1\>;\<g,1\> \times 1} = \rst{\<f\<g,1\>, 1\>} = \rst{f ;\< g,1\>}
= \rst{\<f;g, f\>} = \rst{f g}\,\rst{f} = \rst{f g}$ where the last equality is by \rtwo, \rthree and finally \rone.

Consider the diagram
\[
\xymatrix @C+35pt @R+20pt{
A \ar[r]^{\<f,1\>} \ar[drr]_{\<f g,1\>} &
   B \times A  \ar[r]^{\<g,1\> \times 1}
   &  C \times B \times A \\
&&C \times A \ar[u]_{1 \times \<f,1\>}
}
\]

From this:
\begin{align*}
\<f g,1\>  (1\times \<f,1\>) ( \inv{\<g,1\>}\times 1) \inv{\<f,1\>}
&=\<f,1\>(\<g,1\>\times 1 ) (\inv{\<g,1\>}\times 1) \inv{\<f,1\>}\\
&=\<f,1\> (\rst{g\times 1}) \inv{\<f,1\>}\\
&=\rst{\<f,1\> (g\times 1)}  \<f,1\> \inv{\<f,1\>}\\
&=\rst{\<f,1\> (g\times 1)}  \rst{\<f,1\>}\\
&= \rst{\<f,1\>} \rst{\<f,1\>(g\times 1)}\\
&= \rst{\<f,1\> (g\times 1)}\\
&= \rst{\<f g,1\>}(=\restr{f g})\\
\end{align*}
showing that $1\times \<f,1\>  (\inv{\<g,1\>}\times 1 ) \inv{\<f,1\>}$ is
 a right inverse for $\<f g,1\>$.

For the other direction, note that in general $\inv{h k} = \inv{k}\inv{h}$ and that
we have $\<f g,1\> = \<f,1\> (\<g,1\>\times 1)  (1 \times \inv{\<f,1\>})$, thus
 $(1\times \<f,1\>)  (\inv{\<g,1\>}\times 1) \inv{\<f,1\>}$ will also be a left inverse and
$\<f g,1\>$ is a restriction isomorphism.

\item This follows from the definition of graphic and that
 $\rst{\<f,1\>} = \rst{f} = \restr{\rst{f}}$.

\item Given a discrete object $A$, $1_A$ is graphic as $\<1_A,1\> = \Delta$ and therefore
$\inv{\<1,1\>} =
\inv{\Delta}$. Conversely, if $\<1_A,1\>$ has an inverse, then $\Delta = \<1_A,1\>$ has that same
inverse and therefore the object is discrete.
\eproofenum
\end{proof}

\begin{lemma}\label{lem:a_discrete_crc_is_precisely_a_graphic_crc}
A discrete restriction category is precisely a
graphic Cartesian restriction category.
\end{lemma}
\begin{proof}
The requirement is that $\<f,1\>$ (and $\<\rst{f},1\>$) each have partial inverses. For
$\<f,1\>$, the inverse is
$\rst{(1 \times f)\inv{\Delta}} \pi_1$.

To show this, calculate  the two compositions. First,
\[
\<f,1\> \rst{1 \times f \inv{\Delta}} \pi_1 = \rst{\<f,f\> \inv{\Delta}}\<f,1\>\pi_1 % use R.4
= \rst{f \Delta \inv{\Delta}}\<f,1\>\pi_1 % product
= \rst{f}\<f,1\>\pi_1 % Delta total
= \rst{f}.% product
\]
The other direction is:
\begin{align*}
\rst{(1 \times f)\inv{\Delta}} \pi_1 \<f,1\>
  &= \< \rst{(1 \times f)\inv{\Delta}} \pi_1 f , \rst{(1 \times f)\inv{\Delta}}\pi_1 \>\\ %pdt defn
  &= \< \rst{(1 \times f)\inv{\Delta}} (1 \times f) \pi_1, \rst{(1 \times f)\inv{\Delta}}\pi_1 \>\\ %pi total, natural
  &= \< (1 \times f )\rst{\inv{\Delta}} \pi_1 ,\rst{(1 \times f)\inv{\Delta}}\pi_1 \>\\ %R.4
  &= \< (1 \times f) \rst{\inv{\Delta}} \pi_0 ,\rst{(1 \times f)\inv{\Delta}}\pi_1 \>\\ %below
  &= \< \rst{(1 \times f)\inv{\Delta}} (1 \times f) \pi_0, \rst{(1 \times f)\inv{\Delta}}\pi_1 \>\\ %R.4
%  &= \< \rst{(1 \times f)\inv{\Delta}}\, \rst{(1 \times f)} \pi_0, \rst{(1 \times f)\inv{\Delta}}\pi_1 \>\\
  &= \< \rst{(1 \times f)\inv{\Delta}} \pi_0, \rst{(1 \times f)\inv{\Delta}}\pi_1 \>\\ %(a x b);pi0 = a
  &= \rst{(1 \times f)\inv{\Delta}} \< \pi_0, \pi_1 \>\\ % products
  &= \rst{(1 \times f)\inv{\Delta}}
\end{align*}
The one tricky step is to realize
\begin{align*}
     \rst{\inv{\Delta}} \pi_1
        &= \inv{\Delta} \Delta \pi_1\\
        &= \inv{\Delta}\\
        &= \inv{\Delta} \Delta \pi_0\\
        &= \rst{\inv{\Delta}} \pi_0
\end{align*}

For $\<\rst{f},1\>$, the inverse is $\rst{(1 \times \rst{f})\inv{\Delta}} \pi_1$. Similarly to above,
\[
\<\rst{f},1\> \rst{1 \times \rst{f} \inv{\Delta}} \pi_1 = \rst{\<\rst{f},\rst{f}\> \inv{\Delta}}\<\rst{f},1\>\pi_1 % use R.4
= \rst{\rst{f} \Delta \inv{\Delta}}\<\rst{f},1\>\pi_1 % product
= \rst{\rst{f}}\<\rst{f},1\>\pi_1 % Delta total
= \rst{f}.% product
\]
The other direction follows the same pattern as for $\<f,1\>.$
\end{proof}
% subsection graphic_categories (end)

% section restriction_categories (end)
