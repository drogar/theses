%!TEX root = /Users/gilesb/UofC/thesis/phd-thesis/phd-thesis.tex
\chapter{Turing categories and PCAs} % (fold)
\label{chap:turing_categories}

In this chapter, we review the definition and properties of a Turing category and partial
combinatory algebras
\cite{cockett-hostra08-intro-to-turing,cockett2010:categories-and-computability}. Because of
the theorems of the earlier chapters, we will be able to transfer these ideas in a straightforward
way from discrete Cartesian restriction categories to discrete inverse categories.  Inverse Turing
categories are defined below and correspond to Turing categories using
Theorem~\ref{thm:inverse_turing_category_gives_a_turing_category}: This  provides the link between
reversible computation and standard models of computation as promised in the introduction.

As noted in the introduction, Bennett\cite{bennett:1973reverse} showed how a reversible Turing
machine can emulate a standard Turing machine. As Turing machines can perform the applications of a
partial combinatory algebra, we have a link between inverse Turing categories through Turing
categories to reversible Turing machines.

\section{Turing categories}
\label{sec:turing_category_definitions}
Turing categories provide a categorical formulation for
computability and includes partial combinatory algebras, the partial lambda calculus, and various
other models as given in
\cite{cockett-hostra08-intro-to-turing}.

\begin{definition}[Turing category]\label{def:turing_category}
  Given \X is a Cartesian restriction category:
  \begin{enumerate}[{(}i{)}]
    \item For a map $\txy: A \times X \to Y$, a map $f:B\times X \to Y$ \emph{admits a $\txy$-index}
      when there is a total $\name{f}:B\to A$ such that
      \[
        \xymatrix@C+10pt@R+10pt{
          A\times X \ar[r]^{\txy} & Y \\
          B\times X \ar@{.>}[u]^{\name{f}\times 1_X} \ar[ur]_f
        }
      \]
      commutes.\label{defitem:turing_admit_txy_index}
    \item A map $\txy: A \times X \to Y$ is called a \emph{universal application} if all
      $f:B\times X \to Y$ admit a $\txy$-index.\label{defitem:turing_universal_application}
    \item If $A$ is an object in $\X$ such that for every pair of objects $X,Y$ in \X there is
      a universal application $\tau:A\times X \to Y$, then $A$ is called a \emph{Turing object}.
    \item A Cartesian restriction category that contains a Turing object is called a
      \emph{Turing category}.
  \end{enumerate}
\end{definition}

Note there is no requirement in the definition for the map $\name{f}$ to be unique. When $\name{f}$ is unique
for a specific $\txy$, then that $\txy$ is called \emph{extensional}. In the case where the object
$B$ is the terminal object, then the map $\name{f}$ is a point of $A$ (with $f = (\name{f} \times 1)\txy$) and
$\name{f}$ is referred to as a \emph{code} of $f$.

\begin{example}\label{ex:turing-category-kleene}
  This example is due to Cockett and Hofstra\cite{cockett-hostra08-intro-to-turing}.

  We start with a ``suitable'' enumeration of partial recursive functions $f:\nat\to\nat$. Based on
  the fact that functions such as these can be described by Turing machines, and that Turing
  machines may be enumerated as $\{\phi_0,\phi_1,\ldots\}$, each of these functions can be coded
  into a single number. This may be extended to partial recursive functions of $n$ variables which
  may similarly be enumerated, $\{\phi_0^{(n)},\phi_1^{(n)},\ldots\}$. When $f$ is given by $\phi_e$
  we say $e$ is a \emph{code} for $f$.

  Two facts we will need about the family $\phi_m^{(n)}$:
  \begin{itemize}
    \item \textbf{Universal Functions:} There are partial recursive functions such that for each $n > 0$,
      \[ \Phi^{(n)}(e,x_1,\ldots,x_n) = \phi_e(x_1,\ldots,x_n) \]
      which are called the \emph{universal functions}.
    \item \textbf{Parameter Theorem:} There are primitive recursive functions $S_m^n$ for each $n,m >
      0$ such that:
      \[ \Phi^{(n+m)}(e,x_1,\ldots,x_m,u_1,\ldots,u_n) = \Phi^{(m)}(S_m^n(e,x_1,\ldots,x_m),u_1,\ldots,u_n).\]
  \end{itemize}

  Suppose we choose such an enumeration and use it for the \emph{Kleene-application} on the natural
  numbers, i.e., $\nat\times\nat\xrightarrow{\bullet}\nat$ will be defined as
  \[
    n\bullet x = \phi_n(x).
  \]

  Now, consider the category with objects being the finite powers of $\nat$ and a map $\nat^k \to
  \nat^m$ is an $m$-tuple of partial recursive function of $k$ variables. When $k$ is zero, the map
  simply picks a specific $m$-tuple of $\nat$. We denote this category by $\compn$.
  Then, $\compn$ is a Turing category with $\N$ being a Turing object, using Kleene-application
  as the application map. We know $\N$ is isomorphic to $\N\times \N$ and hence we have $\N\times \N
  \retract \N$. Application is guaranteed to be partial recursive by the universal functions item
  above and the weak universal property of $\bullet$ is a result of the Parameter Theorem.
\end{example}

\begin{definition}\label{def:turing_structure}
  Given $\T$ is a Turing category and $A$ is an object of \T,
  \begin{enumerate}[{(}i{)}]
    \item If $\Upsilon=\{\txy: A\times X \to Y | X,Y \in \objects{\T}\}$, then $\Upsilon$ is called an
      \emph{applicative family} for $A$.
    \item An applicative family $\Upsilon$ is called \emph{universal for $A$} when each $\txy$ is
      a universal application. This is also referred to as a \emph{Turing structure} on $A$.
    \item A pair $(A,\Upsilon)$ where $\Upsilon$ is universal for $A$ is called a \emph{Turing
      structure} on \T.
  \end{enumerate}
\end{definition}

\begin{lemma}\label{lem:turing_object_is_retractable}
  If \T is a Turing category with Turing object $T$, then every object $B$ in \T is a retract of
  $T$.
\end{lemma}
\begin{proof}
  As $T$ is a Turing object, we have a diagram for $\tur{1}{B}$ and $\pi_0:B\times 1 \to B$:
  \[
    \xymatrix@C+10pt@R+10pt{
      T\times 1 \ar[r]^{\tur{1}{B}} & B \\
      B\times 1. \ar@{.>}[u]^{\name{\pi_0}\times 1_1} \ar[ur]_{\pi_0}
    }
  \]
  Note we also have $u_r:B\to B\times 1$ is an
  isomorphism and therefore we have $1_B = u_r \pi_0 = (u_r (\name{\pi_0}\times 1)) \tur{1}{B}$. Hence, we
  have $B \retractmaps{u_r (\name{\pi_0}\times 1)}{\tur{1}{B}} T$.
\end{proof}

This allows for various recognition criteria for Turing categories.

\begin{theorem}\label{thm:turing_recognition}
  A Cartesian restriction category \D is a Turing category if and only if $\D$ has an object $T$
  for which every other object of \D is a retract and $T$ has a universal self-application map
  $\bullet$, written as $T\times T \xrightarrow{\ \bullet\ }T$.
\end{theorem}
\begin{proof}
  The ``only if'' portion follows immediately from setting $T$ to be the Turing object of $\D$ and
  $\bullet = \tur{T}{T}$.

  For the ``if'' direction, we need to construct the family of universal applications
  $\txy : T\times X \to Y$ for each pair of objects $X,Y$ in \D.

  Let us choose pairs of maps that witness the retractions of $X, Y$ of $T$, that is:
  \[
    X \retractmaps{m_X}{r_X} T \quad\text{and}\quad Y\retractmaps{m_Y}{r_Y} T.
  \]
  Define $\txy = (1_T\times m_X ) \bullet r_Y$. Suppose we are given $f:B\times X \to Y$. Consider
  \[
    \xymatrix@C+10pt@R+10pt{
      T \times X \ar[r]^{1_T\times m_X} & T\times T \ar[r]^{\bullet} & T \ar[r]^{r_Y} & Y \\
      B \times X \ar[r]^{1_B\times m_X} \ar[u]^{h\times 1_X}
        & B \times T \ar@{.>}[u]^{h\times 1_T} \ar[r]^{1_B \times r_X}
        & B \times X \ar[ur]^{f} \ar[u]^{f m_Y}
      }
  \]
  where $h$ is the index for the composite map $(1_B \times r_X) f m_Y$. The middle square commutes
  as $\bullet$ is a universal application for $T,T$. The right triangle commutes as $m_Y r_Y =1$.
  The left square commutes as each composite is $h \times m_X$. Noting that the bottom path from
  $B\times X$ to $Y$ is $(1_B \times m_X)(1_B \times r_X)f = f$ and the top path from $T\times X$ to
  $Y$ is our definition of $\txy$, this means $f$ admits the $\txy$-index $h$.
\end{proof}

Note that different splittings (choices of $(m,r)$ pairs) would lead to different $\txy$ maps. In
fact there is no requirement that this is the only way to create a universal applicative family
for $T$.

There is another criteria that also gives a Turing category:

\begin{lemma}\label{lem:t_t_to_t_gives_a_turing_category}
  A Cartesian restriction category \T is a Turing category if:
  \begin{enumerate}[{(}i{)}]
  \item $\T$ has an object $T$ for which every other object of \T is a retract;
  \item $T\times T$ has a map  $T\times T \xrightarrow{\ \circ\ }T$ and for all
    $f:T\to T$ there exists an element, $\code{f}:1\to T$ (which is total) such that
      \[
        \xymatrix{
          T\times T \ar[r]^{\circ} & T \\
          T \ar[ur]_{f} \ar[u]^{\<!\code{f},1\>}
        }
      \]
    is a commutative diagram.
  \end{enumerate}
\end{lemma}
\begin{proof}
  We need only show that $T$ has a universal self-application map and then use
  Theorem~\ref{thm:turing_recognition}.

  $T$ having a universal self-application map, $\bullet$, means for every map $f:B\times T \to T$ there is a
  map, $\name{f}:B\to T$ such that
  \[
    \xymatrix{
      T\times T \ar[r]^{\bullet} & T \\
      B\times T \ar[ur]_{f} \ar[u]^{\name{f}\times1}
    }
  \]
  commutes.

  Let $T\times T\retractmaps{m}{r} T$. Then, consider
  \[
    \xymatrix@C+15pt{
     T\times T \ar[r]^{r\times 1} & T\times T\times T \ar[r]^{1\times m} &T\times T \ar[rr]^{\circ}
       && T \\
     T\times T\times T \ar[u]^{m\times 1} \ar@{=}[ur]\\
     T\times T \ar@{.>}[u]^{\<\code{(r f)},\pi_0,\pi_1\>} \ar@{.>}[uur]_{\<\code{(r f)},\pi_0,\pi_1\>}
       \ar[rr]_{m} \ar@{=}@/_20pt/[rrr]
       & & T \ar[r]_{r} \ar@{.>}[uu]_{\<\code{(r f)},1\>} &T\times T. \ar[uur]_{f}
    }
  \]
  \\[10pt]
  The rightmost quadrilateral commutes by assumption of this lemma. The middle quadrilateral
  commutes due to the properties of the product map and $\pi_0$ and $\pi_1$. The top left triangle
  commutes as $m r = 1$ and the remaining triangle has the same map on both dotted lines.

  Thus, we may conclude that $\bullet \definedas (r \times 1)(1\times m) \circ$ and
  $\name{f}\definedas \<!\code{(rf)},1\> m$ satisfy the requirements of
  Theorem~\ref{thm:turing_recognition} and therefore $T$ is a Turing object in a Turing category.
\end{proof}
\section{Inverse Turing categories}
\label{sec:inverse_turing_categories}
Now, we define inverse Turing categories. The idea is that an inverse Turing category should be a discrete
inverse category \X such that $\Xt$ is a Turing category. A concrete description of this is
developed below.

\begin{definition}\label{def:inverse_turing_category}
  A discrete inverse category \X is an \emph{inverse Turing category} when there is a universal
  object $T$ (i.e., every $B\in\X_o$ is a retract of $T$) in
  \X with a map $\diamond :T\*T \to T\*T$ such that for every map $f:T \to T\*T$ there is a total map
  $\iname{f}: I\to T$ and a map $h_f:T\*T \to T\*T$ with $h_f \in \dmap{T}$ such that $f \xequiv{h_f}
  \inv{\usl}(\iname{f}\*1)\diamond$, i.e., the diagram
  \begin{equation}
    \xymatrix @C=15pt @R=15pt{
      & & T \* T \ar@{.>}[dddd]^{h_f}\\
      &T\*T \ar[ur]^{\diamond} & & \\
      T \ar[ur]^(.3){\inv{\usl}(\iname{f}\*1)} \ar[ddrr]_{f} \\
      & & & \\
      && T\*T
    }\label{dia:inverse-turing-category-code}
  \end{equation}
  commutes.
\end{definition}

First, we observe:

\begin{lemma}\label{lem:discrete_turing_category_inverses_make_inverse_turing_category}
  When $\T$ is a discrete Turing category then $\Inv{\T}$ is an inverse Turing category.
\end{lemma}
\begin{proof}
  By Lemma~\ref{lem:inv_x_is_a_discrete_inverse_category}, we know that $\Inv{\T}$ is a discrete
  inverse category. Thus, all that remains is to show:
  \begin{enumerate}[{(}i{)}]
    \item There is a map $\diamond:T\*T \to T\*T$ in $\Inv{\T}$;
    \item for a  map $f$ in $\Inv{\T}$, there is another map $\iname{f}$ which makes
      Diagram~\ref{dia:inverse-turing-category-code} commute.
  \end{enumerate}

% Do not need $f$ to be invertible, it is just the map $(f,D)$.
  As we are in a Turing category, we know that we have the diagram
      \[
        \xymatrix{
          T\times T \ar[r]^{\circ} & T \\
          T \ar[ur]_{f} \ar[u]^{\<!\code{f},1\>}
        }
      \]
  in $\T$.  The  map $\<!\code{f},1\>$ is invertible by
  \ref{lem:a_discrete_crc_is_precisely_a_graphic_crc} as we are in a discrete Cartesian restriction
  category.

  Expressing this in $\Inv{\T}$, for some $h'_f \in \dmap{T}$ we have:
  \begin{equation}
    \xymatrix @C=15pt @R=15pt{
      & & T \* T\*C \ar@{.>}[dddd]^{h'_f}\\
      &T\*T \ar[ur]^{\circ} & & \\
      T \ar[ur]^(.3){\inv{\usl}(\code{f}\*1)} \ar[ddrr]_{f} \\
      & & & \\
      && T\*D.
    }\label{dia:pre-inverse-turing}
  \end{equation}

  Recall from Lemma~\ref{lem:delta_nabla_maps_are_closed} that $\dmap{T}$ is closed under
  composition and that $1\*f\in \dmap{T}$ for any $f$. In particular, for $f:T\to T\*D$ in
  $\Inv{\T}$, as $D\retract T$, we have $f$ is in the same equivalence class as $f(1\*m_D)$. We see
  this as $\rst{f} = \rst{f(1\*m_D)}$, $1\*m_D\in \dmap{T}$ and the diagram
  \[
    \xymatrix@R-10pt{
      &T\*D \ar@{.>}[dd]^{1\*m_D}\\
      T \ar[ur]^{f} \ar[dr]_{f(1\*m_D)}& \\
      &T\*T
    }
  \]
  obviously commutes.

  We use this and the fact that we have both $T\*C \retract T$
  and $D \retract T$, to add to   Diagram~\ref{dia:pre-inverse-turing}:
  \[
    \xymatrix @C=15pt @R=15pt{
      & & & T \* T \ar@{.>}[d]^{1\*r_{T\*C}}\\
      & & T \* T \* C \ar@{.>}[dddd]^{h'_f} \ar[ur]^{1\*m_{T\*C}} \ar@{=}[r] &T\*T\*C \ar@{.>}[dddd]^{h'_f} \\
      &T\*T \ar[ur]^{\circ} & & \\
      T \ar[ur]^(.3){\inv{\usl}(\code{(f(1\*m_D))}\*1)} \ar[ddrr]_{f} \\
      & & & \\
      && T\*D \ar[dr]_{1\*m_D} &  T\*D \ar[d]^{1\*m_D}\\
      &&&T\*T.
    }
  \]
  But this is the required diagram for an inverse Turing category with $h_f = (1\*r_{T\*C}) h'_f (1\*m_D)$,
  and with $\iname{f} = \code{(f(1\*m_D))}$ and  with $\diamond = \circ(1\* m_{T\*C})$. Therefore
  $\Inv{\T}$ is an inverse Turing category.
\end{proof}

We know that applying the Cartesian Completion to \X, an inverse Turing category, results in \Xt, a
discrete Cartesian restriction category. Moreover, if $A\retractmaps{m_A}{r_A} T$ in \X, then $A
\retractmaps{m_A\inv{\usr}}{r_A\inv{\usr}} T$ in \Xt and thus $T$ will remain universal in
\Xt. Hence, we have the basic requirements for a Turing category as specified in
Theorem~\ref{thm:turing_recognition} and Lemma~\ref{lem:t_t_to_t_gives_a_turing_category}. All that
remains to be shown is that we have a self-application map and a code for each map $f:1\to T$ as in
Lemma~\ref{lem:t_t_to_t_gives_a_turing_category}.

\begin{theorem}\label{thm:inverse_turing_category_gives_a_turing_category}
  When $\X$ is an inverse Turing category, $\Xt$ is a Turing category.
\end{theorem}
\begin{proof}
  From the discussion, we need to specify the self-application map $\circ:T\times T \to T$ and
  $\code{f}:1 \to T$ in \Xt.

  The diagram of Definition~\ref{def:inverse_turing_category}, when raised to $\Xt$
  translates to:
  \[
    \xymatrix@C+25pt@R+10pt{
      T \times T \ar[r]^{(\diamond, T)} &T \\
      T. \ar[u]^{\<\iname{f},1\>} \ar[ur]_{(f, T)}
    }
  \]
  But this corresponds exactly to the requirement of
  Lemma~\ref{lem:t_t_to_t_gives_a_turing_category} with $\circ = (\diamond,T)$ and $\code{(f,T)} =
  \name{f}$.  Finally, noting that $T$ is universal in \X, if we have $(f,B):T\to T$ in \Xt,
  where $B\retractmaps{m_B}{r_B} T$ in \X, we recall that $(f,B) \xequiv{} (f(1\*m_B),T)$ in
  \X. Therefore, it may be  written as above and we therefore have shown that $\Xt$ is a Turing
  category.
\end{proof}

\section{Partial combinatory algebras}
\label{sec:partial_combinatory_algebras}

In a Cartesian restriction category, for any operation $f:A\times A\to A$ define $\multiapp{f}{n}$
for $n\ge 1$  recursively by:
\begin{enumerate}[{(}i{)}]
  \item $\multiapp{f}{1} = f$,
  \item $\multiapp{f}{n+1} = (f\times 1) \multiapp{f}{n}$.
\end{enumerate}

\begin{definition}\label{def:partial_combinatory_algebra}
  A Cartesian restriction category has a \emph{partial combinatory algebra} when it has an object
  $A$ together with:
  \begin{enumerate}[{(}i{)}]
  \item A partial map $\bullet:A\times A \to A$,\label{defitem:pca-1}
  \item two total elements $1\xrightarrow{k}A$ and $1\xrightarrow{s}{A}$ which satisfy\label{defitem:pca-2}
    \[
      \xymatrix@C+25pt{
        A\times A\times A \ar[r]^(.6){(\bullet\times 1)\bullet} & A\\
        A\times A \ar[u]^{k\times1\times1} \ar[ur]_{\pi_1}
      }\quad
      \xymatrix{
        A\times A\times A\times A \ar[r]^(.6){\multibullet{3}}&A\\
        &A\times A \ar[u]_{\bullet}\\
        A\times A\times A \ar[uu]^{s\times1\times1\times1} \ar[r]_(.4){\theta_A'}
          & (A\times A) \times (A\times A), \ar[u]_{\bullet\times\bullet}
      }
    \]
    \item $A\times A \xrightarrow{s\times1\times1} A\times A\times A \xrightarrow{\bullet^2} A$ is total.\label{defitem:pca-3}
  \end{enumerate}
  In the above $\theta' = (1\times1\times\Delta)(1\times c \times 1)a$ where $a$ sets the
  parenthesis as in the diagram.
\end{definition}

Of course, this is more familiarly given equationally by:
\[
   (k\bullet x)\bullet y = x \qquad ((s\bullet x)\bullet y) \bullet z = (x\bullet z) \bullet
   (y\bullet z).
\]
These are the equations of a combinatory algebra where partiality is not considered. As we have
partiality, we also add the requirement that $s\bullet x\bullet y$ is a total map for any $x,y$.

Note that if we have a Turing object $T$ in a Cartesian restriction category, it is a partial
combinatory algebra. All we need to do is to actually define the element $k$ and $s$ by using the
commuting diagrams of Definition~\ref{def:partial_combinatory_algebra}.

Now, we want to consider what are the conditions required for an inverse category \X such that $\Xt$
has a partial combinatory algebra.

In a discrete inverse category, we define the notation $\imultiapp{f}{n}$. For any operation $f:A\* A\to A\*A$
define $\imultiapp{f}{n}$ recursively by:
\begin{enumerate}[{(}i{)}]
\item
  $\imultiapp{f}{1}:A\*A \to A\*A =f =\ $\raisebox{-12pt}{
  \begin{tikzpicture}
    \node [style=nothing] (s1) at (0,1) {};
    \node [style=nothing] (s2) at (.5,1) {};
    \node [style=map] (bullet1) at (.25,.5) {$\scriptstyle f$};
    \node [style=nothing] (e1) at (0,0) {};
    \node [style=nothing] (e2) at (.5,0) {};
    \draw [] (s1) to[out=270,in=125] (bullet1);
    \draw [] (s2) to[out=270,in=55] (bullet1);
    \draw [] (bullet1) to[out=235,in=90] (e1);
    \draw [] (bullet1) to[out=305,in=90] (e2);
  \end{tikzpicture}}.
\item
  $\imultiapp{f}{n+1}:A\*(\*_{n}A)\*A \to A\*(\*_{n+1}A) =\ $\raisebox{-45pt}{
  \begin{tikzpicture}
    \node [style=nothing] (s1) at (-0.25,3.5) {};
    \node [style=nothing] (s1a) at (0.25,3.5) {};
    \node [style=nothing] (topdots) at (.5,3.5) {$\scriptstyle \cdots$};
    \node [style=nothing] (s2) at (.75,3.5) {};
    \node [style=nothing] (s3) at (1,3.5) {};
    \node [style=tensor] (t0) at (.5,3) {$\scriptstyle \*$};
    \node [style=map] (bullet1) at (.25,2.25) {$\scriptstyle \imultiapp{f}{n}$};
    \node [style=map] (bullet2) at (.25,1) {$\scriptstyle f$};
    \node [style=tensor] (t1) at (.75,.5) {$\scriptstyle \*$};
    \node [style=nothing] (e1) at (0,0) {};
    \node [style=nothing] (e2) at (.75,0) {};
    \draw [] (s1) to[out=270,in=125] (bullet1);
    \draw [] (s1a) to[out=270,in=125] (t0);
    \draw [] (s2) to[out=270,in=55] (t0);
    \draw [] (s3) to[out=270,in=55] (bullet2);
    \draw (t0) to[out=270,in=55] (bullet1);
    \draw [] (bullet1) to[out=235,in=125] (bullet2);
    \draw [] (bullet1) to[out=305,in=55] (t1);
    \draw [] (bullet2) to[out=305,in=125] (t1);
    \draw [] (bullet2) to[out=235,in=90] (e1);
    \draw (t1) to (e2);
  \end{tikzpicture}}.
\end{enumerate}

\begin{definition}\label{def:inverse_partial_combinatory_algebra}
  A discrete inverse category \X has an \emph{inverse partial combinatory algebra} when there is an
   object $A$ in \X with a map $A\*A  \xrightarrow{\bullet} A\*A$ and two total elements:
  \[
      1\xrightarrow{k}A \qquad 1\xrightarrow{s}{A}
  \]
  and maps
    $h_k:A\*A\*A\to A\*A, h_s:A\*A\*A\*A\to A\*A\*A\*A$  in $\dmap{A}$ which satisfy the following
    three axioms:\\
    \axiom{iCPA}{1}
    \[
      \xymatrix@C+15pt@R-10pt{
         && A\*A\*A \ar@{.>}[dddd]^{h_k}\\
        &A\* A\* A \ar[ur]^{\imultibullet{2}} \\
        A\*A \ar[ur]^(.4){\inv{\usl}(k\*1\*1)\ \ } \ar[ddrr]_{1} \\
         && & \\
        && A\*A.
      }
    \]
    \axiom{iCPA}{2}
    \[
      \xymatrix@R-10pt{
        &&& A\* A \*A\*A\ar@{.>}[dddddd]^{h_s} \\
        &&A\* A\* A\* A
          \ar[ur]^{\imultibullet{3}}\\
        && \\
        A\* A\* A \ar[uurr]^{\inv{\usl}(s\*1)} \ar[dr]_(.4){\theta_A'}\\
        &(A\* A) \* (A\* A) \ar[dr]_{(\bullet\*\bullet)(1\*c\*1)\ \ } \\
        & &A\* A\* A\*A\ar[dr]_{(\bullet\*1)}\\
        &&&A\*A\*A\*A.\\
      }
    \]
    \axiom{iCPA}{3} $I\*A\* A \xrightarrow{s\*1\*1} A\* A\* A \xrightarrow{\imultibullet{2}} A\*A\*A$ is total.
\end{definition}

\begin{proposition}\label{prop:inverse-pca-iff-pca}
  A discrete inverse category \X has an inverse partial combinatory algebra if and only if $\Xt$ has
  a partial combinatory algebra.
\end{proposition}
\begin{proof}
  When we have a discrete inverse category \X with an inverse partial combinatory algebra, we see
  immediately the map $\bullet:A\*A\to A\*A$ in \X becomes the map $(\bullet,A):A\times A \to A$,
  satisfying \ref{defitem:pca-1} of Definition~\ref{def:partial_combinatory_algebra}. The
  commutative diagrams \axiom{iCPA}{1} and \axiom{iCPA}{2}, when lifted to \Xt, become the diagrams
  for a partial combinatory algebra as given in \ref{defitem:pca-2}, where $(k\inv{\usl}, I)$ and
  $(s\inv{\usl}, I)$ are the $k,s$ of the partial combinatory algebra. Finally, the totality
  requirement, \axiom{iCPA}{3}, gives \ref{defitem:pca-3} of the partial combinatory algebra
  definition.

  Hence, we have shown that an inverse partial combinatory algebra in \X gives a partial combinatory
  algebra in \Xt.

  For the reverse, when we have a partial combinatory algebra over $A$ in $\Xt$, a discrete Cartesian
  restriction category, by Lemma~\ref{lem:a_discrete_crc_is_precisely_a_graphic_crc} we know that
  the map $\<\bullet,1\>$ is invertible and hence is in $\X$. The two maps $\<k,1\>$ and $\<s,1\>$
  are also invertible and therefore are in $\X$.

  Given this, the diagrams of the partial combinatory algebra in \Xt translate directly to the
  \axiom{iCPA}{1} and \axiom{iCPA}{2} where $\bullet$ in \X is the invertible map $\<\bullet,1\>$.

  The totality  of $s \multibullet{2}$ in \Xt then immediately gives us \axiom{iCPA}{3}, the
  totality of $s \imultibullet{2}$ in \X.


\end{proof}

However, we can simplify the definition of an inverse partial combinatory algebra when $A$ is
powerful in \X. (Here, powerful means that $1\retract A$, $A\*A\retract A$, $A\*A\*A\retract A$,
$\ldots$). Note that if $A$ is a partial combinatory algebra in $\Xt$, that guarantees it is
powerful in \Xt. Assuming the retractions are $A^n \retractmaps{m_n}{r_n} A$, we have $m_j r_j =
1$ and $r_j m_j = \rst{r_j m_j}$ for each $j$. Thus, each of the maps are partial inverses in
$\Xt$ and therefore in $\X$. Thus, $A$ is a powerful object in $\X$.

Note that our definition of ``powerful'' does not relate to resource usage or a particular
complexity class. An example of a powerful object is the natural numbers with the Cantor
enumeration.


% We redefine the meaning of the notation $\multiapp{f}{n}$ when in a discrete inverse
% category.

In a discrete inverse category, we redefine the notation $\multiapp{f}{n}$. For any map $f:A\* A\to
A\*A$ where $A$ is a powerful object define $\multiapp{f}{n}$ recursively by:
\begin{enumerate}[{(}i{)}]
\item
  $\multiapp{f}{1}:A\*A \to A\*A =f =\
  \raisebox{-12pt}{\begin{tikzpicture}
    \node [style=nothing] (s1) at (0,1) {};
    \node [style=nothing] (s2) at (.5,1) {};
    \node [style=map] (bullet1) at (.25,.5) {$\scriptstyle f$};
    \node [style=nothing] (e1) at (0,0) {};
    \node [style=nothing] (e2) at (.5,0) {};
    \draw [] (s1) to[out=270,in=125] (bullet1);
    \draw [] (s2) to[out=270,in=55] (bullet1);
    \draw [] (bullet1) to[out=235,in=90] (e1);
    \draw [] (bullet1) to[out=305,in=90] (e2);
  \end{tikzpicture}}$;
\item
  $\multiapp{f}{n+1}:\*_{n+2}A \to A\*A =\
  \raisebox{-35pt}{\begin{tikzpicture}
    \node [style=nothing] (s1) at (0,2.5) {};
    \node [style=nothing] (s2) at (.5,2.5) {};
    \node [style=nothing] (s3) at (1,2.5) {};
    \node [style=map] (bullet1) at (.25,1.75) {$\scriptstyle \multiapp{f}{n}$};
    \node [style=map] (bullet2) at (.25,1) {$\scriptstyle f$};
    \node [style=map] (r1) at (.65,.4) {$\scriptstyle m_2$};
    \node [style=nothing] (e1) at (0,0) {};
    \node [style=nothing] (e2) at (.65,0) {};
    \draw [] (s1) to[out=270,in=125] (bullet1);
    \draw [] (s2) to[out=270,in=55] (bullet1);
    \draw [] (s3) to[out=270,in=55] (bullet2);
    \draw [] (bullet1) to[out=235,in=125] (bullet2);
    \draw [] (bullet1) to[out=305,in=55] (r1);
    \draw [] (bullet2) to[out=305,in=125] (r1);
    \draw [] (bullet2) to[out=235,in=90] (e1);
    \draw [] (r1) to[out=270,in=90] (e2);
  \end{tikzpicture}}$.
\end{enumerate}

\begin{lemma}\label{lem:powerful_inverse_partial_combinatory_algebra}
  Suppose a discrete inverse category \X has a inverse partial combinatory algebra over $A$ and $A$
  is  a powerful object in \X, with $\*^n A \retractmaps{m_n}{r_n} A$. Then \axiom{iCPA}{1},
  \axiom{iCPA}{2} and \axiom{iCPA}{3} may be simplified to:\\
  \axiom{iCPA$'$}{1}
    \[
      \xymatrix@C+15pt{
         && A\*A \ar@{.>}[dddd]^{h'_k}\\
        &A\* A\* A \ar[ur]^{\multibullet{2}} & & \\
        A\* A \ar[ur]^{\inv{\usl}(k\*1\*1)\ \ } \ar[ddrr]_{1} && &\\
         && & \\
        && A\*A. &
      }
    \]
  \axiom{iCPA$'$}{2}
    \[
      \xymatrix{
        &&& A\* A \ar@{.>}[dddddd]^{h'_s} \\
        & &  && \\
        &A\* A\* A\* A
          \ar[uurr]^{\multibullet{3}}\\
        A\* A\* A \ar[ur]^{\inv{\usl}(s\*1)\,} \ar[dr]_(.4){\theta_A'}\\
        &(A\* A) \* (A\* A) \ar[dr]|{(\bullet\*\bullet)(1\*c\*1)(1\*1\*m_2)} \\
        & &A\* A\* A\ar[dr]_{(\bullet\*1)(1\*m_2)\ }& &\\
        &&&A\*A. \\
      }
    \]
   \axiom{iCPA$'$}{3} $I\*A\* A \xrightarrow{s\*1\*1} A\* A\* A \xrightarrow{\multibullet{2}} A\*A$
   is total.
\end{lemma}

\section{Computable functions}
\label{sec:computable_functions}

Given a partial combinatory algebra $A$ in a Cartesian restriction category, one can form \compa,
the category of computable partial functions generated by $A$. These are the maps with an index:
\[
  \xymatrix@C+25pt{
    A\times (\times_n A) \ar[r]^(.6){\multibullet{n}} & A \\
    (\times_n A). \ar[u]^{\name{f}\times 1} \ar[ur]_f
  }
\]
We would like \compa to be a discrete Turing category so that $\Inv{\compa}$ is an inverse Turing
category by
Lemma~\ref{lem:discrete_turing_category_inverses_make_inverse_turing_category}. Unfortunately, there
is no guarantee that \compa is a discrete Turing category.  However, we can define conditions
so that it is true:

\begin{definition}\label{def:discrete_pca}
  Given a discrete object $A$ in a Cartesian restriction category, $A$ has a \emph{discrete partial
    combinatory algebra} when:
  \begin{enumerate}[{(}i{)}]
  \item $A$ has a partial combinatory algebra;
  \item there exists $e:1\to A$, a total element, such that
    \[
      \xymatrix@C+25pt{
        A\times A \times A\ar[r]^(.6){\multibullet{2}} & A, \\
        A\times A  \ar[u]^{e\times 1} \ar[ur]_{\inv{\Delta}}
      }
    \]
    meaning there is a code for $\inv{\Delta}$.
  \end{enumerate}
\end{definition}

We immediately have:
\begin{lemma}\label{lem:comp_a_is_discrete_cart}
  When $A$ has a discrete partial combinatory algebra, then \compa is a discrete Cartesian
  restriction category.
\end{lemma}
By Lemma~\ref{lem:discrete_turing_category_inverses_make_inverse_turing_category}, this means that
$\Inv{\compa}$ is an inverse Turing category. Note it is still the case that there can be a map of
\compa which is invertible in \Xt (i.e., is in \X), but is \emph{not} invertible in \compa.

% chapter turing_categories (end)

%%% Local Variables:
%%% mode: latex
%%% TeX-master: "../phd-thesis"
%%% End: