%!TEX root = /Users/gilesb/UofC/thesis/phd-thesis/phd-thesis.tex

\chapter{Quantum computation}\label{chap:quantum_computation}

\section{Linear algebra} % (fold)
\label{sec:linear_algebra}

Quantum computation requires familiarity with the basics of linear algebra. This section will give
definitions of the terms used throughout this thesis.
\subsection{Basic definitions} % (fold)
\label{sub:basic_definitions}


The first definition needed is that of a \emph{vector space}.

\begin{definition}[Vector Space]
  Given a field $F$, whose elements will be referred to as scalars, a \emph{vector space} over $F$
  is a non-empty set $V$ with two operations, \emph{vector addition} and \emph{scalar
  multiplication}. \emph{Vector addition} is defined as ${+}:V\times V \to V$ and denoted as
  $\vc{v}+\vc{w}$ where $\vc{v},\vc{w}\in V$. The set $V$ must be an abelian group under $+$.
  \emph{Scalar multiplication} is defined as ${}:F\times V \to V$ and denoted as $c\vc{v}$ where
  $c\in F, \vc{v} \in V$. Scalar multiplication distributes over both vector addition and scalar
  addition and is associative. $F$'s multiplicative identity is an identity for scalar
  multiplication.

\end{definition}
The specific algebraic requirements are:
\begin{enumerate}
  \item{}$\forall \vc{u},\vc{v},\vc{w} \in V,\ (\vc{u} +\vc{v}) +\vc{w} =
    \vc{u}+ (\vc{v}+\vc{w})$;
  \item{}$\forall \vc{u},\vc{v} \in V,\ \vc{u} +\vc{v} =
    \vc{v}+ \vc{u}$;
  \item{}$\exists  \vc{0} \in V \mathrm{\ such\ that\ } \forall \vc{v} \in V,
    \vc{0} +\vc{v} =  \vc{v}$;
  \item{}$\forall \vc{u} \in V, \exists \vc{v} \in V \mathrm{\ such\ that\ }
    \vc{u}+ \vc{v} = \vc{0}$;
  \item{}$\forall \vc{u},\vc{v} \in V, c\in F,\
    c(\vc{u}+ \vc{v}) = c\vc{u} + c\vc{v}$;
  \item{}$\forall \vc{u} \in V, c,d\in F,\
    (c+d)\vc{u} = c\vc{u} + d\vc{u}$;
  \item{}$\forall \vc{u} \in V, c,d\in F,\
    (c d)\vc{u} = c(d\vc{u})$;
  \item{}$\forall \vc{u} \in V,\
    1\vc{u} = \vc{u}$.
\end{enumerate}

Examples of vector spaces over $F$ are: $F^{n\times m}$ -- the set of $n\times m$ matrices over
$F$; and $F^n$ -- the $n{-}$fold Cartesian product of $F$. $F^{n\times 1}$, the set of $n\times 1$
matrices over $F$ is also called the space of column vectors, while $F^{1\times n}$, the set of row
vectors. Often, $F^n$ is identified with $F^{n\times 1}$.


This thesis  shall identify $F^n$ with the column vector space over $F$.

\begin{definition}[Linearly independent]
  A subset of vectors $\{\vc{v}_i\}$ of the vector space $V$ is said to be \emph{linearly
  independent} when no finite linear combination of them, $\sum a_j\vc{v}_j$ equals \vc{0} unless
  all the $a_j$ are zero.

\end{definition}

\begin{definition}[Basis]
  A \emph{basis} of a vector space $V$ is a linearly independent subset of $V$ that generates $V$.
  That is, any vector $u \in V$ is a linear combination of the basis vectors.
\end{definition}

\begin{definition}\label{def:linear_map_of_vector_spaces}
  Given $V, W$ are vector spaces over $F$ with $v \in V$ and $s \in F$, then
  if $f:V \to W$ is a group homomorphism such that $f(v s) = f(v) s$, then we say $f$ is a
  \emph{linear map}. Furthermore, a map $f:V\times W \to X$ is called \emph{bilinear} when the
  map $f_v:W\to T$ and $f_w:V\to T$ are linear for each $v\in V$ and $w\in W$, where $f_v$ is the
  map obtained from $f$ by fixing $v\in V$ and $f_w$ is obtained from $f$ by fixing  $w\in W$.
\end{definition}

\begin{definition}\label{def:free_vector_space}
  Given a set $S$, the \emph{free vector space} of $S$ over a field $F$ is the abelian group
  of formal sums $\sum a_i s_i$ where the $s_i$ are the elements of $S$ and $a_i \in F$.
  Formal sums are independent of order. Addition is defined as $(\sum a_i s_i) + (\sum b_i s_i)$ is
  $(\sum (a_i + b_i) s_i)$.
\end{definition}

\begin{definition}\label{def:tensor_product_of_vector_spaces}
  Given vector spaces $V, W$ over the base field $F$, consider the free vector space of $V
  \times W = F(V\times W)$. Next, consider the subspace $T$ of $F(V\times W)$ generated by the
  following equations:
  \begin{align*}
    (v_1,w)+(v_2,w) & = (v_1+v_2,w)\\
    (v,w_1)+(v,w_2) & = (v, w_1+w_2)\\
    s(v,w) &= (s v,w)\\
    s(v,w) &= (v,s w),
  \end{align*}
  where $v,v_1,v_2 \in V$, $w,w_1,w_2 \in W$ and $s\in F$. Then the tensor product of
  $V$ and $W$, written $V\*W$ is $F(V\times W)/T$.
\end{definition}

Elements of the tensor product $V\*W$ are written as $v\*w$ and are the $T$-equivalence class of
$(v,w) \in V\times W$. If $\{v_i\}$ is a basis for $V$ and $\{w_j\}$ is a basis for $W$, then the
elements $\{v_i\*w_j\}$ form a basis for $V\*W$.
% subsection basic_definitions (end)
\subsection{Matrices} % (fold)
\label{sub:matrices}


As mentioned above, the set of $n\times m$ matrices over a field is a vector space. Additionally,
matrices compose and the tensor product of matrices is defined.

Matrix composition is defined as usual. That is, for $A = [a_{ij}] \in F^{m\times n}, B =
[b_{jk}]\in F^{n \times p}$:
  \[
    A \, B = \left[\left(\sum_{j}a_{ij}b_{jk}\right)_{ik}\right] \in F^{m \times p}.
  \]



\begin{definition}[Diagonal matrix]
  A \emph{diagonal matrix} is a matrix where the only non-zero entries are those where the column
  index equals the row index.
\end{definition}

The diagonal matrix $n\times n$ with only $1$'s on the diagonal is the identity for matrix
multiplication, and is designated by $I_n$.

\begin{definition}[Transpose]
  The \emph{transpose} of an $n\times m$ matrix $A=[a_{ij}]$ is an $m\times n$ matrix $A^{t}$ with
  the $i,j$ entry being $a_{ji}$.
\end{definition}

When the base field of a matrix is \complex, the complex numbers, the \emph{conjugate transpose}
(also called the \emph{adjoint}) of an $n\times m$ matrix $A=[a_{ij}]$ is defined as the $m\times
n$ matrix $A^{*}$ with the $i,j$ entry being $\overline{a}_{ji}$, where $\overline{a}$ is the
complex conjugate of $a\in\complex$.

When working with column vectors over \complex, note that $\vc{u} \in \complex^n \implies
\vc{u}^{*} \in \complex^{1\times n}$ and that $\vc{u}^{*}\times \vc{u} \in \complex^{1\times 1}$.
This thesis will use the usual identification of \complex{} with $\complex^{1\times1}$. A column
vector \vc{u} is called a \emph{unit vector} when $\vc{u}^{*}\times \vc{u} = 1$.

\begin{definition}[Trace]
  The \emph{trace}, $Tr(A)$ of a square matrix $A=[a_{ij}]$ is $\sum a_{ii}$.
\end{definition}

\subsubsection{Tensor Product} % (fold)
\label{ssub:tensor_product}


The tensor product of two matrices is the usual Kronecker product:
  \[
    U\otimes V =
    \begin{bmatrix}
      u_{11}V&u_{12}V & \cdots &u_{1m}V\\
      u_{21}V&u_{22}V & \cdots &u_{2m}V \\
      \vdots&\vdots&\ddots\\
      u_{n1}V&u_{n2}V & \cdots &u_{nm}V
    \end{bmatrix}
    =
    \begin{bmatrix}
      u_{11}v_{11}&\cdots&u_{12}v_{11} & \cdots& u_{1m}v_{1q} \\
      u_{11}v_{21}&\cdots&u_{12}v_{21} & \cdots& u_{1m}v_{2q} \\
      \vdots&\vdots&\vdots&\ddots \\
      u_{n1}v_{p1}&\cdots&u_{n2}v_{p1} & \cdots& u_{nm}v_{pq} \\
    \end{bmatrix}
  \]
% subsubsection tensor_product (end)

\subsubsection{Special matrices} % (fold)
\label{ssub:special_matrices}

When working with quantum values certain types of matrices over the complex numbers are of special
interest. These are:
\begin{description}
  \item[Unitary Matrix]: Any $n \times n$  matrix $A$ with $A A^{*} = I\ (= A^{*} A)$.
  \item[Hermitian Matrix]: Any  $n \times n$ matrix $A$ with $A=A^{*}$.
  \item[Positive Matrix]: Any Hermitian matrix $A$ in  $\complex^{n\times n}$
    where $\vc{u}^{*} A \vc{u} \ge 0$ for all vectors  $\vc{u}\in \complex^n$. Note
    that for any Hermitian matrix $A$ and vector $u$,  $\vc{u}^{*} A \vc{u}$ is real.
  \item[Completely Positive Matrix]: Any positive matrix $A$ in  $\complex^{n\times n}$
    where $I_m \otimes A$ is positive.
\end{description}
The matrix
  \[
    {\begin{singlespace}
      \begin{bmatrix}
        0&-i\\
        i&0
      \end{bmatrix}
    \end{singlespace}}
  \]
is an example of a matrix that is \emph{unitary}, \emph{Hermitian}, \emph{positive} and
\emph{completely positive}.


% subsubsection special_matrices (end)

\subsubsection{Superoperators} % (fold)
\label{ssub:superoperators}

A \emph{Superoperator} $S$ is a matrix over \complex{} with the following restrictions:
\begin{enumerate}
  \item{} $S$ is \emph{completely positive}. This implies that $S$ is positive as well.
  \item{} For all positive matrices $A$, $Tr(S\,A) \leq Tr(A)$.
\end{enumerate}
% subsubsection superoperators (end)
% subsection matrices (end)

% section linear_algebra (end)


\section{Quantum computation overview} % (fold)
\label{sec:quantum_computation_overview}

Quantum computation proceeds via the application of reversible transformations --- Unitary
transformations.

The semantics of quantum computation can be defined as a $\dagger$-compact closed category as
introduced in \cite{abramsky02:traces,abramsky05:abstracttraces} and completely positive maps as
discussed in \cite{selinger05:dagger}.

\begin{definition}[Dagger Category]
  A \emph{Dagger Category}\cite{selinger05:dagger} is a category \C together with an operation
  $\dagger$ that is an involutive, identity on objects, contra-variant endofunctor on \C.
\end{definition}
Recalling first that a \emph{symmetric monoidal category} is a category \B with a bi-functor $\*$,
an object $I$ and natural isomorphisms:
\begin{align*}
  a_{A,B,C}&: (A\*B)\*C \to A\* (B\*C)\\
  c_{A,B} &:A\*B \to B\*A\\
  ul_A &:A \to I \* A
\end{align*}
with standard coherence conditions, as in \cite{maclan97:categorieswrkmath}. Note that we
also have a map $ur_A: A \to A\*I$ given by $ur_A = ul_A c_{I,A}$ Furthermore, a \emph{compact
closed category} \C is a symmetric monoidal category where each object $A$ has a dual $A^{*}$
together with the maps:
\begin{align*}
  \eta_A:I \to \dual{A} \* A\\
  \epsilon_A : A\*\dual{A} \to I
\end{align*}
such that
\[
  \xymatrix@C+8pt@R-10pt{
    A \ar[r]^{ur_A} \ar@{=}[dddrr]
      & A\* I \ar[r]^(.4){A\*\eta_A}
      & A\* (\dual{A}\*A) \ar[d]^{a^{-1}}\\
    & & (A\*\dual{A})\*A \ar[d]^{\epsilon\*A}\\
    & & I\* A \ar[d]^{ul^{-1}}\\
    & & A
  }
  \quad\text{ and  }\quad
  \xymatrix@C+8pt@R-10pt{
    \dual{A} \ar[r]^{ul_{\dual{A}}} \ar@{=}[dddrr]
      & I\*\dual{A} \ar[r]^(.4){\eta_{\dual{A}\*\dual{A}}}
      & (\dual{A}\*A)\*\dual{A} \ar[d]^{a}\\
    & & \dual{A}\*(A\*\dual{A}) \ar[d]^{\dual{A}\*\epsilon}\\
    & & \dual{A}\*I \ar[d]^{ur^{-1}}\\
    & & A
  }
\]

From the above, we can define a \emph{Dagger symmetric monoidal category} and a \emph{Dagger
compact closed category}. The latter is referred to as a \emph{strongly compact closed category} in
\cite{abramsky02:traces}, where they were initially introduced. In each case, the $\dagger$ functor
is added in a way that retains coherence with the bi-functor $\*$ and with the dualizing operator.
The coherence implies that the $\dgr{i} = i^{-1}$ for the SMC isomorphisms, that $\dgr{(f\*g)} =
\dgr{f}\*\dgr{g}$ for all maps $f,g$ in the symmetric monoidal category and that
\[
  \xymatrix{
    I \ar[dr]_{\eta_A} \ar[r]^{\dgr{\epsilon_A}} & A\* \dual{A} \ar[d]^{c} \\
    &\dual{A}\*A
  }
\]
commutes for all objects $A$ in the compact closed category.


\begin{example}[\rel]
  \rel is a dagger compact closed category with the dual of an object $A$ is $A$, $\*$ is the
  cartesian product and for $R:A\to B$, we have $\dual{R} = \dgr{R} = \{(y,x) | (x,y) \in R\}$.
\end{example}
\begin{example}[\fdh]
  The category of finite dimensional Hilbert spaces, \fdh is a dagger compact closed category with
  the dual of an object $H$ is the normal Hilbert space dual $H^{*}$, the space of continuous
  linear functions from $H$ to the base field. $\*$ is the normal Hilbert space tensor and and for
  $f:A\to B$, we have $\dgr{f}$ is the unique map such that $\langle f x | y \rangle = \langle y |
  \dgr{f}x \rangle$ for all $x\in A$, $y \in B$.
\end{example}

Additionally, if one has a dagger compact closed category with biproducts where the biproducts and
dagger interact such that $\dgr{p_i} = q_i$, this is called a \emph{biproduct dagger compact closed
category}.

In \cite{selinger05:dagger}, the author continues from this point: Starting with a biproduct dagger
compact closed category $\C$, he creates a new category, $\text{CPM}(\C)$ which has the same
objects as $\C$, but morphisms $f:A \to B$ in $\text{CPM}(\C)$ are given by maps $f:\dual{A} \* A
\to \dual{B} \* B$ in $\C$ which are \emph{completely positive}. Note that \rel and \fdh are
biproduct dagger compact closed categories.

From this, the category $\text{CPM}(\C)^{\+}$, the free biproduct completion of $\text{CPM}(\C)$ is
formed, which is suitable for describing quantum computation semantics. For example, given \fdh as
our starting point, the tensor unit $I$ is the field of complex numbers. The type of
$\mathbf{qubit}$ (in \fdh and by lifting, in $\text{CPM}(\fdh)^{\+}$) is given as $I\+I$. At this
stage, the necessity of the CPM construction to model physical reality can be seen in the following
as in \fdh, the morphisms initialization of a qubit: $init:I\+I \to \mathbf{qubit}$ and destructive
measure: $meas: \mathbf{qubit} \to I\+I$ are inverses. However, in $\text{CPM}(\fdh)^{\+}$, these
same maps are given as
\[
  \dual{\mathbf{qubit}} \* \mathbf{qubit} \xrightarrow{meas} I\+
    I \xrightarrow{init}\dual{\mathbf{qubit}} \* \mathbf{qubit}
\]
by the formulae:
\[
  meas
  \begin{pmatrix}
    a & b \\
    c & d
  \end{pmatrix}
  = (a,d), \qquad init(a,d) =
  \begin{pmatrix}
    a &0 \\
    0 & d
  \end{pmatrix}.
\]
Therefore, the maps are not inverses and reflect the physical reality.


\subsection{Density matrix representation}\label{sec:density}
An alternate representation of quantum states, both pure and mixed, is via \emph{density matrices}.
If the state of a system is represented by some column vector $u$, then the matrix $u u^{*}$ is its
density matrix. Note that if $u = \nu v$ for some complex scalar $\nu$ with norm 1, then $u u^{*} =
(\nu v) (\nu v)^{*} = \nu \bar{\nu} v v^{*} = v v^{*} $. For the mixed state $\sum
\nu_{i}\{v_{i}\}$, the density matrix is $\sum \nu_{i}v_{i}v_{i}^{*}$. Density matrices are
positive hermitian matrices with trace $\le 1$. Note that the trace of the density matrix is the
probability the system has reached this particular value in the computation.

The result of applying the unitary transform $U$ to a state $u$ represented by the density matrix
$A$ is $UAU^{*}$. The measurement operation on a density matrix is derived from the measurement
effects on the \qubit. For example, consider the density matrix for $q=\alpha\kz+\beta\ko$,
$\begin{pmatrix}\alpha\bar{\alpha}&\alpha\bar{\beta}\\ \beta\bar{\alpha} &
\beta\bar{\beta}\end{pmatrix}$. Measuring this \qubit gives either
$\begin{pmatrix}\alpha\bar{\alpha}&0\\ 0& 0\end{pmatrix}$ with probability $|\alpha|^{2}$ or
$\begin{pmatrix}0&0\\ 0 & \beta\bar{\beta}\end{pmatrix}$ with probability $|\beta|^{2}$. If the
results of the measurment are not used, this will result in the density matrix
$\begin{pmatrix}\alpha\bar{\alpha}&0\\ 0 & \beta\bar{\beta}\end{pmatrix}$. This extends linearly so
that if a \qubit is measured in the system whose density matrix is \qsmat{A}{B}{C}{D}, the result
will be the mixed density matrix \qsmat{A}{0}{0}{D}.

It is possible to create a complete partial order on density matrices.

\begin{definition}[L\"owner partial order]\label{def:lownerorder}
  For square complex matrices $A,B$ of the same size, define $A \le B$ if $B-A$ is positive.
\end{definition}

\begin{lemma} \label{lemma:cpodensity}
  Designate $D_{n}$ to be the density matrices of size $n\times n$, then the poset $(D_{n}, \le)$
  is a complete partial order.
\end{lemma}
\begin{proof}
  See \cite{selinger04:qpl}, pp 13--14.
\end{proof}

% section quantum_computation_overview (end)
%!TEX root = /Users/gilesb/UofC/thesis/phd-thesis/phd-thesis.tex
\section{Dagger categories}\label{sec:daggercategories}
Dagger categories generalize the concepts of Hilbert spaces that are required to model quantum
computation. These were introduced in \cite{abramsky04:catsemquantprot} as \emph{strongly compact
closed categories}, an additional structure only on compact closed categories.

Before introducing dagger categories, we define symmetric monoidal categories and compact closed
categories.


\subsection{Symmetric Monoidal Categories} % (fold)
\label{sub:categories_with_additional_structure}

\begin{definition}\label{symmetricmonoidalcat}
  A \emph{symmetric monoidal category}\cite{barr:ctcs,maclan97:categorieswrkmath} \cD{} is a
  category equipped with a monoid $\*$ (a bi-functor $\*:\cD \times \cD \to \cD$) together with
  four families of natural isomorphisms:  $a_{A,B,C}:A\*(B\*C) \to (A\*B)\*C$, $u^r_{A}:A\*I\to A$,
  $u^l_{A}:I\*A \to A$ and $c_{A,B}:A\*B \to B\* A$, which satisfy coherence diagrams and
  equations shown in Figures~\ref{fig:SMC_pentagon}, \ref{fig:SMC_unit}, \ref{fig:SMC_commutes},
  \ref{fig:SMC_unit_symmettry} and \ref{fig:SMC_associativity_symmetry}. The isomorphisms are
  referred to as the \emph{structure isomorphisms}  for the symmetric monoidal category. $I$ is the
  unit of the monoid. A symmetric monoidal category where each of $a_{A,B,C}$, $u^r_{A}$, $u^l_{A}$
  and $c_{A,B}$ are identity maps is called a \emph{strict symmetric monoidal category}.
\end{definition}

\begin{figure}[!htbp]
\[
  \xymatrix@C+25pt{
    A\*(B\*(C\*D) \ar[r]^{a_{A,B,(C\*D)}} \ar[d]_{1\*a_{B,C,D}}
      & (A\*B)\*(C\*D) \ar[r]^{a_{(A\*B),C,D}}
      & ((A\*B)\*C)\*D \ar[d]^{a_{A,B,C}\*1}\\
    A\*((B\*C)\*D) \ar[rr]_{a_{A,(B\*C),D}}
      && (A\*(B\*C))\*D
  }
\]
\caption{Pentagon diagram for associativity in an SMC.}\label{fig:SMC_pentagon}
\end{figure}
\begin{figure}[!htbp]
\[
  \xymatrix@C+5pt@R+10pt{
    A\*(I\*B) \ar[rr]^{a_{A,I,B}} \ar[dr]_{1\*u^l_B}
      && (A\*I)\*B \ar[dl]^{u^r_A \* 1}\\
      &A\*B
  }
\]
\[\text{ and } u^r_I = u^l_I: I\* I \to I\]
\caption{Unit diagram and equation in an SMC.}\label{fig:SMC_unit}
\end{figure}
\begin{figure}[!htbp]
\[
  \xymatrix@C+5pt@R+10pt{
    A\*B \ar[r]^{c_{A,B}} \ar@{=}[dr]
      & B\*A \ar[d]^{c_{B,A}}\\
      &A\*B
  }
\]
\caption{Symmetry in an SMC.}\label{fig:SMC_commutes}
\end{figure}
\begin{figure}[!htbp]
\[
  \xymatrix@C+5pt@R+10pt{
    A\*I \ar[rr]^{c_{A,I}} \ar[dr]_{u^r_A}
      && I\*A \ar[dl]^{u^l_A}\\
      &A
  }
\]
\caption{Unit symmetry in an SMC.}\label{fig:SMC_unit_symmettry}
\end{figure}
\begin{figure}[!htbp]
\[
  \xymatrix@C+15pt@R+10pt{
    (A\*B)\*C \ar[r]^{c_{(A\*B),C}} \ar[d]_{a^{-1}_{A,B,C}}
      & C\*(A\*B) \ar[d]^{a_{C,A,B}}\\
    A\*(B\*C) \ar[d]_{1\*c_{B,C}}
      & (C\*A)\*B \ar[d]^{c_{C,A}\*1}\\
    A\*(C\*B) \ar[r]^{a_{A,C,B}}
      & C\*(A\*B)\text{ ,}
  }\qquad
  \xymatrix@C+15pt@R+10pt{
    A\*(B\*C) \ar[r]^{c_{A,(B\*C)}} \ar[d]_{a_{A,B,C}}
      & (B\*C)\*A \ar[d]^{a^{-1}_{B,C,A}}\\
    (A\*B)\*C \ar[d]_{c_{A,B}\*1}
      & B\*(C\*A) \ar[d]^{1\*c_{C,A}}\\
    (B\*A)\*C \ar[r]^{a^{-1}_{B,A,C}}
      & B\*(A\*C)
  }
\]
\caption{Associativity symmetry in an SMC.}\label{fig:SMC_associativity_symmetry}
\end{figure}
The essence of the coherence diagrams is that any diagram composed solely of the structure
isomorphisms will commute.

\begin{definition}\label{def:compactclosedcat}
A \emph{compact closed category} \cD{} is a symmetric monoidal category with tensor $\*$ where each
object $A$ has a dual $A^{*}$. Additionally, there must exist families of maps $\eta_{A}: I \to
A^{*} \* A$ (the \emph{unit}) and $\epsilon_{A}: A\*A^{*}\to I$ (the \emph{counit}) such that
\[
  \xymatrix@C+20pt{
    A \ar[r]^{u_{A}} \ar@{=}[d]  & A\*I \ar[r]^{1\*\eta_{A}}
        & A\* (A^{*}\*A) \ar[d]^{a_{A,A^{*},A}} \\
    A & I\* A \ar[l]^{u_{A}^{-1}} & (A\* A^{*})\*A \ar[l]^{\*\epsilon_{B}\*1}
    }
  \]
commutes and so does the similar one based on $A^{*}$.
\end{definition}

Given a map $f:A\to B$ in a compact closed category,  define the map $f^{*}:B^{*} \to A^{*}$ as
\[
  \xymatrix@C+10pt{
    B^{*}\ar[r]^{u_{B^{*}}} \ar[d]_{f^{*}}& I\*B^{*} \ar[r]^{\eta_{A}\*1}
      & A^{*}\*A\*B^{*}\ar[d]^{1\*f\*1}\\
    A^{*}&    A^{*}\*I\ar[l]^{u_{A^{*}}^{-1}}  &   A^{*}\*B\*B^{*}\ar[l]^{1\*\epsilon_{B}}.
  }
\]


% subsection categories_with_additional_structure (end)

\subsection{Definitions}\label{sec:daggerdefinitions}

Although dagger categories were introduced in the context of compact closed categories, the concept
of a dagger is definable independently. This was first done in \cite{selinger05:dagger}.

\begin{definition}\label{def:daggercat}
  A \emph{dagger operator} on a category $D$ is a functor $\dagger:\cD^{op}\to \cD$, which is
  involutive in the sense that it is the identity on objects. A \emph{dagger category} is a category
  that has a dagger operator.
\end{definition}

Typically, the dagger is written as a superscript on the morphism. So, if $f:A\to B$ is a map in
\cD, then $\dgr{f}:B\to A$ is a map in \cD{} and is called the \emph{adjoint} of $f$. A map where
$f^{-1} = \dgr{f}$ is called \emph{unitary}. A map $f:A\to A$ with $f=\dgr{f}$ is called
\emph{self-adjoint} or \emph{Hermitian}.

\begin{definition}\label{def:daggersmc}
  A \emph{dagger symmetric monoidal category} is a symmetric monoidal category \cD{} with a dagger
  operator such that:
  \begin{enumerate}
    \item For all maps $f:A\to B$ and $g:C\to D$, $\dgr{(f\*g)} = \dgr{f}\*\dgr{g}:B\*D \to A\* C$;\label{defitem:dagger_smc_one}
    \item The monoid structure isomorphisms $a_{A,B,C}:(A\*B)\* C\to A\*(B\*C)$, $u^l_{A}:I\*A\to
      A$, $u^r_{A}:A\*I \to A$ and  $c_{A,B}:A\*B \to B\*A$ are unitary.\label{defitem:dagger_smc_two}
  \end{enumerate}
\end{definition}


\begin{definition}\label{def:daggercompact}
  A \emph{dagger compact closed category} \cD{} is a dagger symmetric monoidal category
  that is compact closed where the diagram
  \[
    \xymatrix @C+20pt @R+10pt{
      I \ar[r]^{\epsilon^{\dagger}_{A}} \ar[dr]_{\eta_{A}} &A\*A^{*}\ar[d]^{c_{A,A^{*}}}\\
      &A^{*}\* A
    }
  \]
  commutes for all  objects $A$ in \cD.
\end{definition}

\begin{lemma}\label{lemma:daggerbiproducts}
If \cD{} is a dagger category with biproducts, with injections $in_{1},in_{2}$ and projections
$p_{1},p_{2}$, then the following are equivalent:
\begin{enumerate}
  \item $\dgr{p_{i}} = in_{i}, i=1,2$, \label{ldpdgrpisq}
  \item $\dgr{(f\biproduct g)} = \dgr{f}\biproduct \dgr{g}$ and $\dgr{\Delta} = \nabla$,\label{ldpddeltisnab}
  \item $\dgr{\<f,g\>} = [\dgr{f},\dgr{g}]$,\label{ldpdcopisprod}
  \item The map $[\dgr{p_{1}},\dgr{p_{2}}]: \dgr{A} \biproduct \dgr{B} \to \dgr{(A\biproduct B)}$ is
    the identity map.\label{ldpcommute}
%the below diagram commutes:
%  \[
%    \xymatrix @C+20pt @R+10pt{
%      \dgr{A} \biproduct \dgr{B} \ar[d]_{id} \ar[dr]^{[\dgr{p_{1}},\dgr{p_{2}}]}\\
%      A\biproduct B\ar[r]_{id}&\dgr{(A\biproduct B)}.
%    }
%  \]
\end{enumerate}
\end{lemma}
\begin{proof}
  \begin{description}
    \item[\ref{ldpdgrpisq}$\implies$\ref{ldpddeltisnab}] To show $\dgr{\Delta} = \nabla$,
    draw the product cone for $\Delta$,
    \[
      \xymatrix {
        &A \ar[d]^{\Delta} \ar[dr]^{id} \ar[dl]_{id}\\
        A
         & A\biproduct A \ar[l]^{p_{1}}  \ar[r]_{p_{2}}
         & A
      }
    \]
    and apply the dagger functor to it. As $\dgr{p_{i}} = in_{i}$, and $\dagger$ is identity on
    objects, this is now a coproduct diagram and therefore $\dgr{\Delta} = \nabla$.

    For $\dgr{(f\biproduct g)} = \dgr{f}\biproduct\dgr{g}$, start with the diagram defining
    $f\biproduct g$ as a product of the arrows:
    \[
      \xymatrix{
        A\ar[d]_{f}  & A\biproduct B \ar[l]_{p_{1}} \ar[r]^{p_{2}} \ar[d]^{f\biproduct g}&A \ar[d]^{g}\\
        C & C\biproduct D \ar[l]^{p_{1}} \ar[r]_{p_{2}}  & D.
      }
    \]
    Then, apply the dagger functor to this diagram. This is now the diagram defining the
    co-product of maps and therefore $\dgr{(f\biproduct g)} = \dgr{f}\biproduct\dgr{g}$.
    \item[\ref{ldpddeltisnab}$\implies$\ref{ldpdcopisprod}] The calculation showing this is
      \begin{eqnarray*}
        &[\dgr{f},\dgr{g}] & = \nabla; (\dgr{f}\biproduct \dgr{g})\\
        & &=\dgr{\Delta}; (\dgr{f}\biproduct \dgr{g})\\
        & &=\dgr{\Delta}; \dgr{(f\biproduct g)}\\
        & & = \dgr{((f\biproduct g);\Delta)}\\
        & & = \dgr{\<f,g\>}
      \end{eqnarray*}
    \item[\ref{ldpdcopisprod}$\implies$\ref{ldpcommute}]
      Under the assumption,
      \[
        [\dgr{p_{1}},\dgr{p_{2}}] = \dgr{\<p_{1},p_{2}\>}=\dgr{id}=id.
      \]
    \item[\ref{ldpcommute}$\implies$\ref{ldpdgrpisq}] As $[in_{1},in_{2}]:\dgr{A} \biproduct \dgr{B}
      \to \dgr{A} \biproduct \dgr{B} = id = [\dgr{p_{1}},\dgr{p_{2}}]$, we immediately have
      $\dgr{p_{1}} = in_{1}$ and $\dgr{p_{2}} = in_{2}$.
%
%Using the injections and under
%    the assumption, the following diagram commutes:
%      \[
%        \xymatrix @C+20pt @R+10pt{
%          \dgr{A} \biproduct \dgr{B} \ar[d]_{id} \ar[dr]^{[\dgr{p_{1}},\dgr{p_{2}}]}\ar[r]^{[in_{1},in_{2}]}
%            & \dgr{A} \biproduct \dgr{B} \ar[d]^{id}\\
%          A\biproduct B\ar[r]_{id}&\dgr{(A\biproduct B)}
%        }
%      \]
%      and therefore,
  \end{description}
\end{proof}

\begin{definition} \label{def:biproductdaggerccc}
  A \emph{biproduct dagger compact closed category} is a dagger compact closed category with
  biproducts where the conditions of lemma \ref{lemma:daggerbiproducts} hold.
\end{definition}
\subsection{Examples of dagger categories}

\begin{example}[\fdh]\label{ex:fdhilbert_is_dagger_category}
The category of finite dimensional Hilbert spaces is the motivating example for
the creation of the dagger and is, in fact, a biproduct dagger compact closed category. The
biproduct is the direct sum of Hilbert spaces and the tensor for compact closure is the standard
tensor of Hilbert spaces. The dual $H^{*}$ of a space $H$ is the space of all continuous linear
functions from $H$ to the base field. The dagger is defined via the adjoint as being the unique map
$\dgr{f}:B\to A$ such that $\<f a|b\> = \<a | \dgr{f} b\>$ for all $a\in A, b\in B$.
\end{example}

\begin{example}[\rel]\label{ex:rel_is_dagger_category}
The category \rel of sets and relations has the tensor $S\*T = S\times T$, the
Cartesian product and the biproduct $S\biproduct T = S+T$, the disjoint union. This is compact
closed under $A^{*} = A$ and the dagger is the ${}^*$ operation, the relational converse. That is,
if the relation $R=\{(s,t)|s\in S, t\in T\}:S\to T$, then $\dgr{R}=R^*=\{(t,s)|(s,t)\in R\}$.
\end{example}

\begin{example}[Inverse categories]\label{ex:inverse_category_is_dagger_category}
An inverse category \X is also a dagger category when the dagger is defined as the partial inverse.
The unitary maps are the total maps. When the inverse category \X is also a
symmetric monoidal category where the monoid $\*$ is actually a restriction bi-functor, then \X is
a dagger symmetric monoidal category.

Requirement \ref{defitem:dagger_smc_one} of Definition~\ref{def:daggersmc}  is fulfilled, as
\[
  (f\*g) \inv{(f\*g)} = \rst{f\*g}=\rst{f} \*\rst{g} =
   f\inv{f} \* g \inv{g} = (f\*g) (\inv{f} \* \inv{g})
\]
and since the partial inverse of $f\*g$ is unique, $\inv{(f\*g)} = \inv{f} \* \inv{g}$.
Requirement \ref{defitem:dagger_smc_two} is that the structure isomorphisms are unitary. This is, of
course, true as each of them are isomorphisms, hence total and therefore unitary.
\end{example}
%%% Local Variables:
%%% mode: latex
%%% TeX-master: "../../phd-thesis"
%%% End:

%!TEX root = /Users/gilesb/UofC/thesis/phd-thesis/phd-thesis.tex
\section{Semantics of quantum computation}% (fold)
\label{sec:semanticsquantum}

\subsection{Semantics of pure quantum computations}\label{sec:puresemantics}
In \cite{abramsky04:catsemquantprot}, the authors approach the creation of a categorical semantics
for quantum computation independently of a specific language. Rather, they use finitary quantum
mechanics as their reference point.

Finitary quantum mechanics consists of the following:
\begin{enumerate}
  \item The system's state space is represented by a finite dimensional Hilbert space $H$.
    \label{lis:qfm1}
  \item The basic type of the system is that of \qubit --- 2-dimensional Hilbert space --- with the
    computational basis $\{\kz, \ko\}$.\label{lis:qfm2}
  \item Compound systems are tensor products of the components. This is what enables
    \emph{entanglement} as the general form of the system $H\*J$ where $H$ and $J$ are Hilbert
    spaces is
    \[
      \sum_{i=1}^{n}\alpha_{i} (u_{i} \* v_{i})
    \]
    where $u_{i}$ is a basis element of $H$ and $v_{i}$ is a basis element of $J$.\label{lis:qfm3}
  \item The basic transforms are \emph{unitary transformations}. \label{lis:qfm4}
  \item The measurements performable are \emph{self-adjoint} (hermitian) operators - with two
    sub-steps:\label{lis:qfm5}
    \begin{enumerate}
      \item The actual act of measurement. (Preparation).\label{lis:qfm5a}
      \item The communication of the results of the measurement. (Observation).\label{lis:qfm5b}
    \end{enumerate}
\end{enumerate}
The above definition does allow for the possibility of mixed states, as described in section
\ref{sec:density}, but for the remainder of this section, it is assumed both steps of the
measurement are carried out, resulting in pure states only.

\cite{abramsky04:catsemquantprot} gives the interpretation of finitary quantum mechanics in the
context of a biproduct dagger compact closed category, \cD.
\begin{description}
  \item[\ref{lis:qfm1}.] An $n-$dimensional state space $S$ is an object of \cD,
    together with a unitary isomorphism $base_{A}:\+^{n}I\to A$.
  \item[\ref{lis:qfm2}.] A \qubit is a 2 dimensional state space $Q$ with the computational basis
    $base_{Q}:I\+I \to Q$.
  \item[\ref{lis:qfm3}.] Compound systems $A,B$ are described by $A\*B$ and
    $base_{A\*B} = \phi (base_{A}\*base_{B})$ where $\phi:\+^{nm}I \cong(\+^{n}I)\*(\+^{m}I)$ is
    the isomorphism obtained by repeated application of distributivity isomporphisms.
  \item[\ref{lis:qfm4}.] The basic transformations are unitary transformations, i.e., $f$, where
    $\dgr{f} = f^{-1}$.
  \item[\ref{lis:qfm5a}.] A preparation is a morphism $P:I \to A$ which has a corresponding unitary
    morphism $f_{P}:\+^{n}I\to\+^{n}I$ and
    \[
      \xymatrix{
        I \ar[r]^{P} \ar[d]_{i_{1}}& A\\
        \+^{n}I \ar[r]_{f_{P}} & \+^{n}I \ar[u]_{base_{A}}
      }
    \]
    commutes.
  \item[\ref{lis:qfm5b}.] An observation  is an isomorphism $O = \+^{n}O_{i}$ with components
    $O_{i}:A \to I$ which has an unitary automorphism $f_{O}:\+^{n}I\to\+^{n}I$ such that
    \[
      \xymatrix{
        A \ar[r]^{O_{i}} & I\\
        \+^{n}I \ar[r]_{f_{O}}  \ar[u]_{base_{A}} & \+^{n}I \ar[u]_{p_{i}}
      }
    \]
    commutes for all $i=1,\ldots,n$. The observational branches are the individual $O_{i}:A \to I$.
\end{description}
Additionally, the biproduct $\+$ represents distinct branches resulting from measurement.
Accordingly, any operation on a biproduct must be an explicit biproduct, that is $f:A\+B\to C\+D$
will be $f_{1}\+f_{2}$ with $f_{1}:A\to C$ and $f_{2}:B\to D$.

The authors go on to show how this interpretation is sufficient to model quantum teleportation,
logic gate teleportation and entanglement swapping.


\subsection{Complete positivity}\label{sec:completepositivity}
Given a $\dagger$-compact closed category, it is possible to construct its category of completely
positive maps.

\begin{definition}[Positive map]\label{def:positivemap}
  A map $f:A\to A$ in a dagger category is called \emph{positive} if there is an object $B$ and a
  map $g:A\to B$ with $f = g \dgr{g}$
\end{definition}

\begin{definition}[Trace]\label{def:tracecp}
  For $f:A\to A$ in a compact closed category, its \emph{trace} is defined as $tr\, f:I\to I =
  \eta_{A} ; c_{A^{*},A} ; (f\*A^{*}) ; \epsilon$.
\end{definition}

The following lemma gives some properties of positive maps:

\begin{lemma}\label{lemma:positivemaps}
  In any biproduct dagger compact closed category, the following hold:
  \begin{enumerate}
    \item{} $f$ positive $\implies$ $h f \dgr{h}$ is positive for all maps $h$.
    \item{} $id_{A}$ is positive.
    \item If $f:A\to A$ and $g:B\to B$ are positive, so are $f\*g$ and $f\+g$.
    \item $0_{A,A}$ is positive. If $f,g:A\to A$ is positive, so is $f+g$.
    \item $f$ positive $\implies$ $\dgr{f}=f$.
    \item $f$ positive $\implies$ $f^{*}$ and $tr\ f$ are positive.
    \item $f,g:A\to A$ positive $\implies$ $tr (g\,f)$ is positive.
  \end{enumerate}
\end{lemma}
\begin{proof}
  The first six items follow immediately from the definitions and how structure is preserved for
  $(\_)^{\dagger}$. For item 6, note that $g = h\, \dgr{h}$ and $tr(g\,f) = tr(\dgr{h}\,f\,h)$
  which is positive by points 1 and 5.%FIXME - why
\end{proof}

\begin{definition}\label{def:name}
  In a compact closed category, the \emph{name} of a map $f:A\to B$ is the map $\ulcorner f
  \urcorner:I \to A^{*} \* B$ defined as $\eta_{A}; (1\*f)$. This is also called the \emph{matrix} of
  $f$.
\end{definition}

In the case of a positive map $f$, $\ulcorner f \urcorner$ is referred to as a \emph{positive
matrix}.

\begin{definition}\label{def:completelypositive}
  In a dagger compact closed category, a map $f:A^{*}\*A \to B^{*}\* B$ is \emph{completely positive}
  if for all objects $C$ and all positive matrices $f: I \to C^{*} \* A^{*} \* A \* C$ the morphism
  $g ; (1\*f\*1):I \to C^{*} \* B^{*}\* B \* C$ is a positive matrix.
\end{definition}

This now allows us to define the CPM construction.

\begin{definition}\label{def:cpmconstruction}
  Given a dagger compact closed category $\cD$, define \specialcat{CPM(d)} as the category with the
  same objects as $\cD$, and a map $f:A\to B$ in \specialcat{CPM(d)} is a completely positive map
  $f:A^{*}\*A \to B^{*}\* B$ in \cD.
\end{definition}

\specialcat{CPM(d)} is also a dagger compact closed structure, inheriting its tensor from \cD.
There is a functor $F:\cD \to \specialcat{CPM(d)}$ defined as $F(A) = A$ on objects and $F(f)=
f_{*}\*f$ on maps. The image of the structure maps under $F$ are structure maps for
\specialcat{CPM(d)}. The dagger of a map $f$ is the same as its dagger in \cD.

\subsubsection{Biproduct completion}\label{sec:biproduct}
When the \specialcat{CPM} construction is applied to a biproduct dagger compact closed category, it
will not in general retain biproducts. However, it will be monoid enriched by lemma
\ref{lemma:positivemaps}. This allows us to create the biproduct completion.

The biproduct completion of a category \cD, which is enriched in commutative monoids is the
category $\cD^{\+}$ which has as objects finite sequences $\<A_{1},\ldots,A_{n}\>$ where $n\ge 0$.
The morphisms of $\cD^{\+}$ are matrices of the morphisms of \cD. Application and composition of
morphisms is via matrix multiplication. The functor $F(A) = \<A\>$, $F(f)=[f]$ is an embedding of
\cD{} in $\cD^{\+}$. If \cD{} is compact closed and the tensor is linear (i.e., interacts with the
enrichment in a linear fashion), then $\cD^{\+}$ is also compact closed.

Furthermore, if \cD{} is a dagger category and the dagger is linear, then $\cD^{\+}$ will be a
dagger category. The dagger of a map $(f_{i,j})$ in $\cD^{\+}$ is $(\dgr{(f_{j,i})})$.

This gives us the following theorem:

\begin{theorem}\label{theorem:biproductcompletion}
Given \cD, a biproduct dagger compact closed category, \cpm{d} is enriched in commutative monoids
as a dagger compact closed category. Therefore, it is possible to construct its biproduct
completion, \bcpm{d}.
\end{theorem}

Note that the canonical embedding from above, $F$, while it preserves the dagger compact closed
structure, it does \emph{not} preserve biproducts.

%%% Local Variables:
%%% mode: latex
%%% TeX-master: "../../phd-thesis"
%%% End:
