%!TEX root = /Users/gilesb/UofC/thesis/phd-thesis/phd-thesis.tex
\chapter{Matrix categories} % (fold)
\label{cha:matrix_categories}

Inverse categories with all disjoint sums are, in fact, Unique Decomposition
Categories\cite{haghverdi:2000phd}. However, what about inverse categories with disjoint joins?
The previous chapter showed that in the presence of a disjoint sum tensor, an inverse category \X
with disjoint joins also has disjoint sums. This chapter will show how to add a disjoint sum to
such an arbitrary \X.



\section{Matrices} % (fold)
\label{sec:matrix_representation}
In this section, we will show that when given an inverse  category \X with a disjoint joins, one
can define a matrix category based on $\X$, called \imatx. Furthermore, we will show that \imatx is
an inverse category with disjoint sums and that $\X$ embeds within this category.

The types of matrices allowed in the matrix category are subject to certain constraints:
\begin{definition}\label{def:disjoint_sum_matrix}
  Suppose \X is an inverse category with disjoint joins, then a \emph{disjoint sum matrix}
  in \X is a matrix of maps $[f_{i j}]$ where $i \in \{1,\ldots, n\}$ and $j \in \{1,\ldots, m\}$ with
  $f_{i\jay}:A_i \to B_\jay$ which satisfy the two conditions:
  \begin{align}
    &\text{Rows: For each }i,\ \cdperp\{\rst{f_{i \jay}}\}_{\jay=1,\ldots,m}.
      \label{eq:disjoint_sum_matrix_row_perp}\\
    &\text{Columns: For each }j,\ \cdperp\{\wrg{f_{i \jay}}\}_{i=1,\ldots,n}.
    \label{eq:disjoint_sum_matrix_column_perp}
  \end{align}
  In other words, for each column, the ranges of the functions in that column are disjoint and for
  each row the restrictions of the functions in that row are disjoint.
\end{definition}

  % In the above and following we will use the notation $\cp{i}$ for the $i^{\text{th}}$ injection
  % map of the disjoint sum, with $i$ starting at 1. This simply extends the notation introduced in
  % Definition~\ref{def:disjointness_tensor}.

  We will show that this type of matrix corresponds to maps in the category \imatx. In \imatx the
  composition is given by ``matrix multiplication'', with the operations of multiplication and
  addition  replaced with composition in \X and the disjoint join respectively.

\begin{definition}\label{def:inverse_matrix_category}
  Given an inverse category \X with disjoint joins, we define the \emph{inverse
  matrix category} of \X, \imatx, as follows:
  \rcategory{Lists of the objects of $\X$}{
    Disjoint sum matrices $[f_{i\jay}]: [A_i] \to [B_\jay]$. In such a matrix each individual map
    $f_{i\jay}:A_i \to B_{\jay}$ is a map in $\X$}{
    The disjoint sum matrix $I$ --- A diagonal matrix with $1:A_i \to A_i$ at the $i,i$ position and
  zero maps elsewhere}{
    Given $[f_{i\jay}]: [A_i] \to [B_\jay]$ and $[g_{j k}] : [B_\jay] \to [C_k]$, then
    $[h_{i k}] = [f_{i\jay}][g_{\jay k}]: [A_i] \to [C_k]$ is defined as
    $h_{i k} = \djoinbig_\jay f_{i \jay}g_{\jay k}$}{
    We set $\rst{[f_{i\jay}]}$ to be $[f'_{i\jay}]$ where $f'_{i\jay} = 0$ when $i\neq \jay$
    and $f'_{i i} = \djoin_\jay \rst{f_{i\jay}}$}
  In the following, we will use the notation $\diag[d_j]$ for diagonal matrices where the
  $j,j$ entry is $d_j$.
\end{definition}

\begin{lemma}\label{lem:inverse_matrix_category_is_a_restriction_category}
  When $\X$ is an inverse category with disjoint joins, \imatx is an inverse category.
\end{lemma}
\begin{proof}
  We need to show the following:
  \begin{itemize}
    \item Composition is well defined and associative.
    \item The restriction is well defined.
    \item Each map must have a partial inverse.
  \end{itemize}

  \emph{Composition is well defined:} Consider $[h_{i k}] = [f_{i j}] [g_{j k}]$ where $[f_{i j}]:
  [A_1,\ldots, A_n] \to [B_1,\ldots, B_m]$ and $[g_{j k}]: [B_1,\ldots, B_m] \to [C_1,\ldots,
  C_\ell]$. As each of $ [f_{i j}]$ and $ [g_{j k}]$ are disjoint sum matrices, by \axiom{Dis}{7} we know that
  $\cdperp \{f_{i \jay} g_{\jay k}\}$ for each choice of $i$ and $k$. Hence, we know the composition
  $h_{i k} = \djoinbig_\jay f_{i \jay} g_{\jay k}$ is defined and $h_{i k}:A_i\to C_{k}$. We must
  now show that the $h_{ik}$ satisfy Equation~\ref{eq:disjoint_sum_matrix_row_perp} and
  Equation~\ref{eq:disjoint_sum_matrix_column_perp}. Calculating the restriction of row elements,
  \[
    \rst{h_{ik}} = \rst{ \djoinbig_\jay f_{i \jay} g_{\jay k}} = \djoinbig_\jay \rst{ f_{i \jay}
      g_{\jay k}}.
  \]
  By \axiom{DJ}{4}, these are disjoint for the $i$'s when each component is disjoint. But each
  component is of the form $\rst{f_{i\jay}g_{\jay k}}$ and by
  Lemma~\ref{lem:restriction_cats_are_partial_order_enriched}, it is less than or equal to
  $\rst{f_{i\jay}}$ each of which are disjoint by assumption. Therefore, by \axiom{Dis}{3} the
  restriction of each of the entries in a row of $[h_{i k}]$ are disjoint. By a similar argument,
  the range of each of the entries in a column of $[h_{i k}]$ are disjoint.

  Thus the matrix $[h_{i k}]$ is a disjoint sum matrix and is in the category. Therefore
  composition is well-defined.

  \emph{Associativity of composition.} We have
  \begin{align*}
    ([f_{i j}] [g_{j k}]) [h_{k \ell}]
        &= \left[(\djoinbig_j f_{i j}  g_{j k})\right] [h_{k \ell}]\\
        &=\left[ \djoinbig_k (\djoinbig_j f_{i j} g_{j k}) h_{k \ell}\right]\\
        &=\left[ \djoinbig_j f_{i j}  (\djoinbig_k g_{j k} h_{k \ell})\right]\\
        &=[ f_{i j} ]  ([g_{j k}] [h_{k \ell}]).
  \end{align*}

  \emph{The restriction axioms.}
  \begin{align*}
    \rone \quad& \rst{[f_{i\jay}]}[f_{i\jay}] =
      \begin{bmatrix}
        (\djoin_j \rst{f_{1\jay}}) f_{11} & \cdots & (\djoin_\jay \rst{f_{1 \jay}}) f_{1n}\\
        & \vdots & \\
        (\djoin_\jay \rst{f_{m\jay}}) f_{m1} & \cdots & (\djoin_\jay \rst{f_{m\jay}}) f_{m n}
      \end{bmatrix} = [f_{i\jay}].\\
  \end{align*}
  \begin{align*}
    \rtwo \quad & \rst{[f_{i\jay}]} \rst{g_{i\jay}} =  \rst{g_{i\jay}} \rst{[f_{i\jay}]}
      \text{ as diagonal matrices commute and }\djoin\text{ is commutative.}\\
  \end{align*}
  \begin{align*}
    \rthree \quad & \rst{\rst{[f_{i\jay}]}[g_{\jay k}]} =
      \rst{\diag[\djoin_\jay \rst{f_{1\jay}},\ldots,\djoin_\jay \rst{f_{n \jay}}] [g_{\jay k}]}
      \\
      & = \rst{\begin{bmatrix}
        \djoin_\jay \rst{f_{1\jay}} g_{11} & \ldots & \djoin_\jay \rst{f_{1 \jay}} g_{1 k}\\[-4pt]
        & \vdots & \\[-4pt]
        \djoin_\jay \rst{f_{n\jay}} g_{n 1} & \ldots & \djoin_\jay \rst{f_{n \jay}} g_{n k}\\
      \end{bmatrix}} \\
      & = \diag[\djoin_k (\rst{\djoin_\jay (\rst{f_{1\jay}} g_{1k})}), \ldots,
                 \djoin_k (\rst{\djoin_\jay(\rst{f_{n\jay})} g_{n k}})]\\
      & = \diag[\djoin_k (\djoin_\jay(\rst{f_{1\jay}}) \rst{g_{1k}}), \ldots,
                 \djoin_k (\djoin_\jay(\rst{f_{n\jay}}) \rst{g_{n k}})]\\
      & = \diag[(\djoin_\jay(\rst{f_{1\jay}}) \djoin_k  \rst{g_{1k}}), \ldots,
                 (\djoin_\jay(\rst{f_{n\jay}}) \djoin_k  \rst{g_{n k}})] = \rst{[f_{i\jay}]}\,\rst{[g_{\jay k}]}.\\
  \end{align*}
  \begin{align*}
      \rfour \quad & [f_{i\jay}]\rst{[g_{\jay k}]} =
        [f_{i\jay}]\diag_\jay[\djoin_k \rst{g_{\jay k}}]
        \\
      & = \begin{bmatrix}
        f_{11}\djoin_k \rst{g_{1 k}} &\ldots & f_{1n}\djoin_k \rst{g_{n k}} \\[-4pt]
        & \vdots & \\[-4pt]
        f_{m1}\djoin_k \rst{g_{1 k}} &\ldots & f_{m n}\djoin_k \rst{g_{n k}}
      \end{bmatrix}\\[10pt]
      & = \begin{bmatrix}
        \djoin_k f_{11}\rst{g_{1 k}} &\ldots & \djoin_k f_{1n}\rst{g_{n k}} \\[-4pt]
        & \vdots & \\[-4pt]
        \djoin_k f_{m1}\rst{g_{1 k}} &\ldots & \djoin_k f_{m n}\rst{g_{n k}}
      \end{bmatrix}\\[10pt]
      & = \begin{bmatrix}
        \djoin_k \rst{f_{11} g_{1 k}} f_{11} &\ldots & \djoin_k \rst{f_{1n} g_{n k}} f_{1n}\\[-4pt]
        & \vdots & \\[-4pt]
        \djoin_k \rst{f_{m1} g_{1 k}}f_{m1} &\ldots & \djoin_k \rst{f_{m n} g_{n k}}f_{m n}
      \end{bmatrix}\\[10pt]
      & = \begin{bmatrix} % because the f_ij \perp fi1 etc.
        \djoin_\jay\djoin_k \rst{f_{1\jay} g_{\jay k}} f_{11} &\ldots &
          \djoin_\jay\djoin_k \rst{f_{1\jay} g_{\jay k}} f_{1n}\\[-4pt]
        & \vdots & \\[-4pt]
        \djoin_\jay\djoin_k \rst{f_{m\jay} g_{\jay k}}f_{m1} &\ldots &
          \djoin_\jay\djoin_k \rst{f_{m \jay} g_{\jay k}}f_{m n}
      \end{bmatrix}\\[10pt]
      &= \rst{[f_{i\jay}][g_{\jay k}]} [f_{i\jay}].
  \end{align*}

  Thus, \imatx is a restriction category.

  \emph{Existence of partial inverses.}
  The inverse of the map $f = [f_{i\jay}]$ is the map $\inv{f}\definedas[\inv{f_{\jay i}}]$.
  For the off diagonal elements of $f \inv{f}$, they are a disjoint join of \X maps of the form
  $f_{i j} \inv{f_{\ell j}}$. By the definition of a disjoint sum matrix $\wrg{f_{i j}} \cdperp
  \wrg{f_{\ell j}}$, thus by
  Lemma~\ref{def:disjointness_relation}~\ref{lemitem:djv_disjoint_composition_is_0},
  $0 = \wrg{f_{i j}} \wrg{\wrg{f_{\ell j}}} =\wrg{f_{i j}} \wrg{f_{\ell j}}$. Then,
  \[
    f_{i j} \inv{f_{\ell j}} =   f_{i j}\wrg{f_{i j}} \rst{\inv{f_{\ell j}}} \inv{f_{\ell j}}
      =   f_{i j}\wrg{f_{i j}} \wrg{f_{\ell j}} \inv{f_{\ell j}} =  f_{i j} 0  \inv{f_{\ell j}}
      = 0.
  \]
  Therefore all off-diagonal elements are $0$. The $j^{th}$
  diagonal element is $\djoin_k f_{j k}\inv{f_{j_k}} = \djoin \rst{f_{j k}}$ and thus
  $f\inv{f} = \diag_j [\djoin_k \rst{f_{\jay k}}] = \rst{f}$.
\end{proof}

Furthermore, $\imatx$ is actually a disjoint sum category:
\begin{theorem}\label{thm:imatx_is_an_disjoint_sum_category}
  Given \X an inverse category with disjoint joins and restriction zero, \imatx is an inverse
  category with disjoint sums.
\end{theorem}

Lemma~\ref{lem:inverse_matrix_category_is_a_restriction_category} shows that \imatx is an inverse
category. We will prove \imatx has disjoint sums in a series of lemmas.

\begin{lemma}\label{lem:imat_has_restriction_zero}
  Given \X is an inverse restriction category with a restriction zero and a disjoint join,
  then \imatx has a restriction zero.
\end{lemma}
\begin{proof}
  The restriction zero in \imatx is the list $[0]$ where $0$ is the restriction zero in \X.

  For the object $A=[A_1,\ldots,A_n]$, the $0$ map is given by the $n\times 1$ matrix
  $[0,\ldots,0]$. The map from $0$ is given by the $1\times n$ matrix
  $\begin{bmatrix}0\\ \vdots\\ 0\end{bmatrix}$.

\end{proof}

\begin{lemma}\label{lem:imat_is_disjoint_tensor_category}
  Given \X is an inverse restriction category with a restriction zero, $0$, and a disjoint join,
  then the monoid $\+$ defined by list catenation of objects is a disjointness tensor.
\end{lemma}
\begin{proof}
  We first note the monoidal isomorphisms:
  \begin{align*}
    \upl : [0,A_1,A_2,\ldots,A_n] \to [A_1,A_2,\ldots,A_n] \qquad& \upl :=
    \begin{bmatrix}
      0&\cdots&0 \\
      & I_{n\times n}
    \end{bmatrix}\\
    \upr : [A_1,A_2,\ldots,A_n,0] \to [A_1,A_2,\ldots,A_n] \qquad & \upr :=
    \begin{bmatrix}
      & I_{n\times n} \\
      0&\cdots&0
    \end{bmatrix} \\
    a_{\+} : (A\+B)\+C \to A\+ (B\+C) \qquad &a_{\+} := id \\
    c_{\+} : [A_1,\ldots,A_n,B_1,\ldots,B_m] \to [B_1,\ldots,B_m,A_1,\ldots,A_n]  \qquad &
      c_{\+} := \begin{bmatrix}
                      0_{m\times n} & I_{n\times n} \\
                      I_{m\times m} & 0_{n\times m}
                \end{bmatrix}.
  \end{align*}

  The action of $\+$ on maps is given by:
  \[
    [f_{i\jay}]\+ [g_{\ell \kay}] = \begin{bmatrix}
      [f_{i\jay}] & 0 \\
      0 & [g_{\ell \kay}]
      \end{bmatrix}.
  \]

  With this definition, we see that $\+$ is a restriction functor:
  \begin{align*}
    1_X \+ 1_Y &= 1_{X\+Y}, \\
    f_1 g_1 \+ f_2 g_2 &=
    \begin{bmatrix}
      f_1 g_1 & 0 \\ 0 & f_2 g_2
    \end{bmatrix} =
    \begin{bmatrix}
      f_1 & 0 \\ 0 & f_2
    \end{bmatrix}
    \begin{bmatrix}
      g_1 & 0 \\ 0 & g_2
    \end{bmatrix} = (f_1 \+ f_2) (g_1 \+ g_2).
  \end{align*}

  Following Definition~\ref{def:disjointness_tensor}, we note $\icpa = (1\+0)\upr =
  \begin{bmatrix}
    1&0\\0&0
  \end{bmatrix}
  \begin{bmatrix}
    1\\0
  \end{bmatrix}  =
  \begin{bmatrix}
    1\\0
  \end{bmatrix}
  $ and similarly $\icpb =
  \begin{bmatrix}
    0\\1
  \end{bmatrix}$. Suppose we have $f = [f_{i\jay}]$ and $g = [g_{i\jay}]$ where
  $i \in \{1,\ldots,n\}$ and $j\in\{1,2\}$. Further suppose $f\icpa = g\icpa$ and $f\icpa = g\icpa$.
  Therefore, $f \icpa = [f_{i 1}] = [g_{i 1}] = g\icpa$ and
  $f \icpb = [f_{i 2}] = [g_{i 2}] = g\icpb$, but this means that $f = g$ and we may conclude
  $\icpa$ and $\icpb$ are jointly monic. Similarly, $\cp{1} = [1\ 0]$ and $\cp{2} = [0\ 1]$ are
  jointly epic.
\end{proof}

\begin{lemma}\label{lem:imat_tensor_is_a_disjoint_sum_tensor}
  Given \X is an inverse category with a disjoint join and restriction zero, then \imatx
  has a disjoint sum tensor.
\end{lemma}
\begin{proof}
  By Lemma~\ref{lem:imat_is_disjoint_tensor_category}, we know that the tensor defined by
  list catenation is a disjointness tensor. To show that it is a disjoint sum tensor, we must
  show the diagrams and conditions of Definition~\ref{def:disjoint_sum_tensor} hold.

  For the diagram below we show that $\alpha$ exists if and only if
  $f\icpb \tjdown g\icpb$:
  \begin{align*}
    \xymatrix@C+10pt@R+5pt{
      [A] \ar@/_/[ddr]_f \ar@/^/[drr]^g \ar@{.>}[dr]|-{\alpha} \\
        & [X,Y,Z] \ar[d]^{\icp{1,2}} \ar[r]_{\icp{1,3}} & [X,Z] \ar[d]^{\icpa} \\
        & [X,Y] \ar[r]_{\icpa} & [X].
    }
  \end{align*}
  The existence of $f\icpb \tjdown g\icpb$ means there is an $h = [h_1,h_2]: [A]\to[Y,Z]$ such that
  $h\icpa = f\icpb$ and $h\icpb = g\icpb$. From the diagram, given that $f = [f_1,f_2]$ and
  $g = [g_1,g_2]$, we know that $f_1 = f\icpa = g\icpa = g_1$. We also have $h_1 = f_2$ and
  $h_2 = g_2$. If we set $\alpha$ to the matrix $[f_1,f_2,g_2]$, the diagram above commutes.
  We need only show that $\alpha$ is a map in \imatx. As $f, g$ and $h$ are maps in
  \imatx, we know that:
  \begin{align*}
    f_1\cp{1} & \cdperp f_2 \cp{2}\\
    (f_1\cp{1} =) g_1\cp{1} & \cdperp g_2 \cp{2}\\
    (f_2 \cp{2} =) h_1\cp{1} & \cdperp h_2 \cp{2} (= g_2 \cp{2}).\\
  \end{align*}
  From this, we can conclude $\cdperp [f_1 \cp{1}, f_2 \cp{2}, g_2 \cp{3}]$.

  Conversely, suppose we have an $\alpha = [\alpha_1, \alpha_2, \alpha_3]$ that makes the
  above diagram commute. Then  $h := [\alpha_2, \alpha_3]$ is a map in $\X$.
  Since $[\alpha_1,\alpha_3] = g$ and $[\alpha_1,\alpha_2] = f$, we have
  $h\icpa = f\icpb$ and $h\icpb = g\icpb$, hence $h = f\icpb \tjdown g\icpb$.

  The proof that $\beta$ in the diagram below exists if and only if $\cp{2}h\tjup \cp{2}k$
  is similar:
  \[
    \xymatrix@C+5pt{
        &&[A] \\
         [X,Y] \ar[r]_{\cp{1,2}} \ar@/^/[urr]^h &[X,Y,Z] \ar@{.>}[ur]|-{\beta} \\
         [X] \ar[u]^{\cpa} \ar[r]_{\cpa} & [X,Z]. \ar[u]^{\cp{1,3}} \ar@/_/[uur]_k
    }
  \]
\end{proof}

We are now ready to prove Theorem~\ref{thm:imatx_is_an_disjoint_sum_category}, that \imatx has
disjoint sums.

\begin{proof}
  By Lemma~\ref{lem:imat_tensor_is_a_disjoint_sum_tensor}, we know $\imatx$ has a disjoint sum
  tensor and therefore by Proposition~\ref{prop:disjointness_tensor_gives_disjoint_join},
  it has a disjoint join. By Proposition~\ref{prop:a_disjoint_sum_tensor_gives_disjoint_sums}
  we know that $[A,B] = A\+B$ is a disjoint sum of $A$ and $B$ for any two objects in \imatx, and
  hence, \imatx has disjoint sums.
\end{proof}

\section{Equivalence between a disjoint sum category and its matrix category} % (fold)
\label{sec:equivalence_between_an_disjoint_sum_category_and_its_matrix_category}
This section starts by giving a functor from an inverse category with disjoint joins to its matrix
category, followed by exhibiting a reflection between inverse categories with disjoint sums
(Disjoint Sum Cats) and categories with a disjoint join (Disjoint Join Cats). That is,

\begin{equation}
   \xymatrix@R+10pt@C+20pt{
     \text{Disjoint Sum Cats} \ar@{^{(}->}[r]
     & \text{Disjoint Join Cats.} \ar@/_1.8pc/[l]_{\text{Matrix}} \ar@{}@<-1ex>[l]_{\top}
   }\label{fig:disjoint-sum-cats-to-disjoint-join-cats}
\end{equation}

Then, we will provide a restriction functor from the matrix category of an inverse category with a
disjoint sum to itself, i.e., $\imatx \to \X$.  Furthermore, we will show in the case where \X is an
inverse category with a disjoint sum, $\X\cong \imatx$.

\begin{definition}\label{def:functor_from_x_to_imatx}
  Given $\X$ has disjoint joins and restriction zero, define $M:\X \to \imatx$ by:
  \begin{align*}
    \text{On objects:}\ &M(A) := [A].\\
    \text{On maps:}\ &M(f) := [f] \text{ -- The }1\times1\text{ matrix with entry }f.
  \end{align*}
\end{definition}
\begin{lemma}\label{lem:m_is_a_functor}
  The map $M$ from Definition~\ref{def:functor_from_x_to_imatx} is a restriction functor.
\end{lemma}
\begin{proof}
  From the definition of \imatx, we have
  \begin{align*}
    &f: A\to B \text{ if and only if } M(f):M(A)\to M(B)\  (\text{i.e., }[f]: [A]\to [B]),\\
    &M(id_A)  = [id_{A}] = id_M(A),\\
    &M(f g) = [f g] = [f][g] = M(f) M(g),\text{ and}\\
    &M(\rst{f}) = [\rst{f}] = \rst{[f]} = \rst{M(f)}.
  \end{align*}
\end{proof}

Let us represent the category of inverse categories with disjoint sums as \DSum and the
category of inverse categories with disjoint joins as \DJoin.

Note that any inverse category with disjoint sums is an inverse category with disjoint joins. Hence,
we have the obvious forgetful functor $U:\DSum \to \DJoin$. From above, we also have a functor
$Mat:\DJoin \to \DSum$ given by $Mat: \X \mapsto \imatx$. From the definition of \imatx, we see
that we have the correspondence
\[
  \infer{\DJoin(\X,\Y)}{\DSum(\imatx,\Y)}
\]
meaning that we have an adjunction, $Mat\leftadjoint U:\DJoin \to \DSum$, as noted at the beginning
of this section.

Let us now consider when we apply the matrix construction to an inverse category which already has a
disjoint sum:

\begin{definition}\label{def:functor_from_imatx_to_x}
  Given $\X$ has disjoint sums with restriction zero $0$,
  and disjoint sum tensor $\+$, define $S:\imatx \to \X$ by:
  \begin{align*}
    \text{Objects:}\ &S([A_1,A_2, \ldots, A_n]) := A_1\+A_2\+\cdots\+A_n\\
    \text{Maps:}\ &S([f_{i\jay}]) := \djoinbig_i \icp{i} (\djoin_j f_{i\jay} \cp{j}).
  \end{align*}
\end{definition}

\begin{lemma}\label{lem:s_is_a_functor}
  The map $S$ from Definition~\ref{def:functor_from_imatx_to_x} is a restriction functor.
\end{lemma}
\begin{proof}
  From the definition of \imatx, where $A = [A_1,A_2,\ldots,A_n]$, $B=[B_1,B_2,\ldots,B_M]$,
  and $f = [f_{i\jay}]$ we have
  \begin{align*}
    &S(id_A)  = S([id_{A_i}]) = \djoinbig_i \icp{i} (\djoin_j \cp{j}) = id_{S(A)}\\
    &f: A\to B  \iff S(f):S(A)\to S(B) \iff \\
    &\qquad \qquad  \djoinbig_i \icp{i} (\djoin_\jay f_{i\jay} \cp{j}) : A_1\+\cdots\+A_n \to B_1\+\cdots\+B_m\\
    &M(\rst{f}) = [\rst{f}] = \rst{[f]} = \rst{M(f)}.
  \end{align*}
  For composition, we have
  \begin{align*}
    S(f)S(g) & = (\djoinbig_i \icp{i} (\djoin_j f_{i\jay} \cp{j}))
                  (\djoinbig_{j'} \icp{j'} (\djoin_k g_{\jay\kay} \cp{k})) \\
    &= \djoinbig_i \icp{i} \djoinbig_j \djoinbig_{j'} f_{i\jay} \cp{j} \icp{j'}
            (\djoin_k g_{\jay'\kay} \cp{k}) \\
    &= \djoinbig_i \icp{i} \djoinbig_j  f_{i\jay} (\djoin_k g_{\jay\kay} \cp{k}) \\
    &= \djoinbig_i \icp{i} \djoinbig_k (\djoin_j  f_{i\jay}  g_{\jay\kay} \cp{k}) \\
    &= S([\djoin_j f_{i\jay} g_{\jay\kay}]) \\
    &= S(f g).
  \end{align*}

\end{proof}

The functors $S$ and $M$ provide an equivalence:

\begin{proposition}\label{pro:x_and_imatx_are_equivalent}
  Given an inverse category \X with $\+$ a disjoint sum tensor and a restriction zero, then the
  categories $\X$ and \imatx are equivalent.
\end{proposition}
\begin{proof}
  The functors of the equivalence are $S$ from Definition~\ref{def:functor_from_imatx_to_x} and
  $M$ from Definition~\ref{def:functor_from_x_to_imatx}.

  First, we see that $M S:\X \to \X$ is the identity functor as
  \begin{align*}
    \text{Objects:}\ &S(M(A)) = S([A]) = A,\\
    \text{Maps:}\ &S(M(f)) = S([f]) = f.
  \end{align*}

  Next, we need to show that there is a natural transformation and isomorphism $\rho$ such that
  $\rho(S M) = I_{\imatx}$. For each object $A = [A_1,A_2,\ldots,A_n]$, set
  $\rho_A = \begin{bmatrix}\icpa& \cdots &\icp{n}\end{bmatrix}$.

  Note that the functor $S M$ has the following effect:

  \begin{align*}
    \text{On objects:}\quad &M(S([A_1,\ldots, A_n])) = M(A_1\+\cdots\+ A_n) =
      [A_1\+\cdots\+ A_n].\\
    \text{On maps:}\quad &M(S([f_{i\jay}]) = M(\djoinbig_i \icp{i} (\djoin_j f_{i\jay} \cp{j}))
      = [\djoinbig_i \icp{i} (\djoin_j f_{i\jay} \cp{j})].
  \end{align*}
  We can now draw the commuting naturality square for $f=[f_{i\jay}]:[A_i]\to [B_j]$:
  \[
    \xymatrix{
      SM([A_i]) \ar@{=}[r] & [\+_i A_i]
          \ar[rrrr]^{\begin{bmatrix}\icpa& \cdots &\icp{n}\end{bmatrix}}
          \ar[dd]_{SM(f)}
          & &&& [A_i] \ar[dd]^{f}\\
        &&&&&  \\
      SM([B_j]) \ar@{=}[r] & [\+_j B_j]
          \ar[rrrr]_{\begin{bmatrix}\icpa& \cdots &\icp{m}\end{bmatrix}}
          & &&& [B_j]. \\
    }.
  \]

  Following the square by the top--right path from $[\+_i A_i]$ to $[B_j]$, by the definition of
  the maps in the category \imatx, we see each $B_j = \djoin_i\icp{i} f_{i\jay}(\+_i A_i)$.
  Following the left--bottom path, composing $SM(f)$ with $\begin{bmatrix}\icpa& \cdots
  &\icp{m}\end{bmatrix}$ gives us the map
  \begin{multline*}
    \begin{bmatrix}\djoin_i \icp{i} (\djoin_j f_{i\jay}\cp{j})\icpa& \cdots
      &\djoin_i \icp{i} (\djoin_j f_{i\jay}\cp{j})\icp{m}\end{bmatrix} = \\
      \begin{bmatrix}\djoin_i \icp{i}  f_{i 1}& \cdots &\djoin_i \icp{i} f_{i m}\end{bmatrix}.
  \end{multline*}
  Applying this to $[\+_i A_i]$, we see each $B_j = \djoin_i\icp{i} f_{i\jay}(\+_i A_i)$ and the
  two directions are equal.

  Finally, we know that $\inv{\rho_{A_i}} = \begin{bmatrix}\cp{1}\\ \vdots \\\cp{n}\end{bmatrix}$
  and defines an isomorphism between any object of the form $[\+_i A_i]$ and the
  object $[A_1,\ldots,A_n]$.
\end{proof}

Note that as we have an equivalence, we have the adjoint equivalence between hom-sets of
\[
  \infer{\imatx(X, M(Y))}{\X(S(X),Y)}
\]
as $S$ is the left adjoint of $M$.
% subsection equivalence_between_an_disjoint_sum_category_and_its_matrix_category (end)

\begin{example}\label{ex:disjoint_sum_category_maps_are_matrices}
  We may obtain a matrix representative of any map $f:A\+B\to C\+D$ by applying the construction of
  Definition~\ref{def:functor_from_imatx_to_x} in reverse.

  Then given a function   $f:A\+B \to C\+D$ define
  \[ f_M =
     \begin{bmatrix}
       \cpa f \icpa & \cpa f \icpb \\
       \cpb f \icpa & \cpb f \icpb
      \end{bmatrix}.
  \]

  Thus, applying the functor $S$ from Definition~\ref{lem:s_is_a_functor}, we have
  \begin{align*}
     S(f_M) &= \icpa (\cpa f \icpa \cpa \djoin \cpa f \icpb \cpb) \djoin \icpb( \cpb f \icpa\cpa
     \djoin \cpb f \icpb \cpb) \\
     & = \icpa \cpa f (\icpa \cpa \djoin \icpb \cpb) \djoin \icpb \cpb f ( \icpa \cpa \djoin
     \icpb\cpb) \\
     & = \icpa \cpa f \djoin \icpb \cpb f \\
     & = (\icpa \cpa \djoin \icpb \cpb) f \\
     & = f.
  \end{align*}

In particular, we note that we may represent $f:A\to B$ by the matrix
\[
  \begin{bmatrix}
    1f1& 1f 0 \\ 0 f 1 & 0 f 0
  \end{bmatrix}
  =
  \begin{bmatrix}
    f& 0 \\ 0 & 0
  \end{bmatrix}
\]
as $A \cong A\+0$ and $B \cong B\+0$. Equivalently, this is the matrix $\begin{bmatrix}f\end{bmatrix}$.
\end{example}


\begin{definition}\label{def:unique_decomposition_category}
A unique decomposition category\cite{haghverdi:2000phd} is a category where any
\[
  h:A\+B \to C\+D
\]
is uniquely determined by the four maps:
\begin{alignat*}{2}
  \cpa  h  \icpa &: A \to C, & \quad \cpa  h  \icpb &: A \to D,\\
  \cpb  h  \icpa &: B \to C, & \quad \cpb  h  \icpb &: B \to D.
\end{alignat*}
That is, it is writable as the matrix:
\[
  \begin{bmatrix}
    \cpa  h  \cpa^{(-1)} & \cpa  h  \cpb^{(-1)}\\
    \cpb  h  \cpa^{(-1)} &\quad \cpb  h  \cpb^{(-1)}
  \end{bmatrix}:
  A\+B \to C\+D.
\]

The map from $0$ in the category corresponds to the $0$-dimensional matrix,
\[
  0\xrightarrow{[]}B.
\]
%Note also that as $\cpa$ and $\cpb$ are restriction maps, the following diagram commutes:
%\[
%  \xymatrix @C+10pt{
%    A \ar@{=}[d] \ar[r]^{\cpa} & A\+B \\
%    A \ar[r]_{\cpa} & A\+0 \ar[u]_{1+0}
%  }
%\]
\end{definition}

\begin{corollary}\label{cor:disjoint_sum_category_is_udc}
  If $\X$ is an inverse category with disjoint sums, then it is a unique decomposition category.
\end{corollary}

The fact that we get a unique decomposition category is important as Proposition 4.0.11 of
\cite{haghverdi:2000phd} gives a formula for computing a trace, when it exists. In a unique
decomposition category, for a function $f:X\+U \to Y\+U$, which may be represented by the matrix
\[
   f =  \begin{bmatrix}
     f_{11} & f_{12}\\
     f_{21} & f_{22}
  \end{bmatrix},
\]
the unique inductive trace is given by the formula:
\begin{equation}
  Tr^{U}_{X,Y}(f) = f_{11} +\sum_{n=0}^{\infty}f_{12}f_{22}^n f_{21},\label{eq:trace}
\end{equation}
whenever this sum exists. Therefore, for example, an inverse category with countable disjoint sums
will have a trace.

%%% Local Variables:
%%% mode: latex
%%% TeX-master: "../phd-thesis"
%%% End:
