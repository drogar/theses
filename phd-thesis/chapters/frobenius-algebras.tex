%!TEX root = /Users/gilesb/UofC/thesis/phd-thesis/phd-thesis.tex
\chapter{Frobenius Algebras and Quantum Computation} % (fold)
\label{cha:frobenius_algebras_and_quantum_computation}
\section{Definition of a Frobenius Algebra} % (fold)
\label{sec:definition_of_a_frobenius_algebra}
In their most general setting, Frobenius algebras are defined as a finite dimensional algebra
over a field together with a non-degenerate pairing operation. We will continue with the definitions
that make this precise.

\subsection{Frobenius algebra definitions} % (fold)
\label{sub:frobenius_algebra_definitions}


\begin{definition}[Frobenius algebra]\label{def:frobeniusalgebra}
  Given a symmetric monoidal category \cD, a \emph{Frobenius algebra} is an object $X$ of \cD and
  four maps, $\nabla :X\*X \to X$, $e: I \to X$, $\Delta: X\to X\* X$ and $\epsilon:X\to I$, with
  the conditions that $(X,\nabla,e)$ forms a commutative monoid, $(X,\Delta, \epsilon)$ forms a
  commutative comonoid and the diagram
  \[
    \xymatrix{
      X\*X \ar[rr]^{X\*\Delta} \ar[dd]_{\Delta \* X} \ar[dr]^{\nabla}
        && X\*X\*X \ar[dd]^{\nabla\*X}\\
      & X\ar[dr]^{\Delta}\\
      X\*X\*X\ar[rr]_{X\*\nabla}  && X\*X
    }
  \]
  commutes. The Frobenius algebra is \emph{special} when $\Delta; \nabla = 1_{X}$ and
  \emph{commutative} when $\Delta ; c_{X,X} = \Delta$.
\end{definition}
\begin{definition}[$\dagger$-Frobenius algebra]\label{def:daggerfrob}
  A Frobenius algebra in a dagger symmetric monoidal category where $\Delta = \dgr{\nabla}$ and
  $\epsilon=\dgr{u}$ is a $\dagger$\emph{-Frobenius algebra}.
\end{definition}
For an example of a $\dagger$-Frobenius algebra, consider a finite dimensional Hilbert space $H$
with an orthonormal basis $\{\ket{\phi_{i}}\}$ and define $\Delta:H\to H\*H: \ket{\phi_{i}}\mapsto
\ket{\phi_{i}} \* \ket{\phi_{i}}$ and $\epsilon : H\to \C : \ket{\phi_{i}} \mapsto 1$. Then $(H,
\nabla=\dgr{\Delta}, u=\dgr{\epsilon}, \Delta, \epsilon)$ forms a commutative special
$\dagger$-Frobenius algebra.

% subsection frobenius_algebra_definitions (end)

% section definition_of_a_frobenius_algebra (end)


\section{Bases and Frobenius Algebras} % (fold)
\label{sec:bases_and_frobenius_algebras}
In \cite{coeckeetal08:ortho}, the authors provide an algebraic description of orthogonal bases in
finite dimensional Hilbert spaces. As noted in section
\ref{sec:the_category_of_commutative_frobenius_algebras}, an orthonormal basis for such a space is
a special commutative $\dagger$-Frobenius algebra. To show the other direction, given a commutative
$\dagger$-Frobenius algebra, $(H,\nabla,u)$ and for each element $\alpha\in H$, define the right
action of $\alpha$ as $R_{\alpha}:=(id\*\alpha)\, \nabla:H\to H$. Note the use of the fact that
elements $\alpha\in H$ can be considered as linear maps $\alpha:\C \to H:1\mapsto \ket{\alpha}$.
The dagger of a right action is also a right action, $\dgr{R_{\alpha}} = R_{\alpha'}$ where
$\alpha'= u\, \nabla\, (id\* \dgr{\alpha})$, which is a consequence of the Frobenius identities.

The $(\_)'$ construction is actually an involution:
\begin{eqnarray*}
  &(\alpha')' &= u \nabla (id \* \dgr{\alpha'}) \\
  && = u \nabla (id \* \dgr{(u \nabla (id \* \dgr{\alpha}))}\\
  && = u \nabla (id \* ( (id \* \alpha) \Delta \epsilon))\\
  && = (u \* \alpha) (\nabla \* id) (id \* \Delta) (id \*  \epsilon)\\
  && = (u \* \alpha) (id \* \Delta) (\nabla \* id) (id \*  \epsilon)\\
  && = (u \* \alpha)  (id \*  \epsilon)\\
  && = \alpha
\end{eqnarray*}

\begin{lemma}\label{lemma:cstaralgebra}
  Any $\dagger$-Frobenius algebra in \fdh is a $C^{*}$-algebra.
\end{lemma}
\begin{proof}
  The endomorphism monoid of \fdh(H,H) is a $C^{*}$-algebra. From the proceeding,
  \[
    H \cong \fdh(\C,H) \cong R_{[\fdh(\C,H)]}\subseteq\fdh(H,H).
  \]
  This inherits the algebra structure from \fdh(H,H). Furthermore, since any finite dimensional
  involution-closed sub-algebra of a $C^{*}$-algebra is also a $C^{*}$-algebra, this shows the
  $\dagger$-Frobenius algebra is a $C^{*}$-algebra.
\end{proof}

Using the fact that the involution preserving homomorphisms from a finite dimensional commutative
$C^{*}$-algebra to $\C$ form a basis for the dual of the underlying vector space, write these
homomorphisms as $\dgr{\phi_{i}}:H \to \C$. Then their adjoints, $\phi_{i}:\C\to H$ will form a
basis for the space $H$. These are the copyable elements in $H$.

This, together with continued applications of the Frobenius rules and linear algebra allow the
authors to prove:
\begin{theorem}
  Every commutative $\dagger$-Frobenius algebra in \fdh determines an orthogonal basis consisting
  of its copyable elements. Conversely, every orthogonal basis $\{\ket{\phi_{i}}\}_{i}$ determines
  a commutative $\dagger$-Frobenius algebra via \[\Delta:H\to H\*H: \ket{\phi_{i}}\mapsto
  \ket{\phi_{i}} \* \ket{\phi_{i}}\qquad\epsilon : H\to \C : \ket{\phi_{i}} \mapsto 1\] and these
  constructions are inverse to each other.
\end{theorem}

\subsection{Quantum and classical data}\label{sec:quantumclassical}
In \cite{coecke08structures}, the authors build on the results above,
to start from a $\dagger$-symmetric monoidal category and construct the minimal machinery needed to
model quantum and classical computations. For the rest of this section, $\cD$ will be assumed to be
such a category, with $\*$ the monoid tensor and $I$ the unit of the monoid.

\begin{definition}
  A compact structure on an object $A$ in the category $\cD$ is given by the object $A$, an object
  $A^{*}$ called its \emph{dual} and the maps $\eta:I \to A^{*}\* A$, $\epsilon: A\* A^{*} \to I$
  such that the diagrams
  \[
    \xymatrix@C+20pt{
      A^{*} \ar[dr]^{id} \ar[d]_{\eta\*A^{*}} \\
      A^{*} \*A\*A^{*}  \ar[r]_(.6){A^{*} \*\epsilon} & A^{*}
    }
    \hskip{10pt}
    \xymatrix@C+20pt{
      A \ar[r]^(.4){A\*\eta} \ar[dr]_{id} & A\* A^{*}\* A \ar[d]^{\epsilon\*A}\\
      & A
    }
  \]
  commute.
\end{definition}

\begin{definition}[Quantum Structure]\label{def:quantumstructure}
  A \emph{quantum structure} is an object $A$ and map $\eta:I\to A\*A$ such that
  $(A,A,\eta,\dgr{\eta})$ form a compact structure.
\end{definition}
Note that $A$ is self-dual in definition \ref{def:quantumstructure}.

This allows the creation of the category $\cD_{q}$ which has as objects quantum structures and maps
are the maps in $\cD$ between the objects in the quantum structures.

In the category $\cD_{q}$, it is now possible to define the upper and lower $*$ operations on maps,
such that $(f_{*})^{*}= (f^{*})_{*} = \dgr{f}$.
\begin{eqnarray*}
&f^{*} &:= (\eta_{A}\*1) (1 \* f\*1) (1\*\dgr{\eta}_{B})\\
&f_{*} &:= (\eta_{B}\*1) (1 \* \dgr{f}\*1) (1\*\dgr{\eta}_{A})
\end{eqnarray*}

Interestingly, $\cD_{q}$ possesses enough structure to be axiomatized in the same manner as above in
section \ref{sec:puresemantics}, excepting the portions dependent upon biproducts.

Next, define a classical structure on \cD.
\begin{definition}[Classical structure]\label{def:classicalstructure}
  A \emph{classical structure} in \cD{} is an objects $X$ and two maps, $\Delta :X \to X\* X$,
  $\epsilon:X\to I$ such that $X,\dgr{\Delta},\dgr{\epsilon},\Delta,\epsilon$ forms a special
  Frobenius algebra.
\end{definition}

As above, this allows us to define $\cD_{c}$, the category whose objects are the classical
structures of $\cD$ with maps between classical structures being the maps in $\cD$ between the
objects of the classical structure.

Note that a classical structure will induce a quantum structure, setting $\eta_{X}$ to be
$\dgr{\epsilon_{X}}\, \Delta_{X}$.

% section bases_and_frobenius_algebras (end)

\section{The category of Commutative Frobenius Algebras} % (fold)
\label{sec:the_category_of_commutative_frobenius_algebras}
\begin{example}[Commutative Frobenius algebras]\label{example:commfrob}
  Let \X be a symmetric monoidal category and form CFrob(\X) as follows: \paragraph{Objects:}
  Commutative Frobenius algebras\cite{kock04}: A quintuple $(X,\nabla,\eta,\Delta,\epsilon)$ where
  X is a $k$-algebra for some field $k$, and $\nabla :A\*A \to A$, $\eta:k\to A$, $\Delta : A \to
  A\*A$, $\epsilon : A \to k$ are natural maps in the algebra. Additionally, these satisfy
  \[
    \xymatrix @C=40pt @R=25pt{
      A \* A \ar[dd]_{1\*\Delta} \ar[dr]^{\nabla}
        \ar[rr]^{\Delta \* 1} & &
        A \* (A \* A) \ar[dd]^{1 \* \nabla}\\
      & A \ar[dr]^{\Delta} & \\
      (A \* A) \* A \ar[rr]_{\nabla \* 1} & &
        A \* A
    }
  \]
  together with the additional property that $\Delta \nabla = 1$.

  \paragraph{Maps:} Multiplication ($\nabla$) and co-multiplication ($\Delta$) preserving
  homomorphisms which do not necessarily preserve the unit.
\end{example}

\begin{theorem}
  When \X is a symmetric monoidal category, CFrob(\X) is a discrete inverse category.
\end{theorem}
\begin{proof}
  For $f:X \to Y$, define $\inv{f}$ as
  \[
    Y \xrightarrow{1\*\eta} Y\*X \xrightarrow{1\*\Delta}
      Y\*X\*X \xrightarrow{1\*f\*1} Y\*Y\*X \xrightarrow{\nabla\*1}
      Y\*X \xrightarrow{\epsilon\*1}X
  \]
  Using a result from \cite{cockett2002:restcategories1}, we need only show:
  \begin{align*}
    \inv{(\inv{f})} &= f\\
    f\inv{f}f &= f\\
    f\inv{f}g\inv{g} &=g\inv{g} f\inv{f}
  \end{align*}
  We also use the following two identities from \cite{kock04}:
  \begin{align}
    (1\*\eta)\nabla &= id\\
    \Delta(1\*\epsilon) &= id.
  \end{align}

  \begin{align*}
    \inv{\inv{f}} &=(1\*\eta)(1\*\Delta)(1\*(\inv{f})\*1)(\nabla\*1)(\epsilon\*1) \\
    &=(1\*\eta)(1\*\Delta)(1\*((1\*\eta)(1\*\Delta)(1\*f\*1)(\nabla\*1)(\epsilon\*1))\*1)\\
    &\qquad\qquad(\nabla\*1)(\epsilon\*1) \\
    &=(1\*\eta)(1\*\Delta)(1\*1\*\eta)(1\*1\*f\*1\*1)(1\*\nabla\*1\*1)\\
    &\qquad\qquad(1\*\epsilon\*1\*1) (\nabla\*1)(\epsilon\*1)\\
    &=(\eta\*1)(\Delta\*1)(1\*\nabla)(f\*1)(((\eta)(\Delta\*1)(1\*\nabla)(1\*\epsilon))\*1)
      ((1\*\epsilon)\\
    &=(1\*\eta)\nabla \Delta(1\*\epsilon)f(\eta\*1)\nabla\Delta(1\*\epsilon)\\
    &=id_{x}id_{x}\  f \ id_{y} id_{y}\\
    &=f
  \end{align*}
  \begin{align*}
    f\inv{f}f &= f(1\*\eta)(1\*\Delta)(1\*f\*1)(\nabla\*1)(\epsilon\*1)f\\
    &=(1\*\eta)(1\*\Delta)(f\*f\*1)(\nabla\*1)(1\*f)(\epsilon\*1)\\
    &=(1\*\eta)(1\*\Delta)(\nabla\*1)(f\*f)(\epsilon\*1)\\
    &=(1\*\eta)\nabla\Delta(f\*f)(\epsilon\*1)\\
    &=\Delta(f\*f)(\epsilon\*1)\\
    &=f\Delta(\epsilon\*1)\\
    &=f
  \end{align*}
  Finally, to show $f\inv{f}$ and $g\inv{g} $ commute:
  \begin{align*}
    f(1\*\eta)&(1\*\Delta)(1\*f\*1)(\nabla\*1)(\epsilon\*1)g(1\*\eta)(1\*\Delta)(1\*g\*1)
      (\nabla\*1)(\epsilon\*1)\\
    &=(1\*\eta)(1\*\Delta)(\nabla\*1)(f\*1)(\epsilon\*1)(1\*\eta)(1\*\Delta)(\nabla\*1)(g\*1)
      (\epsilon\*1)\\
    &=(1\*\eta)\nabla\Delta(f\*1)(\epsilon\*1)(1\*\eta)\nabla\Delta(g\*1)(\epsilon\*1)\\
    &=\Delta(f\*1)(\epsilon\*1)\Delta(g\*1)(\epsilon\*1)\\
    &=\Delta(1\*\Delta)(f\*g\*1)(\epsilon\*\epsilon\*1)\\
    &=\Delta(1\*\Delta)(g\*f\*1)(\epsilon\*\epsilon\*1)\qquad\qquad\qquad\text{co-commutativity}\\
    &=g\inv{g}f\inv{f}
  \end{align*}

\end{proof}

% section the_category_of_commutative_frobenius_algebras (end)

% chapter frobenius_algebras_and_quantum_computation (end)

%%% Local Variables:
%%% mode: latex
%%% TeX-master: "../phd-thesis"
%%% End:
