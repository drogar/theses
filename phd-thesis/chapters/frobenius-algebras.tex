%!TEX root = /Users/gilesb/UofC/thesis/phd-thesis/phd-thesis.tex
\chapter{Frobenius Algebras and Quantum Computation} % (fold)
\label{cha:frobenius_algebras_and_quantum_computation}
\section{Definition of a Frobenius Algebra} % (fold)
\label{sec:definition_of_a_frobenius_algebra}
In their most general setting, Frobenius algebras are defined as a finite dimensional algebra
over a field together with a non-degenerate pairing operation. We will continue with the definitions
that make this precise.

\subsection{Algebraic setting} % (fold)
\label{sub:algebraic_setting}

We assume the basic algebraic structures of group, ring and field are known. The reader may consult
\cite{lang:algebra} if further details are needed.


\begin{definition}\label{def:vector_space}
  A \emph{vector space} $V$ over the base field $\kappa$ is an Abelian group together with an action
  of $\kappa$ on $V$, $V\times \kappa \to V$. Assuming the group operation on $V$ is given by $+$,
  then the following must also be satisfied when $v,w\in V$ and $i,j\in\kappa$:
  \begin{align}
    (v+w)i & = v i + w i\\
    v(i+j) & = v i + v j\\
    v(i j) & = (v i) j\\
    v1 &= v
  \end{align}
\end{definition}

\begin{definition}\label{def:basis_of_vector_space}
  A set $B = \{b_i\}$ contained in a vector space $T$  over $\kappa$ is a \emph{basis} for $T$ when:
  \begin{itemize}
    \item Every $t \in T$ is equal to $\sum k_i s_i$ for some choice of $k_i\in\kappa$;
    \item There are no pairs $k_1,k_2$ such that $k_1 s_i + k_2 s_j = 0$ for any $s_i, s_j$ unless
      $i = j$.
  \end{itemize}
  The number of elements in a basis for a vector space $V$ is the \emph{dimension} of $V$.
\end{definition}

\begin{definition}\label{def:linear_map_of_vector_spaces}
  Given $V, W$ are vector spaces over $\kappa$ with $v \in V$ and $i \in \kappa$, then
  if $f:V \to W$ is a group homomorphism such that $f(v i) = f(v) i$, then we say $f$ is a
  \emph{linear map}. Furthermore, a map $f:V\times W \to X$ is called \emph{bilinear} when the
  map $f_v:W\to T$ and $f_w:V\to T$ are linear for each $v\in V$ and $w\in W$, where $f_v$ is the
  map obtained from $f$ by fixing $v\in V$ and $f_w$ is obtained from $f$ by fixing  $w\in W$.
\end{definition}

\begin{definition}\label{def:free_vector_space}
  Given a set $S$, the \emph{free vector space} of $S$ over a field $\kappa$ is the abelian group
  of formal sums $\sum a_i s_i$ where the $s_i$ are the elements of $S$ and $a_i \in \kappa$.
  Formal sums are independent of order. Addition is defined as $(\sum a_i s_i) + (\sum b_i s_i)$ is
  $(\sum (a_i + b_i) s_i)$.
\end{definition}

\begin{definition}\label{def:tensor_product_of_vector_spaces}
  Given vector spaces $V, W$ over the base field $\kappa$, consider the free vector space of $V
  \times W = F(V\times W)$. Next, consider the subspace $T$ of $F(V\times W)$ generated by the
  following equations:
  \begin{align*}
    (v_1,w)+(v_2,w) & = (v_1+v_2,w)\\
    (v,w_1)+(v,w_2) & = (v, w_1+w_2)\\
    k(v,w) &= (k v,w)\\
    k(v,w) &= (v,k w),
  \end{align*}
  where $v,v_1,v_2 \in V$, $w,w_1,w_2 \in W$ and $k\in \kappa$. Then the tensor product of
  $V$ and $W$, written $V\*W$ is $F(V\times W)/T$.
\end{definition}

Elements of the tensor product $V\*W$ are written as $v\*w$ and are the $T$-equivalence class of
$(v,w) \in V\times W$. If $\{v_i\}$ is a basis for $V$ and $\{w_j\}$ is a basis for $W$, then the
elements $\{v_i\*w_j\}$ form a basis for $V\*W$.
% subsection algebraic_setting (end)


% section definition_of_a_frobenius_algebra (end)

\section{Bases and Frobenius Algebras} % (fold)
\label{sec:bases_and_frobenius_algebras}
In \cite{coeckeetal08:ortho}, the authors provide an algebraic description of orthogonal bases in
finite dimensional Hilbert spaces. As noted in the example at the end of section
\ref{sec:daggerdefinitions}, an orthonormal basis for such a space is a special commutative
$\dagger$-Frobenius algebra. To show the other direction, given a commutative $\dagger$-Frobenius
algebra, $(H,\nabla,u)$ and for each element $\alpha\in H$, define the right action of $\alpha$ as
$R_{\alpha}:=(id\*\alpha)\, \nabla:H\to H$. Note the use of the fact that elements $\alpha\in H$
can be considered as linear maps $\alpha:\C \to H:1\mapsto \ket{\alpha}$. The dagger of a right
action is also a right action, $\dgr{R_{\alpha}} = R_{\alpha'}$ where $\alpha'= u\, \nabla\, (id\*
\dgr{\alpha})$, which is a consequence of the Frobenius identities.

The $(\_)'$ construction is actually an involution:
\begin{eqnarray*}
  &(\alpha')' &= u \nabla (id \* \dgr{\alpha'}) \\
  && = u \nabla (id \* \dgr{(u \nabla (id \* \dgr{\alpha}))}\\
  && = u \nabla (id \* ( (id \* \alpha) \Delta \epsilon))\\
  && = (u \* \alpha) (\nabla \* id) (id \* \Delta) (id \*  \epsilon)\\
  && = (u \* \alpha) (id \* \Delta) (\nabla \* id) (id \*  \epsilon)\\
  && = (u \* \alpha)  (id \*  \epsilon)\\
  && = \alpha
\end{eqnarray*}

\begin{lemma}\label{lemma:cstaralgebra}
  Any $\dagger$-Frobenius algebra in \fdh is a $C^{*}$-algebra.
\end{lemma}
\begin{proof}
  The endomorphism monoid of \fdh(H,H) is a $C^{*}$-algebra. From the proceeding,
  \[
    H \cong \fdh(\C,H) \cong R_{[\fdh(\C,H)]}\subseteq\fdh(H,H).
  \]
  This inherits the algebra structure from \fdh(H,H). Since any finite dimensional
  involution-closed subalgebra of a $C^{*}$-algebra is also a $C^{*}$-algebra, this shows the
  $\dagger$-Frobenius algebra is a $C^{*}$-algebra.
\end{proof}

Using the fact that the involution preserving homomorphisms from a finite dimensional commutative
$C^{*}$-algebra to $\C$ form a basis for the dual of the underlying vector space, write these
homomorphisms as $\dgr{\phi_{i}}:H \to \C$. Then their adjoints, $\phi_{i}:\C\to H$ will form a
basis for the space $H$. These are the copyable elements in $H$.

This, together with continued applications of the Frobenius rules and linear algebra allow the
authors to prove:
\begin{theorem}
  Every commutative $\dagger$-Frobenius algebra in \fdh determines an orthogonal basis consisting
  of its copyable elements. Conversely, every orthogonal basis $\{\ket{\phi_{i}}\}_{i}$ determines
  a commutative $\dagger$-Frobenius algebra via \[\Delta:H\to H\*H: \ket{\phi_{i}}\mapsto
  \ket{\phi_{i}} \* \ket{\phi_{i}}\qquad\epsilon : H\to \C : \ket{\phi_{i}} \mapsto 1\] and these
  constructions are inverse to each other.
\end{theorem}

\subsection{Quantum and classical data}\label{sec:quantumclassical}
In \cite{coecke08structures}, the authors build on the results above,
to start from a $\dagger$-symmetric monoidal category and construct the minimal machinery needed to
model quantum and classical computations. For the rest of this section, $\cD$ will be assumed to be
such a category, with $\*$ the monoid tensor and $I$ the unit of the monoid.

\begin{definition}
  A compact structure on an object $A$ in the category $\cD$ is given by the object $A$, an object
  $A^{*}$ called its \emph{dual} and the maps $\eta:I \to A^{*}\* A$, $\epsilon: A\* A^{*} \to I$
  such that the diagrams
  \[
    \xymatrix@C+20pt{
      A^{*} \ar[dr]^{id} \ar[d]_{\eta\*A^{*}} \\
      A^{*} \*A\*A^{*}  \ar[r]_(.6){A^{*} \*\epsilon} & A^{*}
    }\hskip-3em
    \xymatrix@C+20pt{
      A \ar[r]^(.4){A\*\eta} \ar[dr]_{id} & A\* A^{*}\* A \ar[d]^{\epsilon\*A}\\
      & A
    }
  \]
  commute.
\end{definition}

\begin{definition}[Quantum Structure]\label{def:quantumstructure}
  A \emph{quantum structure} is an object $A$ and map $\eta:I\to A\*A$ such that
  $(A,A,\eta,\dgr{\eta})$ form a compact structure.
\end{definition}
Note that $A$ is self-dual in definition \ref{def:quantumstructure}.

This allows the creation of the category $\cD_{q}$ which has as objects quantum structures and maps
are the maps in $\cD$ between the objects in the quantum structures.

In the category $\cD_{q}$, it is now possible to define the upper and lower $*$ operations on maps,
such that $(f_{*})^{*}= (f^{*})_{*} = \dgr{f}$.
\begin{eqnarray*}
&f^{*} &:= (\eta_{A}\*1) (1 \* f\*1) (1\*\dgr{\eta}_{B})\\
&f_{*} &:= (\eta_{B}\*1) (1 \* \dgr{f}\*1) (1\*\dgr{\eta}_{A})
\end{eqnarray*}

Interestingly, $\cD_{q}$ possesses enough structure to be axiomatized in the same manner as above in
section \ref{sec:puresemantics}, excepting the portions dependent upon biproducts.

Next, define a classical structure on \cD.
\begin{definition}[Classical structure]\label{def:classicalstructure}
  A \emph{classical structure} in \cD{} is an objects $X$ and two maps, $\Delta :X \to X\* X$,
  $\epsilon:X\to I$ such that $X,\dgr{\Delta},\dgr{\epsilon},\Delta,\epsilon$ forms a special
  Frobenius algebra.
\end{definition}

As above, this allows us to define $\cD_{c}$, the category whose objects are the classical
structures of $\cD$ with maps between classical structures being the maps in $\cD$ between the
objects of the classical structure.

Note that a classical structure will induce a quantum structure, setting $\eta_{X}$ to be
$\dgr{\epsilon_{X}}\, \Delta_{X}$.

% section section_name (end)
\section{The category of Commutative Frobenius Algebras} % (fold)
\label{sec:the_category_of_commutative_frobenius_algebras}

% section the_category_of_commutative_frobenius_algebras (end)
% chapter frobenius_algebras_and_quantum_computation (end)
