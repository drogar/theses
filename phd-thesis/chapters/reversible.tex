%!TEX root = /Users/gilesb/UofC/thesis/phd-thesis/phd-thesis.tex

\chapter{Reversible computation}\label{chap:reversible_computation}

Bennet, in \cite{bennett:1973reverse}, showed that it was possible to
emulate a standard Turing machine via a reversible Turing machine and
vice-versa. This showed the equivalence of  standard and reversible
Turing machines. We reproduce the essence of this proof below.

\section{Reversible Turing machines} % (fold)
\label{sec:reversible_turing_machines}

Turing machines consist of a tape, a read-write head positioned over
the tape, a machine state and a set of instructions. The set of instructions
may be given as a set of transitions determining the movement of the
read-write head, what it writes and the resulting state of the machine.

\begin{definition}\label{def:standard_tape}
  Given an alphabet $A$ which does not contain a space, a tape is in standard
  format when:
  \begin{enumerate}
    \item The tape head is positioned directly over a blank space;
    \item The spaces to the left (the $+1$ direction) contain only
    elements of $A$.
    \item All other spaces of the tape are blank.
  \end{enumerate}
\end{definition}

\begin{definition}\label{def:turing_machine}
  A \emph{standard Turing machine} is given by a set of \emph{turing quintuples}:
  $Q=\{(s,\alpha,\alpha',\delta,s')\}$ where:
  \begin{enumerate}
    \item $s,s' \in S$, a set of predefined states of the machine;
    \item $\alpha, \alpha'\in A$, the set of glyphs that may be
      written on the machine and
    \item $\delta \in \{-1,0,1\}$ corresponds to the movement of the
    read-write head.
  \end{enumerate}
  We require that a particular $s,\alpha$ occur only
  once in the first two positions of the quintuple.
  Furthermore, the tape is in standard format and there are two special
  quintuples contained $Q$:
  \begin{enumerate}
    \item $(s_1,\_,\_,+1,s_2)$, the start quintuple;
    \item $(s_{t-1}, \_, \_, 0, s_{t})$, the end quintuple where
    $t$ is the number of states and is the final state of the machine.
  \end{enumerate}
  Finally, if the machine halts, it will halt in state $s_t$ and the output
  will be in standard format.
\end{definition}

The turing quintuples may also be regarded as giving the data for a partial
function in \sets: $\tau:S\times A \to A\times\{-1,0,1\}\times S$.
\begin{remark}
  A multi-tape Turing machine with $n$ tapes and read-write heads
  can be described by modifying
  definition \ref{def:turing_machine} such that $\alpha$ is an $n$-tuple
  of the set of glyphs for the Turing machine and $\delta$ is an
  $n$-tuple of movement directions.
\end{remark}

\begin{example}\label{exa:simple_turing_program}
  Suppose $S=\{start,run,reset,done\}$, $A=\{0,1,\_\}$ and the Turing machine
  program is given by the quintuples
  \begin{align*}
    &(start,\_,\_,+1,run),\\
    &(run,0,1,+1,run), (run,1,0,+1,run),\\
    &(run,\_,\_,-1,reset),\\
    &(reset,0,0,-1,reset),(reset,1,1,-1,reset),\\
    &(reset,\_,\_,0,done).
  \end{align*}
  This program will perform a ``bit-flip'' of all $0$s and $1$s on the tape
  until it reads a space, reposition the read head to the standard format
  and then it will halt.
\end{example}

As can be seen from \ref{exa:simple_turing_program}, it is \emph{possible}
that a Turing machine program is reversible. If we had chosen the second
quintuple to be $(run,0,0,+1,run)$ instead, the program would not have been
reversible.

The essential property that a Turing machine program would need to be reversible
is that the function $\tau$ defined from the quintuples is injective.

\begin{definition}\label{def:turing_quadruple}
  A \emph{turing quadruple} is given by a quadruple
  \[(s,[b_1,b_2,\ldots,b_n],[b'_1,b'_2,\ldots,b'_n],s')\]
  such that:
  \begin{itemize}
    \item $s,s'\in S$, some set of states;
    \item $b_i\in A  \cup \{\phi\}$ where $A$ is some alphabet;
    \item $b'_i\in A  + \{-1,0,1\}$ ;
    \item $b'_i \in \{-1,0,1\}$ if and only if $b_i = \phi$.
  \end{itemize}
\end{definition}

A turing quadruple explicitly splits the read/write action of the
Turing machine away from the movement. In a particular step for
tape $k$, the
turing machine would either read and write an item or it would not
read and then move.

Any turing quintuple may be split into two
turing quadruples by the addition of a new state $a''$ in $A$, where the
first quadruple will consist of all the read-write operations and
leave the Turing machine in state $a''$. The second quintuple will start
in state $a''$ and all the $b_i$ will be $\phi$, with $b'_i$ being
movements on each of the $n$ tapes.

\begin{definition}\label{def:reversible_turing_quadruple}
  A set of turing quadruples $Q$ is called \emph{reversible set of turing
  quadruples} when given $q_1,q_2\in Q$, with $q_1=(a,[b_i], [b'_i], a')$ and
  $q_2=(c,[d_i], [d'_i], c')$:
\begin{itemize}
  \item  if $a=c$, then there is a \kay where $b_\kay, d_\kay \in A$ and
    $b_\kay \ne d_\kay$;
  \item if $a' = c'$, then there is a \jay with $b'_\jay, d'_\jay \in A$ and
    $b'_\jay \ne d'_\jay$.
\end{itemize}
\end{definition}

Similarly to turing quintuples, turing quadruples may be taken as the
data for a function in \sets:
\[\rho:S\times(A  \cup \{\phi\}) \to (A  + \{-1,0,1\}) \times S.\]
We can see by inspection that $\rho$ is a reversible partial function when
the set of turing quadruples that give $\rho$ is a reversible set of turing
quadruples.

\begin{definition}\label{def:reversible_turing_machine}
  A \emph{reversible Turing machine} is one that is described by a
  set of reversible turing quadruples.
\end{definition}

We will show that a reversible Turing machine with three tapes can emulate
a Turing machine.








% section reversible_turing_machines (end)