%!TEX root = /Users/gilesb/UofC/thesis/phd-thesis/phd-thesis.tex

\chapter{Reversible computation}\label{chap:reversible_computation}

Bennet, in \cite{bennett:1973reverse}, showed that it was possible to
emulate a standard Turing machine via a reversible Turing machine and
vice-versa. This showed the equivalence of  standard and reversible
Turing machines. We reproduce the essence of this proof below.

\section{Reversible Turing machines} % (fold)
\label{sec:reversible_turing_machines}

Turing machines consist of a tape, a read-write head positioned over
the tape, a machine state and a set of instructions. The set of instructions
may be given as a set of transitions determining the movement of the
read-write head, what it writes and the resulting state of the machine.

\begin{definition}\label{def:standard_tape}
  Given an alphabet $A$ which does not contain a space, a tape is in \emph{standard
  format} when:
  \begin{enumerate}
    \item[\axiom{T}{1}] The tape head is positioned directly over a blank space;
    \item[\axiom{T}{2}] The spaces to the left (the $+1$ direction) contain only
    elements of $A$.
    \item[\axiom{T}{3}] All other spaces of the tape are blank.
  \end{enumerate}
\end{definition}

\begin{definition}\label{def:turing_quintuple}
  A \emph{turing quintuple} is a quintuple $(s,\alpha,\alpha',\delta,s')$ where:
  \begin{enumerate}
    \item[\axiom{Q}{1}] $s,s' \in S$, where $S$ is a predefined set of states;
    \item[\axiom{Q}{2}] $\alpha, \alpha'\in A$ is predefined set of glyphs;
    \item[\axiom{Q}{3}] $\delta \in \{-1,0,1\}$.
  \end{enumerate}
\end{definition}

\begin{definition}
  A \emph{standard turing quintuple set} $Q$ consists of a set of turing quintuples
  such that:
  \begin{enumerate}[{(}i{)}]
    \item If $q_1 = (s_1,\alpha_1,\alpha'_1,\delta_1,s'_1)$ and
      $q_2 = (s_2,\alpha_2,\alpha'_2,\delta_2,s'_2)$ are in $Q$, then either
      $s_1 \ne s_2$ or $\alpha_1\ne \alpha_2$ or both are not equal.
    \item There are two special  quintuples contained in $Q$:
      \begin{enumerate}
        \item $(s_1,\blank,\blank,+1,s_2)$\footnote{Here, $\blank$ is used to signify a blank.},
          the \emph{start quintuple};
        \item $(s_{t-1}, \blank, \blank, 0, s_{t})$, the \emph{end quintuple} where
        $t$ is the number of states and is the final state of the machine.
      \end{enumerate}
  \end{enumerate}
\end{definition}

\begin{definition}\label{def:turing_machine}
  A \emph{standard Turing machine} is given by
  \begin{itemize}
    \item[\axiom{TM}{1}] a standard turing quintuple set;
    \item[\axiom{TM}{2}] a tape that starts in standard format;
    \item[\axiom{TM}{3}] and the condition that   and if the machine halts, it will halt in state
      $s_t$, the final state of the end quintuple and the output will be in standard format.
  \end{itemize}
\end{definition}

The turing quintuples may also be regarded as giving the data for a partial
function in \sets: $\tau:S\times A \to A\times\{-1,0,1\}\times S$.
\begin{remark}
  A multi-tape Turing machine with $n$ tapes and read-write heads
  can be described by modifying
  definition \vref{def:turing_machine} such that $\alpha$ is an $n$-tuple
  of the set of glyphs for the Turing machine and $\delta$ is an
  $n$-tuple of movement directions.
\end{remark}

\begin{example}\label{exa:simple_turing_program}
  Suppose $S=\{start,run,reset,done\}$, $A=\{0,1,\blank\}$ and the Turing machine
  program is given by the quintuples
  \begin{align*}
    &(start,\blank,\blank,+1,run),\\
    &(run,0,1,+1,run), (run,1,0,+1,run),\\
    &(run,\blank,\blank,-1,reset),\\
    &(reset,0,0,-1,reset),(reset,1,1,-1,reset),\\
    &(reset,\blank,\blank,0,done).
  \end{align*}
  This program will perform a ``bit-flip'' of all $0$s and $1$s on the tape
  until it reads a space, reposition the read head to the standard format
  and then it will halt.
\end{example}

As we see in example \vref{exa:simple_turing_program}, it is \emph{possible}
that a Turing machine program is reversible. If we had chosen the second
quintuple to be $(run,0,0,+1,run)$ instead, the program would not have been
reversible.

The essential property that a Turing machine program needs to be reversible
is that the function $\tau$ defined from the quintuples is injective. In order
to simplify the discovery the function being injective, we reformulate the
turing quintuples as quadruples.

\begin{definition}\label{def:turing_quadruple}
  A \emph{turing quadruple} is given by a quadruple
  \[(s,[b_1,b_2,\ldots,b_n],[b'_1,b'_2,\ldots,b'_n],s')\]
  such that:
  \begin{itemize}[{(}i{)}]
    \item $s,s'\in S$, some set of states;
    \item $b_\jay\in A  \cup \{\phi\}$ where $A$ is some alphabet;
    \item $b'_\jay\in A  + \{-1,0,1\}$ ;
    \item $b'_\jay \in \{-1,0,1\}$ if and only if $b_\jay = \phi$.
  \end{itemize}
  In this definition, $b_\jay = \phi$ means that the value of tape
  \jay is ignored.
\end{definition}

A turing quadruple explicitly splits the read/write action of the
Turing machine away from the movement. In a particular step for
tape $k$, the
turing machine will either read and write an item or it will not
read and then move.

\begin{remark}\label{rem:quintuple_to_quadruple}
  Any turing quintuple may be split into two
  turing quadruples by the addition of a new state $a''$ in $A$, where the
  first quadruple will consist of all the read-write operations and
  leave the Turing machine in state $a''$. The second quadruple will start
  in state $a''$ and all the $b_\jay$ will be $\phi$, with $b'_\jay$ being
  movements on each of the $n$ tapes.
\end{remark}

\begin{definition}\label{def:reversible_turing_quadruple}
  A set of turing quadruples $Q$ is called \emph{reversible set of turing
  quadruples} when given $q_1,q_2\in Q$, with $q_1=(a,[b_\jay], [b'_\jay], a')$ and
  $q_2=(c,[d_\jay], [d'_\jay], c')$:
\begin{itemize}
  \item[\axiom{RTM}{1}]  if $a=c$, then there is a \kay where $b_\kay, d_\kay \in A$ and
    $b_\kay \ne d_\kay$;
  \item[\axiom{RTM}{2}] if $a' = c'$, then there is a \jay with $b'_\jay, d'_\jay \in A$ and
    $b'_\jay \ne d'_\jay$.
\end{itemize}
\end{definition}

Similarly to turing quintuples, turing quadruples may be taken as the
data for a function in \sets:
\[\rho:S\times(A  \cup \{\phi\}) \to (A  + \{-1,0,1\}) \times S.\]
We can see by inspection that $\rho$ is a reversible partial function when
the set of turing quadruples that give $\rho$ is a reversible set of turing
quadruples.

\begin{definition}\label{def:reversible_turing_machine}
  A \emph{reversible Turing machine} is one that is described by a
  set of reversible turing quadruples.
\end{definition}

We will show that a reversible Turing machine with three tapes can emulate
a Turing machine.

\begin{theorem}[Bennet\cite{bennett:1973reverse}]\label{thm:reversible_turing_machine_emulates_standard}
  Given a standard Turing Machine $M$, it may be emulated by a three tape reversible
  Turing machine $R$. In this case, emulated means:
  \begin{itemize}[{(}i{)}]
    \item $M$ halts on standard input $I$ if and only if $R$ halts on standard input $(I,\blank,\blank)$.
    \item $M$ halts on standard input $I$ producint standard output $O$,
      if and only if $R$ halts on input ($I,\blank,\blank$) producing standard output $(I,\blank,O)$.
  \end{itemize}
\end{theorem}
\begin{proof}
  (Sketch only).

  The crux of the proof is to convert the quintuples of $M$ to the quadruples of
  $R$ as noted in remark \vref{rem:quintuple_to_quadruple}. Explicitly for a single
  tape machine, we have
  \begin{align}
    (s,a,a,\delta,s') \mapsto \left((s,a,a',s''),
          (s'',\phi,\delta,s')\right).\label{eq:quint_to_quad}
  \end{align}
  
  In \vref{eq:quint_to_quad}, $s''$ is a new state for the machine $M$, 
  not in the current set of states.

  Assign an order to the $n$ quintuples of $M$, where the start
  quintuple is the first in the order and the end quintuple comes last. Convert these to
  quadruples as in \vref{eq:quint_to_quad}. 
  
  We then proceed to create three groups of quadruples for $R$.
  We call these \emph{emulation, copy,} and \emph{restore}.

  To create the emulation phase quadruples, we examine the pairs of quadruples of $M$
  in the sorted order and produce a pair of quadruples for $R$.
  \begin{align*}
    \text{Pair 1}\quad (s_1,\blank,\blank,s_1'') & \mapsto (s_1,[\blank,\phi,\blank],[\blank,+1,\blank],e_1)\\
    (s_1'',\phi,\delta,s_2) & \mapsto
          (e_1,[\phi,\blank,\phi],[\delta,1,0],s_2)\\
          & \vdots \\
    \text{Pair \jay}\quad  (s_\kay,a_\jay,a'_\jay,s''_\kay) & 
          \mapsto (s_\kay,[a_\jay,\phi,\blank],[a'_\jay,+1,\blank],e_\jay)\\
     (s_\kay'',\phi,\delta,s_i) & \mapsto
          (e_\jay,[\phi,\blank,\phi],[\delta_\jay,\jay,0],s_i)\\
          & \vdots \\
    \text{Pair }n \quad (s_\ell,\blank,\blank,s_\ell'') & \mapsto (s_\ell,[\blank,\phi,\blank],[\blank,+1,\blank],e_n)\\
     (s_\ell'',\phi,0,s_f) & \mapsto
          (e_n,[\phi,\blank,\phi],[0,n,0],s_f).
  \end{align*}
  By inspection, one can see that even if the quadruples of $M$ were not a reversible set, the
  set created for $R$ is a reversible set, due to the writing of the quadruple index on tape 2.
  Upon completion of the emulation phase, tape 1 will be the same as $M$ would have produced on
  its single tape, tape 2 will be $[1,2,\dots,n]$ and tape 3 will be blanks.
  
  For the copy phase, we create the following quadruples:
  \begin{align*}
    (s_f,[\blank,n,\blank],&[\blank,n,\blank],c_1 )\\
    (c_1,[\phi,\phi,\phi],&[+1,0,+1],c'_1 )\\
    (c'_1,[x,n,\blank],&[x,n,x],c_1 )\quad \text{when } x \ne \blank\\
    (c'_1,[\blank,n,\blank],&[\blank,n,x],c_2 )\\
    (c_2,[\phi,\phi,\phi],&[-1,0,-1],c'_2 )\\
    (c'_2,[x,n,x],&[x,n,x],c_2 )\quad \text{when }x \ne \blank\\
    (c'_2,[\blank,n,_],&[\blank,n,\blank],r_\ell).
  \end{align*}
  In these quadruples, the states $\{c_1,c'_1,c_2,c'_2\}$ should be chosen to be
  distinct from the states in the emulation phase. As an example, set them as follows:
  \[
    c_1 = (\{c\},s_1)\quad c'_1 = (\{c'\},s_1)\quad c_2 = (\{c\},s_f)\quad c'_1 = (\{c'\},s_f).
  \]
  At the completion of this phase, tapes 1 and 2 will be unchanged and tape 3 will be a copy
  of tape 1. 

  Finally we perform the restore phase where the history will be erased and tape 1 reset to
  the input. The quadruples that will accomplish this are:
  \begin{align*}
      \text{Pair }n \quad (r_n,[\phi,n,\phi],&[0,\blank,0],r'_n)\\
      (r'_n,[\blank,\phi,\blank],&[\blank,-1,\blank],r_{n-1})\\
      &\vdots\\
      \text{Pair }\jay \quad (r_\kay,[\phi,\jay,\phi],&[-\delta_\jay,\blank,0],r'_\jay)\\
      (r'_\jay,[a'_\jay,\phi,\blank],&[a_\jay,-1,\blank],r_i)\\
      &\vdots\\
      \text{Pair 1} \quad (r_2,[\phi,1,\phi],&[-1,\blank,0],r'_1)\\
      (r'_1,[\blank,\phi,\blank],&[\blank,-1,\blank],r_1).
  \end{align*}
  The $r$ states are derived from the $s$ states of the emulation phase.
  \[
    r_\jay = (\{r\},s_\jay) \qquad    r'_\jay = (\{r'\}, s_\jay).
  \]
  In this restore phase, the indexes of the states $r$ match up to the indexes of states $s$. The
  quadruples reverse the actions of the emulate phase on tape 1, erase the history on tape 2 and
  make no change to tape 3.

\end{proof}


% section reversible_turing_machines (end)

\section{Reversible automota and linear combinatory algebras} % (fold)
\label{sec:reversible_automota_and_linear_combinatory_algebras}
While reversible Turing machines, as described in \vref{sec:reversible_turing_machines},
show that reversible computing is as powerful as standard computing, they do not 
give us a sense of what may be considered to be happening at a higher level. 

To accomplish that task we examine the results of the paper 
``A Structural Approach to Reversible Computation''\cite{abramsky05:reversible}. In this paper,
Abramsky gives a description of a reversible automaton together with a linear combinatory 
algebra. We will begin by revisiting some definitions and constructions necessary for 
discussing automota. The next subsection will introduce combinatory algebras, after which we will
describe the reversible automata of \cite{abramsky05:reversible} and add a short proof that 
it can emulate a reversible turing machine.

\subsection{Automota} % (fold)
\label{sub:automota}
We will describe the automota as a term-rewriting system. This requires, of course, giving a 
few basic definitions. See, e.g., \cite{termrewriting2003}.
\begin{definition}\label{def:arity}
  An \emph{arity} is a function from a function to the natural numbers. The arity of
  $F$ is the number of inputs (arguments) required by $F$.
\end{definition} 
\begin{definition}\label{def:signature}
  A \emph{signature} $\Sigma$ is a set of \emph{function symbols} $F,G,\ldots$, each of which
  has an arity.  
\end{definition}

\begin{remark}
  We refer to functions with low arity in the following ways:
  \begin{itemize}
    \item $Arity = 0$.  These are known as \emph{nullary} functions or constants.
    \item $Arity = 1$.  These are known as \emph{unary} functions.
    \item $Arity = 2$.  These are known as \emph{binary} functions.
  \end{itemize}
\end{remark}

\begin{definition}\label{def:term_alphabet}
  A \emph{term alphabet} is a set $A$ containing a signature $\Sigma$ and a countably infinite 
  set $X$, the variables. Furthermore, $\Sigma \cap X = \phi$.
\end{definition}

\begin{definition}\label{def:term_algebra}
  A \emph{term algebra} of the term alphabet  $\Sigma \cup X$ is denoted by $T_\Sigma(X)$ and 
  defined as follows:
  \begin{itemize}
    \item $x\in V \implies x \in T_\Sigma$ and
    \item For any $F\in\Sigma$, with $arity(F) = n$, and $\{t_1,\ldots,t_n\} \subseteq T_\Sigma$,
    then $F(t_1,\ldots,t_n)\in T_\Sigma$. In the case where $arity(F) = 0$, we write 
    $F \in T_\Sigma$.
  \end{itemize}
\end{definition}

\begin{definition}\label{def:ground_terms}
  The \emph{ground terms} of a term algebra are those terms that do not contain any variable.
  The set of these terms is designated as $T_\Sigma.$
\end{definition}

\begin{remark}
  Note the ground terms consist of the constants and recursively applying the function
  symbols of $\Sigma$ to them.
\end{remark}

As we are considering rewrite systems, we will need to consider aspects of substitution and 
unification.

\begin{definition}\label{def:substitution}
  A \emph{substitution} is a map $\sigma:\tavar \to \tavar$ which is natural for all function
  symbols in $\Sigma$. In particular if $arity(c) = 0$ then $\sigma(c) = c$.
\end{definition}

Note that given the above definition a substitution $\sigma$ is completely determined by its action
on variables. If $\sigma:X\to X$ and is injective, we call $\sigma$ a renaming. Moreover, if 
$\sigma$ restricted to the variables in a term $t$ is an injective map of $X$ on those variables,
we call $sigma$ a renaming of $t$.

Substitution allows us to define a partial order on $\tavar$, as follows:
\begin{definition}\label{def:subsumption_order}
  In $\tavar$, let $\sigma(t) = s$. Then we say $s$ is an \emph{instance} of $t$, written 
  $s\preceq t$. Moreover, if $\sigma$ is not just a renaming for $t$, then we write $s\prec t$. If
  $\sigma$ is a renaming of $t$, we write $s\simeq t$.
\end{definition}

\begin{lemma}\label{lem:subsumption_is_a_partial_order}
  Subsumption, as defined in \vref{def:subsumption_order} is a partial order, i.e., it is
  transitive and reflexive.
\end{lemma}
\begin{proof}
  % TODO - Prove that subsumption (term algebra substitution is partial order)
\end{proof}

\begin{lemma}\label{lem:most_general_instance}
  Given terms $r,t$ such that there is at least one $s$ with $s\preceq r$ and $s\preceq t$, 
  then there exists a $g$ such that $g\preceq r$ and $g\preceq t$ and for any 
  $s'$ with $s'\preceq r$ and $s'\preceq t$ we will have $s' \preceq g$.
\end{lemma}
\begin{proof}
  % TODO Prove most general instance of terms with common instance
  \prepprooflist
  \begin{enumerate}
    \item Algorithm to compute supremum of $p,q$ terms.
    \item Strict $\prec$ has no infinite ascending chains.
    \item Shows main part - there exists.
    \item Can now show it is unique up to renaming.
  \end{enumerate}
\end{proof}

The subsumption ordering can be used to derive a similar ordering on substitutions:
\begin{definition}\label{def:substituion_ordering}
  $\sigma \preceq \tau$ if and only if there is a $\rho$ with $\sigma = \tau \rho$, where
  $\tau \rho$ is the diagrammatic order composition of the two substitutions.
\end{definition}

\begin{definition}\label{def:unifier}
  For terms $s,t$, if $\sigma(t) = \sigma(s)$, then the substitution $\sigma$ is called a 
  \emph{unifier} for the terms $s,t$.
\end{definition}

\begin{lemma}\label{lem:most_general_unifier}
  If $s,t$ are terms with a unifier $\sigma$, there exists a substitution $\tau$ that 
  unifies $s,t$ such that $\tau \preceq \rho$ whenever $\rho$ unifies $s,t$. $\rho$ is called 
  the \emph{most general unifier} of $s$ and $t$.
\end{lemma}
\begin{proof}
  % TODO Prove most general unifier exists
  Follows from \vref{lem:most_general_instance}.
\end{proof}

\begin{notation}
  Following \cite{abramsky05:reversible}, we write \uniftus if $\sigma$ is the 
  most general unifier of terms $t,u$. 
\end{notation}
% subsection automota (end)

\subsection{Combinatory Algebra} % (fold)
\label{sub:combinatory_algebra}

\begin{definition}\label{def:combinatory_algebra}
  A \emph{combinatory algebra} is an algebra with one binary operation, $\cdot$ written in infix
  notation. The operation is not assumed to be associative. Multi-element expressions such as 
  $a \. b \. c$ are to be taken as associating to the left, that is,
  \[
    a \. b \. c = (a \.b)\.c.
  \]
  The combinatory algebra may possess distinguished elements that are subject to 
  specific rewrite rules.
\end{definition}

\begin{definition}\label{def:combinatory_logic}
  \emph{Combinatory logic} is the combinatory algebra with two distinguished elements, \Kc and \Sc,
  such that the following hold:
  \begin{singlespace}
    \begin{align*}
      &\Kc \cdot x \cdot y &&= x\\
      &\Sc \cdot x \cdot y \cdot z &&= x \cdot z \cdot (y \cdot z).
    \end{align*}
  \end{singlespace}
\end{definition}

Note that combinatory logic does not require a specific set that must be used for the algebra,
simply that it has the two distinguished elements. 

Combinatory logic was shown to be equivalent to the $\lambda$ calculus by 
% TODO Get reference for who showed CL = lambda, and who did it.

For example, we may define the identity combinator \Ic as $\Ic = \Sc \. \Kc \. \Kc$. 
Further combinators may be defined, such as the \Bc combinator, defined by
$\Bc \. a \. b \. c = a \. (b\. c)$. The \Sc and \Kc combinators are complete, in that other
combinators such as \Bc may be defined from them. E.g., $\Bc = \Sc\.(\Kc\.\Sc)\.\Kc$. 
In fact, we may define an alternate combinatory algebra that is equivalent to Combinatory Logic.

\begin{definition}\label{def:bckw_algebra}
  A \emph{BCKW-Combinatory algebra} is a Combinatory Algebra with four distinguish elements, 
  \Bc, \Cc, \Kc, and \Wc subject to the following equations:
  \begin{singlespace}
    \begin{align*}
      &\Bc \. a\.b\.c   &&=a\.(b\.c)&\qquad\qquad\qquad\qquad\qquad\qquad\\
      &\Cc \. a\.b\.c   &&=a\.c\.b\\
      &\Kc \. a\.b      &&=a\\
      &\Wc\. a\.b       && = a\.b\.b
    \end{align*}
  \end{singlespace}
\end{definition}

In fact, a BCKW-Combinatory algebra is equivalent to a Combinatory logic.

\begin{lemma}\label{lem:bckw_combinatory_is_sk_combinatory}
  The distinguished elements of a BCKW-Combinatory algebra may be represented by \Sc and \Kc. 
  Conversly, the \Sc and \Kc of a Combinatory logic may be created from \Bc,\Cc,\Kc and \Wc.
\end{lemma}
\begin{proof}
  For the first statement, we have:
  \begin{singlespace}
    \begin{align*}
      &\Bc &&= \Sc\. (\Kc\.\Sc)\.\Kc\qquad\qquad\qquad\qquad\qquad\qquad\qquad\qquad\\
      &\Cc &&= \Sc\.(\Sc\.(\Kc\.(\Sc\.(\Kc\.\Sc)\.\Kc))\.\Sc)\.(\Kc\.\Kc)\\
      &\Kc &&= \Kc\\
      &\Wc &&= \Sc\.\Sc\.(\Sc\.\Kc).\\
      \\
      \text{Going the other direction, we have:}\\
      &\Ic &&= \Wc\.\Kc\\
      &\Kc &&= \Kc\\
      &\Sc &&= \Bc\.(\Bc\.(\Bc\.\Wc)\.\Cc)\.(\Bc\.\Bc) \text{ and}\\
      &    &&= \Bc\.(\Bc\.\Wc)\.(\Bc\.\Bc\.\Cc).
    \end{align*}
  \end{singlespace}
  
  We show the computations of \Bc and \Sc in detail.
  \begin{singlespace}
    \begin{align*}
      \Bc\.a\.b\.c &=\Sc\. (\Kc\.\Sc)\.\Kc \.a\.b\.c \\
       &= (\Kc\.\Sc)\.a\.(\Kc\.a)\.b\.c\\
       &= \Sc\.(\Kc\.a)\.b\.c\\
       &= \Kc\.a\.c\.(b\.c)\\
       &= a\.(b\.c)\\
       \\
      \Sc\.a\.b\.c&=\Bc\.(\Bc\.\Wc)\.(\Bc\.\Bc\.\Cc)\.a\.b\.c\\
      &=(\Bc\.\Wc)\.((\Bc\.\Bc\.\Cc)\.a)\.b\.c\\
      &=\Bc\.\Wc\.((\Bc\.\Bc\.\Cc)\.a)\.b\.c\\
      &=\Wc\.(((\Bc\.\Bc\.\Cc)\.a)\.b)\.c\\
      &=(((\Bc\.\Bc\.\Cc)\.a)\.b)\.c\.c\\
      &=\Bc\.\Bc\.\Cc\.a\.b\.c\.c\\
      &=\Bc\.(\Cc\.a)\.b\.c\.c\\
      &=(\Cc\.a)\.(b\.c)\.c\\
      &=\Cc\.a\.(b\.c)\.c\\
      &=a\.c\.(b\.c)
    \end{align*}
  \end{singlespace}
\end{proof}

If we use the notation $a^n \. b$ to mean $a\.a\.\cdots\.a\.b$ where $a$ is repeated $n$ times,
then we can terms which correspond to the Church numbers of lambda calculus:
\[
  \bar{n}\equiv (\Sc\.\Bc)^n\.(\Kc\.\Ic)
\]
% TODO find or create definition of B combinator from S and K

\begin{definition}\label{def:functions_representable_in_combinatory_logic}
  A partial function $f:\nat \to \nat$ is \emph{representable} in combinatory logic if there is a
  term $M_f$ such that $M_f\.\bar{n} = \bar{m}$ whenever $f(n) = m$ and $M_f\.\bar{n} $ does
  not have a normal form if $f(n)\uparrow$.
\end{definition}

When we say that combinatory logic with \Sc and \Kc is complete, we mean the following theorem:
\begin{theorem}\label{thm:combinatory_logic_is_complete}
  The partial functions that are representable in combinatory logic are exactly the partial 
  recursive functions.  
\end{theorem}

% subsection combinatory_algebra (end)

\subsection{Linear Combinatory Algebra} % (fold)
\label{sub:linear_combinatory_algebra}

\begin{definition}\label{def:linear_combinatory_algebra}
  A \emph{Linear Combinatory Algebra} $(A,\.,!)$ is an algebra $A$ with an applicative binary 
  operation $\.$, an unary operator $!:A\to A$ and eight distinquished elements: \Bc, \Cc, \Ic,
  \Kc, \Dc, \dc, \Fc and \Wc in $A$ which satisfy the following rules:
  \begin{singlespace}
    \begin{align*}
      &1.\ \Bc\. a\.b\.c     &&=a\.(b\.c)&\qquad\qquad\qquad\qquad\qquad\qquad\\
      &2.\ \Cc \. a\.b\.c    &&=a\.c\.b\\
      &3.\ \Ic \. a          &&=a\\
      &4.\ \Kc \. a\.{!}b    &&=a\\
      &5.\ \Dc\.{!}a         &&=a\\
      &6.\ \dc\.{!}a         &&={!!}a\\
      &7.\ \Fc\.{!}a\.{!}b   &&={!}(a\.b)\\
      &8.\ \Wc\. a\.{!}b     && = a\.{!}b\.{!}b
    \end{align*}
  \end{singlespace}
\end{definition}

Note that a Linear Combinatory Algebra always contains a BCKW-Combinatory algebra.

Define $\Dc' = \Cc\.(\Bc\.\Bc\.\Ic)\.(\Bc\.\Dc\.\Ic)$ and the binary operator $\bullet$ on $A$ such
that $a\bullet b \equiv a\.{!}b$. Then, define the following:
\begin{align*}
  &\Bc_s  &&=\Cc\.(\Bc\.(\Bc\.\Bc\.\Bc)\.(\Dc'\.\Ic))\.(\Cc\.((\Bc\.\Bc)\.\Fc)\.\dc)\\
  &\Cc_s  &&=\Dc'\.\Cc\\
  &\Kc_s  &&=\Dc'\.\Kc\\
  &\Wc_s  &&=\Dc'\.\Wc.
\end{align*}

\begin{lemma}\label{lem:linear_combinatory_algebra_has_a_bckw_algebra}
  Given and Linear Combinatory Algebra $(A,\.,{!})$, then $(A,\bullet)$ is a BCKW-Combinatory
  algebra with $\Bc,\Cc,\Kc,\Wc$ set to $\Bc_s,\Cc_s,\Kc_s,\Wc_s$ from above.
\end{lemma}
\begin{proof}
  We show the calculation for $\Kc_s$, the others are similar.
  \begin{singlespace}
    \begin{align*}
      \Kc_s\bullet a\bullet b &\equiv\Dc'\. \Kc \. {!}a \. {!}b\\
       &= \Cc\.(\Bc\.\Bc\.\Ic)\.(\Bc\.\Dc\.\Ic)\. \Kc \. {!}a \. {!}b\\
       &= (\Bc\.\Bc\.\Ic\. \Kc)\.(\Bc\.\Dc\.\Ic) \. {!}a \. {!}b\\
       &= \Bc\.(\Ic\. \Kc)\.(\Bc\.\Dc\.\Ic) \. {!}a \. {!}b\\
       &= (\Ic\. \Kc)\.((\Bc\.\Dc\.\Ic) \. {!}a) \. {!}b\\
       &= \Kc\.((\Bc\.\Dc\.\Ic) \. {!}a) \. {!}b\\
       &= (\Bc\.\Dc\.\Ic) \. {!}a\\
       &= \Dc\.(\Ic \. {!}a)\\
       &= \Dc\. {!}a\\
       &= a\\
    \end{align*}
  \end{singlespace}
  
\end{proof}

% subsection linear_combinatory_algebra (end)

% section reversible_automota_and_linear_combinatory_algebras (end)


























