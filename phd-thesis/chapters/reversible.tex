%!TEX root = /Users/gilesb/UofC/thesis/phd-thesis/phd-thesis.tex

\chapter{Reversible computation}\label{chap:reversible_computation}

Bennet, in \cite{bennett:1973reverse}, showed that it was possible to
emulate a standard Turing machine via a reversible Turing machine and
vice-versa. This showed the equivalence of  standard and reversible
Turing machines. We reproduce the essence of this proof below.

\section{Reversible Turing machines} % (fold)
\label{sec:reversible_turing_machines}

Turing machines consist of a tape, a read-write head positioned over
the tape, a machine state and a set of instructions. The set of instructions
may be given as a set of transitions determining the movement of the
read-write head, what it writes and the resulting state of the machine.

\begin{definition}\label{def:standard_tape}
  Given an alphabet $A$ which does not contain a space, a tape is in \emph{standard
  format} when:
  \begin{enumerate}
    \item The tape head is positioned directly over a blank space;
    \item The spaces to the left (the $+1$ direction) contain only
    elements of $A$.
    \item All other spaces of the tape are blank.
  \end{enumerate}
\end{definition}

\begin{definition}\label{def:turing_quintuple}
  A \emph{turing quintuple} is a quintuple $(s,\alpha,\alpha',\delta,s')$ where:
  \begin{enumerate}
    \item $s,s' \in S$, where $S$ is a predefined set of states;
    \item $\alpha, \alpha'\in A$ is predefined set of glyphs;
    \item $\delta \in \{-1,0,1\}$.
  \end{enumerate}
\end{definition}

\begin{definition}
  A \emph{standard turing quintuple set} $Q$ consists of a set of turing quintuples
  such that:
  \begin{enumerate}
    \item If $q_1 = (s_1,\alpha_1,\alpha'_1,\delta_1,s'_1)$ and
      $q_2 = (s_2,\alpha_2,\alpha'_2,\delta_2,s'_2)$ are in $Q$, then either
      $s_1 \ne s_2$ or $\alpha_1\ne \alpha_2$ or both are not equal.
    \item There are two special  quintuples contained in $Q$:
      \begin{enumerate}
        \item $(s_1,\_,\_,+1,s_2)$\footnote{Here, $\_$ is used to signify a blank.},
          the \emph{start quintuple};
        \item $(s_{t-1}, \_, \_, 0, s_{t})$, the \emph{end quintuple} where
        $t$ is the number of states and is the final state of the machine.
      \end{enumerate}
  \end{enumerate}
\end{definition}

\begin{definition}\label{def:turing_machine}
  A \emph{standard Turing machine} is given by
  \begin{itemize}
    \item a standard turing quintuple set;
    \item a tape that starts in standard format;
    \item and the condition that   and if the machine halts, it will halt in state
      $s_t$, the final state of the end quintuple and the output will be in standard format.
  \end{itemize}
\end{definition}

The turing quintuples may also be regarded as giving the data for a partial
function in \sets: $\tau:S\times A \to A\times\{-1,0,1\}\times S$.
\begin{remark}
  A multi-tape Turing machine with $n$ tapes and read-write heads
  can be described by modifying
  definition \vref{def:turing_machine} such that $\alpha$ is an $n$-tuple
  of the set of glyphs for the Turing machine and $\delta$ is an
  $n$-tuple of movement directions.
\end{remark}

\begin{example}\label{exa:simple_turing_program}
  Suppose $S=\{start,run,reset,done\}$, $A=\{0,1,\_\}$ and the Turing machine
  program is given by the quintuples
  \begin{align*}
    &(start,\_,\_,+1,run),\\
    &(run,0,1,+1,run), (run,1,0,+1,run),\\
    &(run,\_,\_,-1,reset),\\
    &(reset,0,0,-1,reset),(reset,1,1,-1,reset),\\
    &(reset,\_,\_,0,done).
  \end{align*}
  This program will perform a ``bit-flip'' of all $0$s and $1$s on the tape
  until it reads a space, reposition the read head to the standard format
  and then it will halt.
\end{example}

As can be seen from example \vref{exa:simple_turing_program}, it is \emph{possible}
that a Turing machine program is reversible. If we had chosen the second
quintuple to be $(run,0,0,+1,run)$ instead, the program would not have been
reversible.

The essential property that a Turing machine program needs to be reversible
is that the function $\tau$ defined from the quintuples is injective. In order
to simplify the discovery the function being injective, we reformulate the
turing quintuples as quadruples.

\begin{definition}\label{def:turing_quadruple}
  A \emph{turing quadruple} is given by a quadruple
  \[(s,[b_1,b_2,\ldots,b_n],[b'_1,b'_2,\ldots,b'_n],s')\]
  such that:
  \begin{itemize}
    \item $s,s'\in S$, some set of states;
    \item $b_\jay\in A  \cup \{\phi\}$ where $A$ is some alphabet;
    \item $b'_\jay\in A  + \{-1,0,1\}$ ;
    \item $b'_\jay \in \{-1,0,1\}$ if and only if $b_\jay = \phi$.
  \end{itemize}
  In this definition, $b_\jay = \phi$ means that the value of tape
  \jay is ignored.
\end{definition}

A turing quadruple explicitly splits the read/write action of the
Turing machine away from the movement. In a particular step for
tape $k$, the
turing machine will either read and write an item or it will not
read and then move.

\begin{remark}\label{rem:quintuple_to_quadruple}
  Any turing quintuple may be split into two
  turing quadruples by the addition of a new state $a''$ in $A$, where the
  first quadruple will consist of all the read-write operations and
  leave the Turing machine in state $a''$. The second quadruple will start
  in state $a''$ and all the $b_\jay$ will be $\phi$, with $b'_\jay$ being
  movements on each of the $n$ tapes.
\end{remark}

\begin{definition}\label{def:reversible_turing_quadruple}
  A set of turing quadruples $Q$ is called \emph{reversible set of turing
  quadruples} when given $q_1,q_2\in Q$, with $q_1=(a,[b_\jay], [b'_\jay], a')$ and
  $q_2=(c,[d_\jay], [d'_\jay], c')$:
\begin{itemize}
  \item  if $a=c$, then there is a \kay where $b_\kay, d_\kay \in A$ and
    $b_\kay \ne d_\kay$;
  \item if $a' = c'$, then there is a \jay with $b'_\jay, d'_\jay \in A$ and
    $b'_\jay \ne d'_\jay$.
\end{itemize}
\end{definition}

Similarly to turing quintuples, turing quadruples may be taken as the
data for a function in \sets:
\[\rho:S\times(A  \cup \{\phi\}) \to (A  + \{-1,0,1\}) \times S.\]
We can see by inspection that $\rho$ is a reversible partial function when
the set of turing quadruples that give $\rho$ is a reversible set of turing
quadruples.

\begin{definition}\label{def:reversible_turing_machine}
  A \emph{reversible Turing machine} is one that is described by a
  set of reversible turing quadruples.
\end{definition}

We will show that a reversible Turing machine with three tapes can emulate
a Turing machine.

\begin{theorem}[Bennet\cite{bennett:1973reverse}]\label{thm:reversible_turing_machine_emulates_standard}
  Given a standard Turing Machine $M$, it may be emulated by a three tape reversible
  Turing machine $R$. In this case, emulated means:
  \begin{itemize}
    \item $M$ halts on standard input $I$ if and only if $R$ halts on standard input $(I,\_,\_)$.
    \item $M$ halts on standard input $I$ producint standard output $O$,
      if and only if $R$ halts on input ($I,\_,\_$) producing standard output $(I,\_,O)$.
  \end{itemize}
\end{theorem}
\begin{proof}
  (Sketch only).

  The crux of the proof is to convert the quintuples of $M$ to the quadruples of
  $R$ as noted in remark \vref{rem:quintuple_to_quadruple}. Explicitly,
  \begin{align}
    (s,[a_\jay],[a'_\jay],[\delta_\jay],s') \mapsto \left((s,[a_\jay],[a'_\jay],s''),
          (s'',[\phi,\phi,\ldots,\phi],[\delta_\jay],s')\right).\label{eqn:quint_to_quad}
  \end{align}

  To do that, first assign an order to quintuples of $M$, where the start
  quintuple is the first in the order and the end quintuple comes last. Convert these
  as in the remark. We may then proceed to create three groups of quadruples for $R$.
  We call these \emph{emulation, copy,} and \emph{restore}.

  The emulation phase pairs of quadruples are created directly from the pairs of
  quadruples of $M$:
  \begin{align}
    (s,a,a',s'') & \mapsto (s,[a,\phi,\_],[a',+1,\_],s'')\\
    (s'',\phi,\delta,s') & \mapsto
          (s'',[\phi,\_,\phi],[\delta,m,0],s'),
  \end{align}
  where the $m$ corresponds to this being the $m^{\text{th}}$ pair in the ordering we chose.

  The effect of the emulation phase is to transform tape 1 into the output that $M$ would have
  produced, write the number of pairs of quadruples on tape 2 and write 0s on tape 3.

  For the copy phase



  In \vref{eqn:quint_to_quad}, $s''$ is a fresh state, not in the states of $M$.
\end{proof}


% section reversible_turing_machines (end)