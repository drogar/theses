%!TEX root = /Users/gilesb/UofC/thesis/phd-thesis/phd-thesis.tex
\chapter{Disjoint Sums} % (fold)
\label{chap:disjoint_sums}

Suppose \X is an inverse category  with a restriction zero, and $\+$ is a tensor product on $\X$.
Given specific  conditions regarding the tensor it is possible to define disjointness based upon the
action of the tensor. We are assuming the following naming for the standard monoidal tensor
isomorphisms:
\begin{align*}
   \upl &: 0 \+ A \to A &
   \upr &: A \+ 0 \to A\\
   a_\+ &: (A \+ B) \+ C \to A \+ (B \+ C) &
   c_\+ &: A \+ B \to B \+ A.
 \end{align*}

\section{Disjointness via a tensor}
\label{sec:disjointness_via_a_tensor}

\begin{definition}\label{def:disjointness_tensor}
  Given we have an inverse category \X with restriction zero and a symmetric monoidal tensor
  $\+$, the tensor $\+$ is a \emph{disjointness tensor} when:
  \begin{enumerate}[{(}i{)}]
    \item It is a restriction functor --- i.e., $\_ \+ \_ : \X \times \X \to \X$.
    \item The unit is the restriction zero. ($0 : \boldsymbol{1}\to \X$ picks out the restriction
    zero in \X).
    \item Define $\cpa = \inv{\upr}(1\+0):A\to A\+B$ and $\cpb = \inv{\upl}(0\+1): A\to B\+A$.
      $\cpa$ and $\cpb$ must be jointly epic. That is, if $\cpa f = \cpa g$ and $\cpb f = \cpb g$, then
      $f = g$.
    \item Define $\scpa := (1\+0)\upr: A\+B \to A$ and $\scpb := (0\+1)\upl:A\+B \to B$.
      $\scpa$  and $\scpb$ must be jointly monic. That is, whenever $f\scpa = g \scpa$ and
      $f\scpb = g\scpb$ then $f = g$.
  \end{enumerate}
\end{definition}

\begin{example}[\pinj has a disjointness tensor]\label{ex:pinj_has_disjointness_tensor}
  In \pinj, the disjoint union, $\disjointunion$, is a disjointness tensor. We will designate elements of
  the disjoint union as pairs of the elements of the original sets and the order in the disjoint
  join. That is, when
  \[
     A=\{a\}, B=\{b\}, \text{ then } A\disjointunion B = \{(x,n) | n\in\{1,2\},n =1 \implies x \in A, n = 2
     \implies x\in B\}.
  \]
  Setting $\+$ as $\disjointunion$, we  have the identity for the tensor is $\emptyset$. The action
  of the tensor on maps $f:A\to C = \{(a,c)\}$, $g:B\to D = \{(b,d)\}$ is given by:
  \[
    f\+g:A\+B\to C\+D = \{((x,n),(v,m)) | (x,v)\in f \text{ or } (x,v) \in g\}.
  \]
  From our definitions above, we may define our tensor structure maps:
  \begin{align*}
    \upl &:0\+A \to A, &(a,A) \mapsto a\\
    \upr &:A\+0 \to A, &(a,A) \mapsto a\\
    a_\+ &: (A\+B)\+C, &((a,1),1) \mapsto (a,1)\\
     &  &((b,2),1) \mapsto ((b,1),2)\\
     &  &(c,2) \mapsto ((c,1),2)\\
    c_\+ &: A\+B \to B\+ A & (a,1) \mapsto (a,2)\\
    & & (a,2) \mapsto (a,1)
  \end{align*}

  The map $\cpa = \inv{\upr}(1\+0)$ is given by the mapping $a\in A\mapsto (a,1) \in
  A\+B$. Similarly, $\cpb = \inv{\upl}(0\+1)$ is given by the mapping $a\in A\mapsto (a,2) \in
  B\+A$. We immediately see $\cpa$ and $\cpb$ are jointly epic. Similarly, $\icpa$ and $\icpb$ are
  jointly monic.
\end{example}

\begin{lemma}\label{lem:zero_plus_zero_is_zero}
  Given an inverse category \X with restriction zero and disjointness tensor $\+$, then the map
  $0\+0: A\+B \to C\+D$ is the map $0: A\+B \to C\+D$.
\end{lemma}
\begin{proof}
  Recall the zero map factors through the restriction zero, i.e. $0:A \to B$ is the same as
  saying $A\xrightarrow{!} 0 \xrightarrow{\why}B$. Additionally, as objects, $0\+0 \cong 0$ ---
  the restriction zero.

  Therefore the map $0\+0: A\+B \to C\+D$ is writable as
  \[
    A\+B \xrightarrow{!\+!}0\+0  \xrightarrow{\why\+\why} C\+D,
  \]
  which may then be rewritten as
  \[
    A\+B \xrightarrow{!\+!}0\+0 \xrightarrow{\upl} 0
      \xrightarrow{\inv{\upl}} 0\+0 \xrightarrow{\why\+\why} C\+D.
  \]

  But by the properties of the restriction zero, $(!\+!) \upl = !$ and $\inv{\upl}(\why\+\why) = \why$
  and therefore the map $0\+0: A\+B \to C\+D$ is the same as the map $0: A\+B \to C\+D$.
\end{proof}

%
% This really is immediate = move or amalgamate was the comment.

% \begin{lemma}\label{lem:disjointness_tensor_has_injections}
%   Given an inverse category \X with a restriction zero and a disjointness tensor, the map $\cpa$ is
%   natural in the left component and $\cpb$ is natural in the right, up to isomorphism. This means:
%   \[
%     \cpa (f\+g) = f \cpa \quad\text{and}\quad \cpb (f\+g) = g \cpb.
%   \]
% \end{lemma}
% \begin{proof}
%   For the left and right naturality, we see:
%   \[
%     \cpa (f\+g) = \inv{\upr} (1\+0) (f\+g) = \inv{\upr} (f \+ 0) =
%       f \inv{\upr} (1\+0) = f \cpa,
%   \]
%   and
%   \[
%     \cpb (f\+g) = \inv{\upl} (0\+1) (f\+g) = \inv{\upl} (0 \+ g) =
%       g \inv{\upl} (0\+1) = g \cpb.
%   \]
% \end{proof}

\begin{lemma}\label{lem:tensor_identities}
  Given an inverse category \X with restriction zero and disjointness tensor $\+$,
  $\scpa = \icpa$ and $\scpb = \icpb$ and the following identities hold:
  \begin{enumerate}[{(}i{)}]
    \item $\scp{i} \cp{i} = \rst{\scp{i}}$ and $\cp{i}\scp{i} = \rst{\cp{i}} = 1$;\label{lemitem:disjointness_tensor_identities_1}
    \item $\rst{\scpa} \scpb = 0$ and $\rst{\scpb} \scpa = 0$;\label{lemitem:disjointness_tensor_identities_2}
    \item $\cpb \scpa = 0$, $\cpb \rst{\scpa} = 0$,  $\cpa \scpb = 0$ and
      $\cpa \rst{\scpb} = 0$;\label{lemitem:disjointness_tensor_identities_3}
    \item the maps $\cpa$ and $\cpb$ are monic.\label{lemitem:disjointness_tensor_identities_4}
  \end{enumerate}
\end{lemma}
\begin{proof}
  For \ref{lemitem:disjointness_tensor_identities_1}, recalling that the restriction zero is its own
  partial inverse, we see that
  \[
    \icpa = \inv{(\inv{\upr}(1\+0))}  = \inv{(1\+0)}\upr = (1\+0)\upr = \scpa.
  \]
  Similarly,
  \[
    \icpb = \inv{(\inv{\upl}(0\+1))} = (0\+1)\upl = \scpb.
  \]
  Hence, we may calculate the restriction of $\cpa$,
  \begin{equation*}
    \cpa \scpa   = \inv{\upr}(1\+0) (1\+0)\upr
      = (\inv{\upr}(1\+0))\upr = 1 \inv{\upr}\upr= 1.
  \end{equation*}
  The calculation for $\scpb$ and $\cpb$ is analogous.
  For \ref{lemitem:disjointness_tensor_identities_2}, to show $\rst{\scpa} \scpb = 0$ and
  $\rst{\scpb} \scpa = 0$,
  \[
    \rst{\scpa} \scpb  = \rst{(1\+0)\upr}(0\+1)\upl = \rst{1\+0} (0 \+ 1)\upl = (1\+0) (0\+1)\upl= (0\+0)\upl  = 0,
  \]
  and
  \[
    \rst{\scpb} \scpa  = \rst{(0\+1)\upl}(1\+0)\upr = (0\+1) (1\+0)\upr= (0\+0)\upr= 0.
 \]
  We show \ref{lemitem:disjointness_tensor_identities_3}, $\cp{i} \scp{j} = 0$, $\cp{i}
  \rst{\scp{j}} = 0$ when $i\ne j$,
  \[
    \cpa \scpb = (\inv{\upr}(1\+0))(0\+1) \upl= \inv{\upr}(0\+0)\upl= 0
  \]
  and
  \[
    \cpb \scpa = (\inv{\upl}(0\+1))(1\+0) \upr= \inv{\upl}(0\+0)\upr= 0.
  \]
  As $\rst{\scpa} = 1\+0$ and $\rst{\scpb} = 0 \+1$, we see the other two identities hold as well.

  Finally, to prove \ref{lemitem:disjointness_tensor_identities_4}, we first show $\cpa$ is
  monic. Suppose $f\cpa = g\cpa$. Therefore we must have
  \[
    f = f (\cpa \icpa) = (f \cpa) \icpa = (g \cpa) \icpa = g (\cpa \icpa) = g.
  \]
  The proof that $\cpb$ is monic follows via a similar argument.
\end{proof}

As we have shown that $\scp{i} = \icp{i}$, we will prefer the explicit notation of $\icp{i}$ for the
remainder of this thesis.

\begin{corollary}\label{cor:split_plus_tensor}
  In an inverse category \X with a restriction zero and disjointness tensor, the following
  identities hold:
  \begin{multicols}{2}
    \begin{enumerate}[{(}i{)}]
      \item $\cpa (f \+g) \icpa = f$;
      \item $\cpa (f\+ g) \icpb = 0$;
      \item $\cpb (f \+g) \icpa = 0$;
      \item $\cpb (f\+ g) \icpb = g$.
    \end{enumerate}
  \end{multicols}
  Additionally, if $t$ is a map such that for $i \in \{1,2\}$,
  \[
    \cp{i} t \icp{j} =
    \begin{cases}
      t_i& : \quad i \neq j\\
      0 & : \quad i = j,
    \end{cases}
  \]
  then $t = t_1 \+ t_2$.
\end{corollary}
\begin{proof}
  The calculations for $f\+g$ follow from
  Lemma~\ref{lem:tensor_identities}. For example, $\cpa (f\+g) \icpa = f \cpa \icpa = f$.

  For the second claim, note that we have $\cpa (t \icpa) = t_1 = \cpa ( t_1 \+ t_2)\icpa$ and
  $\cpb  (t \icpa) = 0 = \cpb ( t_1 \+ t_2)\icpa$, hence $t \icpa = ( t_1 \+ t_2)\icpa$. Similarly,
  we see $t\icpb = ( t_1 \+ t_2)\icpb$ and therefore $t = t_1 \+ t_2$.
\end{proof}

\begin{definition}\label{def:up_triangle_and_down_triangle}
  In an inverse category \X with a restriction zero and disjointness tensor, we define a partial
  pairing and a partial copairing operation on arrows in  \X. First, for arrows
  $f:A \to B$ and $g:A \to C$, we define $f\tjdown g$ as
  being the map that makes Diagram~\ref{eq:tjdowndefinition} below commute, when it exists.
  \begin{equation}
    \xymatrix@C+0pt @R+20pt{
      &&A \ar[dll]_{f} \ar[drr]^{g} \ar@{.>}[d]|-{f\tjdown g}\\
      B&&B\+C \ar[ll]^{\icpa} \ar[rr]_{\icpb} && C
      }\label{eq:tjdowndefinition}
  \end{equation}
  Then for $h:B \to A$, $k: C \to A$, $h\tjup k$ is that map that makes
  Diagram~\ref{eq:tjupdefinition} commute, if it exists.
  \begin{equation}
    \xymatrix@C+0pt @R+20pt{
      B\ar[rr]^{\cpa} \ar[drr]_{h} &&B\+C   \ar@{.>}[d]|-{h\tjup k}
        && C \ar[ll]_{\cpb} \ar[dll]^{k}\\
      &&A
      }\label{eq:tjupdefinition}
  \end{equation}
  Due to $\icpa$ and $\icpb$ being jointly monic, $f\tjdown g$ is unique when it exists.
  Similarly, as $\cpa$ and $\cpb$ are jointly epic, $f\tjup g$ is unique when it exists.
\end{definition}
\begin{example}[\pinj]\label{ex:pinj_has_triangle_maps_when_restriction_is_disjoint}
  Continuing from Example~\ref{ex:pinj_has_a_disjoint_join}, we see that $f\tjdown g$ can only exist
  when $\rst{f} \meet \rst{g} = 0$, as it must be a set function, i.e., $f\tjdown g$ of some element
  $a$ must be either $(b,1)$ when  $f(a) = b\in B$ or $(c,2)$ when $g(a) = c\in C$.

  Similarly, $h\tjup k$ can only exist when $\rg{h} \meet \rg{k} = 0$.
\end{example}

The partial pairing and copairing will provide our mechanism for defining disjointness and
eventually the disjoint join of maps. The existence of the pairing map $f\tjdown g$ allows us to
ensure the restrictions of $f$ and $g$ are disjoint, while the copairing map $f\tjup d$ exists only
when the ranges of $f$ and $g$ are disjoint.

To arrive at the disjointness we first give the following lemma detailing properties of the two
operations $\tjdown$ and $\tjup$:

\begin{lemma}\label{lem:properties_of_tjdown_and_tjup}
  Given \X is an inverse category with a restriction zero and a disjointness tensor $\+$ then the
  following relations hold for $\tjdown$ and $\tjup$:
   \begin{enumerate}[{(}i{)}]
    \item If $f \tjdown g$ exists, then $g \tjdown f$ exists. If $f \tjup g$ exists, then
    $g \tjup f$ exists. \label{lemitem:l_r_commute}
    \item $f \tjdown 0$ and $f \tjup 0$ always exist. \label{lemitem:l_r_zero_exists}
    \item When $f\tjdown g$ exists, $\rst{f}(f\tjdown g) = f \tjdown 0$, $\rst{f}g = 0$,
      $\rst{g}(f \tjdown g) = 0 \tjdown g$ and $\rst{g}f = 0$.
      \label{lemitem:l_r_rst_orthogonal_to_zero}
    \item Dually to the previous item, when $f\tjup g$ exists,
      $(f\tjup g)\rg{f} = f \tjup 0$, $g\rg{f} = 0$, $(f \tjup g)\rg{g} = 0 \tjup g$ and
      $f\rg{g} = 0$.\label{lemitem:l_r_rg_orthogonal_to_zero}
    \item When $f\tjdown g$ exists, $f\tjdown g (h\+k) = f h \tjdown g k$.
      \label{lemitem:l_r_l_pull_right}
    \item Dually, when $f \tjup g$ exists, $(h \+ k) f \tjup g = h f \tjup k g$.
      \label{lemitem:l_r_r_pull_left}
    \item When $f \tjdown g$ exists, then $h(f \tjdown g) = h f \tjdown h g$ and when $f \tjup g$
      exists, $(f \tjup g)h = f h \tjup g h$.
      \label{lemitem:l_r_l_universal_r_stable}
    \item If $\rst{f} \tjdown \rst{g}$ exists, then $\rst{f} \tjup \rst{g}$ exists and is
      the partial inverse of $\rst{f} \tjdown \rst{g}$.
      \label{lemitem:l_r_rst_inverses}
    \item If $f\tjdown g$ exists and $f' \le f$, $g' \le g$, then $f' \tjdown g'$ exists.
      \label{lemitem:l_r_less_than}
    \item When $f\tjup g$ exists, $(f\tjup g) \inv{(f\tjup g)} = \rst{f}\+\rst{g}$.
      \label{lemitem:l_r_rst_is_natural}
    \item Given $f\tjdown g$ and $h\tjdown k$ exist, then
      $(f\+h)\tjdown(g\+k) = (f\tjdown g)\+ (h\tjdown k)$. Dually, the existence of $f\tjup g$
      and $h\tjup k$ implies $(f\+h)\tjup(g\+k) = (f\tjup g)\+ (h\tjup k)$.
      \label{lemitem:l_r_preserve_tensor}
  \end{enumerate}
\end{lemma}
\begin{proof}
  \prepprooflist
  \begin{enumerate}[{(}i{)}]
    \item $g \tjdown f = (f \tjdown g)\comp $ and $g \tjup f = \comp(f\tjup g)$.
    \item Consider $f \cpa$. Then $f \cpa \icpa = f$ and
      $f \cpa \icpb = f 0 = 0$. Hence, $f \cpa = f \tjdown 0$.

      Consider $\icpa f$. Then $\cpa \icpa f = f$ and
      $\cpb \icpa f = 0 f = 0$ and therefore $\icpa f = (f\tjup 0)$.
    \item Using Lemma~\ref{lem:tensor_identities}
      \[
        \rst{f}g = \rst{(f\tjdown g)\icpa} (f\tjdown g)\icpb =
          (f\tjdown g) \rst{\icpa} \icpb = 0.
      \]
      Similarly, $\rst{g}f = f\tjdown g \rst{\icpb}\icpa = 0$.

      Recall that $\icpa$ and $\icpb$ are jointly monic. We have
      $\rst{f}(f \tjdown g) \icpa = \rst{f} f = f = (f\tjdown 0) \icpa$ and
      $\rst{f}(f \tjdown g) \icpb = \rst{f}g = 0= (f\tjdown 0) \icpb$.
      Therefore, $\rst{f}(f \tjdown g) = f \tjdown 0$. Similarly,
      $\rst{g}(f \tjdown g) = 0 \tjdown g$.
    \item Using Lemma~\ref{lem:tensor_identities}
      \begin{multline*}
        g\rg{f} = \cpb(f\tjup g)\wrg{(\cpa(f\tjup g))}  =
          \cpb(f\tjup g) \rst{\inv{(f\tjup g)}\icpa} =\\
          \cpb(f\tjup g) \rst{\inv{(f\tjup g)}\rst{\icpa}} =
          \rst{\cpb \rst{(f\tjup g)} \rst{\icpa}}\cpb(f\tjup g) = \\
          \rst{\cpb \rst{\icpa}\rst{(f\tjup g)} }\cpb(f\tjup g) =
          \rst{0}\cpb(f\tjup g) = 0
      \end{multline*}
      Similarly, $f\rg{g} =  0$.

      Recall that $\cpa$ and $\cpb$ are jointly epic. We have
      $\cpa(f \tjup g)\rg{f} = f\rg{f} = f = \cpa(f\tjup 0) $ and
      $\cpb(f \tjup g)\rg{f} = g\rg{f} = 0= \cpb(f\tjup 0)$.
      Therefore, $(f \tjup g)\rg{f} = f \tjup 0$. Similarly,
      $(f \tjup g) \rg{g} = 0 \tjup g$.
    \item Calculating, we have
      \[
        f \tjdown g (h \+ k) \icpa = f\tjdown g \icpa h = f h
      \]
      and
      \[
        f \tjdown g (h \+ k) \icpb = f\tjdown g \icpb k = g k,
      \]
      which means that
      $f \tjdown g (h \+ k) = f h \tjdown g k$ by the joint monic property of $\icpa$, $\icpb$.
    \item The proof for this is dual to \ref{lemitem:l_r_l_pull_right}, and depends on the joint
      epic property of $\cpa$ and $\cpb$.
    \item We are given $f\tjdown g$ exists, therefore $f = (f\tjdown g)\icpa$ and
      $g = (f\tjdown g)\icpb$. But this means $hf = h(f\tjdown g)\icpa$ and
      $h g = h(f\tjdown g)\icpb$, from which we may conclude $hf \tjdown h g = h(f\tjdown g)$ by
      the fact that $\icpa$ and $\icpb$ are jointly monic. The proof of
      $(f\tjup g)h = f h \tjup g h$ is similar.
    \item We are given $\rst{f} = \rst{f}\tjdown\rst{g} \icpa$. Therefore,
      \[
        \rst{f} = \inv{\rst{f}} = \inv{\icpa}\inv{(\rst{f}\tjdown\rst{g})}
          = \cpa \inv{(\rst{f}\tjdown\rst{g})}.
      \]
      Similarly, $\rst{g} = \cpb \inv{(\rst{f}\tjdown\rst{g})}$. But this means
      $\inv{(\rst{f}\tjdown\rst{g})} = \rst{f}\tjup\rst{g}$.
    \item Note that from \ref{lemitem:l_r_l_pull_right}, we know that
      $f \tjdown g  = \rst{f} \tjdown \rst{g} (f\+g)$. We are given $f' \le f$ and $g' \le g$. This
      gives us $\rst{f'}f =f'$, $\rst{g'}g =g'$, $\rst{f'}\,\rst{f} =\rst{f'}$ and
      $\rst{g'}\,\rst{g} =\rst{g'}$. Consider the map
      $\rst{f} \tjdown \rst{g} (\rst{f'}\+\rst{g'})(f\+g)$. Calculating, we see
      \begin{align*}
        \rst{f} \tjdown \rst{g} (\rst{f'}\+\rst{g'})(f\+g)
          &= \rst{f} \tjdown \rst{g} (\rst{f'}\+\rst{g'})(\rst{f'}\+\rst{g'})(f\+g) \\
          & = \rst{f} \tjdown \rst{g} (\rst{f'}\+\rst{g'})(f'\+g') \\
          & = \rst{f}\,\rst{f'} \tjdown \rst{g} \rst{g'} (f'\+g') \\
          & = \rst{f'}\,\rst{f} \tjdown \rst{g'} \rst{g} (f'\+g') \\
          & = \rst{f'} \tjdown \rst{g'}  (f'\+g') \\
          & = f'\tjdown g'.
      \end{align*}
    \item From our diagram for $\tjup$, we know:
      \begin{align*}
        \inv{f} &= \inv{(f\tjup g)} \icpa\text{ and}\\
        \inv{g} &= \inv{(f\tjup g)} \icpb.
      \end{align*}
      As well, we know that $\cpa(f\tjup g) = f$ and
      $\cpa(f\tjup g) = g$.
      Therefore, we have:
      \[
         \cpa(f\tjup g)\inv{(f\tjup g)} \icpa = \rst{f} \text{ and  }
          \cpb(f\tjup g)\inv{(f\tjup g)} \icpb = \rst{g}.
      \]
      As $f\tperp g$, we know that $f\inv{g} = f \rg{g} \inv{g} = 0 \inv{g} = 0$ and therefore,

      \[
         \cpa(f\tjup g)\inv{(f\tjup g)} \icpb = 0 \text{ and  }
          \cpb(f\tjup g)\inv{(f\tjup g)} \icpa = 0.
      \]
      By Corollary~\ref{cor:split_plus_tensor} this means
      $(f\tjup g)\inv{(f\tjup g)} = \rst{f} \+ \rst{g}$.
    \item As $(f\tjdown g)\+ (h\tjdown k)\icpa = (f\tjdown g)$ and
      $(f\tjdown g)\+ (h\tjdown k) \icpb = (h\tjdown k)$, we see that
      $(f\tjdown g)\+ (h\tjdown k)$ satisfies the diagram for $(f\+h)\tjdown(g\+k)$. Dually, as
      $\cpa (f\tjup g)\+ (h\tjup k) = (f\tjup g)$ and
      $\cpb(f\tjup g)\+ (h\tjup k) = (h\tjup k)$, $(f\tjup g)\+ (h\tjup k)$ satisfies
      the diagram for $(f\+h)\tjup(g\+k)$.
  \end{enumerate}
\end{proof}

% \begin{definition}\label{def:general_triangle}
%   In an inverse category \X with a restriction zero and disjointness tensor, we define two partial
%   operations on pairs of arrows in \X to another arrow in \X. First, for arrows
%   $f:A \to B$ and $g:A \to C$, suppose we have arrows $f':A  \to B\+X$ and $g': A \to C\+X$
%   such that $f'\icpa =f$, $g'\icpa = g$ and $f'\icpb = g' \icpb$. Then, we say $f\gtjdown g$ if
%   regardless of the choice of $f',\ g'$ and $X$, there exists a map $\alpha$ that makes
%   diagram \ref{eq:gtjdowndefinition} below commute.
%   \begin{equation}
%     \xymatrix@C+0pt @R+20pt{
%       B\+X&&B\+C\+X \ar[ll]_{\icp{0,2}} \ar[rr]^{\icp{1,2}} && C\+X\\
%       &&A \ar[ull]^{g'} \ar[urr]_{f'} \ar@{.>}[u]|-{\alpha}
%       }\label{eq:gtjdowndefinition}
%   \end{equation}
%
%   Similarly, for $h:B \to A$, $k: C \to A$, and $h': A\+X \to C$, $k': B\+X \to C$, with $h = \cpa
%   h'$ and $k = \cpb k'$, then $h\gtjup k$ if regardless of the choice of $h',\ k'$ and $X$, there
%   exists a map $\beta$ that makes diagram \ref{eq:gtjupdefinition} commute.
%   \begin{equation}
%     \xymatrix@C+0pt @R+20pt{
%       &&A\\
%       B\+X\ar[rr]_{\amalg_{0,2}} \ar[urr]^{h'} &&B\+C \+X  \ar@{.>}[u]|-{\beta}
%         && C\+X \ar[ll]^{\amalg_{1,2}} \ar[ull]_{k'}
%       }\label{eq:gtjupdefinition}
%   \end{equation}
%   In the above diagrams, $\icp{0,2} = (1\+0\+1)(\upr\+1)$, $\amalg_{0,2} =
%   (\inv{\upr}\+1)(1\+0\+1)$, $\icp{1,2}=(0\+1\+1)(\upl\+1)$ and
%   $\cp{1,2} = (\inv{\upl}\+1)(0\+1\+1)$.
% \end{definition}
% \begin{lemma}\label{lem:general_triangle_implies_triangle}
%   In an inverse category $\X$ with a restriction zero and a disjointness tensor, with maps
%   $f:A\to B$, $g: A \to C$, $h: B \to A$ and $k: C \to A$. Then $f\gtjdown g$
%   (respectively $h \gtjup k$) implies that $f\tjdown g$ (respectively $h \tjup k$) exists.
% \end{lemma}
% \begin{proof}
%   Suppose $f \gtjdown g$. Then set $f' = f \inv{\upr}$, $g' = g \inv{\upr}$ and $X = 0$. Then
%   $\alpha: A \to B\+C\+0$ exists and we can see that setting $f\tjdown g = \alpha \upr$ makes the
%   first diagram of Definition~\ref{def:up_triangle_and_down_triangle} commute. In detail,
%   \[
%     \xymatrix@C+0pt @R+10pt{
%       B&&B\+C \ar[ll]_{\icpa} \ar[rr]^{\icpb} && C\\
%       &B\+0\ar[ul]^{\upl}&B\+C\+0 \ar[l]_{\icp{0,2}} \ar[r]^{\icp{1,2}} \ar[u]|-{\upl}
%         & C\+0\ar[ur]_{\upl}\\
%       &&A \ar[ul]^{g \inv{\upl}} \ar[ur]_{f\inv{\upl}} \ar@{.>}[u]|-{\alpha}
%       }
%   \]
%   where the lower triangle commutes by assumption and the upper quadrilaterals commute due to the
%   coherence diagrams of the tensor.
%
%   Similarly, setting $h' = \inv{\upr}h$ and $k' = \inv{\upr}k$ gives us $\beta:B\+C\+0 \to A\+0$ and
%   we set $f\tjup = \beta \upr.$
% \end{proof}

We are now set up to prove that we can create a disjointness relation based on the existence of our
pairing and copairing maps:
\begin{lemma}\label{lem:tensor_disjointness_is_disjointness}
  Define $f \tperp g$ ($f$ is tensor disjoint to $g$) when $f, g: A\to B$ and both $f\tjdown g$ and
  $f\tjup g$ exist.  If \X is an inverse category with a restriction zero and a disjointness tensor
  $\+$ then the relation $\tperp$ is a disjointness relation.
\end{lemma}
\begin{proof}
  We need to show that $\tperp$ satisfies the disjointness axioms. We will use \axiom{Dis}{6'} in
  place of \axiom{Dis}{6} and \axiom{Dis}{7} as discussed in
  Lemma~\ref{lem:disjointness_equivalent_axioms}.
  \setlist[itemize,1]{leftmargin=1.5cm}
  \begin{itemize}
    \item[\axiom{Dis}{1}] We must show $f \tperp 0$. This follows immediately from
      Lemma~\ref{lem:properties_of_tjdown_and_tjup}, item \ref{lemitem:l_r_zero_exists}.
    \item[\axiom{Dis}{2}] Show $f \tperp g$ implies $\rst{f}g = 0$. This is a direct consequence of
      Lemma~\ref{lem:properties_of_tjdown_and_tjup}, item
      \ref{lemitem:l_r_rst_orthogonal_to_zero}.
    \item[\axiom{Dis}{3}] We require $f\tperp g$, $f' \le f$, $g' \le g$ implies $f' \tperp g'$.
      From Lemma~\ref{lem:properties_of_tjdown_and_tjup}, item \ref{lemitem:l_r_less_than}, we
      immediately have $f' \tjdown g'$ exists. Using a similar argument to the proof of this item,
      we also have $f' \tjup g'$ exists and hence $f' \tperp g'$.
    \item[\axiom{Dis}{4}] Commutativity of $\tperp$ follows from the symmetry of the two required
    diagrams, see Lemma~\ref{lem:properties_of_tjdown_and_tjup}, item
      \ref{lemitem:l_r_commute}.
    \item[\axiom{Dis}{5}] Show that if $f\tperp g$ then $h f \tperp h g$ for any map $h$.
      By  Lemma~\ref{lem:properties_of_tjdown_and_tjup}, item
      \ref{lemitem:l_r_l_universal_r_stable}, we know that $h f \tjdown h g$ exists.
      By item \ref{lemitem:l_r_r_pull_left},  $(h f)\tjup (h g) = (h \+ h) (f\tjup g)$
      and therefore $h f \tperp h g$.
    \item[\axiom{Dis}{6'}] We need to show $f\tperp g$ if and only if $\rst{f} \tperp \rst{g}$ and
      $\rg{f}\tperp \rg{g}$. This follows directly from
      Lemma~\ref{lem:properties_of_tjdown_and_tjup}, items
      \ref{lemitem:l_r_l_pull_right} and       \ref{lemitem:l_r_r_pull_left}, which give us
      $f\tjdown g = \rst{f}\tjdown \rst{g} (f\+g)$ and
      $f\tjup g = (f \+ g)\rg{f}\tjup \rg{g}$, where the equalities hold if either side of
      the equation exists.
  \end{itemize}
\end{proof}

\begin{example}[$\tperp$ in \pinj]\label{ex:tensor_perp_in_pinj}
  Referring to Example~\ref{ex:pinj_has_triangle_maps_when_restriction_is_disjoint}, we noted
  $f\tjdown g$ exists when $\rst{f} \meet \rst{g} = 0$ and that $f \tjup g$ exists when $\rg{f} \meet
  \rg{g} = 0$. But this agrees with our initial definition of disjointness $(\perp)$ in \pinj from
  Example~\ref{ex:pinj_has_a_disjointness_relation} and hence we have that $\tperp$ is the same
  relation as $\perp$ in \pinj.
\end{example}

The operations $\tjdown$ and $\tjup$ are sufficient to define a disjointness relation
on an inverse category. However, when we wish to extend this to a disjoint join, we need to prove
$\axiom{DJ}{4}$, That is, we need to show that $\tperp [f,g,h]$ implies $f \tperp (g \tjoin h)$.

Therefore, we add one more assumption regarding our tensor in order to define disjointness.

\begin{definition}\label{def:disjoint_sum_tensor}
  Let \X be an inverse category with a disjointness tensor $\+$ and a restriction zero. Consider the
  commutative diagrams \ref{dia:tensor_complete_left} and \ref{dia:tensor_complete_right}.
  \begin{align}
    &\xymatrix@C+5pt@R+10pt{
      A \ar@/_/[ddr]_f \ar@/^/[drr]^g \ar@{.>}[dr]|-{\alpha} \\
        & X\+Y\+Z \ar[d]^{\icp{1,2}} \ar[r]_{\icp{1,3}} & X\+Z \ar[d]^{\icpa} \\
        & X\+Y \ar[r]_{\icpa} & X
    } \label{dia:tensor_complete_left}\\
    &\xymatrix@C+5pt@R+10pt{
        &&A \\
         X\+Y \ar[r]_{\cp{1,2}} \ar@/^/[urr]^h &X\+Y\+Z \ar@{.>}[ur]|-{\beta} \\
         X \ar[u]^{\cpa} \ar[r]_{\cpa} & X\+Z \ar[u]^{\cp{1,3}} \ar@/_/[uur]_k
    } \label{dia:tensor_complete_right}
  \end{align}
  Then
  $\+$ is a \emph{disjoint sum tensor} when the following two conditions hold:
  \begin{itemize}
    \item $\alpha$ exists if and only if  $f \icpb \tjdown g \icpb$ exists;
    \item $\beta$ exists if and only if $\cpb h \tjup \cpb k$ exists.
  \end{itemize}

\end{definition}

\begin{example}[In \pinj, $\+$ is a disjoint sum tensor]\label{ex:pinj_has_disjoint_sum_tensor}
  In \pinj, Diagram~\ref{dia:tensor_complete_left} means that $f$ and $g$ must agree on those
  elements of $A$ that map to $(x,1)$ in either $X\+Y$ or $X\+Z$. The statement that
  $f \icpb  \tjdown g \icpb$ exists means that if $f(a) = (y,2)$, then $g(a)$ must be undefined and
  vice versa. In such a case $\alpha$ exists and is defined as:
  \[
     \alpha(a) = \begin{cases}
       (x,1) & f(a) = (x,1)\in X\+Y \text{ and }g(a) = (x,1) in X\+Z\\
       (y,2) & f(a) = (y,2)\in X\+Y \text{ and }g(a) \undef\\
       (z,3) & g(a) = (z,2)\in X\+Z \text{ and }f(a) \undef.
       \end{cases}
 \]
 Conversely, if $\alpha$ exists, $\alpha(a)$ must be one of $(x,1), (y,2)$ or $(z,3)$. As
 $f\icpb = \alpha\icp{1,2}\icpb$ and $g\icpb = \alpha\icp{1,3}\icpb$, we see this immediately
 requires that $\rst{f\icpb}\meet\rst{\icpb} =0$ and therefore $f \icpb  \tjdown g \icpb$ exists.

 The reasoning for Diagram~\ref{dia:tensor_complete_left} is similar.
\end{example}
\begin{lemma}\label{lem:complete_disjointness_means_multiple_disjoints}
  Let \X be an inverse category with a disjoint sum tensor as in
  Definition~\ref{def:disjoint_sum_tensor} and we are given $f,g,h:A\to B$ with
  $\tperp[f,g,h]$. Then both $f \tjdown (g \tjdown h)$ and $f\tjup(g\tjup h)$ exist.
\end{lemma}
\begin{proof}
  As all the maps are disjoint, we know the maps $\tjdown$ and $\tjup$ exist for each pair.
  Consider the diagram
  \[
    \xymatrix@C+5pt@R+10pt{
      A \ar@/_/[ddr]_{g\tjdown h} \ar@/^/[drr]^{g\tjdown f} \ar@{.>}[dr]|-{\alpha} \ar[rr]^{f}
        & & B \\
        & B\+B\+B \ar[d]^{\icp{0,1}} \ar[r]_{\icp{0,2}} & B\+B \ar[d]^{\icpa} \ar[u]_{\icpb} \\
        & B\+B \ar[r]_{\icpa} & B
    }
  \]
  where we claim $\alpha = (g\tjdown h)\tjdown f$.

  The lower part of the diagram commutes as it fulfills the conditions of
  Definition~\ref{def:disjoint_sum_tensor}. The upper rightmost triangle of the diagram commutes by
  the definition of $g\tjdown f$. Noting that $\icp{0,1}:B\+B\+B \to B\+B$ is the same map as
  $\icpa:(B\+B)\+B\to (B\+B)$ and $\icp{0,2} \icpb:B\+B\+B \to B\+B \to B$ is the same map as
  $\icpb:(B\+B)\+B\to B$, we see $\alpha$ does make the $\tjdown$ diagram for $g\tjdown h$ and $f$
  commute. Therefore by Lemma~\ref{lem:properties_of_tjdown_and_tjup}, $f\tjdown(g\tjdown h)$
  exists and is equal to $\alpha \com{\+\,\{01,2\}}$.

  A dual diagram and corresponding reasoning shows $f\tjup(g\tjup h)$ exists.
\end{proof}

\begin{lemma}\label{lem:tjdown_and_tjup_associate}
  In an inverse category $\X$ with a disjoint sum tensor, when $\tperp [f,g,h]$, then:
  \begin{enumerate}[{(}i{)}]
    \item   $f \tjdown (g \tjdown h) = ((f \tjdown g) \tjdown h) \assocp$ and both exist,
    \item   $f \tjup (g \tjup h) = ((f \tjup g) \tjup h) \assocp$ and both exist.
  \end{enumerate}
\end{lemma}
\begin{proof}
  Consider the diagram
  \begin{equation}
    \xymatrix@C+5pt@R+10pt{
      A \ar@/_/[ddr]_{f\tjdown g} \ar@/^/[drr]^{f\tjdown h} \ar@{.>}[dr]|-{\alpha} \ar[rr]^{h}
        & & B \\
        & B\+B\+B \ar[d]^{\icp{0,1}} \ar[r]_{\icp{0,2}} & B\+B \ar[d]^{\icpa} \ar[u]_{\icpb} \\
        & B\+B \ar[r]_{\icpa} & B
    }\label{dia:alpha_plus_h}
  \end{equation}
  which gives us $\alpha = (f \tjdown g) \tjdown h: A \to (B\+B)\+B$ and
  $\alpha \assocp :A \to B\+(B\+B)$. Next consider the diagram
  \begin{equation}
    \xymatrix@C+5pt@R+10pt{
      A \ar@/_/[ddr]_{g\tjdown h} \ar@/^/[drr]^{g\tjdown f} \ar@{.>}[dr]|-{\gamma} \ar[rr]^{f}
        & & B \\
        & B\+B\+B \ar[d]^{\icp{0,1}} \ar[r]_{\icp{0,2}} & B\+B \ar[d]^{\icpa} \ar[u]_{\icpb} \\
        & B\+B \ar[r]_{\icpa} & B
    }\label{dia:gamma_plus_f}
  \end{equation}
  which gives us $\gamma \comp = f \tjdown (g\tjdown h): A \to B\+(B\+B)$.

  Note from Diagrams~\ref{dia:alpha_plus_h} and \ref{dia:gamma_plus_f} we have
  \begin{align*}
    \gamma \comp  \icp{0}         =\, &f  = \alpha \assocp\icpa\\
    \gamma \comp  \icp{1} \icp{0} =\, &g  = \alpha \assocp\icpb\icpa\\
    \gamma \comp  \icp{1} \icpb   =\, & h = \alpha \assocp\icpb\icpb.
  \end{align*}
  By the assumption that $\icpa, \icpb$ are jointly monic, we have
  $\alpha = \gamma \comp  \assocp$. Therefore $f \tjdown (g \tjdown h) = (f \tjdown g) \tjdown h$,
  up to the associativity isomorphism.
\end{proof}

We may now state the main result of this chapter:
\begin{proposition}\label{prop:disjointness_tensor_gives_disjoint_join}
  An inverse category with a restriction zero and a disjoint sum tensor has disjoint joins.
\end{proposition}

Given two  maps $f,g$ with $f\tperp g$, we claim the map $f \tjoin g = \rst{f}\tjdown\rst{g} (f\+g)
\rg{f}\tjup\rg{g}$  is a disjoint join. We will prove this after giving an example and a lemma
detailing the properties of $\tjoin$.

For reference, the map $f \tjoin g$ may be visualized as follows:
\[
  \xymatrix@C+20pt @R+10pt{
    &A
      & B  \ar[dr]^{\rg{f}} \ar[d]_{\cpa}\\
    A \ar[ur]^{\rst{f}} \ar@{.>}[r]^{\rst{f}\tjdown \rst{g}} \ar[dr]_{\rst{g}}
      & A\+A \ar[u]_{\icpa} \ar[d]^{\icpb}
        \ar[r]^{f\+g}
       &B\+B \ar@{.>}[r]^{\rg{f}\tjup\rg{g}} & B.\\
    & A & B\ar[ur]_{\rg{g}} \ar[u]^{\cpb}
  }
\]

Using Lemma~\ref{lem:properties_of_tjdown_and_tjup}, we may rewrite this in a variety of
equivalent ways:
\begin{align*}
  f \tjoin g &= \rst{f}\tjdown\rst{g} (f\+g) \rg{f}\tjup\rg{g} \\
  &= f \tjdown g \rg{f}\tjup\rg{g}\\
  & = \rst{f}\tjdown\rst{g}  f\tjup g\\
  & = f \tjdown g (\inv{f}\+\inv{g})f\tjup g.
\end{align*}

In particular, note that $\rst{f}\tjoin \rst{g} = (\rst{f} \tjdown \rst{g})
(\rst{f}\tjup\rst{g})$ as $\rg{\rst{g}} = \rst{g}$.

\begin{example}[Tensor join in \pinj]\label{ex:tensor_join_in_pinj}
  We will use the third equality of those above for $f\tjoin g$, i.e., $\rst{f}\tjdown\rst{g}
  f\tjup g$.

  We have:
  \begin{equation}
    \rst{f}\tjdown\rst{g}(a) =
    \begin{cases}
      (a,1) & \rst{f}(a) = a, \rst{g}\undef\\
      (a,2) & \rst{g}(a) = a, \rst{f}\undef
    \end{cases}\label{eq:definition_of_rf_tjdown_rg_in_pinj}
  \end{equation}
  and
  \begin{equation}
    f\tjup g((a,n)) =
    \begin{cases}
      f(a)  & n = 1\\
      g(a)  & n=2.
    \end{cases}\label{eq:definition_of_f_tjup_g_in_pinj}
  \end{equation}
  Combining Equation~\ref{eq:definition_of_rf_tjdown_rg_in_pinj} with
  Equation~\ref{eq:definition_of_f_tjup_g_in_pinj} then gives us the same definition as that of
  $\djoin$ as given in Example~\ref{ex:pinj_has_a_disjoint_join}.
\end{example}
\begin{lemma}\label{lem:tensor_disjoint_join_properties}
  Let \X be an inverse category with a disjointness tensor and restriction zero. Let \X have the
  maps $f,g: A \to B$ with $f\tperp g$. Then $\tjoin$ has the following properties.
  \begin{enumerate}[{(}i{)}]
    \item For all maps $h:A \to B$, $\rst{f}h \tjoin \rst{g}h = (\rst{f}\tjoin \rst{g})h$.
      \label{lemitem:tdj_rst_universal}
    \item $\rst{f}\tjoin \rst{g} = \rst{f\tjoin g}$. \label{lemitem:tdj_rst_is_rst}
  \end{enumerate}
\end{lemma}
\begin{proof}
  \prepprooflist
  \begin{enumerate}[{(}i{)}]
    \item By Lemma~\ref{lem:disjointness_various}, item \ref{lemitem:djv_universal}, we know that
    $\rst{f}h \tperp \rst{g} h$, hence we can form $\rst{f}h \tjoin \rst{g}h$.
    Also, noting that
      \[
        h \wrg{\rst{f} h} = h \rst{\inv{h}\rst{f}} = \rst{h \inv{h} \rst{f}} h
          = \rst{\rst{h}\rst{f}} h = \rst{f} \rst{h}h = \rst{f} h,
      \]
      we may then calculate from the left hand side as follows:
      \begin{align*}
        \rst{f}h \tjoin \rst{g}h
          & = (\rst{f}h \tjdown \rst{g} h) (\wrg{\rst{f}h}\tjup\wrg{\rst{g} h})\\
          & = (\rst{f} \tjdown \rst{g} ) (h\wrg{\rst{f}h}\tjup h\wrg{\rst{g} h})\\
          & = (\rst{f} \tjdown \rst{g} ) (\rst{f}h\tjup \rst{g} h)\\
          & = (\rst{f} \tjdown \rst{g} ) (\rst{f}\tjup \rst{g} )h\\
          & = (\rst{f} \tjoin \rst{g} ) h.
      \end{align*}
    \item
    Using Lemma~\ref{lem:properties_of_tjdown_and_tjup}, item~\ref{lemitem:l_r_rst_is_natural},
    we can compute:
    \begin{align*}
      \rst{f\tjoin g} & = f \tjoin g \inv{(f\tjoin g)} \\
      & = \left((\rst{f} \tjdown \rst{g})(f\tjup g)\right)
             \left(\inv{(f\tjdown g)} \inv{(\rst{f}\tjdown \rst{g})}\right)\\
      & = \rst{f}\tjdown\rst{g}(f\tjdown g) \inv{(f\tjdown g)} \rst{f}\tjup \rst{g}\\
      & = \rst{f}\tjdown\rst{g}(\rst{f}\+ \rst{g})\rst{f}\tjup \rst{g}\\
      & = \rst{f}\tjdown\rst{g}\rst{f}\tjup \rst{g}\\
      & = \rst{f}\tjoin\rst{g}.
    \end{align*}
  \end{enumerate}
\end{proof}


We may now complete the proof of Proposition~\ref{prop:disjointness_tensor_gives_disjoint_join}:
\begin{proof}
  \setlist[itemize,1]{leftmargin=1.5cm}
  \begin{itemize}
    \item [\axiom{DJ}{1}] We must show $f, g \le f \tjoin g$.
      \begin{align*}
        \rst{f}\,(\rst{f}\tjdown\rst{g})  f\tjup g
          &= (\rst{f}\tjdown\rst{g}) \icpa (\rst{f}\tjdown\rst{g} ) f\tjup g \\
        &=\rst{(\rst{f}\tjdown\rst{g}) \icpa} (\rst{f}\tjdown\rst{g}) f\tjup g  \\
        &=(\rst{f}\tjdown\rst{g}) \rst{\icpa} f\tjup g  \\
        &=(\rst{f}\tjdown\rst{g}) \icpa \cpa f\tjup g \\
        &=((\rst{f}\tjdown\rst{g}) \icpa) (\cpa (f \tjup g)) \\
        &= \rst{f} f \\
        &=f.
      \end{align*}
      Thus, we see $f \le f \tjoin g$. Showing $g \le f \tjoin g$ proceeds in the same manner.
    \item [\axiom{DJ}{2}] We must show that $f \le h,\ g\le h$ and $f\tperp g$ implies
      $f \tjoin g \le h$.
      \begin{align*}
        \rst{f \tjoin g} \,h & = \rst{\rst{f} h \tjoin \rst{g} h} \,h\\
        & = \rst{(\rst{f} \tjoin \rst{g})h} \,h\\
        & = \rst{\rst{(\rst{f} \tjoin \rst{g})}h} \,h\\
        & = \rst{\rst{(\rst{f} \tjoin \rst{g})}h}\,(\rst{f} \tjoin \rst{g})h\\
        & = \rst{(\rst{f} \tjoin \rst{g})h}\,(\rst{f} \tjoin \rst{g})h\\
        & = (\rst{f}\tjoin \rst{g})h\\
        & = (\rst{f}h\tjoin \rst{g}h)\\
        & = (f\tjoin g).
      \end{align*}
    \item [\axiom{DJ}{3}] We must show stability of $\tjoin$, i.e., that
      $h(f\tjoin g) = h f \tjoin h g$.

      \begin{align*}
        h (f\tjoin g) &= h ((\rst{f}\tjdown\rst{g})  (f \tjup g))\\
        &= (h\rst{f}\tjdown h\rst{g}) (f \tjup g)\\
        &= (\rst{h f}h\tjdown \rst{h g}h)  (f \tjup g)\\
        &= (\rst{h f}\tjdown \rst{h g})(h\+h)  (f \tjup g)\\
        &= (\rst{h f}\tjdown \rst{h g}) (h f \tjup h g)\\
        &= h f \tjoin h g.
      \end{align*}


    \item [\axiom{DJ}{4}] We need to show $\tperp [f,g,h]$ if and only if $f \tperp (g \tjoin h)$.
      For the right to left implication, note that the existence of $g\tjoin h$ implies $g \tperp
      h$. We also know $g, h \le g\tjoin h$ by item 1 of this lemma. This gives us that $f \tperp
      g$ and $f \tperp h$, hence $\tperp [f,g,h]$.

      For the left to right implication, we use
      Lemma~\ref{lem:complete_disjointness_means_multiple_disjoints}. As we have $\tperp [f,g,h]$,
      we  know $f\tjdown(g\tjdown h)$ and $f\tjup(g\tjup h)$ exist.

      Recall that $g\tjoin h = (g\tjdown h)(\rg{g}\tjup\rg{h})$. Then the map
      \[
        A\xrightarrow{f\tjdown(g\tjdown h)} B\+B\+B\xrightarrow{1\+(\rg{g}\tjup\rg{h})} B\+B
      \]
      makes the diagram for $f \tjdown (g\tjoin h)$ commute.

      Recalling that  $g\tjoin h =
      (\rst{g}\tjdown \rst{h})(g\tjup h)$, we also see that
      \[
        A\+A\xrightarrow{1\+(\rst{g}\tjdown\rst{h})}A\+A\+A\xrightarrow{f\tjup(g\tjup h)}B
      \]
      provides the witness map for $f \tjup(g\tjoin h)$ and hence $f \tperp (g \tjoin h)$.
  \end{itemize}

\end{proof}
% subsection disjoint_join_via_a_monoidal_tensor (end)
% section tensors_for_disjointness (end)

\section{Disjointness in Frobenius Algebras}
\label{sec:disjointness_in_frobenius_algebras}
\begin{definition}\label{def:perp_in_cfrob}
  As shown in ..., $CFrob(\X)$ is a discrete inverse category. For $f,g:A\to B$, define $f\perp g$
  when
\[
\begin{tikzpicture}
\path node at (0,0) [nabla] (n1) {}
node at (0,2.5) (start) {}
node at (-.5,1) [map] (f) {$\scriptstyle f$}
node at (.5,1) [map] (g) {$\scriptstyle g$}
node at (0,2) [delta] (d) {};
\draw [] (d) to (start);
\draw [] (n1) to (0,-0.5);
\draw [] (d) to[out=305,in=90] (g);
\draw [] (d) to[out=235,in=90] (f);
\draw [-] (f) to[out=270,in=125] (n1);
\draw [-] (g) to[out=270,in=55] (n1);
\end{tikzpicture}
\ \raisebox{45pt}{$= 0.$}\
\]
\end{definition}

\begin{lemma}\label{lem:cfrobperp_is_a_disjointness_relation}
  The relation $\perp$ of Definition~\ref{def:perp_in_cfrob} is a disjointness relation.
\end{lemma}
\begin{proof}
We need to show the seven axioms of the disjointness relation hold. Note that we will show
\axiom{Dis}{6} early on as its result will be used in some of the other axiom proofs.\\
\axiom{Dis}{1}: For all $f:A\to B,\ f\cdperp 0$.\\
\[
\begin{tikzpicture}
\path node at (0,0) [nabla] (n1) {}
node at (0,2.5) (start) {}
node at (-.5,1) [map] (f) {$\scriptstyle f$}
node at (.5,1) [map] (z) {$\scriptstyle 0$}
node at (0,2) [delta] (d) {};
\draw [] (d) to (start);
\draw [] (n1) to (0,-0.5);
\draw [] (d) to[out=235,in=90] (f);
\draw [] (d) to[out=305,in=90] (z);
\draw [-] (f) to[out=270,in=125] (n1);
\draw [-] (z) to[out=270,in=55] (n1);
\end{tikzpicture}
\ \raisebox{45pt}{$=$}\
\begin{tikzpicture}
\path node at (0,0) [nabla] (n1) {}
node at (0,2.5) (start) {}
node at (-.5,1) [map] (f) {$\scriptstyle f$}
node at (.5,.75) [map] (z) {$\scriptstyle 0$}
node at (.5,1.25) [map] (r_z) {$\scriptstyle \rst{0}$}
node at (0,2) [delta] (d) {};
\draw [] (d) to (start);
\draw [] (n1) to (0,-0.5);
\draw [] (d) to[out=235,in=90] (f);
\draw [] (d) to[out=305,in=90] (r_z);
\draw [] (r_z) to (z);
\draw [-] (f) to[out=270,in=125] (n1);
\draw [-] (z) to[out=270,in=55] (n1);
\end{tikzpicture}
\ \raisebox{45pt}{$=$}\
\begin{tikzpicture}
\path node at (0,0) [nabla] (n1) {}
node at (0,2.75) [map] (t_z) {$\scriptstyle 0(=\rst{0})$}
node at (-.5,1) [map] (f) {$\scriptstyle f$}
node at (.5,1) [map] (z) {$\scriptstyle 0$}
node at (0,2) [delta] (d) {};
\draw [] (d) to (t_z);
\draw [] (n1) to (0,-0.5);
\draw [] (d) to[out=235,in=90] (f);
\draw [] (d) to[out=305,in=90] (z);
\draw [-] (f) to[out=270,in=125] (n1);
\draw [-] (z) to[out=270,in=55] (n1);
\end{tikzpicture}
\ \raisebox{45pt}{$= 0.$}
\]
\axiom{Dis}{6}: $f\cdperp g$ implies $\rst{f} \cdperp \rst{g}$ and $\rg{f}\cdperp\rg{g}$.\\
We will show the details of $\rst{f} \cdperp \rst{g}$, using $\rst{f} = f\inv{f}$ and the definition of
$\inv{f}$ as given in Theorem~\ref{thm:cfrob_is_a_discrete_inverse_category}. The proof of $\inv{f}f
= \rg{f} \perp \rg{g} = \inv{g}g$ is similar.
\[
\begin{tikzpicture}
\path node at (0,0) [nabla] (n1) {}
node at (0,2.5) (start) {}
node at (-.5,1) [map] (f) {$\scriptstyle \rst{f}$}
node at (.5,1) [map] (g) {$\scriptstyle \rst{g}$}
node at (0,2) [delta] (d) {};
\draw [] (d) to (start);
\draw [] (n1) to (0,-0.5);
\draw [] (d) to[out=305,in=90] (g);
\draw [] (d) to[out=235,in=90] (f);
\draw [-] (f) to[out=270,in=125] (n1);
\draw [-] (g) to[out=270,in=55] (n1);
\end{tikzpicture}
\ \raisebox{45pt}{$=$}\
\begin{tikzpicture}
    \path node at (0.5,3.5) [delta] (start) {}
    node at (0,2.5) [eta] (eta1) {}
    node at (0,2) [delta] (d) {}
    node at (-1.2,1.5) [map] (f1) {$\scriptstyle f$}
    node at (-.5,1.5) [map] (f) {$\scriptstyle f$}
    node at (-1,1) [nabla] (n1) {}
    node at (-1,.5) [epsilon] (e1) {}
    node at (2,2.5) [eta] (e_tag) {}
    node at (2,2) [delta] (d_g) {}
    node at (.8,1.5) [map] (g1) {$\scriptstyle g$}
    node at (1.5,1.5) [map] (g) {$\scriptstyle g$}
    node at (1,1) [nabla] (n_g) {}
    node at (1,.5) [epsilon] (e_g) {}
    node at (1,-.5) [nabla] (end) {};
    \draw [] (start) to[out=235,in=90] (f1);
    \draw [] (start) to[out=305,in=90] (g1);
    \draw [] (f1) to (n1);
    \draw [] (eta1) to (d);
    \draw [] (d) to (end);
    \draw [] (d) to (f);
    \draw [] (f) to (n1);
    \draw [] (n1) to (e1);
    \draw [] (g1) to (n_g);
    \draw [] (e_tag) to (d_g);
    \draw [] (d_g) to[out=305,in=55] (end);
    \draw [] (d_g) to (g);
    \draw [] (g) to (n_g);
    \draw [] (n_g) to (e_g);
\end{tikzpicture}
\ \raisebox{45pt}{$ = $}\
\begin{tikzpicture}
    \path node at (0,4) [delta] (start) {}
    node at (-1,3.5) [eta] (eta1) {}
    node at (1,3.5) [eta] (eta2) {}
    node at (-1,3) [delta] (d1) {}
    node at (1,3) [delta] (d2) {}
    node at (-.5,2.5) [nabla] (n1) {}
    node at (.5,2.5) [nabla] (n2) {}
    node at (-.5,2) [map] (f1) {$\scriptstyle f$}
    node at (.5,2) [map] (g2) {$\scriptstyle g$}
    node at (-.5,1.5) [epsilon] (e1) {}
    node at (.5,1.5) [epsilon] (e2) {}
    node at (0,.5) [nabla] (n) {};
    \draw [] (start) to (0,4.5);
    \draw [] (start) to (n1);
    \draw [] (start) to (n2);
    \draw [] (eta1) to (d1);
    \draw [] (eta2) to (d2);
    \draw [] (d1) to[out=235,in=125] (n);
    \draw [] (d1) to (n1);
    \draw [] (d2) to (n2);
    \draw [] (d2) to[out=305,in=55] (n);
    \draw [] (n1) to (f1);
    \draw [] (n2) to (g2);
    \draw [] (f1) to (e1);
    \draw [] (g2) to (e2);
    \draw [] (n)  to (0,0);
\end{tikzpicture}
\ \raisebox{45pt}{$=$}\
\begin{tikzpicture}
    \path node at (0,4) [delta] (start) {}
    node at (-.5,3) [delta] (d1) {}
    node at (.5,3) [delta] (d2) {}
    node at (-.25,2) [map] (f1) {$\scriptstyle f$}
    node at (.25,2) [map] (g2) {$\scriptstyle g$}
    node at (-.25,1.5) [epsilon] (e1) {}
    node at (.25,1.5) [epsilon] (e2) {}
    node at (0,0) [nabla] (n) {};
    \draw [] (start) to (0,4.5);
    \draw [] (start) to (d1);
    \draw [] (start) to (d2);
    \draw [] (d1) to[out=235,in=125] (n);
    \draw [] (d1) to (f1);
    \draw [] (d2) to (g2);
    \draw [] (d2) to[out=305,in=55] (n);
    \draw [] (f1) to (e1);
    \draw [] (g2) to (e2);
    \draw [] (n)  to (0,-.5);
\end{tikzpicture}
\ \raisebox{45pt}{$=$}
\]
\[
\begin{tikzpicture}
    \path node at (0,4) [delta] (start) {}
    node at (.5,3.5) [delta] (d2) {}
    node at (0,2.5) [delta] (d1) {}
    node at (-.25,2) [map] (f1) {$\scriptstyle f$}
    node at (.25,2) [map] (g2) {$\scriptstyle g$}
    node at (0,1.5) [nabla] (n12) {}
    node at (0,1) [epsilon] (e2) {}
    node at (0,0) [nabla] (n) {};
    \draw [] (start) to (0,4.5);
    \draw [] (start) to[out=235,in=125] (n);
    \draw [] (start) to (d2);
    \draw [] (d2) to[out=235,in=90] (d1);
    \draw [] (d1) to (f1);
    \draw [] (d1) to (g2);
    \draw [] (d2) to[out=305,in=55] (n);
    \draw [] (f1) to (n12);
    \draw [] (g2) to (n12);
    \draw [] (n12) to (e2);
    \draw [] (n)  to (0,-.5);
\end{tikzpicture}
\ \raisebox{45pt}{$=$}\
\begin{tikzpicture}
    \path node at (0,3) [delta] (start) {}
    node at (.5,2.5) [delta] (d2) {}
    node at (0,1.75) [map] (z) {$\scriptstyle 0$}
    node at (0,1) [epsilon] (e2) {}
    node at (0,0) [nabla] (n) {};
    \draw [] (start) to (0,3.5);
    \draw [] (start) to[out=235,in=125] (n);
    \draw [] (start) to (d2);
    \draw [] (d2) to[out=235,in=90] (z);
    \draw [] (d2) to[out=270,in=55] (n);
    \draw [] (z) to (e2);
    \draw [] (n)  to (0,-.5);
\end{tikzpicture}
\ \raisebox{45pt}{$=$}\
\begin{tikzpicture}
\path node at (0,0) [nabla] (n1) {}
node at (0,2.5) (start) {}
node at (.5,1) [map] (z) {$\scriptstyle 0$}
node at (0,2) [delta] (d) {};
\draw [] (d) to (start);
\draw [] (n1) to (0,-0.5);
\draw [] (d) to[out=305,in=90] (z);
\draw [] (d) to[out=235,in=125] (n1);
\draw [-] (z) to[out=270,in=55] (n1);
\end{tikzpicture}
\ \raisebox{45pt}{$= 0.$}
\]
\axiom{Dis}{2}: $f\cdperp g$ implies $\rst{f} g = 0$.\\
In this proof, we use the result of \axiom{Dis}{6}, i.e., that $\rst{f}\perp\rst{g}$.
\[
\raisebox{25pt}{
\begin{tikzpicture}
  \begin{pgfonlayer}{nodelayer}
    \node [style=map] (0) at (-2, 1) {$\scriptstyle \rst{f}$};
    \node [style=map] (1) at (-2, 0) {$\scriptstyle g$};
    \end{pgfonlayer}
    \begin{pgfonlayer}{edgelayer}
      \draw (0) to (1);
      \draw (0) to (-2,1.5);
      \draw (1) to (-2, -.5);
      \end{pgfonlayer}
\end{tikzpicture}
}
\ \raisebox{45pt}{$=$}\
\begin{tikzpicture}
  \begin{pgfonlayer}{nodelayer}
    \node [style=map] (0) at (-2, 3) {$\scriptstyle \rst{f}$};
    \node [style=map] (1) at (-2, 2.25) {$\scriptstyle g$};
    \node [style=delta] (2) at (-1.25, 4) {};
    \node [style=nabla] (3) at (-1.25, 1.25) {};
    \node [style=map] (4) at (-0.5, 3) {$\scriptstyle \rst{f}$};
    \node [style=map] (5) at (-0.5, 2.25) {$\scriptstyle g$};
    \end{pgfonlayer}
    \begin{pgfonlayer}{edgelayer}
      \draw (0) to (1);
      \draw (1) to[out=270,in=125] (3);
      \draw (5) to[out=270,in=55] (3);
      \draw (4) to (5);
      \draw (2) to[out=305,in=90] (4);
      \draw (2) to[out=235,in=90] (0);
      \draw (2) to (-1.25,4.5);
      \draw (3) to (-1.25,.75);
      \end{pgfonlayer}
\end{tikzpicture}
\ \raisebox{45pt}{$=$}\
\begin{tikzpicture}
  \begin{pgfonlayer}{nodelayer}
    \node [style=map] (0) at (-2, 3) {$\scriptstyle \rst{f}$};
    \node [style=map] (1) at (-2, 2.25) {$\scriptstyle g$};
    \node [style=delta] (2) at (-1.25, 4) {};
    \node [style=nabla] (3) at (-1.25, 1.25) {};
    \node [style=map] (4) at (-0.5, 2.25) {$\scriptstyle g$};
    \end{pgfonlayer}
    \begin{pgfonlayer}{edgelayer}
      \draw (0) to (1);
      \draw (1) to[out=270,in=125] (3);
      \draw (4) to[out=270,in=55] (3);
      \draw (2) to[out=235,in=90] (0);
      \draw (2) to[out=305,in=90] (4);
      \draw (3) to (-1.25,.75);
      \draw (2) to (-1.25,4.5);
      \end{pgfonlayer}
\end{tikzpicture}
\ \raisebox{45pt}{$=$}\
\begin{tikzpicture}
  \begin{pgfonlayer}{nodelayer}
    \node [style=map] (0) at (-2, 2.75) {$\scriptstyle \rst{f}$};
    \node [style=map] (1) at (-1.25, 0.75) {$\scriptstyle g$};
    \node [style=delta] (2) at (-1.25, 4) {};
    \node [style=nabla] (3) at (-1.25, 1.5) {};
    \node [style=map] (4) at (-0.5, 2.75) {$\scriptstyle \rst{g}$};
    \end{pgfonlayer}
    \begin{pgfonlayer}{edgelayer}
      \draw (4) to[out=270,in=55] (3);
      \draw (2) to[out=235,in=90] (0);
      \draw (2) to[out=305,in=90] (4);
      \draw (0) to[out=270,in=125] (3);
      \draw (3) to (1);
      \draw (1) to (-1.25,.25);
      \draw (2) to (-1.25,4.5);
      \end{pgfonlayer}
\end{tikzpicture}
\ \raisebox{45pt}{$=$}\
\raisebox{25pt}{
\begin{tikzpicture}
  \begin{pgfonlayer}{nodelayer}
    \node [style=map] (0) at (-1.25, 1.75) {$\scriptstyle 0$};
    \node [style=map] (1) at (-1.25, 0.75) {$\scriptstyle g$};
    \end{pgfonlayer}
    \begin{pgfonlayer}{edgelayer}
      \draw (0) to (1);
      \draw (1) to (-1.25,.25);
      \draw (0) to (-1.25,2.25);
      \end{pgfonlayer}
\end{tikzpicture}
}
\ \raisebox{45pt}{$=0$}
\]
\axiom{Dis}{3}: $f\cdperp g,\ f' \le f,\ g' \le g$ implies $f' \cdperp g'$.\\
\[
\begin{tikzpicture}
\path node at (0,0) [nabla] (n1) {}
node at (0,2.5) (start) {}
node at (-.5,1) [map] (f) {$\scriptstyle f'$}
node at (.5,1) [map] (g) {$\scriptstyle g'$}
node at (0,2) [delta] (d) {};
\draw [] (d) to (start);
\draw [] (n1) to (0,-0.5);
\draw [] (d) to[out=305,in=90] (g);
\draw [] (d) to[out=235,in=90] (f);
\draw [-] (f) to[out=270,in=125] (n1);
\draw [-] (g) to[out=270,in=55] (n1);
\end{tikzpicture}
\ \raisebox{45pt}{$=$}\
\begin{tikzpicture}
  \begin{pgfonlayer}{nodelayer}
    \node [style=map] (0) at (-2, 3) {$\scriptstyle \rst{f'}$};
    \node [style=map] (1) at (-2, 2.25) {$\scriptstyle f$};
    \node [style=delta] (2) at (-1.25, 4) {};
    \node [style=nabla] (3) at (-1.25, 1.25) {};
    \node [style=map] (4) at (-0.5, 3) {$\scriptstyle \rst{g'}$};
    \node [style=map] (5) at (-0.5, 2.25) {$\scriptstyle g$};
    \end{pgfonlayer}
    \begin{pgfonlayer}{edgelayer}
      \draw (0) to (1);
      \draw (1) to[out=270,in=125] (3);
      \draw (5) to[out=270,in=55] (3);
      \draw (4) to (5);
      \draw (2) to[out=305,in=90] (4);
      \draw (2) to[out=235,in=90] (0);
      \draw (2) to (-1.25,4.5);
      \draw (3) to (-1.25,.75);
      \end{pgfonlayer}
\end{tikzpicture}
\ \raisebox{45pt}{$=$}\
\begin{tikzpicture}
\path
node at (0,3) (start) {}
node at (0,2.5) [map] (fg) {$\scriptstyle \rst{f'}\,\rst{g'}$}
node at (0,2) [delta] (d) {}
node at (-.5,1) [map] (f) {$\scriptstyle f$}
node at (.5,1) [map] (g) {$\scriptstyle g$}
 node at (0,0) [nabla] (n1) {};
\draw [] (d) to (fg);
\draw [] (start) to (fg);
\draw [] (n1) to (0,-0.5);
\draw [] (d) to[out=305,in=90] (g);
\draw [] (d) to[out=235,in=90] (f);
\draw [-] (f) to[out=270,in=125] (n1);
\draw [-] (g) to[out=270,in=55] (n1);
\end{tikzpicture}
\ \raisebox{45pt}{$=$}\
\raisebox{25pt}{
\begin{tikzpicture}
\path
node at (0,1.5) (start) {}
node at (0,1) [map] (fg) {$\scriptstyle \rst{f'}\,\rst{g'}$}
node at (0,.5) [map] (z) {$\scriptstyle 0$};
\draw [] (z) to (fg);
\draw [] (start) to (fg);
\draw [] (z) to (0,0);
\end{tikzpicture}
}
\ \raisebox{45pt}{$=0$}
\]
\axiom{Dis}{4}: $f\cdperp g$ implies $g \cdperp f$.\\
This follows directly from the co-commutativity of $\Delta$.\\
\axiom{Dis}{5}: $f\cdperp g$ implies $h f \cdperp h g$.\\
This follows directly from the naturality of $\Delta$.\\
\axiom{Dis}{7}: $\rst{f}\cdperp \rst{g},\ \rg{h}\cdperp \rg{k}$ implies $f h \cdperp g k$.\\
\[
\begin{tikzpicture}
\path node at (0,0) [nabla] (n1) {}
node at (0,2.5) (start) {}
node at (-.5,1) [map] (fh) {$\scriptstyle f h$}
node at (.5,1) [map] (gk) {$\scriptstyle g k$}
node at (0,2) [delta] (d) {};
\draw [] (d) to (start);
\draw [] (n1) to (0,-0.5);
\draw [] (d) to[out=305,in=90] (gk);
\draw [] (d) to[out=235,in=90] (fh);
\draw [-] (fh) to[out=270,in=125] (n1);
\draw [-] (gk) to[out=270,in=55] (n1);
\end{tikzpicture}
\ \raisebox{45pt}{$=$}\
\begin{tikzpicture}
\path node at (0,0) [nabla] (n1) {}
node at (0,2.5) (start) {}
node at (-.5,1.5) [map] (rf) {$\scriptstyle \rst{f}$}
node at (-.5,1) [map] (fh) {$\scriptstyle f h$}
node at (-.5,.5) [map] (rngh) {$\scriptstyle \rg{h}$}
node at (.5,1.5) [map] (rg) {$\scriptstyle \rst{g}$}
node at (.5,1) [map] (gk) {$\scriptstyle g k$}
node at (.5,.5) [map] (rngk) {$\scriptstyle \rg{k}$}
node at (0,2) [delta] (d) {};
\draw [] (d) to (start);
\draw [] (n1) to (0,-0.5);
\draw [] (d) to[out=305,in=90] (rg);
\draw [] (d) to[out=235,in=90] (rf);
\draw (rf) to (fh);
\draw (fh) to (rngh);
\draw (rg) to (gk);
\draw (gk) to (rngk);
\draw [-] (rngh) to[out=270,in=125] (n1);
\draw [-] (rngk) to[out=270,in=55] (n1);
\end{tikzpicture}
\ \raisebox{45pt}{$=$}\
\begin{tikzpicture}
\path node at (0,0) [nabla] (n1) {}
node at (0,2.5) (start) {}
node at (-.5,1) [map] (fh) {$\scriptstyle f h$}
node at (-.5,.5) [map] (rnghrngk) {$\scriptstyle \rg{h}\rg{k}$}
node at (.5,1.5) [map] (rfrg) {$\scriptstyle \rst{f}\rst{g}$}
node at (.5,1) [map] (gk) {$\scriptstyle g k$}
node at (0,2) [delta] (d) {};
\draw [] (d) to (start);
\draw [] (n1) to (0,-0.5);
\draw [] (d) to[out=305,in=90] (rfrg);
\draw [] (d) to[out=235,in=90] (fh);
\draw (fh) to (rnghrngk);
\draw (rfrg) to (gk);
\draw [-] (rnghrngk) to[out=270,in=125] (n1);
\draw [-] (gk) to[out=270,in=55] (n1);
\end{tikzpicture}
\ \raisebox{45pt}{$=$}\
\begin{tikzpicture}
\path node at (0,0) [nabla] (n1) {}
node at (0,2.5) (start) {}
node at (-.5,1) [map] (fh) {$\scriptstyle f h$}
node at (-.5,.5) [map] (rnghrngk) {$\scriptstyle 0$}
node at (.5,1.5) [map] (rfrg) {$\scriptstyle 0$}
node at (.5,1) [map] (gk) {$\scriptstyle g k$}
node at (0,2) [delta] (d) {};
\draw [] (d) to (start);
\draw [] (n1) to (0,-0.5);
\draw [] (d) to[out=305,in=90] (rfrg);
\draw [] (d) to[out=235,in=90] (fh);
\draw (fh) to (rnghrngk);
\draw (rfrg) to (gk);
\draw [-] (rnghrngk) to[out=270,in=125] (n1);
\draw [-] (gk) to[out=270,in=55] (n1);
\end{tikzpicture}
\ \raisebox{45pt}{$=$}\
\begin{tikzpicture}
\path node at (0,0) [nabla] (n1) {}
node at (0,2.5) (start) {}
node at (-.5,1) [map] (fh) {$\scriptstyle 0$}
node at (.5,1) [map] (gk) {$\scriptstyle 0$}
node at (0,2) [delta] (d) {};
\draw [] (d) to (start);
\draw [] (n1) to (0,-0.5);
\draw [] (d) to[out=305,in=90] (gk);
\draw [] (d) to[out=235,in=90] (fh);
\draw [-] (fh) to[out=270,in=125] (n1);
\draw [-] (gk) to[out=270,in=55] (n1);
\end{tikzpicture}
\ \raisebox{45pt}{$= 0.$}\
\]
\end{proof}

%%% Local Variables:
%%% mode: latex
%%% TeX-master: "../../phd-thesis"
%%% End:


%%% Local Variables:
%%% mode: latex
%%% TeX-master: "../phd-thesis"
%%% End:

%%% Local Variables:
%%% mode: latex
%%% TeX-master: "../phd-thesis"
%%% End:
