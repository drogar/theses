%!TEX root = /Users/gilesb/UofC/thesis/phd-thesis/phd-thesis.tex
\chapter{Inverse sum categories} % (fold)
\label{cha:inverse_sum_categories}

Throughout this chapter we will work with a number of relations between maps and operations on
pairs of maps. Suppose we have a relation $\lozenge$ between maps $f,g:B\to C$, i.e., $f \lozenge
g$. We will refer to $\lozenge$ as \emph{stable} whenever given a $h:A \to B$, then $h f \lozenge
h g$. We will refer to $\lozenge$ as \emph{universal} whenever given a $k:C \to D$, then $f k
\lozenge g k$.


\section{Coproducts in restriction categories} % (fold)
\label{sec:coproducts_in_restriction_categories}
\subsection{Coproducts} % (fold)
\label{sub:coproducts}

Restriction categories may also have coproducts and initial objects.
\begin{definition}
  In a restriction category \X, a coproduct is a \emph{restriction coproduct} when the embeddings
  $\cpa$ and $\cpb$ are total.
\end{definition}

\begin{lemma}\label{lem:restriction_coproduct_is_restriction_functor}
  The definition of restriction coproduct implies the following:
  \begin{enumerate}[{(}i{)}]
    \item $\restr{f+g} = \restr{f} + \restr{g}$ which means $+$ is a restriction functor.
    \item $\nabla:A+A\to A$ is total.
    \item $?:0 \to A$ is total, where $0$ is the initial object in the restriction category.
  \end{enumerate}
\end{lemma}
\begin{proof}
  \prepprooflist
  \begin{enumerate}[{(}i{)}]
    \item \textbf{$+$ is a restriction functor.}
      Consider the diagram:
      \[
        \xymatrix @C=30pt @R=35pt{
          A \ar[rr]_{f} \ar[dr]_{\cpa} && A' \ar[dr]^{\cpa'}\\
          &A+B \ar@{.>}[rr]^{f+g} & & A' + B'\\
          B \ar[rr]_{g} \ar[ur]^{\cpb} && B'. \ar[ur]_{\cpb'}
        }
      \]
      In order to show $\restr{f+g} = \restr{f} + \restr{g}$, it suffices to show that
      $\cpa\restr{f+g} = \cpa(\restr{f} + \restr{g}) = \restr{f}\cpa$.
      \begin{align*}
        \cpa\restr{f+g} &= \restr{\cpa(f+g)}\cpa & \rfour\\
        &= \restr{f \cpa'} \cpa &\text{coproduct diagram}\\
        &= \restr{f \restr{\cpa'}} \cpa
          & \text{Lemma }\refitem{lem:restrictionvarious}{lemitem:rv_3}\\
        &= \restr{f} \cpa & \cpa'\text{ total}.
      \end{align*}
    \item   \textbf{$\nabla:A+A\to A$ is total.}
      By the definition of $\nabla$ ($=\<1 | 1 \>$) and the co-product, the following diagram
      commutes,
      \[
        \xymatrix @C=30pt @R=35pt{
          &A + A \ar[d]_{\nabla}\\
          A \ar@{=}[r] \ar[ur]^{\cpa}&A \ar@{=}[r] & A \ar[ul]_{\cpb}
        }
      \]
      resulting in:
      \[
         \cpa \restr{\nabla}  = \restr{\cpa \nabla} \cpa = \restr{1} \cpa = \cpa.
      \]
      Similarly, $\cpb \restr{\nabla} = \cpb$, hence, the restriction of $\nabla$ is $1$ and
      therefore $\nabla$ is total.
    \item  \textbf{$\why:0 \to A$ is total.}
      This follows from
      \[
        \xymatrix @C=30pt @R=35pt{
          0 \ar[dr]_{\why} \ar[r]^{\cpb} &A + 0 \\
          & A \ar@{=}[u]
        }
      \]
      so $\why$ can be defined as the total coproduct injection.
  \end{enumerate}



\end{proof}

Recall that when an object is both initial and terminal, it is referred to as a zero object and
denoted as $0$. This gives rise to the zero map $0_{A,B}: A \to 0 \to B$ between any two objects.

\begin{definition}\label{def:restriction_zero}
  Given a restriction category \X with a zero object, then $0$ is a \emph{restriction zero} when
  for each object $A$ in \X we have $\restr{0_{A,A}} = 0_{A,A}$.
\end{definition}

\begin{lemma}[Cockett-Lack]\label{lem:restriction_zero_equivalences}
  For a restriction category \X, the following are equivalent:
  \begin{enumerate}[{(}i{)}]
    \item \X has a restriction zero;
    \item \X has an initial object $0$ and terminal object $1$ and each initial map $z_A$ is a
      restriction monic;
    \item \X has a terminal object $1$ and each terminal map $t_A$ is a restriction retraction.
  \end{enumerate}
\end{lemma}
% subsection coproducts (end)




\subsection{Inverse categories with restriction coproducts} % (fold)
\label{sub:inverse_categories_with_restriction_coproducts}


\begin{proposition}
  An inverse category \X with restriction coproducts is a pre-order.
\end{proposition}
\begin{proof}
  By Lemma~\ref{lem:restriction_coproduct_is_restriction_functor}, we know $\nabla$ is total and
  therefore $\nabla\inv{\nabla} = 1$. From the coproduct diagrams, we have $\cpa\nabla =1$ and
  $\cpb\nabla = 1$. But this gives us $\inv{\nabla}\inv{\cpa} = \inv{(\cpa\nabla)} = 1$ and
  similarly $\inv{\nabla}\inv{\cpb} =1$. Hence, $\inv{\nabla} = \cpa$ \emph{and} $\inv{\nabla} =
  \cpb$.

  This means for parallel maps $f,g:A \to B$, we have
  \[
    f = \cpa \<f|g\> = \inv{\nabla} \<f|g\> = \cpb\<f|g\> = g
  \]
  and therefore \X is a pre-order.
\end{proof}
% subsection inverse_categories_with_restriction_coproducts (end)
% section coproducts_in_restriction_categories (end)

\section{Disjointness in an inverse category} % (fold)
\label{sec:disjointness_in_an_inverse_category}
In the following, we will add two related structures to an inverse category with a
restriction zero. This structure is meant to be evocative of the \emph{join} concept in a
restriction category. We will first re-iterate some of the basic definitions and lemmas
about joins.
\subsection{Joins in restriction categories} % (fold)
\label{sub:joins_in_restriction_categories}

\begin{definition}\label{def:joins}
  Given \R is a restriction category with a restriction zero, then \R is said to have \emph{joins}
  whenever there is an operator $\join$ defined between compatible maps (from
  Definition~\vref{def:compatible_maps}) such that:
  \begin{itemize}
    \item $f \le f \join g$ and $g \le f \join g$,
    \item $\rst{f\join g} = \rst{f}\join \rst{g}$,
    \item $f,g \le h$ implies that $f\join g \le h$ and
    \item $h(f\join g) = h f \join h g$.
  \end{itemize}
\end{definition}

For example, in the restriction category $\Par$, the join is given by:
\[(f \join g) (x) = \begin{cases}
  f(x)( = g(x))& \text{ when both $f$ and $g$ are defined;}\\
  f(x)& \text{ when only $f$ is defined;}\\
  g(x)& \text{ when only $g$ is defined;}\\
  \uparrow & \text{ when both $f$ and $g$ are undefined.}
\end{cases}
\]

\begin{lemma}\label{lem:meet_distributes_over_join}
  If \R is a meet restriction category with joins, then the meet distributes over the join, i.e.,
  \[
    h\meet(f\join g) = (h\meet f)\join (h\meet g).
  \]
\end{lemma}
\begin{proof}
  \begin{align*}
    h\meet(f\join g) &= \rst{(f\join g)} h\meet (f\join g)\\
    &= (\rst{f}\join \rst{g}) h\meet (f\join g)\\
    &= (\rst{f}(h\meet (f\join g))) \join (\rst{g}(h\meet (f\join g)))\\
    &= (h\meet \rst{f}(f\join g)) \join (h\meet \rst{g}(f\join g)))\\
    &= (h\meet (f\join g)) \join (h\meet (f\join g))).
  \end{align*}
\end{proof}
% subsection joins_in_restriction_categories (end)


\subsection{Disjointness relations} % (fold)
\label{sub:disjointness_relations}
In this subsection, we will define a disjointness relationship between maps and
explore alternate characterizations of this relation on the restriction idempotents of objects.

\begin{definition}\label{def:disjointness_relation}
  In an inverse category \X with a restriction zero, the relation $\cdperp$ between two parallel
  maps $f, g:A \to B$ is called a \emph{disjointness relation} when it satisfies the following
  properties:
  \begin{align*}
    \axiom{Dis}{1}\quad &\text{ For all }f:A\to B,\ f\cdperp 0;\\
    \axiom{Dis}{2}\quad &f\cdperp g\text{ implies }\rst{f} g = 0; \\
    \axiom{Dis}{3}\quad &f\cdperp g,\ f' \le f,\ g' \le g\text{ implies }f' \cdperp g';\\
    \axiom{Dis}{4}\quad &f\cdperp g\text{ implies }g \cdperp f;\\
    \axiom{Dis}{5}\quad &f\cdperp g\text{ implies }h f \cdperp h g; \text{ (Stable)}\\
    \axiom{Dis}{6}\quad &f\cdperp g\text{ implies }\rst{f} \cdperp \rst{g}
      \text{ and }\rg{f}\cdperp\rg{g};\\
    \axiom{Dis}{7}\quad &\rst{f}\cdperp \rst{g},\ \rg{h}\cdperp \rg{k}\text{ implies }f h
      \cdperp g k.
  \end{align*}
\end{definition}

\begin{lemma}\label{lem:disjointness_equivalent_axioms}
  In Definition~\ref{def:disjointness_relation}, provided we retain $\axiom{Dis}{1-5}$, we may
  replace $\axiom{Dis}{6}$ and $\axiom{Dis}{7}$ by:
  \begin{align*}
    \axiom{Dis}{6'}\quad &f\cdperp g\text{ if and only if }\rst{f} \cdperp \rst{g}
      \text{ and }\rg{f}\cdperp\rg{g}.
  \end{align*}
\end{lemma}
\begin{proof}
  Given $\axiom{Dis}{6}$ and $\axiom{Dis}{7}$, the \emph{only if} direction of $\axiom{Dis}{6'}$ is
  immediate. To show the \emph{if} direction, assume $\rst{f} \cdperp \rst{g}$ and
  $\rg{f}\cdperp\rg{g}$. This also means that $\rst{\rst{f}} \cdperp \rst{\rst{g}}$. Then, by
  \axiom{Dis}{7}, $\rst{f} f \cdperp \rst{g}g$ and therefore $f \cdperp g$.

  Conversely, assume we are given \axiom{Dis}{6'}. Then, \axiom{Dis}{6} follows immediately. To show
  \axiom{Dis}{7}, assume we have $\rst{f}\cdperp \rst{g},\ \rg{h}\cdperp \rg{k}$. As
  $\rst{f h} \le \rst{f}$ and $\rst{g k}\le\rst{g}$, by \axiom{Dis}{3}, we know that
  $\rst{f h} \cdperp \rst{g k}$. Similarly, $\wrg{f h} \le \rg{h}$ and $\wrg{g k} \le \rg{k}$,
  giving us $\wrg{f h}\cdperp \wrg{g k}$. Then, from \axiom{Dis}{6'} we may conclude
  $f h \cdperp g k$, showing \axiom{Dis}{7} holds.
\end{proof}
\begin{lemma}\label{lem:disjointness_various}
  In an inverse category \X with $\cdperp$ a disjointness relation:
  \begin{enumerate}[{(}i{)}]
    \item $f \cdperp g$ if and only if $\inv{f}\cdperp \inv{g}$; \label{lemitem:djv_inverses}
    \item $f \cdperp g$ implies $f h \cdperp g h$ (Universal);\label{lemitem:djv_universal}
    \item $f \cdperp g$ implies $f\rg{g} = 0$;
    \item if $m,n$ are monic, then $f m \cdperp g n$ implies $\rst{f} \perp \rst{g}$;
      \label{lemitem:djv_monic_implies}
    \item if $m,n$ are monic, then $\inv{m} f \cdperp \inv{n}g$ implies $\rg{f} \perp \rg{g}$.
  \end{enumerate}
\end{lemma}
\begin{proof}
  \prepprooflist
  \begin{enumerate}[{(}i{)}]
    \item Assume $f \cdperp g$. By \axiom{Dis}{6}, we have $\rst{f} \cdperp \rst{g}$ and
      $\rg{f}\cdperp\rg{g}$. Since $\rg{f} = \rst{\inv{f}}$ and $\rst{f} = \wrg{\inv{f}}$, this
      means $\rst{\inv{f}} \cdperp \rst{\inv{g}}$ and $\wrg{\inv{f}}\cdperp\wrg{\inv{g}}$. By
      \axiom{Dis}{6'} from Lemma~\ref{lem:disjointness_equivalent_axioms}, we have
      $\inv{f} \cdperp \inv{g}$. The converse follows with  a similar argument.
    \item Assume $f \cdperp g$. By the previous item, we have $\inv{f}\cdperp\inv{g}$. By
      \axiom{Dis}{5}, $\inv{h}\inv{f}\cdperp\inv{h}\inv{g}$, giving us $\inv{(f h)} \cdperp
      \inv{(g h)}$. By this Lemma, item \ref{lemitem:djv_inverses}, we now have $f h \cdperp g h$.
    \item Assume $f \cdperp g$. From item \ref{lemitem:djv_inverses} and reflexivity, we know that
      $\inv{g}
      \cdperp \inv{f}$ and therefore $\rst{\inv{g}}\inv{f} = \rg{g}\inv{f}= 0$. However, in an
      inverse category, $\inv{0} = 0$ and therefore $0 = \inv{(\rg{g}\inv{f})} = f \inv{\rg{g}} = f
      \rg{g}$.
    \item Assume $f m \cdperp g n$ where $m, n$ are monic. By \axiom{Dis}{6}, this gives us
      $\rst{f m} \cdperp \rst{g n}$. By Lemma~\ref{lem:restrictionvarious},
      $\rst{f m} = \rst{f \rst{m}} = \rst{f 1} = \rst{f}$ and therefore $\rst{f} \cdperp \rst{g}$.
    \item This is a corollary to this Lemma, item~\ref{lemitem:djv_monic_implies}. By assumption,
      we have $\inv{m} f \cdperp  \inv{n}g$ and therefore $\inv{f}m \cdperp \inv{g} n$. By the
      previous item, this means $\rst{\inv{f}} \cdperp \rst{\inv{g}}$ and hence $\rg{f} \cdperp \rg{g}$.
  \end{enumerate}
\end{proof}

We may define the disjointness relation via its action in \open{a}.

\begin{definition}\label{def:disjointness_in_open_x}
  Given an inverse category \X, a relation $\ocdperpsub{A} \subseteq \open{A}^2$ for each $
  A \in \mathrm{ob}(\X)$, is an \emph{open disjointness} relation when for all $e, e' \in \open{A}$
  \begin{align*}
    \axiom{$\mathcal{O}$dis}{1}\quad &1 \ocdperpsub{A} 0; \\
    \axiom{$\mathcal{O}$dis}{2}\quad &e \ocdperpsub{A} e' \text{ implies }e' \ocdperpsub{A} e;\\
    \axiom{$\mathcal{O}$dis}{3}\quad &e \ocdperpsub{A} e' \text{ implies }e e' = 0;\\
    \axiom{$\mathcal{O}$dis}{4}\quad &e \ocdperpsub{A} e' \text{ implies }\rst{f e} \ocdperpsub{B}
      \rst{f e'}\text{ for all }f:B \to A;\\
    \axiom{$\mathcal{O}$dis}{5}\quad &e \ocdperpsub{A} e' \text{ implies }\wrg{e g} \ocdperpsub{C}
      \wrg{e' g}\text{ for all }g:A \to C; \\
    \axiom{$\mathcal{O}$dis}{6}\quad &e \ocdperpsub{A} e',\ e_1 \le e,\ e_1' \le e'
      \text{ implies }e_1 \ocdperpsub{A} e_1'.
  \end{align*}
\end{definition}

We will normally write $\ocdperp$ rather than $\ocdperpsub{A}$ where the object is either clear or
not germane to the point under discussion.

\begin{proposition}\label{prop:disjointness_is_open_disjointness}
  If $\cdperp$ is a disjointness relation in $\X$, it is an open disjointness
  relation on the restriction idempotents.
\end{proposition}
\begin{proof}
  \prepprooflist
  \setlist[enumerate,1]{leftmargin=1.5cm}
  \begin{enumerate}
    \item[\axiom{$\mathcal{O}$dis}{1}] This follows immediately from \axiom{Dis}{1} by taking
      $f = 1$.
    \item[\axiom{$\mathcal{O}$dis}{2}] Reflexivity follows directly from  \axiom{Dis}{4}.
    \item[\axiom{$\mathcal{O}$dis}{3}] By \axiom{Dis}{2}, $0 = \rst{e} e' = e e'$.
    \item[\axiom{$\mathcal{O}$dis}{4}] Given $e \cdperp e'$, we have $f e \cdperp f e'$ by
      \axiom{Dis}{5}. Then, by \axiom{Dis}{6} we may conclude $\rst{f e} \cdperp \rst{f e'}$.
    \item[\axiom{$\mathcal{O}$dis}{5}] This follows from the above item, using $\inv{g}$ for $f$.
      This means we have $\rst{\inv{g}e} \cdperp \rst{\inv{g}e'}$. But this gives us $\rst{\inv{(e
      g)}} \cdperp \rst{\inv{(e' g)}}$. Recalling from
      Lemma~\ref{lem:inverse_categories_are_range_categories} that $\rg{k} = \rst{\inv{k}}$, we may
      conclude $\wrg{e g} \cdperp \wrg{e' g}$.
    \item[\axiom{$\mathcal{O}$dis}{6}] Assuming $e \cdperp e'$ and $e_1 \le e,\ e'_1 \le e'$, by
      \axiom{Dis}{3}, $e_1 \cdperp e'_1$.
  \end{enumerate}
  Therefore, $\cdperp$ acts as an open disjointness relation on $\open{A}^2$.

\end{proof}

\begin{definition}\label{def:extended_open_disjointness}
  If $\ocdperp$ is an open disjointness relation in $\X$, then we may define a relation
  $\perpab \in \X(A,B)^2$ by
  \[
    \infer{f\perpab g}{f,g:A\to B,\ \rst{f}\ocdperp\rst{g},\ \rg{f}\ocdperp\rg{g}}.
  \]
  We call $\perpab$ an \emph{extended disjointness relation}.
\end{definition}

\begin{proposition}\label{prop:extended_disjointness_is_a_disjointness_relation}
  If $\cdperp$ is an extended disjointness relation based on $\ocdperp$ in \X, then
  $\cdperp$ is a disjointness relation in \X.
\end{proposition}
\begin{proof}
  \prepprooflist
  \setlist[enumerate,1]{leftmargin=1.5cm}
  \begin{enumerate}
    \item[\axiom{Dis}{1}] We need to show $f \cdperp 0$ for any $f$. We know that $1 \ocdperp 0$
      and therefore $\rst{f} \ocdperp 0$ and $\rg{f} \ocdperp 0$, as $\rst{f}\le 1$ and $\rg{f}\le
      1$. This gives us $f \cdperp 0$.
    \item[\axiom{Dis}{2}] Assume $f \cdperp g$, i.e., $\rst{f}\ocdperp \rst{g}$. Then, $\rst{f}g =
      \rst{f}\rst{g}g = 0 g = 0$.
    \item[\axiom{Dis}{3}] We are given $f \cdperp g$, $f' \le f$ and $g' \le g$. By
      Lemma~\refitem{lem:restriction_cats_are_partial_order_enriched}{lemitem:rst_ordering_2}
      $\rst{f'} \le \rst{f}$ and $\rst{g'} \le \rst{g}$. Then, by \axiom{$\mathcal{O}$dis}{6}, as
      $\rst{f} \ocdperp \rst{g}$ we have $\rst{f'} \ocdperp \rst{g'}$. By
      Lemma~\refitem{lem:ordering_of_restriction_and_range}{lemitem:ordering_2}, we have $\wrg{f'}
      \le \rg{f}$ and $\wrg{g'} \le \rg{g}$. Then, by \axiom{$\mathcal{O}$dis}{6}, as $\rg{f}
      \ocdperp \rg{g}$ we have $\wrg{f'} \ocdperp \wrg{g'}$. This means $f' \cdperp g'$.
    \item[\axiom{Dis}{4}] Reflexivity of $\cdperp$ follows immediately from the reflexivity of
      $\ocdperp$.
    \item[\axiom{Dis}{5}] Assume $f \cdperp g$, i.e., $\rst{f}\ocdperp \rst{g}$ and $\rg{f}
      \ocdperp \rg{g}$. Then we have $\rst{hf}\ocdperp \rst{h g}$ by \axiom{$\mathcal{O}$dis}{4}. By
      Lemma~\refitem{lem:ordering_of_restriction_and_range}{lemitem:ordering_1} we have $\wrg{h f}
      \le \rg{f}$ and $\wrg{h g} \le \rg{g}$. Therefore we have $\wrg{h f}\ocdperp \wrg{h g}$ by
      \axiom{$\mathcal{O}$dis}{6} and therefore $h f \cdperp h g$.
    \item[\axiom{Dis}{6}] This follows directly from definition
      \ref{def:extended_open_disjointness}.
    \item[\axiom{Dis}{7}] We assume $\rst{f} \cdperp \rst{g}$ and $\rg{h} \cdperp \rg{k}$. By
      definition \ref{def:extended_open_disjointness} we have $\rst{f} \ocdperp \rst{g}$ and
      $\rg{h} \ocdperp \rg{k}$. By
      Lemma~\refitem{lem:restriction_cats_are_partial_order_enriched}{lemitem:rst_ordering_3}, we
      have $\rst{f h} \le \rst{f}$ and $\rst{g k} \le \rst{g}$. Therefore, $\rst{f h} \ocdperp
      \rst{g k}$ by \axiom{$\mathcal{O}$dis}{6}. By
      Lemma~\refitem{lem:ordering_of_restriction_and_range}{lemitem:ordering_1}, $\wrg{f h} \le
      \rg{h}$ and $\wrg{g k} \le \rg{k}$, giving us $\wrg{f h}\ocdperp \wrg{g k}$ also by
      \axiom{$\mathcal{O}$dis}{6}. This means $f h \cdperp g k$.
  \end{enumerate}
\end{proof}
\begin{theorem}\label{thm:open_disjointness_is_disjointness}
   To give a disjointness relation $\perp$ on $\X$ is to give an open disjointness relation
   $\ocdperp$ on \X.
\end{theorem}
\begin{proof}
  Suppose we are given the disjointness relation $\perp$. By
  Proposition~\ref{prop:disjointness_is_open_disjointness}, this is an open disjointness relation
  on each of the sets of idempotents, $\open{A}$. We will label that relation $\ocdperp$.

  Use Definition~\ref{def:extended_open_disjointness} to create an extended
  disjointness relation based on $\ocdperp$, signify it by $\eocdperp$.
  By Proposition~\ref{prop:extended_disjointness_is_a_disjointness_relation}, $\eocdperp$ is
  a disjointness relation on $\X$.

  Assume $f\perp g$. We have $\rst{f}\ocdperp\rst{g}$ and $\rg{f}\ocdperp\rg{g}$ by
  \axiom{Dis}{6} and Proposition~\ref{prop:disjointness_is_open_disjointness}. Then, from
  Definition~\ref{def:extended_open_disjointness}, we have $f \eocdperp g$.

  Assume $f \eocdperp g$. This means we must have had $\rst{f}\ocdperp\rst{g}$ and
  $\rg{f}\ocdperp\rg{g}$
  by Definition~\ref{def:extended_open_disjointness} and therefore $\rst{f}\perp\rst{g}$ and
  $\rg{f}\perp\rg{g}$. By Proposition~\ref{lem:disjointness_various}, we have $f \perp g$.


  Now, suppose we are given the open disjointness relation $\ocdperp$. Similar to above, we can
  construct the extended disjointness relation $\perp$ by
  Definition~\ref{def:extended_open_disjointness}. From the disjointness relation $\perp$, we have
  the open disjointness relation $\overline{\perp}$ by
  Lemma~\ref{prop:disjointness_is_open_disjointness}.

  Assume $e \ocdperp e'$. As this means both $\rst{e} \ocdperp \rst{e'}$ and
  $\rg{e} \ocdperp \wrg{e'}$, we have $e \perp e'$. By
  Proposition~\ref{prop:disjointness_is_open_disjointness} this means $e \overline{\perp} e'$.

  If we are given that $e \overline{\perp} e'$, then we know that $e \perp e'$ by
  Proposition~\ref{prop:disjointness_is_open_disjointness}. From
  Definition~\ref{def:extended_open_disjointness}, this requires that $\rst{e} \ocdperp \rst{e'}$
  and $\rg{e} \ocdperp \wrg{e'}$, but that just means $e \ocdperp e'$.
\end{proof}

Note that while we have worked with binary disjointness throughout this section, one may extend
the concept to lists of maps simply by considering disjointness pairwise. I.e., we have
$\cdperp [f_1,f_2,\ldots,f_n]$ if and only if $f_i \cdperp f_k$ whenever $i \ne j$.

Disjointness is additional structure on a restriction category, i.e., it is possible to have more
than one disjointness relation on the category.
\begin{example}\label{ex:disjointness_is_a_structure_not_a_property}
  Consider the restriction category \inj. Here, the objects are sets and maps are
  the partial injective set functions, where $\rst{f}=id_{|\dom(f)}$. The restriction zero is the
  empty map (i.e., $\dom(0) = \rng(0) = \emptyset$).

  We may define the disjointness relation $\perp$ by $f \perp g$ if and only if
  $\dom(f)\intersect\dom(g)=\emptyset$ and $\rng(f)\intersect\rng(g) = \emptyset$. It is reasonably
  straightforward to verify \axiom{Dis}{1} through \axiom{Dis}{7}. For example, take \axiom{Dis}{7}:
  \begin{proof}
    We are given $\rst{f}\cdperp\rst{g}$ and $\rg{h}\cdperp\rg{k}$. This means
    \[
      \dom f\intersect\dom g=\emptyset \text{ and }\rng h\intersect\rng k = \emptyset.
    \]
    Note that in    general for partial injective functions $m$ and $n$ we have
    $\dom{m n} \subseteq \dom{m}$ and that $\rng{m n} \subseteq \rng{n}$. Hence we have
    \begin{align*}
      \dom{f h}\intersect\dom{g k}&\subseteq \dom{f}\intersect \dom{g}= \emptyset\\
      \rng{f h}\intersect\rng{g k}&\subseteq \rng{h}\intersect \rng{k}= \emptyset.
    \end{align*}
    Therefore, $f h \cdperp g k$.
  \end{proof}

  We may define a different disjointness relation, $\perp'$, on the same restriction category.
  Define $f \perp' g$ if and only if one of $f$ or $g$ is the restriction 0, $\emptyset$. As
  $0 = \rst{0} = \rg{0} = h 0 = 0 k$, all of the seven disjointness axioms are easily verifiable.
\end{example}

Although disjointness is additional structure on a restriction category, one can use the
disjointness structure of a base category (or categories) to define a disjointness structure
on derived categories, such as the product category.

\begin{lemma}\label{lem:disjointness_is_derivable_on_a_product_category}
  If $\X$ and $\Y$ are inverse categories with restriction zeros and respective disjointness
  relations $\perp$ and $\perp'$, then we may construct a disjointness relation
  $\perp_{\times}$ on $\X\times\Y$.
\end{lemma}
\begin{proof}
  Recall that product categories are defined component-wise. These definitions extend to the
  restriction, the inverse and the restriction zero. That is:
  \begin{itemize}
    \item If $(f,g)$ is a map in $\X\times\Y$, then $\inv{(f,g)} = (\inv{f}, \inv{g})$;
    \item If $(f,g)$ is a map in $\X\times\Y$, then $\rst{(f,g)} = (\rst{f}, \rst{g})$;
    \item The map $(0_X,0_Y)$ is the restriction zero in $\X\times\Y$.
  \end{itemize}

  Following this pattern, for $(f,g)$ and $(h,k)$ maps in $\X\times\Y$, $(f,g) \perp_{\times}(h,k)$
  iff $f\perp h$ and $g\perp' k$.

  Verifying the disjointness axioms is straightforward, we show axioms 2 and 5. Proofs of the
  others are similar.
  {
  \setlist[itemize,1]{leftmargin=1.5cm}
  \begin{itemize}
    \item [\axiom{Dis}{2}]: Given $(f,g)\perp_{\times}(h,k)$, we have
      $\rst{(f,g)}(h,k) =  (\rst{f},\rst{g})(h,k) = (\rst{f} h, \rst{g} k) = (0, 0) = 0$.
    \item [\axiom{Dis}{5}]: We are given $(f,g)\perp_{\times}(h,k)$. Consider the map $z = (x,y)$
      in $\X \times \Y$. We know that $x f \perp x h$ and $y g \perp y k$, therefore we have
      $z(f,g) = (x f, y g) \perp_{\times} (x h, y k) = z(h,k)$.
  \end{itemize}
  }
\end{proof}
% subsection disjointness_relations (end)
\subsection{Disjoint joins} % (fold)
\label{sub:disjoint_joins}

We now consider additional structure on the inverse category, dependent upon the disjointness
relation.
\begin{definition}\label{def:disjoint_join}
  An \emph{inverse category with disjoint joins} is an inverse category $\X$, with a restriction
  $0$, a disjointness relation $\perp$ and a binary operator on disjointness parallel maps:
  \[
    \infer{f \djoin g : A \to B}{f: A\to B,\ g: A\to B,\ f \perp g}
  \]
  where the following hold:
  \begin{align*}
    \axiom{DJ}{1}\quad & f \le f \djoin g\text{ and }g \le f \djoin g; \\
    \axiom{DJ}{2}\quad & f \le h,\ g \le h\text{ and }f\perp g\text{ implies }f \djoin g \le h;\\
    \axiom{DJ}{3}\quad & h(f \djoin g) = h f \djoin h g. \text{ (Stable)} \\
    \axiom{DJ}{4}\quad & \cdperp [f, g, h]\text{ if and only if }f \perp (g\djoin h).\\
  \end{align*}
  The binary operator, $\djoin$, is referred to as the \emph{disjoint join}.
\end{definition}

Note that \axiom{DJ}{1} with \axiom{DJ}{2} immediately gives us that there is only one disjoint
join given a specific disjointness relation.
\begin{lemma}\label{lem:disjoint_join_is_unique}
  Suppose \X in an inverse category with disjoint joins, with the join $\djoin$ and that it
  has a second disjoint join, $\altjoin$. Then $f \djoin g = f \altjoin g$ for all maps $f,g$ in
  \X.
\end{lemma}
\begin{proof}
  The first axiom tells us:
  \[
    f,g \le f\djoin g\text{ and }f,g \le f \altjoin g.
  \]
  Using the second axiom, we may therefore conclude $f \djoin g \le f \altjoin g$ and
  $f \altjoin g \le f\djoin g$, hence $f \djoin g = f \altjoin g$.
\end{proof}
\begin{lemma}\label{lem:join_is_associative_and_commutative_monoid}
  In an inverse category with disjoint joins, the disjoint join respects the restriction and is
  universal. Additionally, it is a partial associative and
  commutative operation, with identity $0$. That is, the following hold:
  \begin{enumerate}[{(}i{)}]
    \item $\rst{f\djoin g} = \rst{f} \djoin \rst{g}$;
    \item $(f \djoin g)k = f k \djoin g k$ (Universal);
    \item $f \perp g,\ g\perp h,\ f\perp h$ implies that $(f\djoin g)\djoin h = f\djoin(g\djoin
      h)$; \label{lemitem:associative_join}
    \item $f \perp g$ implies $f \djoin g = g \djoin f$; \label{lemitem:commutative_join}
    \item $f \djoin 0 = f$. \label{lemitem:identity_for_join}
  \end{enumerate}
\end{lemma}
\begin{proof}
  \prepprooflist
  \begin{enumerate}[{(}i{)}]
    \item As $\rst{f}, \rst{g} \le \rst{f\djoin g}$, we immediately have
      $\rst{f} \djoin \rst{g} \le \rst{f \djoin g}$. To show the other direction, consider
      \begin{align*}
        \rst{f}(\rst{f} \djoin \rst{g}) (f \djoin g)
        &= (\rst{f}\,\rst{f} \djoin \rst{f}\rst{g})(f \djoin g) &\text{\axiom{DJ}{3}}\\
        &= \rst{f} (f\djoin g) &\text{Lemma~\ref{lem:restrictionvarious}, \axiom{Dis}{2}}\\
        &= f.
      \end{align*}
      Hence, we have $f \le (\rst{f} \djoin \rst{g}) (f \djoin g)$ and similarly, so is $g$. By
      \axiom{DJ}{2} and that $\rst{f} \djoin \rst{g}$ is a restriction idempotent, we then have
      \[
        f \djoin g \le (\rst{f} \djoin \rst{g}) (f \djoin g) \le f \djoin g
      \]
      and therefore $f \djoin g = (\rst{f} \djoin \rst{g}) (f \djoin g)$. By
      Lemma~\ref{lem:restriction_cats_are_partial_order_enriched}, $\rst{f\djoin g} \le \rst{f}
      \djoin \rst{g}$ and so $\rst{f\djoin g} = \rst{f}\djoin \rst{g}$.
    \item First consider when $f, g$ and $k$ are restriction idempotents, say $e_0, e_1 $ and $e_2$.
      Then, we have $(e_0 \djoin e_1)e_2 = e_2(e_0 \djoin e_1) = e_2 e_0 \djoin e_2 e_1 =
      e_0 e_2 \djoin e_1 e_2$.
      Next, note that for general $f,g,h$, we have $f k \djoin g k \le (f\djoin g) k$ as both
      $f k, g k \le (f\djoin g) k$.
      By Lemma~\ref{lem:restriction_cats_are_partial_order_enriched}, we need only show that their
      restrictions are equal:
      \begin{align*}
        \rst{(f\djoin g)k} &= \rst{\rst{f\djoin g}(f\djoin g) k } & \rone \\
        & = \rst{f\djoin g}\, \rst{(f\djoin g)k } & \rthree\\
        & = (\rst{f}\djoin \rst{g}) \rst{(f\djoin g)k }  & \text{previous item}\\
        & = \rst{f}\,\rst{(f\djoin g)k } \djoin \rst{g}\, \rst{(f\djoin g)k }
          &\text{idempotent universal}\\
        & = \rst{\rst{f} (f\djoin g)k} \djoin \rst{\rst{g} (f\djoin g)k} & \rthree\\
        & = \rst{f k} \djoin \rst{g k}\\
        & = \rst{f k \djoin g k}.
      \end{align*}
      Therefore, as the restrictions are equal, we have shown $(f\djoin g)k = f k \djoin g k$.
    \item \emph{Associativity}: Note that \axiom{DJ}{4} shows that both sides of the equation
      exist. To show they are equal, we show that they are less than or equal to each other.

      From the definitions, we know that $f\djoin g, h \le (f\djoin g)\djoin h$, which also means
      $f, g \le (f\djoin g)\djoin h$. Similarly, $g\djoin h \le (f\djoin g)\djoin h $ and then $f
      \djoin (g\djoin h)\le (f\djoin g)\djoin h$. Conversely, $f,g,h \le f \djoin (g\djoin h)$ and
      therefore $(f\djoin g)\djoin h \le f \djoin (g\djoin h)$ and both sides are equal.
    \item \emph{Commutativity}: Note first that both $f$ and $g$ are less than or equal to both
      $f\djoin g$ and $g \djoin f$, by \axiom{DJ}{1}. By \axiom{DJ}{2}, we have $f \djoin g \le
      g\djoin f$ and $g\djoin f \le f \djoin g$ and we may conclude $f \djoin g = g \djoin f$.
    \item \emph{Identity}: By \axiom{DJ}{1}, $f \le f \djoin 0$. As $0 \le f$ and $f \le f$, by
      \axiom{DJ}{2}, $f \djoin 0 \le f$ and we have $f = f \djoin 0$.
  \end{enumerate}
\end{proof}

Note that the previous lemma and proof of associativity allows a simple inductive argument which
shows that having binary disjoint joins extends unambiguously to disjoint joins of an arbitrary
finite collection of disjoint maps.

We will write $[f_i]$ to signify a list of maps, where each $f_i: A \to B$. For disjointness,
$\cdperp [f_i]$ will mean that $f_j \cdperp f_k$ where $j\ne k$ and $f_j, f_k \in [f_i]$. Finally,
$\djoin [f_i]$ will mean the disjoint join of all maps $f_i$, i.e.,
$f_1 \djoin f_2 \djoin \cdots \djoin f_n$.

\begin{lemma}\label{lem:arbitrary_disjoint_joins}
  In an inverse category with disjoint joins, $\cdperp [f_i]$ if and only if
  $\djoin [f_i]$ is defined unambiguously.
\end{lemma}
\begin{proof}
  Using \axiom{Dj}{4}, proceed as in the proof of
  Lemma~\refitem{lem:join_is_associative_and_commutative_monoid}{lemitem:associative_join},
  inducting on $n$.
\end{proof}


\begin{lemma}\label{lem:disjoint_joins_only_when_all_disjoint}
  Given $\X$ is an inverse category with a disjoint join, then if $f_i, g_j : A \to B$ and $\cdperp
  [f_i]$ and $\cdperp [g_j]$, then $\djoin [f_i] \cdperp \djoin [g_j]$ if and only $f_i \cdperp
  g_j$ for all $i,j$.
\end{lemma}
\begin{proof}
  Assume $\djoin [f_i] \cdperp \djoin [g_j]$. By \axiom{Dj}{4} and associativity, we have
  $\djoin [f_i] \cdperp g_j$ for each $j$. Using the reflexivity of $\cdperp$,
  \axiom{Dj}{4} and associativity, we have $f_i \cdperp g_j$ for each $i$ and $j$.

  Assume $f_i \cdperp g_j$ for each $i$ and $j$. Then by \axiom{Dj}{4} and associativity, $f_i
  \cdperp \djoin [g_j]$ for each $i$. Applying \axiom{Dj}{4} again, we have $\djoin [f_i] \cdperp
  \djoin [g_j]$.
\end{proof}

Following the same method as in the previous section, we show that the product of two
inverse categories with disjoint joins has a disjoint join.
\begin{lemma}\label{lem:disjoint_join_is_in_product_category}
  Given $\X, \Y$ are inverse categories with disjoint joins, $\djoin$ and $\djoin'$ respectively,
  then the category $\X \times \Y$ is an inverse category with disjoint joins.
\end{lemma}
\begin{proof}
  From Lemma~\ref{lem:disjointness_is_derivable_on_a_product_category}, we know $\X\times\Y$ has a
  disjointness relation that is defined point-wise.  We therefore define $\djoin_{\times}$ the
  disjoint join on $\X\times\Y$ by
  \begin{equation}
    (f,g)\djoin_{\times}(h,k) = (f \djoin h, g \djoin' k). \label{eq:disjoint_join_on_product}
  \end{equation}
  We now prove each of the axioms in Definition~\ref{def:disjoint_join} hold.
  \setlist[itemize,1]{leftmargin=1.5cm}
  \begin{itemize}
    \item [\axiom{DJ}{1}] From Equation~\eqref{eq:disjoint_join_on_product}, we see that since
      $f,h \le f \djoin h$ and $g,k \le g \djoin' k$, we have $(f,g) \le (f,g)\djoin_{\times}(h,k)$
      and  $(h,k) \le (f,g)\djoin_{\times}(h,k)$.
    \item [\axiom{DJ}{2}] Suppose $(f,g) \le (x,y)$, $(h,k) \le (x,y)$ and $(f,g) \perp_{\times}
      (h,k)$. Then regarding it point-wise, we have $(f,g)\djoin_{\times}(h,k) = (f \djoin h, g
      \djoin' k) \le (x,y)$.
    \item [\axiom{DJ}{3}] $(x,y)\left((f,g)\djoin_{\times}(h,k)\right) = (x(f \djoin h), y(g
      \djoin' k)) = (x f \djoin x h, y g \djoin' y k) = (x f,y g)\djoin_{\times}(x h,y k) =
      ((x,y)(f,g))\djoin_{\times}((x,y)(h,k))$.
    \item [\axiom{DJ}{4}] Given $\perp_{\times}\!\![(f,g),(h,k),(x,y)]$, we know $f \perp (h\djoin
      x)$ and $g \perp' (k\djoin' y)$. Hence, $(f,g) \perp_{\times}((h,k)\djoin_{\times}(x,y))$. The
      opposite direction is similar.
  \end{itemize}
\end{proof}

% subsection disjoint_joins (end)

\subsection{Monoidal Tensors for disjointness} % (fold)
\label{sub:tensors_for_disjointness}

When we have a monoidal tensor $\+$ on $\X$, an inverse category with a restriction zero,
there are certain conditions where it is possible to define disjointness based upon the action of the
tensor. We are assuming the following naming for the standard monoidal tensor
isomorphisms:
\begin{align*}
   \upl &: 0 \+ A \to A\\
   \upr &: A \+ 0 \to A\\
   a_\+ &: (A \+ B) \+ C \to A \+ (B \+ C) \\
   c_\+ &: A \+ B \to B \+ A.
 \end{align*}
We also require the tensor isomorphisms above be natural.

\begin{definition}\label{def:disjointness_tensor}
  Given we have an inverse category \X with restriction zero and a symmetric monoidal tensor
  $\+$, the tensor $\+$ is a \emph{disjointness tensor} when:
  \begin{itemize}
    \item It is a restriction functor --- i.e., $\_ \+ \_ : \X \times \X \to \X$.
    \item The unit is the restriction zero. ($0 : \boldsymbol{1}\to \X$ picks out the restriction
    zero in \X).
    \item Define $\cpa = \inv{\upr}(1\+0):A\to A\+B$ and $\cpb = \inv{\upl}(0\+1): A\to B\+A$.
      $\cpa$ and $\cpb$ must be jointly epic. That is, if $\cpa f = \cpa g$ and $\cpb f = \cpb g$, then
      $f = g$.
    \item Define $\scpa := (1\+0)\upr: A\+B \to A$ and $\scpb := (0\+1)\upl:A\+B \to B$.
      $\scpa$  and $\scpb$ must be jointly monic. That is, whenever $f\scpa = g \scpa$ and
      $f\scpb = g\scpb$ then $f = g$.
  \end{itemize}

\end{definition}

\begin{lemma}\label{lem:zero_plus_zero_is_zero}
  Given an inverse category \X with restriction zero and disjointness tensor $\+$, then the map
  $0\+0: A\+B \to C\+D$ is the map $0: A\+B \to C\+D$.
\end{lemma}
\begin{proof}
  Recall the zero map factors through the restriction zero, i.e. $0:A \to B$ is the same as
  saying $A\xrightarrow{!} 0 \xrightarrow{\why}B$. Additionally, as objects, $0\+0 \cong 0$ ---
  the restriction zero.

  Therefore the map $0\+0: A\+B \to C\+D$ is writable as
  \[
    A\+B \xrightarrow{!\+!}0\+0  \xrightarrow{\why\+\why} C\+D,
  \]
  which may then be rewritten as
  \[
    A\+B \xrightarrow{!\+!}0\+0 \xrightarrow{\upl} 0
      \xrightarrow{\inv{\upl}} 0\+0 \xrightarrow{\why\+\why} C\+D.
  \]

  But by the properties of the restriction zero, $(!\+!) \upl = !$ and $\inv{\upl}(\why\+\why) = \why$
  and therefore the map $0\+0: A\+B \to C\+D$ is the same as the map $0: A\+B \to C\+D$.
\end{proof}


\begin{lemma}\label{lem:disjointness_tensor_has_injections}
  Given an inverse category \X with a restriction zero and a disjointness tensor, the map $\cpa$ is
  natural in the left component and $\cpb$ is natural in the right, up to isomorphism. This means:
  \[
    \cpa (f\+g) = f \cpa \quad\text{and}\quad \cpb (f\+g) = g \cpb.
  \]
\end{lemma}
\begin{proof}
  For the left and right naturality, we see:
  \[
    \cpa (f\+g) = \inv{\upr} (1\+0) (f\+g) = \inv{\upr} (f \+ 0) =
      f \inv{\upr} (1\+0) = f \cpa,
  \]
  and
  \[
    \cpb (f\+g) = \inv{\upl} (0\+1) (f\+g) = \inv{\upl} (0 \+ g) =
      g \inv{\upl} (0\+1) = g \cpb.
  \]
\end{proof}



\begin{lemma}\label{lem:tensor_identities}
  Given an inverse category \X with restriction zero and disjointness tensor $\+$,
  $\scpa = \icpa$ and $\scpb = \icpb$ and the following identities hold:
  \begin{enumerate}
    \item $\scp{i} \cp{i} = \rst{\scp{i}}$ and $\cp{i}\scp{i} = \rst{\cp{i}} = 1$;
    \item $\rst{\scpa} \scpb = 0$ and $\rst{\scpb} \scpa = 0$;
    \item $\cpb \scpa = 0$, $\cpb \rst{\scpa} = 0$,  $\cpa \scpb = 0$ and
      $\cpa \rst{\scpb} = 0$;
    \item the maps $\cpa$ and $\cpb$ are monic.
  \end{enumerate}
  Accordingly,
\end{lemma}
\begin{proof}
  For item 1, recalling that the restriction zero is its own partial inverse,
  we see that
  \[
    \icpa = \inv{(\inv{\upr}(1\+0))}  = \inv{(1\+0)}\upr = (1\+0)\upr = \scpa.
  \]
  Similarly,
  \[
    \icpb = \inv{(\inv{\upl}(0\+1))} = (0\+1)\upl = \scpb.
  \]

  Hence, we may calculate the restriction of $\cpa$,
  \begin{equation*}
    \cpa \scpa   = \inv{\upr}(1\+0) (1\+0)\upr
      = (\inv{\upr}(1\+0))\upr = 1 \inv{\upr}\upr= 1.
  \end{equation*}
  The calculation for $\scpb$ and $\cpb$ is analogous.

  To show $\rst{\scpa} \scpb = 0$ and $\rst{\scpb} \scpa = 0$,
  \begin{align*}
    \rst{\scpa} \scpb  &= \rst{(1\+0)\upr}(0\+1)\upl\\
      &= \rst{1\+0} (0 \+ 1)\upl\\
      &= (1\+0) (0\+1)\upl\\
      &= (0\+0)\upl  = 0,
  \end{align*}
  and
  \begin{align*}
    \rst{\scpb} \scpa  &= \rst{(0\+1)\upl}(1\+0)\upr \\
     &= (0\+1) (1\+0)\upr\\
     &= (0\+0)\upr\\
     &= 0.
 \end{align*}

  To show $\cp{i} \scp{j} = 0$, $\cp{i} \rst{\scp{j}} = 0$ when $i\ne j$,
  \begin{align*}
    \cpa \scpb &= (\inv{\upr}(1\+0))(0\+1) \upl\\
       &= \inv{\upr}(0\+0)\upl\\
       &= 0
  \end{align*}
  and
  \begin{align*}
    \cpb \scpa &= (\inv{\upl}(0\+1))(1\+0) \upr\\
      &= \inv{\upl}(0\+0)\upr\\
       &= 0.
  \end{align*}
  As $\rst{\scpa} = 1\+0$ and $\rst{\scpb} = 0 \+1$, we see the other two identities hold as well.

  To prove $\cpa$ is monic, suppose $f\cpa = g\cpa$. Therefore we must have
  \[
    f = f (\cpa \icpa) = (f \cpa) \icpa = (g \cpa) \icpa = g (\cpa \icpa) = g.
  \]
  The proof that $\cpb$ is monic follows via a similar argument.
\end{proof}

As we have shown that $\scp{i} = \icp{i}$, we will prefer the explicit notation of $\icp{i}$ for
rest of this paper.

\begin{corollary}\label{cor:split_plus_tensor}
  In an inverse category \X with a restriction zero and disjointness tensor, the following
  identities hold:
  \begin{multicols}{2}
    \begin{enumerate}[{(}i{)}]
      \item $\cpa (f \+g) \icpa = f$;
      \item $\cpa (f\+ g) \icpb = 0$;
      \item $\cpb (f \+g) \icpa = 0$;
      \item $\cpb (f\+ g) \icpb = g$.
    \end{enumerate}
  \end{multicols}
  Additionally, if $t$ is a map such that for $i \in \{1,2\}$,
  \[
    \cp{i} t \icp{j} =
    \begin{cases}
      t_i& : \quad i \neq j\\
      0 & : \quad i = j,
    \end{cases}
  \]
  then $t = t_1 \+ t_2$.
\end{corollary}
\begin{proof}
  The calculations for $f\+g$ follow from Lemma~\ref{lem:disjointness_tensor_has_injections} and
  Lemma~\ref{lem:tensor_identities}. For example, $\cpa (f\+g) \icpa = f \cpa \icpa = f$.

  For the second claim, note that we have $\cpa (t \icpa) = t_1 = \cpa ( t_1 \+ t_2)\icpa$ and
  $\cpb  (t \icpa) = 0 = \cpb ( t_1 \+ t_2)\icpa$, hence $t \icpa = ( t_1 \+ t_2)\icpa$. Similarly,
  we see $t\icpb = ( t_1 \+ t_2)\icpb$ and therefore $t = t_1 \+ t_2$.
\end{proof}

\begin{definition}\label{def:up_triangle_and_down_triangle}
  In an inverse category \X with a restriction zero and disjointness tensor, we define two partial
  operations on pairs of arrows in \X to another arrow in \X. First, for arrows
  $f:A \to B$ and $g:A \to C$, we define $f\tjdown g$ as
  being the map that makes diagram \eqref{eq:tjdowndefinition} below commute, when it exists.
  \begin{equation}
    \xymatrix@C+0pt @R+20pt{
      &&A \ar[dll]^{g} \ar[drr]_{f} \ar@{.>}[d]|-{f\tjdown g}\\
      B&&B\+C \ar[ll]_{\icpa} \ar[rr]^{\icpb} && C
      }\label{eq:tjdowndefinition}
  \end{equation}
  Then for $h:B \to A$, $k: C \to A$, $h\tjup k$ is that map that makes diagram
  \eqref{eq:tjupdefinition} commute, if it exists.
  \begin{equation}
    \xymatrix@C+0pt @R+20pt{
      B\ar[rr]_{\cpa} \ar[drr]^{h} &&B\+C   \ar@{.>}[d]|-{h\tjup k}
        && C \ar[ll]^{\cpb} \ar[dll]_{k}\\
      &&A
      }\label{eq:tjupdefinition}
  \end{equation}
  Due to $\icpa$ and $\icpb$ being jointly monic, $f\tjdown g$ is unique when it exists.
  Similarly, as $\cpa$ and $\cpb$ are jointly epic, $f\tjup g$ is unique when it exists.
\end{definition}

We give a lemma exploring the
behaviour of the two operations: $\tjdown$ and $\tjup$.

\begin{lemma}\label{lem:properties_of_tjdown_and_tjup}
  Given \X is an inverse category with a restriction zero and a disjointness tensor $\+$ then the
  following relations hold for $\tjdown$ and $\tjup$:
   \begin{enumerate}[{(}i{)}]
    \item If $f \tjdown g$ exists, then $g \tjdown f$ exists. If $f \tjup g$ exists, then
    $g \tjup f$ exists. \label{lemitem:l_r_commute}
    \item $f \tjdown 0$ and $f \tjup 0$ always exist. \label{lemitem:l_r_zero_exists}
    \item When $f\tjdown g$ exists, $\rst{f}(f\tjdown g) = f \tjdown 0$, $\rst{f}g = 0$,
      $\rst{g}(f \tjdown g) = 0 \tjdown g$ and $\rst{g}f = 0$.
      \label{lemitem:l_r_rst_orthogonal_to_zero}
    \item Dually to the previous item, when $f\tjup g$ exists,
      $(f\tjup g)\rg{f} = f \tjup 0$, $g\rg{f} = 0$, $(f \tjup g)\rg{g} = 0 \tjup g$ and
      $f\rg{g} = 0$.\label{lemitem:l_r_rg_orthogonal_to_zero}
    \item When $f\tjdown g$ exists, $f\tjdown g (h\+k) = f h \tjdown g k$.
      \label{lemitem:l_r_l_pull_right}
    \item Dually, when $f \tjup g$ exists, $(h \+ k) f \tjup g = h f \tjup k g$.
      \label{lemitem:l_r_r_pull_left}
    \item When $f \tjdown g$ exists, then $h(f \tjdown g) = h f \tjdown h g$ and when $f \tjup g$
      exists, $(f \tjup g)h = f h \tjup g h$.
      \label{lemitem:l_r_l_universal_r_stable}
    \item If $\rst{f} \tjdown \rst{g}$ exists, then $\rst{f} \tjup \rst{g}$ exists and is
      the partial inverse of $\rst{f} \tjdown \rst{g}$.
      \label{lemitem:l_r_rst_inverses}
    \item If $f\tjdown g$ exists and $f' \le f$, $g' \le g$, then $f' \tjdown g'$ exists.
      \label{lemitem:l_r_less_than}
    \item When $f\tjup g$ exists, $(f\tjup g) \inv{(f\tjup g)} = \rst{f}\+\rst{g}$.
      \label{lemitem:l_r_rst_is_natural}
    \item Given $f\tjdown g$ and $h\tjdown k$ exist, then
      $(f\+h)\tjdown(g\+k) = (f\tjdown g)\+ (h\tjdown k)$. Dually, the existence of $f\tjup g$
      and $h\tjup k$ implies $(f\+h)\tjup(g\+k) = (f\tjup g)\+ (h\tjup k)$.
      \label{lemitem:l_r_preserve_tensor}
  \end{enumerate}
\end{lemma}
\begin{proof}
  \prepprooflist
  \begin{enumerate}[{(}i{)}]
    \item $g \tjdown f = (f \tjdown g)\comp $ and $g \tjup f = \comp(f\tjup g)$.
    \item Consider $f \cpa$. Then $f \cpa \icpa = f$ and
      $f \cpa \icpb = f 0 = 0$. Hence, $f \cpa = f \tjdown 0$.

      Consider $\icpa f$. Then $\cpa \icpa f = f$ and
      $\cpb \icpa f = 0 f = 0$ and therefore $\icpa f = (f\tjup 0)$.
    \item Using Lemma~\ref{lem:tensor_identities}
      \[
        \rst{f}g = \rst{(f\tjdown g)\icpa} (f\tjdown g)\icpb =
          (f\tjdown g) \rst{\icpa} \icpb = 0.
      \]
      Similarly, $\rst{g}f = f\tjdown g \rst{\icpb}\icpa = 0$.

      Recall that $\icpa$ and $\icpb$ are jointly monic. We have
      $\rst{f}(f \tjdown g) \icpa = \rst{f} f = f = (f\tjdown 0) \icpa$ and
      $\rst{f}(f \tjdown g) \icpb = \rst{f}g = 0= (f\tjdown 0) \icpb$.
      Therefore, $\rst{f}(f \tjdown g) = f \tjdown 0$. Similarly,
      $\rst{g}(f \tjdown g) = 0 \tjdown g$.
    \item Using Lemma~\ref{lem:tensor_identities}
      \begin{multline*}
        g\rg{f} = \cpb(f\tjup g)\wrg{(\cpa(f\tjup g))}  =
          \cpb(f\tjup g) \rst{\inv{(f\tjup g)}\icpa} =\\
          \cpb(f\tjup g) \rst{\inv{(f\tjup g)}\rst{\icpa}} =
          \rst{\cpb \rst{(f\tjup g)} \rst{\icpa}}\cpb(f\tjup g) = \\
          \rst{\cpb \rst{\icpa}\rst{(f\tjup g)} }\cpb(f\tjup g) =
          \rst{0}\cpb(f\tjup g) = 0
      \end{multline*}
      Similarly, $f\rg{g} =  0$.

      Recall that $\cpa$ and $\cpb$ are jointly epic. We have
      $\cpa(f \tjup g)\rg{f} = f\rg{f} = f = \cpa(f\tjup 0) $ and
      $\cpb(f \tjup g)\rg{f} = g\rg{f} = 0= \cpb(f\tjup 0)$.
      Therefore, $(f \tjup g)\rg{f} = f \tjup 0$. Similarly,
      $(f \tjup g) \rg{g} = 0 \tjup g$.
    \item Calculating, we have
      \[
        f \tjdown g (h \+ k) \icpa = f\tjdown g \icpa h = f h
      \]
      and
      \[
        f \tjdown g (h \+ k) \icpb = f\tjdown g \icpb k = g k,
      \]
      which means that
      $f \tjdown g (h \+ k) = f h \tjdown g k$ by the joint monic property of $\icpa$, $\icpb$.
    \item The proof for this is dual to the previous item, and depends on the joint epic property
      of $\cpa$ and $\cpb$.
    \item We are given $f\tjdown g$ exists, therefore $f = (f\tjdown g)\icpa$ and
      $g = (f\tjdown g)\icpb$. But this means $hf = h(f\tjdown g)\icpa$ and
      $h g = h(f\tjdown g)\icpb$, from which we may conclude $hf \tjdown h g = h(f\tjdown g)$ by
      the fact that $\icpa$ and $\icpb$ are jointly monic. The proof of
      $(f\tjup g)h = f h \tjup g h$ is similar.
    \item We are given $\rst{f} = \rst{f}\tjdown\rst{g} \icpa$. Therefore,
      \[
        \rst{f} = \inv{\rst{f}} = \inv{\icpa}\inv{(\rst{f}\tjdown\rst{g})}
          = \cpa \inv{(\rst{f}\tjdown\rst{g})}.
      \]
      Similarly, $\rst{g} = \cpb \inv{(\rst{f}\tjdown\rst{g})}$. But this means
      $\inv{(\rst{f}\tjdown\rst{g})} = \rst{f}\tjup\rst{g}$.
    \item Note that from item \ref{lemitem:l_r_l_pull_right}, we know that
      $f \tjdown g  = \rst{f} \tjdown \rst{g} (f\+g)$. We are given $f' \le f$ and $g' \le g$. This
      gives us $\rst{f'}f =f'$, $\rst{g'}g =g'$, $\rst{f'}\,\rst{f} =\rst{f'}$ and
      $\rst{g'}\,\rst{g} =\rst{g'}$. Consider the map
      $\rst{f} \tjdown \rst{g} (\rst{f'}\+\rst{g'})(f\+g)$. Calculating, we see
      \begin{align*}
        \rst{f} \tjdown \rst{g} (\rst{f'}\+\rst{g'})(f\+g)
          &= \rst{f} \tjdown \rst{g} (\rst{f'}\+\rst{g'})(\rst{f'}\+\rst{g'})(f\+g) \\
          & = \rst{f} \tjdown \rst{g} (\rst{f'}\+\rst{g'})(f'\+g') \\
          & = \rst{f}\,\rst{f'} \tjdown \rst{g} \rst{g'} (f'\+g') \\
          & = \rst{f'}\,\rst{f} \tjdown \rst{g'} \rst{g} (f'\+g') \\
          & = \rst{f'} \tjdown \rst{g'}  (f'\+g') \\
          & = f'\tjdown g'.
      \end{align*}
    \item From our diagram for $\tjup$, we know:
      \begin{align*}
        \inv{f} &= \inv{(f\tjup g)} \icpa\text{ and}\\
        \inv{g} &= \inv{(f\tjup g)} \icpb.
      \end{align*}
      As well, we know that $\cpa(f\tjup g) = f$ and
      $\cpa(f\tjup g) = g$.
      Therefore, we have:
      \[
         \cpa(f\tjup g)\inv{(f\tjup g)} \icpa = \rst{f} \text{ and  }
          \cpb(f\tjup g)\inv{(f\tjup g)} \icpb = \rst{g}.
      \]
      As $f\tperp g$, we know that $f\inv{g} = f \rg{g} \inv{g} = 0 \inv{g} = 0$ and therefore,

      \[
         \cpa(f\tjup g)\inv{(f\tjup g)} \icpb = 0 \text{ and  }
          \cpb(f\tjup g)\inv{(f\tjup g)} \icpa = 0.
      \]
      By Corollary~\ref{cor:split_plus_tensor} this means
      $(f\tjup g)\inv{(f\tjup g)} = \rst{f} \+ \rst{g}$.
    \item As $(f\tjdown g)\+ (h\tjdown k)\icpa = (f\tjdown g)$ and
      $(f\tjdown g)\+ (h\tjdown k) \icpb = (h\tjdown k)$, we see that
      $(f\tjdown g)\+ (h\tjdown k)$ satisfies the diagram for $(f\+h)\tjdown(g\+k)$. Dually, as
      $\cpa (f\tjup g)\+ (h\tjup k) = (f\tjup g)$ and
      $\cpb(f\tjup g)\+ (h\tjup k) = (h\tjup k)$, $(f\tjup g)\+ (h\tjup k)$ satisfies
      the diagram for $(f\+h)\tjup(g\+k)$.
  \end{enumerate}
\end{proof}

% \begin{definition}\label{def:general_triangle}
%   In an inverse category \X with a restriction zero and disjointness tensor, we define two partial
%   operations on pairs of arrows in \X to another arrow in \X. First, for arrows
%   $f:A \to B$ and $g:A \to C$, suppose we have arrows $f':A  \to B\+X$ and $g': A \to C\+X$
%   such that $f'\icpa =f$, $g'\icpa = g$ and $f'\icpb = g' \icpb$. Then, we say $f\gtjdown g$ if
%   regardless of the choice of $f',\ g'$ and $X$, there exists a map $\alpha$ that makes
%   diagram \eqref{eq:gtjdowndefinition} below commute.
%   \begin{equation}
%     \xymatrix@C+0pt @R+20pt{
%       B\+X&&B\+C\+X \ar[ll]_{\icp{0,2}} \ar[rr]^{\icp{1,2}} && C\+X\\
%       &&A \ar[ull]^{g'} \ar[urr]_{f'} \ar@{.>}[u]|-{\alpha}
%       }\label{eq:gtjdowndefinition}
%   \end{equation}
%
%   Similarly, for $h:B \to A$, $k: C \to A$, and $h': A\+X \to C$, $k': B\+X \to C$, with $h = \cpa
%   h'$ and $k = \cpb k'$, then $h\gtjup k$ if regardless of the choice of $h',\ k'$ and $X$, there
%   exists a map $\beta$ that makes diagram \eqref{eq:gtjupdefinition} commute.
%   \begin{equation}
%     \xymatrix@C+0pt @R+20pt{
%       &&A\\
%       B\+X\ar[rr]_{\amalg_{0,2}} \ar[urr]^{h'} &&B\+C \+X  \ar@{.>}[u]|-{\beta}
%         && C\+X \ar[ll]^{\amalg_{1,2}} \ar[ull]_{k'}
%       }\label{eq:gtjupdefinition}
%   \end{equation}
%   In the above diagrams, $\icp{0,2} = (1\+0\+1)(\upr\+1)$, $\amalg_{0,2} =
%   (\inv{\upr}\+1)(1\+0\+1)$, $\icp{1,2}=(0\+1\+1)(\upl\+1)$ and
%   $\cp{1,2} = (\inv{\upl}\+1)(0\+1\+1)$.
% \end{definition}
% \begin{lemma}\label{lem:general_triangle_implies_triangle}
%   In an inverse category $\X$ with a restriction zero and a disjointness tensor, with maps
%   $f:A\to B$, $g: A \to C$, $h: B \to A$ and $k: C \to A$. Then $f\gtjdown g$
%   (respectively $h \gtjup k$) implies that $f\tjdown g$ (respectively $h \tjup k$) exists.
% \end{lemma}
% \begin{proof}
%   Suppose $f \gtjdown g$. Then set $f' = f \inv{\upr}$, $g' = g \inv{\upr}$ and $X = 0$. Then
%   $\alpha: A \to B\+C\+0$ exists and we can see that setting $f\tjdown g = \alpha \upr$ makes the
%   first diagram of Definition~\ref{def:up_triangle_and_down_triangle} commute. In detail,
%   \[
%     \xymatrix@C+0pt @R+10pt{
%       B&&B\+C \ar[ll]_{\icpa} \ar[rr]^{\icpb} && C\\
%       &B\+0\ar[ul]^{\upl}&B\+C\+0 \ar[l]_{\icp{0,2}} \ar[r]^{\icp{1,2}} \ar[u]|-{\upl}
%         & C\+0\ar[ur]_{\upl}\\
%       &&A \ar[ul]^{g \inv{\upl}} \ar[ur]_{f\inv{\upl}} \ar@{.>}[u]|-{\alpha}
%       }
%   \]
%   where the lower triangle commutes by assumption and the upper quadrilaterals commute due to the
%   coherence diagrams of the tensor.
%
%   Similarly, setting $h' = \inv{\upr}h$ and $k' = \inv{\upr}k$ gives us $\beta:B\+C\+0 \to A\+0$ and
%   we set $f\tjup = \beta \upr.$
% \end{proof}

\begin{definition}\label{def:disjointness_via_a_disjointness_tensor}
  Define $f \tperp g$ when $f, g: A\to B$ and both $f\tjdown g$ and $f\tjup g$.
\end{definition}

\begin{lemma}\label{lem:tensor_disjointness_is_disjointness}
  If \X is an inverse category with a restriction zero and a disjointness tensor $\+$ then
  the relation $\tperp$ is a disjointness relation.
\end{lemma}
\begin{proof}
  We need to show that $\tperp$ satisfies the disjointness axioms. We will use \axiom{Dis}{6'} in
  place of \axiom{Dis}{6} and \axiom{Dis}{7} as discussed in
  Lemma~\ref{lem:disjointness_equivalent_axioms}.
  \setlist[itemize,1]{leftmargin=1.5cm}
  \begin{itemize}
    \item[\axiom{Dis}{1}] We must show $f \tperp 0$. This follows immediately from
      Lemma~\ref{lem:properties_of_tjdown_and_tjup}, item \ref{lemitem:l_r_zero_exists}.
    \item[\axiom{Dis}{2}] Show $f \tperp g$ implies $\rst{f}g = 0$. This is a direct consequence of
      Lemma~\ref{lem:properties_of_tjdown_and_tjup}, item
      \ref{lemitem:l_r_rst_orthogonal_to_zero}.
    \item[\axiom{Dis}{3}] We require $f\tperp g$, $f' \le f$, $g' \le g$ implies $f' \tperp g'$.
      From Lemma~\ref{lem:properties_of_tjdown_and_tjup}, item \ref{lemitem:l_r_less_than}, we
      immediately have $f' \tjdown g'$ exists. Using a similar argument to the proof of this item,
      we also have $f' \tjup g'$ exists and hence $f' \tperp g'$.
    \item[\axiom{Dis}{4}] Commutativity of $\tperp$ follows from the symmetry of the two required
    diagrams, see Lemma~\ref{lem:properties_of_tjdown_and_tjup}, item
      \ref{lemitem:l_r_commute}.
    \item[\axiom{Dis}{5}] Show that if $f\tperp g$ then $h f \tperp h g$ for any map $h$.
      By  Lemma~\ref{lem:properties_of_tjdown_and_tjup}, item
      \ref{lemitem:l_r_l_universal_r_stable}, we know that $h f \tjdown h g$ exists.
      By item \ref{lemitem:l_r_r_pull_left},  $(h f)\tjup (h g) = (h \+ h) (f\tjup g)$
      and therefore $h f \tperp h g$.
    \item[\axiom{Dis}{6'}] We need to show $f\tperp g$ if and only if $\rst{f} \tperp \rst{g}$ and
      $\rg{f}\tperp \rg{g}$. This follows directly from
      Lemma~\ref{lem:properties_of_tjdown_and_tjup}, items
      \ref{lemitem:l_r_l_pull_right} and       \ref{lemitem:l_r_r_pull_left}, which give us
      $f\tjdown g = \rst{f}\tjdown \rst{g} (f\+g)$ and
      $f\tjup g = (f \+ g)\rg{f}\tjup \rg{g}$, where the equalities hold if either side of
      the equation exists.
  \end{itemize}
\end{proof}

The operations $\tjdown$ and $\tjup$ are sufficient to define a disjointness relation
on an inverse category. However, when we wish to extend this to a disjoint join, we run into
problems when trying to prove $\axiom{DJ}{4}$. Specifically, there is not enough information to
show that $\tperp [f,g,h]$ implies $f \tperp (g \tjoin h)$.

Therefore, we add one more assumption regarding our tensor in order to define disjointness.

\begin{definition}\label{def:disjoint_sum_tensor}
  Let \X be an inverse category with a disjointness tensor $\+$ and a restriction zero. Consider
  diagrams \ref{dia:tensor_complete_left} and \ref{dia:tensor_complete_right}.
  \begin{align}
    &\xymatrix@C+5pt@R+10pt{
      A \ar@/_/[ddr]_f \ar@/^/[drr]^g \ar@{.>}[dr]|-{\alpha} \\
        & X\+Y\+Z \ar[d]^{\icp{1,2}} \ar[r]_{\icp{1,3}} & X\+Z \ar[d]^{\icpa} \\
        & X\+Y \ar[r]_{\icpa} & X
    } \label{dia:tensor_complete_left}\\
    &\xymatrix@C+5pt@R+10pt{
        &&A \\
         X\+Y \ar[r]_{\cp{1,2}} \ar@/^/[urr]^h &X\+Y\+Z \ar@{.>}[ur]|-{\beta} \\
         X \ar[u]^{\cpa} \ar[r]_{\cpa} & X\+Z \ar[u]^{\cp{1,3}} \ar@/_/[uur]_k
    } \label{dia:tensor_complete_right}
  \end{align}
  Then
  $\+$ is a \emph{disjoint sum tensor} when the following two conditions hold:
  \begin{itemize}
    \item $\alpha$ exists if and only if  $f \icpb \tjdown g \icpb$ exists;
    \item $\beta$ exists if and only if $\cpb h \tjup \cpb k$ exists.
  \end{itemize}

\end{definition}
\begin{lemma}\label{lem:complete_disjointness_means_multiple_disjoints}
  Let \X be an inverse category with a disjoint sum tensor as in
  Definition~\ref{def:disjoint_sum_tensor} and we are given $f,g,h:A\to B$ with
  $\tperp[f,g,h]$. Then both $f \tjdown (g \tjdown h)$ and $f\tjup(g\tjup h)$ exist.
\end{lemma}
\begin{proof}
  As all the maps are disjoint, we know the maps $\tjdown$ and $\tjup$ exist for each pair.
  Consider the diagram
  \[
    \xymatrix@C+5pt@R+10pt{
      A \ar@/_/[ddr]_{g\tjdown h} \ar@/^/[drr]^{g\tjdown f} \ar@{.>}[dr]|-{\alpha} \ar[rr]^{f}
        & & B \\
        & B\+B\+B \ar[d]^{\icp{0,1}} \ar[r]_{\icp{0,2}} & B\+B \ar[d]^{\icpa} \ar[u]_{\icpb} \\
        & B\+B \ar[r]_{\icpa} & B
    }
  \]
  where we claim $\alpha = (g\tjdown h)\tjdown f$.

  The lower part of the diagram commutes as it fulfills the conditions of
  Definition~\ref{def:disjoint_sum_tensor}. The upper rightmost triangle of the diagram commutes by
  the definition of $g\tjdown f$. Noting that $\icp{0,1}:B\+B\+B \to B\+B$ is the same map as
  $\icpa:(B\+B)\+B\to (B\+B)$ and $\icp{0,2} \icpb:B\+B\+B \to B\+B \to B$ is the same map as
  $\icpb:(B\+B)\+B\to B$, we see $\alpha$ does make the $\tjdown$ diagram for $g\tjdown h$ and $f$
  commute. Therefore by Lemma~\ref{lem:properties_of_tjdown_and_tjup}, $f\tjdown(g\tjdown h)$
  exists and is equal to $\alpha \com{\+\,\{01,2\}}$.

  A dual diagram and corresponding reasoning shows $f\tjup(g\tjup h)$ exists.
\end{proof}

\begin{lemma}\label{lem:tjdown_and_tjup_associate}
  In an inverse category $\X$ with a disjoint sum tensor, when $\tperp [f,g,h]$, then:
  \begin{enumerate}
    \item   $f \tjdown (g \tjdown h) = ((f \tjdown g) \tjdown h) \assocp$ and both exist,
    \item   $f \tjup (g \tjup h) = ((f \tjup g) \tjup h) \assocp$ and both exist.
  \end{enumerate}
\end{lemma}
\begin{proof}
  Consider the diagram
  \begin{equation}
    \xymatrix@C+5pt@R+10pt{
      A \ar@/_/[ddr]_{f\tjdown g} \ar@/^/[drr]^{f\tjdown h} \ar@{.>}[dr]|-{\alpha} \ar[rr]^{h}
        & & B \\
        & B\+B\+B \ar[d]^{\icp{0,1}} \ar[r]_{\icp{0,2}} & B\+B \ar[d]^{\icpa} \ar[u]_{\icpb} \\
        & B\+B \ar[r]_{\icpa} & B
    }\label{dia:alpha_plus_h}
  \end{equation}
  which gives us $\alpha = (f \tjdown g) \tjdown h: A \to (B\+B)\+B$ and
  $\alpha \assocp :A \to B\+(B\+B)$. Next consider the diagram
  \begin{equation}
    \xymatrix@C+5pt@R+10pt{
      A \ar@/_/[ddr]_{g\tjdown h} \ar@/^/[drr]^{g\tjdown f} \ar@{.>}[dr]|-{\gamma} \ar[rr]^{f}
        & & B \\
        & B\+B\+B \ar[d]^{\icp{0,1}} \ar[r]_{\icp{0,2}} & B\+B \ar[d]^{\icpa} \ar[u]_{\icpb} \\
        & B\+B \ar[r]_{\icpa} & B
    }\label{dia:gamma_plus_f}
  \end{equation}
  which gives us $\gamma \comp = f \tjdown (g\tjdown h): A \to B\+(B\+B)$.

  Note from Diagrams~\ref{dia:alpha_plus_h} and \ref{dia:gamma_plus_f} we have
  \begin{align*}
    \gamma \comp  \icp{0}         =\, &f  = \alpha \assocp\icpa\\
    \gamma \comp  \icp{1} \icp{0} =\, &g  = \alpha \assocp\icpb\icpa\\
    \gamma \comp  \icp{1} \icpb   =\, & h = \alpha \assocp\icpb\icpb.
  \end{align*}
  By the assumption that $\icpa, \icpb$ are jointly monic, we have
  $\alpha = \gamma \comp  \assocp$. Therefore $f \tjdown (g \tjdown h) = (f \tjdown g) \tjdown h$,
  up to the associativity isomorphism.
\end{proof}
\begin{definition}\label{def:tensor_disjoint_join}
  Let \X be an inverse category with a disjointness tensor and restriction zero. Assume we have two
  maps $f,g: A \to B$ with $f\tperp g$. Then define the map $f \tjoin g = \rst{f}\tjdown\rst{g}
  (f\+g) \rg{f}\tjup\rg{g}$.
\end{definition}


For reference, the map $f \tjoin g$ may be visualized as follows:
\[
  \xymatrix@C+20pt @R+10pt{
    &A
      & B  \ar[dr]^{\rg{f}} \ar[d]_{\cpa}\\
    A \ar[ur]^{\rst{f}} \ar@{.>}[r]^{\rst{f}\tjdown \rst{g}} \ar[dr]_{\rst{g}}
      & A\+A \ar[u]_{\icpa} \ar[d]^{\icpb}
        \ar[r]^{f\+g}
       &B\+B \ar@{.>}[r]^{\rg{f}\tjup\rg{g}} & B.\\
    & A & B\ar[ur]_{\rg{g}} \ar[u]^{\cpb}
  }
\]

Using Lemma~\ref{lem:properties_of_tjdown_and_tjup}, we may rewrite this in a variety of
equivalent ways:
\begin{align*}
  f \tjoin g &= \rst{f}\tjdown\rst{g} (f\+g) \rg{f}\tjup\rg{g} \\
  &= f \tjdown g \rg{f}\tjup\rg{g}\\
  & = \rst{f}\tjdown\rst{g}  f\tjup g\\
  & = f \tjdown g (\inv{f}\+\inv{g})f\tjup g.
\end{align*}

In particular, note that $\rst{f}\tjoin \rst{g} = (\rst{f} \tjdown \rst{g})
(\rst{f}\tjup\rst{g})$ as $\rg{\rst{g}} = \rst{g}$.

\begin{lemma}\label{lem:tensor_disjoint_join_properties}
  Let \X be an inverse category with a disjointness tensor and restriction zero. Let \X have the
  maps $f,g: A \to B$ with $f\tperp g$. Then $\tjoin$ has the following properties.
  \begin{enumerate}[{(}i{)}]
    \item For all maps $h:A \to B$, $\rst{f}h \tjoin \rst{g}h = (\rst{f}\tjoin \rst{g})h$.
      \label{lemitem:tdj_rst_universal}
    \item $\rst{f}\tjoin \rst{g} = \rst{f\tjoin g}$. \label{lemitem:tdj_rst_is_rst}
  \end{enumerate}
\end{lemma}
\begin{proof}
  \begin{enumerate}[{(}i{)}]
    \item By Lemma~\ref{lem:disjointness_various}, item \ref{lemitem:djv_universal}, we know that
    $\rst{f}h \tperp \rst{g} h$, hence we can form $\rst{f}h \tjoin \rst{g}h$.
    Also, noting that
      \[
        h \wrg{\rst{f} h} = h \rst{\inv{h}\rst{f}} = \rst{h \inv{h} \rst{f}} h
          = \rst{\rst{h}\rst{f}} h = \rst{f} \rst{h}h = \rst{f} h,
      \]
      we may then calculate from the left hand side as follows:
      \begin{align*}
        \rst{f}h \tjoin \rst{g}h
          & = (\rst{f}h \tjdown \rst{g} h) (\wrg{\rst{f}h}\tjup\wrg{\rst{g} h})\\
          & = (\rst{f} \tjdown \rst{g} ) (h\wrg{\rst{f}h}\tjup h\wrg{\rst{g} h})\\
          & = (\rst{f} \tjdown \rst{g} ) (\rst{f}h\tjup \rst{g} h)\\
          & = (\rst{f} \tjdown \rst{g} ) (\rst{f}\tjup \rst{g} )h\\
          & = (\rst{f} \tjoin \rst{g} ) h.
      \end{align*}
    \item
    Using Lemma~\ref{lem:properties_of_tjdown_and_tjup}, item~\ref{lemitem:l_r_rst_is_natural},
    we can compute:
    \begin{align*}
      \rst{f\tjoin g} & = f \tjoin g \inv{(f\tjoin g)} \\
      & = \left((\rst{f} \tjdown \rst{g})(f\tjup g)\right)
             \left(\inv{(f\tjdown g)} \inv{(\rst{f}\tjdown \rst{g})}\right)\\
      & = \rst{f}\tjdown\rst{g}(f\tjdown g) \inv{(f\tjdown g)} \rst{f}\tjup \rst{g}\\
      & = \rst{f}\tjdown\rst{g}(\rst{f}\+ \rst{g})\rst{f}\tjup \rst{g}\\
      & = \rst{f}\tjdown\rst{g}\rst{f}\tjup \rst{g}\\
      & = \rst{f}\tjoin\rst{g}.
    \end{align*}
  \end{enumerate}
\end{proof}


\begin{proposition}\label{prop:disjointness_tensor_gives_disjoint_join}
  Let \X be an inverse category with a disjoint sum tensor and restriction zero. Assume we have two
  maps $f,g$ with $f\tperp g$. Then the map $f \tjoin g$ from
  Definition~\ref{def:tensor_disjoint_join} is a disjoint join.
\end{proposition}
\begin{proof}
  \setlist[itemize,1]{leftmargin=1.5cm}
  \begin{itemize}
    \item [\axiom{DJ}{1}] We must show $f, g \le f \tjoin g$.
      \begin{align*}
        \rst{f}\,(\rst{f}\tjdown\rst{g})  f\tjup g
          &= (\rst{f}\tjdown\rst{g}) \icpa (\rst{f}\tjdown\rst{g} ) f\tjup g \\
        &=\rst{(\rst{f}\tjdown\rst{g}) \icpa} (\rst{f}\tjdown\rst{g}) f\tjup g  \\
        &=(\rst{f}\tjdown\rst{g}) \rst{\icpa} f\tjup g  \\
        &=(\rst{f}\tjdown\rst{g}) \icpa \cpa f\tjup g \\
        &=((\rst{f}\tjdown\rst{g}) \icpa) (\cpa (f \tjup g)) \\
        &= \rst{f} f \\
        &=f.
      \end{align*}
      Thus, we see $f \le f \tjoin g$. Showing $g \le f \tjoin g$ proceeds in the same manner.
    \item [\axiom{DJ}{2}] We must show that $f \le h,\ g\le h$ and $f\tperp g$ implies
      $f \tjoin g \le h$.
      \begin{align*}
        \rst{f \tjoin g} \,h & = \rst{\rst{f} h \tjoin \rst{g} h} \,h\\
        & = \rst{(\rst{f} \tjoin \rst{g})h} \,h\\
        & = \rst{\rst{(\rst{f} \tjoin \rst{g})}h} \,h\\
        & = \rst{\rst{(\rst{f} \tjoin \rst{g})}h}\,(\rst{f} \tjoin \rst{g})h\\
        & = \rst{(\rst{f} \tjoin \rst{g})h}\,(\rst{f} \tjoin \rst{g})h\\
        & = (\rst{f}\tjoin \rst{g})h\\
        & = (\rst{f}h\tjoin \rst{g}h)\\
        & = (f\tjoin g).
      \end{align*}
    \item [\axiom{DJ}{3}] We must show stability of $\tjoin$, i.e., that
      $h(f\tjoin g) = h f \tjoin h g$.

      \begin{align*}
        h (f\tjoin g) &= h ((\rst{f}\tjdown\rst{g})  (f \tjup g))\\
        &= (h\rst{f}\tjdown h\rst{g}) (f \tjup g)\\
        &= (\rst{h f}h\tjdown \rst{h g}h)  (f \tjup g)\\
        &= (\rst{h f}\tjdown \rst{h g})(h\+h)  (f \tjup g)\\
        &= (\rst{h f}\tjdown \rst{h g}) (h f \tjup h g)\\
        &= h f \tjoin h g.
      \end{align*}


    \item [\axiom{DJ}{4}] We need to show $\tperp [f,g,h]$ if and only if $f \tperp (g \tjoin h)$.
      For the right to left implication, note that the existence of $g\tjoin h$ implies $g \tperp
      h$. We also know $g, h \le g\tjoin h$ by item 1 of this lemma. This gives us that $f \tperp
      g$ and $f \tperp h$, hence $\tperp [f,g,h]$.

      For the left to right implication, we use
      Lemma~\ref{lem:complete_disjointness_means_multiple_disjoints}. As we have $\tperp [f,g,h]$,
      we  know $f\tjdown(g\tjdown h)$ and $f\tjup(g\tjup h)$.

      Recall that $g\tjoin h = (g\tjdown h)(\rg{g}\tjup\rg{h})$. Then the map
      \[
        A\xrightarrow{f\tjdown(g\tjdown h)} B\+B\+B\xrightarrow{1\+(\rg{g}\tjup\rg{h})} B\+B
      \]
      makes the diagram for $f \tjdown (g\tjoin h)$ commute.

      Recalling that  $g\tjoin h =
      (\rst{g}\tjdown \rst{h})(g\tjup h)$, we also see that
      \[
        A\+A\xrightarrow{1\+(\rst{g}\tjdown\rst{h})}A\+A\+A\xrightarrow{f\tjup(g\tjup h)}B
      \]
      provides the witness map for $f \tjup(g\tjoin h)$ and hence $f \tperp (g \tjoin h)$.
  \end{itemize}

\end{proof}

% sub-subsection disjoint_join_via_a_monoidal_tensor (end)
% subsection tensors_for_disjointness (end)

\section{Disjointness in Frobenius Algebras}
\label{sec:disjointness_in_frobenius_algebras}
\begin{definition}\label{def:perp_in_cfrob}
  As shown in ..., $CFrob(\X)$ is a discrete inverse category. For $f,g:A\to B$, define $f\perp g$
  when
\[
\begin{tikzpicture}
\path node at (0,0) [nabla] (n1) {}
node at (0,2.5) (start) {}
node at (-.5,1) [map] (f) {$\scriptstyle f$}
node at (.5,1) [map] (g) {$\scriptstyle g$}
node at (0,2) [delta] (d) {};
\draw [] (d) to (start);
\draw [] (n1) to (0,-0.5);
\draw [] (d) to[out=305,in=90] (g);
\draw [] (d) to[out=235,in=90] (f);
\draw [-] (f) to[out=270,in=125] (n1);
\draw [-] (g) to[out=270,in=55] (n1);
\end{tikzpicture}
\ \raisebox{45pt}{$= 0.$}\
\]
\end{definition}

\begin{lemma}\label{lem:cfrobperp_is_a_disjointness_relation}
  The relation $\perp$ of Definition~\ref{def:perp_in_cfrob} is a disjointness relation.
\end{lemma}
\begin{proof}
We need to show the seven axioms of the disjointness relation hold. Note that we will show
\axiom{Dis}{6} early on as its result will be used in some of the other axiom proofs.\\
\axiom{Dis}{1}: For all $f:A\to B,\ f\cdperp 0$.\\
\[
\begin{tikzpicture}
\path node at (0,0) [nabla] (n1) {}
node at (0,2.5) (start) {}
node at (-.5,1) [map] (f) {$\scriptstyle f$}
node at (.5,1) [map] (z) {$\scriptstyle 0$}
node at (0,2) [delta] (d) {};
\draw [] (d) to (start);
\draw [] (n1) to (0,-0.5);
\draw [] (d) to[out=235,in=90] (f);
\draw [] (d) to[out=305,in=90] (z);
\draw [-] (f) to[out=270,in=125] (n1);
\draw [-] (z) to[out=270,in=55] (n1);
\end{tikzpicture}
\ \raisebox{45pt}{$=$}\
\begin{tikzpicture}
\path node at (0,0) [nabla] (n1) {}
node at (0,2.5) (start) {}
node at (-.5,1) [map] (f) {$\scriptstyle f$}
node at (.5,.75) [map] (z) {$\scriptstyle 0$}
node at (.5,1.25) [map] (r_z) {$\scriptstyle \rst{0}$}
node at (0,2) [delta] (d) {};
\draw [] (d) to (start);
\draw [] (n1) to (0,-0.5);
\draw [] (d) to[out=235,in=90] (f);
\draw [] (d) to[out=305,in=90] (r_z);
\draw [] (r_z) to (z);
\draw [-] (f) to[out=270,in=125] (n1);
\draw [-] (z) to[out=270,in=55] (n1);
\end{tikzpicture}
\ \raisebox{45pt}{$=$}\
\begin{tikzpicture}
\path node at (0,0) [nabla] (n1) {}
node at (0,2.75) [map] (t_z) {$\scriptstyle 0(=\rst{0})$}
node at (-.5,1) [map] (f) {$\scriptstyle f$}
node at (.5,1) [map] (z) {$\scriptstyle 0$}
node at (0,2) [delta] (d) {};
\draw [] (d) to (t_z);
\draw [] (n1) to (0,-0.5);
\draw [] (d) to[out=235,in=90] (f);
\draw [] (d) to[out=305,in=90] (z);
\draw [-] (f) to[out=270,in=125] (n1);
\draw [-] (z) to[out=270,in=55] (n1);
\end{tikzpicture}
\ \raisebox{45pt}{$= 0.$}
\]
\axiom{Dis}{6}: $f\cdperp g$ implies $\rst{f} \cdperp \rst{g}$ and $\rg{f}\cdperp\rg{g}$.\\
We will show the details of $\rst{f} \cdperp \rst{g}$, using $\rst{f} = f\inv{f}$ and the definition of
$\inv{f}$ as given in Theorem~\ref{thm:cfrob_is_a_discrete_inverse_category}. The proof of $\inv{f}f
= \rg{f} \perp \rg{g} = \inv{g}g$ is similar.
\[
\begin{tikzpicture}
\path node at (0,0) [nabla] (n1) {}
node at (0,2.5) (start) {}
node at (-.5,1) [map] (f) {$\scriptstyle \rst{f}$}
node at (.5,1) [map] (g) {$\scriptstyle \rst{g}$}
node at (0,2) [delta] (d) {};
\draw [] (d) to (start);
\draw [] (n1) to (0,-0.5);
\draw [] (d) to[out=305,in=90] (g);
\draw [] (d) to[out=235,in=90] (f);
\draw [-] (f) to[out=270,in=125] (n1);
\draw [-] (g) to[out=270,in=55] (n1);
\end{tikzpicture}
\ \raisebox{45pt}{$=$}\
\begin{tikzpicture}
    \path node at (0.5,3.5) [delta] (start) {}
    node at (0,2.5) [eta] (eta1) {}
    node at (0,2) [delta] (d) {}
    node at (-1.2,1.5) [map] (f1) {$\scriptstyle f$}
    node at (-.5,1.5) [map] (f) {$\scriptstyle f$}
    node at (-1,1) [nabla] (n1) {}
    node at (-1,.5) [epsilon] (e1) {}
    node at (2,2.5) [eta] (e_tag) {}
    node at (2,2) [delta] (d_g) {}
    node at (.8,1.5) [map] (g1) {$\scriptstyle g$}
    node at (1.5,1.5) [map] (g) {$\scriptstyle g$}
    node at (1,1) [nabla] (n_g) {}
    node at (1,.5) [epsilon] (e_g) {}
    node at (1,-.5) [nabla] (end) {};
    \draw [] (start) to[out=235,in=90] (f1);
    \draw [] (start) to[out=305,in=90] (g1);
    \draw [] (f1) to (n1);
    \draw [] (eta1) to (d);
    \draw [] (d) to (end);
    \draw [] (d) to (f);
    \draw [] (f) to (n1);
    \draw [] (n1) to (e1);
    \draw [] (g1) to (n_g);
    \draw [] (e_tag) to (d_g);
    \draw [] (d_g) to[out=305,in=55] (end);
    \draw [] (d_g) to (g);
    \draw [] (g) to (n_g);
    \draw [] (n_g) to (e_g);
\end{tikzpicture}
\ \raisebox{45pt}{$ = $}\
\begin{tikzpicture}
    \path node at (0,4) [delta] (start) {}
    node at (-1,3.5) [eta] (eta1) {}
    node at (1,3.5) [eta] (eta2) {}
    node at (-1,3) [delta] (d1) {}
    node at (1,3) [delta] (d2) {}
    node at (-.5,2.5) [nabla] (n1) {}
    node at (.5,2.5) [nabla] (n2) {}
    node at (-.5,2) [map] (f1) {$\scriptstyle f$}
    node at (.5,2) [map] (g2) {$\scriptstyle g$}
    node at (-.5,1.5) [epsilon] (e1) {}
    node at (.5,1.5) [epsilon] (e2) {}
    node at (0,.5) [nabla] (n) {};
    \draw [] (start) to (0,4.5);
    \draw [] (start) to (n1);
    \draw [] (start) to (n2);
    \draw [] (eta1) to (d1);
    \draw [] (eta2) to (d2);
    \draw [] (d1) to[out=235,in=125] (n);
    \draw [] (d1) to (n1);
    \draw [] (d2) to (n2);
    \draw [] (d2) to[out=305,in=55] (n);
    \draw [] (n1) to (f1);
    \draw [] (n2) to (g2);
    \draw [] (f1) to (e1);
    \draw [] (g2) to (e2);
    \draw [] (n)  to (0,0);
\end{tikzpicture}
\ \raisebox{45pt}{$=$}\
\begin{tikzpicture}
    \path node at (0,4) [delta] (start) {}
    node at (-.5,3) [delta] (d1) {}
    node at (.5,3) [delta] (d2) {}
    node at (-.25,2) [map] (f1) {$\scriptstyle f$}
    node at (.25,2) [map] (g2) {$\scriptstyle g$}
    node at (-.25,1.5) [epsilon] (e1) {}
    node at (.25,1.5) [epsilon] (e2) {}
    node at (0,0) [nabla] (n) {};
    \draw [] (start) to (0,4.5);
    \draw [] (start) to (d1);
    \draw [] (start) to (d2);
    \draw [] (d1) to[out=235,in=125] (n);
    \draw [] (d1) to (f1);
    \draw [] (d2) to (g2);
    \draw [] (d2) to[out=305,in=55] (n);
    \draw [] (f1) to (e1);
    \draw [] (g2) to (e2);
    \draw [] (n)  to (0,-.5);
\end{tikzpicture}
\ \raisebox{45pt}{$=$}
\]
\[
\begin{tikzpicture}
    \path node at (0,4) [delta] (start) {}
    node at (.5,3.5) [delta] (d2) {}
    node at (0,2.5) [delta] (d1) {}
    node at (-.25,2) [map] (f1) {$\scriptstyle f$}
    node at (.25,2) [map] (g2) {$\scriptstyle g$}
    node at (0,1.5) [nabla] (n12) {}
    node at (0,1) [epsilon] (e2) {}
    node at (0,0) [nabla] (n) {};
    \draw [] (start) to (0,4.5);
    \draw [] (start) to[out=235,in=125] (n);
    \draw [] (start) to (d2);
    \draw [] (d2) to[out=235,in=90] (d1);
    \draw [] (d1) to (f1);
    \draw [] (d1) to (g2);
    \draw [] (d2) to[out=305,in=55] (n);
    \draw [] (f1) to (n12);
    \draw [] (g2) to (n12);
    \draw [] (n12) to (e2);
    \draw [] (n)  to (0,-.5);
\end{tikzpicture}
\ \raisebox{45pt}{$=$}\
\begin{tikzpicture}
    \path node at (0,3) [delta] (start) {}
    node at (.5,2.5) [delta] (d2) {}
    node at (0,1.75) [map] (z) {$\scriptstyle 0$}
    node at (0,1) [epsilon] (e2) {}
    node at (0,0) [nabla] (n) {};
    \draw [] (start) to (0,3.5);
    \draw [] (start) to[out=235,in=125] (n);
    \draw [] (start) to (d2);
    \draw [] (d2) to[out=235,in=90] (z);
    \draw [] (d2) to[out=270,in=55] (n);
    \draw [] (z) to (e2);
    \draw [] (n)  to (0,-.5);
\end{tikzpicture}
\ \raisebox{45pt}{$=$}\
\begin{tikzpicture}
\path node at (0,0) [nabla] (n1) {}
node at (0,2.5) (start) {}
node at (.5,1) [map] (z) {$\scriptstyle 0$}
node at (0,2) [delta] (d) {};
\draw [] (d) to (start);
\draw [] (n1) to (0,-0.5);
\draw [] (d) to[out=305,in=90] (z);
\draw [] (d) to[out=235,in=125] (n1);
\draw [-] (z) to[out=270,in=55] (n1);
\end{tikzpicture}
\ \raisebox{45pt}{$= 0.$}
\]
\axiom{Dis}{2}: $f\cdperp g$ implies $\rst{f} g = 0$.\\
In this proof, we use the result of \axiom{Dis}{6}, i.e., that $\rst{f}\perp\rst{g}$.
\[
\raisebox{25pt}{
\begin{tikzpicture}
  \begin{pgfonlayer}{nodelayer}
    \node [style=map] (0) at (-2, 1) {$\scriptstyle \rst{f}$};
    \node [style=map] (1) at (-2, 0) {$\scriptstyle g$};
    \end{pgfonlayer}
    \begin{pgfonlayer}{edgelayer}
      \draw (0) to (1);
      \draw (0) to (-2,1.5);
      \draw (1) to (-2, -.5);
      \end{pgfonlayer}
\end{tikzpicture}
}
\ \raisebox{45pt}{$=$}\
\begin{tikzpicture}
  \begin{pgfonlayer}{nodelayer}
    \node [style=map] (0) at (-2, 3) {$\scriptstyle \rst{f}$};
    \node [style=map] (1) at (-2, 2.25) {$\scriptstyle g$};
    \node [style=delta] (2) at (-1.25, 4) {};
    \node [style=nabla] (3) at (-1.25, 1.25) {};
    \node [style=map] (4) at (-0.5, 3) {$\scriptstyle \rst{f}$};
    \node [style=map] (5) at (-0.5, 2.25) {$\scriptstyle g$};
    \end{pgfonlayer}
    \begin{pgfonlayer}{edgelayer}
      \draw (0) to (1);
      \draw (1) to[out=270,in=125] (3);
      \draw (5) to[out=270,in=55] (3);
      \draw (4) to (5);
      \draw (2) to[out=305,in=90] (4);
      \draw (2) to[out=235,in=90] (0);
      \draw (2) to (-1.25,4.5);
      \draw (3) to (-1.25,.75);
      \end{pgfonlayer}
\end{tikzpicture}
\ \raisebox{45pt}{$=$}\
\begin{tikzpicture}
  \begin{pgfonlayer}{nodelayer}
    \node [style=map] (0) at (-2, 3) {$\scriptstyle \rst{f}$};
    \node [style=map] (1) at (-2, 2.25) {$\scriptstyle g$};
    \node [style=delta] (2) at (-1.25, 4) {};
    \node [style=nabla] (3) at (-1.25, 1.25) {};
    \node [style=map] (4) at (-0.5, 2.25) {$\scriptstyle g$};
    \end{pgfonlayer}
    \begin{pgfonlayer}{edgelayer}
      \draw (0) to (1);
      \draw (1) to[out=270,in=125] (3);
      \draw (4) to[out=270,in=55] (3);
      \draw (2) to[out=235,in=90] (0);
      \draw (2) to[out=305,in=90] (4);
      \draw (3) to (-1.25,.75);
      \draw (2) to (-1.25,4.5);
      \end{pgfonlayer}
\end{tikzpicture}
\ \raisebox{45pt}{$=$}\
\begin{tikzpicture}
  \begin{pgfonlayer}{nodelayer}
    \node [style=map] (0) at (-2, 2.75) {$\scriptstyle \rst{f}$};
    \node [style=map] (1) at (-1.25, 0.75) {$\scriptstyle g$};
    \node [style=delta] (2) at (-1.25, 4) {};
    \node [style=nabla] (3) at (-1.25, 1.5) {};
    \node [style=map] (4) at (-0.5, 2.75) {$\scriptstyle \rst{g}$};
    \end{pgfonlayer}
    \begin{pgfonlayer}{edgelayer}
      \draw (4) to[out=270,in=55] (3);
      \draw (2) to[out=235,in=90] (0);
      \draw (2) to[out=305,in=90] (4);
      \draw (0) to[out=270,in=125] (3);
      \draw (3) to (1);
      \draw (1) to (-1.25,.25);
      \draw (2) to (-1.25,4.5);
      \end{pgfonlayer}
\end{tikzpicture}
\ \raisebox{45pt}{$=$}\
\raisebox{25pt}{
\begin{tikzpicture}
  \begin{pgfonlayer}{nodelayer}
    \node [style=map] (0) at (-1.25, 1.75) {$\scriptstyle 0$};
    \node [style=map] (1) at (-1.25, 0.75) {$\scriptstyle g$};
    \end{pgfonlayer}
    \begin{pgfonlayer}{edgelayer}
      \draw (0) to (1);
      \draw (1) to (-1.25,.25);
      \draw (0) to (-1.25,2.25);
      \end{pgfonlayer}
\end{tikzpicture}
}
\ \raisebox{45pt}{$=0$}
\]
\axiom{Dis}{3}: $f\cdperp g,\ f' \le f,\ g' \le g$ implies $f' \cdperp g'$.\\
\[
\begin{tikzpicture}
\path node at (0,0) [nabla] (n1) {}
node at (0,2.5) (start) {}
node at (-.5,1) [map] (f) {$\scriptstyle f'$}
node at (.5,1) [map] (g) {$\scriptstyle g'$}
node at (0,2) [delta] (d) {};
\draw [] (d) to (start);
\draw [] (n1) to (0,-0.5);
\draw [] (d) to[out=305,in=90] (g);
\draw [] (d) to[out=235,in=90] (f);
\draw [-] (f) to[out=270,in=125] (n1);
\draw [-] (g) to[out=270,in=55] (n1);
\end{tikzpicture}
\ \raisebox{45pt}{$=$}\
\begin{tikzpicture}
  \begin{pgfonlayer}{nodelayer}
    \node [style=map] (0) at (-2, 3) {$\scriptstyle \rst{f'}$};
    \node [style=map] (1) at (-2, 2.25) {$\scriptstyle f$};
    \node [style=delta] (2) at (-1.25, 4) {};
    \node [style=nabla] (3) at (-1.25, 1.25) {};
    \node [style=map] (4) at (-0.5, 3) {$\scriptstyle \rst{g'}$};
    \node [style=map] (5) at (-0.5, 2.25) {$\scriptstyle g$};
    \end{pgfonlayer}
    \begin{pgfonlayer}{edgelayer}
      \draw (0) to (1);
      \draw (1) to[out=270,in=125] (3);
      \draw (5) to[out=270,in=55] (3);
      \draw (4) to (5);
      \draw (2) to[out=305,in=90] (4);
      \draw (2) to[out=235,in=90] (0);
      \draw (2) to (-1.25,4.5);
      \draw (3) to (-1.25,.75);
      \end{pgfonlayer}
\end{tikzpicture}
\ \raisebox{45pt}{$=$}\
\begin{tikzpicture}
\path
node at (0,3) (start) {}
node at (0,2.5) [map] (fg) {$\scriptstyle \rst{f'}\,\rst{g'}$}
node at (0,2) [delta] (d) {}
node at (-.5,1) [map] (f) {$\scriptstyle f$}
node at (.5,1) [map] (g) {$\scriptstyle g$}
 node at (0,0) [nabla] (n1) {};
\draw [] (d) to (fg);
\draw [] (start) to (fg);
\draw [] (n1) to (0,-0.5);
\draw [] (d) to[out=305,in=90] (g);
\draw [] (d) to[out=235,in=90] (f);
\draw [-] (f) to[out=270,in=125] (n1);
\draw [-] (g) to[out=270,in=55] (n1);
\end{tikzpicture}
\ \raisebox{45pt}{$=$}\
\raisebox{25pt}{
\begin{tikzpicture}
\path
node at (0,1.5) (start) {}
node at (0,1) [map] (fg) {$\scriptstyle \rst{f'}\,\rst{g'}$}
node at (0,.5) [map] (z) {$\scriptstyle 0$};
\draw [] (z) to (fg);
\draw [] (start) to (fg);
\draw [] (z) to (0,0);
\end{tikzpicture}
}
\ \raisebox{45pt}{$=0$}
\]
\axiom{Dis}{4}: $f\cdperp g$ implies $g \cdperp f$.\\
This follows directly from the co-commutativity of $\Delta$.\\
\axiom{Dis}{5}: $f\cdperp g$ implies $h f \cdperp h g$.\\
This follows directly from the naturality of $\Delta$.\\
\axiom{Dis}{7}: $\rst{f}\cdperp \rst{g},\ \rg{h}\cdperp \rg{k}$ implies $f h \cdperp g k$.\\
\[
\begin{tikzpicture}
\path node at (0,0) [nabla] (n1) {}
node at (0,2.5) (start) {}
node at (-.5,1) [map] (fh) {$\scriptstyle f h$}
node at (.5,1) [map] (gk) {$\scriptstyle g k$}
node at (0,2) [delta] (d) {};
\draw [] (d) to (start);
\draw [] (n1) to (0,-0.5);
\draw [] (d) to[out=305,in=90] (gk);
\draw [] (d) to[out=235,in=90] (fh);
\draw [-] (fh) to[out=270,in=125] (n1);
\draw [-] (gk) to[out=270,in=55] (n1);
\end{tikzpicture}
\ \raisebox{45pt}{$=$}\
\begin{tikzpicture}
\path node at (0,0) [nabla] (n1) {}
node at (0,2.5) (start) {}
node at (-.5,1.5) [map] (rf) {$\scriptstyle \rst{f}$}
node at (-.5,1) [map] (fh) {$\scriptstyle f h$}
node at (-.5,.5) [map] (rngh) {$\scriptstyle \rg{h}$}
node at (.5,1.5) [map] (rg) {$\scriptstyle \rst{g}$}
node at (.5,1) [map] (gk) {$\scriptstyle g k$}
node at (.5,.5) [map] (rngk) {$\scriptstyle \rg{k}$}
node at (0,2) [delta] (d) {};
\draw [] (d) to (start);
\draw [] (n1) to (0,-0.5);
\draw [] (d) to[out=305,in=90] (rg);
\draw [] (d) to[out=235,in=90] (rf);
\draw (rf) to (fh);
\draw (fh) to (rngh);
\draw (rg) to (gk);
\draw (gk) to (rngk);
\draw [-] (rngh) to[out=270,in=125] (n1);
\draw [-] (rngk) to[out=270,in=55] (n1);
\end{tikzpicture}
\ \raisebox{45pt}{$=$}\
\begin{tikzpicture}
\path node at (0,0) [nabla] (n1) {}
node at (0,2.5) (start) {}
node at (-.5,1) [map] (fh) {$\scriptstyle f h$}
node at (-.5,.5) [map] (rnghrngk) {$\scriptstyle \rg{h}\rg{k}$}
node at (.5,1.5) [map] (rfrg) {$\scriptstyle \rst{f}\rst{g}$}
node at (.5,1) [map] (gk) {$\scriptstyle g k$}
node at (0,2) [delta] (d) {};
\draw [] (d) to (start);
\draw [] (n1) to (0,-0.5);
\draw [] (d) to[out=305,in=90] (rfrg);
\draw [] (d) to[out=235,in=90] (fh);
\draw (fh) to (rnghrngk);
\draw (rfrg) to (gk);
\draw [-] (rnghrngk) to[out=270,in=125] (n1);
\draw [-] (gk) to[out=270,in=55] (n1);
\end{tikzpicture}
\ \raisebox{45pt}{$=$}\
\begin{tikzpicture}
\path node at (0,0) [nabla] (n1) {}
node at (0,2.5) (start) {}
node at (-.5,1) [map] (fh) {$\scriptstyle f h$}
node at (-.5,.5) [map] (rnghrngk) {$\scriptstyle 0$}
node at (.5,1.5) [map] (rfrg) {$\scriptstyle 0$}
node at (.5,1) [map] (gk) {$\scriptstyle g k$}
node at (0,2) [delta] (d) {};
\draw [] (d) to (start);
\draw [] (n1) to (0,-0.5);
\draw [] (d) to[out=305,in=90] (rfrg);
\draw [] (d) to[out=235,in=90] (fh);
\draw (fh) to (rnghrngk);
\draw (rfrg) to (gk);
\draw [-] (rnghrngk) to[out=270,in=125] (n1);
\draw [-] (gk) to[out=270,in=55] (n1);
\end{tikzpicture}
\ \raisebox{45pt}{$=$}\
\begin{tikzpicture}
\path node at (0,0) [nabla] (n1) {}
node at (0,2.5) (start) {}
node at (-.5,1) [map] (fh) {$\scriptstyle 0$}
node at (.5,1) [map] (gk) {$\scriptstyle 0$}
node at (0,2) [delta] (d) {};
\draw [] (d) to (start);
\draw [] (n1) to (0,-0.5);
\draw [] (d) to[out=305,in=90] (gk);
\draw [] (d) to[out=235,in=90] (fh);
\draw [-] (fh) to[out=270,in=125] (n1);
\draw [-] (gk) to[out=270,in=55] (n1);
\end{tikzpicture}
\ \raisebox{45pt}{$= 0.$}\
\]
\end{proof}

%%% Local Variables:
%%% mode: latex
%%% TeX-master: "../../phd-thesis"
%%% End:


% section disjointness_in_an_inverse_category (end)
\section{Inverse sum categories} % (fold)
\label{sec:inverse_sum_categories}

\subsection{Inverse sums} % (fold)
\label{sub:inverse_sums}

\begin{definition}\label{def:inverse_sum}
  In an inverse category with disjoint joins, an object $X$ is the \emph{inverse sum} of $A$ and
  $B$ when there exist maps $i_1,\ i_2,\ \xa,\ \xb$ such that:
  \begin{enumerate}[{(}i{)}]
    \item $i_1$ and $i_2$ are monic;
    \item $i_1 : A \to X$, $i_2: B \to X$, $\xa: X \to A$ and $\xb: X \to B$.
    \item $\inv{i_1} = \xa$ and $\inv{i_2} = \xb$.
    \item $\inv{i_1}i_1 \perp \inv{i_2}i_2$ and $\inv{i_1}i_1 \djoin \inv{i_2}i_2 = 1_X$.
  \end{enumerate}
  $i_1$ and $i_2$ will be referred to as the \emph{injection} maps of the inverse sum.
\end{definition}

\begin{lemma}\label{lem:inverse_sum_is_unique}
  The inverse sum $X$ of $A$ and $B$ is unique up to isomorphism.
\end{lemma}
\begin{proof}
  Assume we have two inverse sums over $A$ and $B$:
  \[
    \xymatrix{
      A\ar[r]^{i_1} &X\ar@/^9pt/[l]^{x_0} \ar@/_9pt/[r]_{x_1} & B \ar[l]_{i_2}
    }
    \qquad  \text{ and  }\qquad
    \xymatrix{
      A\ar[r]^{\jay_1} &Y\ar@/^9pt/[l]^{y_1} \ar@/_9pt/[r]_{y_2} & B \ar[l]_{\jay_2}
    }.
  \]
  We will show that the map $x_1 \jay_1 \djoin x_2 \jay_2 : X \to Y$ is an isomorphism.

  Note by the fact that $i_2$ is monic, we may conclude from the definition that
  $0 = \rst{x_1 i_1 x_2}$ and therefore $0 = x_1 i_1 x_2$. Then, given that $x_1$ is the inverse
  of the monic $i_1$, we may calculate
  $0 = \rg{0} = \wrg{x_2 i_1 x_2} = \rst{\inv{x_2}\, \inv{i_1}\, i_1} =
  \rst{\inv{x_2}\, \inv{i_1}} = \wrg{i_1 x_2}$. From this we see $i_1 x_2 = 0$. Similarly, we have
  $i_2 x_2 = 0$, $j_1 y_2 =0$ and $j_2 y_1 = 0$.

  Next, by Lemma~\ref{lem:disjointness_various}, we know that $\rst{\xa}\cdperp\rst{\xb}$ as both
  $i_1$ and $i_2$ are monic. By the same lemma, $\rg{\jay_1} \cdperp \rg{\jay_2}$ as $y_1, y_2$
  are the inverses of monic maps.  Then, from \axiom{Dis}{7}, we have
  $x_2 \jay_1 \cdperp x_2 \jay_2$, hence we may form $x_2 \jay_1 \djoin x_2 \jay_2 : X \to Y$.

  Similarly, we may form the map $y_1 i_1 \djoin y_2 i_2 : Y \to X$. Computing their composition:
  \begin{align*}
    (x_2 \jay_1 \djoin x_2 \jay_2)(y_1 i_1 \djoin y_2 i_2)
      &= (x_2 \jay_1 (y_1 i_1 \djoin y_2 i_2))\djoin (x_2 \jay_2(y_1 i_1 \djoin y_2 i_2))\\
      &= x_2 \jay_1 y_1 i_1 \djoin x_2 \jay_1 y_2 i_2 \djoin x_2 \jay_2 y_1 i_1 \djoin x_2
        \jay_2 y_2 i_2 \\
      &= x_2\, 1\, i_1 \djoin x_2\, 0\, i_2 \djoin x_2\, 0\, i_1 \djoin x_2\, 1\, i_2\\
      &= x_2 i_1 \djoin x_2 i_2 = 1.
  \end{align*}
  Computing the other direction,
  \begin{align*}
    (y_1 i_1 \djoin y_2 i_2)(x_2 \jay_1 \djoin x_2 \jay_2)
      &= (y_1 i_1 (x_2 \jay_1 \djoin x_2 \jay_2) )\djoin (y_2 i_2(x_2 \jay_1 \djoin x_2 \jay_2))\\
      &= y_1 i_1 x_2 \jay_1 \djoin y_1 i_1 x_2 \jay_2 \djoin y_2 i_2 x_2 \jay_1
        \djoin y_2 i_2 x_2 \jay_2\\
      &= y_1\, 1\, \jay_1 \djoin y_1\, 0\, \jay_2 \djoin y_2\, 0\, \jay_1 \djoin y_2\, 1\, \jay_2\\
      &= y_1 \jay_1 \djoin y_2 \jay_2 = 1.\\
  \end{align*}
  This shows that the map between any two inverse sums over the same two objects is an isomorphism.
\end{proof}

\begin{lemma}\label{lem:all_maps_to_inverse_sum_are_disjoint}
  Suppose $X$ is the inverse sum of $A$ and $B$ in the inverse category \X. Then for all maps
  $f:C \to A$ and $g: C \to B$, the composition with the injections is disjoint, that is,
  $f i_1 \perp g i_2$. (This is not right - need further thought...)
\end{lemma}
\begin{proof}
  First note $f i_1 = f i_1 \wrg{i_1} = f i_1 \inv{i_1} i_1$ and similarly, $g i_2 = g i_2
  \inv{i_2} i_2$.
\end{proof}

\begin{definition}\label{def:inverse_sum_category}
  Suppose \X is an inverse category with disjoint joins $\djoin$ based on a disjointness
  relation $\cdperp$ and a restriction zero. If every pair of objects has an inverse sum as in
  Definition~\ref{def:inverse_sum}, we call the category an \emph{inverse sum} category. For any
  two objects $A,B$ in \X, we write their inverse sum as $\invsum{A}{B}$.
\end{definition}

\begin{lemma}\label{lem:tensor_disjoint_sum_cats_are_inverse_sum_categories}
  Let \X be an inverse category with a restriction 0 and a disjoint sum tensor $\+$. Then $\X$ is
  an inverse sum category.
\end{lemma}
\begin{proof}
  We claim that setting $i_i = \cp{i}$ and $x_i = \icp{i}$ and setting $X = A\+B$ produces inverse
  sums in \X and show this satisfies the four conditions of Definition~\ref{def:inverse_sum}.
  \begin{enumerate}[{(}i{)}]
    \item From Lemma~\ref{lem:tensor_identities}, we know that $\cpa$ and $\cpb$ are monic maps.
    \item $\cpa : A \to A\+B$, $\cpb: B \to A\+B$, $\icpa: A\+B \to A$ and $\icpb: A\+B \to B$.
    \item $\inv{\cpa} = \icpa$ and $\inv{\cpb} = \icpb$.
    \item $\inv{i_1}i_1 = 1\+ 0 \tperp 0\+1 = \inv{i_2}i_2$ as
      $1\+0 \tjdown 0\+1 = (\inv{\upr}\+\inv{\upl})$ and
      $1\+0 \tjup 0\+1 = (\icpa\+\icpb)$. For their join,
      $(1\+0)\tjoin (0\+1) = (\inv{\upr}\+\inv{\upl})(\icpa\+\icpb) =
      \inv{\upr}\icpa \+ \inv{\upl}\icpb = 1\+1 = 1$.
  \end{enumerate}
\end{proof}

\begin{lemma}\label{lem:inverse_sums_have_identity_zero}
  If $A$ is an object in \X, an inverse sum category, then $A+0$ is isomorphic to $A$.
\end{lemma}
\begin{proof}
  We write the inverse sum diagram:
  \[
    \xymatrix{
      A\ar[r]^{1} &A\ar@/^9pt/[l]^{1} \ar@/_9pt/[r]_{0} & 0 \ar[l]_{0}.
    }
  \]
\end{proof}
\begin{lemma}\label{lem:functor_preserving_joins_preserves_inverse_sums}
  Suppose $\X$ is an inverse sum category and $\Y$ is an inverse category with a restriction zero.
  Further, suppose $F:\X \to \Y$ is a restriction functor which preserves disjoint joins. Then,
  $F$ preserves inverse sums.
\end{lemma}
\begin{proof}
  In \X, consider the inverse sum over $A$ and $B$,
  \[
    \xymatrix{
      A\ar[r]^{i_1} &X\ar@/^9pt/[l]^{x_0} \ar@/_9pt/[r]_{x_1} & B \ar[l]_{i_2}
    }.
  \]
  The functor $F$ maps this as follows:
  \[
    \xymatrix@R+10pt@C+10pt{
      F(A)\ar[r]^{F(i_1)} &F(X)\ar@/^13pt/[l]^{F(x_0)} \ar@/_13pt/[r]_{F(x_1)} & F(B)\ar[l]_{F(i_2)}
    }.
  \]
  As $F$ is a restriction functor, we immediately have $F(x_0) = F(\inv{i_1}) = \inv{F(i_1)}$ and
  $F(x_1) = \inv{F(i_2)}$. Since $F$ preserves the disjoint join, we also have
  $\inv{F(i_1)}F(i_1) \cdperp\inv{F(i_2)}F(i_2)$ and
  $\inv{F(i_1)}F(i_1) \djoin \inv{F(i_2)}F(i_2) = 1$.

  Finally, as $F$ is a restriction functor, it preserves monics, hence $F(i_1)$ and $F(i_2)$ are
  both monic and therefore $F(X)$ is the inverse sum of $F(A)$ and $F(B)$.

\end{proof}
\begin{lemma}\label{lem:inverse_sum_maps_are_perp}
  Given \X an inverse sum category and maps $f:A \to C$ and $g:B\to D$ in \X. Then
  $\inv{i_1} f i_1 \cdperp \inv{i_2} g i_2 : \invsum{A}{B}\to \invsum{A}{B}$.
\end{lemma}
\begin{proof}
  Note that $\rst{\inv{i_1} f i_1} = \rst{\inv{i_1} f} \le \rst{\inv{i_1}}$ and similarly
  $\rst{\inv{i_2} g i_2} \le \rst{\inv{i_2}}$. Then, by \axiom{Dis}{3}, we have
  $\rst{\inv{i_1} f i_1} \cdperp \rst{\inv{i_2} g i_2}$.
  As $\wrg{\inv{i_1} f i_1} \le \wrg{i_1}$ and  $\wrg{\inv{i_2} g i_2} \le \wrg{i_2}$, we
  have $\wrg{\inv{i_1} f i_1} \cdperp \wrg{\inv{i_2} g i_2}$ and by
  Lemma~\ref{lem:disjointness_various}, this means  $\inv{i_1} f i_1\cdperp \inv{i_2} g i_2$.
\end{proof}


\begin{lemma}\label{lem:inverse_sums_have_unique_maps_out}
  Given $\X$ is an inverse sum category. Denote the inverse sum of objects $A,B$ of $\X$ by
  $\invsum{A}{B}$. Then for objects $A,B$ and $X$ with maps $f:A\to X$ and $g:B\to X$ such that
  $\rg{f} \cdperp \rg{g}$, there exists a unique map $h$ making the following diagram commute.
  \[
    \xymatrix@R+10pt@C+28pt {
      A \ar[dr]^{f} \ar[d]_{i_1}
        \\
      \invsum{A}{B} \ar@{.>}[r]^{h}
        & X\text{.}\\
      B \ar[ur]_{g} \ar[u]_{i_2}
    }
  \]
  We use the notation $f\rgp g$ for the unique map $h$.
\end{lemma}
\begin{proof}
  As $\rg{f} \cdperp \rg{g}$ and $\rst{\inv{i_1}} \cdperp \rst{\inv{i_2}}$ we may form the map $h'
  = \inv{i_1}f \djoin \inv{i_2}g$. By its construction, $h'$ is a map from $\invsum{A}{B}$ to $X$
  which makes the diagram commute. Suppose now that both maps $v$ and $w$ are such maps. Then we
  have
  \[
    (\inv{i_1}i_1) v = (\inv{i_1}i_1) w \quad\text{ and }\quad(\inv{i_2}i_2) v = (\inv{i_2}i_2) w.
  \]
  As $\inv{i_1}i_1 \cdperp \inv{i_2}i_2$, by Lemmas~\ref{lem:disjointness_various} and
  \ref{lem:join_is_associative_and_commutative_monoid}, we know that $(\inv{i_1}i_1) v \cdperp
  (\inv{i_2}i_2) v$ and $(\inv{i_1}i_1) w \cdperp (\inv{i_2}i_2) w$ allowing us to form their
  respective disjoint joins. As the disjoint joins of equal maps remains equal, we have
  \begin{align*}
    (\inv{i_1}i_1) v \djoin (\inv{i_2}i_2) v &=     (\inv{i_1}i_1) w \djoin (\inv{i_2}i_2) w \\
    (\inv{i_1}i_1 \djoin \inv{i_2}i_2 )v &=     (\inv{i_1}i_1 \djoin \inv{i_2}i_2) w \\
    (1 )v &=     (1) w \\
        v &=  w.
  \end{align*}
\end{proof}

\begin{corollary}\label{cor:inverse_sums_have_unique_maps_in}
  Given $\X$ is an inverse sum category. Then for objects $A,B$ and $X$ with maps $f:X\to A$
  and $g:X\to B$ such that
  $\rst{f} \cdperp \rst{g}$, there exists a unique map $h$ making the following diagram commute.
  \[
    \xymatrix@R+10pt@C+28pt {
      &A \ar[d]_{i_1}
        \\
      X   \ar[dr]_{g} \ar[ur]^{f}  \ar@{.>}[r]^{h} & \invsum{A}{B}\\
      & B \ar[u]_{i_2}\text{.}
    }
  \]
  We use the notation $f\rstp g$ for the unique map $h$.
\end{corollary}
\begin{proof}
  This is simply the dual of Lemma~\ref{lem:inverse_sums_have_unique_maps_out}. The unique map $h$
  in this case is $f i_1 \djoin g i_2$.
\end{proof}

\begin{corollary}\label{cor:inverse_sums_have_unique_maps}
  Suppose $\X$ is an inverse sum category. Then for objects $A,B,C$ and $D$ with maps $f:A\to C$
  and $g:B\to D$, there exists a unique map $h$ making the following diagram commute.
  \[
    \xymatrix@R+10pt@C+28pt {
      A \ar[r]^{f} \ar[d]_{i_1}
        & C \ar[d]_{i_1} \\
      \invsum{A}{B} \ar@{.>}[r]^{h}
        & \invsum{C}{D}\\
      B \ar[r]_{g} \ar[u]_{i_2}
        & D \ar[u]_{i_2}\text{.}
    }
  \]
  We use the notation $\invsum{f}{g}$ for the map $h$.
\end{corollary}
\begin{proof}
  This follows directly from Lemma~\ref{lem:inverse_sums_have_unique_maps_out} by setting
  $X = \invsum{C}{D}$. The unique map in this case is $\inv{i_1} f i_1 \djoin \inv{i_2} g i_2$.
\end{proof}
\begin{lemma}\label{lem:functor_preserving_inverse_sums_preserves_joins}
  Suppose $\X$ and $\Y$ are inverse sum categories and $F:\X \to \Y$ is a restriction functor which
  preserves inverse sums. Then, $F$ preserves disjoint joins.
\end{lemma}
\begin{proof}
  By stating that $F$ preserves the inverse sum, we mean it preserves diagrams derived via the
  properties of the inverse sum, and specifically, it will preserve the diagrams of
  Lemma~\ref{lem:inverse_sums_have_unique_maps_out} and
  Corollaries~\ref{cor:inverse_sums_have_unique_maps_in} and
  \ref{cor:inverse_sums_have_unique_maps}.

  Suppose we are given $f, g: A \to B$ with $f \cdperp g$. In the inverse sum category, we know that
  $f \djoin g = (\rst{f}i_1 \djoin \rst{g} i_2) (\inv{i_1} f i_1 \djoin \inv{i_2} g i_2)
  (\inv{i_1} \rg{f} \djoin \inv{i_2} \rg{g})$, as this follows by:
  \begin{enumerate}
    \item Apply Corollary~\ref{cor:inverse_sums_have_unique_maps_in} to $\rst{f}$ and $\rst{g}$;
    \item then apply Corollary~\ref{cor:inverse_sums_have_unique_maps} to $f, g$;
    \item finally apply Lemma~\ref{lem:inverse_sums_have_unique_maps_out} to $\rg{f}, \rg{g}$.
  \end{enumerate}

  Thus, we have that $f\djoin g =
  (\rst{f} \rstp \rst{g})(\invsum{f}{g})(\rg{f} \rgp \rg{g})$. As $F$ preserves
  the inverse sum, this gives us:
  \begin{align*}
    F(f\djoin g) &= F(\rst{f}\rstp \rst{g})F(\invsum{f}{g})F(\rg{f} \rgp \rg{g})\\
    &=(F(\rst{f})\rstp F(\rst{g}))(\invsum{F(f)}{F(g)})(F(\rg{f}) \rgp F(\rg{g}))\\
    &=(\rst{F(f)}\rstp \rst{F(g)})(\invsum{F(f)}{F(g)})(\wrg{F(f)} \rgp \wrg{F(g)})\\
    &=F(f) \djoin F(g).
  \end{align*}

  The last line is due to \Y being an inverse sum category as well.

\end{proof}


% subsection inverse_sums (end)

\subsection{Inverse sum tensor} % (fold)
\label{sub:inverse_sum_tensor}
\begin{definition}\label{def:inverse_sum_tensor}
  An \emph{inverse sum tensor} in an inverse category \X with disjoint joins $\djoin$ based
  on a disjointness relation $\cdperp$ and a restriction zero
  is given by a tensor combined with two restriction monics, $\cpa$ and $\cpb$. The data for the
  tensor is:
 \begin{align*}
    \_ \+ \_ &: \X \times \X \to \X\ \ \text{(a restriction functor preserving disjoint joins)}\\
    0 &: \boldsymbol{1}\to \X \\
    \upl &: 0 \+ A \to A\\
    \upr &: A \+ 0 \to A\\
    a_\+ &: (A \+ B) \+ C \to A \+ (B \+ C) \\
    c_\+ &: A \+ B \to B \+ A\\
    \cpa &: A \to A \+ B\\
    \cpb &: B \to A \+ B
  \end{align*}
  where $\upl, \upr, a_\+, c_\+$ are all isomorphisms and the standard symmetric monoidal
  equations and coherence diagrams hold. The unit of the tensor, $0: \boldsymbol{1}\to \X$, is the
  restriction zero of the category. We specifically note that preserving disjoint joins means the
  tensor obeys the following two equations:
  \begin{align}
    f \cdperp g, \ h \cdperp k&\text{ implies } f\+ h \cdperp g \+ k
    \label{eq:invsum_preserve_perp}\\
    f \cdperp g, \ h \cdperp k &\text{ implies } (f\djoin g)\+(h\djoin k) = (f \+h )\djoin (g\+k).
    \label{eq:invsum_preserve_join}
  \end{align}

\end{definition}


\begin{lemma}\label{lem:a_disjoint_sum_tensor_is_an_inverse_sum_tensor}
  Given an inverse category \X with a disjoint sum tensor $\+$ as in
  Definition~\ref{def:disjoint_sum_tensor}, then $\+$ is an inverse sum tensor.
\end{lemma}
\begin{proof}
  From the data of the disjoint sum tensor, the only thing remaining to show is that the tensor
  preserves the disjoint join.

  Suppose we have $f \tperp g$ and $\ h \tperp k$.
  From Lemma~\ref{lem:properties_of_tjdown_and_tjup}, item~\ref{lemitem:l_r_preserve_tensor}, we
  know both $(f\+h)\tjdown(g\+k)$ and $(f\+h)\tjup(g\+k)$ exist, hence $(f\+h)\tperp(g\+k)$.
  This shows condition~\eqref{eq:invsum_preserve_perp}.

  For condition~\eqref{eq:invsum_preserve_join}, we compute from the right hand side:
  \begin{align*}
    (f\+h)\tjoin(g\+k) &= (f\+h)\tjdown (g\+k) \wrg{(f\+h)}\tjup \wrg{(g\+k)}\\
    &= \left( (f\tjdown g)\+ (h \tjdown k) \right)
       \left( (\rg{f}\+\rg{h}) \tjup (\rg{g}\+\rg{k}) \right) \\
    &= \left( (f\tjdown g)\+ (h \tjdown k) \right)
       \left( (\rg{f}\tjup \rg{g}) \+ (\rg{h}\tjup\rg{k}) \right) \\
    &= \left( (f\tjdown g)(\rg{f}\tjup \rg{g})\right) \+
       \left((h \tjdown k) (\rg{h}\tjup\rg{k}) \right) \\
    &=  (f\tjoin g)\+(h \tjoin k).
  \end{align*}
  The second and third lines above again use Lemma~\ref{lem:properties_of_tjdown_and_tjup},
  item~\ref{lemitem:l_r_preserve_tensor}.

\end{proof}

\begin{lemma}\label{lem:an_inverse_sum_tensor_gives_inverse_sums}
  If $\+$ is an inverse sum tensor in the inverse category \X, then $A\+B \cong A+B$, an inverse
  sum of $A$ and $B$.
\end{lemma}
\begin{proof}
  As $\+$ is a restriction functor from $\X\times\X$ to $\X$, this actually follows immediately
  from Lemma~\ref{lem:functor_preserving_joins_preserves_inverse_sums}. It may also be proven
  directly:

  Draw the inverse sum diagram:
  \[
    \xymatrix @C+60pt{
      A\ar[r]^{i_1=\inv{\upr} (1\+0)}
        & A\+B \ar@/^15pt/[l]^{x_0=(1\+0)\upr} \ar@/_15pt/[r]_{x_1=(0\+1)\upl}
          & B \ar[l]_{i_2=\inv{\upl} (0\+1)}
    }.
  \]
  Therefore, we have $\inv{i_1}i_1 =
  (1\+0)\upr \inv{\upr} (1\+0) = (1\+0)(1\+0) = (1\+0)$. Similarly, $\inv{i_2} i_2 = (0\+1)$.
  Since $0\cdperp 1$, we have $\inv{i_1}i_1 \cdperp \inv{i_2} i_2$.

  By the functorality of $\+$ and that it preserves disjoint joins, we have $(1\+0) \djoin (0 \+1)
  = (1\djoin 0 ) \+ (0\djoin 1) = 1 \+ 1 = 1_{A\+B}$. Hence $A\+ B$ is an inverse sum of $A$ and
  $B$ and by Lemma~\ref{lem:inverse_sum_is_unique} it is isomorphic to $A+B$.
\end{proof}

Conversely, we can show that given a tensor which produces inverse sums, that tensor will
be an inverse sum tensor.

\begin{lemma}\label{lem:inverse_sums_give_inverse_sum_tensors}
  Suppose we have an inverse category $\X$ with restriction zero, a disjointness relation $\cdperp$, a
  disjoint join $\djoin$ and a symmetric monoidal tensor $\+$, with natural restriction monics
  $\cpa: A\to A\+B$ and $\cpb: B \to A\+B$. Further suppose that $A\+B$ is an inverse sum under $\cpa$ and
  $\cpb$. When $f,g :A \to B$ and $h,k:C\to D$ with $f \cdperp g$ and $h \cdperp k$, then $f
  \+h \cdperp g\+k$ and $(f\+h)\djoin (g\+k) = (f\djoin g)\+ (h\djoin k)$.
\end{lemma}
\begin{proof}
  Similarly, this follows immediately from
  Lemma~\ref{lem:functor_preserving_inverse_sums_preserves_joins}. We show it directly below:
  \begin{equation}
    \xymatrix@R+10pt@C+28pt {
      A \ar[r]^{\rst{f}} \ar[d]_{\cpa} \ar@/^20pt/[rr]^f
        & A \ar[d]_{\cpa} \ar[r]^{f\djoin g} & B \ar[d]_{\cpa}\\
      A\+C \ar[r]^{\rst{f}\+\rst{h}}_{(1)}
        & A\+C  \ar[r]^{f\djoin g \+ h\djoin k}_{(2)} & B \+D \\
      C \ar[r]_{\rst{h}} \ar[u]_{\cpb} \ar@/_20pt/[rr]_h
        & C \ar[u]_{\cpb} \ar[r]_{h\djoin k} & D \ar[u]_{\cpb}
    }\label{dia:sum_preserve_joins1}
  \end{equation}
  Consider $\inv{\cpa} \rst{f}\cpa$. As this is idempotent and we are in an inverse category, we
  know that $\inv{\cpa} \rst{f}\cpa = \rst{\inv{\cpa} \rst{f}\cpa} = \rst{\inv{\cpa}\rst{f}} =
  \wrg{\rst{f}\cpa}$. Similarly, $\inv{\cpb}\rst{h}\cpb = \wrg{\rst{h}\cpb}$. By \axiom{Dis}{5} and
  \axiom{Dis}{6}, we know that $\wrg{\rst{f}\cpa} \cdperp \wrg{\rst{g}\cpa}$ and $\wrg{\rst{h}\cpb}
  \cdperp \wrg{\rst{k}\cpb}$. As shown in the proof of
  Lemma~\ref{lem:inverse_sum_is_unique}, we know $\wrg{\cpa}\cdperp \wrg{\cpb}$. Hence, by
  \axiom{Dis}{3}, we have $\wrg{\rst{x}\cpa} \cdperp \wrg{\rst{y}\cpb}$ for any maps $x,y$.

  Hence, we can form the map $\wrg{\rst{f}\cpa} \djoin \wrg{\rst{h}\cpb}$. Referring to the
  Diagram~\ref{dia:sum_preserve_joins1}, by Corollary~\ref{cor:inverse_sums_have_unique_maps} there is
  a unique map at location $(1)$ which makes the diagram commute --- currently given as
  $\rst{f}\+\rst{h}$. The map $\wrg{\rst{f}\cpa} \djoin \wrg{\rst{h}\cpb}$ also makes the diagram commute.
  Hence, we have $\wrg{\rst{f}\cpa} \djoin \wrg{\rst{h}\cpb} = \rst{f}\+\rst{h}$. Similarly,
  $\wrg{\rst{g}\cpa} \djoin \wrg{\rst{k}\cpb} = \rst{g}\+\rst{k}$. By
  Lemma~\ref{lem:disjoint_joins_only_when_all_disjoint}, this means $\rst{f\+h} \cdperp \rst{g\+k}$.

  Using a similar argument based on the diagram

  \begin{equation}
    \xymatrix@R+10pt@C+28pt {
      A \ar[r]^{f\djoin g} \ar[d]_{\cpa} \ar@/^20pt/[rr]^f
        & B \ar[r]^{f} \ar[d]_{\cpa}  & B \ar[d]_{\cpa}\\
      A\+C  \ar[r]^{f\djoin g \+ h\djoin k}_{(3)}
        & B\+D \ar[r]^{\rg{f}\+\rg{h}}_{(4)}   & B \+D \\
      C  \ar[r]_{h\djoin k} \ar[u]_{\cpb} \ar@/_20pt/[rr]_h
        & D\ar[r]_{\rg{h}} \ar[u]_{\cpb}  & D \ar[u]_{\cpb}
    }\label{dia:sum_preserve_joins2}
  \end{equation}
  we can show $\wrg{f\+h} \cdperp \wrg{g\+k}$ and therefore $f\+h \cdperp g\+k$.

  This allows us to form the map $(f\+h) \djoin (g\+k)$. Once again, as the objects are inverse
  sums, the map at $(3)$ in Diagram~\ref{dia:sum_preserve_joins2} is unique.
  However, we see that both $f\djoin g \+ h\djoin k$ and $(f\+h) \djoin (g\+k)$ fulfill this
  requirement and hence they are equal.
\end{proof}


\begin{definition}\label{def:inverse_sum_tensor_category}
  An inverse category $\X$ with restriction zero, a disjointness relation $\cdperp$, a disjoint
  join $\djoin$ and an inverse sum tensor $\+$ is called an \emph{inverse sum tensor category}.
\end{definition}

\begin{corollary}\label{cor:inverse_sum_of_maps_is_disjoint_join}
  In an inverse sum tensor category, $f \+g$ is given by $\inv{i_1} f i_1 \djoin \inv{i_2} g i_2$.
\end{corollary}
\begin{proof}
  Recall that in the proof of Lemma~\ref{lem:inverse_sum_is_unique} that we showed $\rst{\inv{i_1}}
  \cdperp \rst{\inv{i_2}}$ and $\wrg{i_1}\perp \wrg{i_2}$. Since $\rst{x f} \le \rst{x}$, by
  \axiom{Dis}{3} and \axiom{Dis}{7}, we know that $\inv{i_1} f i_1 \perp \inv{i_2} g i_2$ and we
  can therefore form the disjoint join.
\end{proof}


\subsection{Matrices} % (fold)
\label{sub:matrix_representation}
In this sub-section, we will show that when given an inverse  category \X with a disjoint sum
tensor, one can define a
matrix category based on $\X$. We will call this category \imatx. Furthermore, we will
show that \imatx is an inverse category and that $\X$ embeds within this category.

\begin{definition}\label{def:inverse_sum_matrix}
  Given \X is an inverse category with a disjoint sum tensor. Then an \emph{inverse sum matrix} in \X
  is a matrix of maps $[f_{i j}]$ where $i \in \{1,\ldots, n\}$ and $j \in \{1,\ldots, m\}$ with
  $f_{i\jay}:A_i \to B_\jay$ which satisfy the two conditions:
  \begin{align}
    &\text{For each }i,\ \cdperp[f_{i \jay} \cp{\jay}]_{\jay=1,\ldots,m}
    \text{ where } \cp{\jay}: B_\jay \to B_1\+B_2\+\cdots\+B_m.
      \label{eq:inverse_sum_matrix_row_perp}\\
    &\text{For each }j,\ \cdperp[\inv{\cp{i}} f_{i \jay}]_{i=1,\ldots,n}
    \text{ where } \inv{\cp{i}} = \icp{i} : A_1\+A_2\+\cdots\+A_n \to A_i.
    \label{eq:inverse_sum_matrix_column_perp}
  \end{align}
\end{definition}

  In the above and following we will use the notation $\cp{i}$ for the $i^{\text{th}}$ injection
  map of the disjoint sum tensor, with $i$ starting at 1. This simply extends the notation
  introduced in Definition~\ref{def:disjointness_tensor}.

  We will show that this type of matrix corresponds to maps in the category \imatx, where
  composition is given by ``matrix multiplication'', where the operations of multiplication and
  addition are replaced with composition in \X and the disjoint join respectively.

\begin{definition}\label{def:inverse_matrix_category}
  Given an inverse category \X with a disjoint sum tensor, we define the \emph{inverse
  matrix category} of \X, \imatx, as follows:
  \rcategory{Non-empty lists of the objects of $\X$.}{
    Inverse sum matrices $[f_{i\jay}]: [A_i] \to [B_\jay]$. In such a matrix each individual map
    $f_{i\jay}:A_i \to B_{\jay}$ is a map in $\X$. For each $j$, $B_\jay$ is given by
    applying the map $\djoin_i \icp{i} f_{i\jay}$ to the object $\+_i A_i$}{
    The inverse sum matrix $I$.}{
    Given $[f_{i\jay}]: [A_i] \to [B_\jay]$ and $[g_{j k}] : [B_\jay] \to [C_k]$, then
    $[h_{i k}] = [f_{i\jay}][g_{\jay k}]: [A_i] \to [C_k]$ is defined as
    $h_{i k} = \djoinbig_\jay f_{i \jay}g_{\jay k}$.}{
    We set $\rst{[f_{i\jay}]}$ to be $[f'_{i\jay}]$ where $f'_{i\jay} = 0$ when $i\neq \jay$
    and $f'_{i i} = \djoin_\jay \rst{f_{i\jay}}$.}
  In the following, we will use the notation $\diag[d_1,d_2,\ldots,d_n]$ for a diagonal $n\times n$
  matrix with entries $[d_1,d_2,\ldots,d_n]$ and $\diag_j[d_j]$ for diagonal matrices where the
  $j,j$ entry is $d_j$.
  %TODO: Get the notation better
\end{definition}

\begin{lemma}\label{lem:inverse_matrix_category_is_a_restriction_category}
  When $\X$ is an inverse sum category, \imatx is a restriction category.
\end{lemma}
\begin{proof}
  We need to show the following:
  \begin{itemize}
    \item Composition is well defined and associative.
    \item The restriction is well defined.
  \end{itemize}

  \emph{Composition is well defined:} Consider $[h_{i k}] = [f_{i j}] [g_{j k}]$ where $[f_{i j}]:
  [A_1,\ldots, A_n] \to [B_1,\ldots, B_m]$ and $[g_{j k}]: [B_1,\ldots, B_m] \to [C_1,\ldots,
  C_\ell]$. By supposition, we know $h_{i k} = \djoinbig_\jay f_{i \jay} g_{\jay k}$. As each of
  the maps are inverse sum matrices, we know that $\cdperp [f_{i \jay} \cp{j}]$ and
  $\cdperp[\inv{\cp{j}}g_{\jay k}]$. Hence, for each $\jay$ we know the composition $f_{i \jay}
  \cp{j} \inv{\cp{j}}g_{\jay k} = f_{i \jay} g_{\jay k}$ is defined and from $A_i$ to $C_{k}$. By
  the stability and universality of $\djoin$, we know $h_{i k}$ exists and by the definition of
  $\djoin$, we have each $h_{i k}: A_i \to C_k$ and hence composition is well-defined.

  \emph{Associativity of composition.} We have
  \begin{align*}
    ([f_{i j}] [g_{j k}]) [h_{k \ell}]
        &= \left[(\djoinbig_j f_{i j}  g_{j k})\right] [h_{k \ell}]\\
        &=\left[ \djoinbig_k (\djoinbig_j f_{i j} g_{j k}) h_{k \ell}\right]\\
        &=\left[ \djoinbig_j f_{i j}  (\djoinbig_k g_{j k} h_{k \ell})\right]\\
        &=[ f_{i j} ]  ([g_{j k}] [h_{k \ell}]).
  \end{align*}

  \emph{The restriction axioms.}
  \begin{align*}
    \rone \quad& \rst{[f_{i\jay}]}[f_{i\jay}] =
      \begin{bmatrix}
        (\djoin_j \rst{f_{1\jay}}) f_{11} & \cdots & (\djoin_\jay \rst{f_{1 \jay}}) f_{1n}\\
        & \vdots & \\
        (\djoin_\jay \rst{f_{m\jay}}) f_{m1} & \cdots & (\djoin_\jay \rst{f_{m\jay}}) f_{m n}
      \end{bmatrix} = [f_{i\jay}].\\
  \end{align*}
  \begin{align*}
    \rtwo \quad & \rst{[f_{i\jay}]} \rst{g_{i\jay}} =  \rst{g_{i\jay}} \rst{[f_{i\jay}]}
      \text{ as diagonal matrices commute and }\djoin\text{ is also commutative.}\\
  \end{align*}
  \begin{align*}
    \rthree \quad & \rst{\rst{[f_i\jay]}[g_{\jay k}]} =
      \rst{\diag[\djoin_\jay \rst{f_{1\jay}},\ldots,\djoin_\jay \rst{f_{n \jay}}] [g_{\jay k}]}
      \\[10pt]
      & = \rst{\begin{bmatrix}
        \djoin_\jay \rst{f_{1\jay}} g_{11} & \ldots & \djoin_\jay \rst{f_{1 \jay}} g_{1 k}\\
        & \vdots & \\
        \djoin_\jay \rst{f_{n\jay}} g_{n 1} & \ldots & \djoin_\jay \rst{f_{n \jay}} g_{n k}\\
      \end{bmatrix}} \\[10pt]
      & = \diag[\djoin_k (\rst{\djoin_\jay (\rst{f_{1\jay}} g_{1k})}), \ldots,
                 \djoin_k (\rst{\djoin_\jay(\rst{f_{n\jay})} g_{n k}})]\\
      & = \diag[\djoin_k (\djoin_\jay(\rst{f_{1\jay}}) \rst{g_{1k}}), \ldots,
                 \djoin_k (\djoin_\jay(\rst{f_{n\jay}}) \rst{g_{n k}})]\\
      & = \diag[(\djoin_\jay(\rst{f_{1\jay}}) \djoin_k  \rst{g_{1k}}), \ldots,
                 (\djoin_\jay(\rst{f_{n\jay}}) \djoin_k  \rst{g_{n k}})]\\
      & = \rst{[f_i\jay]}\,\rst{[g_{\jay k}]}.\\
  \end{align*}
  \begin{align*}
      \rfour \quad & [f_{i\jay}]\rst{[g_{\jay k}]} =
        [f_{i\jay}]\diag_\jay[\djoin_k \rst{g_{\jay k}}]
        \\[10pt]
      & = \begin{bmatrix}
        f_{11}\djoin_k \rst{g_{1 k}} &\ldots & f_{1n}\djoin_k \rst{g_{n k}} \\
        & \vdots & \\
        f_{m1}\djoin_k \rst{g_{1 k}} &\ldots & f_{m n}\djoin_k \rst{g_{n k}}
      \end{bmatrix}\\[10pt]
      & = \begin{bmatrix}
        \djoin_k f_{11}\rst{g_{1 k}} &\ldots & \djoin_k f_{1n}\rst{g_{n k}} \\
        & \vdots & \\
        \djoin_k f_{m1}\rst{g_{1 k}} &\ldots & \djoin_k f_{m n}\rst{g_{n k}}
      \end{bmatrix}\\[10pt]
      & = \begin{bmatrix}
        \djoin_k \rst{f_{11} g_{1 k}} f_{11} &\ldots & \djoin_k \rst{f_{1n} g_{n k}} f_{1n}\\
        & \vdots & \\
        \djoin_k \rst{f_{m1} g_{1 k}}f_{m1} &\ldots & \djoin_k \rst{f_{m n} g_{n k}}f_{m n}
      \end{bmatrix}\\[10pt]
      & = \begin{bmatrix} % because the f_ij \perp fi1 etc.
        \djoin_\jay\djoin_k \rst{f_{1\jay} g_{\jay k}} f_{11} &\ldots &
          \djoin_\jay\djoin_k \rst{f_{1\jay} g_{\jay k}} f_{1n}\\
        & \vdots & \\
        \djoin_\jay\djoin_k \rst{f_{m\jay} g_{\jay k}}f_{m1} &\ldots &
          \djoin_\jay\djoin_k \rst{f_{m \jay} g_{\jay k}}f_{m n}
      \end{bmatrix}\\[10pt]
      &= \rst{[f_{i\jay}][g_{\jay k}]} [f_{i\jay}].
  \end{align*}
\end{proof}

Note that when $\X$ is an inverse category with a disjoint join, \imatx is also an inverse
category. The inverse of the map $f = [f_{i\jay}]$ is the map $\inv{f}:=[\inv{f_\jay i}]$. Recalling
that the rows and columns of $f$ are each disjoint, we see that the composition
$f \inv{f} = \diag_i [\djoin_j \rst{f_{i\jay}}] = \rst{f}$.

\begin{lemma}\label{lem:imat_has_restriction_zero}
  Given \X is an inverse restriction category with a restriction zero and a disjoint join,
  then \imatx has a restriction zero.
\end{lemma}
\begin{proof}
  The restriction zero in \imatx is the list $[0]$ where $0$ is the restriction zero in \X.

  For the object $A=[A_1,\ldots,A_n]$, the $0$ map is given by the $n\times 1$ matrix
  $[0,\ldots,0]$. The map from $0$ is given by the $1\times n$ matrix
  $\begin{bmatrix}0\\ \vdots\\ 0\end{bmatrix}$.

\end{proof}

\begin{lemma}\label{lem:imat_is_disjoint_tensor_category}
  Given \X is an inverse restriction category with a restriction zero, $0$, and a disjoint join,
  then the monoid $\+$ defined by list catenation of objects is a disjointness tensor.
\end{lemma}
\begin{proof}
  We first note the monoidal isomorphisms:
  \begin{align*}
    \upl : [0,A_1,A_2,\ldots,A_n] \to [A_1,A_2,\ldots,A_n] \qquad& \upl :=
    \begin{bmatrix}
      0&\cdots&0 \\
      & I_{n\times n}
    \end{bmatrix}\\
    \upr : [A_1,A_2,\ldots,A_n,0] \to [A_1,A_2,\ldots,A_n] \qquad & \upr :=
    \begin{bmatrix}
      & I_{n\times n} \\
      0&\cdots&0
    \end{bmatrix} \\
    a_{\+} : (A\+B)\+C \to A\+ (B\+C) \qquad &a_{\+} := id \\
    c_{\+} : [A_1,\ldots,A_n,B_1,\ldots,B_m] \to [B_1,\ldots,B_m,A_1,\ldots,A_n]  \qquad &
      c_{\+} := \begin{bmatrix}
                      0_{m\times n} & I_{n\times n} \\
                      I_{m\times m} & 0_{n\times m}
                \end{bmatrix}.
  \end{align*}

  The action of $\+$ on maps is given by:
  \[
    [f_{i\jay}]\+ [g_{\ell \kay}] = \begin{bmatrix}
      [f_{i\jay}] & 0 \\
      0 & [g_{\ell \kay}]
      \end{bmatrix}.
  \]

  With this definition, we see that $\+$ is a restriction functor:
  \begin{align*}
    id_X \+ id_Y &= id_{X\+Y}, \\
    f_1 g_1 \+ f_2 g_2 &= h_1 \+ h_2 =
    \begin{bmatrix}
      h_1 & 0 \\ 0 & h_2
    \end{bmatrix} =
    \begin{bmatrix}
      f_1 & 0 \\ 0 & g_1
    \end{bmatrix}
    \begin{bmatrix}
      f_2 & 0 \\ 0 & g_2
    \end{bmatrix} = (f_1 \+ g_1) (f_2 \+ g_2).
  \end{align*}

  Following Definition~\ref{def:disjointness_tensor}, we note $\icpa = (1\+0)\upr =
  \begin{bmatrix}
    1&0\\0&0
  \end{bmatrix}
  \begin{bmatrix}
    1\\0
  \end{bmatrix}  =
  \begin{bmatrix}
    1\\0
  \end{bmatrix}
  $ and similarly $\icpb =
  \begin{bmatrix}
    0\\1
  \end{bmatrix}$. Suppose we have $f = [f_{i\jay}]$ and $g = [g_{i\jay}]$ where
  $i \in \{1,\ldots,n\}$ and $j\in\{1,2\}$. Further suppose $f\icpa = g\icpa$ and $f\icpa = g\icpa$.
  Therefore, $f \icpa = [f_{i 1}] = [g_{i 1}] = g\icpa$ and
  $f \icpb = [f_{i 2}] = [g_{i 2}] = g\icpb$, but this means that $f = g$ and we may conclude
  $\icpa$ and $\icpb$ are jointly monic. Similarly, $\cp{1} = [1\ 0]$ and $\cp{2} = [0\ 1]$ are
  jointly epic.
\end{proof}

\begin{lemma}\label{lem:imat_tensor_is_a_disjoint_sum_tensor}
  Given \X is an inverse category with a disjoint join and restriction zero, then \imatx
  has a disjoint sum tensor.
\end{lemma}
\begin{proof}
  By Lemma~\ref{lem:imat_is_disjoint_tensor_category}, we know that the tensor defined by
  list catenation is a disjoint tensor. To show that it is a disjoint sum tensor, we must
  fulfill Definition~\ref{def:disjoint_sum_tensor}.

  First, for the diagram below, we show that $\alpha$ exists if and only if
  $f\icpb \tjdown g\icpb$. Note that diagram assumes all the solid arrows exist and make the
  diagram a commutative diagram.
  \begin{align*}
    \xymatrix@C+5pt{
      [A] \ar@/_/[ddr]_f \ar@/^/[drr]^g \ar@{.>}[dr]|-{\alpha} \\
        & [X,Y,Z] \ar[d]^{\icp{1,2}} \ar[r]_{\icp{1,3}} & [X,Z] \ar[d]^{\icpa} \\
        & [X,Y] \ar[r]_{\icpa} & [X].
    }
  \end{align*}
  The existence of $f\icpb \tjdown g\icpb$ means there is an $h = [h_1,h_2]: [A]\to[Y,Z]$ such that
  $h\icpa = f\icpb$ and $h\icpb = g\icpb$. From the diagram, given that $f = [f_1,f_2]$ and
  $g = [g_1,g_2]$, we know that $f_1 = f\icpa = g\icpa = g_1$. We also have $h_1 = f_2$ and
  $h_2 = g_2$. If we set $\alpha$ to the matrix $[f_1,f_2,g_2]$, the diagram above commutes.
  We need only show that $\alpha$ is a map in \imatx. As $f, g$ and $h$ are maps in
  \imatx, we know that:
  \begin{align*}
    f_1\cp{1} & \cdperp f_2 \cp{2}\\
    f_1\cp{1} = g_1\cp{1} & \cdperp g_2 \cp{2}\\
    f_2 \cp{2} = h_1\cp{1} & \cdperp h_2 \cp{2} = g_2 \cp{2}.\\
  \end{align*}
  From this, we can conclude $\cdperp [f_1 \cp{1}, f_2 \cp{2}, g_2 \cp{3}]$.

  Conversely, suppose we have an $\alpha = [\alpha_1, \alpha_2, \alpha_3]$ that makes the
  above diagram commute. Then  $h := [\alpha_2, \alpha_3]$ is a map in $\X$.
  Since $[\alpha_1,\alpha_3] = g$ and $[\alpha_1,\alpha_2] = f$, we have
  $h\icpa = f\icpb$ and $h\icpb = g\icpb$, hence $h = f\icpb \tjdown g\icpb$.

  The proof that $\beta$ in the diagram below exists if and only if $\cp{2}h\tjup \cp{2}k$
  is similar.
  \[
    \xymatrix@C+5pt{
        &&[A] \\
         [X,Y] \ar[r]_{\cp{1,2}} \ar@/^/[urr]^h &[X,Y,Z] \ar@{.>}[ur]|-{\beta} \\
         [X] \ar[u]^{\cpa} \ar[r]_{\cpa} & [X,Z]. \ar[u]^{\cp{1,3}} \ar@/_/[uur]_k
    }
  \]
\end{proof}

\begin{theorem}\label{thm:imatx_is_an_inverse_sum_category}
  Given \X an inverse category with a disjoint sum tensor and restriction zero, \imatx is
  an inverse sum category.
\end{theorem}
\begin{proof}
  By Lemma~\ref{lem:imat_tensor_is_a_disjoint_sum_tensor}, we know $\imatx$ has a disjoint sum
  tensor and therefore by Proposition~\ref{prop:disjointness_tensor_gives_disjoint_join},
  it has a disjoint join. By Lemmas~\ref{lem:an_inverse_sum_tensor_gives_inverse_sums} and
  \ref{lem:a_disjoint_sum_tensor_is_an_inverse_sum_tensor}, we know that $[A,B] = A\+B$ is
  an inverse sum of $A$ and $B$ for any two objects in \imatx, and hence, \imatx is an inverse
  sum category.
\end{proof}

\subsection{Equivalence between an inverse sum category and its matrix category} % (fold)
\label{sub:equivalence_between_an_inverse_sum_category_and_its_matrix_category}
In this sub-section we will provide restriction functors between an inverse sum category
\X and its matrix category \imatx. Furthermore, we will show these functors form an
equivalence between these two categories.

\begin{definition}\label{def:functor_from_x_to_imatx}
  Given $\X$ is an inverse sum category with disjoint join $\cdperp$ and restriction zero,
  define $M:\X \to \imatx$ by:
  \begin{align*}
    \text{Objects:}\ &M(A) := [A]\\
    \text{Maps:}\ &M(f) := [f] \text{ -- The }1\times1\text{ matrix with entry }f.
  \end{align*}
\end{definition}
\begin{lemma}\label{lem:m_is_a_functor}
  The map $M$ from Definition~\ref{def:functor_from_x_to_imatx} is a restriction functor.
\end{lemma}
\begin{proof}
  From the definition of \imatx, we have
  \begin{align*}
    &f: A\to B  \iff M(f):M(A)\to M(B)\quad ([f]: [A]\to [B])\\
    &M(id_A)  = [id_{A}] = id_M(A)\\
    &M(f g) = [f g] = [f][g] = M(f) M(g)\\
    &M(\rst{f}) = [\rst{f}] = \rst{[f]} = \rst{M(f)}.
  \end{align*}
\end{proof}

\begin{definition}\label{def:functor_from_imatx_to_x}
  Given $\X$ is an inverse sum category with disjoint join $\cdperp$ and restriction zero $0$,
  and inverse sum tensor $\+$ define $S:\imatx \to \X$ by:
  \begin{align*}
    \text{Objects:}\ &S([A_1,A_2, \ldots, A_n]) := A_1\+A_2\+\cdots\+A_n\\
    \text{Maps:}\ &S([f_{i\jay}]) := \djoinbig_i \icp{i} (\djoin_j f_{i\jay} \cp{j}).
  \end{align*}
\end{definition}

\begin{lemma}\label{lem:s_is_a_functor}
  The map $S$ from Definition~\ref{def:functor_from_imatx_to_x} is a restriction functor.
\end{lemma}
\begin{proof}
  From the definition of \imatx, where $A = [A_1,A_2,\ldots,A_n]$, $B=[B_1,B_2,\ldots,B_M]$,
  and $f = [f_{i\jay}]$ we have
  \begin{align*}
    &S(id_A)  = S([id_{A_i}]) = \djoinbig_i \icp{i} (\djoin_j \cp{j}) = id_{S(A)}\\
    &f: A\to B  \iff S(f):S(A)\to S(B) \iff \\
    &\qquad \qquad  \djoinbig_i \icp{i} (\djoin_\jay f_{i\jay} \cp{j}) : A_1\+\cdots\+A_n \to B_1\+\cdots\+B_m\\
    &M(\rst{f}) = [\rst{f}] = \rst{[f]} = \rst{M(f)}.
  \end{align*}
  For composition, we have
  \begin{align*}
    S(f)S(g) & = (\djoinbig_i \icp{i} (\djoin_j f_{i\jay} \cp{j}))
                  (\djoinbig_{j'} \icp{j'} (\djoin_k g_{\jay\kay} \cp{k})) \\
    &= \djoinbig_i \icp{i} \djoinbig_j \djoinbig_{j'} f_{i\jay} \cp{j} \icp{j'}
            (\djoin_k g_{\jay'\kay} \cp{k}) \\
    &= \djoinbig_i \icp{i} \djoinbig_j  f_{i\jay} (\djoin_k g_{\jay\kay} \cp{k}) \\
    &= \djoinbig_i \icp{i} \djoinbig_k (\djoin_j  f_{i\jay}  g_{\jay\kay} \cp{k}) \\
    &= S([\djoin_j f_{i\jay} g_{\jay\kay}]) \\
    &= S(f g).
  \end{align*}

\end{proof}

\begin{proposition}\label{pro:x_and_imatx_are_equivalent}
  Given an inverse category \X with a disjoint sum tensor $\+$ and restriction zero, then the
  categories $\X$ and \imatx are equivalent.
\end{proposition}
\begin{proof}
  The functors of the equivalence are $S$ from Definition~\ref{def:functor_from_imatx_to_x} and
  $M$ from Definition~\ref{def:functor_from_x_to_imatx}.

  First, we see that $M S:\X \to \X$ is the identity functor as
  \begin{align*}
    \text{Objects:}\ &S(M(A)) = S([A]) = A,\\
    \text{Maps:}\ &S(M(f)) = S([f]) = f.
  \end{align*}

  Next, we need to show that there is a natural transformation and isomorphism $\rho$ such that
  $\rho(S M) = I_{\imatx}$. For each object $[A_1,A_2,\ldots,A_n]$, set
  $\rho A = \begin{bmatrix}\icpa& \cdots &\icp{n}\end{bmatrix}$.

  Note that the functor $S M$ has the following effect:

  \begin{align*}
    \text{Objects:}\quad &M(S([A_1,\ldots, A_n])) = M(A_1\+\cdots\+ A_n) =
      [A_1\+\cdots\+ A_n]\\
    \text{Maps:}\quad &M(S([f_{i\jay}]) = M(\djoinbig_i \icp{i} (\djoin_j f_{i\jay} \cp{j}))
      = [\djoinbig_i \icp{i} (\djoin_j f_{i\jay} \cp{j})].
  \end{align*}
  We can now draw the commuting naturality square for $f=[f_{i\jay}]:[A_i]\to [B_j]$:
  \[
    \xymatrix{
      SM([A_i]) \ar@{=}[r] & [\+_i A_i]
          \ar[rrrr]^{\begin{bmatrix}\icpa& \cdots &\icp{n}\end{bmatrix}}
          \ar[dd]_{SM(f)}
          & &&& [A_i] \ar[dd]^{f}\\
        &&&&&  \\
      SM([B_j]) \ar@{=}[r] & [\+_j B_j]
          \ar[rrrr]_{\begin{bmatrix}\icpa& \cdots &\icp{m}\end{bmatrix}}
          & &&& [B_j]. \\
    }.
  \]

  Following the square by the top--right path from $[\+_i A_i]$ to $[B_j]$, by the definition of
  the maps in the category \imatx, we see each $B_j = \djoin_i\icp{i} f_{i\jay}(\+_i A_i)$.
  Following the left--bottom path, composing $SM(f)$ with $\begin{bmatrix}\icpa& \cdots
  &\icp{m}\end{bmatrix}$ gives us the map
  \begin{multline*}
    \begin{bmatrix}\djoin_i \icp{i} (\djoin_j f_{i\jay}\cp{j})\icpa& \cdots
      &\djoin_i \icp{i} (\djoin_j f_{i\jay}\cp{j})\icp{m}\end{bmatrix} = \\
      \begin{bmatrix}\djoin_i \icp{i}  f_{i 1}& \cdots &\djoin_i \icp{i} f_{i m}\end{bmatrix}.
  \end{multline*}
  Applying this to $[\+_i A_i]$, we see each $B_j = \djoin_i\icp{i} f_{i\jay}(\+_i A_i)$ and the
  two directions are equal.

  Finally, we know that $\inv{\rho_{A_i}} = \begin{bmatrix}\cp{1}\\ \vdots \\\cp{n}\end{bmatrix}$
  and defines an isomorphism between any object of the form $[\+_i A_i]$ and the
  object $[A_1,\ldots,A_n]$.
\end{proof}
% subsection equivalence_between_an_inverse_sum_category_and_its_matrix_category (end)
\begin{lemma}\label{lem:inverse_sum_category_maps_are_matrices}
  In an inverse sum tensor category, any map $f:A\+B \to C\+D$ may be represented in a matrix form.
  Composition of maps may be computed by multiplication of the matrices, with composition taking
  the place of base level multiplication and $\djoin$ in the place of addition.
\end{lemma}
\begin{proof}
  Recall from Lemma~\ref{lem:an_inverse_sum_tensor_gives_inverse_sums} that $A\+B$ and $C\+D$ are
  inverse sums. Referencing Definition~\ref{def:inverse_sum}, define $e_0 = \inv{i_1}i_1$ and $e_1
  = \inv{i_2}i_2$ and note that $e_0 \cdperp e_1$, $e_0 \djoin e_1 = 1$. Then given a function
  $f:A\+B \to C\+D$ define
  \[ f_M =
     \begin{bmatrix}
       e_0 f e_0 & e_0 f e_1 \\
       e_1 f e_0 & e_1 f e_1
      \end{bmatrix}.
  \]
  Since $e_0 \cdperp e_1$, the maps in the rows of $f_M$ are disjoint by the
  stability of the disjointness relation. Similarly, the maps in the columns are disjoint by
  universality. We have $e_0 f e_0 \djoin e_0 f e_1 = e_0 f$ and $e_1 f e_0 \djoin e_1 f e_1 = e_1
  f$. Once again, each of these maps are disjoint by universality. Finally,
  $e_0 f \djoin e_1 f = (e_0 \djoin  e_1) f = f$ and hence we may recover the initial map whenever
  we have a matrix of this form. We will call this computation the distinct join of $f_M$.

  Next, consider $f_M \times g_M$. As each $e_i$ is idempotent, this is
  \[ f_M \times g_M =
     \begin{bmatrix}
       e_0 f e_0 g e_0 \djoin e_0 f e_1 g e_0 & e_0 f e_0 g e_1 \djoin e_0 f e_1 g e_1 \\
       e_1 f e_0 g e_0 \djoin e_1 f e_1 g e_0 & e_1 f e_0 g e_1 \djoin e_1 f e_1 g e_1
      \end{bmatrix}
      =
     \begin{bmatrix}
       e_0 f g e_0  & e_0 f  g e_1  \\
       e_1 f  g e_0 & e_1  g f e_1
      \end{bmatrix}
  \]
  where the distinct joins are well defined due to the stability and universality of the join.
  We can see that the distinct join of $f_M \times g_M = f g$ and as such we have composition.

\end{proof}

In particular, we note that we may represent $f:A\to B$ by the matrix
\[
  \begin{bmatrix}
    1f1& 1f 0 \\ 0 f 1 & 0 f 0
  \end{bmatrix}
  =
  \begin{bmatrix}
    f& 0 \\ 0 & 0
  \end{bmatrix}
\]
as $A \cong A\+0$ and $B \cong B\+0$.

We now turn to examining a category of specialized matrices over an inverse sum category. In
general, the
matrix category $\imat{(\X)}$ will have objects that are lists of objects in $\X$,
$X=(X_1,\ldots,X_m)$. Maps between lists will be matrices $[f_{i j}]: (X_1,\ldots,X_m) \to
(Y_1,\ldots,Y_n)$. We will only consider maps whose matrices have disjoint rows, i.e., if
$[f_{i j}]$ is a matrix, it must have $f_{i j} \cdperp f_{i k}$ for all $i$ whenever $j \ne k$.




% sub-subsection matrix_representation (end)







% A unique decomposition category is a category where any
% \[
%   h:A\+B \to C\+D
% \]
% is uniquely determined by:
% \begin{alignat*}{2}
%   \cpa  h  \cpa^{(-1)} &: A \to C & \quad \cpa  h  \cpb^{(-1)} &: A \to D\\
%   \cpb  h  \cpa^{(-1)} &: B \to C & \quad \cpb  h  \cpb^{(-1)} &: B \to D
% \end{alignat*}
% i.e., is writable as the matrix:
% \[
%   \begin{bmatrix}
%     \cpa  h  \cpa^{(-1)} & \cpa  h  \cpb^{(-1)}\\
%     \cpb  h  \cpa^{(-1)} &\quad \cpb  h  \cpb^{(-1)}
%   \end{bmatrix}:
%   A\+B \to C\+D.
% \]
% See \cite{haghverdi:2000phd}.
%
% The $0$ map in the category corresponds to the $0$-dimensional matrix,
% \[
%   0\xrightarrow{[]}B.
% \]
%
% Note also that as $\cpa$ and $\cpb$ are restriction maps, the following diagram commutes:
% \[
%   \xymatrix @C+10pt{
%     A \ar@{=}[d] \ar[r]^{\cpa} & A\+B \\
%     A \ar[r]_{\cpa} & A\+0 \ar[u]_{1+0}
%   }
% \]
%
% The co-diagonal does not exist, except at the $0$ object. To see this, recall this is an inverse
% category and the co-diagonal is a total map. Therefore, given $\nabla: A \+ A \to A$, then
% $\nabla^{(-1)} : A \to A \+ A$ and $\nabla \nabla^{(-1)} = 1_{A\+A}, \nabla^{(-1)} \nabla = 1_{A}$.

% subsection inverse_sum_tensor (end)

% section inverse_sums_categories (end)
\section{Completing a distributive inverse category} % (fold)
\label{sec:completing_a_distributive_inverse_category}
\subsection{Distributive restriction categories} % (fold)
\label{sub:distributive_restriction_categories}


\begin{definition}
  A Cartesian restriction category with a restriction zero and coproducts is called
  \emph{distributive} when there is an isomorphism $\rho$ such that
  \[
    A\times (B+C) \xrightarrow{\rho} (A\times B) +(A\times C).
  \]
\end{definition}

In a distributive inverse category, we lack:
\[
  A  \xrightarrow{!} 1
\]
and
\[
  A+A \xrightarrow{\nabla} A.
\]
% subsection distributive_restriction_categories (end)

\subsection{Distributive inverse categories} % (fold)
\label{sub:distributive_inverse_categories}
% \note{
% The following two \emph{should} be equivalent. Either
% \begin{itemize}
%   \item $\*$ preserves sums
%   \item $\*$ preserves disjoint
% \end{itemize}
%
% Break down the map into disjoint items, if equivalent on pieces, then they are
% equivalent.
% }
\begin{definition}\label{def:distributive_inverse_category}
  A \emph{distributive inverse category} \D consists of the following:
  \begin{itemize}
    \item \D is an inverse category;
    \item \D has an inverse product tensor, $\*$, per Sub-Section~\ref{sub:inverse_products};
    \item \D has an inverse sum tensor, $\+$, per Definition~\ref{def:inverse_sum_tensor} and
    \item There is a family of isomorphisms, $d$, such that
    \begin{equation}
      \xymatrix{
      && A\*B\\
      A \* (B \+ C) \ar[rr]^{d} \ar[urr]^{1\*\icpa} \ar[drr]_{1\*\icpb}
        && (A\*B) \+ (A \* C) \ar[u]_{\icpa} \ar[d]^{\icpb}\\
      && A\*C,
      }\label{eq:inverse_distributive_laws}
    \end{equation}
    commutes in \D for any choices of objects $A, B, C$.
  \end{itemize}
\end{definition}

Note that as we are operating in an inverse category, we also have the inverse of
diagram~\ref{eq:inverse_distributive_laws} available to us. That is,
\begin{equation}
  \xymatrix{
  A\*B \ar[d]_{\cpa} \ar[drr]^{1\*\cpa}\\
  (A\*B) \+ (A \* C)  \ar[rr]^{\inv{d}}  &&A \* (B \+ C) \\
  A\*C\ar[u]^{\cpb} \ar[urr]_{1\*\cpb}
  }\label{eq:inverse_distributive_laws_reversed}
\end{equation}
is also a commuting diagram in \D.

\begin{definition}\label{def:join_of_perp_rst_maps}
  Suppose \X is an inverse category with a disjoint join tensor $\+$ and a restriction zero. Then
  for maps $f:A \to B$ and $g:A \to C$ with $\rst{f} \perp \rst{g}$, define the map $[f,g]:A \to
  B\+C$ as $(f \cpa) \djoin (g \cpb)$. This is well defined as $\wrg{\cpa} \perp \wrg{\cpb}$ and
  therefore by \axiom{Dis}{7}, $f\cpa \perp g\cpb$.
\end{definition}

\begin{lemma}
  Suppose \X is an inverse category \X\ with:
  \begin{itemize}
    \item a disjoint join tensor $\+$,
    \item a restriction zero, and
    \item an inverse  product tensor $\*$ which distributes over disjoint joins, (i.e.,
      $f\*(g\djoin h) = (f\*g) \djoin (f\* h)$).
  \end{itemize}
  Then, \X is an inverse distributive  category.
\end{lemma}
\begin{proof}
  By assumption, we have the first three items of
  Definition~\ref{def:distributive_inverse_category}. Therefore, we need to construct an
  isomorphism $d$ such that diagram~\ref{eq:inverse_distributive_laws} commutes. We claim that the
  map $d = [1\* \icpa, 1\* \icpb]$ does this.

  First, note that the typing of $d$ is correct. By Definition~\ref{def:join_of_perp_rst_maps},
  \[
    d = ((1 \* \icpa) \cpa )\djoin ((1\*\icpb) \cpb) : A \* (B \+ C) \to (A\*B)\+(A\*C)
  \]
  as
  \begin{align*}
      A \* (B \+ C) & \xrightarrow{(1\*\icpa)} A\* B \xrightarrow{\icpa} (A\*B)\+(A\*C),\\
      A \* (B \+ C) & \xrightarrow{(1\*\icpb)} A\* C \xrightarrow{\icpb} (A\*B)\+(A\*C).
  \end{align*}

  Next, we need to show $d$ is an isomorphism. We will do this by showing both $\rst{d} = 1$ and
  $\rst{\inv{d}} = 1$.
  As a consequence of Lemma~\ref{lem:join_is_associative_and_commutative_monoid}, we know
  the inverse of $d$ is
  \[
    \inv{((1 \* \icpa) \cpa )}\djoin \inv{((1\*\icpb) \cpb)} =
      ( \icpa(1 \* \cpa) ) \djoin ( \icpb(1\*\cpb)).
  \]

  Having $\*$ distribute over the disjoint sum means that for any maps $f,h,k$ with $h\perp k$, we
  have $f \* (h\djoin k) = (f\*h) \djoin (f\*k)$. We use this in the calculation of the restriction
  of $f$:
  \begin{align*}
    \rst{((1 \* \icpa) \cpa )\djoin ((1\*\icpb) \cpb)}
      &= \rst{((1 \* \icpa) \cpa )}\djoin \rst{((1\*\icpb) \cpb)} \\
      &= (1\* \rst{\icpa}) \djoin (1\* \rst{\icpb}) \\
      &= (1\* (\rst{\icpa} \djoin \rst{\icpb}) \\
      &= 1\* ((1\+0) \djoin (0\+1) )\\
      &=1 \* 1 = 1
  \end{align*}
  and for the inverse,
  \begin{align*}
    \rst{( \icpa(1 \* \cpa) ) \djoin ( \icpb(1\*\cpb))}
      &= \rst{(\icpa(1 \* \cpa) )}\djoin \rst{(\icpb(1\*\cpb))} \\
      &= \rst{(\icpa\rst{(1 \* \cpa)} )}\djoin \rst{(\icpb\rst{(1\*\cpb)})} \\
      &= \rst{(\icpa)}\djoin \rst{(\icpb)}\\
      &= (1\+0) \djoin (0\+1) \\
      &= 1.
  \end{align*}

  Hence, $[1\* \icpa, 1\* \icpb]$ is an isomorphism.
  Finally, we must show that diagram~\ref{eq:inverse_distributive_laws} commutes:
  \begin{align*}
    d \icpa & = \left( ((1 \* \icpa) \cpa )\djoin ((1\*\icpb) \cpb) \right)\icpa \\
     & = \left(((1 \* \icpa) \cpa )\icpa \right) \djoin \left( ((1\*\icpb) \cpb) \icpa\right) \\
     & = \left((1 \* \icpa) 1 \right) \djoin \left( (1\*\icpb) 0\right) \\
     & = (1 \* \icpa) \djoin 0\\
     & = 1 \* \icpa
  \end{align*}
  and
  \begin{align*}
    d \icpb & = \left( ((1 \* \icpa) \cpa )\djoin ((1\*\icpb) \cpb) \right)\icpb \\
     & = \left(((1 \* \icpa) \cpa )\icpb \right) \djoin \left( ((1\*\icpb) \cpb) \icpb\right) \\
     & = 0 \djoin (1 \* \icpb)\\
     & = 1 \* \icpb.
  \end{align*}
  This shows the fourth condition is satisfied and $\X$ is a distributive inverse category.
\end{proof}

We have seen that a second tensor distributing over the disjoint joins implies that we have an
inverse distributive category. We now show the converse is true.
\begin{lemma}\label{lem:distributive_means_distribute_over_triangles}
  Given an inverse distributive category \X, then $h\*(f\tjdown g) = (h\* f) \tjdown (h\* g)$
  whenever $f\tjdown g$ exists and $h\*(f\tjup g) = (h\* f) \tjup (h\* g)$ whenever
  $f\tjup g$ exists.
\end{lemma}
\begin{proof}
  Let $h:A \to C$, $f:B \to D$ and $g: B \to E$. Consider the following diagram:
  \[
    \xymatrix@R+20pt@C+10pt{
      & && B\*E\\
      A\*C \ar[rr]^{h\*(f\tjdown g)} \ar[drrr]_{h\*g} \ar[urrr]^{h\*f}
        && B \* (D \+ E) \ar[r]^{\cong} \ar[ur]_{1\*\icpa} \ar[dr]^{1\*\icpb}
        & (B\*D) \+ (B \* E) \ar[d]^{\icpb} \ar[u]_{\icpa}\\
      & && B\*D.
    }
  \]
  The two leftmost  triangles commute by the diagram for $f\tjdown g$. The right hand triangles
  commute as per Definition~\ref{def:distributive_inverse_category}. By the uniqueness of the
  $\tjdown$ operation we see  $h\*(f\tjdown g) = (h\* f)\tjdown(h\* g)$,

  The argument for showing $h\*(f\tjup g) = (h\* f)\tjup(h\* g)$ follows the same methodology.
  %TODO: Is the isomorphism important? Do we need to do something with it?
\end{proof}
\begin{lemma}\label{lem:distributive_means_distribute_over_join}
  Given an inverse distributive category \X, then $\*$ distributes over the disjoint join.
\end{lemma}
\begin{proof}
  First recall the definition of $f\djoin g = (\rst{f}\tjdown\rst{g})(f\tjup g)$. In order
  to show $h\*(f\djoin g) = (h\*f)\djoin (h\*g)$, we need to show that
  \begin{equation}
    h\*(\rst{f}\tjdown\rst{g})(f\tjup g) = (\rst{h\*f}\tjdown\rst{h\*g})(h\*f\tjup h\*g).
    \label{eq:distribute_over_left_right}
  \end{equation}

  Since $h\*(\rst{f}\tjdown\rst{g})(f\tjup g) =
  (\rst{h}\*(\rst{f}\tjdown\rst{g}))(h\*(f\tjup g))$, Equation~(\ref{eq:distribute_over_left_right})
  follows directly from  Lemma~\ref{lem:distributive_means_distribute_over_triangles} and the fact
  that $\*$ is a  restriction functor.
\end{proof}

\begin{corollary}\label{cor:inverse_product_and_disjointness}
  Suppose we have an inverse distributive category \X. Then,
  \begin{enumerate}[{(}i{)}]
    \item if $f \perp g$, then $h\*f \perp h\*  g$ for any $h$,
    \item  if  $f\perp g : A\to B$ and $h \perp k: C\to D$, then $(f\*h)\perp (g\*k)$.
   \end{enumerate}
\end{corollary}
\begin{proof}
  \prepprooflist
  \begin{enumerate}[{(}i{)}]
    \item As $f\perp g$, we have $f\tjup g$ and $f\tjdown g$. By
      Lemma~\ref{lem:distributive_means_distribute_over_triangles}, both $h\*f \tjup h\*g$ and
      $h\*f \tjdown h\* g$ exist and therefore $h\*f \perp h\*g$.
    \item By the previous item, we have that $((f\djoin g)\*h) \perp ((f\djoin g)\* k)$. Then, by
      \axiom{DJ}{1} and \axiom{Dis}{3} we have $(f\*h)\perp (g\*k)$.
  \end{enumerate}
\end{proof}
% subsection distributive_inverse_categories (end)

\subsection{Discrete inverse categories with inverse sums} % (fold)
\label{sub:discrete_inverse_categories_with_inverse_sums}

We now consider the case where we have a discrete inverse category with inverse product
tensor $\*$ and a disjoint join tensor $\+$, with the $\*$ tensor preserves the
disjoint join.

A map in \Xt is related to a map in \X in the following way:
\[
  \infer{A\xrightarrow{f} B\*C \text{ in }\X.}{A \xrightarrow{\ (f,C)\ } B \text{ in } \Xt}
\]

Our goal is to show that an inverse sum in a distributive inverse category becomes a co-product
in \Xt.

\begin{lemma}\label{lem:x_tilde_has_a_restriction_zero}
  Given $\X$ is a distributive inverse category, then $\Xt$, the discrete inverse category created
  from $\X$, has a restriction zero.
\end{lemma}
\begin{proof}
  Recall from Theorem~\ref{thm:discrete_inverse_categories_are_equivalent_to_discrete_restriction_categories}
  that $\X$ is equivalent as a category to $\Xt$ under the identity on objects
  functor
  \[
    {\raisebox{-18pt}{$\wtf:\X \to \Xt;$}} \quad
    \xymatrix{A\ar[d]^{f}\\B}\ \raisebox{-18pt}{$\mapsto$}\, \xymatrix{A\ar[d]^{(f\inv{\usr},1)}\\B}
    \,\raisebox{-18pt}{.}
  \]

  In $\X$, we know $0$ is a terminal and initial object, with maps
  $A \xrightarrow{t_A} 0 $ and $0 \xrightarrow{z_A} A$, where  $\rst{0_{A,A}} = 0_{A,A} = t_A z_A$.

  First we note that $0$ is both initial and terminal in $\Xt$, with the terminal maps being
  $\wtf(t_A)$ and initial maps being $\wtf(z_A)$.

  As was also shown in
  Theorem~\ref{thm:discrete_inverse_categories_are_equivalent_to_discrete_restriction_categories},
$\wtf$ is a restriction functor, so in $\Xt$ we have
  \[
    0_{A,A} = \wtf(t_A)\wtf(z_A) = \wtf(t_A z_A) = \wtf(0_{A,A}) = \wtf(\rst{0_{A,A}}) =
    \rst{\wtf(0_{A,A})} = \rst{0_{A,A}}.
  \]
  Hence, $0_{A,A}$ is a restriction zero in $\Xt$.
\end{proof}

\begin{lemma}\label{lem:maps_from_inverse_sums_have_disjoint_equivalences}
  In a distributive inverse category \X:
  \begin{enumerate}[{(}i{)}]
    \item Given $f : A \to Y\*C$, we can construct $f' : A \to Y \* (C\+D)$ for some object $D$ such
      that $f \xequiv{} f'$.
    \item Given $g : B \to Y\*D$, we can construct $g'' : B \to Y \* (C\+D)$ for some object $C$ such
      that $g \xequiv{} g''$.
    \item Given $f : A \to Y \* C$, $g : A\to Y \* D$, then the $f', g''$ as constructed by the
      previous points satisfy $\icpa f' \perp \icpb g''$, where $\icpa f',\,\icpb g'': A\+B \to Y\*(C\+D)$.
  \end{enumerate}
\end{lemma}
\begin{proof}
  \prepprooflist
  \begin{enumerate}[{(}i{)}]
  \item Set $f' = f(1\*\cpa)$.
    To show $f\equiv f'$, we must first show their restriction is the same.
    \begin{align*}
      \rst{f(1\*\cpa)} &= \rst{f\rst{(1\*\cpa)}}\\
      & = \rst{f 1}\\
      &=  \rst{f}
    \end{align*}
    Now we detail the mediating map between $f$ and $f'$:
    \[
    \xymatrix @C=25pt @R=25pt{
      & & Y \* C \ar@{.>}[dr]^{\Delta\* 1}\\
      & & & Y \* (Y\* C) \ar@{.>}[dd]^{1\*1\*\cpa}\\
      A \ar[uurr]^f \ar[ddrr]_{f'} \\
      & & & Y \* Y\*(C\+ D) \ar@{.>}[dl]^(.4){\inv{\Delta}\* 1}\\
      && Y\*(C\+ D)\*X.
    }
    \]
  To show this commutes, we primarily use that $\Delta\inv{Delta} = 1$
  \begin{align*}
    f(\Delta\*1)&(1\*1\*\cpa)(\inv{\Delta}\*1) \\
    &= f(\Delta\*1)(\inv{\Delta} \* \cpa)\\
    &= f(1 \* \cpa)\\
    &= f'
  \end{align*}
  and hence $f\xequiv{h}f'$ where $h = (1\*\cpa)$.
  \item For this item, we set $g'' = g(1\*\cpb)$. The proof that $g\xequiv{k}g''$, where
    $k = (1\*\cpb)$ is done in the same way as the previous point.
  \item In order to show  $\icpa f' \perp \icpb g''$, we will proceed by showing their restrictions
    and ranges are disjoint.
    As $\rst{\inv{\cpa}} \perp \rst{\inv{\cpb}}$ and $\rst{\inv{\cpa}f'} \le \rst{\inv{\cpa}}$ and
    $\rst{\inv{\cpb}g''} \le \rst{\inv{\cpb}}$, we immediately have
    $\rst{\inv{\cpa}f'} \perp \rst{\inv{\cpb}g'}$.

    For the ranges, we have
    \begin{align*}
      \wrg{\inv{\cpa} (f (1\*\cpa))} &=
      \rst{((1\*\icpa)\inv{f})\cpa}\\
      &= \rst{((1\*\icpa)\inv{f}}\\
      &\le \rst{(1\*\icpa)}
    \end{align*}
    and similarly
    \[
      \wrg{\inv{\cpb} g''} \le \rst{(1\*\icpb)}.
    \]

    Using Lemma~\ref{lem:disjointness_various}
    we know that $\rst{(\icpa)}\perp \rst{(\icpb)}$. From
    Corollary~\ref{cor:inverse_product_and_disjointness}
    we conclude that $\rst{(1\*\icpa)}\perp \rst{(1\*\icpb)}$ and giving us
    $\wrg{\inv{\cpa}f'} \perp \wrg{\inv{\cpb}g''}$ and therefore $\inv{\cpa}f' \perp \inv{\cpb}g''$.
  \end{enumerate}
\end{proof}
\begin{proposition}\label{thm:x_tilde_has_coproducts_if_x_is_inverse_distributive_category}
  Given \X is an distributive inverse category, then the category \Xt has co-products.
\end{proposition}
\begin{proof}
  The tensor object, $A\+B$ in \X will become the co-product of $A, B$ in \Xt.

  The injection maps of the co-product are $i_1 =(\cpa \usr, 1)$ and $i_2 = (\cpb \usr, 1)$.

  Consider the following diagram in \Xt:
  \[
    \xymatrix{
      A \ar[drrr]^{(f,C)} \ar[dr]_{i_1} \\
      & A\+B \ar@{.>}[rr]^h & & Y. \\
      B \ar[urrr]_{(g,D)} \ar[ur]^{i_2}
    }
  \]

  In \X, this comes from the diagram:
  \begin{equation}
    \xymatrix{
      A \ar[dr]_{\cpa} \ar[rr]^f && Y\*C\\
      & A\+B\\
      B \ar[ur]^{\cpb} \ar[rr]^g && Y\*D\\
    }\label{diag:pre-coproduct}
  \end{equation}
  where the extraneous unit isomorphisms are removed.

  This corresponds to the conditions of
  Lemma~\ref{lem:maps_from_inverse_sums_have_disjoint_equivalences}. Hence by that lemma we may
  revise Diagram~\eqref{diag:pre-coproduct} as
  \begin{equation}
    \xymatrix@R+10pt {
      A \ar[d]^{\cpa} \ar[drrr]^{f'}\\
      A\+B \ar[rrr]^(.4){\icpa f' \djoin \icpb g''} &&& Y\*(C\+D)\\
      B \ar[u]^{\cpb} \ar[urrr]_{g''}\\
    }\label{diag:pre-coproduct-with-common-map}
  \end{equation}
  where $f'$ and $g''$ are respectively equivalent to $f,g$.

  Lifting Diagram~\ref{diag:pre-coproduct-with-common-map} to \X, we see this corresponds to the
  desired co-product diagram, where  $h$ in \Xt is the map $(\icpa f' \djoin \icpb g'', (C\+D))$.

  By construction, in \X, we have
  \[
    \cpa (\icpa f' \djoin \cpa\icpb g'') = (\cpa\icpa f') \djoin (\cpa\icpb g'') = f' \djoin 0 = f'
  \]
  and
  \[
    \cpb (\icpa f' \djoin \cpa\icpb g'') = g''.
  \]
  Hence, in \Xt, we have  $(i_1 \usr, 1) h = f$ and $(i_2 \usr, 1) h = g$.

  All that remains to be shown is that $h$ is unique.

  Suppose there is another $(k,E)$ in \Xt such that it satisfies the coproduct properties, i.e.,
  that $i_1 (k,E) = (f',C\+D)$ and $i_2 (k,E) = (g'',C\+D)$. In \X, $k:A\+B \to Y \+E$ and we have
  \begin{align*}
    \cpa k &\xequiv{x} f'\quad\text{and}\\
    \cpb k &\xequiv{y} g''
  \end{align*}
  where the maps $x, y: Y\*E \to Y\*(C\+D)$ fulfill the respective equivalence diagrams.

  The above gives us $\icpa\cpa k \xequiv{} \icpa f'$ and $\icpb\cpb k \xequiv{} \icpb g''$.
  As we know that $\*$ preserves the disjoint join, and the equivalence diagrams consist of maps
  under $\*$, we now have:
  \begin{align*}
    h&=\icpa f' \djoin \icpb g''\\
    &\xequiv{} \icpa\cpa k \djoin \icpb\cpb k\\
    &= (\icpa\cpa  \djoin \icpb\cpb) k\\
    &= 1 k = k.
  \end{align*}

  Hence, $h \xequiv{} k$ in $\X$ and $(h, C\+D) = (k,E)$ in \Xt, meaning
 %TODO: Show that the joint epicness of cpa, cpb give us a unique item.
 $A\+B$ in \X is the co-product of $A$ and $B$ in \Xt.
\end{proof}


% chapter inverse_sum_categories (end)

%%% Local Variables:
%%% mode: latex
%%% TeX-master: "../phd-thesis"
%%% End:
