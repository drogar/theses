%!TEX root = /Users/gilesb/UofC/thesis/phd-thesis/phd-thesis.tex
\chapter*{Overview of thesis chapters}

\section*{Introduction}

This chapter will give a brief introduction to and an explanation of both reversible and quantum
computing. I will discuss the relationship of both of these, along with a brief introduction to
their categorical semantics. I will discuss the equivalence of reversible Turing machines and
standard Turing machines.



\section*{Abstract Computability}

This chapter includes a basic introduction to category theory. Specific areas include definitions
of categories, natural transformations, functors and tensors. It introduces limits and
co-limits, focussing on products and co-products.

This chapter then moves on to introduce restriction categories and discuss how Cartesian
restriction categories can be used to model standard computing. I will introduce Turing categories
and show how this can be used to create a Partial Combinatory Algebra.

\section*{Inverse categories and Reversible computing}

An inverse category is a specific type of restriction category, in which each map has a partial
inverse. An inverse category corresponds to a restriction category in the same way a groupoid
corresponds to a category.

Inverse categories are explored, along with some basic results regarding products and idempotent
splitting. The inverse product is introduced, along with the concept of a discrete inverse
category and the relationship between a discrete inverse category and a Cartesian restriction
category.

Next, I explore the inverse sum, disjointness, and the disjoint join of maps in an inverse
category. These provide a way to work with objects which behave like co-products in the inverse
category. The interaction of the inverse sum and inverse product will be explained.

The chapter will define inverse Turing Categories.

The foregoing is based on two papers which are in preparation (joint work with R. Cockett).

To conclude the chapter, I will provide a detailed proof of the equivalence of reversible Turing
machines and standard Turing machines and connect this to inverse categories.

\section*{Quantum Computation}
This chapter introduces quantum computation, focussing on the semantics of quantum computing as
described using $\dagger$-categories. This will include examples of ``Toy'' quantum semantics.

\section*{Frobenius Algebras in $\dagger$-Categories}

In this chapter, I highlight the connection between the model of reversible computing (inverse
categories) introduced in this thesis, and that of quantum computing. Frobenius Algebras provide a
way of describing the basis used in quantum computation.


\section*{$D[\omega]$ based $\dagger$-categories and exact synthesis}

I will give an example of a toy quantum semantics on matrices over $D[\omega]$ (where
$D = \integers[\frac{1}{2}]$). The primary result of this chapter is that these matrices are
equivalent to the Clifford Group + $T$ over multiple qubits.

In preparation for the main result, I will discuss the issue of gate synthesis, where an arbitrary
quantum transform is to be expressed in terms of a set of base gates. An overview of the history of
both approximate and exact synthesis will be given.

Then, the chapter continues with an algorithm for exact synthesis of single-qubit transforms over
the Clifford group + $T$, together with a normal form and characterization of these. This is based
on a paper that is an extension of work done by Matsumoto and Amano (joint work with P. Selinger.)

Finally, I present the algorithm for exact synthesis over the Clifford group of multi-qubit
transforms and characterize those transforms that may be exactly synthesized. This is based on a
paper published in the journal Physical Review A (joint work with P. Selinger).

