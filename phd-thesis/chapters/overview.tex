%!TEX root = /Users/gilesb/UofC/thesis/phd-thesis/phd-thesis.tex
\chapter*{Overview of thesis chapters}

\section*{Introduction}

This chapter will give a brief introduction to and an explanation of both reversible and quantum
computing. How they are be related will be discussed, along with a brief introduction to their
categorical semantics. We will discuss the equivalence of reversible Turing machines and
standard Turing machines.



\section*{Abstract Computability}

This chapter will include a basic introduction to category theory. Specific areas introduced will
included definitions of categories, natural transformations and functors. It will introduce limits
and co-limits, focussing on products and co-products.

This chapter will start with an introduction to restriction categories and how Cartesian restriction
categories can be used to model standard computing. We will introduce Turing categories and show how
this can be used to create a Partial Combinatory Algebra.

\section*{Inverse categories and Reversible computing}

An inverse category is a specific type of restriction category, in which each map has a partial
inverse. An inverse category corresponds to a restriction category in the same way a groupoid
corresponds to a category.

Inverse categories will be explored, along with some basic results regarding products and idempotent
splitting. The inverse product will be introduced, along with the concept of a discrete inverse
category and the relationship between a discrete inverse category and a Cartesian restriction
category.

Next, we explore the inverse sum, disjointness and the disjoint join of maps in an inverse
category, allowing us to work with objects which behave like co-products in the inverse category.
The interaction of the inverse sum and inverse product will be explained.

The chapter will define inverse Turing Categories.

The foregoing is based on two papers which are in preparation (joint work with R. Cockett).

To conclude the chapter, we will provide a detailed proof of the equivalence of reversible Turing
machines and standard Turing machines and connect this to inverse categories.

\section*{Quantum Computation}
The initial part of this chapter will introduce quantum circuits, following which, we will explore
the semantics of quantum computing as described using $\dagger$-categories. This will include
examples of ``Toy'' quantum semantics, specifically that of matrices over $D[\omega]$ (where
$D = \integers[\frac{1}{2}]$).

\section*{Frobenius Algebras in $\dagger$-Categories}

In this chapter, we highlight the connection between the model of reversible computing (inverse
categories) introduced in this thesis, and that of quantum computing. Frobenius Algebras provide a
way of describing the basis used in quantum computation for a specific model of quantum semantics.


\section*{$D[\omega]$ based $\dagger$ categories and exact synthesis}

We will revisit the example of a toy quantum semantics, that of the Clifford Group +$T$ over
multiple qubits. As part of that, we will discuss the issue of gate synthesis, where an arbitrary
quantum transform is to be expressed in terms of a set of base gates. An overview of the history of
both approximate and exact synthesis will be given.

Then, the chapter will present an algorithm for exact synthesis of single-qubit transforms over the
Clifford group, together with a normal form and characterization of these. This is based on a paper
that is an extension of work done by Matsumoto and Amano (joint work with P. Selinger.)

Finally, we will present an algorithm for exact synthesis over the Clifford group of multi-qubit
transforms and characterize those transforms that may be exactly synthesized. This is based on a
paper published in the journal Physical Review A (joint work with P. Selinger).

