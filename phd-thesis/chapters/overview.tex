%!TEX root = /Users/gilesb/UofC/thesis/phd-thesis/phd-thesis.tex
\chapter*{Overview of thesis chapters}

\section*{Introduction}

This chapter will give a brief introduction to and explanation of both reversible and quantum
computing. Comparisons to each other and how they may be related will be discussed here, along
with a brief introduction to their semantics as computational models.

This chapter will include a basic introduction to category theory. Specific areas introduced will
included definitions of categories, natural transformations and functors. It will introduce limits
and co-limits, focussing on products and co-products.

This chapter will also include an introduction to restriction categories.

\section*{Reversible Computing}

This will introduce the subject of reversible computing, explaining the equivalence to standard
computing at the level of Turing machines. (Based on work by Bennet). It will also provide an
example of a reversible language.

The semantics of reversible computing will be examined briefly.

\section*{Abstract Computability}

This chapter will start with an introduction to restriction categories and how Cartesian restriction
categories can be used to model standard computing. We will introduce Turing categories and show how
this can be used to create a Partial Combinatory Algebra.

\section*{Inverse categories}

Inverse categories are a special type of restriction categories, in which each map has a partial
inverse. They correspond to restriction categories in the same way Groupoids correspond to
categories.

Inverse categories will be explored, along with some basic results regarding products and splits of
categories. The inverse product will be introduced, along with the concept of a discrete inverse
category and the relationship to Cartesian restriction categories.

The next step is to explore the inverse sum, disjointness and the disjoint join of maps in an
inverse category, allowing us to work with objects which behave like co-products in the inverse
category. The interaction of the inverse sum and inverse product will be explained.

The chapter will conclude with remarks on Inverse Turing Categories.

This is based on two papers which are in preparation (joint work with R. Cockett).

\section*{Quantum Computation}
The initial part of this chapter will introduce quantum circuits, following which, we will explore
the semantics of quantum computing as described using $\dagger$-categories. This will include
examples of ``Toy'' quantum semantics.

\section*{Frobenius Algebras}

In this chapter, we highlight the connection between the model of reversible computing (inverse
categories) introduced in this thesis, and that of quantum computing. Frobenius Algebras provide a
way of describing the basis used in quantum computation for a specific model of quantum semantics.

\section*{Transformations of Quantum Programs}

PROBABLY REMOVE!!!!

The current understanding of how to treat iteration and folding in Quantum circuits and algorithms
is somewhat lacking. This chapter will present unpublished work (done under the supervision of P.
Selinger) exploring this area. It will include a treatment of necessary conditions for a quantum
routine, its inputs and outputs, which would allow transforming the routine into either an iterated
or folded routine. Algorithms to compute this transform are also provided.

\section*{$D[\omega]$ based $\dagger$ categories}

We will show a specific example of a toy quantum semantics, that of the Clifford Group +$T$ over
multiple qubits. Additionally, we will discuss the issue of gate synthesis, where an arbitrary
quantum transform is to be expressed in terms of a set of base gates. The histories of both
approximate and exact synthesis will be reviewed.

Then, the chapter will present an algorithm for exact synthesis of single-qubit transforms over the
Clifford group, together with a normal form and characterization of these. This is based on a paper
that is an extension of work done by Matsumoto and Amano (joint work with P. Selinger.)

Finally, we will present an algorithm for exact synthesis over the Clifford group of multi-qubit
transforms and characterize those transforms that may be exactly synthesized. This based on a paper
published in the journal Physical Review A (joint work with P. Selinger).

