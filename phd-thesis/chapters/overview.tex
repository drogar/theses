%!TEX root = /Users/gilesb/UofC/thesis/phd-thesis/phd-thesis.tex
\chapter*{Overview of thesis chapters}

\section*{Category Theory}

This chapter will include a basic introduction to category theory. Specific areas introduced will 
included definitions of categories, natural transformations and functors. It will introduce limits
and colimts, focussing on products and co-products.

This chapter will also include an introduction to restriction categories.

\section*{Reversible Computing}

This will introduce the subject of reversible computing, explaining the equivalence to 
standard computing at the level of Turing machines and automata. (Based on work by Bennet and 
Abramsky respectively). It will also provide at least one example of a reversible language. (To be
picked.)

\section*{Inverse categories}

Inverse categories will be explored, along with some basic results regarding the product and
their split categories. The inverse product will be introduced, along with the concept of
a discrete inverse category and the relationship to Cartesian restriction categories. 

The inverse co-product and inverse categories with inverse co-products will also be addressed in
this chapter.

This is based on work to be submitted. 

\section*{Quantum Computation}
The basics of quantum computation and circuits will be introduced.

\section*{Transformations of Quantum Programs}
The current understanding of how to treat iteration and folding in Quantum circuits and 
algorithms is somewhat lacking. This chapter will present unpublished work (done under the
supervision of P. Selinger) exploring this area. It will include a treatment of necessary 
conditions for a quantum routine, its inputs and outputs, which would allow transforming
the routine into either an iterated or folded routine. Algorithms to compute this transform
are also provided.

\section*{Synthesis of Quantum transformations}
This chapter will discuss the issue of gate synthesis, where an arbitrary quantum transform is to
be expressed in terms of a set of base gates. The histories of both approximate and exact 
synthesis will be reviewed.

Then, the chapter will present an algorithm for exact synthesis of single-qubit transforms over
the Clifford group, together
with a normal form and characterization of these. This work is an extension of work done by
Matsumoto and Amano and is expected to be submitted for publication soon. This was joint work with
P. Selinger

Finally, we will present an algorithm for exact synthesis over the Clifford group
of multi-qubit transfroms and characterize those transforms that may be exactly synthesized. This
was joint work with P. Selinger and was published in the Physical Review A. 

\section*{Conclusion and future work}

\begin{itemize}
  \item Highlight the similarities between reversible and quantum computing.
  \item Give examples of discrete inverse categories that use some quantum structures.
  \item Discuss n-qubit algorithm and improvments.
  \item Discuss possibilities of normal form in n-qubit case.
\end{itemize}