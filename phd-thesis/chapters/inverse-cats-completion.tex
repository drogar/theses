%!TEX root = /Users/gilesb/UofC/thesis/phd-thesis/phd-thesis.tex
\chapter{From a discrete inverse category to a Cartesian restriction category} % (fold)
\label{chap:completing_a_discrete_inverse_category}

The purpose of this chapter is to prove that the category of discrete inverse categories is
equivalent to the the category of discrete Cartesian restriction categories. We will show
how to construct a discrete Cartesian restriction category, \Xt, from a discrete inverse
category, \X. Then, we will give the equivalence functors between \X and \Xt.

\section{The restriction category \hypXt} % (fold)
\label{sec:the_restriction_category_hypxt}


We begin by giving the construction of $\Xt$. This is followed by the details of the equivalence
relation and then the proof that $\Xt$ is a discrete Cartesian restriction category.

\begin{definition}\label{def:xt}
  When \X is an inverse category, define \Xt\ as:
  \categoryns{objects as in \X;}
  {
    A map $(f,C):A\to B$ in \Xt is the equivalence class of the map $f:A\to B\*C$ in \X (detailed below in
    Definition~\ref{def:xequivalence}). We have the following relationship between maps in \Xt and \X: %
    \[
      \infer{A\xrightarrow{f} B\*C \text{ in }\X}{A \xrightarrow{\ (f,C)\ } B \text{ in } \Xt};
    \]
  }
  {% identity
    by
    \[
      \infer{A\xrightarrow{\inv{u_{\*}^r}}A\* 1}
            {A \xrightarrow{(\inv{u_{\*}^r}, 1)} A};
    \]
  }
  {% composition
    given by
    \[
      \infer{
        \infer{A\xrightarrow{(f (g\*1) a_{\*},C' \* B')} C}
              {A\xrightarrow{f (g\*1) a_{\*}} C \* (C' \* B')}
            }
            {A \xrightarrow{\ (f,B')\ } B \xrightarrow{\ (g,C')\ } C}.
    \]
  }

\end{definition}

When considering an \Xt\ map $(f,C):A\to B$ in \X, we occasionally use the notation $f:A\to
\xtdmn{B}{C}$ ($\equiv f:A\to B\* C$).

\subsection{Equivalence classes of maps in \hypX} % (fold)
\label{sec:equivalence_classes_of_maps_in_hypx}


\begin{definition}\label{def:xequivalence}
  In a discrete inverse category \X, the map $f$ is equivalent to $f'$ in \X when
  $\restr{f} = \restr{f'}$ in \X and Diagram~\ref{fig:xt_map_equivalence_diagram} commutes for some
  map $h\in\dmap{B}$.
\begin{figure}[H]
  \[
    \xymatrix @C=40pt @R=15pt{
      & B \* C  \ar@{.>}[dd]^{h} \\
      A \ar[ur]^f \ar[dr]_{f'}\\
      & B\* C'
    }
  \]
\caption{Equivalence diagram for constructing maps in \Xt.}
\label{fig:xt_map_equivalence_diagram}
\end{figure}
\end{definition}

\begin{notation}\label{notn:xequivalence}
  When $f$ is equivalent to $g$ as in Definition~\ref{def:xequivalence} via the mediating map $h$,
  this is written as:
  \[
    f\xequiv{h}g.
  \]
\end{notation}

\begin{lemma}\label{lem:mediating_map_equivalence_is_symmetric_reflexive_and_transitive}
  Definition~\ref{def:xequivalence} gives a symmetric, reflexive equivalence class of maps in \X.
\end{lemma}
\begin{proof}
  \prepprooflist
  \begin{description}
    \itembf{Reflexivity: } Choose $h$ as the identity map.
    \itembf{Symmetry: } Suppose $f\xequiv{h}g$. Then, $\restr{f} = \restr{g}$ and $f h = g$.
      Applying $\inv{h}$, we have
      \[
        g \inv{h} = f h \inv{h} = f \restr{h} = \restr{f h} f
        = \restr{g} f = \restr{f} f = f.
      \]

      Thus, $g\xequiv{\inv{h}} f$.

    \itembf{Transitivity: } Suppose $f\xequiv{h} f'$ and $f' \xequiv{k} f''$, i.e., $fh = f'$ and
    $f'k = f''$. We immediately have $f h k = f''$ and by
    Lemma~\ref{lem:delta_nabla_maps_are_closed}, we know $h k = \dmap{(hk)}$ and therefore we have
    an equivalence.
  \end{description}
\end{proof}

% end_fold sec:equivalence_classes_of_maps_in_hypx

\subsection{\hypXt is a restriction category} % (fold)
\label{subsec:the_restriction_category_hypxt}


\begin{lemma}\label{lem:xt_is_a_category}
  \Xt\ as defined above is a category.
\end{lemma}
\begin{proof}
  The maps are well defined, as shown in
  Lemma~\ref{lem:mediating_map_equivalence_is_symmetric_reflexive_and_transitive}. The existence of
  the identity map is due to the tensor $\*$ being defined on \X, an inverse category, hence
  $\inv{\usr}$ is defined.

  It remains to show the composition is associative and that $(\inv{\usr}, 1)$ acts as an identity
  in \Xt. For all of these, we will make use of
  Lemma~\ref{lem:delta_nabla_maps_are_closed}~\ref{lemitem:delta_nabla_3},  which states $1\*f \in
  \dmap{A}$ for all $f$.

  \textbf{Associativity:}
  Consider
  \[
    A\xrightarrow{(f,B')}B\xrightarrow{(g,C')} C \xrightarrow{(h,D')}D.
  \]

  To show the associativity of this in \Xt, we need to show in \X that
  \[
    \restr{(f(g\*1)a_{\*}) (h\*1)a_{\*}} = \restr{f (((g(h\*1)a_{\*})\*1) a_{\*})}
  \]
  and that there exists a mediating map between the two of them.

  To see that the restrictions are equal, first note that by the functorality of $\*$, for any two
  maps $u$ and $v$, we have $u v \* 1 = (u\*1) (v\*1)$. Second, the naturality of $a_{\*}$ gives us
  that $a_{\*} (h \* 1) = ((h\*1)\*1) a_{\*}$. Thus,
  \begin{align*}
    \restr{f(g\*1)a_{\*} (h\*1)a_{\*}}
      & = \restr{f(g\*1)a_{\*} (h\*1)\restr{a_{\*}}}
    & \text{Lemma \ref{lem:restrictionvarious}} \\
    & = \restr{f(g\*1)a_{\*} (h\*1)} & \restr{a_{\*}}=1 \\
    & = \restr{f(g\*1) ((h\*1)\*1) a_{\*} } & a_{\*}\text{ natural} \\
    & = \restr{f(g\*1) ((h\*1)\*1) }
      &  \text{Lemma \ref{lem:restrictionvarious}}\\
    & = \restr{f(g\*1) ((h\*1)\*1) (a_{\*}\*1)}
      & \text{Lemma \ref{lem:restrictionvarious}}\\
    & = \restr{f((g (h\*1)a )\*1)} & \text{ see above}\\
    & = \restr{f((g (h\*1)a )\*1)a_{\*}} & a_{\*}.
  \end{align*}

  For the mediating map, see the diagram below, where the calculation is in \X. The path starting at
  the top left at $A$ and going right to \xtdmn{D}{D' \* (C' \* B')} is grouping parentheses to the
  left. Starting at $A$ and then going down to $\xtdmn{(\xtdmn{D}{D' \* C'})}{B'}$ followed by
  right to $ \xtdmn{D}{(D' \* C') \* B'}$ is grouping parentheses to the right. The
  commutativity of the diagram is shown by the commutativity of the internal portions, which all
  follow from the standard coherence diagrams for the tensor and naturality of association.

  \[
    \xymatrix @C=35pt @R=35pt{
      A \ar[d]_f \ar[r]^(.4){f (g\*1) a_{\*}} \ar[dr]^(.4){f(g\*1)} &
        \xtdmn{C}{C' \* B'} \ar[r]^{h\*1}
        & \xtdmn{(\xtdmn{D}{D'})}{C' \* B'} \ar[r]^{a_{\*}}
        & \xtdmn{D}{D' \* (C' \* B')}
        \ar@{.>}[dd]^{1\*\inv{a_{\*}}}\\
      \xtdmn{B}{B'} \ar[d]_{(g (h\*1) a_{\*})\*1} \ar[r]^{g\*1}
        &\xtdmn{(\xtdmn{C}{C'})}{B'} \ar[u]_{a_{\*}} \ar[r]^(.4){(h\*1)\*1}
        & \xtdmn{(\xtdmn{(\xtdmn{D}{D'})}{C'})}{B'}
        \ar[u]_{a_\*} \ar[dll]_{a_{\*}\*1}
      \\
      \xtdmn{(\xtdmn{D}{D' \* C'})}{B'}  \ar[rrr]_{a_{\*}}
        &&& \xtdmn{D}{(D' \* C') \* B'}. %\ar @/^14pt/ [uu]^{1\*a_{\*}}
    }
  \]
  Therefore, we can conclude
  \[
    (f(g\*1)a_{\*}) (h\*1)a_{\*} \xequiv{1\*\inv{a_{\*}}} f (((g(h\*1)a_{\*})\*1) a_{\*})
  \]
  which gives us that composition in \Xt is associative.

  \textbf{Identity:} This requires:
  \[
    (f,C) (\inv{\usr}, 1) = (f,C) = (\inv{\usr}, 1) (f,C)
  \]
  for all maps $A\xrightarrow{(f,C)}B$ in \Xt.
  By Lemma~\ref{lem:restrictionvarious} we have
  $\restr{f (\inv{\usr}\*1) a_{\*}} = \restr{f}$. Then, calculating in \X, we have a mediating map
  of $1 \* \usl$ as shown below.
  \[
    \xymatrix @C=50pt @R=45pt{
      A \ar[r]^f \ar[ddrrr]_f&
        B \* C \ar[r]^(.4){\inv{\usr}\*1}
        \ar@/_20pt/[rr]_{1 \* \inv{\usl}}
        \ar@{=}[ddrr]
        & (B \* 1) \* C \ar[r]^{a_{\*}}
        & B \* (1 \* C) \ar@{.>}[dd]^{1 \* \usl} \\
      \\
      &&& B \* C.
    }
  \]
  $\restr{\inv{\usr} (f \*1)  a_{\*}} = \restr{f}$ by the naturality of $\inv{\usr}$ and
  Lemma~\ref{lem:restrictionvarious}. The diagram
  \[
    \xymatrix @C=50pt @R=45pt{
      A \ar@{=}[dd] \ar[r]^{\inv{\usr}} \ar[dr]_{f}
        &      A \* 1 \ar[r]^{f \* 1}
        & (B \* C) \* 1 \ar[r]^{a_{\*}}
        & B \* (C \* 1)\ar@{.>}[dd]^{1 \* \usr} \\
      &B\*C \ar[ur]^{\inv{\usr}} \ar[urr]_{1\*\inv{\usr}} \ar@{=}[drr]\\
      A \ar[rrr]^f &&& B \* C
    }
  \]
  shows our mediating map is $1 \* \usr$.
\end{proof}


Define the restriction in \Xt\ as follows:
\[
  \infer{A\xrightarrow{\restr{f}  \inv{u_{\*}^r}} A\*1 \text{ in }\X.}
        {\infer{A\xrightarrow{\restr{(f,C)}}A}
               {A\xrightarrow{(f,C)}B}
        }
\]

\begin{lemma}\label{lem:xt_is_a_restriction_category}
  The category \Xt with restriction defined as above is a restriction category.
\end{lemma}
\begin{proof}
  Given the above definition the four restriction axioms must now be checked. For the remainder of
  this proof, all diagrams will be in $\X$. Again, we will make use of
  Lemma~\ref{lem:delta_nabla_maps_are_closed}~\ref{lemitem:delta_nabla_3},  which states $1\*f \in
  \dmap{A}$ for all $f$.

  \rone ($\restr{f} f = f$). Calculating the restriction of the left hand side in
      \X, we have:
      \begin{align*}
        \restr{\rst{f}\inv{\usr} (f\*1) a_{\*}} & = \restr{\rst{f}\inv{\usr} (f\*1)}
          & \text{Lemma \ref{lem:restrictionvarious}}\\
        & = \restr{\rst{f}f \inv{\usr}}  & \inv{\usr} \text{ natural}\\
        & = \restr{f \inv{\usr}}  & \text{ \rone in }\X\\
        & = \restr{f } & \text{Lemma \ref{lem:restrictionvarious}}.
      \end{align*}

      Then, the following diagram
      \[
        \xymatrix @C=40pt @R=25pt{
          A \ar[r]^{\restr{f} \inv{\usr}}
          \ar @/_25pt/[ddrrr]_f  \ar[drr]^{\restr{f}f}
          &A \* 1 \ar[r]^{f\* 1}
          &(A \* B) \* 1 \ar[r]^{a_{\*}} \ar[ddr]^{\usr}
          &A \* (B \* 1) \ar@{.>}[dd]^{1 \* {\usr}}\\
          &&A\*B \ar[u]^{\inv{\usr}} \ar@{=} [dr]\\
          && &A \* B
        }
      \]
      shows $\rst{f}\inv{\usr} (f\*1) a_{\*} \xequiv{1 \* {\usr}} f$ in \X and therefore $\rst{f}f
      = f$ in \Xt.

     \rtwo ($\restr{g} \restr{f} = \restr{f} \restr{g}$). We must show
      \begin{equation}
        \rst{f}\inv{\usr} ((\restr{g}\inv\usr)\*1)) a_{\*} \xequiv{}
        \rst{g}\inv{\usr}((\restr{f}\inv\usr)\*1) a_{\*}.
        \label{eq:rtwo_in_xt}
      \end{equation}
      The restriction of the left hand
      side equals the restriction of the right hand side as seen below:
      \begin{align*}
        \restr{\rst{f}\inv{\usr} ((\restr{g}\inv\usr)\*1)) a_{\*}}
        & = \restr{\rst{f}(\restr{g}\inv\usr)\inv{\usr} a_{\*}} & \inv{\usr}   \text{ natural}\\
        & = \restr{\rst{g}\restr{f}\inv\usr\inv{\usr} a_{\*}} &  \text{\rtwo in }\X\\
        & = \restr{\rst{g}\inv{\usr}((\restr{f}\inv\usr)\*1) a_{\*}} & \inv{\usr}   \text{natural}.
      \end{align*}

      %\note{!!!!! Above calc works without being under restr, therefore have we not just shown
      %that $\restr{g}  \restr{f} = \restr{f}  \restr{g}$ ????
      %Note that the mediating map is $id$!!!!!!!!!}

      The below diagram commutes by the naturality of $\usr$ and the tensor coherence,
      \[
        \xymatrix @C=48pt @R=55pt{
          A \ar[r]^{\restr{g} \inv{\usr}}
            \ar[dr]_{\restr{f}\restr{g}}^{\restr{g}\restr{f}}
            \ar[d]_{\restr{f}\inv{\usr}}
            &A\*1 \ar[r]^(.4){(\rst{f}\inv{\usr})\*1}
            & (A \*1) \* 1
            \ar[r]^{a_{\*}}  \ar[dl]_{\usr \usr}
            & A \* (1\*1) \ar@{.>}[dd]_{1\*id}\\ %\ar @/^25pt/ @{=}[dd]
          A\*1 \ar[d]_{(\rst{g}\inv{\usr})\*1}
            &A \ar@/_25pt/[ur]_{\inv{\usr}\inv{\usr}} \\
             %\ar@/^25pt/[dl]^(.3){\inv{\usr}\inv{\usr}}\\
          (A\*1)\*1 \ar[rrr]_{a_\*} \ar[ur]^{\usr \usr}
            &&& A \* (1\*1)
        }
      \]
      which allows us to conclude $\rst{f} \rst{g} = \rst{g} \rst{f}$ in \Xt.

    \rthree ($\restr{\restr{f} g} = \restr{f} \restr{g}$ ). We must show
      \begin{equation}
        \restr{(\restr{f} \inv{\usr}) (g\* 1) a_{\*}} \inv{\usr} \xequiv{}
        (\restr{f} \inv{\usr})(\rst{g} \inv{\usr}\* 1) a_{\*}.
        \label{eq:rthree_in_xt}
      \end{equation}
      As above, the first step is
      to show that the restrictions of each side of Equation~\ref{eq:rthree_in_xt} are the same.
      Computing the restriction of the left
      hand side in \X:
      \begin{align*}
        \rst{\restr{(\restr{f} \inv{\usr}) (g\* 1) a_{\*}} \inv{\usr}}
        & = \rst{\restr{(\restr{f} \inv{\usr}) (g\* 1) a_{\*}}} & \text{Lemma \ref{lem:restrictionvarious}}\\
        & = \restr{(\restr{f} \inv{\usr}) (g\* 1) a_{\*}} &
          \text{Lemma \ref{lem:restrictionvarious}}\\
        & = \restr{\restr{f} g \inv{\usr} a_{\*}} & \inv{\usr} \text{ natural}\\
        & = \restr{\restr{f} g } & \text{Lemma \ref{lem:restrictionvarious}}\\
        & = \restr{f} \rst{g} & \text{\rthree in }\X.
      \end{align*}
      The restriction of the right hand side computes in \X as:
      \begin{align*}
        &\rst{(\restr{f} \inv{\usr})(\rst{g} \inv{\usr}\* 1) a_{\*}}\\
        & = \rst{(\restr{f} \inv{\usr}) (\rst{g} \inv{\usr}\* 1) } &  \text{Lemma \ref{lem:restrictionvarious}}\\
        & = \rst{\restr{f}  \rst{g} \inv{\usr}\inv{\usr} } &  \inv{\usr} \text{ natural}\\
        & = \rst{\restr{f}  \rst{g} } &\text{Lemma \ref{lem:restrictionvarious}}\\
        & = \restr{f} \rst{g} & \text{Lemma \ref{lem:restrictionvarious}}.
      \end{align*}

      Additionally, we see $\rst{\rst{f} g}$ in \Xt is expressed in \X as:
      \begin{align*}
        &\restr{(\restr{f} \inv{\usr})(g\* 1) a_{\*}} \inv{\usr} \\
        & = \rst{f} \inv{\usr} \rst{g\* 1} & \text{\rthree, \rfour} \\
        & = \rst{f} \rst{g} \inv{\usr} & \*
          \text{a restriction bi-functor, }\inv{\usr}\text{ natural.}
      \end{align*}

      The following diagram in \X follows the right hand side of Equation~\ref{eq:rthree_in_xt}
      with the top curved arrow and the left hand side of Equation~\ref{eq:rthree_in_xt} with the
      bottom curved arrow. Note that we are using that
      $\restr{(\restr{f} \inv{\usr}) (g\* 1) a_{\*}} = \restr{f}\rst{g}$ as shown above.
      \[
        \xymatrix @C=33pt @R=20pt{
          A \ar@/^45pt/[rrrrr]^{\restr{f} \inv{\usr}(\restr{g} \inv{\usr} \* 1)a_{\*}}
            \ar@/_65pt/[dddddrrrrr]_{\restr{f}\, \restr{g}\inv{\usr}}
            \ar[r]^(.7){\restr{f}}
            \ar[dr]_(.7){\restr{f}}
            &A \ar[r]^{\inv{\usr}}
            \ar@{=}[d]
            &A \* 1 \ar[r]^{\restr{g}\*1}
            \ar[dd]_{\usr}
            &A \* 1 \ar[r]^(.41){\inv{\usr}\* 1}
            \ar[ddd]_{\usr}
            &(A \* 1) \* 1 \ar[r]^{a_{\*}}
            \ar[dddd]_{\usr}
            & A \* (1\*1) \ar@{.>}[ddddd]^{1\*\usr}\\
          &A\ar@{=}[dr]\\
          &&A \ar[dr]_{\restr{g}}\\
          &&&A \ar[dr]_{\inv\usr}\\
          &&&&A\*1 \ar@{=}[dr]\\
          &&&&&A\*1.
        }
      \]
      Hence, in \X, $\restr{(\restr{f} \inv{\usr}) (g\* 1) a_{\*}} \inv{\usr} \xequiv{1\*\usr}
      (\restr{f} \inv{\usr}) (\rst{g} \inv{\usr}\* 1) a_{\*}$ and therefore $\restr{\restr{f} g} =
      \restr{f} \restr{g}$ in \Xt.

      \rfour ($f \restr{g} = \restr{f g} f$). We must show
      \begin{equation}
        f (\restr{g} \inv{\usr}\* 1) a_{\*} \xequiv{}
        \rst{f (g \* 1)} \inv{\usr} (f\* 1) a_{\*}.
        \label{eq:rfour_in_xt}
      \end{equation}
      The restriction of the left hand side of Equation~\ref{eq:rfour_in_xt} is:
      \begin{align*}
        \rst{f (\restr{g} \inv{\usr}\* 1) a_{\*}}
          & = \rst{f (\restr{g} \inv{\usr}\* 1)} & \text{Lemma \ref{lem:restrictionvarious}} \\
        & = \rst{f \rst{g} \inv{\usr}} \* \rst{f} & \* \text{ restriction functor}\\
        & = \rst{f \rst{g}} \* \rst{f} & \text{Lemma \ref{lem:restrictionvarious}} \\
        & = \rst{f (\rst{g} \* 1)}
      \end{align*}
      and the restriction of the right hand side of Equation~\ref{eq:rfour_in_xt}  is:
      \begin{align*}
        \rst{\rst{f (g \* 1)} \inv{\usr} (f\* 1) a_{\*}}
          & = \rst{\rst{f (g \* 1)} \inv{\usr} (f\* 1) } &  \text{Lemma \ref{lem:restrictionvarious}}\\
        & = \rst{\rst{f (g \* 1)} f \inv{\usr}  } & \inv{\usr} \text{ natural}\\
        & = \rst{f \rst{(g \* 1)}  \inv{\usr}  } & \text{ \rfour for }\X\\
        & = \rst{f (\rst{g} \* 1)  \inv{\usr}  } & \* \text{ a restriction functor}\\
        & = \rst{f (\rst{g} \* 1)    } & \text{Lemma \ref{lem:restrictionvarious}.}
      \end{align*}
      Computing the right hand side of Equation~\ref{eq:rfour_in_xt} in \X,
      \begin{align*}
        \rst{f(g\* 1)a_{\*}} \inv{\usr} (f\* 1) a_{\*}
          & = \rst{f(g\* 1)} f \inv{\usr} a_{\*} & \inv{\usr}\text{ natural,}\\
        & = f (\rst{g} \* 1) \inv{\usr} a_{\*} & \rfour.
      \end{align*}

      Thus,
      \[
        \xymatrix @C=40pt @R=35pt{
          A \ar[r]^f \ar[dr]_{f}
            & B\* C \ar[rr]^{\restr{g} \inv{\usr}\* 1}
            &
            & (B\* 1) \* C \ar[r]^{a_{\*}}
            & B \* (1 \* C)\ar@{.>}[d]^{1\*c_{\*}}\\
          & B\*C \ar[r]_{\rst{g}\*1}
            & B \* C \ar[r]_{\inv{\usr}}
            & (B \* C) \* 1 \ar[r]_{a_{\*}}
            & B \* (C\*1)
        }
      \]
  and hence, \Xt\ is a restriction category.
\end{proof}
% subsection the_restriction_category_hypxt (end)
\subsection{\hypXt is a discrete Cartesian restriction category} % (fold)
\label{sec:the_category_hypxt_is_cartesian}



\begin{lemma}\label{lem:tensor_unit_of_x_is_terminal_object_of_xt}
  The unit of the inverse product in \X is the terminal object in \Xt.
\end{lemma}
\begin{proof}
  The unique map to the terminal object for any object $A$ in \Xt is the equivalence class of maps
  represented by $(\inv{\usl},A)$. For this to be a terminal object, the diagram
  \[
    \xymatrix @C=40pt @R=25pt{
      X \ar[r]^{\restr{(f,C)}} \ar[d]^{(f,C)} & X \ar[r]^{!_X}  &\top  \\
      Y \ar[urr]_{!_Y}
    }
  \]
  must commute for all choices of $f$. Translating this to \X, this is the same as requiring
  \[
    \xymatrix @C=40pt @R=25pt{
      X \ar[r]^{\restr{f}} \ar[d]^{f} & X \ar[r]^{\inv{\usr} }
      & X\*1 \ar[r]^{\inv{\usl}}  &1\*X\*1 \ar@{.>}[dl]_{1\*(\usr f)}  \\
      Y \*C\ar[rr]_{\inv{\usl}} & & 1 \*Y \* C
    }
  \]
  to commute, which is true by \rone and from the coherence diagrams for the inverse product tensor.
\end{proof}

Next,we show that the category \Xt\ has restriction products, given by the action of \wtc on
the $\*$ tensor in \X.

First, define total maps $\pi_0$, $\pi_1$ in \Xt by:
\begin{align}
  \pi_0:\qquad & A \* B \xrightarrow{(1,B)} A, \label{eq:defn:pia}\\
  \pi_1:\qquad & A \* B \xrightarrow{(c_{\*},A)} B. \label{eq:defn:pib}
\end{align}

\begin{definition}\label{def:product_map_in_xt}
  Given a discrete inverse category $\X$, suppose we are given the maps $ Z \xrightarrow{(f,C)} A$
  and $Z \xrightarrow{(g,C')} B$ in $\Xt$. Then define $\<(f,C),(g,C')\>$as
  \begin{equation}
    Z\xrightarrow{(\Delta  ( f \* g )  (1\* c_{\*} \* 1), C\* C')} A \* B\label{eq:defn:fg}
  \end{equation}
  where associativity is assumed as needed. Note that with the associativity maps, this is actually:
  \begin{equation}
    Z\xrightarrow{(\Delta  ( f \* g )  a_{\*} (1\*\inv{a_{\*}})
      (1\* (c_{\*} \* 1)) (1\*a_{\*}) \inv{a_{\*}}, C\* C')} A \* B.\label{eq:defn:fg2}
  \end{equation}
\end{definition}

\begin{lemma}\label{lem:tensor_on_x_is_the_restriction_product_on_xt}
  On \Xt, $\*$ is a restriction product with projections $\pi_0, \pi_1$ and the product of maps
  $f, g$ being $\<f,g\>$.
\end{lemma}
\begin{proof}
  From the definition above, as $1$ and $c_{\*}$ are isomorphisms, the maps $\pi_0, \pi_1$ are
  total.

  In order to show that $\rst{\<f,g\>} = \rst{f}\,\rst{g}$, first reduce the left hand side:
  \begin{align*}
    \rst{\<f,g\>}
      &=\rst{\Delta(f\*g)(1\*c_{\*}\*1)}\inv{\usr}&\text{in }\X, \text{ definition of restriction}\\
    &=\rst{\Delta(f\*g)}\inv{\usr} &\\
    &=\rst{\Delta\rst{(f\*g)}}\inv{\usr} &\text{from Lemma \ref{lem:restrictionvarious}}\\
    &=\rst{\Delta(\rst{f}\*\rst{g})}\inv{\usr} &\*\text{ is a restriction functor}\\
    &=\rst{\rst{f}\,\rst{g}\,\Delta(1\*1)}\inv{\usr}
      &\text{Lemma  \ref{lem:properties_of_delta_and_tensor_in_a_discrete_inverse_category}(\ref{le:deltaefg}) twice}\\
    &=\rst{\rst{f}\,\rst{g}}\inv{\usr} &\text{Lemma  \ref{lem:restrictionvarious}}\\
    &=\rst{f}\,\rst{g}\inv{\usr}  &\text{Lemma  \ref{lem:restrictionvarious}.}\\
  \end{align*}

  Then, the right hand side reduces as:
  \begin{align*}
    \rst{f} \rst{g}
    &= \rst{f}\inv{\usr}(\rst{g}\inv{\usr} \* 1) a_{\*} & \text{in \X by definitions}\\
    &= \rst{f} \rst{g}\inv{\usr}\inv{\usr} a_{\*} &  \inv{\usr}\text{ natural.}
  \end{align*}
  The restriction of the left hand side and the right hand side, in \X, is $\rst{\rst{f} \rst{g}}$.
  This is done by applying Lemma~\ref{lem:restrictionvarious} once on the left and
  thrice on the right.

  Thus, this shows $\rst{\<f,g\>}=\rst{f} \rst{g}$ in \Xt where the mediating map in \X is
  $1\*\usr$.

  Next, to show $\<f,g\> \pi_0 \le f$ (and $\<f,g\> \pi_1 \le g$), it is required to show
  $\rst{\<f,g\>\pi_{0}} f = \<f,g\> \pi_{0}$. Calculating the left side, we see:
  \begin{align*}
    \rst{\<f,g\>\pi_{0}} f &=\rst{\<f,g\>\rst{\pi_{0}}} f &\text{Lemma \ref{lem:restrictionvarious}}\\
    &=\rst{\<f,g\>} f &\pi_{0}\text{ is total}\\
    &=\rst{f}\,\rst{g}\, f&\text{ by above}\\
    &=\rst{g} \rst{f} f & \rtwo\\
    &=\rst{g} f& \rone.
  \end{align*}
  Now, turning to the right hand side:
  \begin{align*}
    \<f,g\>\pi_{0} &= \Delta(f\*g)(1\*c_{\*}\*1) 1 &\text{in \X, by definition.}
  \end{align*}
  To show these are equal in \Xt, we need to first show the restrictions are the same in \X and
  then show there is a mediating map between the images in \X. The restriction of $\rst{g} f$ is
  $\rst{f} \rst{g}$ immediately by \rthree and \rtwo. For the right hand side, calculate in \X:
  \begin{align*}
    \rst{\Delta(f\*g)(1\*c_{\*}\*1)}
      & = \rst{\Delta(f\*g)} & \text{Lemma  \ref{lem:restrictionvarious}}\\
    & = \Delta(f\*g)(\inv{f} \* \inv{g})\inv{\Delta} & \X \text{ is an inverse category}\\
    & = \Delta(\rst{f} \* \rst{g})\inv{\Delta} & \\
    & = \rst{f} \rst{g} \Delta\inv{\Delta}
      & \text{Lemma \ref{lem:properties_of_delta_and_tensor_in_a_discrete_inverse_category}(\ref{le:deltaefg}) twice}\\
    & = \rst{f} \rst{g}.
  \end{align*}

  The diagram below shows the required mediating map. By
  Lemma~\ref{lem:delta_nabla_maps_are_closed}, $\Delta\in\dmap{A}$, $1\*k\in\dmap{A}$ and $\dmap{A}$
  is closed under composition.
  \[
    \xymatrix @C=27pt @R=25pt{
      & &  A \* C\ar@{.>}[d]^{\Delta}\\
      &Z\ar[ur]^{f}&  A\* C \* A \* C\ar@{.>}[d]^{1\*\inv{f}}\\
      Z\ar[ur]^{\rst{g}}\ar[dr]_{\Delta(f\*g)}& &  A \*C \* Z \ar@{.>}[d]^{1 \* 1 \* g}\\
      &A\*C\*B\* C'\ar[dr]_{1\*c_{\*}\*1}&  A\*C\*B\*C'\ar@{.>}[d]^{1\*c_{\*}\*1}\\
      & & A\*B\*C\*C'.
    }
  \]
\end{proof}

% subsection the_category_hypxt_is_cartesian (end)

At this point, we have shown that \Xt is a restriction category with restriction products. This
leads us to the following theorem:

\begin{theorem}\label{thm:xt_is_a_discrete_crc_when_x_is_an_inverse_category}
  For any inverse category \X, the category \Xt is a discrete restriction category.
\end{theorem}
\begin{proof}
  The fact that \Xt is a Cartesian restriction category is immediate from
  Lemmas~\ref{lem:xt_is_a_category}, \ref{lem:xt_is_a_restriction_category},
  \ref{lem:tensor_unit_of_x_is_terminal_object_of_xt} and
  \ref{lem:tensor_on_x_is_the_restriction_product_on_xt}.

  To show that it is discrete, we need only show that the map $(\Delta \inv{\usr},1)$ is in the
  same equivalence class as $\Xt$'s $\Delta(= \<1,1\> = \<(\inv{\usr},1),(\inv{\usr},1))$. As both
  $\Delta$ and $\inv{\usr}$ are total, the restriction of each side is the same, namely $1$. The
  diagram below uses Lemma~\ref{lem:delta_nabla_maps_are_closed}~\ref{lemitem:delta_nabla_3} and
  shows that the two maps are in the same equivalence class.
  \[
    \xymatrix @C=190pt @R=40pt{
      & A \* A \* 1 \ar@{.>}[d]^{1\*\inv{\usr}}\\
      A \ar[ur]^{\Delta\inv{\usr}}
        \ar[r]_{\Delta(\inv{\usr}\*\inv{\usr})(1\*c_{\*}\*1)}& A\*A\*1\*1
    }
  \]
\end{proof}
\section{Equivalence between discrete inverse and discrete Cartesian restriction categories} % (fold)
\label{sec:equivalence_of_categories}

This section will show that the category of discrete inverse categories (with functors that preserve
the inverse tensor) is equivalent to the category of discrete restriction categories (with
restriction functors which preserve the product).

We will follow this path to showing the equivalence:
\begin{enumerate}
\item Give a functor \Invf from discrete Cartesian restriction categories to discrete inverse
  categories and show that it is full and faithful.
\item Draw the universal diagram:
\begin{equation}
  \xymatrix @C=65pt @R=40pt{
    \X \ar[r]^{\eta} \ar[rd]_{F}& \Inv{\Xt} \ar[d]^{\Inv{F^{\#}}}\\
    &\Inv{\D}
  }
  \label{dia:universal_property_of_inverse_categories}
\end{equation}
and show that there exists a functor $F^{\#}:\Xt \to \D$ which makes the diagram commute. As we have
$\Invf$ full and faithful, we may conclude it is unique.
\item Show that $\eta$ in Diagram~\ref{dia:universal_property_of_inverse_categories} is an isomorphism.
\end{enumerate}
Once we have completed these steps, we may then conclude that there is an equivalence of discrete
inverse categories to discrete Cartesian restriction categories.

In the following, $\X$
will always be a discrete inverse category, $\D$ and $\B$ will be discrete restriction categories.

The functor $\Invf$ maps a discrete restriction category to its inverse subcategory and
maps functors between discrete restriction categories to a functor having the same action on the
partial inverses. That is, given $G:\B \to \D$, then:
\begin{align*}
  &\Inv{G}: \Inv{\B} \to \Inv{\D}\\
  &\Inv{G}(A) = G{A}&\text{(all objects of $\D$ are in $Inv(\D)$)}\\
  &\Inv{G}(f) = G(f)&\text{(restriction functors preserve partial inverse)}.
\end{align*}
\begin{lemma}\label{lem:inv_is_full_and_faithful}
  The functor $\Invf$ from the category of discrete restriction categories to the category of
  discrete inverse categories is full and faithful.
\end{lemma}

\begin{proof}
  To show fullness, we must show $\Invf$ is surjective on hom-sets. Given a functor between two
  categories in the image of $\Invf$, i.e., $G:\Inv{\B}\to \Inv{\D}$, construct a functor
  $H:\B\to\D$ as follows:
  \begin{description}
    \item{Action on objects:} $H(A) = G(A),$
    \item{Objects on maps:} $H(f) = G(\<f,1\>)\pi_0.$
  \end{description}
  $H$ is well defined as we know $\<f,1\>$ is an invertible map and therefore in the domain of $G$.
  To see $H$ is a functor:
  \begin{align*}
    H(1) &= G(\<1,1\>)\pi_0 = \Delta_{\D}\pi_0 = 1,\\
    H(f g) &= G(\<f g,1\>)\pi_0 = \<G(fg),1\>\pi_0 = G(f)\<G(g,1)\>\pi_0 \\
    &= \<G(f),1\>\pi_0 \<G(g),1\>\pi_0 = G(\<f,1\>)\pi_0 G(\<g,1\>)\pi_0 = H(f)H(g).
  \end{align*}
  But on any invertible map, $H(f) = G(\<f,1\>)\pi_0 = \<G(f),1\>\pi_0 = G(f)$ and therefore
  $\Inv{H} = G$, so $\Invf$ is full.

  Next, assume we have $F,G:\B\to\D$ with $\Inv{F} = \Inv{G}$. Considering $F(f)$ and $F(g)$, we
  know $F(\<f,1\>) = G(\<f,1\>) $ as $\<f,1\>$ is invertible. Thus, as the functors preserve the
  product structure, we have
  \[
    F(f) = F(\<f,1\>)F(\pi_0)= G(\<f,1\>)G(\pi_0)= G(f).
  \]
  Thus, $\Invf$ is faithful.
\end{proof}


Next we define  $\eta:\X \to \Inv{\Xt}$ as the identity on objects functor where for a map $f$ in
\X, $\eta: f \mapsto (f\inv{\usr},1)$. $\eta$ is a functor as
\begin{align*}
  \eta(1) & = (\inv{\usr},1)\\
  \eta(fg) &= (fg\inv{\usr},1) \xequiv{} (f\inv{\usr},1)(g\inv{\usr},1) = \eta{f}\eta{g}.
\end{align*}

Now, we may define the functor $F^{\#}:\Xt \to \D$. Recall that $\Inv{\D}$ is a sub-category of $\D$
having the same objects, but only the invertible maps. Given a functor $F:\X \to \Inv{\D}$ define
$F^{\#}$ as follows:
\begin{description}
  \item{Objects:} $F^{\#}:A \mapsto F(A) \in \D_{o}$
  \item{Arrows:} $F^{\#}:(f,C) \mapsto F(f\pi_0)\in \D_{m}$
\end{description}

In order to show \ref{dia:universal_property_of_inverse_categories} is a universal diagram, we
proceed with a series of lemmas building to the result.


\begin{lemma}\label{lem:universal_diagram_is_a_commutative_diagram}
  Diagram~\ref{dia:universal_property_of_inverse_categories} above commutes and is a universal diagram.
  That is,
  \[
    \eta\Inv{F^{\#}} = F
  \]
  and $F^{\#}$ is unique.
\end{lemma}
\begin{proof}
  Using our definitions above, given a map $f$ in \X, then:
  \begin{align*}
    \Inv(F^{\#})(\eta(f) &= \Inv(F^{\#})((f\inv{\usr},1))\\
    &= F^{\#}((f\inv{\usr},1))\\
    &= F(f\inv{\usr}\pi_0)\\
    &= F(f)\\
  \end{align*}
  % Chasing maps around the diagram, we have:
  % \[
  %   \xymatrix @C=35pt @R=40pt{
  %     f \ar@{|->}[rr]^{\eta} \ar@{|->}[rd]_{F}&& (f\inv{\usr},1) \ar@{|->}[d]^{\Inv{F^{\#}}}\\
  %     &F(f) \ar@{=}[r] & F(f\inv{\usr})\pi_0.
  %   }
  % \]
  As $\eta$ is identity on the objects, Diagram~\ref{dia:universal_property_of_inverse_categories}
  commutes.
  The uniqueness of $F^{\#}$ follows immediately from Lemma~\ref{lem:inv_is_full_and_faithful},
  i.e., $\Invf$ is faithful.
\end{proof}
\begin{corollary}\label{cor:eta_and_xt_are_universal_for_inv}
  The category \Xt and functor $\eta:\X\to \Inv{\Xt}$ is a universal pair for the functor $\Invf$.
\end{corollary}
\begin{proof}
  Immediate and Lemma~\ref{lem:universal_diagram_is_a_commutative_diagram}.
\end{proof}

We may now proceed to show $\eta$ is an isomorphism, but we need a lemma first showing that all
invertible maps in $\Xt$ are the equivalence class of the form  $(f \inv{\usr},1)$ for some $f$.

\begin{lemma}\label{lem:all_invertible_maps_in_xt_are_of_the_form_f_inv_usr}
  For any discrete inverse category $\X$, all invertible maps $(g,C):A\to B$ in $\Xt$ are in the
  equivalence class of $(f \inv{\usr},1)$ for some $f:A\to B$.
\end{lemma}
\begin{proof}
  As $(g,C)$ is invertible in \Xt, the map $\inv{(g,C)}: B \to A$ exists. The map $\inv{(g,C)}$ must be in
  the equivalence class of some map $k:B \to A \* D$. By construction, the map  $\rst{(k,D)}$ is
  in the equivalence class of the map $\rst{k}\inv{\usr}:B \to B\*1$ in \X. This means,
  in \X, there is an $n$ such that
  \[
    \xymatrix @C=45pt @R=25pt{
      B \ar[r]^{k} \ar[rrd]_{\rst{k}\inv{\usr}}
        & A \*D \ar[r]^{g\*1}&B\*C\*D  \ar@{.>}[d]^{n} \\
      && B\* 1
    }
  \]
  commutes.

  Starting with $g:A\to B\*C$, construct the map $f$ in \X with the following diagram:
  \[
    \xymatrix @C=225pt @R=20pt{
      A \ar[r]^{g} \ar@{.>}[rdddddd]_{f}& B \*C \ar[d]^{\Delta\*1}\\
      &B\*B\*C \ar[d]^{1\*k\*1}\\
      &B\*A\*D\*C \ar[d]^{1\*g\*1\*1}\\
      &B\*B\*C\*D\*C \ar[d]^{\inv{\Delta}\*1\*c_{\*}}\\
      &B\*C\*C\*D \ar[d]^{1\*\inv{\Delta}\*1}\\
      &B\*C\*D \ar[d]^{n \usl}\\
      &B.
    }
  \]
  By its construction, $f:A\to B$ in \X and $(f\inv{\usr},1)$ are in the same equivalence class as
  $(g,C)$.

\end{proof}

\begin{lemma}\label{lem:the_functor_eta_is_an_isomorphism}
  The functor $\eta:\X \to \Inv{\Xt}$ is an isomorphism.
\end{lemma}
\begin{proof}
  As $\eta$ is an identity on objects functor, we need only show that it is full and faithful.
  Referring to Lemma~\ref{lem:all_invertible_maps_in_xt_are_of_the_form_f_inv_usr} above, we
  immediately see that $\eta$ is full. For faithful, if we assume $(f\inv{\usr},1)$ is equal in \Xt
  to $(g\inv{\usr},1)$. This means in \X, that $\rst{f} = \rst{g}$ and there is a $h\in\dmap{B}$ such that
  \[
    \xymatrix @C=40pt @R=15pt{
      & B \* 1 \ar@{.>}[dd]^{h} \\
      A \ar[ur]^{f\inv{\usr}} \ar[dr]_{g\inv{\usr}}\\
      & B\* 1.
    }
  \]
  But as $h = (\Delta\*1)(1\*h)(\inv{\Delta}\*1)$, and letting $\ell =  \inv{\usr}h\usr$ this
  simplifies to $g = f \Delta (1\*\ell) \inv{\Delta}$. But by
  Lemma~\ref{lem:properties_of_delta_and_tensor_in_a_discrete_inverse_category}\ref{le:restfg},
  $\Delta (1\* \ell) \inv{\Delta} = \rst{\Delta (1\* \ell) \inv{\Delta}}$. Setting $\Delta (1\* \ell)
  \inv{\Delta}$ as $k$, we have $g = f \rst{k}$. This gives us:
  \[
    g = f \rst{k} = \rst{f k} f = \rst{f \rst{k}} f = \rst{g} f = \rst{f} f = f.
  \]
  This shows $\eta$ is faithful and hence an isomorphism between $\X$ and $\Inv{\Xt}$.
\end{proof}

\begin{theorem}\label{thm:discrete_inverse_categories_are_equivalent_to_discrete_restriction_categories}
  The category of discrete inverse categories (objects are discrete inverse categories, maps are
  inverse tensor preserving functors) is equivalent to the category of discrete restriction
  categories (objects are discrete restriction categories, maps are the Cartesian restriction
  functors).
\end{theorem}
\begin{proof}
  Letting $\wtf:\X \to \Xt$ be the functor that takes $X$ to its $\wtc$ completion, then from the
  above lemmas, we have shown that we have an adjoint:
  \begin{equation}
    (\eta,\varepsilon):\wtf \vdash \Invf :D_{i c} \to D_{r c}. \label{eq:inv_and_wtf_are_adjoint_pair}
  \end{equation}
  By Lemma~\ref{lem:the_functor_eta_is_an_isomorphism} we know $\eta$ is an isomorphism. But this
  means the functor $\wtf$ is full and faithful, as shown in, e.g., Proposition 2.2.6 of
  \cite{cockett2009:ctcs}. From lemma \ref{lem:inv_is_full_and_faithful} we know that $\Invf$ is
  full and faithful. But again by the previous reference, this means $\varepsilon$ is an
  isomorphism. Thus, by Corollary \ref{cor:eta_and_xt_are_universal_for_inv} and Proposition 2.2.7
  of \cite{cockett2009:ctcs} we have the equivalence of the two categories.
\end{proof}

Thus, we may now draw out the relationship between Cartesian restriction categories, discrete
Cartesian restriction categories and discrete inverse categories:

\[
  \xymatrix{
    \text{DCartRest} \ar@{^{(}->}[r]^{\top} & \text{CartRest} \ar@/_/[l]\\
    \text{DInv}\ar@{<->}[u]^{\cong}
  }
\]
where the arrow from discrete Cartesian restriction categories to Cartesian restriction categories
is the standard embedding and the reverse arrow picks out the discrete objects in the Cartesian
restriction category. Of course, the terminal object is always discrete, as noted in
Example~\ref{ex:terminal_object_is_discrete}.

% subsection equivalence_of_categories (end)
\section{Examples of the \texorpdfstring{\wtc}{tilde} construction} % (fold)
\label{sec:examples_of_the_wtf_construction}

\begin{example}[Different inverse products produce different $\Xt$]
  \label{example:completing_a_finite_discrete_inverse_category}
\end{example}
Continuing from Example~\ref{example:invprodisstructure}, recall the discrete category of 4
elements with two different tensors. Completing these gives two different lattices. They are either
the straight line lattice, or the diamond semi-lattice. Below are the details of these constructions.

Recall $\D$ has four elements $a,b,c$ and $d$, and there are two possible inverse product tensors,
given in Table~\ref{tab:two_different_inverse_products}. (Repeated here for your convenience).
\begin{table*}[!htbp]
  \begin{center}
  \begin{tabular}{|l||c|c|c|c|}
    \hline
    $\*$&a&b&c&d\\ \hline \hline
    a&a&a&a&a\\ \hline
    b&a&b&\textbf{b}&b\\ \hline
    c&a&\textbf{b}&c&c \\ \hline
    d&a&b&c&d \\ \hline
  \end{tabular}
  \qquad
  \begin{tabular}{|l||c|c|c|c|} \hline
    $\odot$&a&b&c&d\\ \hline \hline
    a&a&a&a&a\\ \hline
    b&a&b&\textbf{a}&b\\ \hline
    c&a&\textbf{a}&c&c \\ \hline
    d&a&b&c&d \\ \hline
  \end{tabular}
  \end{center}
  \caption[]{Two different inverse products on the same category.} %[] option removes from LOT
\end{table*}


Define $\Delta$ as the identity map. Then, for the first tensor, $\*$ of
Table~\ref{tab:two_different_inverse_products}, $\widetilde{\D}$ has the following
maps
\begin{align*}
  %a \xrightarrow{(id,a)} a \ (\equiv a \xrightarrow{(id, b)} a \equiv a \xrightarrow{(id,c)} a
  %\equiv a \xrightarrow{(id,d)} a), \qquad a \xrightarrow{(id,a)} b, \qquad a \xrightarrow{(id, a)}
  %c , \qquad a \xrightarrow{(id, a)} d \\
  a \xrightarrow{(id,a)\ (\equiv (id, b) \equiv (id,c) \equiv (id,d))} a, \qquad a
    \xrightarrow{(id,a)} b, \qquad a \xrightarrow{(id, a)} c , \qquad a \xrightarrow{(id, a)} d \\
  %b \xrightarrow{(id,b)} b \ (\equiv b \xrightarrow{(id, c)} b \equiv b \xrightarrow{(id, d)} b ),
  %\qquad b \xrightarrow{(id,b)} c, \qquad b \xrightarrow{(id,b)} d\\
  b \xrightarrow{(id,b) \ (\equiv (id, c) \equiv (id, d))} b , \qquad b \xrightarrow{(id,b)} c,
    \qquad b \xrightarrow{(id,b)} d\\
  c \xrightarrow{(id, c) \ (\equiv (id, d))} c ,   \qquad c \xrightarrow{(id,c)} d\\
  d \xrightarrow{(id,d)} d
\end{align*}
resulting in the straight-line ($a \to b \to c \to d$) lattice. The tensor in \D becomes the meet
and hence is a categorical product in $\widetilde{\D}$. Note that the only partial inverses in
$\widetilde{\D}$ are the identity functions and that for all maps $f$, $\<f,1\> = id$.

With the second tensor, $\odot$ from Table~\ref{tab:two_different_inverse_products}, we have:
\begin{align*}
  a \xrightarrow{(id,a) \ (\equiv (id, b) \equiv (id,c) \equiv (id,d))} a, \qquad
    a \xrightarrow{(id,a)} b, \qquad a \xrightarrow{(id, a)} c , \qquad a \xrightarrow{(id, a)} d \\
  b \xrightarrow{(id,b) \ (\equiv (id, d))} b ,  \qquad b \xrightarrow{(id,b)} d\\
  c \xrightarrow{(id, c) \ (\equiv (id, d))} c,   \qquad c \xrightarrow{(id,c)} d\\
  d \xrightarrow{(id,d)} d
\end{align*}
resulting in the ``diamond'' lattice, \raisebox{19pt}{
$
  \xymatrix @R=3pt @C=8pt {
    & b \ar[dr]\\
    a \ar[ur] \ar[dr] & &d\\
    & c \ar[ur]
  }
$}. Once again, the tensor in \D becomes the meet.

\begin{example}[Lattice completion.]\label{example:lattice_completion}
  Suppose we have a set together with an idempotent, commutative, associative operation $\wedge$ on
  the set, giving us a lattice, \Lat. Further suppose the set is partially ordered via $\le$ with
  the order being compatible with $\wedge$.

  Then, we may create a pullback square for any $x' \le x,\ y' \le x$ with
  \[
    \xymatrix{
      &x\\
      x' \ar[ur]_{\le} & & y' \ar[ul]^{\le}\\
      &x'\wedge y'.  \ar[ul]_{\le} \ar[ur]^{\le}
    }
  \]

  Considering \Lat as a category, we see that all maps are monic and therefore, we may create a
  partial map category $\text{Par}(\Lat,\Mstab)$ where the stable system of monics are all the maps.

  Then $\widetilde{\text{Par}(\Lat,\Mstab)}$ becomes the completion of the lattice over $\wedge$.
\end{example}
\begin{example}[$\widetilde{\text{\pinj}}$ is \Par]\label{ex:tilde_pinj_is_par}
  Noting that the objects of both \pinj and \Par are sets, we simply need to show that any partial
  function is in the equivalence class of some $f$, a map in \pinj.

  Suppose we are given $g : A \to B = \{(a,b) | a \in A, b\in B\}$, a partial function in sets. Of
  course, if it is a partial injective function, then $g$ is in the equivalence class of $(g,\{*\})$
  and we are done.  If it is not injective, that means there are one or more elements of $B$ which
  appear in the left hand element of $g$ multiple times. Construct a new function $h$ as follows:
  \begin{equation}
     h \definedas \{(a,(b,a)) | (a,b) \in g\} \label{eq:new_partial_injective}
  \end{equation}
  By its definition, $h : A \to B \*A$ is injective, $(h,A): A \to B$ coincides with $g$ and therefore
  we see that using the \wtc construction on \pinj results in \Par.
\end{example}
% subsection examples_of_the_wtf_construction (end)


% chapter inverse_categories (end)
%%% Local Variables:
%%% mode: latex
%%% TeX-master: "../phd-thesis"
%%% End:

%%% Local Variables:
%%% mode: latex
%%% TeX-master: "../phd-thesis"
%%% End:
