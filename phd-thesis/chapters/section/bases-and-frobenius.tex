

\section{Bases and Frobenius Algebras} % (fold)
\label{sec:bases_and_frobenius_algebras}
In \cite{coeckeetal08:ortho}, the authors provide an algebraic description of orthogonal bases in
finite dimensional Hilbert spaces. As noted in section
\ref{sec:the_category_of_commutative_frobenius_algebras}, an orthonormal basis for such a space is
a special commutative $\dagger$-Frobenius algebra. To show the other direction, given a commutative
$\dagger$-Frobenius algebra, $(H,\nabla,u)$ and for each element $\alpha\in H$, define the right
action of $\alpha$ as $R_{\alpha}:=(id\*\alpha)\, \nabla:H\to H$. Note the use of the fact that
elements $\alpha\in H$ can be considered as linear maps $\alpha:\C \to H:1\mapsto \ket{\alpha}$.
The dagger of a right action is also a right action, $\dgr{R_{\alpha}} = R_{\alpha'}$ where
$\alpha'= u\, \nabla\, (id\* \dgr{\alpha})$, which is a consequence of the Frobenius identities.

The $(\_)'$ construction is actually an involution:
\begin{eqnarray*}
  &(\alpha')' &= u \nabla (id \* \dgr{\alpha'}) \\
  && = u \nabla (id \* \dgr{(u \nabla (id \* \dgr{\alpha}))}\\
  && = u \nabla (id \* ( (id \* \alpha) \Delta \epsilon))\\
  && = (u \* \alpha) (\nabla \* id) (id \* \Delta) (id \*  \epsilon)\\
  && = (u \* \alpha) (id \* \Delta) (\nabla \* id) (id \*  \epsilon)\\
  && = (u \* \alpha)  (id \*  \epsilon)\\
  && = \alpha
\end{eqnarray*}

\begin{lemma}\label{lemma:cstaralgebra}
  Any $\dagger$-Frobenius algebra in \fdh is a $C^{*}$-algebra.
\end{lemma}
\begin{proof}
  The endomorphism monoid of \fdh(H,H) is a $C^{*}$-algebra. From the proceeding,
  \[
    H \cong \fdh(\C,H) \cong R_{[\fdh(\C,H)]}\subseteq\fdh(H,H).
  \]
  This inherits the algebra structure from \fdh(H,H). Furthermore, since any finite dimensional
  involution-closed sub-algebra of a $C^{*}$-algebra is also a $C^{*}$-algebra, this shows the
  $\dagger$-Frobenius algebra is a $C^{*}$-algebra.
\end{proof}

Using the fact that the involution preserving homomorphisms from a finite dimensional commutative
$C^{*}$-algebra to $\C$ form a basis for the dual of the underlying vector space, write these
homomorphisms as $\dgr{\phi_{i}}:H \to \C$. Then their adjoints, $\phi_{i}:\C\to H$ will form a
basis for the space $H$. These are the copyable elements in $H$.

This, together with continued applications of the Frobenius rules and linear algebra allow the
authors to prove:
\begin{theorem}
  Every commutative $\dagger$-Frobenius algebra in \fdh determines an orthogonal basis consisting
  of its copyable elements. Conversely, every orthogonal basis $\{\ket{\phi_{i}}\}_{i}$ determines
  a commutative $\dagger$-Frobenius algebra via \[\Delta:H\to H\*H: \ket{\phi_{i}}\mapsto
  \ket{\phi_{i}} \* \ket{\phi_{i}}\qquad\epsilon : H\to \C : \ket{\phi_{i}} \mapsto 1\] and these
  constructions are inverse to each other.
\end{theorem}

\subsection{Quantum and classical data}\label{sec:quantumclassical}
In \cite{coecke08structures}, the authors build on the results above,
to start from a $\dagger$-symmetric monoidal category and construct the minimal machinery needed to
model quantum and classical computations. For the rest of this section, $\cD$ will be assumed to be
such a category, with $\*$ the monoid tensor and $I$ the unit of the monoid.

\begin{definition}
  A compact structure on an object $A$ in the category $\cD$ is given by the object $A$, an object
  $A^{*}$ called its \emph{dual} and the maps $\eta:I \to A^{*}\* A$, $\epsilon: A\* A^{*} \to I$
  such that the diagrams
  \[
    \xymatrix@C+20pt{
      A^{*} \ar[dr]^{id} \ar[d]_{\eta\*A^{*}} \\
      A^{*} \*A\*A^{*}  \ar[r]_(.6){A^{*} \*\epsilon} & A^{*}
    }
    \text{ and }
    \xymatrix@C+20pt{
      A \ar[r]^(.4){A\*\eta} \ar[dr]_{id} & A\* A^{*}\* A \ar[d]^{\epsilon\*A}\\
      & A
    }
  \]
  commute.
\end{definition}

\begin{definition}[Quantum Structure]\label{def:quantumstructure}
  A \emph{quantum structure} is an object $A$ and map $\eta:I\to A\*A$ such that
  $(A,A,\eta,\dgr{\eta})$ form a compact structure.
\end{definition}
Note that $A$ is self-dual in definition \ref{def:quantumstructure}.

This allows the creation of the category $\cD_{q}$ which has as objects quantum structures and maps
are the maps in $\cD$ between the objects in the quantum structures.

In the category $\cD_{q}$, it is now possible to define the upper and lower $*$ operations on maps,
such that $(f_{*})^{*}= (f^{*})_{*} = \dgr{f}$.
\begin{eqnarray*}
&f^{*} &:= (\eta_{A}\*1) (1 \* f\*1) (1\*\dgr{\eta}_{B})\\
&f_{*} &:= (\eta_{B}\*1) (1 \* \dgr{f}\*1) (1\*\dgr{\eta}_{A})
\end{eqnarray*}

Interestingly, $\cD_{q}$ possesses enough structure to be axiomatized in the same manner as above in
section \ref{sec:puresemantics}, excepting the portions dependent upon biproducts.

Next, define a classical structure on \cD.
\begin{definition}[Classical structure]\label{def:classicalstructure}
  A \emph{classical structure} in \cD{} is an objects $X$ and two maps, $\Delta :X \to X\* X$,
  $\epsilon:X\to I$ such that $X,\dgr{\Delta},\dgr{\epsilon},\Delta,\epsilon$ forms a special
  Frobenius algebra.
\end{definition}

As above, this allows us to define $\cD_{c}$, the category whose objects are the classical
structures of $\cD$ with maps between classical structures being the maps in $\cD$ between the
objects of the classical structure.

Note that a classical structure will induce a quantum structure, setting $\eta_{X}$ to be
$\dgr{\epsilon_{X}}\, \Delta_{X}$.

% section bases_and_frobenius_algebras (end)

%%% Local Variables:
%%% mode: latex
%%% TeX-master: "../../phd-thesis"
%%% End:
