\section{Disjointness in Frobenius Algebras}
\label{sec:disjointness_in_frobenius_algebras}
An important example of a disjointness system arises naturally from Frobenius Algebras. Recall that
we defined a zero map in Section~\ref{sec:coproducts_in_inverse_categories} whenever there was a
zero object. This may also be defined directly:

\begin{definition}\label{def:zero_map}
  A family of maps in a symmetric monoidal category with monoid $\*$ is called a \emph{zero map},
  written as $0$ whenever:
  \begin{itemize}
    \item For all $f, g$, $f 0 g = 0$;
    \item For all $f$, $f \* 0 = 0$ and $ 0 \* f = 0$.
  \end{itemize}
\end{definition}

\begin{lemma}\label{lem:cfrobx-has-a-zero-when-x-has-zero}
  If $\X$ is a symmetric monoidal category with a zero map, then CFrob(\X) has a zero map.
\end{lemma}
\begin{proof}
  As $\Delta 0 = 0 = 0 \Delta$ and $0 \nabla = 0 = \nabla 0$, the zero maps are in CFrob(\X). Since
  composition is inherited from \X, the first item of Definition~\ref{def:zero_map} is
  satisfied. Finally, since the tensor in \X is the tensor in CFrob(\X), the second item is also
  satisfied. Hence, CFrob(\X) has zero maps.
\end{proof}

\begin{lemma}\label{lem:cfrobperp_is_a_disjointness_relation}
  As shown in Lemma~\ref{lem:cfrobx_is_an_inverse_category} $CFrob(\X)$ is a discrete inverse
  category. Suppose \X has zero maps. Then for $f,g:A\to B$, define $f\perp g$  when
\[
\begin{tikzpicture}
\path node at (0,0) [nabla] (n1) {}
node at (0,2.5) (start) {}
node at (-.5,1) [map] (f) {$\scriptstyle f$}
node at (.5,1) [map] (g) {$\scriptstyle g$}
node at (0,2) [delta] (d) {};
\draw [] (d) to (start);
\draw [] (n1) to (0,-0.5);
\draw [] (d) to[out=305,in=90] (g);
\draw [] (d) to[out=235,in=90] (f);
\draw [-] (f) to[out=270,in=125] (n1);
\draw [-] (g) to[out=270,in=55] (n1);
\end{tikzpicture}
\ \raisebox{40pt}{$= 0.$}\
\]

  The relation $\perp$ is a disjointness relation.
\end{lemma}
\begin{proof}
We need to show the seven axioms of the disjointness relation hold. Note that we will show
\axiom{Dis}{6} early on as its result will be used to establish some of the other axioms.\\
\axiom{Dis}{1}: For all $f:A\to B,\ f\cdperp 0$.\\
\[
\begin{tikzpicture}
\path node at (0,0) [nabla] (n1) {}
node at (0,2.5) (start) {}
node at (-.5,1) [map] (f) {$\scriptstyle f$}
node at (.5,1) [map] (z) {$\scriptstyle 0$}
node at (0,2) [delta] (d) {};
\draw [] (d) to (start);
\draw [] (n1) to (0,-0.5);
\draw [] (d) to[out=235,in=90] (f);
\draw [] (d) to[out=305,in=90] (z);
\draw [-] (f) to[out=270,in=125] (n1);
\draw [-] (z) to[out=270,in=55] (n1);
\end{tikzpicture}
\ \raisebox{45pt}{$=$}\
\begin{tikzpicture}
\path node at (0,0) [nabla] (n1) {}
node at (0,2.5) (start) {}
node at (-.5,1) [map] (f) {$\scriptstyle f$}
node at (.5,.75) [map] (z) {$\scriptstyle 0$}
node at (.5,1.25) [map] (r_z) {$\scriptstyle \rst{0}$}
node at (0,2) [delta] (d) {};
\draw [] (d) to (start);
\draw [] (n1) to (0,-0.5);
\draw [] (d) to[out=235,in=90] (f);
\draw [] (d) to[out=305,in=90] (r_z);
\draw [] (r_z) to (z);
\draw [-] (f) to[out=270,in=125] (n1);
\draw [-] (z) to[out=270,in=55] (n1);
\end{tikzpicture}
\ \raisebox{45pt}{$=$}\
\begin{tikzpicture}
\path node at (0,0) [nabla] (n1) {}
node at (0,2.75) [map] (t_z) {$\scriptstyle 0(=\rst{0})$}
node at (-.5,1) [map] (f) {$\scriptstyle f$}
node at (.5,1) [map] (z) {$\scriptstyle 0$}
node at (0,2) [delta] (d) {};
\draw [] (d) to (t_z);
\draw [] (n1) to (0,-0.5);
\draw [] (d) to[out=235,in=90] (f);
\draw [] (d) to[out=305,in=90] (z);
\draw [-] (f) to[out=270,in=125] (n1);
\draw [-] (z) to[out=270,in=55] (n1);
\end{tikzpicture}
\ \raisebox{45pt}{$= 0.$}
\]
\axiom{Dis}{6}: $f\cdperp g$ implies $\rst{f} \cdperp \rst{g}$ and $\rg{f}\cdperp\rg{g}$.\\
We will show the details of $\rst{f} \cdperp \rst{g}$, using $\rst{f} = f\inv{f}$ and the definition of
$\inv{f}$ as given in Theorem~\ref{thm:cfrob_is_a_discrete_inverse_category}. The proof of
$(\inv{f}f=) \rg{f} \perp \rg{g} (= \inv{g}g)$ is similar.
\[
\begin{tikzpicture}
\path node at (0,0) [nabla] (n1) {}
node at (0,2.5) (start) {}
node at (-.5,1) [map] (f) {$\scriptstyle \rst{f}$}
node at (.5,1) [map] (g) {$\scriptstyle \rst{g}$}
node at (0,2) [delta] (d) {};
\draw [] (d) to (start);
\draw [] (n1) to (0,-0.5);
\draw [] (d) to[out=305,in=90] (g);
\draw [] (d) to[out=235,in=90] (f);
\draw [-] (f) to[out=270,in=125] (n1);
\draw [-] (g) to[out=270,in=55] (n1);
\end{tikzpicture}
\ \raisebox{45pt}{$=$}\
\begin{tikzpicture}
    \path node at (0.5,3.5) [delta] (start) {}
    node at (0,2.5) [eta] (eta1) {}
    node at (0,2) [delta] (d) {}
    node at (-1.2,1.5) [map] (f1) {$\scriptstyle f$}
    node at (-.5,1.5) [map] (f) {$\scriptstyle f$}
    node at (-1,1) [nabla] (n1) {}
    node at (-1,.5) [epsilon] (e1) {}
    node at (2,2.5) [eta] (e_tag) {}
    node at (2,2) [delta] (d_g) {}
    node at (.8,1.5) [map] (g1) {$\scriptstyle g$}
    node at (1.5,1.5) [map] (g) {$\scriptstyle g$}
    node at (1,1) [nabla] (n_g) {}
    node at (1,.5) [epsilon] (e_g) {}
    node at (1,-.5) [nabla] (end) {};
    \draw [] (start) to[out=235,in=90] (f1);
    \draw [] (start) to[out=305,in=90] (g1);
    \draw [] (f1) to (n1);
    \draw [] (eta1) to (d);
    \draw [] (d) to (end);
    \draw [] (d) to (f);
    \draw [] (f) to (n1);
    \draw [] (n1) to (e1);
    \draw [] (g1) to (n_g);
    \draw [] (e_tag) to (d_g);
    \draw [] (d_g) to[out=305,in=55] (end);
    \draw [] (d_g) to (g);
    \draw [] (g) to (n_g);
    \draw [] (n_g) to (e_g);
\end{tikzpicture}
\ \raisebox{45pt}{$ = $}\
\begin{tikzpicture}
    \path node at (0,4) [delta] (start) {}
    node at (-1,3.5) [eta] (eta1) {}
    node at (1,3.5) [eta] (eta2) {}
    node at (-1,3) [delta] (d1) {}
    node at (1,3) [delta] (d2) {}
    node at (-.5,2.5) [nabla] (n1) {}
    node at (.5,2.5) [nabla] (n2) {}
    node at (-.5,2) [map] (f1) {$\scriptstyle f$}
    node at (.5,2) [map] (g2) {$\scriptstyle g$}
    node at (-.5,1.5) [epsilon] (e1) {}
    node at (.5,1.5) [epsilon] (e2) {}
    node at (0,.5) [nabla] (n) {};
    \draw [] (start) to (0,4.5);
    \draw [] (start) to (n1);
    \draw [] (start) to (n2);
    \draw [] (eta1) to (d1);
    \draw [] (eta2) to (d2);
    \draw [] (d1) to[out=235,in=125] (n);
    \draw [] (d1) to (n1);
    \draw [] (d2) to (n2);
    \draw [] (d2) to[out=305,in=55] (n);
    \draw [] (n1) to (f1);
    \draw [] (n2) to (g2);
    \draw [] (f1) to (e1);
    \draw [] (g2) to (e2);
    \draw [] (n)  to (0,0);
\end{tikzpicture}
\ \raisebox{45pt}{$=$}\
\begin{tikzpicture}
    \path node at (0,4) [delta] (start) {}
    node at (-.5,3) [delta] (d1) {}
    node at (.5,3) [delta] (d2) {}
    node at (-.25,2) [map] (f1) {$\scriptstyle f$}
    node at (.25,2) [map] (g2) {$\scriptstyle g$}
    node at (-.25,1.5) [epsilon] (e1) {}
    node at (.25,1.5) [epsilon] (e2) {}
    node at (0,0) [nabla] (n) {};
    \draw [] (start) to (0,4.5);
    \draw [] (start) to (d1);
    \draw [] (start) to (d2);
    \draw [] (d1) to[out=235,in=125] (n);
    \draw [] (d1) to (f1);
    \draw [] (d2) to (g2);
    \draw [] (d2) to[out=305,in=55] (n);
    \draw [] (f1) to (e1);
    \draw [] (g2) to (e2);
    \draw [] (n)  to (0,-.5);
\end{tikzpicture}
\ \raisebox{45pt}{$=$}
\]
\[
\begin{tikzpicture}
    \path node at (0,4) [delta] (start) {}
    node at (.5,3.5) [delta] (d2) {}
    node at (0,2.5) [delta] (d1) {}
    node at (-.25,2) [map] (f1) {$\scriptstyle f$}
    node at (.25,2) [map] (g2) {$\scriptstyle g$}
    node at (0,1.5) [nabla] (n12) {}
    node at (0,1) [epsilon] (e2) {}
    node at (0,0) [nabla] (n) {};
    \draw [] (start) to (0,4.5);
    \draw [] (start) to[out=235,in=125] (n);
    \draw [] (start) to (d2);
    \draw [] (d2) to[out=235,in=90] (d1);
    \draw [] (d1) to (f1);
    \draw [] (d1) to (g2);
    \draw [] (d2) to[out=305,in=55] (n);
    \draw [] (f1) to (n12);
    \draw [] (g2) to (n12);
    \draw [] (n12) to (e2);
    \draw [] (n)  to (0,-.5);
\end{tikzpicture}
\ \raisebox{45pt}{$=$}\
\begin{tikzpicture}
    \path node at (0,3) [delta] (start) {}
    node at (.5,2.5) [delta] (d2) {}
    node at (0,1.75) [map] (z) {$\scriptstyle 0$}
    node at (0,1) [epsilon] (e2) {}
    node at (0,0) [nabla] (n) {};
    \draw [] (start) to (0,3.5);
    \draw [] (start) to[out=235,in=125] (n);
    \draw [] (start) to (d2);
    \draw [] (d2) to[out=235,in=90] (z);
    \draw [] (d2) to[out=270,in=55] (n);
    \draw [] (z) to (e2);
    \draw [] (n)  to (0,-.5);
\end{tikzpicture}
\ \raisebox{45pt}{$=$}\
\begin{tikzpicture}
\path node at (0,0) [nabla] (n1) {}
node at (0,2.5) (start) {}
node at (.5,1) [map] (z) {$\scriptstyle 0$}
node at (0,2) [delta] (d) {};
\draw [] (d) to (start);
\draw [] (n1) to (0,-0.5);
\draw [] (d) to[out=305,in=90] (z);
\draw [] (d) to[out=235,in=125] (n1);
\draw [-] (z) to[out=270,in=55] (n1);
\end{tikzpicture}
\ \raisebox{45pt}{$= 0.$}
\]
\axiom{Dis}{2}: $f\cdperp g$ implies $\rst{f} g = 0$.\\
In this proof, we use the result of \axiom{Dis}{6}, i.e., that $\rst{f}\perp\rst{g}$.
\[
\raisebox{25pt}{
\begin{tikzpicture}
  \begin{pgfonlayer}{nodelayer}
    \node [style=map] (0) at (-2, 1) {$\scriptstyle \rst{f}$};
    \node [style=map] (1) at (-2, 0) {$\scriptstyle g$};
    \end{pgfonlayer}
    \begin{pgfonlayer}{edgelayer}
      \draw (0) to (1);
      \draw (0) to (-2,1.5);
      \draw (1) to (-2, -.5);
      \end{pgfonlayer}
\end{tikzpicture}
}
\ \raisebox{45pt}{$=$}\
\begin{tikzpicture}
  \begin{pgfonlayer}{nodelayer}
    \node [style=map] (0) at (-2, 3) {$\scriptstyle \rst{f}$};
    \node [style=map] (1) at (-2, 2.25) {$\scriptstyle g$};
    \node [style=delta] (2) at (-1.25, 4) {};
    \node [style=nabla] (3) at (-1.25, 1.25) {};
    \node [style=map] (4) at (-0.5, 3) {$\scriptstyle \rst{f}$};
    \node [style=map] (5) at (-0.5, 2.25) {$\scriptstyle g$};
    \end{pgfonlayer}
    \begin{pgfonlayer}{edgelayer}
      \draw (0) to (1);
      \draw (1) to[out=270,in=125] (3);
      \draw (5) to[out=270,in=55] (3);
      \draw (4) to (5);
      \draw (2) to[out=305,in=90] (4);
      \draw (2) to[out=235,in=90] (0);
      \draw (2) to (-1.25,4.5);
      \draw (3) to (-1.25,.75);
      \end{pgfonlayer}
\end{tikzpicture}
\ \raisebox{45pt}{$=$}\
\begin{tikzpicture}
  \begin{pgfonlayer}{nodelayer}
    \node [style=map] (0) at (-2, 3) {$\scriptstyle \rst{f}$};
    \node [style=map] (1) at (-2, 2.25) {$\scriptstyle g$};
    \node [style=delta] (2) at (-1.25, 4) {};
    \node [style=nabla] (3) at (-1.25, 1.25) {};
    \node [style=map] (4) at (-0.5, 2.25) {$\scriptstyle g$};
    \end{pgfonlayer}
    \begin{pgfonlayer}{edgelayer}
      \draw (0) to (1);
      \draw (1) to[out=270,in=125] (3);
      \draw (4) to[out=270,in=55] (3);
      \draw (2) to[out=235,in=90] (0);
      \draw (2) to[out=305,in=90] (4);
      \draw (3) to (-1.25,.75);
      \draw (2) to (-1.25,4.5);
      \end{pgfonlayer}
\end{tikzpicture}
\ \raisebox{45pt}{$=$}\
\begin{tikzpicture}
  \begin{pgfonlayer}{nodelayer}
    \node [style=map] (0) at (-2, 2.75) {$\scriptstyle \rst{f}$};
    \node [style=map] (1) at (-1.25, 0.75) {$\scriptstyle g$};
    \node [style=delta] (2) at (-1.25, 4) {};
    \node [style=nabla] (3) at (-1.25, 1.5) {};
    \node [style=map] (4) at (-0.5, 2.75) {$\scriptstyle \rst{g}$};
    \end{pgfonlayer}
    \begin{pgfonlayer}{edgelayer}
      \draw (4) to[out=270,in=55] (3);
      \draw (2) to[out=235,in=90] (0);
      \draw (2) to[out=305,in=90] (4);
      \draw (0) to[out=270,in=125] (3);
      \draw (3) to (1);
      \draw (1) to (-1.25,.25);
      \draw (2) to (-1.25,4.5);
      \end{pgfonlayer}
\end{tikzpicture}
\ \raisebox{45pt}{$=$}\
\raisebox{25pt}{
\begin{tikzpicture}
  \begin{pgfonlayer}{nodelayer}
    \node [style=map] (0) at (-1.25, 1.75) {$\scriptstyle 0$};
    \node [style=map] (1) at (-1.25, 0.75) {$\scriptstyle g$};
    \end{pgfonlayer}
    \begin{pgfonlayer}{edgelayer}
      \draw (0) to (1);
      \draw (1) to (-1.25,.25);
      \draw (0) to (-1.25,2.25);
      \end{pgfonlayer}
\end{tikzpicture}
}
\ \raisebox{45pt}{$=0$}
\]
\axiom{Dis}{3}: $f\cdperp g,\ f' \le f,\ g' \le g$ implies $f' \cdperp g'$.\\
\[
\begin{tikzpicture}
\path node at (0,0) [nabla] (n1) {}
node at (0,2.5) (start) {}
node at (-.5,1) [map] (f) {$\scriptstyle f'$}
node at (.5,1) [map] (g) {$\scriptstyle g'$}
node at (0,2) [delta] (d) {};
\draw [] (d) to (start);
\draw [] (n1) to (0,-0.5);
\draw [] (d) to[out=305,in=90] (g);
\draw [] (d) to[out=235,in=90] (f);
\draw [-] (f) to[out=270,in=125] (n1);
\draw [-] (g) to[out=270,in=55] (n1);
\end{tikzpicture}
\ \raisebox{45pt}{$=$}\
\begin{tikzpicture}
  \begin{pgfonlayer}{nodelayer}
    \node [style=map] (0) at (-2, 3) {$\scriptstyle \rst{f'}$};
    \node [style=map] (1) at (-2, 2.25) {$\scriptstyle f$};
    \node [style=delta] (2) at (-1.25, 4) {};
    \node [style=nabla] (3) at (-1.25, 1.25) {};
    \node [style=map] (4) at (-0.5, 3) {$\scriptstyle \rst{g'}$};
    \node [style=map] (5) at (-0.5, 2.25) {$\scriptstyle g$};
    \end{pgfonlayer}
    \begin{pgfonlayer}{edgelayer}
      \draw (0) to (1);
      \draw (1) to[out=270,in=125] (3);
      \draw (5) to[out=270,in=55] (3);
      \draw (4) to (5);
      \draw (2) to[out=305,in=90] (4);
      \draw (2) to[out=235,in=90] (0);
      \draw (2) to (-1.25,4.5);
      \draw (3) to (-1.25,.75);
      \end{pgfonlayer}
\end{tikzpicture}
\ \raisebox{45pt}{$=$}\
\begin{tikzpicture}
\path
node at (0,3) (start) {}
node at (0,2.5) [map] (f_g) {$\scriptstyle \rst{f'}\,\rst{g'}$}
node at (0,2) [delta] (d) {}
node at (-.5,1) [map] (f) {$\scriptstyle f$}
node at (.5,1) [map] (g) {$\scriptstyle g$}
 node at (0,0) [nabla] (n1) {};
\draw [] (d) to (f_g);
\draw [] (start) to (f_g);
\draw [] (n1) to (0,-0.5);
\draw [] (d) to[out=305,in=90] (g);
\draw [] (d) to[out=235,in=90] (f);
\draw [-] (f) to[out=270,in=125] (n1);
\draw [-] (g) to[out=270,in=55] (n1);
\end{tikzpicture}
\ \raisebox{45pt}{$=$}\
\raisebox{25pt}{
\begin{tikzpicture}
\path
node at (0,1.5) (start) {}
node at (0,1) [map] (f_g) {$\scriptstyle \rst{f'}\,\rst{g'}$}
node at (0,.5) [map] (z) {$\scriptstyle 0$};
\draw [] (z) to (f_g);
\draw [] (start) to (f_g);
\draw [] (z) to (0,0);
\end{tikzpicture}
}
\ \raisebox{45pt}{$=0$}
\]
\axiom{Dis}{4}: $f\cdperp g$ implies $g \cdperp f$.\\
This follows directly from the co-commutativity of $\Delta$.\\
\axiom{Dis}{5}: $f\cdperp g$ implies $h f \cdperp h g$.\\
This follows directly from the naturality of $\Delta$.\\
\axiom{Dis}{7}: $\rst{f}\cdperp \rst{g},\ \rg{h}\cdperp \rg{k}$ implies $f h \cdperp g k$.\\
\[
\begin{tikzpicture}
\path node at (0,0) [nabla] (n1) {}
node at (0,2.5) (start) {}
node at (-.5,1) [map] (f_h) {$\scriptstyle f h$}
node at (.5,1) [map] (g_k) {$\scriptstyle g k$}
node at (0,2) [delta] (d) {};
\draw [] (d) to (start);
\draw [] (n1) to (0,-0.5);
\draw [] (d) to[out=305,in=90] (g_k);
\draw [] (d) to[out=235,in=90] (f_h);
\draw [-] (f_h) to[out=270,in=125] (n1);
\draw [-] (g_k) to[out=270,in=55] (n1);
\end{tikzpicture}
\ \raisebox{45pt}{$=$}\
\begin{tikzpicture}
\path node at (0,0) [nabla] (n1) {}
node at (0,2.5) (start) {}
node at (-.5,1.5) [map] (r-f) {$\scriptstyle \rst{f}$}
node at (-.5,1) [map] (f_h) {$\scriptstyle f h$}
node at (-.5,.5) [map] (rngh) {$\scriptstyle \rg{h}$}
node at (.5,1.5) [map] (rg) {$\scriptstyle \rst{g}$}
node at (.5,1) [map] (gk) {$\scriptstyle g k$}
node at (.5,.5) [map] (rngk) {$\scriptstyle \rg{k}$}
node at (0,2) [delta] (d) {};
\draw [] (d) to (start);
\draw [] (n1) to (0,-0.5);
\draw [] (d) to[out=305,in=90] (rg);
\draw [] (d) to[out=235,in=90] (r-f);
\draw (r-f) to (f_h);
\draw (f_h) to (rngh);
\draw (rg) to (gk);
\draw (gk) to (rngk);
\draw [-] (rngh) to[out=270,in=125] (n1);
\draw [-] (rngk) to[out=270,in=55] (n1);
\end{tikzpicture}
\ \raisebox{45pt}{$=$}\
\begin{tikzpicture}
\path node at (0,0) [nabla] (n1) {}
node at (0,2.5) (start) {}
node at (-.5,1) [map] (fh) {$\scriptstyle f h$}
node at (-.5,.5) [map] (rnghrngk) {$\scriptstyle \rg{h}\rg{k}$}
node at (.5,1.5) [map] (rfrg) {$\scriptstyle \rst{f}\rst{g}$}
node at (.5,1) [map] (gk) {$\scriptstyle g k$}
node at (0,2) [delta] (d) {};
\draw [] (d) to (start);
\draw [] (n1) to (0,-0.5);
\draw [] (d) to[out=305,in=90] (rfrg);
\draw [] (d) to[out=235,in=90] (fh);
\draw (fh) to (rnghrngk);
\draw (rfrg) to (gk);
\draw [-] (rnghrngk) to[out=270,in=125] (n1);
\draw [-] (gk) to[out=270,in=55] (n1);
\end{tikzpicture}
\ \raisebox{45pt}{$=$}\
\begin{tikzpicture}
\path node at (0,0) [nabla] (n1) {}
node at (0,2.5) (start) {}
node at (-.5,1) [map] (fh) {$\scriptstyle f h$}
node at (-.5,.5) [map] (rnghrngk) {$\scriptstyle 0$}
node at (.5,1.5) [map] (rfrg) {$\scriptstyle 0$}
node at (.5,1) [map] (gk) {$\scriptstyle g k$}
node at (0,2) [delta] (d) {};
\draw [] (d) to (start);
\draw [] (n1) to (0,-0.5);
\draw [] (d) to[out=305,in=90] (rfrg);
\draw [] (d) to[out=235,in=90] (fh);
\draw (fh) to (rnghrngk);
\draw (rfrg) to (gk);
\draw [-] (rnghrngk) to[out=270,in=125] (n1);
\draw [-] (gk) to[out=270,in=55] (n1);
\end{tikzpicture}
\ \raisebox{45pt}{$=$}\
\begin{tikzpicture}
\path node at (0,0) [nabla] (n1) {}
node at (0,2.5) (start) {}
node at (-.5,1) [map] (fh) {$\scriptstyle 0$}
node at (.5,1) [map] (gk) {$\scriptstyle 0$}
node at (0,2) [delta] (d) {};
\draw [] (d) to (start);
\draw [] (n1) to (0,-0.5);
\draw [] (d) to[out=305,in=90] (gk);
\draw [] (d) to[out=235,in=90] (fh);
\draw [-] (fh) to[out=270,in=125] (n1);
\draw [-] (gk) to[out=270,in=55] (n1);
\end{tikzpicture}
\ \raisebox{45pt}{$= 0.$}\
\]
\end{proof}

%%% Local Variables:
%%% mode: latex
%%% TeX-master: "../../phd-thesis"
%%% End:
