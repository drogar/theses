\section{The category of Commutative Frobenius Algebras} % (fold)
\label{sec:the_category_of_commutative_frobenius_algebras}
Dagger categories generalize the category of Hilbert spaces which is often used to model quantum
computation. These were introduced in \cite{abramsky04:catsemquantprot} as \emph{strongly compact
closed categories}, an additional structure on compact closed categories.

Before introducing dagger categories, we define compact closed
categories.



\begin{definition}\label{def:compactclosedcat}
A \emph{compact closed category} \cD{} is a symmetric monoidal category with tensor $\*$ where each
object $A$ has a dual $A^{*}$. Additionally, there must exist families of maps $\eta_{A}: I \to
A^{*} \* A$ (the \emph{unit}) and $\epsilon_{A}: A\*A^{*}\to I$ (the \emph{counit}) such that
\[
  \xymatrix@C+15pt{
    A \ar[r]^{u_{A}} \ar@{=}[d]  & A\*I \ar[r]^(.4){1\*\eta_{A}}
        & A\* (A^{*}\*A) \ar[d]^{a_{A,A^{*},A}} \\
    A & I\* A \ar[l]^{u_{A}^{-1}} & (A\* A^{*})\*A \ar[l]^(.6){\epsilon_{A}\*1}
    }\text{ and }
  \xymatrix@C+15pt{
    A^{*} \ar[r]^{u_{A^*}} \ar@{=}[d]  & I\*A^* \ar[r]^(.4){\eta_{A}\*1}
        & (A^{*}\* A)\*A^{*} \ar[d]^{a_{A^{*},A,A^{*}}^{-1}} \\
    A^* & A^*\*I \ar[l]^{u_{A^*}^{-1}} & A^{*}\*(A\*A^{*}) \ar[l]^(.6){1\*\epsilon_{A}}
    }
  \]
commute.
\end{definition}

Given a map $f:A\to B$ in a compact closed category,  define the map $f^{*}:B^{*} \to A^{*}$ as
\[
  \xymatrix@C+10pt{
    B^{*}\ar[r]^{u_{B^{*}}} \ar[d]_{f^{*}}& I\*B^{*} \ar[r]^{\eta_{A}\*1}
      & A^{*}\*A\*B^{*}\ar[d]^{1\*f\*1}\\
    A^{*}&    A^{*}\*I\ar[l]^{u_{A^{*}}^{-1}}  &   A^{*}\*B\*B^{*}\ar[l]^{1\*\epsilon_{B}}.
  }
\]


%!TEX root = /Users/gilesb/UofC/thesis/phd-thesis/phd-thesis.tex
\subsection{Dagger categories}\label{ssec:daggercategories}

Although dagger categories were introduced in the context of compact closed categories, the concept
of a dagger is definable independently. This was first done in \cite{selinger05:dagger}.

\begin{definition}\label{def:daggercat}
  A \emph{dagger} on a category $D$ is a functor $\dagger:\dual{\cD}\to \cD$, which is  involutive,
  that is, $\dgr{\dgr{f}} = f$ and which is the identity on objects. A \emph{dagger category} is a
  category that has a dagger.
\end{definition}

Typically, the dagger is written as a superscript on the morphism. So, if $f:A\to B$ is a map in
\cD, then $\dgr{f}:B\to A$ is a map in \cD{} and is called the \emph{adjoint} of $f$. A map where
$f^{-1} = \dgr{f}$ is called \emph{unitary}. A map $f:A\to A$ with $f=\dgr{f}$ is called
\emph{self-adjoint} or \emph{Hermitian}.

\begin{definition}\label{def:daggersmc}
  A \emph{dagger symmetric monoidal category} is a symmetric monoidal category \cD{} with a dagger
  operator such that:
  \begin{enumerate}[{(}i{)}]
    \item For all maps $f:A\to B$ and $g:C\to D$, $\dgr{(f\*g)} = \dgr{f}\*\dgr{g}:B\*D \to A\* C$;\label{defitem:dagger_smc_one}
    \item The monoid structure isomorphisms $a_{A,B,C}:(A\*B)\* C\to A\*(B\*C)$, $u^l_{A}:I\*A\to
      A$, $u^r_{A}:A\*I \to A$ and  $c_{A,B}:A\*B \to B\*A$ are unitary.\label{defitem:dagger_smc_two}
  \end{enumerate}
\end{definition}


\begin{definition}\label{def:daggercompact}
  A \emph{dagger compact closed category} \cD{} is a dagger symmetric monoidal category
  that is compact closed where the diagram
  \[
    \xymatrix @C+20pt @R+10pt{
      I \ar[r]^{\epsilon^{\dagger}_{A}} \ar[dr]_{\eta_{A}} &A\*A^{*}\ar[d]^{c_{A,A^{*}}}\\
      &A^{*}\* A
    }
  \]
  commutes for all  objects $A$ in \cD.
\end{definition}

\begin{lemma}\label{lemma:daggerbiproducts}
If \cD{} is a dagger category with biproducts, with injections $in_{1},in_{2}$ and projections
$p_{1},p_{2}$, then the following are equivalent:
\begin{enumerate}[{(}i{)}]
  \item $\dgr{p_{i}} = in_{i}, i=1,2$, \label{ldpdgrpisq}
  \item $\dgr{(f\biproduct g)} = \dgr{f}\biproduct \dgr{g}$ and $\dgr{\Delta} = \nabla$,\label{ldpddeltisnab}
  \item $\dgr{\<f,g\>} = [\dgr{f},\dgr{g}]$,\label{ldpdcopisprod}
  \item The map $[\dgr{p_{1}},\dgr{p_{2}}]: \dgr{A} \biproduct \dgr{B} \to \dgr{(A\biproduct B)}$ is
    the identity map.\label{ldpcommute}
%the below diagram commutes:
%  \[
%    \xymatrix @C+20pt @R+10pt{
%      \dgr{A} \biproduct \dgr{B} \ar[d]_{id} \ar[dr]^{[\dgr{p_{1}},\dgr{p_{2}}]}\\
%      A\biproduct B\ar[r]_{id}&\dgr{(A\biproduct B)}.
%    }
%  \]
\end{enumerate}
\end{lemma}
\begin{proof}
  \begin{description}
    \item[\ref{ldpdgrpisq}$\implies$\ref{ldpddeltisnab}] To show $\dgr{\Delta} = \nabla$,
    draw the product cone for $\Delta$,
    \[
      \xymatrix {
        &A \ar[d]^{\Delta} \ar[dr]^{id} \ar[dl]_{id}\\
        A
         & A\biproduct A \ar[l]^{p_{1}}  \ar[r]_{p_{2}}
         & A
      }
    \]
    and apply the dagger functor to it. As $\dgr{p_{i}} = in_{i}$, and $\dagger$ is identity on
    objects, this is now a coproduct diagram and therefore $\dgr{\Delta} = \nabla$.

    For $\dgr{(f\biproduct g)} = \dgr{f}\biproduct\dgr{g}$, start with the diagram defining
    $f\biproduct g$ as a product of the arrows:
    \[
      \xymatrix{
        A\ar[d]_{f}  & A\biproduct B \ar[l]_{p_{1}} \ar[r]^{p_{2}} \ar[d]^{f\biproduct g}&A \ar[d]^{g}\\
        C & C\biproduct D \ar[l]^{p_{1}} \ar[r]_{p_{2}}  & D.
      }
    \]
    Then, apply the dagger functor to this diagram. This is now the diagram defining the
    coproduct of maps and therefore $\dgr{(f\biproduct g)} = \dgr{f}\biproduct\dgr{g}$.
    \item[\ref{ldpddeltisnab}$\implies$\ref{ldpdcopisprod}] The calculation showing this is
      \begin{eqnarray*}
        &[\dgr{f},\dgr{g}] & = \nabla; (\dgr{f}\biproduct \dgr{g})\\
        & &=\dgr{\Delta}; (\dgr{f}\biproduct \dgr{g})\\
        & &=\dgr{\Delta}; \dgr{(f\biproduct g)}\\
        & & = \dgr{((f\biproduct g);\Delta)}\\
        & & = \dgr{\<f,g\>}.
      \end{eqnarray*}
    \item[\ref{ldpdcopisprod}$\implies$\ref{ldpcommute}]
      Under the assumption,
      \[
        [\dgr{p_{1}},\dgr{p_{2}}] = \dgr{\<p_{1},p_{2}\>}=\dgr{id}=id.
      \]
    \item[\ref{ldpcommute}$\implies$\ref{ldpdgrpisq}] As $[in_{1},in_{2}]:\dgr{A} \biproduct \dgr{B}
      \to \dgr{A} \biproduct \dgr{B} = id = [\dgr{p_{1}},\dgr{p_{2}}]$, we immediately have
      $\dgr{p_{1}} = in_{1}$ and $\dgr{p_{2}} = in_{2}$.
%
%Using the injections and under
%    the assumption, the following diagram commutes:
%      \[
%        \xymatrix @C+20pt @R+10pt{
%          \dgr{A} \biproduct \dgr{B} \ar[d]_{id} \ar[dr]^{[\dgr{p_{1}},\dgr{p_{2}}]}\ar[r]^{[in_{1},in_{2}]}
%            & \dgr{A} \biproduct \dgr{B} \ar[d]^{id}\\
%          A\biproduct B\ar[r]_{id}&\dgr{(A\biproduct B)}
%        }
%      \]
%      and therefore,
  \end{description}
\end{proof}

\begin{definition} \label{def:biproductdaggerccc}
  A \emph{biproduct dagger compact closed category} is a dagger compact closed category with
  biproducts where the conditions of lemma \ref{lemma:daggerbiproducts} hold.
\end{definition}
\subsection{Examples of dagger categories}

\begin{example}[\fdh]\label{ex:fdhilbert_is_dagger_category}
The category of finite dimensional Hilbert spaces is the motivating example for
the creation of the dagger and is, in fact, a biproduct dagger compact closed category. The
biproduct is the direct sum of Hilbert spaces and the tensor for compact closure is the standard
tensor of Hilbert spaces. The dual $H^{*}$ of a space $H$ is the space of all continuous linear
functions from $H$ to the base field. The dagger is defined via the adjoint as being the unique map
$\dgr{f}:B\to A$ such that $\<f a|b\> = \<a | \dgr{f} b\>$ for all $a\in A, b\in B$.
\end{example}

\begin{example}[\rel]\label{ex:rel_is_dagger_category}
The category \rel of sets and relations has the tensor $S\*T \definedas S\times T$ and the biproduct
$S\biproduct T \definedas S\disjointunion T$. This is compact closed under $A^{*} \definedas A$ and
the dagger is the relational converse. That is, if the relation
$R=\{(s,t)|s\in S, t\in T\}:S\to T$, then $\dgr{R}=R^*=\{(t,s)|(s,t)\in R\}$.
\end{example}

\begin{example}[Inverse categories]\label{ex:inverse_category_is_dagger_category}
An inverse category \X is also a dagger category when the dagger is defined as the partial inverse.
The unitary maps are the total maps. When the inverse category \X is also a
symmetric monoidal category where the monoid $\*$ is actually a restriction bi-functor, then \X is
a dagger symmetric monoidal category.

Requirement \ref{defitem:dagger_smc_one} of Definition~\ref{def:daggersmc}  is fulfilled, as
\[
  (f\*g) \inv{(f\*g)} = \rst{f\*g}=\rst{f} \*\rst{g} =
   f\inv{f} \* g \inv{g} = (f\*g) (\inv{f} \* \inv{g})
\]
and since the partial inverse of $f\*g$ is unique, $\inv{(f\*g)} = \inv{f} \* \inv{g}$.
Requirement \ref{defitem:dagger_smc_two} is that the structure isomorphisms are unitary. This is, of
course, true as each of them are isomorphisms, hence total and therefore unitary.
\end{example}
%%% Local Variables:
%%% mode: latex
%%% TeX-master: "../../phd-thesis"
%%% End:

\section{Frobenius Algebras} % (fold)
\label{sec:frobenius_algebras}
In their most general setting, Frobenius algebras are defined as a finite dimensional algebra
over a field together with a non-degenerate pairing operation. We will continue with the definitions
that make this precise.

\subsection{Frobenius algebra definitions} % (fold)
\label{sub:frobenius_algebra_definitions}


\begin{definition}\label{def:frobeniusalgebra}
  Given a symmetric monoidal category \cD, a \emph{Frobenius algebra} is an object $X$ of \cD and
  four maps, $\nabla :X\*X \to X$, $e: I \to X$, $\Delta: X\to X\* X$ and $\epsilon:X\to I$, with
  the conditions that $(X,\nabla,e)$ forms a commutative monoid, $(X,\Delta, \epsilon)$ forms a
  commutative comonoid and the diagram
  \[
    \xymatrix{
      X\*X \ar[rr]^{X\*\Delta} \ar[dd]_{\Delta \* X} \ar[dr]^{\nabla}
        && X\*X\*X \ar[dd]^{\nabla\*X}\\
      & X\ar[dr]^{\Delta}\\
      X\*X\*X\ar[rr]_{X\*\nabla}  && X\*X
    }
  \]
  commutes. The Frobenius algebra is \emph{special} when $\Delta \nabla = 1_{X}$ and
  \emph{commutative} when $\Delta c_{X,X} = \Delta$.
\end{definition}
\begin{definition}\label{def:daggerfrob}
  A Frobenius algebra in a dagger symmetric monoidal category where $\Delta = \dgr{\nabla}$ and
  $\epsilon=\dgr{u}$ is a $\dagger$\emph{-Frobenius algebra}.
\end{definition}
For an example of a $\dagger$-Frobenius algebra, consider a finite dimensional Hilbert space $H$
with an orthonormal basis $\{\ket{\phi_{i}}\}$ and define $\Delta:H\to H\*H: \ket{\phi_{i}}\mapsto
\ket{\phi_{i}} \* \ket{\phi_{i}}$ and $\epsilon : H\to \complex : \ket{\phi_{i}} \mapsto 1$. Then $(H,
\nabla=\dgr{\Delta}, u=\dgr{\epsilon}, \Delta, \epsilon)$ forms a commutative special
$\dagger$-Frobenius algebra.

% subsection frobenius_algebra_definitions (end)

\subsection{Bases and Frobenius Algebras} % (fold)
\label{sub:bases_and_frobenius_algebras}
In \cite{coeckeetal08:ortho}, Coecke et. al. provide an algebraic description of orthogonal bases in
finite dimensional Hilbert spaces. Additionally,  an orthonormal basis for such a space is
a special commutative $\dagger$-Frobenius algebra. To show the other direction, given a commutative
$\dagger$-Frobenius algebra, $(H,\nabla,u)$ and for each element $\alpha\in H$, define the right
action of $\alpha$ as $R_{\alpha}:=(id\*\alpha)\, \nabla:H\to H$. Note the use of the fact that
elements $\alpha\in H$ can be considered as linear maps $\alpha:\complex \to H:1\mapsto \ket{\alpha}$.
The dagger of a right action is also a right action, $\dgr{R_{\alpha}} = R_{\alpha'}$ where
$\alpha'= u\, \nabla\, (id\* \dgr{\alpha})$, which is a consequence of the Frobenius identities.

The $(\_)'$ construction is actually an involution:
\begin{eqnarray*}
  &(\alpha')' &= u \nabla (id \* \dgr{\alpha'}) \\
  && = u \nabla (id \* \dgr{(u \nabla (id \* \dgr{\alpha}))}\\
  && = u \nabla (id \* ( (id \* \alpha) \Delta \epsilon))\\
  && = (u \* \alpha) (\nabla \* id) (id \* \Delta) (id \*  \epsilon)\\
  && = (u \* \alpha) (id \* \Delta) (\nabla \* id) (id \*  \epsilon)\\
  && = (u \* \alpha)  (id \*  \epsilon)\\
  && = \alpha
\end{eqnarray*}

\begin{lemma}\label{lemma:cstaralgebra}
  Any $\dagger$-Frobenius algebra in \fdh is a $C^{*}$-algebra.
\end{lemma}
\begin{proof}
  The endomorphism monoid of \fdh(H,H) is a $C^{*}$-algebra. From the proceeding, we have
  \[
    H \cong \fdh(\complex,H) \cong R_{[\fdh(\complex,H)]}\subseteq\fdh(H,H).
  \]
  This inherits the algebra structure from \fdh(H,H). Furthermore, since any finite dimensional
  involution-closed sub-algebra of a $C^{*}$-algebra is also a $C^{*}$-algebra, this shows the
  $\dagger$-Frobenius algebra is a $C^{*}$-algebra.
\end{proof}

Using the fact that the involution preserving homomorphisms from a finite dimensional commutative
$C^{*}$-algebra to $\complex$ form a basis for the dual of the underlying vector space, write these
homomorphisms as $\dgr{\phi_{i}}:H \to \complex$. Then their adjoints, $\phi_{i}:\complex\to H$ will form a
basis for the space $H$. These are the copyable elements in $H$.

This, together with continued applications of the Frobenius rules and linear algebra allow the
authors to prove the following Theorem.
\begin{theorem}
  Every commutative $\dagger$-Frobenius algebra in \fdh determines an orthogonal basis consisting
  of its copyable elements. Conversely, every orthogonal basis $\{\ket{\phi_{i}}\}_{i}$ determines
  a commutative $\dagger$-Frobenius algebra via \[\Delta:H\to H\*H: \ket{\phi_{i}}\mapsto
  \ket{\phi_{i}} \* \ket{\phi_{i}}\qquad\epsilon : H\to \complex : \ket{\phi_{i}} \mapsto 1\] and these
  constructions are inverse to each other.
\end{theorem}

% subsection bases_and_frobenius_algebras (end)

\subsection{Quantum and classical data}\label{sec:quantumclassical}
In \cite{coecke08structures}, Coecke et.al. build on the results of \cite{coeckeetal08:ortho}
to start from a $\dagger$-symmetric monoidal category and construct the minimal machinery needed to
model quantum and classical computations. For the rest of this section, $\cD$ will be assumed to be
such a category, with $\*$ the monoid tensor and $I$ the unit of the monoid.

\begin{definition}\label{def:compact_structure}
  A compact structure on an object $A$ in the category $\cD$ is given by the object $A$, an object
  $A^{*}$ called its \emph{dual} and the maps $\eta:I \to A^{*}\* A$, $\epsilon: A\* A^{*} \to I$
  such that the diagrams
  \[
    \xymatrix@C+20pt{
      A^{*} \ar[dr]^{id} \ar[d]_{\eta\*A^{*}} \\
      A^{*} \*A\*A^{*}  \ar[r]_(.6){A^{*} \*\epsilon} & A^{*}
    }
    \text{ and }
    \xymatrix@C+20pt{
      A \ar[r]^(.4){A\*\eta} \ar[dr]_{id} & A\* A^{*}\* A \ar[d]^{\epsilon\*A}\\
      & A
    }
  \]
  commute.
\end{definition}

\begin{definition}\label{def:quantumstructure}
  A \emph{quantum structure} is an object $A$ and map $\eta:I\to A\*A$ such that
  $(A,A,\eta,\dgr{\eta})$ form a compact structure.
\end{definition}
Note that $A$ is self-dual in definition \ref{def:quantumstructure}.

This allows the creation of the category $\cD_{q}$ which has as objects quantum structures and maps
are the maps in $\cD$ between the objects in the quantum structures.

In the category $\cD_{q}$, it is now possible to define the upper and lower $*$ operations on maps,
such that $(f_{*})^{*}= (f^{*})_{*} = \dgr{f}$:
\begin{eqnarray*}
&f^{*} &:= (\eta_{A}\*1) (1 \* f\*1) (1\*\dgr{\eta}_{B}),\\
&f_{*} &:= (\eta_{B}\*1) (1 \* \dgr{f}\*1) (1\*\dgr{\eta}_{A}).
\end{eqnarray*}

Next, define a classical structure on \cD.
\begin{definition}\label{def:classicalstructure}
  A \emph{classical structure} in \cD{} is an object $X$ together with two maps, $\Delta :X \to X\* X$,
  $\epsilon:X\to I$ such that $(X,\dgr{\Delta},\dgr{\epsilon},\Delta,\epsilon)$ forms a special
  Frobenius algebra.
\end{definition}

As above, this allows us to define $\cD_{c}$, the category whose objects are the classical
structures of $\cD$. The maps in $\cD_{c}$ are given by the maps in $\cD$ between the
objects of the classical structure.

Note that a classical structure will induce a quantum structure, setting $\eta_{X}$ to be
$\dgr{\epsilon_{X}}\, \Delta_{X}$.


Later on, in \ref{sec:the_category_of_commutative_frobenius_algebras}, we will show that commutative
special Frobenius algebras possess a specialized inverse category structure.
% subsection quantum_and_classical_data (end)


% section frobenius_algebras (end)

%%% Local Variables:
%%% mode: latex
%%% TeX-master: "../../phd-thesis"
%%% End:


\subsection{\CFrob is an inverse category}\label{ssec:cfrob_x_is_an_inverse_category}
\begin{example}[Commutative separable Frobenius algebras\cite{kock04}]\label{example:commfrob}
  Let \X be a symmetric monoidal category and form \CFrob as follows: \paragraph{\textbf{Objects:}}
  Commutative separable Frobenius algebras: These are quintuples
  $(A,\nabla,\eta,\Delta,\epsilon)$ where $A$ is an object of \X with the following maps:
  $\nabla :A\*A \to A$, $\eta:I\to A$, $\Delta : A \to A\*A$, $\epsilon : A \to I$ which are natural
  maps in \X, with $(A,\nabla,\eta)$ a monoid and $(A,\Delta,\epsilon)$ a comonoid. Additionally,
  these satisfy
  \[
    \xymatrix @C=10pt @R=20pt{
      A \* A \ar[dd]_{1\*\Delta} \ar[dr]^{\nabla}
        \ar[rr]^{\Delta \* 1} & &
        A \* (A \* A) \ar[dd]^{1 \* \nabla}\\
      & A \ar[dr]^{\Delta} & \\
      (A \* A) \* A \ar[rr]_{\nabla \* 1} & &
        A \* A\\
      &*!<0pt,-25pt>{\text{\textbf{Frobenius}}}
    }
  \]
  together with the additional property that $\Delta \nabla = 1$ (separable).

  \paragraph{\textbf{Maps:}} The maps of \X between the objects of \X which preserve multiplication ($\nabla$)
  and comultiplication ($\Delta$) but do not necessarily preserve the units.
  This means a map $f$ must satisfy the following commuting diagrams:
  \[
    \xymatrix@C+25pt{
      A \ar[d]_{\Delta} \ar[r]^{f} & B \ar[d]^{\Delta}\\
      A\*A \ar[r]_{f\*f} & B\* B
    }
    \text{ and }
    \xymatrix@C+25pt{
      A\*A \ar[d]_{\nabla} \ar[r]^{f\*f}& B\*B \ar[d]^{\nabla}\\
      A \ar[r]_{f} & B.
    }
  \]
\end{example}

\begin{lemma}\label{lem:cfrobx_is_an_inverse_category}
  When \X is a symmetric monoidal category, \CFrob is an inverse category.
\end{lemma}
\begin{proof}
  We need to show that \CFrob has restrictions and that each map has a partial inverse. We do
  this by exhibiting the partial inverse of a map.
  For $f:X \to Y$, define $\inv{f}$ as
  \[
    Y \xrightarrow{1\*\eta} Y\*X \xrightarrow{1\*\Delta}
      Y\*X\*X \xrightarrow{1\*f\*1} Y\*Y\*X \xrightarrow{\nabla\*1}
      Y\*X \xrightarrow{\epsilon\*1}X.
  \]
  As a string diagram, this looks like:
  \[
  \begin{tikzpicture}
    \path node at (-.5,3) (start) {}
    node at (0,2.5) [eta] (eta1) {}
    node at (0,2) [delta] (d) {}
    node at (-.25,1.5) [map] (f) {$\scriptstyle f$}
    node at (-.5,1) [nabla] (n1) {}
    node at (-.5,.5) [epsilon] (e1) {}
    node at (0,0) (end) {};
    \draw [] (start) to[out=270,in=125] (n1);
    \draw [] (eta1) to (d);
    \draw [] (d) to[out=305,in=90] (end);
    \draw [] (d) to[out=235,in=90] (f);
    \draw [] (f) to[out=270,in=55] (n1);
    \draw [-] (n1) to (e1);
  \end{tikzpicture}
  \ \raisebox{25pt}{\text{.}}
  \]

  In the following proofs, we also use the following two identities from \cite{kock04}:
  \begin{align}
    (1\*\eta)\nabla &= 1,\\
    \Delta(1\*\epsilon) &= 1.
  \end{align}
  Diagrammatically, this is:
  \[
    \begin{tikzpicture}
    \path   node at (.5,1) (start) {}
    node at (0,1) [eta] (eta1) {}
    node at (.25,.5) [nabla] (n1) {}
    node at (.25,0) (end) {};
    \draw [] (eta1) to[out=270,in=125] (n1);
    \draw [] (start) to[out=270,in=55] (n1);
    \draw [] (n1)   to (end);
  \end{tikzpicture}
  \ \raisebox{15pt}{\text{= }}
  \begin{tikzpicture}
    \path node at (0,1) (start) {}
    node at (0,0) (end) {};
    \draw [-] (start) to (end);
  \end{tikzpicture}
  \ \raisebox{15pt}{\text{=}}
  \begin{tikzpicture}
    \path node at (0,1) (start) {}
    node at (0,.5) [delta] (d1) {}
    node at (-.25,0) (end) {}
    node at (.25,0) [epsilon] (e1) {};
    \draw [] (start) to (d1);
    \draw [] (d1) to[out=305,in=90] (e1);
    \draw [] (d1) to[out=235,in=90] (end);
  \end{tikzpicture}
  \ \raisebox{15pt}{.}
  \]
  Note that when combined with the Frobenius identities, this allows transforms of the following
  types:
  \[
  \begin{tikzpicture}
    \path node at (0,1.5) (s1) {}
    node at (.75,1.5) (s2) {}
    node at (0,1) [delta] (d1) {}
    node at (.5,.5) [nabla] (n1) {}
    node at (0,0) (end) {}
    node at (.5,0) [epsilon] (e1) {};
    \draw [] (s1) to (d1);
    \draw [] (s2) to[out=270,in=55] (n1);
    \draw [] (d1) to[out=235,in=90] (end);
    \draw [] (d1) to[out=305,in=125] (n1);
    \draw [] (n1) to (e1);
  \end{tikzpicture}
  \raisebox{15pt}{$=$}
  \begin{tikzpicture}
    \path node at (0,1.5) (s1) {}
    node at (.5,1.5) (s2) {}
    node at (.25,1) [nabla] (n1) {}
    node at (.25,.5) [delta] (d1) {}
    node at (0,0) (end) {}
    node at (.5,0) [epsilon] (e1) {};
    \draw [] (s1) to[out=270,in=125] (n1);
    \draw [] (s2) to[out=270,in=55] (n1);
    \draw [] (d1) to[out=235,in=90] (end);
    \draw [] (d1) to[out=305,in=90] (e1);
    \draw [] (n1) to (d1);
  \end{tikzpicture}
  \raisebox{15pt}{$=$}
  \begin{tikzpicture}
    \path node at (0,1.5) (s1) {}
    node at (.5,1.5)  (s2) {}
    node at (.25,1) [nabla] (n1) {}
    node at (.25,0.5) (end) {};
    \draw [] (s1) to[out=270,in=125] (n1);
    \draw [] (s2) to[out=270,in=55] (n1);
    \draw [] (n1) to (end);
  \end{tikzpicture}
  \raisebox{15pt}{ and }
  \begin{tikzpicture}
    \path node at (0,1.5)  [eta](s1) {}
    node at (.75,1.5) (s2) {}
    node at (0,1) [delta] (d1) {}
    node at (.5,.5) [nabla] (n1) {}
    node at (0,0) (end) {}
    node at (.5,0) (e1) {};
    \draw [] (s1) to (d1);
    \draw [] (s2) to[out=270,in=55] (n1);
    \draw [] (d1) to[out=235,in=90] (end);
    \draw [] (d1) to[out=305,in=125] (n1);
    \draw [] (n1) to (e1);
  \end{tikzpicture}
  \raisebox{15pt}{$=$}
  \begin{tikzpicture}
    \path node at (0,1.5)  [eta] (s1) {}
    node at (.5,1.5) (s2) {}
    node at (.25,1) [nabla] (n1) {}
    node at (.25,.5) [delta] (d1) {}
    node at (0,0) (end) {}
    node at (.5,0)  (e1) {};
    \draw [] (s1) to[out=270,in=125] (n1);
    \draw [] (s2) to[out=270,in=55] (n1);
    \draw [] (d1) to[out=235,in=90] (end);
    \draw [] (d1) to[out=305,in=90] (e1);
    \draw [] (n1) to (d1);
  \end{tikzpicture}
  \raisebox{15pt}{$=$}
  \begin{tikzpicture}
    \path node at (.25,1.5) (s1) {}
    node at (.25,1) [delta] (d1) {}
    node at (0,0.5) (end) {}
    node at (.5,0.5) (end1) {};
    \draw [] (s1) to (d1);
    \draw [] (d1) to[out=255,in=90] (end);
    \draw [] (d1) to[out=305,in=90] (end1);
  \end{tikzpicture}
  \raisebox{15pt}{.}
  \]

  First, we must show that $\inv{f}$ is a map in the category, i.e., that $\Delta (\inv{f} \*
  \inv{f}) = \inv{f} \Delta$ and $(\inv{f} \* \inv{f})\nabla = \nabla \inv{f}$. We show this for
  $\Delta$ using string diagrams, starting from $\Delta(\inv{f} \*
  \inv{f})$. The proof for the preservation of $\nabla$ proceeds in a similar manner.
  \[
  \begin{tikzpicture}
    \path node at (0,3) (start) {}
    node at (-.75,2.5) [eta] (eta1) {}
    node at (0,2.5) [delta] (d0) {}
    node at (.75,2.5) [eta] (eta2) {}
    node at (-.75,2) [delta] (d) {}
    node at (.75,2) [delta] (d2) {}
    node at (-.5,1.5) [map] (f) {$\scriptstyle f$}
    node at (.5,1.5) [map] (f2) {$\scriptstyle f$}
    node at (-.25,1) [nabla] (n1) {}
    node at (.25,1) [nabla] (n2) {}
    node at (-.25,.5) [epsilon] (e1) {}
    node at (.25,.5) [epsilon] (e2) {}
    node at (-.75,0) (end) {}
    node at (.75,0) (end2) {};
    \draw [] (start) to (d0);
    \draw [] (eta1) to (d);
    \draw [] (eta2) to (d2);
    \draw (d0) to[out=235,in=55] (n1);
    \draw (d0) to[out=305,in=125] (n2);
    \draw [] (d) to[out=235,in=90] (end);
    \draw [] (d) to[out=305,in=90] (f);
    \draw [] (f) to[out=270,in=125] (n1);
    \draw [-] (n1) to (e1);
    \draw [] (d2) to[out=305,in=90] (end2);
    \draw [] (d2) to[out=235,in=90] (f2);
    \draw [] (f2) to[out=270,in=55] (n2);
    \draw [-] (n2) to (e2);
  \end{tikzpicture}
  \raisebox{45pt}{$=$}
  \begin{tikzpicture}
    \path node at (0,3.5) (start) {}
    node at (-.75,2.5) [eta] (eta1) {}
    node at (.75,3) [eta] (eta2) {}
    node at (-.75,2) [delta] (d) {}
    node at (.75,2.5) [delta] (d2) {}
    node at (-.5,1.5) [map] (f) {$\scriptstyle f$}
    node at (.5,2) [map] (f2) {$\scriptstyle f$}
    node at (-.25,1) [nabla] (n1) {}
    node at (.25,1.5) [nabla] (n2) {}
    node at (-.25,.5) [epsilon] (e1) {}
    node at (-.75,0) (end) {}
    node at (.75,0) (end2) {};
    \draw [] (start) to[out=270,in=125] (n2);
    \draw [] (eta1) to (d);
    \draw [] (eta2) to (d2);
    \draw [] (d) to[out=235,in=90] (end);
    \draw [] (d) to[out=305,in=90] (f);
    \draw [] (f) to[out=270,in=125] (n1);
    \draw [-] (n1) to (e1);
    \draw [] (d2) to[out=305,in=90] (end2);
    \draw [] (d2) to[out=235,in=90] (f2);
    \draw [] (f2) to[out=270,in=55] (n2);
    \draw [-] (n2) to[out=270,in=55] (n1);
  \end{tikzpicture}
  \raisebox{45pt}{$=$}
  \begin{tikzpicture}
    \path node at (-0.25,3.5) (start) {}
    node at (-.75,3) [eta] (eta1) {}
    node at (.75,3) [eta] (eta2) {}
    node at (-.75,2.5) [delta] (d) {}
    node at (.75,2.5) [delta] (d2) {}
    node at (-.5,2) [map] (f) {$\scriptstyle f$}
    node at (.5,2) [map] (f2) {$\scriptstyle f$}
    node at (0,1.5) [nabla] (n1) {}
    node at (0.1,1) [nabla] (n2) {}
    node at (0.1,.5) [epsilon] (e1) {}
    node at (-.75,0) (end) {}
    node at (.5,0) (end2) {};
    \draw [] (start) to[out=270,in=55] (n2);
    \draw [] (eta1) to (d);
    \draw [] (eta2) to (d2);
    \draw [] (d) to[out=235,in=90] (end);
    \draw [] (d) to[out=305,in=90] (f);
    \draw [] (f) to[out=270,in=125] (n1);
    \draw [-] (n1) to[out=270,in=125] (n2);
    \draw [] (d2) to[out=305,in=90] (end2);
    \draw [] (d2) to[out=235,in=90] (f2);
    \draw [] (f2) to[out=270,in=55] (n1);
    \draw [-] (n2) to (e1);
  \end{tikzpicture}
  \raisebox{45pt}{$=$}
  \begin{tikzpicture}
    \path node at (.75,3.5) (start) {}
    node at (-.25,3) [eta] (eta1) {}
    node at (.25,3) [eta] (eta2) {}
    node at (-.25,2.5) [delta] (d) {}
    node at (.25,2.5) [delta] (d2) {}
    node at (0,2) [nabla] (n1) {}
    node at (0,1.5) [map] (f) {$\scriptstyle f$}
    node at (0.25,1) [nabla] (n2) {}
    node at (0.25,.5) [epsilon] (e1) {}
    node at (-.5,0) (end) {}
    node at (.75,0) (end2) {};
    \draw [] (start) to[out=270,in=55] (n2);
    \draw [] (eta1) to (d);
    \draw [] (eta2) to (d2);
    \draw [] (d) to[out=235,in=90] (end);
    \draw [] (d) to[out=305,in=125] (n1);
    \draw [] (d2) to[out=305,in=90] (end2);
    \draw [] (d2) to[out=235,in=55] (n1);
    \draw [-] (n1) to[out=270,in=90] (f);
    \draw [] (f) to[out=270,in=125] (n2);
    \draw [-] (n2) to (e1);
  \end{tikzpicture}
  \raisebox{45pt}{$=$}
  \begin{tikzpicture}
    \path node at (.75,3.5) (start) {}
    node at (.25,3) [eta] (eta2) {}
    node at (.25,2.5) [delta] (d2) {}
    node at (-0.25,2) [delta] (d) {}
    node at (0,1.5) [map] (f) {$\scriptstyle f$}
    node at (0.25,1) [nabla] (n2) {}
    node at (0.25,.5) [epsilon] (e1) {}
    node at (-.5,0) (end) {}
    node at (.5,0) (end2) {};
    \draw [] (start) to[out=270,in=55] (n2);
    \draw [] (eta2) to (d2);
    \draw [] (d2) to[out=305,in=90] (end2);
    \draw [] (d2) to[out=235,in=90] (d);
    \draw [] (d) to[out=235,in=90] (end);
    \draw [] (d) to[out=305,in=90] (f);
    \draw [] (f) to[out=270,in=125] (n2);
    \draw [-] (n2) to (e1);
  \end{tikzpicture}
  \ \raisebox{45pt}{$=$}
  \begin{tikzpicture}
    \path node at (.75,3) (start) {}
    node at (0,2.5) [eta] (eta2) {}
    node at (0,2) [delta] (d2) {}
    node at (-0.25,1.5) [delta] (d) {}
    node at (.25,1.5) [map] (f) {$\scriptstyle f$}
    node at (0.5,1) [nabla] (n2) {}
    node at (0.5,.5) [epsilon] (e1) {}
    node at (-.5,0) (end) {}
    node at (0,0) (exit) {};
    \draw [] (start) to[out=270,in=55] (n2);
    \draw [] (eta2) to (d2);
    \draw [] (d2) to[out=235,in=90] (d);
    \draw [] (d2) to[out=305,in=90] (f);
    \draw [] (d) to[out=235,in=90] (end);
    \draw [] (d) to[out=305,in=90] (exit);
    \draw [] (f) to[out=270,in=125] (n2);
    \draw [-] (n2) to (e1);
  \end{tikzpicture}
  \raisebox{45pt}{$=\inv{f}\Delta$.}
  \]
  Thus, $\inv{f}$ is a map in the category whenever $f$ is.

  If $\inv{f}$ is truly a partial inverse, we may then define $\rst{f} = f \inv{f}$.
  Using Theorem 2.20 from \cite{cockett2002:restcategories1}, we need only show:
  \begin{align}
    \inv{(\inv{f})} &= f\label{eq:finvinv_is_f}\\
    f\inv{f}f &= f\label{eq:ffinvf_is_f}\\
    f\inv{f}g\inv{g} &=g\inv{g} f\inv{f}.\label{eq:ffinv_commutes_gginv}
  \end{align}
  Proof of Equation~\ref{eq:finvinv_is_f}: $\inv{(\inv{f})} =$
  \[
  \begin{tikzpicture}
    \path node at (-.75,4) (start) {}
    node at (0,3.5) [eta] (eta2) {}
    node at (0,3) [delta] (d2) {}
    node at (0,2.5) [eta] (eta1) {}
    node at (0,2) [delta] (d1) {}
    node at (-.25,1.5) [map] (f) {$\scriptstyle f$}
    node at (-.5,1) [nabla] (n1) {}
    node at (-.5,.5) [epsilon] (e1) {}
    node at (-.5,0) [nabla] (n2) {}
    node at (-.5,-.5) [epsilon] (e2) {}
    node at (.25,-1) (end) {};
    \draw [] (start) to[out=270,in=125] (n2);
    \draw [] (eta2) to (d2);
    \draw [] (d2) to[out=235,in=125] (n1);
    \draw [] (d2) to[out=305,in=90] (end);
    \draw [] (eta1) to (d1);
    \draw [] (d1) to[out=305,in=55] (n2);
    \draw [] (d1) to[out=235,in=90] (f);
    \draw [] (f) to[out=270,in=55] (n1);
    \draw [-] (n1) to (e1);
    \draw [-] (n2) to (e2);
  \end{tikzpicture}
  \ \raisebox{70pt}{\text{=}}
  \begin{tikzpicture}
    \path node at (.25,4) (start) {}
    node at (-.5,3.5) [eta] (eta2) {}
    node at (-.5,3) [delta] (d2) {}
    node at (0,2.5) [eta] (eta1) {}
    node at (0,2) [delta] (d1) {}
    node at (-.25,1.5) [map] (f) {$\scriptstyle f$}
    node at (-.5,1) [nabla] (n1) {}
    node at (-.5,.5) [epsilon] (e1) {}
    node at (0,0) [nabla] (n2) {}
    node at (0,-.5) [epsilon] (e2) {}
    node at (-.5,-1) (end) {};
    \draw [] (start) to[out=270,in=55] (n2);
    \draw [] (eta2) to (d2);
    \draw [] (d2) to[out=305,in=125] (n1);
    \draw [] (d2) to[out=235,in=90] (end);
    \draw [] (eta1) to (d1);
    \draw [] (d1) to[out=305,in=125] (n2);
    \draw [] (d1) to[out=235,in=90] (f);
    \draw [] (f) to[out=270,in=55] (n1);
    \draw [-] (n1) to (e1);
    \draw [-] (n2) to (e2);
  \end{tikzpicture}
  \ \raisebox{70pt}{\text{= }}
  \begin{tikzpicture}
    \path node at (.25,4) (start) {}
    node at (-.5,1) [eta] (eta2) {}
    node at (-.5,.5) [delta] (d2) {}
    node at (0,3.5) [eta] (eta1) {}
    node at (0,3) [delta] (d1) {}
    node at (-.25,1.5) [map] (f) {$\scriptstyle f$}
    node at (-.25,0) [nabla] (n1) {}
    node at (-.25,-.5) [epsilon] (e1) {}
    node at (.25,2.5) [nabla] (n2) {}
    node at (.25,2) [epsilon] (e2) {}
    node at (-.5,-1) (end) {};
    \draw [] (start) to[out=270,in=55] (n2);
    \draw [] (eta2) to (d2);
    \draw [] (d2) to[out=305,in=125] (n1);
    \draw [] (d2) to[out=235,in=90] (end);
    \draw [] (eta1) to (d1);
    \draw [] (d1) to[out=305,in=125] (n2);
    \draw [] (d1) to[out=235,in=90] (f);
    \draw [] (f) to[out=270,in=55] (n1);
    \draw [-] (n1) to (e1);
    \draw [-] (n2) to (e2);
  \end{tikzpicture}
  \ \raisebox{70pt}{\text{= }}
  \begin{tikzpicture}
    \path node at (.5,4) (start) {}
    node at (0,3.5) [eta] (eta1) {}
    node at (.25,3) [nabla] (n1) {}
    node at (.25,2.5) [delta] (d1) {}
    node at (.5,2) [epsilon] (e1) {}
    node at (0,1.5) [map] (f) {$\scriptstyle f$}
    node at (-.5,1) [eta] (eta2) {}
    node at (-.25,.5) [nabla] (n2) {}
    node at (-.25,0) [delta] (d2) {}
    node at (0,-.5) [epsilon] (e2) {}
    node at (-.5,-1) (end) {};
    \draw [] (start) to[out=270,in=55] (n1);
    \draw [] (eta1) to[out=270,in=125] (n1);
    \draw [] (n1) to (d1);
    \draw [] (d1) to[out=305,in=90] (e1);
    \draw [] (d1) to[out=235,in=90] (f);
    \draw [] (f) to[out=270,in=55] (n2);
    \draw [] (eta2) to[out=270,in=125] (n2);
    \draw [-] (n2) to (d2);
    \draw [] (d2) to[out=305,in=125] (e2);
    \draw [] (d2) to[out=235,in=90] (end);
  \end{tikzpicture}
  \ \raisebox{70pt}{\text{=}}
  \begin{tikzpicture}
    \path node at (.5,4) (start) {}
    node at (0,1.5) [map] (f) {$\scriptstyle f$}
    node at (-.5,-1) (end) {};
    \draw [] (start) to[out=270,in=90] (f);
    \draw [] (f) to[out=270,in=90] (end);
  \end{tikzpicture}
  \ \raisebox{70pt}{\text{= }$f$.}
  \]
  Proof of Equation~\ref{eq:ffinvf_is_f}: $f \inv{f} f =$
  \[
  \begin{tikzpicture}
    \path node at (-.5,3) (start) {}
    node at (0,2.5) [eta] (eta1) {}
    node at (0,2) [delta] (d) {}
    node at (-.75,1.5) [map] (f1) {$\scriptstyle f$}
    node at (-.25,1.5) [map] (f2) {$\scriptstyle f$}
    node at (.25,1.5) [map] (f3) {$\scriptstyle f$}
    node at (-.5,1) [nabla] (n1) {}
    node at (-.5,.5) [epsilon] (e1) {}
    node at (0,0) (end) {};
    \draw [] (start) to[out=270,in=90] (f1);
    \draw [] (f1) to [out=270,in=125] (n1);
    \draw [] (eta1) to (d);
    \draw [] (d) to[out=305,in=90] (f3);
    \draw [] (f3) to[out=270,in=90] (end);
    \draw [] (d) to[out=235,in=90] (f2);
    \draw [] (f2) to[out=270,in=55] (n1);
    \draw [-] (n1) to (e1);
  \end{tikzpicture}
  \ \raisebox{40pt}{$=$ }
  \begin{tikzpicture}
    \path node at (-.5,3) (start) {}
    node at (0,2.5) [eta] (eta1) {}
    node at (0,2) [delta] (d) {}
    node at (-.5,1.5) [nabla] (n1) {}
    node at (-.5,1) [map] (f2) {$\scriptstyle f$}
    node at (.25,1) [map] (f3) {$\scriptstyle f$}
    node at (-.5,.5) [epsilon] (e1) {}
    node at (0,0) (end) {};
    \draw [] (start) to [out=270,in=125] (n1);
    \draw (n1) to (f2);
    \draw (f2) to (e1);
    \draw [] (eta1) to (d);
    \draw [] (d) to[out=305,in=90] (f3);
    \draw [] (f3) to[out=270,in=90] (end);
    \draw [] (d) to[out=235,in=55] (n1);
  \end{tikzpicture}
  \ \raisebox{40pt}{$=$ }
  \begin{tikzpicture}
    \path node at (-.5,3) (start) {}
    node at (0,2.5) [eta] (eta1) {}
    node at (-.25,2) [nabla] (n1) {}
    node at (-.25,1.5) [delta] (d) {}
    node at (-.5,1) [map] (f2) {$\scriptstyle f$}
    node at (0,1) [map] (f3) {$\scriptstyle f$}
    node at (-.5,.5) [epsilon] (e1) {}
    node at (0,0) (end) {};
    \draw [] (start) to[out=270,in=125] (n1);
    \draw [] (eta1) to[out=270,in=55] (n1);
    \draw [] (n1) to (d);
    \draw [] (d) to[out=305,in=90] (f3);
    \draw [] (d) to[out=235,in=90] (f2);
    \draw (f2) to (e1);
    \draw [] (f3) to[out=270,in=90] (end);
  \end{tikzpicture}
  \ \raisebox{40pt}{\text{= }}
  \begin{tikzpicture}
    \path node at (-.5,3) (start) {}
    node at (-.25,1.5) [map] (f3) {$\scriptstyle f$}
    node at (-.25,1) [delta] (d) {}
    node at (-.5,.5) [epsilon] (e1) {}
    node at (0,0) (end) {};
    \draw [] (start) to[out=270,in=90] (f3);
    \draw [] (f3) to (d);
    \draw [] (d) to[out=305,in=90] (end);
    \draw [] (d) to[out=235,in=90] (e1);
  \end{tikzpicture}
  \ \raisebox{40pt}{\text{= }}
  \begin{tikzpicture}
    \path node at (-.5,3) (start) {}
    node at (-.25,1.5) [map] (f3) {$\scriptstyle f$}
    node at (0,0) (end) {};
    \draw [] (start) to[out=270,in=90] (f3);
    \draw [] (f3) to[out=270,in=90] (end);
  \end{tikzpicture}
  \ \raisebox{40pt}{$= f$.}
  \]
  Proof of Equation~\ref{eq:ffinv_commutes_gginv}:  $f\inv{f}g\inv{g} =$

  \[
  \begin{tikzpicture}
    \path node at (-.5,3) (start) {}
    node at (0,2.5) [eta] (eta1) {}
    node at (0,2) [delta] (d1) {}
    node at (1,2.5) [eta] (eta2) {}
    node at (1,2) [delta] (d2) {}
    node at (-.75,1.5) [map] (f1) {$\scriptstyle f$}
    node at (-.25,1.5) [map] (f2) {$\scriptstyle f$}
    node at (.25,1.5) [map] (g1) {$\scriptstyle g$}
    node at (.75,1.5) [map] (g2) {$\scriptstyle g$}
    node at (-.5,1) [nabla] (n1) {}
    node at (-.5,.5) [epsilon] (e1) {}
    node at (.5,1) [nabla] (n2) {}
    node at (.5,.5) [epsilon] (e2) {}
    node at (.75,0) (end) {};
    \draw [] (start) to[out=270,in=90] (f1);
    \draw [] (eta1) to (d1);
    \draw [] (eta2) to (d2);
    \draw [] (d1) to[out=235,in=90] (f2);
    \draw [] (d1) to[out=305,in=90] (g1);
    \draw [] (d2) to[out=235,in=90] (g2);
    \draw [] (d2) to[out=305,in=90] (end);
    \draw [] (f1) to [out=270,in=125] (n1);
    \draw [] (f2) to[out=270,in=55] (n1);
    \draw [] (g1) to[out=270,in=125] (n2);
    \draw [] (g2) to[out=270,in=55] (n2);
    \draw [-] (n1) to (e1);
    \draw [-] (n2) to (e2);
  \end{tikzpicture}
  \ \raisebox{40pt}{$=$ }
  \begin{tikzpicture}
    \path node at (-.5,3) (start) {}
    node at (0,2.5) [eta] (eta1) {}
    node at (0,2) [delta] (d1) {}
    node at (1,2.5) [eta] (eta2) {}
    node at (1,2) [delta] (d2) {}
    node at (-.5,1.5) [nabla] (n1) {}
    node at (.5,1.5) [nabla] (n2) {}
    node at (-.5,1) [map] (f1) {$\scriptstyle f$}
    node at (.5,1) [map] (g1) {$\scriptstyle g$}
    node at (-.5,.5) [epsilon] (e1) {}
    node at (.5,.5) [epsilon] (e2) {}
    node at (.75,0) (end) {};
    \draw [] (start) to[out=270,in=125] (n1);
    \draw [] (eta1) to (d1);
    \draw [] (eta2) to (d2);
    \draw [] (d1) to[out=235,in=55] (n1);
    \draw [] (d1) to[out=305,in=125] (n2);
    \draw [] (d2) to[out=235,in=55] (n2);
    \draw [] (d2) to[out=305,in=90] (end);
    \draw [-] (n1) to (f1);
    \draw [-] (n2) to (g1);
    \draw [] (f1) to (e1);
    \draw [] (g1) to (e2);
  \end{tikzpicture}
  \ \raisebox{40pt}{$=$ }
  \begin{tikzpicture}
    \path node at (-.25,3) (start) {}
    node at (0,2.5) [eta] (eta1) {}
    node at (-.25,2) [nabla] (n1) {}
    node at (-.25,1.5) [delta] (d1) {}
    node at (.5,1.5) [eta] (eta2) {}
    node at (-.5,1) [map] (f1) {$\scriptstyle f$}
    node at (.25,1) [nabla] (n2) {}
    node at (-.5,.5) [epsilon] (e1) {}
    node at (.25,.5) [delta] (d2) {}
    node at (0,0) [map] (g1) {$\scriptstyle g$}
    node at (0,-.5) [epsilon] (e2) {}
    node at (.25,-1) (end) {};
    \draw [] (start) to[out=270,in=125] (n1);
    \draw [] (eta1) to[out=270,in=55] (n1);
    \draw [] (n1) to (d1);
    \draw [] (eta2) to[out=270,in=55] (n2);
    \draw [] (d1) to[out=235,in=90] (f1);
    \draw [] (d1) to[out=305,in=125] (n2);
    \draw [] (f1) to (e1);
    \draw [] (n2) to (d2);
    \draw [] (d2) to[out=235,in=90] (g1);
    \draw [] (d2) to[out=305,in=90] (end);
    \draw [] (g1) to (e2);
  \end{tikzpicture}
  \ \raisebox{40pt}{$=$ }
  \begin{tikzpicture}
    \path node at (-.25,2.5) (start) {}
    node at (-.25,1.5) [delta] (d1) {}
    node at (-.5,.5) [map] (f1) {$\scriptstyle f$}
    node at (-.5,0) [epsilon] (e1) {}
    node at (.25,1) [delta] (d2) {}
    node at (0,.5) [map] (g1) {$\scriptstyle g$}
    node at (0,0) [epsilon] (e2) {}
    node at (.25,-.5) (end) {};
    \draw [] (start) to (d1);
    \draw [] (d1) to[out=235,in=90] (f1);
    \draw [] (d1) to[out=305,in=90] (d2);
    \draw [] (f1) to (e1);
    \draw [] (d2) to[out=235,in=90] (g1);
    \draw [] (d2) to[out=305,in=90] (end);
    \draw [] (g1) to (e2);
  \end{tikzpicture}
  \ \raisebox{40pt}{$=$ }
  \begin{tikzpicture}
    \path node at (-.25,2.5) (start) {}
    node at (.25,2) [delta] (d2) {}
    node at (-.25,1.5) [delta] (d1) {}
    node at (-.5,.5) [map] (f1) {$\scriptstyle f$}
    node at (-.5,0) [epsilon] (e1) {}
    node at (0,.5) [map] (g1) {$\scriptstyle g$}
    node at (0,0) [epsilon] (e2) {}
    node at (.25,-.5) (end) {};
    \draw [] (start) to[out=270,in=90] (d2);
    \draw [] (d2) to[out=235,in=90] (d1);
    \draw [] (d2) to[out=305,in=90] (end);
    \draw [] (d1) to[out=235,in=90] (f1);
    \draw [] (d1) to[out=305,in=90] (g1);
    \draw [] (f1) to (e1);
    \draw [] (g1) to (e2);
  \end{tikzpicture}
  \ \raisebox{40pt}{$=$ }
  \begin{tikzpicture}
    \path node at (-.25,2.5) (start) {}
    node at (.25,2) [delta] (d2) {}
    node at (-.25,1.5) [delta] (d1) {}
    node at (-.5,1) [map] (g1) {$\scriptstyle g$}
    node at (-.5,.5) [epsilon] (e2) {}
    node at (0,1) [map] (f1) {$\scriptstyle f$}
    node at (0,.5) [epsilon] (e1) {}
    node at (.25,0) (end) {};
    \draw [] (start) to[out=270,in=90] (d2);
    \draw [] (d2) to[out=235,in=90] (d1);
    \draw [] (d2) to[out=305,in=90] (end);
    \draw [] (d1) to[out=235,in=90] (g1);
    \draw [] (d1) to[out=305,in=90] (f1);
    \draw [] (f1) to (e1);
    \draw [] (g1) to (e2);
  \end{tikzpicture}
  \ \raisebox{40pt}{$= g\inv{g}f\inv{f}$ }
\]
where the last step is accomplished by reversing all the previous diagrammatic steps.
Hence, \CFrob is an inverse category.
\end{proof}

\begin{theorem}\label{thm:cfrob_is_a_discrete_inverse_category}
  When \X is a symmetric monoidal category, \CFrob is a discrete inverse category.
\end{theorem}
\begin{proof}
  Lemma~\ref{lem:cfrobx_is_an_inverse_category} shows \CFrob is an inverse category. We
  need to show the conditions of Definition~\ref{def:inverse_product} are met.

  First, we see that the tensor of $\X$ is a tensor in \CFrob. $A\*B$ is an object in \CFrob
  with $\Delta_{A\*B} = (\Delta_A\*\Delta_B)(1\*c_{\*}\*1)$,
  $\nabla_{A\*B} =  (1\*c_{\*}\*1)(\nabla_A\*\nabla_B)$,
  $\eta_{A\*B} = \Delta_I(\eta_A \* \eta_B)$, and
  $\epsilon_{A\*B} =  (\epsilon_A\*\epsilon_B)\nabla_I$.

  The map $\Delta : A \to A\*A$ is a map in \CFrob. To show it preserves $\Delta$, we need to
  show $\Delta_A \Delta_{A\*A} = \Delta_A (\Delta_A \* \Delta_A)$:
  \[
  \raisebox{20pt}{$\Delta_A \Delta_{A\*A} =$}
  \begin{tikzpicture}
    \path node at (0,1.5) (start) {}
    node at (0,1) [delta] (d0) {}
    node at (-.25,.5) [delta] (d1) {}
    node at (.25,.5) [delta] (d2) {}
    node at (-.35,0) (e1) {}
    node at (-.15,0) (e2) {}
    node at (.15,0) (e3) {}
    node at (.35,0) (e4) {};
    \draw [] (start) to[out=270,in=90] (d0);
    \draw [] (d0) to[out=235,in=90] (d1);
    \draw [] (d0) to[out=305,in=90] (d2);
    \draw [] (d1) to[out=235,in=90] (e1);
    \draw [] (d1) to[out=305,in=90] (e3);
    \draw [] (d2) to[out=235,in=90] (e2);
    \draw [] (d2) to[out=305,in=90] (e4);
  \end{tikzpicture}
  \raisebox{20pt}{$=$}
  \begin{tikzpicture}
    \path node at (0,2) (start) {}
    node at (-.25,1.5) [delta] (d0) {}
    node at (0,1) [delta] (d1) {}
    node at (.25,.5) [delta] (d2) {}
    node at (-.35,0) (e1) {}
    node at (-.15,0) (e2) {}
    node at (.15,0) (e3) {}
    node at (.35,0) (e4) {};
    \draw [] (start) to[out=270,in=90] (d0);
    \draw [] (d0) to[out=235,in=90] (e1);
    \draw [] (d0) to[out=305,in=90] (d1);
    \draw [] (d1) to[out=235,in=90] (e3);
    \draw [] (d1) to[out=305,in=90] (d2);
    \draw [] (d2) to[out=235,in=90] (e2);
    \draw [] (d2) to[out=305,in=90] (e4);
  \end{tikzpicture}
  \raisebox{20pt}{$=$}
  \begin{tikzpicture}
    \path node at (0,2) (start) {}
    node at (-.25,1.5) [delta] (d0) {}
    node at (.25,1) [delta] (d1) {}
    node at (0,.5) [delta] (d2) {}
    node at (-.35,0) (e1) {}
    node at (-.15,0) (e2) {}
    node at (.15,0) (e3) {}
    node at (.35,0) (e4) {};
    \draw [] (start) to[out=270,in=90] (d0);
    \draw [] (d0) to[out=235,in=90] (e1);
    \draw [] (d0) to[out=305,in=90] (d1);
    \draw [] (d1) to[out=235,in=90] (d2);
    \draw [] (d1) to[out=305,in=90] (e4);
    \draw [] (d2) to[out=235,in=90] (e3);
    \draw [] (d2) to[out=305,in=90] (e2);
  \end{tikzpicture}
  \raisebox{20pt}{$=$}
  \begin{tikzpicture}
    \path node at (0,2) (start) {}
    node at (-.25,1.5) [delta] (d0) {}
    node at (.25,1) [delta] (d1) {}
    node at (0,.5) [delta] (d2) {}
    node at (-.35,0) (e1) {}
    node at (-.15,0) (e2) {}
    node at (.15,0) (e3) {}
    node at (.35,0) (e4) {};
    \draw [] (start) to[out=270,in=90] (d0);
    \draw [] (d0) to[out=235,in=90] (e1);
    \draw [] (d0) to[out=305,in=90] (d1);
    \draw [] (d1) to[out=235,in=90] (d2);
    \draw [] (d1) to[out=305,in=90] (e4);
    \draw [] (d2) to[out=235,in=90] (e2);
    \draw [] (d2) to[out=305,in=90] (e3);
  \end{tikzpicture}
  \raisebox{20pt}{$=\Delta_A (\Delta_A \* \Delta_A).$}
  \]
  Note that in the last step, we simply reverse the various associativity steps used previously.

  To show that $\Delta$ preserves the $\nabla$, we must show that
  $(\Delta_A\*\Delta_A)\nabla_{A\*A} = \nabla_A \Delta_A$. Starting with $(\Delta_A\*\Delta_A)\nabla_{A\*A} =$
  \[
  \begin{tikzpicture}
    \path node at (0,1.5) (s1) {}
    node at (.5,1.5) (s2) {}
    node at (0,1) [delta] (d0) {}
    node at (.5,1) [delta] (d1) {}
    node at (0,.5) [nabla] (n0) {}
    node at (.5,.5) [nabla] (n1) {}
    node at (0,0) (e0) {}
    node at (.5,0) (e1) {};
    \draw [] (s1) to[out=270,in=90] (d0);
    \draw [] (s2) to[out=270,in=90] (d1);
    \draw [] (d0) to[out=235,in=125] (n0);
    \draw [] (d0) to[out=305,in=125] (n1);
    \draw [] (d1) to[out=235,in=55] (n0);
    \draw [] (d1) to[out=305,in=55] (n1);
    \draw [] (n0) to[out=270,in=90] (e0);
    \draw [] (n1) to[out=270,in=90] (e1);
  \end{tikzpicture}
  \raisebox{20pt}{$=$}
  \begin{tikzpicture}
    \path node at (0,2.5) (s1) {}
    node at (.5,2.5) (s2) {}
    node at (.5,2) [delta] (d1) {}
    node at (0,1.5) [nabla] (n0) {}
    node at (0,1) [delta] (d0) {}
    node at (0,.5) [nabla] (n1) {}
    node at (0,0) (e0) {}
    node at (.5,0) (e1) {};
    \draw [] (s1) to[out=270,in=125] (n0);
    \draw [] (s2) to[out=270,in=90] (d1);
    \draw [] (d1) to[out=235,in=55] (n0);
    \draw [] (d1) to[out=305,in=55] (n1);
    \draw [] (n0) to[out=270,in=90] (d0);
    \draw [] (d0) to[out=235,in=125] (n1);
    \draw [] (d0) to[out=305,in=90] (e1);
    \draw [] (n1) to[out=270,in=90] (e0);
  \end{tikzpicture}
  \raisebox{20pt}{$=$}
  \begin{tikzpicture}
    \path node at (0,2.5) (s1) {}
    node at (.25,2.5) (s2) {}
    node at (.25,2) [nabla] (n0) {}
    node at (.25,1.5) [delta] (d1) {}
    node at (0,1) [delta] (d0) {}
    node at (0,.5) [nabla] (n1) {}
    node at (0,0) (e0) {}
    node at (.5,0) (e1) {};
    \draw [] (s1) to[out=270,in=125] (n0);
    \draw [] (s2) to[out=270,in=55] (n0);
    \draw [] (n0) to[out=270,in=90] (d1);
    \draw [] (d1) to[out=235,in=90] (d0);
    \draw [] (d1) to[out=305,in=55] (n1);
    \draw [] (d0) to[out=235,in=125] (n1);
    \draw [] (d0) to[out=305,in=90] (e1);
    \draw [] (n1) to[out=270,in=90] (e0);
  \end{tikzpicture}
  \raisebox{20pt}{$=$}
  \begin{tikzpicture}
    \path node at (0,2.5) (s1) {}
    node at (.25,2.5) (s2) {}
    node at (.25,2) [nabla] (n0) {}
    node at (.25,1.5) [delta] (d1) {}
    node at (0,1) [delta] (d0) {}
    node at (0,.5) [nabla] (n1) {}
    node at (0,0) (e0) {}
    node at (.5,0) (e1) {};
    \draw [] (s1) to[out=270,in=125] (n0);
    \draw [] (s2) to[out=270,in=55] (n0);
    \draw [] (n0) to[out=270,in=90] (d1);
    \draw [] (d1) to[out=235,in=90] (d0);
    \draw [] (d1) to[out=305,in=90] (e1);
    \draw [] (d0) to[out=235,in=125] (n1);
    \draw [] (d0) to[out=305,in=55] (n1);
    \draw [] (n1) to[out=270,in=90] (e0);
  \end{tikzpicture}
  \raisebox{20pt}{$=$}
  \begin{tikzpicture}
    \path node at (0,2.5) (s1) {}
    node at (.5,2.5) (s2) {}
    node at (.25,2) [nabla] (n0) {}
    node at (.25,1.5) [delta] (d1) {}
    node at (0,1) (e0) {}
    node at (.5,1) (e1) {};
    \draw [] (s1) to[out=270,in=125] (n0);
    \draw [] (s2) to[out=270,in=55] (n0);
    \draw [] (n0) to[out=270,in=90] (d1);
    \draw [] (d1) to[out=235,in=90] (e0);
    \draw [] (d1) to[out=305,in=90] (e1);
  \end{tikzpicture}
  \raisebox{20pt}{$= \nabla_A \Delta_A$.}
  \]
  Note that the proof uses the ``special'' property in a non-trivial way.

  Thus, we have a $\Delta$ in \CFrob. As $\nabla = \inv{\Delta}$, the Frobenius requirement for
  the inverse product is immediately fulfilled. Commutativity, cocommutativity, associativity,
  coassociativity and the exchange rule all follow from the properties of the commutative Frobenius
  algebras and therefore \CFrob is a discrete inverse category.
\end{proof}

Note that the category \CFrob possesses additional structure over that of a general discrete inverse
category, specifically, the existence of unit maps $\eta:I \to A$ and $\epsilon:A\to I$ for each
object $A$.

We may also consider \CFrob as a bicategory\cite{leinster1998basic}, with the following data:
\begin{enumerate}[{(}i{)}]
  \item Objects of the bicategory are the objects of \CFrob, that is, the objects of the underlying
  symmetric monoidal category \X.
  \item The hom-sets $\CFrob(A,B)$ are each categories, with the objects being the maps between $A$
    and $B$ (elements of the hom-set) and the maps being the partial ordering given by the
    restriction as shown in Lemma~\ref{lem:restriction_cats_are_partial_order_enriched}.
  \item The composition functor is based upon the composition in \X. By
    Lemma~\ref{lem:restriction_cats_are_partial_order_enriched} we have that $f \le g$, $h\le k$
    gives $f h \le g k$. The identity functor $I_A:1\to \CFrob(A,A)$ maps to the identity map in
    each hom-set.
  \item The associativity and identity transforms are identities, i.e. $f(g h) = (f g) h$ and $f I_A
    = f = I_A f$, hence this is a 2-category.
\end{enumerate}
With this data, we see that the Frobenius structure of $(A,\nabla,\eta,\Delta,\epsilon)$ ensure that
\CFrob is actually a Cartesian bicategory as defined in Carboni and Walters
\cite{carboni1987cartesian}. Moreover, as we have $\nabla \Delta = (\Delta \*1) (1\* \nabla)$, this
satisfies the further condition that each object is \emph{discrete} and therefore is considered a
``bicategory of relations'' as defined in \cite{carboni1987cartesian}, Definition~2.1.
Carboni and Walters describe a number of consequences resulting when the base bicategory is locally
posetal.

Frobenius algebras are not the only structure of interest when considering symmetric monoidal
categories. Coecke, Paquette and Pavlovi\'c In  \cite{coecke08classical}, model quantum and
classical computations in a $\dagger$-symmetric monoidal category $\cD$. Their basic definitions for
include a compact structure, a quantum structure and a classical structure, that latter of which is
a special Frobenius algebra:

\begin{definition}\label{def:compact_structure}
  A \emph{compact structure} on an object $A$ in the category $\cD$ is given by the object $A$, an object
  $A^{*}$ called its \emph{dual} and the maps $\eta:I \to A^{*}\* A$, $\epsilon: A\* A^{*} \to I$
  such that the diagrams
  \[
    \xymatrix@C+20pt{
      A^{*} \ar[dr]^{id} \ar[d]_{\eta\*A^{*}} \\
      A^{*} \*A\*A^{*}  \ar[r]_(.6){A^{*} \*\epsilon} & A^{*}
    }
    \text{ and }
    \xymatrix@C+20pt{
      A \ar[r]^(.4){A\*\eta} \ar[dr]_{id} & A\* A^{*}\* A \ar[d]^{\epsilon\*A}\\
      & A
    }
  \]
  commute.
\end{definition}

\begin{definition}\label{def:quantumstructure}
  A \emph{quantum structure} is an object $A$ and map $\eta:I\to A\*A$ such that
  $(A,A,\eta,\dgr{\eta})$ form a compact structure.
\end{definition}
Note that $A$ is self-dual in definition \ref{def:quantumstructure}.


\begin{definition}\label{def:classicalstructure}
  A \emph{classical structure} in \cD{} is an object $X$ together with two maps, $\Delta :X \to X\* X$,
  $\epsilon:X\to I$ such that $(X,\dgr{\Delta},\dgr{\epsilon},\Delta,\epsilon)$ forms a special
  Frobenius algebra.
\end{definition}

Coecke, Paquette and Pavlovi\'c then examine two categories based on these structures: The category
$\cD_q$, the category whose objects are quantum structures in \cD, with
\[
  \cD_q((A,\eta_A),(B,\eta_B)) \definedas \cD(A,B)
\] and  $\cD_c$, the category whose maps are classical structures in
\cD, with
\[
  \cD_q((X,\Delta_X,\epsilon_X), (Y,\Delta_Y,\epsilon_Y)) \definedas \cD(A,B).
\]
The creation of these categories and the use of the classical structure motivated the examination of
\CFrob in this section.


\begin{remark}
  For an alternate way of using inverse categories to describe quantum computation, refer to
  Example~\ref{ex:partial-isometries-are-inverse-cats} and Hines and Braunstein's paper ``The
  structure of partial isometries'' \cite{hines2010structure}. This paper discusses how the category
  of partial isometries (an inverse category) may be considered a reasonable interpretation of
  von~Neumann-Birkhoff quantum logic\cite{birkhoff1936logic}. At the same time, they show this model
  in inconsistent with that of \cite{abramsky04:catsemquantprot} in that one can not model
  teleportation and in fact, is not even compact closed.
\end{remark}

% section the_category_of_commutative_frobenius_algebras (end)

%%% Local Variables:
%%% mode: latex
%%% TeX-master: "../../phd-thesis"
%%% End:
