\section{The category of Commutative Frobenius Algebras} % (fold)
\label{sec:the_category_of_commutative_frobenius_algebras}
\begin{example}[Commutative Frobenius algebras]\label{example:commfrob}
  Let \X be a symmetric monoidal category and form CFrob(\X) as follows: \paragraph{Objects:}
  Commutative Frobenius algebras\cite{kock04}: A quintuple $(A,\nabla,\eta,\Delta,\epsilon)$ where
  $A$ is a $k$-algebra for some field $k$, and $\nabla :A\*A \to A$, $\eta:k\to A$, $\Delta : A \to
  A\*A$, $\epsilon : A \to k$ are natural maps in the algebra. Additionally, these satisfy
  \[
    \xymatrix @C=40pt @R=25pt{
      A \* A \ar[dd]_{1\*\Delta} \ar[dr]^{\nabla}
        \ar[rr]^{\Delta \* 1} & &
        A \* (A \* A) \ar[dd]^{1 \* \nabla}\\
      & A \ar[dr]^{\Delta} & \\
      (A \* A) \* A \ar[rr]_{\nabla \* 1} & &
        A \* A
    }
  \]
  together with the additional property that $\Delta \nabla = 1$.

  \paragraph{Maps:} Multiplication ($\nabla$) and co-multiplication ($\Delta$) preserving
  homomorphisms which do not necessarily preserve the unit.
\end{example}

\begin{theorem}\label{thm:cfrob_is_a_discrete_inverse_category}
  When \X is a symmetric monoidal category, CFrob(\X) is a discrete inverse category.
\end{theorem}
\begin{proof}
  For $f:X \to Y$, define $\inv{f}$ as
  \[
    Y \xrightarrow{1\*\eta} Y\*X \xrightarrow{1\*\Delta}
      Y\*X\*X \xrightarrow{1\*f\*1} Y\*Y\*X \xrightarrow{\nabla\*1}
      Y\*X \xrightarrow{\epsilon\*1}X.
  \]
  As a string diagram, this looks like:
  \[
  \begin{tikzpicture}
    \path node at (-.5,3) (starty) {}
    node at (0,2.5) [eta] (eta1) {}
    node at (0,2) [delta] (d) {}
    node at (-.25,1.5) [map] (f) {$\scriptstyle f$}
    node at (-.5,1) [nabla] (n1) {}
    node at (-.5,.5) [epsilon] (e1) {}
    node at (0,0) (end) {};
    \draw [] (starty) to[out=270,in=125] (n1);
    \draw [] (eta1) to (d);
    \draw [] (d) to[out=305,in=90] (end);
    \draw [] (d) to[out=235,in=90] (f);
    \draw [] (f) to[out=270,in=55] (n1);
    \draw [-] (n1) to (e1);
  \end{tikzpicture}
  \ \raisebox{25pt}{\text{.}}
  \]
  Using a result from \cite{cockett2002:restcategories1}, we need only show:
  \begin{align}
    \inv{(\inv{f})} &= f\label{eq:finvinv_is_f}\\
    f\inv{f}f &= f\label{eq:ffinvf_is_f}\\
    f\inv{f}g\inv{g} &=g\inv{g} f\inv{f}.\label{eq:ffinv_commutes_gginv}
  \end{align}
  We also use the following two identities from \cite{kock04}:
  \begin{align}
    (1\*\eta)\nabla &= id\\
    \Delta(1\*\epsilon) &= id.
  \end{align}
  Diagrammatically, this is:
  \[
    \begin{tikzpicture}
    \path   node at (.5,1) (start) {}
    node at (0,1) [eta] (eta1) {}
    node at (.25,.5) [nabla] (n1) {}
    node at (.25,0) (end) {};
    \draw [] (eta1) to[out=270,in=125] (n1);
    \draw [] (start) to[out=270,in=55] (n1);
    \draw [] (n1)   to (end);
  \end{tikzpicture}
  \ \raisebox{15pt}{\text{= }}
  \begin{tikzpicture}
    \path node at (0,1) (starty) {}
    node at (0,0) (end) {};
    \draw [-] (starty) to (end);
  \end{tikzpicture}
  \ \raisebox{15pt}{\text{=}}
  \begin{tikzpicture}
    \path node at (0,1) (start) {}
    node at (0,.5) [delta] (d1) {}
    node at (-.25,0) (end) {}
    node at (.25,0) [epsilon] (e1) {};
    \draw [] (start) to (d1);
    \draw [] (d1) to[out=305,in=90] (e1);
    \draw [] (d1) to[out=235,in=90] (end);
  \end{tikzpicture}
  \ \raisebox{15pt}{.}
  \]
  Proof of Equation~\ref{eq:finvinv_is_f}:
  \[
  \begin{tikzpicture}
    \path node at (-.75,4) (starty) {}
    node at (0,3.5) [eta] (eta2) {}
    node at (0,3) [delta] (d2) {}
    node at (0,2.5) [eta] (eta1) {}
    node at (0,2) [delta] (d1) {}
    node at (-.25,1.5) [map] (f) {$\scriptstyle f$}
    node at (-.5,1) [nabla] (n1) {}
    node at (-.5,.5) [epsilon] (e1) {}
    node at (-.5,0) [nabla] (n2) {}
    node at (-.5,-.5) [epsilon] (e2) {}
    node at (.25,-1) (end) {};
    \draw [] (starty) to[out=270,in=125] (n2);
    \draw [] (eta2) to (d2);
    \draw [] (d2) to[out=235,in=125] (n1);
    \draw [] (d2) to[out=305,in=90] (end);
    \draw [] (eta1) to (d1);
    \draw [] (d1) to[out=305,in=55] (n2);
    \draw [] (d1) to[out=235,in=90] (f);
    \draw [] (f) to[out=270,in=55] (n1);
    \draw [-] (n1) to (e1);
    \draw [-] (n2) to (e2);
  \end{tikzpicture}
  \ \raisebox{50pt}{\text{=}}
  \begin{tikzpicture}
    \path node at (.25,4) (starty) {}
    node at (-.5,3.5) [eta] (eta2) {}
    node at (-.5,3) [delta] (d2) {}
    node at (0,2.5) [eta] (eta1) {}
    node at (0,2) [delta] (d1) {}
    node at (-.25,1.5) [map] (f) {$\scriptstyle f$}
    node at (-.5,1) [nabla] (n1) {}
    node at (-.5,.5) [epsilon] (e1) {}
    node at (0,0) [nabla] (n2) {}
    node at (0,-.5) [epsilon] (e2) {}
    node at (-.5,-1) (end) {};
    \draw [] (starty) to[out=270,in=55] (n2);
    \draw [] (eta2) to (d2);
    \draw [] (d2) to[out=305,in=125] (n1);
    \draw [] (d2) to[out=235,in=90] (end);
    \draw [] (eta1) to (d1);
    \draw [] (d1) to[out=305,in=125] (n2);
    \draw [] (d1) to[out=235,in=90] (f);
    \draw [] (f) to[out=270,in=55] (n1);
    \draw [-] (n1) to (e1);
    \draw [-] (n2) to (e2);
  \end{tikzpicture}
  \ \raisebox{50pt}{\text{= }}
  \begin{tikzpicture}
    \path node at (.25,4) (starty) {}
    node at (-.5,1) [eta] (eta2) {}
    node at (-.5,.5) [delta] (d2) {}
    node at (0,3.5) [eta] (eta1) {}
    node at (0,3) [delta] (d1) {}
    node at (-.25,1.5) [map] (f) {$\scriptstyle f$}
    node at (-.25,0) [nabla] (n1) {}
    node at (-.25,-.5) [epsilon] (e1) {}
    node at (.25,2.5) [nabla] (n2) {}
    node at (.25,2) [epsilon] (e2) {}
    node at (-.5,-1) (end) {};
    \draw [] (starty) to[out=270,in=55] (n2);
    \draw [] (eta2) to (d2);
    \draw [] (d2) to[out=305,in=125] (n1);
    \draw [] (d2) to[out=235,in=90] (end);
    \draw [] (eta1) to (d1);
    \draw [] (d1) to[out=305,in=125] (n2);
    \draw [] (d1) to[out=235,in=90] (f);
    \draw [] (f) to[out=270,in=55] (n1);
    \draw [-] (n1) to (e1);
    \draw [-] (n2) to (e2);
  \end{tikzpicture}
  \ \raisebox{50pt}{\text{= }}
  \begin{tikzpicture}
    \path node at (.5,4) (starty) {}
    node at (0,3.5) [eta] (eta1) {}
    node at (.25,3) [nabla] (n1) {}
    node at (.25,2.5) [delta] (d1) {}
    node at (.5,2) [epsilon] (e1) {}
    node at (0,1.5) [map] (f) {$\scriptstyle f$}
    node at (-.5,1) [eta] (eta2) {}
    node at (-.25,.5) [nabla] (n2) {}
    node at (-.25,0) [delta] (d2) {}
    node at (0,-.5) [epsilon] (e2) {}
    node at (-.5,-1) (end) {};
    \draw [] (starty) to[out=270,in=55] (n1);
    \draw [] (eta1) to[out=270,in=125] (n1);
    \draw [] (n1) to (d1);
    \draw [] (d1) to[out=305,in=90] (e1);
    \draw [] (d1) to[out=235,in=90] (f);
    \draw [] (f) to[out=270,in=55] (n2);
    \draw [] (eta2) to[out=270,in=125] (n2);
    \draw [-] (n2) to (d2);
    \draw [] (d2) to[out=305,in=125] (e2);
    \draw [] (d2) to[out=235,in=90] (end);
  \end{tikzpicture}
  \ \raisebox{50pt}{\text{=}}
  \begin{tikzpicture}
    \path node at (.5,4) (starty) {}
    node at (0,1.5) [map] (f) {$\scriptstyle f$}
    node at (-.5,-1) (end) {};
    \draw [] (starty) to[out=270,in=90] (f);
    \draw [] (f) to[out=270,in=90] (end);
  \end{tikzpicture}
  \ \raisebox{50pt}{\text{= }$f$.}
  \]
  Proof of Equation~\ref{eq:ffinvf_is_f}:
  \[
  \begin{tikzpicture}
    \path node at (-.5,3) (starty) {}
    node at (0,2.5) [eta] (eta1) {}
    node at (0,2) [delta] (d) {}
    node at (-.75,1.5) [map] (f1) {$\scriptstyle f$}
    node at (-.25,1.5) [map] (f2) {$\scriptstyle f$}
    node at (.25,1.5) [map] (f3) {$\scriptstyle f$}
    node at (-.5,1) [nabla] (n1) {}
    node at (-.5,.5) [epsilon] (e1) {}
    node at (0,0) (end) {};
    \draw [] (starty) to (f1);
    \draw [] (f1) to [out=270,in=125] (n1);
    \draw [] (eta1) to (d);
    \draw [] (d) to[out=305,in=90] (f3);
    \draw [] (f3) to[out=270,in=90] (end);
    \draw [] (d) to[out=235,in=90] (f2);
    \draw [] (f2) to[out=270,in=55] (n1);
    \draw [-] (n1) to (e1);
  \end{tikzpicture}
  \ \raisebox{45pt}{$=$ }
  \begin{tikzpicture}
    \path node at (-.5,3) (starty) {}
    node at (0,2.5) [eta] (eta1) {}
    node at (0,2) [delta] (d) {}
    node at (-.5,1.5) [nabla] (n1) {}
    node at (-.5,1) [map] (f2) {$\scriptstyle f$}
    node at (.25,1) [map] (f3) {$\scriptstyle f$}
    node at (-.5,.5) [epsilon] (e1) {}
    node at (0,0) (end) {};
    \draw [] (starty) to [out=270,in=125] (n1);
    \draw (n1) to (f2);
    \draw (f2) to (e1);
    \draw [] (eta1) to (d);
    \draw [] (d) to[out=305,in=90] (f3);
    \draw [] (f3) to[out=270,in=90] (end);
    \draw [] (d) to[out=235,in=55] (n1);
  \end{tikzpicture}
  \ \raisebox{45pt}{$=$ }
  \begin{tikzpicture}
    \path node at (-.5,3) (starty) {}
    node at (0,2.5) [eta] (eta1) {}
    node at (-.25,2) [nabla] (n1) {}
    node at (-.25,1.5) [delta] (d) {}
    node at (-.5,1) [map] (f2) {$\scriptstyle f$}
    node at (0,1) [map] (f3) {$\scriptstyle f$}
    node at (-.5,.5) [epsilon] (e1) {}
    node at (0,0) (end) {};
    \draw [] (starty) to[out=270,in=125] (n1);
    \draw [] (eta1) to[out=270,in=55] (n1);
    \draw [] (n1) to (d);
    \draw [] (d) to[out=305,in=90] (f3);
    \draw [] (d) to[out=235,in=90] (f2);
    \draw (f2) to (e1);
    \draw [] (f3) to[out=270,in=90] (end);
  \end{tikzpicture}
  \ \raisebox{45pt}{\text{= }}
  \begin{tikzpicture}
    \path node at (-.5,3) (starty) {}
    node at (-.25,1.5) [map] (f3) {$\scriptstyle f$}
    node at (-.25,1) [delta] (d) {}
    node at (-.5,.5) [epsilon] (e1) {}
    node at (0,0) (end) {};
    \draw [] (starty) to[out=270,in=90] (f3);
    \draw [] (f3) to (d);
    \draw [] (d) to[out=305,in=90] (end);
    \draw [] (d) to[out=235,in=90] (e1);
  \end{tikzpicture}
  \ \raisebox{45pt}{\text{= }}
  \begin{tikzpicture}
    \path node at (-.5,3) (starty) {}
    node at (-.25,1.5) [map] (f3) {$\scriptstyle f$}
    node at (0,0) (end) {};
    \draw [] (starty) to[out=270,in=90] (f3);
    \draw [] (f3) to[out=270,in=90] (end);
  \end{tikzpicture}
  \ \raisebox{45pt}{\text{= f.}}
  \]
  Proof of Equation~\ref{eq:ffinv_commutes_gginv}:  $f\inv{f}g\inv{g} =$

  \[
  \begin{tikzpicture}
    \path node at (-.5,3) (starty) {}
    node at (0,2.5) [eta] (eta1) {}
    node at (0,2) [delta] (d1) {}
    node at (1,2.5) [eta] (eta2) {}
    node at (1,2) [delta] (d2) {}
    node at (-.75,1.5) [map] (f1) {$\scriptstyle f$}
    node at (-.25,1.5) [map] (f2) {$\scriptstyle f$}
    node at (.25,1.5) [map] (g1) {$\scriptstyle g$}
    node at (.75,1.5) [map] (g2) {$\scriptstyle g$}
    node at (-.5,1) [nabla] (n1) {}
    node at (-.5,.5) [epsilon] (e1) {}
    node at (.5,1) [nabla] (n2) {}
    node at (.5,.5) [epsilon] (e2) {}
    node at (.75,0) (end) {};
    \draw [] (starty) to (f1);
    \draw [] (eta1) to (d1);
    \draw [] (eta2) to (d2);
    \draw [] (d1) to[out=235,in=90] (f2);
    \draw [] (d1) to[out=305,in=90] (g1);
    \draw [] (d2) to[out=235,in=90] (g2);
    \draw [] (d2) to[out=305,in=90] (end);
    \draw [] (f1) to [out=270,in=125] (n1);
    \draw [] (f2) to[out=270,in=55] (n1);
    \draw [] (g1) to[out=270,in=125] (n2);
    \draw [] (g2) to[out=270,in=55] (n2);
    \draw [-] (n1) to (e1);
    \draw [-] (n2) to (e2);
  \end{tikzpicture}
  \ \raisebox{45pt}{$=$ }
  \begin{tikzpicture}
    \path node at (-.5,3) (starty) {}
    node at (0,2.5) [eta] (eta1) {}
    node at (0,2) [delta] (d1) {}
    node at (1,2.5) [eta] (eta2) {}
    node at (1,2) [delta] (d2) {}
    node at (-.5,1.5) [nabla] (n1) {}
    node at (.5,1.5) [nabla] (n2) {}
    node at (-.5,1) [map] (f1) {$\scriptstyle f$}
    node at (.5,1) [map] (g1) {$\scriptstyle g$}
    node at (-.5,.5) [epsilon] (e1) {}
    node at (.5,.5) [epsilon] (e2) {}
    node at (.75,0) (end) {};
    \draw [] (starty) to[out=270,in=125] (n1);
    \draw [] (eta1) to (d1);
    \draw [] (eta2) to (d2);
    \draw [] (d1) to[out=235,in=55] (n1);
    \draw [] (d1) to[out=305,in=125] (n2);
    \draw [] (d2) to[out=235,in=55] (n2);
    \draw [] (d2) to[out=305,in=90] (end);
    \draw [-] (n1) to (f1);
    \draw [-] (n2) to (g1);
    \draw [] (f1) to (e1);
    \draw [] (g1) to (e2);
  \end{tikzpicture}
  \ \raisebox{45pt}{$=$ }
  \begin{tikzpicture}
    \path node at (-.25,3) (starty) {}
    node at (0,2.5) [eta] (eta1) {}
    node at (-.25,2) [nabla] (n1) {}
    node at (-.25,1.5) [delta] (d1) {}
    node at (.5,1.5) [eta] (eta2) {}
    node at (-.5,1) [map] (f1) {$\scriptstyle f$}
    node at (.25,1) [nabla] (n2) {}
    node at (-.5,.5) [epsilon] (e1) {}
    node at (.25,.5) [delta] (d2) {}
    node at (0,0) [map] (g1) {$\scriptstyle g$}
    node at (0,-.5) [epsilon] (e2) {}
    node at (.25,-1) (end) {};
    \draw [] (starty) to[out=270,in=125] (n1);
    \draw [] (eta1) to[out=270,in=55] (n1);
    \draw [] (n1) to (d1);
    \draw [] (eta2) to[out=270,in=55] (n2);
    \draw [] (d1) to[out=235,in=90] (f1);
    \draw [] (d1) to[out=305,in=125] (n2);
    \draw [] (f1) to (e1);
    \draw [] (n2) to (d2);
    \draw [] (d2) to[out=235,in=90] (g1);
    \draw [] (d2) to[out=305,in=90] (end);
    \draw [] (g1) to (e2);
  \end{tikzpicture}
  \ \raisebox{45pt}{$=$ }
  \begin{tikzpicture}
    \path node at (-.25,2.5) (starty) {}
    node at (-.25,1.5) [delta] (d1) {}
    node at (-.5,.5) [map] (f1) {$\scriptstyle f$}
    node at (-.5,0) [epsilon] (e1) {}
    node at (.25,1) [delta] (d2) {}
    node at (0,.5) [map] (g1) {$\scriptstyle g$}
    node at (0,0) [epsilon] (e2) {}
    node at (.25,-.5) (end) {};
    \draw [] (starty) to (d1);
    \draw [] (d1) to[out=235,in=90] (f1);
    \draw [] (d1) to[out=305,in=90] (d2);
    \draw [] (f1) to (e1);
    \draw [] (d2) to[out=235,in=90] (g1);
    \draw [] (d2) to[out=305,in=90] (end);
    \draw [] (g1) to (e2);
  \end{tikzpicture}
  \ \raisebox{45pt}{$=$ }
  \begin{tikzpicture}
    \path node at (-.25,2.5) (starty) {}
    node at (.25,2) [delta] (d2) {}
    node at (-.25,1.5) [delta] (d1) {}
    node at (-.5,.5) [map] (f1) {$\scriptstyle f$}
    node at (-.5,0) [epsilon] (e1) {}
    node at (0,.5) [map] (g1) {$\scriptstyle g$}
    node at (0,0) [epsilon] (e2) {}
    node at (.25,-.5) (end) {};
    \draw [] (starty) to[out=270,in=90] (d2);
    \draw [] (d2) to[out=235,in=90] (d1);
    \draw [] (d2) to[out=305,in=90] (end);
    \draw [] (d1) to[out=235,in=90] (f1);
    \draw [] (d1) to[out=305,in=90] (g1);
    \draw [] (f1) to (e1);
    \draw [] (g1) to (e2);
  \end{tikzpicture}
  \ \raisebox{45pt}{$=$ }
  \begin{tikzpicture}
    \path node at (-.25,2.5) (starty) {}
    node at (.25,2) [delta] (d2) {}
    node at (-.25,1.5) [delta] (d1) {}
    node at (-.5,1) [map] (g1) {$\scriptstyle g$}
    node at (-.5,.5) [epsilon] (e2) {}
    node at (0,1) [map] (f1) {$\scriptstyle f$}
    node at (0,.5) [epsilon] (e1) {}
    node at (.25,0) (end) {};
    \draw [] (starty) to[out=270,in=90] (d2);
    \draw [] (d2) to[out=235,in=90] (d1);
    \draw [] (d2) to[out=305,in=90] (end);
    \draw [] (d1) to[out=235,in=90] (g1);
    \draw [] (d1) to[out=305,in=90] (f1);
    \draw [] (f1) to (e1);
    \draw [] (g1) to (e2);
  \end{tikzpicture}
  \ \raisebox{45pt}{$= g\inv{g}f\inv{f}$ }
\]
where the last step is accomplished by reversing all the previous diagrammatic steps.

\end{proof}

% section the_category_of_commutative_frobenius_algebras (end)

%%% Local Variables:
%%% mode: latex
%%% TeX-master: "../../phd-thesis"
%%% End:
