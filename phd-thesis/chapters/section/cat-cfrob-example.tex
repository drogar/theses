\section{The category of Commutative Frobenius Algebras} % (fold)
\label{sec:the_category_of_commutative_frobenius_algebras}
\begin{example}[Commutative Frobenius algebras]\label{example:commfrob}
  Let \X be a symmetric monoidal category and form CFrob(\X) as follows: \paragraph{Objects:}
  Commutative Frobenius algebras\cite{kock04}: A quintuple $(A,\nabla,\eta,\Delta,\epsilon)$ where
  $A$ is a $k$-algebra for some field $k$, and $\nabla :A\*A \to A$, $\eta:k\to A$, $\Delta : A \to
  A\*A$, $\epsilon : A \to k$ are natural maps in the algebra. Additionally, these satisfy
  \[
    \xymatrix @C=40pt @R=25pt{
      A \* A \ar[dd]_{1\*\Delta} \ar[dr]^{\nabla}
        \ar[rr]^{\Delta \* 1} & &
        A \* (A \* A) \ar[dd]^{1 \* \nabla}\\
      & A \ar[dr]^{\Delta} & \\
      (A \* A) \* A \ar[rr]_{\nabla \* 1} & &
        A \* A
    }
  \]
  together with the additional property that $\Delta \nabla = 1$.

  \paragraph{Maps:} Multiplication ($\nabla$) and co-multiplication ($\Delta$) preserving
  homomorphisms which do not necessarily preserve the unit.
\end{example}

\begin{theorem}\label{thm:cfrob_is_a_discrete_inverse_category}
  When \X is a symmetric monoidal category, CFrob(\X) is a discrete inverse category.
\end{theorem}
\begin{proof}
  For $f:X \to Y$, define $\inv{f}$ as
  \[
    Y \xrightarrow{1\*\eta} Y\*X \xrightarrow{1\*\Delta}
      Y\*X\*X \xrightarrow{1\*f\*1} Y\*Y\*X \xrightarrow{\nabla\*1}
      Y\*X \xrightarrow{\epsilon\*1}X
  \]
  As a string diagram, this looks like:
  \[
  \begin{tikzpicture}
    \path node at (-1,3) (starty) {}
    node at (0,2.5) [eta] (eta1) {}
    node at (0,2) [delta] (d) {}
    node at (-.5,1.5) [shape=rectangle,draw] (f) {$\scriptstyle f$}
    node at (-1,1) [nabla] (n1) {}
    node at (-1,.5) [epsilon] (e1) {}
    node at (0,0) (end) {};
    \draw [] (starty) to (n1);
    \draw [] (eta1) to (d);
    \draw [] (d) to (end);
    \draw [] (d) to (f);
    \draw [] (f) to (n1);
    \draw [-] (n1) to (e1);
  \end{tikzpicture}
  \]
  Using a result from \cite{cockett2002:restcategories1}, we need only show:
  \begin{align*}
    \inv{(\inv{f})} &= f\\
    f\inv{f}f &= f\\
    f\inv{f}g\inv{g} &=g\inv{g} f\inv{f}
  \end{align*}
  We also use the following two identities from \cite{kock04}:
  \begin{align}
    (1\*\eta)\nabla &= id\\
    \Delta(1\*\epsilon) &= id.
  \end{align}

  \begin{align*}
    \inv{\inv{f}} &=(1\*\eta)(1\*\Delta)(1\*(\inv{f})\*1)(\nabla\*1)(\epsilon\*1) \\
    &=(1\*\eta)(1\*\Delta)(1\*((1\*\eta)(1\*\Delta)(1\*f\*1)(\nabla\*1)(\epsilon\*1))\*1)\\
    &\qquad\qquad(\nabla\*1)(\epsilon\*1) \\
    &=(1\*\eta)(1\*\Delta)(1\*1\*\eta)(1\*1\*f\*1\*1)(1\*\nabla\*1\*1)\\
    &\qquad\qquad(1\*\epsilon\*1\*1) (\nabla\*1)(\epsilon\*1)\\
    &=(\eta\*1)(\Delta\*1)(1\*\nabla)(f\*1)(((\eta)(\Delta\*1)(1\*\nabla)(1\*\epsilon))\*1)
      ((1\*\epsilon)\\
    &=(1\*\eta)\nabla \Delta(1\*\epsilon)f(\eta\*1)\nabla\Delta(1\*\epsilon)\\
    &=id_{x}id_{x}\  f \ id_{y} id_{y}\\
    &=f
  \end{align*}
  \begin{align*}
    f\inv{f}f &= f(1\*\eta)(1\*\Delta)(1\*f\*1)(\nabla\*1)(\epsilon\*1)f\\
    &=(1\*\eta)(1\*\Delta)(f\*f\*1)(\nabla\*1)(1\*f)(\epsilon\*1)\\
    &=(1\*\eta)(1\*\Delta)(\nabla\*1)(f\*f)(\epsilon\*1)\\
    &=(1\*\eta)\nabla\Delta(f\*f)(\epsilon\*1)\\
    &=\Delta(f\*f)(\epsilon\*1)\\
    &=f\Delta(\epsilon\*1)\\
    &=f
  \end{align*}
  Finally, to show $f\inv{f}$ and $g\inv{g} $ commute:
  \begin{align*}
    f(1\*\eta)&(1\*\Delta)(1\*f\*1)(\nabla\*1)(\epsilon\*1)g(1\*\eta)(1\*\Delta)(1\*g\*1)
      (\nabla\*1)(\epsilon\*1)\\
    &=(1\*\eta)(1\*\Delta)(\nabla\*1)(f\*1)(\epsilon\*1)(1\*\eta)(1\*\Delta)(\nabla\*1)(g\*1)
      (\epsilon\*1)\\
    &=(1\*\eta)\nabla\Delta(f\*1)(\epsilon\*1)(1\*\eta)\nabla\Delta(g\*1)(\epsilon\*1)\\
    &=\Delta(f\*1)(\epsilon\*1)\Delta(g\*1)(\epsilon\*1)\\
    &=\Delta(1\*\Delta)(f\*g\*1)(\epsilon\*\epsilon\*1)\\
    &=\Delta(1\*\Delta)(g\*f\*1)(\epsilon\*\epsilon\*1)\qquad\qquad\qquad\text{co-commutativity}\\
    &=g\inv{g}f\inv{f}
  \end{align*}

\end{proof}

% section the_category_of_commutative_frobenius_algebras (end)

%%% Local Variables:
%%% mode: latex
%%% TeX-master: "../phd-thesis"
%%% End:
