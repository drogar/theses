%!TEX root = /Users/gilesb/UofC/thesis/phd-thesis/phd-thesis.tex
\section{Dagger categories}\label{sec:daggercategories}
Dagger categories generalize the concepts of Hilbert spaces that are required to model quantum
computation. These were introduced in \cite{abramsky04:catsemquantprot} as \emph{strongly compact
closed categories}, an additional structure only on compact closed categories.


\subsection{Definitions}\label{sec:daggerdefinitions}

Although dagger categories were introduced in the context of compact closed categories, the concept
of a dagger is definable independently. This was first done in \cite{selinger05:dagger}.

\begin{definition}\label{def:daggercat}
  A dagger operator on a category $D$ is an involutive, identity on objects
  contravariant functor $\dagger:\cD\to \cD$. A dagger category
  is a category that has a dagger operator.
\end{definition}

Typically, the dagger is written as a superscript on the morphism. So, if $f:A\to B$ is a map in
\cD, then $\dgr{f}:B\to A$ is a map in \cD{} and is called the \emph{adjoint} of $f$. A map where
$f^{-1} = \dgr{f}$ is called \emph{unitary}. A map $f:A\to A$ with $f=\dgr{f}$ is called
\emph{self-adjoint} or \emph{Hermitian}.

\begin{definition}\label{def:daggersmc}
  A \emph{dagger symmetric monoidal category} is a symmetric monoidal category \cD{} with a dagger
  operator such that the dagger interacts coherently with the monoid to preserve the symmetric
  monoidal structure.
\end{definition}

The coherence requirements in definition \ref{def:daggersmc} are in addition to the standard
coherence diagrams for a symmetric monoidal category. The additional coherence requirements are:
\begin{itemize}
  \item For all maps $f:A\to B$ and $g:C\to D$, $\dgr{(f\*g)} = \dgr{f}\*\dgr{g}:B\*D \to A\* C$;
  \item The monoid structure isomorphisms $a_{A,B,C}:(A\*B)\to C$, $u_{A}:A\to I\*A$ and
    $c_{A,B}:A\*B \to B\*A$ are unitary.
\end{itemize}

\begin{definition}\label{def:daggercompact}
  A \emph{dagger compact closed category} \cD{} is a dagger symmetric monoidal category
  that is compact closed where the diagram
  \[
    \xymatrix @C+20pt @R+10pt{
      I \ar[r]^{\epsilon^{\dagger}_{A}} \ar[dr]_{\eta_{A}} &A\*A^{*}\ar[d]^{\sigma_{A,A^{*}}}\\
      &A^{*}\* A
    }
  \]
  commutes for all  objects $A$ in \cD.
\end{definition}

A category \cD{} is said to have \emph{finite biproducts} when it has a zero object $\mathbf{0}$
(an object that is both initial and terminal) and when each pair of objects $A,B$ have a biproduct
$A\+B$. In such a category the unique map $A\to \mathbf{0} \to B$ is designated as $\zeroob_{A,B}$.

Note that a category with finite biproducts is enriched in commutative monoids, where if $f,g:A\to
B$, define $f+g:A\to B$ as $\<id_{A}, id_{A}\>\, (f\+g)\, [id_{B},id_{B}]$. The unit for the
addition is $\zeroob_{A,B}$. In the future, $\<id, id\>$ will be designated by $\Delta$ and
$[id,id]$ will be designated by $\nabla$.

\begin{lemma}\label{lemma:daggerbiproducts}
If \cD{} is a dagger category with biproducts, with injections $in_{1},in_{2}$ and projections
$p_{1},p_{2}$, then the following are equivalent.
\begin{enumerate}
  \item $\dgr{p_{i}} = in_{i}, i=1,2$, \label{ldpdgrpisq}
  \item $\dgr{(f\+g)} = \dgr{f}\+\dgr{g}$ and $\dgr{\Delta} = \nabla$,\label{ldpddeltisnab}
  \item $\dgr{\<f,g\>} = [\dgr{f},\dgr{g}]$,\label{ldpdcopisprod}
  \item the below diagram commutes.\label{ldpcommute}
  \[
    \xymatrix @C+20pt @R+10pt{
      \dgr{A} \+ \dgr{B} \ar[d]_{id} \ar[dr]^{[\dgr{p_{1}},\dgr{p_{2}}]}\\
      A\+B\ar[r]_{id}&\dgr{(A\+B)}
    }
  \]
\end{enumerate}
\end{lemma}
\begin{proof}
  \begin{description}
    \item[\ref{ldpdgrpisq}$\implies$\ref{ldpddeltisnab}] To show $\dgr{\Delta} = \nabla$,
    draw the product cone for $\Delta$,
    \[
      \xymatrix {
        &A \ar[d]^{\Delta} \ar[dr]^{id} \ar[dl]_{id}\\
        A
         & A\+A \ar[l]^{p_{1}}  \ar[r]_{p_{2}}
         & A
      }
    \]
    and apply the dagger functor to it. As $\dgr{p_{i}} = in_{i}$, and $\dagger$ is identity on
    objects, this is now a coproduct diagram and therefore $\dgr{\Delta} = \nabla$.

    For $\dgr{(f\+g)} = \dgr{f}\+\dgr{g}$, start with the diagram defining $f\+g$ as a product of
    the arrows:
    \[
      \xymatrix{
        A\ar[d]_{f}  & A\+B \ar[l]_{p_{1}} \ar[r]^{p_{2}} \ar[d]^{f\*g}&A \ar[d]^{g}\\
        C & C\+D \ar[l]^{p_{1}} \ar[r]_{p_{2}}  & D.
      }
    \]
    Then, apply the dagger functor to this diagram. This is now the diagram defining the
    co-product of maps and therefore $\dgr{(f\+g)} = \dgr{f}\+\dgr{g}$.
    \item[\ref{ldpddeltisnab}$\implies$\ref{ldpdcopisprod}] The calculation showing this is
      \begin{eqnarray*}
        &[\dgr{f},\dgr{g}] & = \nabla; (\dgr{f}\+\dgr{g})\\
        & &=\dgr{\Delta}; (\dgr{f}\+\dgr{g})\\
        & &=\dgr{\Delta}; \dgr{(f\+g)}\\
        & & = \dgr{((f\+g);\Delta)}\\
        & & = \dgr{\<f,g\>}
      \end{eqnarray*}
    \item[\ref{ldpdcopisprod}$\implies$\ref{ldpcommute}]
      Under the assumption,
      \begin{eqnarray*}
        &[\dgr{p_{1}},\dgr{p_{2}}] &= \dgr{\<p_{1},p_{2}\>}\\
        &&=\dgr{id}\\
        &&=id
      \end{eqnarray*}
      and therefore the diagram commutes.

    \item[\ref{ldpcommute}$\implies$\ref{ldpdgrpisq}] Using the injections and under
    the assumption, the following diagram commutes:
      \[
        \xymatrix @C+20pt @R+10pt{
          \dgr{A} \+ \dgr{B} \ar[d]_{id} \ar[dr]^{[\dgr{p_{1}},\dgr{p_{2}}]}\ar[r]^{[in_{1},in_{2}]}
            & \dgr{A} \+ \dgr{B} \ar[d]^{id}\\
          A\+B\ar[r]_{id}&\dgr{(A\+B)}
        }
      \]
      and therefore, $\dgr{p_{1}} = in_{1}$ and $\dgr{p_{2}} = in_{2}$.
  \end{description}
\end{proof}

\begin{definition} \label{def:biproductdaggerccc}
  A \emph{biproduct dagger compact closed category} is a dagger compact closed category with
  biproducts where the conditions of lemma \ref{lemma:daggerbiproducts} hold.
\end{definition}
\subsection{Examples of dagger categories}

\paragraph{\fdh{}:} The category of finite dimensional Hilbert spaces is the motivating example for
the creation of the dagger and is, in fact, a biproduct dagger compact closed category. The
biproduct is the direct sum of Hilbert spaces and the tensor for compact closure is the standard
tensor of Hilbert spaces. The dual $H^{*}$ of a space $H$ is the space of all continuous linear
functions from $H$ to the base field. The dagger is defined via the adjoint as being the unique map
$\dgr{f}:B\to A$ such that $\<f a|b\> = \<a | \dgr{f} b\>$ for all $a\in A, b\in B$.

\paragraph{\rel{}:} The category \rel of sets and relations has the tensor $S\*T = S\times T$, the
Cartesian product and the biproduct $S\+T = S+T$, the disjoint union. This is compact closed under
$A^{*} = A$ and the dagger is the relational converse. That is, if the relation $R=\{(s,t)|s\in S,
t\in T\}:S\to T$, then $\dgr{R}=\{(t,s)|(s,t)\in R\}(=R^{*})$.

\paragraph{Inverse categories:}
An inverse category \X is also a dagger category when the dagger is defined as the partial inverse.
The unitary maps are the total maps which are isomorphisms. If the inverse category \X is also a
symmetric monoidal category where the monoid $\*$ is actually a restriction bi-functor, then \X is
a dagger symmetric monoidal category. This follows from
\[
  (f\*g) \inv{(f\*g)} = \rst{f\*g}=\rst{f} \*\rst{g} =
   f\inv{f} \* g \inv{g} = (f\*g) (\inv{f} \* \inv{g})
\]
but since the partial inverse of $f\*g$ is unique, $\inv{f\*g} = \inv{f} \* \inv{g}$. Finally,
since all the structure isomorphisms are total maps, they are unitary and \X is a dagger symmetric
monoidal restriction category.
