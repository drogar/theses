%!TEX root = /Users/gilesb/UofC/thesis/phd-thesis/phd-thesis.tex
\section{Semantics of quantum computation}% (fold)
\label{sec:semanticsquantum}
\subsection{Semantics of QPL}\label{sec:semanticsqpl}
\subsubsection{QPL Basics}\label{sec:qplbasics}
In \cite{selinger04:qpl} Dr. Selinger provides a denotational semantics for a quantum programming,
QPL, with the slogan of ``quantum data with classical control''. This slogan refers to the semantic
representation described in the paper, where explicit classical branching based on a classical
value is described by a \bit value with specific probabilities of being 0 or 1.

QPL is defined via a collection of functional flowchart components, where ``functional''
specifically means that each flowchart is a function from its inputs to its outputs. These
components describe the basic operations on \bits and \qbits. Edges between the components
represent the data (\bits and \qbits). These edges are labelled with a typing context and annotated
with a tuple of density matrices, describing the probability distribution of the classical data and
the state of the quantum data. In the case of purely classical data, this annotation will be a
tuple of probabilities, whereas in the case of purely quantum data, it will be a single density
matrix.

\begin{figure}[ht]
\[
  \xymatrix{
    *+<10pt>{\text{\bf Allocate bit}}
    \ar[d]^{\Gamma=A}\\
    *+[F-]{\text{new bit }b :=0}
    \ar[d]^{b:\text{\bf bit},\ \Gamma=(A,0)}\\
    {}
  }\hskip-1em
  \xymatrix{
    *+<10pt>{\text{\bf Assign bit}}
    \ar[d]^{b:\text{\bf bit},\ \Gamma=(A,B)}\\
    *+[F-]{b :=0}
    \ar[d]^{b:\text{\bf bit},\ \Gamma=(A+B,0)}\\
    {}
  }
  \xymatrix{
    *+<10pt>{\text{\bf }}
    \ar[d]^{b:\text{\bf bit},\ \Gamma=(A,B)}\\
    *+[F-]{b :=1}
    \ar[d]^{b:\text{\bf bit},\ \Gamma=(0,A+B)}\\
    {}
  }
\]
\[
  \xymatrix{
    *+<10pt>{\text{\bf Discard bit}}
    \ar[d]^{b:\text{\bf bit},\ \Gamma=(A,B)}\\
    *+[F-]{\text{discard }b}
    \ar[d]^{b:\text{\bf bit},\ \Gamma=A+B}\\
    {}
  }
  \xymatrix@!C@C=-2em@R+0ex{
     {\text{\bf Branch}} & {}
     \\
     &   *+{\text{\bf branch}~ b}
     \branchbox{a}{.35}{2ex}{2.5ex}
     \ar"a0";[dl]_<(0.2){\bf 0}_<>(.7){b:\text{\bf bit},\ \Gamma=(A,0)}
     \ar"a1";[dr]^<(0.2){\bf 1}^<>(.7){b:\text{\bf bit},\ \Gamma=(0,B)}
     \ar[u];"au"^<>(.5){b:\text{\bf bit},\ \Gamma=(A,B)}\\
     {}& &
  }
\]
\caption{Classical flowcharts}\label{fig:classicalflow}
\end{figure}


\begin{figure}[ht]
\[
  \xymatrix@!C@C=-2em@R+0ex{
    {} \ar[dr]_{\Gamma=A}   &*+<10pt>{\text{\bf Merge}} & {}  \ar[dl]^{\Gamma=B}
     \\
     & *{\circ} \ar[d]^<>(.5){\Gamma=A+B}
     \\
     &
   }
   \xymatrix{
    *+<10pt>{\text{\bf Initial}}\\
    *{\circ} \ar[d]^{\Gamma=0}\\
    {}
  }
  \xymatrix{
    *+<10pt>{\text{\bf Permute}} \ar[d]_{b_{i}:\text{bit},q_{j}:\text{qbit},\Gamma=A_{i}}\\
    *+[F-]{\text{permute} \phi} \ar[d]_{b_{\phi(i)}:\text{bit},q_{\phi(j)}:\text{qbit},\Gamma=A_{\phi(i)}}\\
    {}
  }
\]
\caption{General flowcharts}\label{fig:generalflow}
\end{figure}

\begin{figure}[ht]
\[
  \xymatrix{
    *+<10pt>{\text{\bf Allocate qubit}}
    \ar[d]^{\Gamma=A}\\
    *+[F-]{\text{new qubit }q :=0}
    \ar[d]^{q:\text{\bf qubit},\ \Gamma=\qsmatss{A}{0}{0}{0}}\\
    {}
  }\hskip-2em
  \xymatrix{
    *+<10pt>{\text{\bf Unitary Transform}}
    \ar[d]^{\vec{q}:\text{\bf qubit},\ \Gamma=A}\\
    *+[F-]{q *= S}
    \ar[d]^{\vec{q}:\text{\bf qubit},\ \Gamma=(S \* I) A (S \* I)^{*}}\\
    {}
  }
\]
\[
  \xymatrix@!C@C=-2em@R+0ex{
    {\text{\bf Discard qubit}}
    \ar[d]^{q:\text{\bf qubit},\ \Gamma=\qsmatss{A}{B}{C}{D}}\\
    *+[F-]{\text{discard } q}
    \ar[d]^{\Gamma=A+D}\\
    {}
  }
\]
\[
  \xymatrix@!C@C=-2em@R+0ex{
    {\text{\bf Measure}} & {}
    \\
    &   *+{\text{\bf measure}~ q}
    \branchbox{a}{.35}{2ex}{2.5ex}
    \ar"a0";[dl]_<(0.2){\bf 0}_<>{q:\text{\bf qubit},\ \Gamma=\qsmatss{A}{0}{0}{0}}
    \ar"a1";[dr]^<(0.2){\bf 1}^<>{q:\text{\bf qubit},\ \Gamma=\qsmatss{0}{0}{0}{D}}
    \ar[u];"au"^<>(.5){q:\text{\bf qubit},\ \Gamma=\qsmatss{A}{B}{C}{D}}\\
    {}& &
  }
\]
\caption{Quantum flowcharts}\label{fig:quantumflow}
\end{figure}

In figure \ref{fig:classicalflow}, the annotation $\Gamma$ consists of a tuple of probabilities,
with $n$ \bits requiring $2^{n}$ probabilities for their description. In figure
\ref{fig:quantumflow}, $\Gamma$ will consist of a density matrix of size $2^{m}\times 2^{m}$ for
$m$ \qubits. Note also that in figure \ref{fig:quantumflow}, the notation $\vec{q}$ indicates an
ordered set of \qubits.

In QPL, the classical operations consist of: \emph{Allocate bit, Assignment, Discard bit} and
\emph{Branch}. The quantum operations are:\emph{Allocate qubit, Unitary Transform, Discard qubit}
and \emph{Measure}. The operations applicable to both types of data are \emph{Merge, Initial} and
\emph{Permute}. These are found in Figure \ref{fig:generalflow}.

When components are combined, the type annotation $\Gamma$ consists of a tuple of density matrices.
Flowchart components must be combined so that they are connected via edges with identical typing
judgements. Flowcharts may have arbitrary numbers of input and output edges. By convention,
component flow is from the top down and programs are read is the same manner.

The semantics of a component is the function that calculates the matrix tuple(s) of the output
edges when given the matrix tuple of the input edges. Each of these functions is linear and
preserves adjoints. They also preserve positivity and the sum of the traces of the output edges
equals the sum of the traces of the input edges, which can be viewed as the probability of leaving
a fragment is the same as the probability of entering a fragment.

\subsubsection{Looping, subroutines and recursion}
In the flow chart representation of QPL, looping occurs when one edge is connected to a component
above the component originating the edge. Subroutines are represented by boxes with double left and
right lines. A subroutine may have multiple input and output edges and is considered shorthand for
the flowchart making up the subroutine. For example, see figure \ref{fig:procandloop}, where the
subroutine $Proc1$ accepts two \qubits $q,r$ as input and produces either two \qubits $c,d$ or a
single \qubit $q$. In the case when the output is the single \qubit $q$, the output is looped back
to be merged with the original input.

\begin{figure}
\[
  \xymatrix@!C@C=-4em@R-.7ex{
    *[.]{} \ar[dr]_<>(.5){q:\text{\bf qubit}}
    &&
    *[.]{}
    \\
    &*{\circ} \ar[d]^<>(.5){q:\text{\bf qubit}}
    \\
    &*+[F-]{\text{\bf new qubit}~r:={\bf 0}}\ar[d]^<>(.5){q,r:\text{\bf qubit}}
    \\
    &*+{\text{\bf input}~q,r}
    \\
    *+{\text{\bf output1}~c,d}
    \ar[d]^<>(.5){c,d:\text{\bf qubit}}
    &&
    *+{\text{\bf output2}~a}
    \procbox{[ll].[].[ul]}{Proc1}
    \ar`d[drr]`[r]`[uuuu]_<>(.5){a:\text{\bf qubit}}`^dl[uuul][uuul]
    &&
    *[.]{}
    \\
    *[.]{}
    &&
    *[.]{}
    &&
    *[.]{}
  }
\]
\caption{Example of a subroutine and loop}\label{fig:procandloop}
\end{figure}


\begin{figure}
\[
  \xymatrix@!C@C=1em@R-.7ex{
    *[.]{}
    &&\\
    *[.]{} \ar[dr]_<>(.5){n \text{ inputs}}
    &&
    \\
    &*+<10pt>[F-]{X}\ar[dl]_<>(.5){m \text{ outputs}}
     \ar`d[dr]`^u`[u]`^dl[][]
    &&
    *[.]{}
    \\
    *[.]{}
    &&
    *[.]{}
    &&
    *[.]{}
  } \hskip-6em
  \xymatrix@C=7pt@R=5pt@H+2pt{
    & \ar[dd]_{A} & & \\
    &  & & *{\circ}\ar[d] \\
    &\save [].[rr]!C="b1"*[F] \frm{}\restore \ar`d[dl]`[dddd]^{F_{11}(A)}[ldddd] & *+{X}
      & \ar[ddd]^{F_{21}(A)}\\
    & & &\\
      & *{\circ} \ar[d]& & \\
    &\save [].[rr]!C="b2"*[F] \frm{}\restore \ar`d[dl][dl] & *+{X} &\ar[ddd]^{F_{22}F_{21}(A)}\\
    *{\circ} \ar[ddd]_{F_{11}(A)+F_{12}F_{21}(A)}& & & \\
     & *{\circ} \ar[d]& & \\
    &\save [].[rr]!C="b2"*[F] \frm{}\restore   \ar`d[dl][dl] & *+{X}
      &\ar@{-}[dd]^{F_{22}F_{22}F_{21}(A)}\\
    *{\circ} \ar@{. }[d]& & &\\
    \ar[d]_{G(A)} & & & *{\vdots}\\
    & & &
  }
\]
\caption{Unwinding a loop}\label{fig:loopunwinding}
\end{figure}


Semantics of looping is based on ``infinite unwinding''. It is interesting to note this is similar
to the method used in \cite{kozen-semanticsprobabilistic} where a \texttt{while} program
construction is unwound to
\begin{quote}
  \texttt{if} $\neg B$ \texttt{then} $I$ \texttt{else} $S$; \texttt{while} $B$ \texttt{do} $S$
  \texttt{od fi};
\end{quote}
and the semantics of \texttt{while} is given as a fixpoint $W$ of the equation $W=e_{\neg B} +
(e_{B} ; S ; W)$. Referring to figure \ref{fig:loopunwinding}, the input to $X$ is $n+k$ density
matrices, the output is $m+k$ density matrices, where the $k$ matrices partake in the loop. This
can be written as $F(A,C) = (B,D)$ with $A=(A_{1},\ldots,A_{n})$, $B=(B_{1},\ldots,B_{m})$, where
$F$ is the linear function giving the semantics of $X$. This allows creation of four component
functions, where $F(A,0) = (F_{11}(A),F_{21}(A))$ and $F(0,C) = (F_{12}(C),F_{22}(C))$.

Following the right hand side of figure \ref{fig:loopunwinding}, the state of the edges at the end
are given by
\begin{equation}
  G(A) = F_{11}(A)+\sum_{i=0}^{\infty}F_{12}(F^{i}_{22}(F_{21}(A)))\label{eqn:loopunwinding}
\end{equation}
I will show later that this is a convergent sum.

The semantics of subroutines without recursion is the same as ``in-lining'' the subroutine at the
place of its call. The first requirement for this is that the program handle renaming of variables
as the formal parameters of the subroutine may have different names than the calling parameters.
For \bits, renaming $b$ to $c$ may be accomplished by the fragment
\begin{quote}
  \textbf{new bit $c:=0$;\\  branch $b$ 0>\{ \} 1>\{$c:=1$\} ;\\
   discard $b$}.
\end{quote}
For \qbits, renaming $q$ to $r$ is done by the fragment
\begin{quote}
  \textbf{new qubit $p :=0$; \\
  q,p *=CNOT; \\
  p,q *=CNOT; \\
  discard q}.
\end{quote}
The second requirement is that the program needs to be able to extend the semantic context. That
is, suppose that a subroutine $X$ is defined with input typing $\Gamma$, with semantic function
$F$. This means that starting with $\Gamma=A$, applying the subroutine $X$ gives us the typing and
context $\Gamma'=F(A)$. Inlining a subroutine requires that the addition of an arbitrary number of
\bits and \qbits to the context and be able to derive the semantics. But, since each of the
components of flow charts and looping are linear functions, this is a straightforward induction on
the structure of the subroutine. The proof for one of the cases is below.
\begin{lemma}[Context extension]\label{lemma:contextextension}
Given a subroutine $X$ in context $\Gamma=A$ with semantics $F$ (i.e., applying $X$ to $\Gamma=A$
gives $\Gamma'=F(A)$),
\begin{itemize}
  \item{} The result of $X$ in context $b:$bit, $\Gamma=(A,B)$ is $b:$bit, $\Gamma'=(F(A),F(B))$.
  \item{} The result of $X$ in context $q:$qubit, $\Gamma=\qsmat{A}{B}{C}{D}$ is\\ $q:$qubit, $\Gamma'=\qsmat{F(A)}{F(B)}{F(C)}{F(D)}$.
\end{itemize}
\end{lemma}
\begin{proof}
  \textbf{Case $X=$``allocate bit''}
  The semantics of allocate bit is $F(A) = (A,0)$, where the number of $0$ density matrices is the same as the number
  of density matrices in $A$. When the additional context is a \bit, the starting context is $x:$bit$,\Gamma=(A,B)$, where again the number and
  dimensions of density matrices in $A$ and $B$ agree. After applying $X$,
  $b:$bit,$x:$bit$\Gamma'=(A,B,0,0)$. Next, permute $x$ and $b$, to retain $x$ in the correct order and get
  $x:$bit,$b:$bit$\Gamma'=(A,0,B,0) = ((A,0),(B,0) = (F(A),F(B))$.

  When the additional context is a \qubit, the starting context is $x:$qubit$,\Gamma=\qsmat{A}{B}{C}{D}$. After allocation of a bit,
  \begin{eqnarray*}
    &b:\text{bit},x:\text{qubit},\Gamma&=(\qsmat{A}{B}{C}{D}, \qsmat{0}{0}{0}{0}) \\
    &&= \qsmat{(A,0)}{(B,0)}{(C,0)}{(D,0)}\\
    &&=\qsmat{F(A)}{F(B)}{F(C)}{F(D)}.
  \end{eqnarray*}
\end{proof}

As the semantics of each of the components of the flowchart language are linear functions, the
technique used in the example case in Lemma \ref{lemma:contextextension} is applicable for each of
these flowchart components. Furthermore, as the semantics are compositional, this extends to
looping.

For recursive subroutines, a variant of the infinite unwinding is used. A recursive subroutine is
one that calls itself in some way. If we have a subroutine $X$, let $X(Y)$ be the flowchart defined
as $X$ with the recursive call to itself replaced with a call to $Y$. As the semantics are
compositional, there is some function $\Theta$ such that give the semantics of $X(Y)$ from the
semantics of $Y$. Let $Y_{0}$ be a non-terminating program and define $Y_{i}$ by the equation
$Y_{i+1} = X(Y_{i})$. Denote the semantics of $Y_{i}$ by $F_{i}$. Then $F_{0} = 0$ and $F_{i+1} =
\Theta(F_{i})$. From this, define
\begin{equation}\label{eqn:semrecursion}
  X =\lim_{i\to\infty}F_{i}.
\end{equation}
The existence of this limit will be discussed below in section \ref{sec:catsemanticsofqpl}.

An important point to note here is that the semantics of an arbitrary $G$ may actually reduce the
trace, that is, there may be a non-zero chance the program will not terminate.

\subsubsection{Categorical semantics of QPL}\label{sec:catsemanticsofqpl}
While the exposition above referred to the functions described in the flowchart components as the
semantics of the program, this section will give a formal definition of the categorical semantics.

\begin{definition}[Signature]\label{def:signature}
  A \emph{signature} is a list of positive non-zero integers, $\sigma=n_{1},\ldots,n_{s}$ which is
  associated with the complex vector space $V_{\sigma}=\C^{n_{1}\times
  n_{1}}\times\cdots\times\C^{n_{s}\times n_{s}}$
\end{definition}
Designate the elements of $V_{\sigma}$ by tuples of matrices, $A=(A_{1},\ldots,A_{s})$. The trace
of $A$ will be the sum of the traces of the tuple matrices. A will be said to have a specific
property when all matrices in the tuple have that property, e.g., positive, hermitian.

From the above, now define the category $V$ with objects being signatures $\sigma$ and maps from
$\sigma$ to $\tau$ being any complex linear function from $V_{\sigma}$ to $V_{\tau}$. Define $\+$
by concatenation of signatures. Then, $\+$ is both a product and coproduct in $V$. The co-pair map
$[F,G]:\sigma\+\sigma' \to \tau$ is defined as $[F,G](A,B) = F(A)+F(B)$, while the pairing map
$\<F,G\>:\sigma \to \tau\+\tau'$ is defined as $\<F,G\>(A)=(FA,GA)$. Additionally, define the
tensor $\*$ on $\sigma=n_{1},\ldots,n_{s}$ and $\tau=m_{1},\ldots,m_{t}$ as \[\sigma\*\tau =
n_{1}m_{1},n_{1}m_{2},\ldots,n_{s},m_{t}.\] This tensor, together with the unit $I=1$ makes $V$ a
symmetric monoidal category. Note that it is also distributive with $\tau\*(\sigma \+\sigma') =
(\tau\*\sigma)\+(\tau\*\sigma')$.

As $V$ is equivalent to the category of finite dimensional vector spaces, we will need to restrict
the morphisms to those that can occur as programs in QPL. $V$ has too many morphisms, for instance
the signature $1,1$ (which will be designated as \bit) is isomorphic to the signature $2$ (which
will be designated as \qbit).

\begin{definition}[Superoperator]
  Given $F:V_{\sigma}\to V_{\tau}$, define:
  \begin{itemize}
    \item $F$ as \emph{positive} if $F(A)$ is positive for all positive $A$;
    \item $F$ as \emph{completely positive} if $id_{\rho}\*F:V_{\rho\*\sigma}\to V_{\rho\*\tau}$ is
      positive for all $\rho$;
    \item $F$ as a \emph{superoperator} if it is completely positive and $\text{tr }F(A)\le$~tr~$A$
      for all positive $A$.
  \end{itemize}
\end{definition}
The definition of a superoperator is trace \emph{non-increasing} rather than trace
\emph{preserving} due to the possibility of non-termination in programs.

Considering superoperators in the category $V$, there a number of properties that hold. It is
immediate to see that an identity map is a superoperator and that compositions of superoperators
are again superoperators. The canonical injections $i_{1}:\sigma\to \sigma\+\tau$ and
$i_{2}:\tau\to \sigma\+\tau$ are superoperators. The remain properties of interest are detailed in
the following lemma.
\begin{lemma}\label{lemma:superoperator}
In the category $V$, the following hold:
\begin{enumerate}
  \item  If $F:\sigma\to\tau$ and $G:\sigma'\to\tau$ are superoperators, so is
  $[F,G]:\sigma\+\sigma'\to\tau$.
  \item If $F:\sigma\to\sigma'$ and $G:\tau\to\tau'$ are superoperators, then so are $F\+G$ and
    $F\*G$.
  \item if $id_{\nu}\*F$ is positive for all one element signatures $\nu$, then $F$ is completely
    positive.
  \item Given $U$, a unitary $n\times n$ matrix, then $F:n \to n$ defined as $F(A) = UAU^{*}$ is a
    superoperator.
  \item If $T_{1},T_{2}$ are $n\times n$ matrices such that $T_{1}^{*}T_{1} + T_{2}^{*}T_{2} = I$,
    then $F:n\to n,n$ defined as $F(A) = (T_{1}AT_{1}^{*}, T_{2}AT_{2}^{*})$ is a superoperator.
\end{enumerate}
\end{lemma}
\begin{proof}
  For statement 1, as $tr\ (F(A)+F(B)) = (tr\ F(A))+(tr\ F(B)) \le tr\ A + tr\ B = tr\ (A,B)$, note
  that $[F,G]$ satisfies the trace condition. Secondly, because of distributivity $id_{\rho}\*[F,G]
  = [id_{\rho}\*F,id_{\rho}\*G]$ and the complete positivity follows. The first assertion of
  statement 2 follows in a similar manner. As $F\*G = (F\*id_{\tau}) (id_{\sigma'}\*G)$, note that
  each element of the composition is a superoperator, hence so is $F\*G$. In statement 3, note that
  any signature $\nu$ of length $n$ is equal to a coproduct of $n$ single element signatures,
  $\nu_{1}\+\cdots\+\nu_{n}$. Then using distributivity, $id_{\nu}\*F = (id_{\nu_{1}}\*F)\+\cdots\+
  (id_{\nu_{1}}\*F)$ which by assumption and statement 2 is positive. Hence, $F$ is completely
  positive. For statement 4, as $U$ is unitary, it is immediate that $F$ is positive and preserves
  the trace. Note also that $(id_{n} \* F)(A) = (I\*U)A(I\*U)^{*}$ where $I$ is the $n\times n$
  identity matrix. However, $I\*U$ is also a unitary matrix, therefore $(id_{n} \* F)$ is positive
  for all $n$ and by the previous point, $F$ is completely positive and therefore a superoperator.
  For the last statement, by construction $F$ preserves both positivity and trace and by a similar
  argument to the previous point, it is a superoperator.
\end{proof}

At this point there is now sufficient machinery to define the category $Q$ which will be used for
the categorical semantics for QPL. Define $Q$ as the subcategory of $V$ having the same objects,
but only superoperators as morphisms. By lemma \ref{lemma:superoperator}, this is a valid
subcategory, which inherits $\+$ as a coproduct. It is not a product as the diagonal morphism
increases the trace and is therefore not a superoperator.

$Q$ is also a CPO enriched category. First, note that superoperators send density matrices to
density matrices (positive hermitian matrices with trace $\le 1$). Designating $D_{\sigma}$ to be
the subset of density matrix tuples contained in $V_{\sigma}$, then for any superoperator $F$, it
can be restricted to the density matrices. This restricted function preserves the L\"owner order
from definition \ref{def:lownerorder} and it preserves the least upper bounds of sequences. From
this, given signatures $\sigma$ and $\tau$, define a partial order on $Q(\sigma,\tau)$ by $F\le G$
when $\forall \nu, A\in D_{\nu\*\sigma}: (id_{\nu} \* F)(A) \le (id_{\nu} \* G)(A)$.

\begin{lemma}\label{lemma:qiscpoenriched}
The poset $Q(\sigma,\tau)$ is a complete partial order. Composition, co-pairing and tensor are
Scott-continuous and therefore $Q$ is CPO-enriched.
\end{lemma}

As $Q$ is a CPO enriched category, it is possible to define a monoidal trace over the coproduct
monoid. Recall that if $F:\sigma \+ \tau \to \sigma' \+ \tau$, then $tr\ F$ is a map, $\sigma \to
\sigma'$. Given such an $F$, construct the trace as follows:
\begin{itemize}
  \item Define $T_{0}$ as the constant zero function.
  \item Define $T_{i+1} = F ; [id_{\sigma'},i_{2} H_{i}]: \sigma\+ \tau \to \sigma'$.
\end{itemize}
Then, $T_{0}\le T_{1}$ as 0 is the least element in the partial order. $T_{i}\le T_{i+1}$ for all
$i$ as all the categorical operations are monotonic due to the CPO enrichment. Therefore, now
define $T = \vee_{i} T_{i}:\sigma\+ \tau \to \sigma'$. Finally, define $tr\ F = i_{1}; T: \sigma
\to \sigma'$.

This trace construction may be compared to the loop semantics construction in section
\ref{sec:qplbasics}. For $F$ as above, we may decompose it into components
$F_{11}:\sigma\to\sigma'$, $F_{21}:\sigma\to\tau$, $F_{12}:\tau\to\sigma'$ and
$F_{22}:\tau\to\tau$. This gives us
\begin{eqnarray*}
  &T_{0}(A,0) &= 0,\\
  &T_{1}(A,0) &= F_{11}(A),\\
  &T_{2}(A,0) &= F_{11}(A) + F_{12}F_{21}(A),\\
  &&\vdots
\end{eqnarray*}
which brings us to
\begin{equation*}
  (Tr\ F)(A) = T(A,0) = F_{11}(A) +\sum_{i=0}^{\infty}F_{12}(F_{22}^{i}(F_{21}(A))).
\end{equation*}
This is the same construction as equation (\ref{eqn:loopunwinding}) and will be used for the
interpretation of loops. In particular, this justifies the convergence of the infinite sum in that
equation.

At this point, we now have the information required to give an interpretation of the quantum flow
charts of QPL in the category $Q$. There are two types, \interp{\bit{}}$= 1,1$ and
\interp{\qubit{}}$=2$. The interpretation of basic operations is given in Table \ref{tab:interp}.
\begin{table*}[ht]
  \caption{Interpretation of QPL operations}
  \begin{center}
    \begin{eqnarray*}
      \interp{\text{new bit }b:=0} &=newbit:I\to \bit: & a\mapsto (a,0)\\
      \interp{\text{discard }b} &=discardbit:\bit\to I: & (a,b)\mapsto a+b\\
      \interp{b:=0} &=set_{0}:\bit\to\bit:&(a,b)\mapsto(a+b,0)\\
      \interp{b:=1} &=set_{1}:\bit\to\bit:&(a,b)\mapsto(0,a+b)\\
      \interp{\text{branch }b}&=branch: \bit\to\bit\+\bit:&(a,b)\mapsto(a,0,0,b)\\
      \interp{\text{new qbit }q:=0} &=newqbit:I\to \qbit:
        & a\mapsto \begin{pmatrix}a&0\\0&0\end{pmatrix}\\
      \interp{\text{discard }q} &=discardqbit:\qbit\to I:
        & \begin{pmatrix}a&b\\c&d\end{pmatrix}\mapsto a+d\\
      \interp{\vec{q}*=U} &=unitary_{U}:\qbit^{n}\to\qbit^{n}:&A\mapsto UAU^{*}\\
      \interp{\text{measure }q}&=measure: \qbit\to\qbit\+\qbit :
        &\begin{pmatrix}a&b\\c&d\end{pmatrix}\mapsto(\begin{pmatrix}a&0\\0&0\end{pmatrix},
        \begin{pmatrix}0&0\\0&d\end{pmatrix})\\
      \interp{\text{merge}} &= merge:I\+I \to I : &(a,b)\mapsto (a+b)\\
      \interp{\text{initial}}&=initial: 0 \to I : &() \mapsto 0\\
      \interp{\text{permute }\Phi} &= permute_{\Phi}: \+_{i}A_{i} \to \+_{i}A_{\Phi(i)}
    \end{eqnarray*}
  \end{center}
  \label{tab:interp}
\end{table*}

Additionally, if a type context $\Gamma$ is $x_{i}:T_{i}$, then
$\interp{\Gamma}=\*_{i}\interp{A_{i}}$. If $\bar{\Theta}$ is a list of typing context,
$\Theta_{i}$, then $\interp{\bar{\Theta}} = \+_{i}\interp{\Theta_{i}}$. For the various types of
composite flowcharts, the interpretation is as follows:
\begin{itemize}
  \item If the context $\Gamma$ is added to flowchart $A$, producing $B$, then
    $\interp{B} = \interp{A}\*\interp{\Gamma}$.
  \item If the outputs of flowchart $A$ are connected to the inputs of $B$, giving flowchart $C$,
    then $\interp{C}=\interp{A} ; \interp{B}$.
  \item If flowchart $C$ is made up of parallel flowcharts $A$ and $B$, then
    $\interp{C} = \interp{A}\+\interp{B}$.
  \item if flowchart $C$ is a loop on flowchart $A$, then $\interp{C} = tr\,(\interp{X})$.
\end{itemize}

For procedures, it is necessary to consider abstract variable flowcharts with specified types,
$R_{i}:\overline{\Theta_{i}} \to \overline{\Theta'_{i}}$. With these variable flowcharts allowed,
these may be interpreted in a specified environment $\kappa$ which maps the $R_{i}$ to specific
morphisms of $Q$ with the appropriate type. Then, $\interp{R_{i}}_{\kappa} = \kappa{R_{i}}$ and if
$A$ is a flow chart using $R_{i}$, its interpretation relative to $\kappa$ may be built up
inductively via the operations above, giving a function $\Omega_{A}$ which will map the
environments to a specific map in $Q$.

For recursion, consider the recursive subroutine defined as $P=T(P)$ for a flowchart $T$. Then
$\Omega_{T}:Q(\sigma,\tau)\to Q(\sigma,\tau)$ will be a Scott-continuous function. In this case,
$\interp{Y}$ will be the least fixed point of $\Omega_{T}$. There is an increasing sequence for all
$i\ge0$, $S_{i} \le S_{i+1}$ given by $S_{0} = 0$ and $S_{i+1} = \Omega_{T}(S_{i})$. This gives the
interpretation of $P$ as \[\interp{P} = \vee_{i} S_{i} = \lim_{i} S_{i}.\] This corresponds to
equation (\ref{eqn:semrecursion}) above, which shows this is the correct interpretation for
recursive procedures and since $Q$ is a CPO enriched category with a least point, therefore this
limit exists and therefore the the limit in equation (\ref{eqn:semrecursion}) will exist.


\subsubsection{Conclusions for QPL}\label{sec:conclusionsqpl}
\paragraph{Data types}
As can be seen by the proceeding pages, creating the categorical machinery for a semantic
interpretation of QPL is quite detailed. The paper \cite{selinger04:qpl} goes on to prove
soundness, completeness and provides some alternative syntaxes for QPL. The subject of structured
types is discussed briefly. Tuple types $(\Gamma,\Lambda)$ may be constructed as
$\interp{\Gamma}\*\interp{\Lambda}$. This immediately provides types such as fixed length classical
or quantum integers, characters, and so forth. Sum types can similarly be added as $\Gamma \+
\Lambda$, noting that the ``choice'' between the two types remains classical. The one primary
weakness in the type system is not allowing structured recursive types such as \type{List}. This
weakness is addressed in a follow on paper, \cite{huet2007}. In this paper, the major change is
that rather than restricting to a tuple of integers for the objects of $V$ and $Q$, they consider
arbitrary families of integers as the objects, defining a new category $Q^{\infty}$. Definitions
such as positive, Hermitian, the L\"owner order and trace follow in a straightforward manner, as
does the definition $V_{\sigma} = \prod_{i\in|\sigma|}\C^{\sigma_{i}\times\sigma_{i}}$, noting this
is now an infinite product. Note that in the infinite dimensional case there is no canonical basis
for $V_{\sigma}$ and therefore no canonical isomorphism between $V_{\sigma\*\tau}$ and
$V_{\sigma}\*V_{\tau}$. To rectify this, the authors refine the allowed morphisms in the category
$Q^{\infty}$. First, the define the category $\overline{Q^{\infty}}$ as having infinite signatures
as objects, but maps $f:\sigma\to\tau$ are maps $f:D_{\sigma}\to D_{\tau}$ ($D_{\sigma}$ are the
density matrix tuples of $V_{\sigma}$). These $f$ are called \emph{positive operators}. They are
required to extend to linear maps $\overline{f}: V_{\sigma} \to V_{\tau}$ and be continuous for the
L\"owner order. These maps are definable as a matrix of maps over finite dimensional spaces
$f_{ij}:\C^{\sigma_{i}\times\sigma_{i}}\to\C^{\tau_{i}\times\tau_{i}}$, calling this the
\emph{operator matrix}.

One can now define the tensor of two positive operators $f,g$ by tensoring their respective
operator matrices. Then, following the finite case, the positive operator $f:\sigma\to\tau$ with
operator matrix $F$ is a superoperator, if when $ID_{\gamma}$ is the operator matrix for the
identity on the signature $\gamma$, $ID_{\gamma} \* F$ is a positive operator. The infinite case
superoperators follow the desired properties as in the finite case and the category $Q^{\infty}$ is
defined as the category with objects being infinite signatures and morphisms these superoperators.

From this, the authors show that any endofunctor definable via an ``arithmetic'' equation involving
the coproduct and tensor will give rise to a data type in the category. In particular, one can
define \qubit lists as \[QList = 1 \+ (\qbit \* QList)\] and trees of \qbits as \[QTree =
\qbit\+(QTree \* QTree).\]

\paragraph{Quantum communication}
QPL makes no attempts to handle communication or transmission of quantum data. This will not be
addressed in this thesis.

\paragraph{Higher order functions}
QPL is defined as a functional language. One of the expectations of modern functional languages is
that programs themselves are first class objects, that is, they may be operated on by the program.
Typical uses are partial evaluation and passing a subroutine of a specified type for use by another
subroutine. In quantum computation, the primary issue with this in how does one guarantee the
no-cloning, no-erasing rules with respect to quantum data. Work on a quantum lambda calculus,
\cite{tonder03:qcsemantics,valiron:thesis}, has attempted to address this, albeit primarily with
operational rather than denotational semantics. In, \cite{selinger04:towardssemantics}, the author
explores the use of cones rather than vector spaces to create a denotational semantics, but finds
that the candidates fail to provide the correct answer over the base types. We will not be
considering the higher-order issues further in this paper.

\subsection{Semantics of pure quantum computations}\label{sec:puresemantics}
In \cite{abramsky04:catsemquantprot}, the authors approach the creation of a categorical semantics
for quantum computation independently of a specific language. Rather, they use finitary quantum
mechanics as their reference point.

Finitary quantum mechanics consists of the following:
\begin{enumerate}
  \item The system's state space is represented by a finite dimensional Hilbert space $H$.
    \label{lis:qfm1}
  \item The basic type of the system is that of \qubit --- 2-dimensional Hilbert space --- with the
    computational basis $\{\kz, \ko\}$.\label{lis:qfm2}
  \item Compound systems are tensor products of the components. This is what enables
    \emph{entanglement} as the general form of the system $H\*J$ where $H$ and $J$ are Hilbert
    spaces is
    \[
      \sum_{i=1}^{n}\alpha_{i} (u_{i} \* v_{i})
    \]
    where $u_{i}$ is a basis element of $H$ and $v_{i}$ is a basis element of $J$.\label{lis:qfm3}
  \item The basic transforms are \emph{unitary transformations}. \label{lis:qfm4}
  \item The measurements performable are \emph{self-adjoint} (hermitian) operators - with two
    sub-steps:\label{lis:qfm5}
    \begin{enumerate}
      \item The actual act of measurement. (Preparation).\label{lis:qfm5a}
      \item The communication of the results of the measurement. (Observation).\label{lis:qfm5b}
    \end{enumerate}
\end{enumerate}
The above definition does allow for the possibility of mixed states, as described in section
\ref{sec:density}, but for the remainder of this section, it is assumed both steps of the
measurement are carried out, resulting in pure states only.

\cite{abramsky04:catsemquantprot} gives the interpretation of finitary quantum mechanics in the
context of a biproduct dagger compact closed category, \cD.
\begin{description}
  \item[\ref{lis:qfm1}.] An $n-$dimensional state space $S$ is an object of \cD,
    together with a unitary isomorphism $base_{A}:\+^{n}I\to A$.
  \item[\ref{lis:qfm2}.] A \qubit is a 2 dimensional state space $Q$ with the computational basis
    $base_{Q}:I\+I \to Q$.
  \item[\ref{lis:qfm3}.] Compound systems $A,B$ are described by $A\*B$ and
    $base_{A\*B} = \phi (base_{A}\*base_{B})$ where $\phi:\+^{nm}I \cong(\+^{n}I)\*(\+^{m}I)$ is
    the isomorphism obtained by repeated application of distributivity isomporphisms.
  \item[\ref{lis:qfm4}.] The basic transformations are unitary transformations, i.e., $f$, where
    $\dgr{f} = f^{-1}$.
  \item[\ref{lis:qfm5a}.] A preparation is a morphism $P:I \to A$ which has a corresponding unitary
    morphism $f_{P}:\+^{n}I\to\+^{n}I$ and
    \[
      \xymatrix{
        I \ar[r]^{P} \ar[d]_{i_{1}}& A\\
        \+^{n}I \ar[r]_{f_{P}} & \+^{n}I \ar[u]_{base_{A}}
      }
    \]
    commutes.
  \item[\ref{lis:qfm5b}.] An observation  is an isomorphism $O = \+^{n}O_{i}$ with components
    $O_{i}:A \to I$ which has an unitary automorphism $f_{O}:\+^{n}I\to\+^{n}I$ such that
    \[
      \xymatrix{
        A \ar[r]^{O_{i}} & I\\
        \+^{n}I \ar[r]_{f_{O}}  \ar[u]_{base_{A}} & \+^{n}I \ar[u]_{p_{i}}
      }
    \]
    commutes for all $i=1,\ldots,n$. The observational branches are the individual $O_{i}:A \to I$.
\end{description}
Additionally, the biproduct $\+$ represents distinct branches resulting from measurement.
Accordingly, any operation on a biproduct must be an explicit biproduct, that is $f:A\+B\to C\+D$
will be $f_{1}\+f_{2}$ with $f_{1}:A\to C$ and $f_{2}:B\to D$.

The authors go on to show how this interpretation is sufficient to model quantum teleportation,
logic gate teleportation and entanglement swapping.


\subsection{Complete positivity}\label{sec:completepositivity}
Given a $\dagger$-compact closed category, it is possible to construct its category of completely
positive maps.

\begin{definition}[Positive map]\label{def:positivemap}
  A map $f:A\to A$ in a dagger category is called \emph{positive} if there is an object $B$ and a
  map $g:A\to B$ with $f = g \dgr{g}$
\end{definition}

\begin{definition}[Trace]\label{def:tracecp}
  For $f:A\to A$ in a compact closed category, its \emph{trace} is defined as $tr\, f:I\to I =
  \eta_{A} ; c_{A^{*},A} ; (f\*A^{*}) ; \epsilon$.
\end{definition}

The following lemma gives some properties of positive maps:

\begin{lemma}\label{lemma:positivemaps}
  In any biproduct dagger compact closed category, the following hold:
  \begin{enumerate}
    \item{} $f$ positive $\implies$ $h f \dgr{h}$ is positive for all maps $h$.
    \item{} $id_{A}$ is positive.
    \item If $f:A\to A$ and $g:B\to B$ are positive, so are $f\*g$ and $f\+g$.
    \item $0_{A,A}$ is positive. If $f,g:A\to A$ is positive, so is $f+g$.
    \item $f$ positive $\implies$ $\dgr{f}=f$.
    \item $f$ positive $\implies$ $f^{*}$ and $tr\ f$ are positive.
    \item $f,g:A\to A$ positive $\implies$ $tr (g\,f)$ is positive.
  \end{enumerate}
\end{lemma}
\begin{proof}
  The first six items follow immediately from the definitions and how structure is preserved for
  $(\_)^{\dagger}$. For item 6, note that $g = h\, \dgr{h}$ and $tr(g\,f) = tr(\dgr{h}\,f\,h)$
  which is positive by points 1 and 5.%FIXME - why
\end{proof}

\begin{definition}\label{def:name}
  In a compact closed category, the \emph{name} of a map $f:A\to B$ is the map $\ulcorner f
  \urcorner:I \to A^{*} \* B$ defined as $\eta_{A}; (1\*f)$. This is also called the \emph{matrix} of
  $f$.
\end{definition}

In the case of a positive map $f$, $\ulcorner f \urcorner$ is referred to as a \emph{positive
matrix}.

\begin{definition}\label{def:completelypositive}
  In a dagger compact closed category, a map $f:A^{*}\*A \to B^{*}\* B$ is \emph{completely positive}
  if for all objects $C$ and all positive matrices $f: I \to C^{*} \* A^{*} \* A \* C$ the morphism
  $g ; (1\*f\*1):I \to C^{*} \* B^{*}\* B \* C$ is a positive matrix.
\end{definition}

This now allows us to define the CPM construction.

\begin{definition}\label{def:cpmconstruction}
  Given a dagger compact closed category $\cD$, define \specialcat{CPM(d)} as the category with the
  same objects as $\cD$, and a map $f:A\to B$ in \specialcat{CPM(d)} is a completely positive map
  $f:A^{*}\*A \to B^{*}\* B$ in \cD.
\end{definition}

\specialcat{CPM(d)} is also a dagger compact closed structure, inheriting its tensor from \cD.
There is a functor $F:\cD \to \specialcat{CPM(d)}$ defined as $F(A) = A$ on objects and $F(f)=
f_{*}\*f$ on maps. The image of the structure maps under $F$ are structure maps for
\specialcat{CPM(d)}. The dagger of a map $f$ is the same as its dagger in \cD.

\subsubsection{Biproduct completion}\label{sec:biproduct}
When the \specialcat{CPM} construction is applied to a biproduct dagger compact closed category, it
will not in general retain biproducts. However, it will be monoid enriched by lemma
\ref{lemma:positivemaps}. This allows us to create the biproduct completion.

The biproduct completion of a category \cD, which is enriched in commutative monoids is the
category $\cD^{\+}$ which has as objects finite sequences $\<A_{1},\ldots,A_{n}\>$ where $n\ge 0$.
The morphisms of $\cD^{\+}$ are matrices of the morphisms of \cD. Application and composition of
morphisms is via matrix multiplication. The functor $F(A) = \<A\>$, $F(f)=[f]$ is an embedding of
\cD{} in $\cD^{\+}$. If \cD{} is compact closed and the tensor is linear (i.e., interacts with the
enrichment in a linear fashion), then $\cD^{\+}$ is also compact closed.

Furthermore, if \cD{} is a dagger category and the dagger is linear, then $\cD^{\+}$ will be a
dagger category. The dagger of a map $(f_{i,j})$ in $\cD^{\+}$ is $(\dgr{(f_{j,i})})$.

This gives us the following theorem:

\begin{theorem}\label{theorem:biproductcompletion}
Given \cD, a biproduct dagger compact closed category, \cpm{d} is enriched in commutative monoids
as a dagger compact closed category. Therefore, it is possible to construct its biproduct
completion, \bcpm{d}.
\end{theorem}

Note that the canonical embedding from above, $F$, while it preserves the dagger compact closed
structure, it does \emph{not} preserve biproducts.

%%% Local Variables:
%%% mode: latex
%%% TeX-master: "../../phd-thesis"
%%% End:
