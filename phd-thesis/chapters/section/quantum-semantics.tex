%!TEX root = /Users/gilesb/UofC/thesis/phd-thesis/phd-thesis.tex
\section{Semantics of quantum computation}% (fold)
\label{sec:semanticsquantum}

\subsection{Semantics of pure quantum computations}\label{sec:puresemantics}
In \cite{abramsky04:catsemquantprot}, the authors approach the creation of a categorical semantics
for quantum computation independently of a specific language. Rather, they use finitary quantum
mechanics as their reference point.

Finitary quantum mechanics consists of the following:
\begin{enumerate}
  \item The system's state space is represented by a finite dimensional Hilbert space $H$.
    \label{lis:qfm1}
  \item The basic type of the system is that of \qubit --- 2-dimensional Hilbert space --- with the
    computational basis $\{\kz, \ko\}$.\label{lis:qfm2}
  \item Compound systems are tensor products of the components. This is what enables
    \emph{entanglement} as the general form of the system $H\*J$ where $H$ and $J$ are Hilbert
    spaces is
    \[
      \sum_{i=1}^{n}\alpha_{i} (u_{i} \* v_{i})
    \]
    where $u_{i}$ is a basis element of $H$ and $v_{i}$ is a basis element of $J$.\label{lis:qfm3}
  \item The basic transforms are \emph{unitary transformations}. \label{lis:qfm4}
  \item The measurements performable are \emph{self-adjoint} (Hermitian) operators - with two
    sub-steps:\label{lis:qfm5}
    \begin{enumerate}
      \item The actual act of measurement. (Preparation).\label{lis:qfm5a}
      \item The communication of the results of the measurement. (Observation).\label{lis:qfm5b}
    \end{enumerate}
\end{enumerate}
The above definition does allow for the possibility of mixed states, as described in section
\ref{sec:density}, but for the remainder of this section, it is assumed both steps of the
measurement are carried out, resulting in pure states only.

\cite{abramsky04:catsemquantprot} gives the interpretation of finitary quantum mechanics in the
context of a biproduct dagger compact closed category, \cD.
\begin{description}
  \item[\ref{lis:qfm1}.] An $n-$dimensional state space $S$ is an object of \cD,
    together with a unitary isomorphism $base_{A}:\+^{n}I\to A$.
  \item[\ref{lis:qfm2}.] A \qubit is a 2 dimensional state space $Q$ with the computational basis
    $base_{Q}:I\+I \to Q$.
  \item[\ref{lis:qfm3}.] Compound systems $A,B$ are described by $A\*B$ and
    $base_{A\*B} = \phi (base_{A}\*base_{B})$ where $\phi:\+^{n m}I \cong(\+^{n}I)\*(\+^{m}I)$ is
    the isomorphism obtained by repeated application of distributivity isomorphisms.
  \item[\ref{lis:qfm4}.] The basic transformations are unitary transformations, i.e., $f$, where
    $\dgr{f} = f^{-1}$.
  \item[\ref{lis:qfm5a}.] A preparation is a morphism $P:I \to A$ which has a corresponding unitary
    morphism $f_{P}:\+^{n}I\to\+^{n}I$ and
    \[
      \xymatrix{
        I \ar[r]^{P} \ar[d]_{i_{1}}& A\\
        \+^{n}I \ar[r]_{f_{P}} & \+^{n}I \ar[u]_{base_{A}}
      }
    \]
    commutes.
  \item[\ref{lis:qfm5b}.] An observation  is an isomorphism $O = \+^{n}O_{i}$ with components
    $O_{i}:A \to I$ which has an unitary automorphism $f_{O}:\+^{n}I\to\+^{n}I$ such that
    \[
      \xymatrix{
        A \ar[r]^{O_{i}} & I\\
        \+^{n}I \ar[r]_{f_{O}}  \ar[u]_{base_{A}} & \+^{n}I \ar[u]_{p_{i}}
      }
    \]
    commutes for all $i=1,\ldots,n$. The observational branches are the individual $O_{i}:A \to I$.
\end{description}
Additionally, the biproduct $\+$ represents distinct branches resulting from measurement.
Accordingly, any operation on a biproduct must be an explicit biproduct, that is $f:A\+B\to C\+D$
will be $f_{1}\+f_{2}$ with $f_{1}:A\to C$ and $f_{2}:B\to D$.

The authors go on to show how this interpretation is sufficient to model quantum teleportation,
logic gate teleportation and entanglement swapping.


\subsection{Complete positivity}\label{sec:completepositivity}
Given a $\dagger$-compact closed category, it is possible to construct its category of completely
positive maps.

\begin{definition}[Positive map]\label{def:positivemap}
  A map $f:A\to A$ in a dagger category is called \emph{positive} if there is an object $B$ and a
  map $g:A\to B$ with $f = g \dgr{g}$
\end{definition}

\begin{definition}[Trace]\label{def:tracecp}
  For $f:A\to A$ in a compact closed category, its \emph{trace} is defined as $tr\, f:I\to I =
  \eta_{A} ; c_{A^{*},A} ; (f\*A^{*}) ; \epsilon$.
\end{definition}

The following lemma gives some properties of positive maps:

\begin{lemma}\label{lemma:positivemaps}
  In any biproduct dagger compact closed category, the following hold:
  \begin{enumerate}
    \item{} $f$ positive $\implies$ $h f \dgr{h}$ is positive for all maps $h$.
    \item{} $id_{A}$ is positive.
    \item If $f:A\to A$ and $g:B\to B$ are positive, so are $f\*g$ and $f\+g$.
    \item $0_{A,A}$ is positive. If $f,g:A\to A$ is positive, so is $f+g$.
    \item $f$ positive $\implies$ $\dgr{f}=f$.
    \item $f$ positive $\implies$ $f^{*}$ and $tr\ f$ are positive.
    \item $f,g:A\to A$ positive $\implies$ $tr (g\,f)$ is positive.
  \end{enumerate}
\end{lemma}
\begin{proof}
  The first six items follow immediately from the definitions and how structure is preserved for
  $(\_)^{\dagger}$. For item 6, note that $g = h\, \dgr{h}$ and $tr(g\,f) = tr(\dgr{h}\,f\,h)$
  which is positive by points 1 and 5.%FIXME - why
\end{proof}

\begin{definition}\label{def:name}
  In a compact closed category, the \emph{name} of a map $f:A\to B$ is the map $\ulcorner f
  \urcorner:I \to A^{*} \* B$ defined as $\eta_{A}; (1\*f)$. This is also called the \emph{matrix} of
  $f$.
\end{definition}

In the case of a positive map $f$, $\ulcorner f \urcorner$ is referred to as a \emph{positive
matrix}.

\begin{definition}\label{def:completelypositive}
  In a dagger compact closed category, a map $f:A^{*}\*A \to B^{*}\* B$ is \emph{completely positive}
  if for all objects $C$ and all positive matrices $f: I \to C^{*} \* A^{*} \* A \* C$ the morphism
  $g ; (1\*f\*1):I \to C^{*} \* B^{*}\* B \* C$ is a positive matrix.
\end{definition}

This now allows us to define the CPM construction.

\begin{definition}\label{def:cpmconstruction}
  Given a dagger compact closed category $\cD$, define \specialcat{CPM(d)} as the category with the
  same objects as $\cD$, and a map $f:A\to B$ in \specialcat{CPM(d)} is a completely positive map
  $f:A^{*}\*A \to B^{*}\* B$ in \cD.
\end{definition}

\specialcat{CPM(d)} is also a dagger compact closed structure, inheriting its tensor from \cD.
There is a functor $F:\cD \to \specialcat{CPM(d)}$ defined as $F(A) = A$ on objects and $F(f)=
f_{*}\*f$ on maps. The image of the structure maps under $F$ are structure maps for
\specialcat{CPM(d)}. The dagger of a map $f$ is the same as its dagger in \cD.

\subsubsection{Biproduct completion}\label{sec:biproduct}
When the \specialcat{CPM} construction is applied to a biproduct dagger compact closed category, it
will not in general retain biproducts. However, it will be monoid enriched by lemma
\ref{lemma:positivemaps}. This allows us to create the biproduct completion.

The biproduct completion of a category \cD, which is enriched in commutative monoids is the
category $\cD^{\+}$ which has as objects finite sequences $\<A_{1},\ldots,A_{n}\>$ where $n\ge 0$.
The morphisms of $\cD^{\+}$ are matrices of the morphisms of \cD. Application and composition of
morphisms is via matrix multiplication. The functor $F(A) = \<A\>$, $F(f)=[f]$ is an embedding of
\cD{} in $\cD^{\+}$. If \cD{} is compact closed and the tensor is linear (i.e., interacts with the
enrichment in a linear fashion), then $\cD^{\+}$ is also compact closed.

Furthermore, if \cD{} is a dagger category and the dagger is linear, then $\cD^{\+}$ will be a
dagger category. The dagger of a map $(f_{i,j})$ in $\cD^{\+}$ is $(\dgr{(f_{j,i})})$.

This gives us the following theorem:

\begin{theorem}\label{theorem:biproductcompletion}
Given \cD, a biproduct dagger compact closed category, \cpm{d} is enriched in commutative monoids
as a dagger compact closed category. Therefore, it is possible to construct its biproduct
completion, \bcpm{d}.
\end{theorem}

Note that the canonical embedding from above, $F$, while it preserves the dagger compact closed
structure, it does \emph{not} preserve biproducts.

%%% Local Variables:
%%% mode: latex
%%% TeX-master: "../../phd-thesis"
%%% End:
