%!TEX root = /Users/gilesb/UofC/thesis/phd-thesis/phd-thesis.tex
\section{Coproducts in restriction categories} % (fold)
\label{sec:coproducts_in_restriction_categories}
\subsection{Coproducts} % (fold)
\label{sub:coproducts}

Restriction categories may also have coproducts and initial objects.
\begin{definition}
  In a restriction category \X, a coproduct is a \emph{restriction coproduct} when the embeddings
  $\cpa$ and $\cpb$ are total.
\end{definition}

\begin{lemma}\label{lem:restriction_coproduct_is_restriction_functor}
  The definition of restriction coproduct implies the following:
  \begin{enumerate}[{(}i{)}]
    \item $\restr{f+g} = \restr{f} + \restr{g}$ which means $+$ is a restriction functor.
    \item $\nabla:A+A\to A$ is total.
    \item $?:0 \to A$ is total, where $0$ is the initial object in the restriction category.
  \end{enumerate}
\end{lemma}
\begin{proof}
  \prepprooflist
  \begin{enumerate}[{(}i{)}]
    \item \textbf{$+$ is a restriction functor.}
      Consider the diagram:
      \[
        \xymatrix @C=30pt @R=35pt{
          A \ar[rr]_{f} \ar[dr]_{\cpa} && A' \ar[dr]^{\cpa'}\\
          &A+B \ar@{.>}[rr]^{f+g} & & A' + B'\\
          B \ar[rr]_{g} \ar[ur]^{\cpb} && B' \ar[ur]_{\cpb'}
        }
      \]
      In order to show $\restr{f+g} = \restr{f} + \restr{g}$, it suffices to show that
      $\cpa\restr{f+g} = \cpa(\restr{f} + \restr{g}) = \restr{f}\cpa$.
      \begin{align*}
        \cpa\restr{f+g} &= \restr{\cpa(f+g)}\cpa & \rfour\\
        &= \restr{f \cpa'} \cpa &\text{coproduct diagram}\\
        &= \restr{f \restr{\cpa'}} \cpa
          & \text{Lemma }\refitem{lem:restrictionvarious}{lemitem:rv_3}\\
        &= \restr{f} \cpa & \cpa'\text{ total}
      \end{align*}
    \item   \textbf{$\nabla:A+A\to A$ is total.}
      By the definition of $\nabla$ ($=\<1 | 1 \>$) and the co-product, the following diagram
      commutes,
      \[
        \xymatrix @C=30pt @R=35pt{
          &A + A \ar[d]_{\nabla}\\
          A \ar@{=}[r] \ar[ur]^{\cpa}&A \ar@{=}[r] & A \ar[ul]_{\cpb}
        }
      \]
      resulting in:
      \begin{align*}
        \cpa \restr{\nabla} & = \restr{\cpa \nabla} \cpa\\
        &= \restr{1} \cpa\\
        &= \cpa
      \end{align*}
      Similarly, $\cpb \restr{\nabla} = \cpb$, hence, the restriction of $\nabla$ is $1$ and
      therefore $\nabla$ is total.
    \item  \textbf{$?:0 \to A$ is total.}
      This follows from
      \[
        \xymatrix @C=30pt @R=35pt{
          0 \ar[dr]_{?} \ar[r]^{\cpb} &A + 0 \\
          & A \ar@{=}[u]
        }
      \]
      so $?$ can be defined as the total coproduct injection.
  \end{enumerate}



\end{proof}

Recall that when an object is both initial and terminal, it is referred to as a zero object and
denoted as $0$. This gives rise to the zero map $0_{A,B}: A \to 0 \to B$ between any two objects.

\begin{definition}\label{def:restriction_zero}
  Given a restriction category \X with a zero object, then $0$ is a \emph{restriction zero} when
  for each object $A$ in \X we have $\restr{0_{A,A}} = 0_{A,A}$.
\end{definition}

\begin{lemma}[Cockett-Lack]\label{lem:restriction_zero_equivalences}
  For a restriction category \X, the following are equivalent:
  \begin{enumerate}[{(}i{)}]
    \item \X has a restriction zero;
    \item \X has an initial object $0$ and terminal object $1$ and each initial map $z_A$ is a
      restriction monic;
    \item \X has a terminal object $1$ and each terminal map $t_A$ is a restriction retraction.
  \end{enumerate}
\end{lemma}
% subsection coproducts (end)




\subsection{Inverse categories with restriction coproducts} % (fold)
\label{sub:inverse_categories_with_restriction_coproducts}


\begin{proposition}
  An inverse category \X with restriction coproducts is a preorder.
\end{proposition}
\begin{proof}
  By Lemma \ref{lem:restriction_coproduct_is_restriction_functor}, we know $\nabla$ is total and
  therefore $\nabla\inv{\nabla} = 1$. From the coproduct diagrams, we have $\cpa\nabla =1$ and
  $\cpb\nabla = 1$. But this gives us $\inv{\nabla}\inv{\cpa} = \inv{\cpa\nabla} = 1$ and
  $\inv{\nabla}\inv{\cpb} =1$. Hence, $\inv{\nabla} = \cpa$ \emph{and} $\inv{\nabla} = \cpb$.

  This means for parallel maps $f,g:A \to B$, we have
  \[
    f = \cpa \<f|g\> = \inv{\nabla} \<f|g\> = \cpb\<f|g\> = g
  \]
  and therefore \X is a preorder.
\end{proof}
% subsection inverse_categories_with_restriction_coproducts (end)
% section coproducts_in_restriction_categories (end)

\section{Disjointness in an inverse category} % (fold)
\label{sec:disjointness_in_an_inverse_category}
In the following, we will add two related structures to an inverse category with a
restriction zero. This structure is meant to be evocative of the concept of \emph{join} in a
restriction category.


\subsection{Disjointness relations} % (fold)
\label{sub:disjointness_relations}
In this subsection, we will define a disjointness relationship between maps and
explore alternate characterizations of this relation on the restriction idempotents of objects.

\begin{definition}\label{def:disjointness_relation}
  In an inverse category \X with a restriction zero, the relation $\cdperp$ between two parallel maps
  $f, g:A \to B$ is called a \emph{disjointess relation} when it satisfies the following
  properties:
  \begin{align*}
    \axiom{Dis}{1}\quad &\text{ For all }f:A\to B,\ f\cdperp 0;\\
    \axiom{Dis}{2}\quad &f\cdperp g\text{ implies }\rst{f} g = 0; \\
    \axiom{Dis}{3}\quad &f\cdperp g,\ f' \le f,\ g' \le g\text{ implies }f' \cdperp g';\\
    \axiom{Dis}{4}\quad &f\cdperp g\text{ implies }g \cdperp f;\\
    \axiom{Dis}{5}\quad &f\cdperp g\text{ implies }h f \cdperp h g; \text{ (Stable)}\\
    \axiom{Dis}{6}\quad &f\cdperp g\text{ implies }\rst{f} \cdperp \rst{g}
      \text{ and }\rg{f}\cdperp\rg{g}.\\
    \axiom{Dis}{7}\quad &\rst{f}\cdperp \rst{g},\ \rg{h}\cdperp \rg{k}\text{ implies }f h
      \cdperp g k;
  \end{align*}
\end{definition}

\begin{lemma}\label{lem:disjointness_equivalent_axioms}
  In Definition~\ref{def:disjointness_relation}, provided we retain $\axiom{Dis}{1-5}$ we may
  replace $\axiom{Dis}{6}$ and $\axiom{Dis}{7}$ by:
  \begin{align*}
    \axiom{Dis}{6'}\quad &f\cdperp g\text{ if and only if }\rst{f} \cdperp \rst{g}
      \text{ and }\rg{f}\cdperp\rg{g}.
  \end{align*}
\end{lemma}
\begin{proof}
  Given $\axiom{Dis}{6}$ and $\axiom{Dis}{7}$, the \emph{only if} direction of $\axiom{Dis}{6'}$ is
  immediate. To show the \emph{if} direction, assume $\rst{f} \cdperp \rst{g}$ and
  $\rg{f}\cdperp\rg{g}$. This also means that $\rst{\rst{f}} \cdperp \rst{\rst{g}}$. Then, by
  \axiom{Dis}{7}, $\rst{f} f \cdperp \rst{g}g$ and therefore $f \cdperp g$.

  Conversely, assume we are given \axiom{Dis}{6'}. Then, \axiom{Dis}{6} follows immediately. To show
  \axiom{Dis}{7}, assume we have $\rst{f}\cdperp \rst{g},\ \rg{h}\cdperp \rg{k}$. As
  $\rst{f h} \le \rst{f}$ and $\rst{g k}\le\rst{g}$, by \axiom{Dis}{3}, we know that
  $\rst{f h} \cdperp \rst{g k}$. Similarly, $\wrg{f h} \le \rg{h}$ and $\wrg{g k} \le \rg{k}$,
  giving us $\wrg{f h}\cdperp \wrg{g k}$. Then, from \axiom{Dis}{6'} we may conclude
  $f h \cdperp g k$, showing \axiom{Dis}{7} holds.
\end{proof}
\begin{lemma}\label{lem:disjointness_various}
  In an inverse category \X with $\cdperp$ a disjointness relation:
  \begin{enumerate}[{(}i{)}]
    \item $f \cdperp g$ if and only if $\inv{f}\cdperp \inv{g}$; \label{lemitem:djv_inverses}
    \item $f \cdperp g$ implies $f h \cdperp g h$ (Universal);\label{lemitem:djv_universal}
    \item $f \cdperp g$ implies $f\rg{g} = 0$;
    \item if $m,n$ are monic, then $f m \cdperp g n$ implies $\rst{f} \perp \rst{g}$;
    \item if $m,n$ are monic, then $\inv{m} f \cdperp \inv{n}g$ implies $\rg{f} \perp \rg{g}$;
  \end{enumerate}
\end{lemma}
\begin{proof}
  \prepprooflist
  \begin{enumerate}[{(}i{)}]
    \item Assume $f \cdperp g$. Then we know that $\rst{f} \cdperp \rst{g}$ and
      $\rg{f}\cdperp\rg{g}$. But since $\rg{f} = \rst{\inv{f}}$ and $\rst{f} = \wrg{\inv{f}}$, this
      means $\rst{\inv{f}} \cdperp \rst{\inv{g}}$ and $\wrg{\inv{f}}\cdperp\wrg{\inv{g}}$ and again
      by the first item of this lemma, we have $\inv{f} \cdperp \inv{g}$. The converse follows with
      a similar argument.
    \item Assume $f \cdperp g$. By the previous item, we have $\inv{f}\cdperp\inv{g}$. By
      \axiom{Dis}{5}, $\inv{h}\inv{f}\cdperp\inv{h}\inv{g}$, giving us $\inv{(f h)} \cdperp
      \inv{(g h)}$. Again by the previous item, we now have $f h \cdperp g h$.
    \item Assume $f \cdperp g$. From item \ref{lemitem:djv_inverses} and reflexivity, we know that
      $\inv{g}
      \cdperp \inv{f}$ and therefore $\rst{\inv{g}}\inv{f} = \rg{g}\inv{f}= 0$. However, in an
      inverse category, $\inv{0} = 0$ and therefore $0 = \inv{(\rg{g}\inv{f})} = f \inv{\rg{g}} = f
      \rg{g}$.
    \item Assume we have $f m \cdperp g n$ where $m, n$ are monic. By \axiom{Dis}{6},
      $\rst{f m} \cdperp \rst{g n}$. By Lemma~\ref{lem:restrictionvarious}, $\rst{f m} = \rst{f
      \rst{m}} = \rst{f 1} = \rst{f}$ and therefore $\rst{f} \cdperp \rst{g}$.
    \item This is a corollary to the previous item. By assumption, we have $\inv{m} f \cdperp
      \inv{n}g$ and therefore $\inv{f}m \cdperp \inv{g} n$. By the previous item, this means
      $\rst{\inv{f}} \cdperp \rst{\inv{g}}$ and hence $\rg{f} \cdperp \rg{g}$.
  \end{enumerate}
\end{proof}

We may define the disjointness relation via its action in \open{a}.

\begin{definition}\label{def:disjointness_in_open_x}
  Given an inverse category \X, a relation $\ocdperpsub{A} \subseteq \open{A}^2$ for each $
  A \in \mathrm{ob}(\X)$, is an \emph{open disjointness} relation when for all $e, e' \in \open{A}$
  \begin{align*}
    \axiom{$\mathcal{O}$dis}{1}\quad &1 \ocdperpsub{A} 0; \\
    \axiom{$\mathcal{O}$dis}{2}\quad &e \ocdperpsub{A} e' \text{ implies }e' \ocdperpsub{A} e;\\
    \axiom{$\mathcal{O}$dis}{3}\quad &e \ocdperpsub{A} e' \text{ implies }e e' = 0;\\
    \axiom{$\mathcal{O}$dis}{4}\quad &e \ocdperpsub{A} e' \text{ implies }\rst{f e} \ocdperpsub{B}
      \rst{f e'}\text{ for all }f:B \to A;\\
    \axiom{$\mathcal{O}$dis}{5}\quad &e \ocdperpsub{A} e' \text{ implies }\wrg{e g} \ocdperpsub{C}
      \wrg{e' g}\text{ for all }g:A \to C; \\
    \axiom{$\mathcal{O}$dis}{6}\quad &e \ocdperpsub{A} e',\ e_1 \le e,\ e_1' \le e'
      \text{ implies }e_1 \ocdperpsub{A} e_1'.
  \end{align*}
\end{definition}

We will normally write $\ocdperp$ rather than $\ocdperpsub{A}$ where the object is either clear or
not germane to the point under discussion.

\begin{proposition}\label{prop:disjointness_is_open_disjointness}
  If $\cdperp$ is a disjointness relation in $\X$, it is an open disjointness
  relation on the restriction idempotents.
\end{proposition}
\begin{proof}
  \prepprooflist
  \setlist[enumerate,1]{leftmargin=1.5cm}
  \begin{enumerate}
    \item[\axiom{$\mathcal{O}$dis}{1}] This follows immediately from \axiom{Dis}{1} by taking $f = 1$.
    \item[\axiom{$\mathcal{O}$dis}{2}] Reflexivity follows directly from  \axiom{Dis}{4}.
    \item[\axiom{$\mathcal{O}$dis}{3}] By \axiom{Dis}{2}, $0 = \rst{e} e' = e e'$.
    \item[\axiom{$\mathcal{O}$dis}{4}] Given $e \cdperp e'$, we have $f e \cdperp f e'$ by \axiom{Dis}{5}. Then,
      by \axiom{Dis}{6} we may conclude $\rst{f e} \cdperp \rst{f e'}$.
    \item[\axiom{$\mathcal{O}$dis}{5}] This follows from the above item, using $\inv{g}$ for $f$. This means we
      have $\rst{\inv{g}e} \cdperp \rst{\inv{g}e'}$. But this gives us $\rst{\inv{(e g)}} \cdperp
      \rst{\inv{(e'g)}}$. Recalling from Lemma \ref{lem:inverse_categories_are_range_categories}
      that $\rg{k} = \rst{\inv{k}}$, we may conclude $\wrg{e g} \cdperp \wrg{e' g}$.
    \item[\axiom{$\mathcal{O}$dis}{6}] Assuming $e \cdperp e'$ and $e_1 \le e,\ e'_1 \le e'$, by \axiom{Dis}{3}, $e_1 \cdperp
      e'_1$.
  \end{enumerate}
  Therefore, $\cdperp$ acts as an open disjointness relation on $\open{A}^2$.

\end{proof}

\begin{definition}\label{def:extended_open_disjointness}
  If $\ocdperp$ is an open disjointness relation in $\X$, then we may define a relation
  $\perpab \in \X(A,B)^2$ by
  \[
    \infer{f\perpab g}{f,g:A\to B,\ \rst{f}\ocdperp\rst{g},\ \rg{f}\ocdperp\rg{g}}.
  \]
  We call $\perpab$ an \emph{extended disjointness relation}.
\end{definition}

\begin{proposition}\label{prop:extended_disjointness_is_a_disjointness_relation}
  If $\cdperp$ is an extended disjointess relation based on $\ocdperp$ in \X, then
  $\cdperp$ is a disjointness relation in \X.
\end{proposition}
\begin{proof}
  \prepprooflist
  \setlist[enumerate,1]{leftmargin=1.5cm}
  \begin{enumerate}
    \item[\axiom{Dis}{1}] We need to show $f \cdperp 0$ for any $f$. We know that $1 \ocdperp 0$ and therefore
      $\rst{f} \ocdperp 0$ and $\rg{f} \ocdperp 0$, as $\rst{f}\le 1$ and $\rg{f}\le 1$. This gives
      us $f \cdperp 0$.
    \item[\axiom{Dis}{2}] Assume $f \cdperp g$, i.e., $\rst{f}\ocdperp \rst{g}$. Then, $\rst{f}g = \rst{f}\rst{g}g
      = 0 g = 0$.
    \item[\axiom{Dis}{3}] We are given $f \cdperp g$, $f' \le f$ and $g' \le g$. By lemma
      \refitem{lem:ordering_of_restriction_and_range}{lemitem:ordering_3} $\rst{f'} \le \rst{f}$
      and $\rst{g'} \le \rst{g}$. Then, by \axiom{$\mathcal{O}$dis}{6}, as $\rst{f} \ocdperp \rst{g}$ we have
      $\rst{f'} \ocdperp \rst{g'}$. By
      \refitem{lem:ordering_of_restriction_and_range}{lemitem:ordering_4}, we have $\wrg{f'} \le
      \rg{f}$ and $\wrg{g'} \le \rg{g}$. Then, by \axiom{$\mathcal{O}$dis}{6}, as $\rg{f} \ocdperp \rg{g}$ we
      have $\wrg{f'} \ocdperp \wrg{g'}$. This means $f' \cdperp g'$.
    \item[\axiom{Dis}{4}] Reflexivity of $\cdperp$ follows immediately from the reflexivity of
      $\ocdperp$.
    \item[\axiom{Dis}{5}] Assume $f \cdperp g$, i.e., $\rst{f}\ocdperp \rst{g}$ and $\rg{f}
      \ocdperp \rg{g}$. Then we have $\rst{hf}\ocdperp \rst{hg}$ by \axiom{$\mathcal{O}$dis}{4}. By lemma
      \refitem{lem:ordering_of_restriction_and_range}{lemitem:ordering_2} we have $\wrg{h f} \le
      \rg{f}$ and $\wrg{h g} \le \rg{g}$. Therefore we have $\wrg{h f}\ocdperp \wrg{h g}$ by
      \axiom{$\mathcal{O}$dis}{6} and therefore $h f \cdperp h g$.
    \item[\axiom{Dis}{6}] This follows directly from definition
      \ref{def:extended_open_disjointness}.
    \item[\axiom{Dis}{7}] We assume $\rst{f} \cdperp \rst{g}$ and $\rg{h} \cdperp \rg{k}$. By
      definition \ref{def:extended_open_disjointness} we have $\rst{f} \ocdperp \rst{g}$ and
      $\rg{h} \ocdperp \rg{k}$. By lemma
      \refitem{lem:ordering_of_restriction_and_range}{lemitem:ordering_1}, we have $\rst{fh} \le
      \rst{f}$ and $\rst{g k} \le \rst{g}$. Therefore, $\rst{f h} \ocdperp \rst{g k}$ by
      \axiom{$\mathcal{O}$dis}{6}. By \refitem{lem:ordering_of_restriction_and_range}{lemitem:ordering_2},
      $\wrg{f h} \le \rg{h}$ and $\wrg{g k} \le \rg{k}$, giving us $\wrg{f h}\ocdperp \wrg{g k}$
      also by \axiom{$\mathcal{O}$dis}{6}. This means $f h \cdperp g k$.
  \end{enumerate}
\end{proof}
\begin{theorem}\label{thm:open_disjointness_is_disjointness}
   To give a disjointness relation $\perp$ on $\X$ is to give an open disjointess relation
   $\ocdperp$ on \X.
\end{theorem}
\begin{proof}
  Suppose we are given the disjointess relation $\perp$. By
  Proposition~\ref{prop:disjointness_is_open_disjointness}, this is an open disjointness relation
  on each of the sets of idempotents, $\open{A}$. We will label that relation $\ocdperp$.

  Use Definition~\ref{def:extended_open_disjointness} to create an extended
  disjointness relation based on $\ocdperp$, signify it by $\eocdperp$.
  By Proposition~\ref{prop:extended_disjointness_is_a_disjointness_relation}, $\eocdperp$ is
  a disjointness relation on $\X$.

  Assume $f\perp g$. Then we have $\rst{f}\ocdperp\rst{g}$ and $\rg{f}\ocdperp\rg{g}$ by
  \axiom{Dis}{6} and Proposition~\ref{prop:disjointness_is_open_disjointness}. Then, from
  Definition~\ref{def:extended_open_disjointness}, we have $f \eocdperp g$.

  Assume $f \eocdperp g$. Then we must have had $\rst{f}\ocdperp\rst{g}$ and $\rg{f}\ocdperp\rg{g}$
  by Definition~\ref{def:extended_open_disjointness} and therefore $\rst{f}\perp\rst{g}$ and
  $\rg{f}\perp\rg{g}$. Then, by Proposition~\ref{lem:disjointness_various}, we have $f \perp g$.


  Now, suppose we are given the open disjointness relation $\ocdperp$. Similar to above, we can
  construct the extended disjointess relation $\perp$ by
  Definition~\ref{def:extended_open_disjointness}. From the disjointness relation $\perp$, we have
  the open disjointess relation $\overline{\perp}$ by
  Lemma~\ref{prop:disjointness_is_open_disjointness}.

  Assume $e \ocdperp e'$. As this means both $\rst{e} \ocdperp \rst{e'}$ and
  $\rg{e} \ocdperp \wrg{e'}$, we have $e \perp e'$. By
  Proposition~\ref{prop:disjointness_is_open_disjointness} this means $e \overline{\perp} e'$.

  If we are given that $e \overline{\perp} e'$, then we know that $e \perp e'$ by
  Proposition~\ref{prop:disjointness_is_open_disjointness}. From
  Definition~\ref{def:extended_open_disjointness}, this requires that $\rst{e} \ocdperp \rst{e'}$
  and $\rg{e} \ocdperp \wrg{e'}$, but that just means $e \ocdperp e'$.
\end{proof}

Note that while we have worked with binary disjointness throughout this section, one may extend
the concept to lists of maps simply by considering disjointness pairwise. I.e., we have
$\cdperp [f_1,f_2,\ldots,f_n]$ if and only if $f_i \cdperp f_k$ whenever $i \ne j$.

Disjointness is additional structure on a restriction category, i.e., it is possible to have more
than one disjointness relation on the category.
\begin{example}\label{ex:disjointness_is_a_structure_not_a_property}
  Consider the restriction category \inj. Here, the objects are sets and maps are
  the partial injective set functions, where $\rst{f}=id_{|\dom(f)}$. The restriction zero is the
  empty map (i.e., $\dom(0) = \rng(0) = \emptyset$).

  We may define the disjointness relation $\perp$ by $f \perp g$ if and only if
  $\dom(f)\intersect\dom(g)=\emptyset$ and $\rng(f)\intersect\rng(g) = \emptyset$. It is reasonably
  straightforward to verify \axiom{Dis}{1} through \axiom{Dis}{7}. For example, take \axiom{Dis}{7}:
  \begin{proof}
    We are given $\rst{f}\cdperp\rst{g}$ and $\rg{h}\cdperp\rg{k}$. This means
    \[
      \dom f\intersect\dom g=\emptyset \text{ and }\rng h\intersect\rng k = \emptyset.
    \]
    Note that in    general for partial injective functions $m$ and $n$ we have
    $\dom{m n} \subseteq \dom{m}$ and that $\rng{m n} \subseteq \rng{n}$. Hence we have
    \begin{align*}
      \dom{f h}\intersect\dom{g k}&\subseteq \dom{f}\intersect \dom{g}= \emptyset\\
      \rng{f h}\intersect\rng{g k}&\subseteq \rng{h}\intersect \rng{k}= \emptyset.
    \end{align*}
    Therefore, $f h \cdperp g k$.
  \end{proof}

  We may define a different disjointness relation, $\perp'$, on the same restriction category.
  Define $f \perp' g$ if and only if one of $f$ or $g$ is the restriction 0, $\emptyset$. As
  $0 = \rst{0} = \rg{0} = h 0 = 0 k$, all of the seven disjointness axioms are easily verifiable.
\end{example}

Although disjointness is additional structure on a restriction category, one can use the
disjointness structure of a base category (or categories) to define a disjointness structure
on derived categories, such as the product category.

\begin{lemma}\label{lem:disjointness_is_derivable_on_a_product_category}
  If $\X$ and $\Y$ are inverse categories with restriction zeros and respective disjointness
  relations $\perp$ and $\perp'$, then we may construct a disjointness relation
  $\perp_{\times}$ on $\X\times\Y$.
\end{lemma}
\begin{proof}
  Recall that product categories are defined component-wise. These definitions extend to the
  restriction, the inverse and the restriction zero. That is:
  \begin{itemize}
    \item If $(f,g)$ is a map in $\X\times\Y$, then $\inv{(f,g)} = (\inv{f}, \inv{g})$;
    \item If $(f,g)$ is a map in $\X\times\Y$, then $\rst{(f,g)} = (\rst{f}, \rst{g})$;
    \item The map $(0_X,0_Y)$ is the restriction zero in $\X\times\Y$.
  \end{itemize}

  Following this pattern, for $(f,g)$ and $(h,k)$ maps in $\X\times\Y$, $(f,g) \perp_{\times}(h,k)$
  iff $f\perp h$ and $g\perp' k$.

  Verifying the disjointness axioms is straightforward, we show axioms 2 and 5. Proofs of the
  others are similar.
  {
  \setlist[itemize,1]{leftmargin=1.5cm}
  \begin{itemize}
    \item [\axiom{Dis}{2}]: Given $(f,g)\perp_{\times}(h,k)$, we have
      $\rst{(f,g)}(h,k) =  (\rst{f},\rst{g})(h,k) = (\rst{f} h, \rst{g} k) = (0, 0) = 0$.
    \item [\axiom{Dis}{5}]: We are given $(f,g)\perp_{\times}(h,k)$. Consider the map $z = (x,y)$
      in $\X \times \Y$. We know that $x f \perp x h$ and $y g \perp y k$, therefore we have
      $z(f,g) = (x f, y g) \perp_{\times} (x h, y k) = z(h,k)$.
  \end{itemize}
  }
\end{proof}
% subsection disjointness_relations (end)
\subsection{Disjoint joins} % (fold)
\label{sub:disjoint_joins}

We now consider additional structure on the inverse category, dependant upon the disjointness
relation.
\begin{definition}\label{def:disjoint_join}
  An \emph{inverse category with disjoint joins} is an inverse category $\X$, with a restriction
  $0$, a disjointness relation $\perp$ and a binary operator on disjointness parallel maps:
  \[
    \infer{f \djoin g : A \to B}{f: A\to B,\ g: A\to B,\ f \perp g}
  \]
  where the following hold:
  \begin{align*}
    \axiom{DJ}{1}\quad & f \le f \djoin g\text{ and }g \le f \djoin g; \\
    \axiom{DJ}{2}\quad & f \le h,\ g \le h\text{ and }f\perp g\text{ implies }f \djoin g \le h;\\
    \axiom{DJ}{3}\quad & h(f \djoin g) = h f \djoin h g. \text{ (Stable)} \\
    \axiom{DJ}{4}\quad & \cdperp [f, g, h]\text{ if and only if }f \perp (g\djoin h).\\
  \end{align*}
  The binary operator, $\djoin$, is referred to as the \emph{disjoint join}.
\end{definition}

Note that \axiom{DJ}{1} with \axiom{DJ}{2} immediately gives us that there is only one disjoint
join given a specific disjointness relation.
\begin{lemma}\label{lem:disjoint_join_is_unique}
  Suppose \X in an inverse category with disjoint joins, with the join $\djoin$ and that it
  has a second disjoint join, $\altjoin$. Then $f \djoin g = f \altjoin g$ for all maps $f,g$ in
  \X.
\end{lemma}
\begin{proof}
  The first axiom tells us:
  \[
    f,g \le f\djoin g\text{ and }f,g \le f \altjoin g.
  \]
  Using the second axiom, we may therefore conclude $f \djoin g \le f \altjoin g$ and
  $f \altjoin g \le f\djoin g$, hence $f \djoin g = f \altjoin g$.
\end{proof}
\begin{lemma}\label{lem:join_is_associative_and_commutative_monoid}
  In an inverse category with disjoint joins, the disjoint join respects the restriction and is
  universal. Additionally, it is a partial associative and
  commutative operation, with identity $0$. That is, the following hold:
  \begin{enumerate}[{(}i{)}]
    \item $\rst{f\djoin g} = \rst{f} \djoin \rst{g}$;
    \item $(f \djoin g)k = f k \djoin g k$ (Universal);
    \item $f \perp g,\ g\perp h,\ f\perp h$ implies that $(f\djoin g)\djoin h = f\djoin(g\djoin
      h)$; \label{lemitem:associative_join}
    \item $f \perp g$ implies $f \djoin g = g \djoin f$; \label{lemitem:commutative_join}
    \item $f \djoin 0 = f$. \label{lemitem:identity_for_join}
  \end{enumerate}
\end{lemma}
\begin{proof}
  \prepprooflist
  \begin{enumerate}[{(}i{)}]
    \item As $\rst{f}, \rst{g} \le \rst{f\djoin g}$, we immediately have
      $\rst{f} \djoin \rst{g} \le \rst{f \djoin g}$. To show the other direction, consider
      \begin{align*}
        \rst{f}(\rst{f} \djoin \rst{g}) (f \djoin g)
        &= (\rst{f}\,\rst{f} \djoin \rst{f}\rst{g})(f \djoin g) &\text{\axiom{DJ}{3}}\\
        &= \rst{f} (f\djoin g) &\text{Lemma~\ref{lem:restrictionvarious}, \axiom{Dis}{2}}\\
        &= f.
      \end{align*}
      Hence, we have $f \le (\rst{f} \djoin \rst{g}) (f \djoin g)$ and similarly, so is $g$. By
      \axiom{DJ}{2} and that $\rst{f} \djoin \rst{g}$ is a restriction idempotent, we then have
      \[
        f \djoin g \le (\rst{f} \djoin \rst{g}) (f \djoin g) \le f \djoin g
      \]
      and therefore $f \djoin g = (\rst{f} \djoin \rst{g}) (f \djoin g)$. By
      Lemma~\ref{lem:restriction_cats_are_partial_order_enriched}, $\rst{f\djoin g} \le \rst{f}
      \djoin \rst{g}$ and so $\rst{f\djoin g} = \rst{f}\djoin \rst{g}$.
    \item First consider when $f, g$ and $k$ are restriction idempotents, say $e_0, e_1 $ and $e_2$.
      Then, we have $(e_0 \djoin e_1)e_2 = e_2(e_0 \djoin e_1) = e_2 e_0 \djoin e_2 e_1 =
      e_0 e_2 \djoin e_1 e_2$.
      Next, note that for general $f,g,h$, we have $f k \djoin g k \le (f\djoin g) k$ as both
      $fk, gk \le (f\djoin g) k$.
      By Lemma~\ref{lem:restriction_cats_are_partial_order_enriched}, we need only show that their
      restrictions are equal:
      \begin{align*}
        \rst{(f\djoin g)k} &= \rst{\rst{f\djoin g}(f\djoin g) k } & \rone \\
        & = \rst{f\djoin g}\, \rst{(f\djoin g)k } & \rthree\\
        & = (\rst{f}\djoin \rst{g}) \rst{(f\djoin g)k }  & \text{previous item}\\
        & = \rst{f}\,\rst{(f\djoin g)k } \djoin \rst{g}\, \rst{(f\djoin g)k }
          &\text{idempotent universal}\\
        & = \rst{\rst{f} (f\djoin g)k} \djoin \rst{\rst{g} (f\djoin g)k} & \rthree\\
        & = \rst{f k} \djoin \rst{g k}\\
        & = \rst{f k \djoin g k}.
      \end{align*}
      Therefore, as the restrictions are equal, we have shown $(f\djoin g)k = f k \djoin g k$.
    \item \emph{Associativity}: Note that \axiom{DJ}{4} shows that both sides of the equation
      exist. To show they are equal, we show that they are less than or equal to each other.

      From the definitions, we know that $f\djoin g, h \le (f\djoin g)\djoin h$, which also means
      $f, g \le (f\djoin g)\djoin h$. Similarly, $g\djoin h \le (f\djoin g)\djoin h $ and then $f
      \djoin (g\djoin h)\le (f\djoin g)\djoin h$. Conversely, $f,g,h \le f \djoin (g\djoin h)$ and
      therefore $(f\djoin g)\djoin h \le f \djoin (g\djoin h)$ and both sides are equal.
    \item \emph{Commutativity}: Note first that both $f$ and $g$ are less than or equal to both
      $f\djoin g$ and $g \djoin f$, by \axiom{DJ}{1}. By \axiom{DJ}{2}, we have $f \djoin g \le
      g\djoin f$ and $g\djoin f \le f \djoin g$ and we may conclude $f \djoin g = g \djoin f$.
    \item \emph{Identity}: By \axiom{DJ}{1}, $f \le f \djoin 0$. As $0 \le f$ and $f \le f$, by
      \axiom{DJ}{2}, $f \djoin 0 \le f$ and we have $f = f \djoin 0$.
  \end{enumerate}
\end{proof}

Note that the previous lemma and proof of associativity allows a simple inductive argument which
shows that having binary disjoint joins extends unambiguously to disjoint joins of an arbitrary
finite collection of disjoint maps.

We will write $[f_i]$ to signify a list of maps, where each $f_i: A \to B$. For disjointness,
$\cdperp [f_i]$ will mean that $f_j \cdperp f_k$ where $j\ne k$ and $f_j, f_k \in [f_i]$. Finally,
$\djoin [f_i]$ will mean the disjoint join of all maps $f_i$, i.e.,
$f_1 \djoin f_2 \djoin \cdots \djoin f_n$.

\begin{lemma}\label{lem:arbitrary_disjoint_joins}
  In an inverse category with disjoint joins, $\cdperp [f_i]$ if and only if
  $\djoin [f_i]$ is defined unambiguously.
\end{lemma}
\begin{proof}
  Using \axiom{Dj}{4}, proceed as in the proof of
  Lemma~\refitem{lem:join_is_associative_and_commutative_monoid}{lemitem:associative_join},
  inducting on $n$.
\end{proof}


\begin{lemma}\label{lem:disjoint_joins_only_when_all_disjoint}
  Given $\X$ is an inverse category with a disjoint join, then if $f_i, g_j : A \to B$ and $\cdperp
  [f_i]$ and $\cdperp [g_j]$, then $\djoin [f_i] \cdperp \djoin [g_j]$ if and only $f_i \cdperp
  g_j$ for all $i,j$;
\end{lemma}
\begin{proof}
  Assume $\djoin [f_i] \cdperp \djoin [g_j]$. Then by \axiom{Dj}{4} and associativity, we have
  $\djoin [f_i] \cdperp g_j$ for each $j$. Then, applying the reflexivity of $\cdperp$,
  \axiom{Dj}{4} and associativity, we have $f_i \cdperp g_j$ for each $i$ and $j$.

  Assume $f_i \cdperp g_j$ for each $i$ and $j$. Then by \axiom{Dj}{4} and associativity, $f_i
  \cdperp \djoin [g_j]$ for each $i$. Applying \axiom{Dj}{4} again, we have $\djoin [f_i] \cdperp
  \djoin [g_j]$.
\end{proof}

Following the same method as in the previous section, we show that the product of two
inverse categories with disjoint joins has a disjoint join.
\begin{lemma}\label{lem:disjoint_join_is_in_product_category}
  Given $\X, \Y$ are inverse categories with disjoint joins, $\djoin$ and $\djoin'$ respectively,
  then the category $\X \times \Y$ is an inverse category with disjoint joins.
\end{lemma}
\begin{proof}
  From Lemma~\ref{lem:disjointness_is_derivable_on_a_product_category}, we know $\X\times\Y$ has a
  disjointness relation that is defined pointwise.  We therefore define $\djoin_{\times}$ the
  disjoint join on $\X\times\Y$ by
  \begin{equation}
    (f,g)\djoin_{\times}(h,k) = (f \djoin h, g \djoin' k) \label{eq:disjoint_join_on_product}
  \end{equation}
  We now prove each of the axioms in Definition~\ref{def:disjoint_join} hold.
  \setlist[itemize,1]{leftmargin=1.5cm}
  \begin{itemize}
    \item [\axiom{DJ}{1}] From Equation~\eqref{eq:disjoint_join_on_product}, we see that since
      $f,h \le f \djoin h$ and $g,k \le g \djoin' k$, we have $(f,g) \le (f,g)\djoin_{\times}(h,k)$
      and  $(h,k) \le (f,g)\djoin_{\times}(h,k)$.
    \item [\axiom{DJ}{2}] Suppose $(f,g) \le (x,y)$, $(h,k) \le (x,y)$ and $(f,g) \perp_{\times} (h,k)$. Then
      regarding it point-wise, we have
      $(f,g)\djoin_{\times}(h,k) = (f \djoin h, g \djoin' k)   \le (x,y)$.
    \item [\axiom{DJ}{3}] $(x,y)\left((f,g)\djoin_{\times}(h,k)\right) = (x(f \djoin h), y(g \djoin' k))
      = (x f \djoin x h, y g \djoin' y k) = (x f,y g)\djoin_{\times}(x h,y k) =
      ((x,y)(f,g))\djoin_{\times}((x,y)(h,k))$.
    \item [\axiom{DJ}{4}] Given $\perp_{\times}\!\![(f,g),(h,k),(x,y)]$, we know $f \perp (h\djoin x)$ and
      $g \perp' (k\djoin' y)$. Hence, $(f,g) \perp_{\times}((h,k)\djoin_{\times}(x,y))$. The
      opposite direction is similar.
  \end{itemize}
\end{proof}

% subsection disjoint_joins (end)

\subsection{Monoidal Tensors for disjointness} % (fold)
\label{sub:tensors_for_disjointness}

Suppose we are given a monoidal tensor $\+$ on $\X$, an inverse category with a restriction zero. Under
certain conditions, it is possible to define disjointness based upon the action of the tensor. Note
that throughout, we are assuming the following naming for the standard monoidal tensor isomorphisms.
\begin{align*}
   \upl &: 0 \+ A \to A\\
   \upr &: A \+ 0 \to A\\
   a_\+ &: (A \+ B) \+ C \to A \+ (B \+ C) \\
   c_\+ &: A \+ B \to B \+ A.
 \end{align*}
Note we also require the tensor isomorphisms above be be natural.

\begin{definition}\label{def:disjointness_tensor}
  Suppose we are given an inverse category \X with restriction zero and a symmetric monoidal tensor
  $\+$. Then $\+$ is a \emph{disjointness tensor} when:
  \begin{itemize}
    \item It is a restriction functor --- i.e., $\_ \+ \_ : \X \times \X \to \X$.
    \item The unit is the restriction zero. ($0 : \boldsymbol{1}\to \X$ picks out the restriction
    zero in \X).
    \item Define $\Pi_0 := (1\+0)\upr: A\+B \to A$ and $\Pi_1 := (0\+1)\upl:A\+B \to B$. Then
      $\Pi_0$  and $\Pi_1$ are jointly monic. That is, whenever $f\Pi_0 = g \Pi_0$ and
      $f\Pi_1 = g\Pi_1$ then $f = g$.
    \item Define $\cpa = \inv{\upr}(1\+0):A\to A\+A$ and $\cpb = \inv{\upl}(0\+1): A\to A\+A$. Then
      $\cpa$ and $\cpb$ are jointly epic. That is, if $\cpa f = \cpa g$ and $\cpb f = \cpb g$, then
      $f = g$.
  \end{itemize}

\end{definition}

\begin{lemma}\label{lem:zero_plus_zero_is_zero}
  Given an inverse category \X with restriction zero and disjointness tensor $\+$, then the map
  $0\+0: A\+B \to C\+D$ is the map $0: A\+B \to C\+D$.
\end{lemma}
\begin{proof}
  Recall the zero map factors through the restriction zero, i.e. $0:A \to B$ is the same as
  saying $A\xrightarrow{!} 0 \xrightarrow{?}B$. Additionally, as objects, $0\+0 \cong 0$ ---
  the restriction zero.

  Therefore the map $0\+0: A\+B \to C\+D$ is writable as
  \[
    A\+B \xrightarrow{!\+!}0\+0  \xrightarrow{?\+?} C\+D,
  \]
  which may then be rewritten as
  \[
    A\+B \xrightarrow{!\+!}0\+0 \xrightarrow{\upl} 0
      \xrightarrow{\inv{\upl}} 0\+0 \xrightarrow{?\+?} C\+D.
  \]

  But by the properties of the restriction zero, $(!\+!) \upl = !$ and $\inv{\upl}(?\+?) = ?$
  and therefore the map $0\+0: A\+B \to C\+D$ is the same as the map $0: A\+B \to C\+D$.
\end{proof}


\begin{lemma}\label{lem:disjointness_tensor_has_injections}
  Given an inverse category \X with a restriction zero and a disjointness tensor, the map $\cpa$ is
  natural in the left component and $\cpb$ is natural in the right, up to isomorphism. This means:
  \[
    \cpa (f\+g) = f \cpa \quad\text{and}\quad \cpb (f\+g) = g \cpb.
  \]
\end{lemma}
\begin{proof}
  For the left and right naturality, we see:
  \[
    \cpa (f\+g) = \inv{\upr} (1\+0) (f\+g) = \inv{\upr} (f \+ 0) =
      f \inv{\upr} (1\+0) = f \cpa,
  \]
  and
  \[
    \cpb (f\+g) = \inv{\upl} (0\+1) (f\+g) = \inv{\upl} (0 \+ g) =
      g \inv{\upl} (0\+1) = g \cpb.
  \]
\end{proof}



\begin{lemma}\label{lem:tensor_identities}
  Given an inverse category \X with restriction zero and disjointness tensor $\+$, then the
  following hold:
  \begin{enumerate}
    \item $\Pi_i \cp{i} = \rst{\Pi_i}$ and $\cp{i}\Pi_i = \rst{\cp{i}} = 1$;
    \item $\rst{\Pi_0} \Pi_1 = 0$ and $\rst{\Pi_1} \Pi_0 = 0$;
    \item $\cpb \Pi_0 = 0$, $\cpb \rst{\Pi_0} = 0$,  $\cpa \Pi_1 = 0$ and  $\cpa \rst{\Pi_1} = 0$;
    \item the maps $\cpa$ and $\cpb$ are monic.
  \end{enumerate}
\end{lemma}
\begin{proof}
  For item 1, recalling that the restriction zero is its own partial inverse, we see from their
  definitions that $\Pi_i = \inv{\cp{i}}$. Calculating the restriction of $\cpa$,
  \begin{equation*}
    \cpa \Pi_0   = \inv{\upr}(1\+0) (1\+0)\upr
      = (\inv{\upr}(1\+0))\upr = 1 \inv{\upr}\upr= 1.
  \end{equation*}
  The calculation for $\Pi_1$ and $\cpb$ are analogous.

  For the second item,
  \[
    \rst{\Pi_0} \Pi_1  = \rst{(1\+0)\upr}(0\+1)\upl  = \rst{1\+0} (0 \+ 1)\upl
     = (1\+0) (0\+1)\upl
      = (0\+0)\upl  = 0,
  \]
  and
  \[
    \rst{\Pi_1} \Pi_0  = \rst{(0\+1)\upl}(1\+0)\upr
     = (0\+1) (1\+0)\upr
     = (0\+0)\upr
     = 0.
 \]

  For the third item, we see
  \[
    \cpa \Pi_1 = (\inv{\upr}(1\+0))(0\+1) \upl
       = \inv{\upr}(0\+0)\upl\\
       = 0
  \]
  and
  \[
    \cpb \Pi_0 = (\inv{\upl}(0\+1))(1\+0) \upr
       = \inv{\upl}(0\+0)\upr
       = 0.
  \]
  As $\rst{\Pi_0} = 1\+0$ and $\rst{\Pi_1} = 0 \+1$, we see the other two identities hold as well.

  To prove $\cpa$ is monic, suppose $f\cpa = g\cpa$. Therefore we must have
  \[
    f = f (\cpa \Pi_0) = (f \cpa) \Pi_0 = (g \cpa) \Pi_0 = g (\cpa \Pi_0) = g.
  \]
  The proof that $\cpb$ is monic is similar.
\end{proof}
\begin{corollary}\label{cor:split_plus_tensor}
  In an inverse category \X with a restriction zero and disjointness tensor, the following hold:
  \begin{multicols}{2}
    \begin{enumerate}[{(}i{)}]
      \item $\cpa f \+g \Pi_0 = f$;
      \item $\cpa f\+ g \Pi_1 = 0$;
      \item $\cpb f \+g \Pi_0 = 0$;
      \item $\cpb f\+ g \Pi_1 = g$.
    \end{enumerate}
  \end{multicols}
  Additionally, if $t$ is a map such that for $i \in \{0,1\}$,
  \[
    \cp{i} t \Pi_j =
    \begin{cases}
      t_i& : \quad i \neq j\\
      0 & : \quad i = j,
    \end{cases}
  \]
  then $t = t_0 \+ t_1$
\end{corollary}
\begin{proof}
  The calculations for $f\+g$ follow from Lemma~\ref{lem:disjointness_tensor_has_injections} and
  Lemma~\ref{lem:tensor_identities}. For example, $\cpa f\+g \Pi_0 = f \cpa \Pi_0 = f$.

  For the second claim, note that we have $\cpa (t \Pi_0) = t_0 = \cpa ( t_0 \+ t_1)\Pi_0$ and
  $\cpb  (t \Pi_0) = 0 = \cpb ( t_0 \+ t_1)\Pi_0$, hence $t \Pi_0 = ( t_0 \+ t_1)\Pi_0$. Similary,
  we see $t\Pi_1 = ( t_0 \+ t_1)\Pi_1$ and therefore $t = t_0 \+ t+1$.
\end{proof}
\begin{definition}\label{def:up_triangle_and_down_triangle}
  In an inverse category \X with a restriction zero and disjointness tensor, we define two partial
  operations on pairs of arrows in \X to another arrow in \X. First, for arrows
  $f:A \to B$ and $g:A \to C$, we define $f\tjleft g$ as
  being the map that makes diagram \eqref{eq:tjleftdefinition} below commute, when it exists.
  \begin{equation}
    \xymatrix@C+0pt @R+20pt{
      B&&B\+C \ar[ll]_{\Pi_0} \ar[rr]^{\Pi_1} && C\\
      &&A \ar[ull]^{g} \ar[urr]_{f} \ar@{.>}[u]|-{f\tjleft g}
      }\label{eq:tjleftdefinition}
  \end{equation}
  Then for $h:B \to A$, $k: C \to A$, $h\tjright k$ is that map that makes diagram
  \eqref{eq:tjrightdefinition} commute, if it exists.
  \begin{equation}
    \xymatrix@C+0pt @R+20pt{
      &&A\\
      B\ar[rr]_{\cpa} \ar[urr]^{h} &&B\+C   \ar@{.>}[u]|-{h\tjright k}
        && C \ar[ll]^{\cpb} \ar[ull]_{k}
      }\label{eq:tjrightdefinition}
  \end{equation}
  Due to $\Pi_0$ and $\Pi_1$ being jointly monic, $f\tjleft g$ is unique when it exists.
  Similarly, as $\cpa$ and $\cpb$ are jointly epic, $f\tjright g$ is unique when it exists.
\end{definition}

We give a lemma exploring the
behaviour of the two operations: $\tjleft$ and $\tjright$.

\begin{lemma}\label{lem:properties_of_tjleft_and_tjright}
  Given \X is an inverse category with a restriction zero and a disjointness tensor $\+$ then the
  following relations hold for $\tjleft$ and $\tjright$:
   \begin{enumerate}[{(}i{)}]
    \item If $f \tjleft g$ exists, then $g \tjleft f$ exists. If $f \tjright g$ exists, then
    $g \tjright f$ exists. \label{lemitem:l_r_commute}
    \item $f \tjleft 0$ and $f \tjright 0$ always exist. \label{lemitem:l_r_zero_exists}
    \item When $f\tjleft g$ exists, $\rst{f}(f\tjleft g) = f \tjleft 0$, $\rst{f}g = 0$,
    $\rst{g}(f \tjleft g) = 0 \tjleft g$ and $\rst{g}f = 0$.\label{lemitem:l_r_rst_orthogonal_to_zero}
    \item Dually to the previous item, when $f\tjright g$ exists,
    $(f\tjright g)\rg{f} = f \tjright 0$, $g\rg{f} = 0$,
    $(f \tjright g)\rg{g} = 0 \tjright g$ and $f\rg{g} = 0$.\label{lemitem:l_r_rg_orthogonal_to_zero}
    \item When $f\tjleft g$ exists, $f\tjleft g (h\+k) = f h \tjleft g k$.
      \label{lemitem:l_r_l_pull_right}
    \item Dually, when $f \tjright g$ exists, $(h \+ k) f \tjright g = h f \tjright k g$.
      \label{lemitem:l_r_r_pull_left}
    \item When $f \tjleft g$ exists, then $h(f \tjleft g) = h f \tjleft hg$ and when $f \tjright g$
      exists, $(f \tjright g)h = f h \tjright g h$.
      \label{lemitem:l_r_l_universal_r_stable}
    \item If $\rst{f} \tjleft \rst{g}$ exists, then $\rst{f} \tjright \rst{g}$ exists and is
      the partial inverse of $\rst{f} \tjleft \rst{g}$.
      \label{lemitem:l_r_rst_inverses}
    \item If $f\tjleft g$ exists and $f' \le f$, $g' \le g$, then $f' \tjleft g'$ exists.
      \label{lemitem:l_r_less_than}
    \item When $f\tjright g$ exists, $(f\tjright g) \inv{(f\tjright g)} = \rst{f}\+\rst{g}$.
      \label{lemitem:l_r_rst_is_natural}
    \item Given $f\tjleft g$ and $h\tjleft k$ exist, then
      $(f\+h)\tjleft(g\+k) = (f\tjleft g)\+ (h\tjleft k)$. Dually, the existence of $f\tjright g$
      and $h\tjright k$ implies $(f\+h)\tjright(g\+k) = (f\tjright g)\+ (h\tjright k)$.
      \label{lemitem:l_r_preserve_tensor}
  \end{enumerate}
\end{lemma}
\begin{proof}
  \prepprooflist
  \begin{enumerate}[{(}i{)}]
    \item $g \tjleft f = (f \tjleft g)\comp $ and $g \tjright f = \comp(f\tjright g)$.
    \item Consider $f \cpa$. Then $f \cpa \Pi_0 = f$ and
      $f \cpa \Pi_1 = f 0 = 0$. Hence, $f \cpa = f \tjleft 0$.

      Consider $\Pi_0 f$. Then $\cpa \Pi_0 f = f$ and
      $\cpb \Pi_0 f = 0 f = 0$ and therefore $\Pi_0 f = (f\tjright 0)$.
    \item Using Lemma~\ref{lem:tensor_identities}
      \[
        \rst{f}g = \rst{(f\tjleft g)\Pi_0} (f\tjleft g)\Pi_1 =
          (f\tjleft g) \rst{\Pi_0} \Pi_1 = 0.
      \]
      Similarly, $\rst{g}f = f\tjleft g \rst{\Pi_1}\Pi_0 = 0$.

      Recall that $\Pi_0$ and $\Pi_1$ are jointly monic. We have
      $\rst{f}(f \tjleft g) \Pi_0 = \rst{f} f = f = (f\tjleft 0) \Pi_0$ and
      $\rst{f}(f \tjleft g) \Pi_1 = \rst{f}g = 0= (f\tjleft 0) \Pi_1$.
      Therefore, $\rst{f}(f \tjleft g) = f \tjleft 0$. Similarly,
      $\rst{g}(f \tjleft g) = 0 \tjleft g$.
    \item Using Lemma~\ref{lem:tensor_identities}
      \begin{multline*}
        g\rg{f} = \cpb(f\tjright g)\wrg{(\cpa(f\tjright g))}  =
          \cpb(f\tjright g) \rst{\inv{(f\tjright g)}\Pi_0} \\
          = \cpb(f\tjright g) \rst{\inv{(f\tjright g)}\rst{\Pi_0}} =
          \rst{\cpb \rst{(f\tjright g)} \rst{\Pi_0}}\cpb(f\tjright g) =\\
          \rst{\cpb \rst{\Pi_0}\rst{(f\tjright g)} }\cpb(f\tjright g) =
          \rst{0}\cpb(f\tjright g) = 0
      \end{multline*}
      Similarly, $f\rg{g} =  0$.

      Recall that $\cpa$ and $\cpb$ are jointly epic. We have
      $\cpa(f \tjright g)\rg{f} = f\rg{f} = f = \cpa(f\tjright 0) $ and
      $\cpb(f \tjright g)\rg{f} = g\rg{f} = 0= \cpb(f\tjright 0)$.
      Therefore, $(f \tjright g)\rg{f} = f \tjright 0$. Similarly,
      $(f \tjright g) \rg{g} = 0 \tjright g$.
    \item Calculating, we have
      \[
        f \tjleft g (h \+ k) \Pi_0 = f\tjleft g \Pi_0 h = f h \text{ and }
        f \tjleft g (h \+ k) \Pi_1 = f\tjleft g \Pi_1 k = g k,
      \]
      but this means that
      $f \tjleft g (h \+ k) = f h \tjleft g k$ by the joint monic property of $\Pi_0$, $\Pi_1$.
    \item The proof for this is dual to the previous item, and depends on the joint epic property
      of $\cpa$ and $\cpb$.
    \item We are given $f\tjleft g$ exists, therefore $f = (f\tjleft g)\Pi_0$ and
      $g = (f\tjleft g)\Pi_1$. But this means $h f = h(f\tjleft g)\Pi_0$ and
      $h g = h(f\tjleft g)\Pi_1$, from which we may conclude $h f \tjleft h g = h(f\tjleft g)$ by
      the fact that $\Pi_0$ and $\Pi_1$ are jointly monic. The proof of
      $(f\tjright g)h = f h \tjright g h$ is similar.
    \item We are given $\rst{f} = \rst{f}\tjleft\rst{g} \Pi_0$. Therefore,
      \[
        \rst{f} = \inv{\rst{f}} = \inv{\Pi_0}\inv{(\rst{f}\tjleft\rst{g})}
          = \cpa \inv{(\rst{f}\tjleft\rst{g})}.
      \]
      Similarly, $\rst{g} = \cpb \inv{(\rst{f}\tjleft\rst{g})}$. But this means
      $\inv{(\rst{f}\tjleft\rst{g})} = \rst{f}\tjright\rst{g}$.
    \item Note that from item \ref{lemitem:l_r_l_pull_right}, we know that
      $f \tjleft g  = \rst{f} \tjleft \rst{g} (f\+g)$. We are given $f' \le f$ and $g' \le g$. This
      gives us $\rst{f'}f =f'$, $\rst{g'}g =g'$, $\rst{f'}\,\rst{f} =\rst{f'}$ and
      $\rst{g'}\,\rst{g} =\rst{g'}$. Consider the map
      $\rst{f} \tjleft \rst{g} (\rst{f'}\+\rst{g'})(f\+g)$. Calculating, we see
      \begin{align*}
        \rst{f} \tjleft \rst{g} (\rst{f'}\+\rst{g'})(f\+g)
          &= \rst{f} \tjleft \rst{g} (\rst{f'}\+\rst{g'})(\rst{f'}\+\rst{g'})(f\+g) \\
          & = \rst{f} \tjleft \rst{g} (\rst{f'}\+\rst{g'})(f'\+g') \\
          & = \rst{f}\,\rst{f'} \tjleft \rst{g} \rst{g'} (f'\+g') \\
          & = \rst{f'}\,\rst{f} \tjleft \rst{g'} \rst{g} (f'\+g') \\
          & = \rst{f'} \tjleft \rst{g'}  (f'\+g') \\
          & = f'\tjleft g'.
      \end{align*}
    \item From our diagram for $\tjright$, we know that $\inv{f} = \inv{(f\tjright g)} \Pi_0$ and
      $\inv{g} = \inv{(f\tjright g)} \Pi_1$. As well, we know that $\cpa(f\tjright g) = f$ and
      $\cpa(f\tjright g) = g$.
      Therefore, we have:
      \[
         \cpa(f\tjright g)\inv{(f\tjright g)} \Pi_0 = \rst{f} \text{ and  }
          \cpb(f\tjright g)\inv{(f\tjright g)} \Pi_1 = \rst{g}.
      \]
      As $f\tperp g$, we know that $f\inv{g} = f \rg{g} \inv{g} = 0 \inv{g} = 0$ and therefore,

      \[
         \cpa(f\tjright g)\inv{(f\tjright g)} \Pi_1 = 0 \text{ and  }
          \cpb(f\tjright g)\inv{(f\tjright g)} \Pi_0 = 0.
      \]
      By Corollary~\ref{cor:split_plus_tensor} this means
      $(f\tjright g)\inv{(f\tjright g)} = \rst{f} \+ \rst{g}$.
    \item As $(f\tjleft g)\+ (h\tjleft k)\Pi_0 = (f\tjleft g)$ and
      $(f\tjleft g)\+ (h\tjleft k) \Pi_1 = (h\tjleft k)$, we see that
      $(f\tjleft g)\+ (h\tjleft k)$ satisfies the diagram for $(f\+h)\tjleft(g\+k)$. Dually, as
      $\cpa (f\tjright g)\+ (h\tjright k) = (f\tjright g)$ and
      $\cpb(f\tjright g)\+ (h\tjright k) = (h\tjright k)$, $(f\tjright g)\+ (h\tjright k)$ satisfies
      the diagram for $(f\+h)\tjright(g\+k)$.
  \end{enumerate}
\end{proof}

% \begin{definition}\label{def:general_triangle}
%   In an inverse category \X with a restriction zero and disjointness tensor, we define two partial
%   operations on pairs of arrows in \X to another arrow in \X. First, for arrows
%   $f:A \to B$ and $g:A \to C$, suppose we have arrows $f':A  \to B\+X$ and $g': A \to C\+X$
%   such that $f'\Pi_0 =f$, $g'\Pi_0 = g$ and $f'\Pi_1 = g' \Pi_1$. Then, we say $f\gtjleft g$ if
%   regardless of the choice of $f',\ g'$ and $X$, there exists a map $\alpha$ that makes
%   diagram \eqref{eq:gtjleftdefinition} below commute.
%   \begin{equation}
%     \xymatrix@C+0pt @R+20pt{
%       B\+X&&B\+C\+X \ar[ll]_{\Pi_{0,2}} \ar[rr]^{\Pi_{1,2}} && C\+X\\
%       &&A \ar[ull]^{g'} \ar[urr]_{f'} \ar@{.>}[u]|-{\alpha}
%       }\label{eq:gtjleftdefinition}
%   \end{equation}
%
%   Similarly, for $h:B \to A$, $k: C \to A$, and $h': A\+X \to C$, $k': B\+X \to C$, with $h = \cpa
%   h'$ and $k = \cpb k'$, then $h\gtjright k$ if regardless of the choice of $h',\ k'$ and $X$, there
%   exists a map $\beta$ that makes diagram \eqref{eq:gtjrightdefinition} commute.
%   \begin{equation}
%     \xymatrix@C+0pt @R+20pt{
%       &&A\\
%       B\+X\ar[rr]_{\amalg_{0,2}} \ar[urr]^{h'} &&B\+C \+X  \ar@{.>}[u]|-{\beta}
%         && C\+X \ar[ll]^{\amalg_{1,2}} \ar[ull]_{k'}
%       }\label{eq:gtjrightdefinition}
%   \end{equation}
%   In the above diagrams, $\Pi_{0,2} = (1\+0\+1)(\upr\+1)$, $\amalg_{0,2} =
%   (\inv{\upr}\+1)(1\+0\+1)$, $\Pi_{1,2}=(0\+1\+1)(\upl\+1)$ and
%   $\cp{1,2} = (\inv{\upl}\+1)(0\+1\+1)$.
% \end{definition}
% \begin{lemma}\label{lem:general_triangle_implies_triangle}
%   In an inverse category $\X$ with a restriction zero and a disjointness tensor, with maps
%   $f:A\to B$, $g: A \to C$, $h: B \to A$ and $k: C \to A$. Then $f\gtjleft g$
%   (respectively $h \gtjright k$) implies that $f\tjleft g$ (respectively $h \tjright k$) exists.
% \end{lemma}
% \begin{proof}
%   Suppose $f \gtjleft g$. Then set $f' = f \inv{\upr}$, $g' = g \inv{\upr}$ and $X = 0$. Then
%   $\alpha: A \to B\+C\+0$ exists and we can see that setting $f\tjleft g = \alpha \upr$ makes the
%   first diagram of Definition~\ref{def:up_triangle_and_down_triangle} commute. In detail,
%   \[
%     \xymatrix@C+0pt @R+10pt{
%       B&&B\+C \ar[ll]_{\Pi_0} \ar[rr]^{\Pi_1} && C\\
%       &B\+0\ar[ul]^{\upl}&B\+C\+0 \ar[l]_{\Pi_{0,2}} \ar[r]^{\Pi_{1,2}} \ar[u]|-{\upl}
%         & C\+0\ar[ur]_{\upl}\\
%       &&A \ar[ul]^{g \inv{\upl}} \ar[ur]_{f\inv{\upl}} \ar@{.>}[u]|-{\alpha}
%       }
%   \]
%   where the lower triangle commutes by assumption and the upper quadrilaterals commute due to the
%   coherence diagrams of the tensor.
%
%   Similarly, setting $h' = \inv{\upr}h$ and $k' = \inv{\upr}k$ gives us $\beta:B\+C\+0 \to A\+0$ and
%   we set $f\tjright = \beta \upr.$
% \end{proof}

\begin{definition}\label{def:disjointness_via_a_disjointness_tensor}
  Define $f \tperp g$ when $f, g: A\to B$ and both $f\tjleft g$ and $f\tjright g$.
\end{definition}

\begin{lemma}\label{lem:tensor_disjointness_is_disjointness}
  If \X is an inverse category with a restriction zero and a disjointness tensor $\+$ then
  the relation $\tperp$ is a disjointness relation.
\end{lemma}
\begin{proof}
  We need to show that $\tperp$ satisfies the disjointness axioms. We will use \axiom{Dis}{6'} in
  place of \axiom{Dis}{6} and \axiom{Dis}{7} as discussed in
  Lemma~\ref{lem:disjointness_equivalent_axioms}.
  \setlist[itemize,1]{leftmargin=1.5cm}
  \begin{itemize}
    \item[\axiom{Dis}{1}] We must show $f \tperp 0$. This follows immediately from
      Lemma~\ref{lem:properties_of_tjleft_and_tjright}, item \ref{lemitem:l_r_zero_exists}.
    \item[\axiom{Dis}{2}] Show $f \tperp g$ implies $\rst{f}g = 0$. This is a direct consequence of
      Lemma~\ref{lem:properties_of_tjleft_and_tjright}, item
      \ref{lemitem:l_r_rst_orthogonal_to_zero}.
    \item[\axiom{Dis}{3}] We require $f\tperp g$, $f' \le f$, $g' \le g$ implies $f' \tperp g'$.
      From Lemma~\ref{lem:properties_of_tjleft_and_tjright}, item \ref{lemitem:l_r_less_than}, we
      immediately have $f' \tjleft g'$ exists. Using a similar argument to the proof of this item,
      we also have $f' \tjright g'$ exists and hence $f' \tperp g'$.
    \item[\axiom{Dis}{4}] Commutativity of $\tperp$ follows from the symmetry of the two required
    diagrams, see Lemma~\ref{lem:properties_of_tjleft_and_tjright}, item
      \ref{lemitem:l_r_commute}.
    \item[\axiom{Dis}{5}] Show that if $f\tperp g$ then $h f \tperp h g$ for any map $h$.
      By  Lemma~\ref{lem:properties_of_tjleft_and_tjright}, item
      \ref{lemitem:l_r_l_universal_r_stable}, we know that $h f \tjleft h g$ exists.
      By item \ref{lemitem:l_r_r_pull_left},  $(h f)\tjright (h g) = (h \+ h) (f\tjright g)$
      and therefore $h f \tperp h g$.
    \item[\axiom{Dis}{6'}] We need to show $f\tperp g$ if and only if $\rst{f} \tperp \rst{g}$ and
      $\rg{f}\tperp \rg{g}$. This follows directly from
      Lemma~\ref{lem:properties_of_tjleft_and_tjright}, items
      \ref{lemitem:l_r_l_pull_right} and       \ref{lemitem:l_r_r_pull_left}, which give us
      $f\tjleft g = \rst{f}\tjleft \rst{g} (f\+g)$ and
      $f\tjright g = (f \+ g)\rg{f}\tjright \rg{g}$, where the equalities hold if either side of
      the equation exists.
  \end{itemize}
\end{proof}

The operations $\tjleft$ and $\tjright$ are sufficient to define a disjointness relation
on an inverse category. However, when we wish to extend this to a disjoint join, we run into
problems when trying to prove $\axiom{DJ}{4}$. Specifically, there is not enough information to
show that $\tperp [f,g,h]$ implies $f \tperp (g \tjoin h)$.

Therefore, we add one more assumption regarding our tensor in order to define disjointness.

\begin{definition}\label{def:disjoint_sum_tensor}
  Let \X be an inverse category with a disjointness tensor $\+$ and a restriction zero. Consider
  diagrams \ref{dia:tensor_complete_left} and \ref{dia:tensor_complete_right}.
  \begin{align}
    &\xymatrix@C+5pt{
      A \ar@/_/[ddr]_f \ar@/^/[drr]^g \ar@{.>}[dr]|-{\alpha} \\
        & X\+Y\+Z \ar[d]^{\Pi_{0,1}} \ar[r]_{\Pi_{0,2}} & X\+Z \ar[d]^{\Pi_0} \\
        & X\+Y \ar[r]_{\Pi_0} & X
    } \label{dia:tensor_complete_left}\\
    &\xymatrix@C+5pt{
        &&A \\
         X\+Y \ar[r]_{\cp{0,1}} \ar@/^/[urr]^h &X\+Y\+Z \ar@{.>}[ur]|-{\beta} \\
         X \ar[u]^{\cpa} \ar[r]_{\cpa} & X\+Y \ar[u]^{\cp{0,2}} \ar@/_/[uur]_k
    } \label{dia:tensor_complete_right}
  \end{align}
  Then
  $\+$ is a \emph{disjoint sum tensor} when the following two conditions hold:
  \begin{itemize}
    \item $\alpha$ exists if and only if  $f \Pi_1 \tjleft g \Pi_1$ exists;
    \item $\beta$ exists if and only if $\cpb h \tjright \cpb k$ exists.
  \end{itemize}

\end{definition}
\begin{lemma}\label{lem:complete_disjointness_means_multiple_disjoints}
  Let \X be an inverse category with a disjoint sum tensor as in
  Definition~\ref{def:disjoint_sum_tensor} and we are given $f,g,h:A\to B$ with
  $\tperp[f,g,h]$. Then both $f \tjleft (g \tjleft h)$ and $f\tjright(g\tjright h)$ exist.
\end{lemma}
\begin{proof}
  As all the maps are disjoint, we know that each pair's $\tjleft$ map and $\tjright$ maps
  exist.  Consider the diagram
  \[
    \xymatrix@C+5pt{
      A \ar@/_/[ddr]_{g\tjleft h} \ar@/^/[drr]^{g\tjleft f} \ar@{.>}[dr]|-{\alpha} \ar[rr]^{f}
        & & B \\
        & B\+B\+B \ar[d]^{\Pi_{0,1}} \ar[r]_{\Pi_{0,2}} & B\+B \ar[d]^{\Pi_0} \ar[u]^{\Pi_1} \\
        & B\+B \ar[r]_{\Pi_0} & B
    }
  \]
  where we claim $\alpha = (g\tjleft h)\tjleft f$.

  The lower part of the diagram commutes as it fulfills the conditions of
  Definition~\ref{def:disjoint_sum_tensor}. The upper triangle of the diagram commutes by the
  definition of $g\tjleft f$. Noting that $\Pi_{0,1}:B\+B\+B \to B\+B$ is the same map as
  $\Pi_0:(B\+B)\+B\to (B\+B)$ and $\Pi_{0,2} \Pi_1:B\+B\+B \to B\+B \to B$ is the same map as
  $\Pi_1:(B\+B)\+B\to B$, we see $\alpha$ does make the $\tjleft$ diagram for $g\tjleft h$ and $f$
  commute. Therefore by Lemma~\ref{lem:properties_of_tjleft_and_tjright}, $f\tjleft(g\tjleft h)$
  exists and is equal to $\alpha \com{\+\,\{01,2\}}$.

  A dual diagram and reasoning shows $f\tjright(g\tjright h)$ exists.
\end{proof}

\begin{lemma}\label{lem:tjleft_and_tjright_associate}
  In an inverse category $\X$ with a disjoint sum tensor, when $\tperp [f,g,h]$, then:
  \begin{enumerate}
    \item   $f \tjleft (g \tjleft h) = ((f \tjleft g) \tjleft h) \assocp$ and both exist;
    \item   $f \tjright (g \tjright h) = ((f \tjright g) \tjright h) \assocp$ and both exist;
  \end{enumerate}
\end{lemma}
\begin{proof}
  Consider the diagram
  \[
    \xymatrix@C+5pt{
      A \ar@/_/[ddr]_{f\tjleft g} \ar@/^/[drr]^{f\tjleft h} \ar@{.>}[dr]|-{\alpha} \ar[rr]^{h}
        & & B \\
        & B\+B\+B \ar[d]^{\Pi_{0,1}} \ar[r]_{\Pi_{0,2}} & B\+B \ar[d]^{\Pi_0} \ar[u]^{\Pi_1} \\
        & B\+B \ar[r]_{\Pi_0} & B
    }
  \]
  which gives us $\alpha = (f \tjleft g) \tjleft h: A \to (B\+B)\+B$ and
  $\alpha \assocp :A \to B\+(B\+B)$. Next consider the diagram
  \[
    \xymatrix@C+5pt{
      A \ar@/_/[ddr]_{g\tjleft h} \ar@/^/[drr]^{g\tjleft f} \ar@{.>}[dr]|-{\gamma} \ar[rr]^{f}
        & & B \\
        & B\+B\+B \ar[d]^{\Pi_{0,1}} \ar[r]_{\Pi_{0,2}} & B\+B \ar[d]^{\Pi_0} \ar[u]^{\Pi_1} \\
        & B\+B \ar[r]_{\Pi_0} & B
    }
  \]
  which gives us $\gamma \comp = f \tjleft (g\tjleft h): A \to B\+(B\+B)$.

  Note from the diagrams we have:
  \begin{align*}
    \gamma \comp  \Pi_{0}         =\, &f  = \alpha \assocp\Pi_0\\
    \gamma \comp  \Pi_{1} \Pi_{0} =\, &g  = \alpha \assocp\Pi_1\Pi_0\\
    \gamma \comp  \Pi_{1} \Pi_1   =\, & h = \alpha \assocp\Pi_1\Pi_1.
  \end{align*}
  Hence, by the assumption that $\Pi_0, \Pi_1$ are jointly monic, we have
  $\alpha = \gamma \comp  \assocp$ and hence $f \tjleft (g \tjleft h) = (f \tjleft g) \tjleft h$,
  up to the associativity isomorphism.
\end{proof}
\begin{definition}\label{def:tensor_disjoint_join}
  Let \X be an inverse category with a disjointness tensor and restriction zero. Assume we have two
  maps $f,g: A \to B$ with $f\tperp g$. Then define the map $f \tjoin g = \rst{f}\tjleft\rst{g}
  (f\+g) \rg{f}\tjright\rg{g}$.
\end{definition}


For reference, the map $f \tjoin g$ may be visualized as follows:
\[
  \xymatrix@C+20pt @R+10pt{
    &A
      & B  \ar[dr]^{\rg{f}} \ar[d]_{\cpa}\\
    A \ar[ur]^{\rst{f}} \ar@{.>}[r]^{\rst{f}\tjleft \rst{g}} \ar[dr]_{\rst{g}}
      & A\+A \ar[u]_{\Pi_0} \ar[d]^{\Pi_1}
        \ar[r]^{f\+g}
       &B\+B \ar@{.>}[r]^{\rg{f}\tjright\rg{g}} & B\\
    & A & B\ar[ur]_{\rg{g}} \ar[u]^{\cpb}
  }
\]

Using Lemma~\ref{lem:properties_of_tjleft_and_tjright}, we may rewrite this in a variety of
equivalent ways:
\begin{align*}
  f \tjoin g &= \rst{f}\tjleft\rst{g} (f\+g) \rg{f}\tjright\rg{g} \\
  &= f \tjleft g \rg{f}\tjright\rg{g}\\
  & = \rst{f}\tjleft\rst{g}  f\tjright g\\
  & = f \tjleft g (\inv{f}\+\inv{g})f\tjright g
\end{align*}

In particular, note that $\rst{f}\tjoin \rst{g} = (\rst{f} \tjleft {g}) (\rst{f}\tjright\rst{g})$ as
$\rg{\rst{g}} = \rst{g}$.

\begin{lemma}\label{lem:tensor_disjoint_join_properties}
  Let \X be an inverse category with a disjointness tensor and restriction zero. Let \X have the
  maps $f,g: A \to B$ with $f\tperp g$. Then $\tjoin$ has the following properties.
  \begin{enumerate}[{(}i{)}]
    \item For all maps $h:A \to B$, $\rst{f}h \tjoin \rst{g}h = (\rst{f}\tjoin \rst{g})h$. \label{lemitem:tdj_rst_universal}
    \item $\rst{f}\tjoin \rst{g} = \rst{f\tjoin g}$. \label{lemitem:tdj_rst_is_rst}
  \end{enumerate}
\end{lemma}
\begin{proof}
  \begin{enumerate}[{(}i{)}]
    \item By Lemma~\ref{lem:disjointness_various}, item \ref{lemitem:djv_universal}, we know that
    $\rst{f}h \tperp \rst{g} h$, hence we can form $\rst{f}h \tjoin \rst{g}h$.
    Also, noting that
      \[
        h \wrg{\rst{f} h} = h \rst{\inv{h}\rst{f}} = \rst{h \inv{h} \rst{f}} h
          = \rst{\rst{h}\rst{f}} h = \rst{f} \rst{h}h = \rst{f} h,
      \]
      we may then calculate from the left hand side as follows:
      \begin{align*}
        \rst{f}h \tjoin \rst{g}h
          & = (\rst{f}h \tjleft \rst{g} h) (\wrg{\rst{f}h}\tjright\wrg{\rst{g} h})\\
          & = (\rst{f} \tjleft \rst{g} ) (h\wrg{\rst{f}h}\tjright h\wrg{\rst{g} h})\\
          & = (\rst{f} \tjleft \rst{g} ) (\rst{f}h\tjright \rst{g} h)\\
          & = (\rst{f} \tjleft \rst{g} ) (\rst{f}\tjright \rst{g} )h\\
          & = (\rst{f} \tjoin \rst{g} ) h.
      \end{align*}
    \item
    Using Lemma~\ref{lem:properties_of_tjleft_and_tjright}, item~\ref{lemitem:l_r_rst_is_natural},
    we can compute:
    \begin{align*}
      \rst{f\tjoin g} & = f \tjoin g \inv{(f\tjoin g)} \\
      & = \left((\rst{f} \tjleft \rst{g})(f\tjright g)\right)
             \left(\inv{(f\tjleft g)} \inv{(\rst{f}\tjleft \rst{g})}\right)\\
      & = \rst{f}\tjleft\rst{g}(f\tjleft g) \inv{(f\tjleft g)} \rst{f}\tjright \rst{g}\\
      & = \rst{f}\tjleft\rst{g}(\rst{f}\+ \rst{g})\rst{f}\tjright \rst{g}\\
      & = \rst{f}\tjleft\rst{g}\rst{f}\tjright \rst{g}\\
      & = \rst{f}\tjoin\rst{g}
    \end{align*}
  \end{enumerate}
\end{proof}


\begin{proposition}\label{prop:disjointness_tensor_gives_disjoint_join}
  Let \X be an inverse category with a disjoint sum tensor and restriction zero. Assume we have two
  maps $f,g$ with $f\tperp g$. Then the map $f \tjoin g$ from
  Definition~\ref{def:tensor_disjoint_join} is a disjoint join.
\end{proposition}
\begin{proof}
  \setlist[itemize,1]{leftmargin=1.5cm}
  \begin{itemize}
    \item [\axiom{DJ}{1}] We must show $f, g \le f \tjoin g$. Computing,
      \begin{align*}
        \rst{f}\,(\rst{f}\tjleft\rst{g})  f\tjright g
          &= (\rst{f}\tjleft\rst{g}) \Pi_0 (\rst{f}\tjleft\rst{g} ) f\tjright g \\
        &=\rst{(\rst{f}\tjleft\rst{g}) \Pi_0} (\rst{f}\tjleft\rst{g}) f\tjright g  \\
        &=(\rst{f}\tjleft\rst{g}) \rst{\Pi_0} f\tjright g  \\
        &=(\rst{f}\tjleft\rst{g}) \Pi_0 \cpa f\tjright g \\
        &=((\rst{f}\tjleft\rst{g}) \Pi_0) (\cpa (f \tjright g)) \\
        &= \rst{f} f \\
        &=f
      \end{align*}
      we see $f \le f \tjoin g$. Showing $g \le f \tjoin g$ proceeds in the same manner.
    \item [\axiom{DJ}{2}] We must show that $f \le h,\ g\le h$ and $f\tperp g$ implies
      $f \tjoin g \le h$. First, note that
      \begin{align*}
        \rst{f \tjoin g} \,h & = \rst{\rst{f} h \tjoin \rst{g} h} \,h\\
        & = \rst{(\rst{f} \tjoin \rst{g})h} \,h\\
        & = \rst{\rst{(\rst{f} \tjoin \rst{g})}h} \,h\\
        & = \rst{\rst{(\rst{f} \tjoin \rst{g})}h}\,(\rst{f} \tjoin \rst{g})h\\
        & = \rst{(\rst{f} \tjoin \rst{g})h}\,(\rst{f} \tjoin \rst{g})h\\
        & = (\rst{f}\tjoin \rst{g})h\\
        & = (\rst{f}h\tjoin \rst{g}h)\\
        & = (f\tjoin g)
      \end{align*}
    \item [\axiom{DJ}{3}] We must show stability of $\tjoin$, i.e., that
      $h(f\tjoin g) = h f \tjoin h g$.

      \begin{align*}
        h (f\tjoin g) &= h ((\rst{f}\tjleft\rst{g})  (f \tjright g))\\
        &= (h\rst{f}\tjleft h\rst{g}) (f \tjright g)\\
        &= (\rst{h f}h\tjleft \rst{h g}h)  (f \tjright g)\\
        &= (\rst{h f}\tjleft \rst{h g})(h\+h)  (f \tjright g)\\
        &= (\rst{h f}\tjleft \rst{h g}) (h f \tjright hg)\\
        &= h f \tjoin h g
      \end{align*}


    \item [\axiom{DJ}{4}] We need to show $\tperp [f,g,h]$ if and only if $f \tperp (g \tjoin h)$.
      For the right to left implication, note that the existence of $g\tjoin h$ implies $g \tperp h$.
      We also know $g, h \le g\tjoin h$ by item 1 of this lemma. This gives us that $f \tperp g$ and
      $f \tperp h$, hence $\tperp [f,g,h]$.

      For the left to right implication, we use
      Lemma~\ref{lem:complete_disjointness_means_multiple_disjoints}. As we have $\tperp [f,g,h]$,
      we  know $f\tjleft(g\tjleft h)$ and $f\tjright(g\tjright h)$.

      Recall that $g\tjoin h = (g\tjleft h)(\rg{g}\tjright\rg{h})$. Then the map
      \[
        A\xrightarrow{f\tjleft(g\tjleft h)} B\+B\+B\xrightarrow{1\+(\rg{g}\tjright\rg{h})} B\+B
      \]
      makes the diagram for $f \tjleft (g\tjoin h)$ commute.

      Recalling that  $g\tjoin h =
      (\rst{g}\tjleft \rst{h})(g\tjright h)$, we also see that
      \[
        A\+A\xrightarrow{1\+(\rst{g}\tjleft\rst{h})}A\+A\+A\xrightarrow{f\tjright(g\tjright h)}B
      \]
      provides the witness map for $f \tjright(g\tjoin h)$ and hence $f \tperp (g \tjoin h)$.
  \end{itemize}

\end{proof}
% subsection tensors_for_disjointness (end)

% section disjointness_in_an_inverse_category (end)
\section{Inverse sum categories} % (fold)
\label{sec:inverse_sum_categories}

\subsection{Inverse sums} % (fold)
\label{sub:inverse_sums}

\begin{definition}\label{def:inverse_sum}
  In an inverse category with disjoint joins, an object $X$ is the \emph{inverse sum} of $A$ and
  $B$ when there exist maps $i_0,\ i_1,\ \xa,\ \xb$ such that:
  \begin{enumerate}[{(}i{)}]
    \item $i_0$ and $i_1$ are monic;
    \item $i_0 : A \to X$, $i_1: B \to X$, $\xa: X \to A$ and $\xb: X \to B$.
    \item $\inv{i_0} = \xa$ and $\inv{i_1} = \xb$.
    \item $\inv{i_0}i_0 \perp \inv{i_1}i_1$ and $\inv{i_0}i_0 \djoin \inv{i_1}i_1 = 1_X$.
  \end{enumerate}
  $i_0$ and $i_1$ will be referred to as the \emph{injection} maps of the inverse sum.
\end{definition}

\begin{lemma}\label{lem:inverse_sum_is_unique}
  The inverse sum $X$ of $A$ and $B$ is unique up to isomorphism.
\end{lemma}
\begin{proof}
  Assume we have two inverse sums over $A$ and $B$:
  \[
    \xymatrix{
      A\ar[r]^{i_0} &X\ar@/^9pt/[l]^{x_0} \ar@/_9pt/[r]_{x_1} & B \ar[l]_{i_1}
    }
    \qquad  \text{ and  }\qquad
    \xymatrix{
      A\ar[r]^{\jay_0} &Y\ar@/^9pt/[l]^{y_0} \ar@/_9pt/[r]_{y_1} & B \ar[l]_{\jay_1}
    }.
  \]
  We will show that the map $x_0 \jay_0 \djoin x_1 \jay_1 : X \to Y$ is an isomorphism.

  Note by the fact that $i_1$ is monic, we may conclude from the definition that
  $0 = \rst{x_0 i_0 x_1}$ and therefore $0 = x_0 i_0 x_1$. Then, given that $x_0$ is the inverse
  of the monic $i_0$, we may calculate
  $0 = \rg{0} = \wrg{x_0 i_0 x_1} = \rst{\inv{x_1}\, \inv{i_0}\, i_0} =
  \rst{\inv{x_1}\, \inv{i_0}} = \wrg{i_0 x_1}$. From this we see $i_0 x_1 = 0$. Similarly, we have
  $i_1 x_0 = 0$, $j_0 y_1 =0$ and $j_1 y_0 = 0$.

  Next, by Lemma~\ref{lem:disjointness_various}, we know that $\rst{\xa}\cdperp\rst{\xb}$ as both
  $i_0$ and $i_1$ are monic. By the same lemma, $\rg{\jay_0} \cdperp \rg{\jay_1}$ as $y_0, y_1$
  are the inverses of monic maps.  Then, from \axiom{Dis}{7}, we have
  $x_0 \jay_0 \cdperp x_1 \jay_1$, hence we may form $x_0 \jay_0 \djoin x_1 \jay_1 : X \to Y$.

  Similarly, we may form the map $y_0 i_0 \djoin y_1 i_1 : Y \to X$. Computing their composition:
  \begin{align*}
    (x_0 \jay_0 \djoin x_1 \jay_1)(y_0 i_0 \djoin y_1 i_1)
      &= (x_0 \jay_0 (y_0 i_0 \djoin y_1 i_1))\djoin (x_1 \jay_1(y_0 i_0 \djoin y_1 i_1))\\
      &= x_0 \jay_0 y_0 i_0 \djoin x_0 \jay_0 y_1 i_1 \djoin x_1 \jay_1 y_0 i_0 \djoin x_1
        \jay_1 y_1 i_1 \\
      &= x_0\, 1\, i_0 \djoin x_0\, 0\, i_1 \djoin x_1\, 0\, i_0 \djoin x_1\, 1\, i_1\\
      &= x_0 i_0 \djoin x_1 i_1 = 1.
  \end{align*}
  Computing the other direction,
  \begin{align*}
    (y_0 i_0 \djoin y_1 i_1)(x_0 \jay_0 \djoin x_1 \jay_1)
      &= (y_0 i_0 (x_0 \jay_0 \djoin x_1 \jay_1) )\djoin (y_1 i_1(x_0 \jay_0 \djoin x_1 \jay_1))\\
      &= y_0 i_0 x_0 \jay_0 \djoin y_0 i_0 x_1 \jay_1 \djoin y_1 i_1 x_0 \jay_0 \djoin y_1 i_1 x_1 \jay_1\\
      &= y_0\, 1\, \jay_0 \djoin y_0\, 0\, \jay_1 \djoin y_1\, 0\, \jay_0 \djoin y_1\, 1\, \jay_1\\
      &= y_0 \jay_0 \djoin y_1 \jay_1 = 1.\\
  \end{align*}
  This shows that the map between any two inverse sums is an isomorphism.


\end{proof}

\begin{definition}\label{def:inverse_sum_category}
  Suppose \X is an inverse category with with disjoint joins $\djoin$ based on a disjointness
  relation $\cdperp$ and a restriction zero. If every pair of objects has an inverse sum as in
  Definition~\ref{def:inverse_sum}, we call the category an \emph{inverse sum} category. For any
  two objects $A,B$ in \X, we write their inverse sum as $\invsum{A}{B}$.
\end{definition}

\begin{lemma}\label{lem:tensor_disjoint_sum_cats_are_inverse_sum_categories}
  Let \X be an inverse category with a restriction 0 and a disjoint sum tensor $\+$. Then $\X$ is
  an inverse sum category.
\end{lemma}
\begin{proof}
  We claim that setting $i_i = \cp{i}$ and $x_i = \Pi_i$ and setting $X = A\+B$ produces inverse
  sums in \X and show this satisfies the four conditions of Definition~\ref{def:inverse_sum}.
  \begin{enumerate}[{(}i{)}]
    \item From Lemma~\ref{lem:tensor_identities}, we know that $\cpa$ and $\cpb$ are monic maps.
    \item $\cpa : A \to A\+B$, $\cpb: B \to A\+B$, $\Pi_0: A\+B \to A$ and $\Pi_1: A\+B \to B$.
    \item $\inv{\cpa} = \Pi_0$ and $\inv{\cpb} = \Pi_1$.
    \item $\inv{i_0}i_0 = 1\+ 0 \tperp 0\+1 = \inv{i_1}i_1$ as
      $1\+0 \tjleft 0\+1 = (\inv{\upr}\+\inv{\upl})$ and
      $1\+0 \tjright 0\+1 = (\Pi_0\+\Pi_1)$. For their join,
      $(1\+0)\tjoin (0\+1) = (\inv{\upr}\+\inv{\upl})(\Pi_0\+\Pi_1) =
      \inv{\upr}\Pi_0 \+ \inv{\upl}\Pi_1 = 1\+1 = 1$.
  \end{enumerate}
\end{proof}

\begin{lemma}\label{lem:inverse_sums_have_identity_zero}
  If $A$ is an object in \X, an inverse sum category, then $A+0$ is isomorphic to $A$.
\end{lemma}
\begin{proof}
  We write the inverse sum diagram:
  \[
    \xymatrix{
      A\ar[r]^{1} &A\ar@/^9pt/[l]^{1} \ar@/_9pt/[r]_{0} & 0 \ar[l]_{0}.
    }
  \]
\end{proof}
\begin{lemma}\label{lem:functor_preserving_joins_preserves_inverse_sums}
  Suppose $\X$ is an inverse sum category and $\Y$ is an inverse category with a restriction zero.
  Further, suppose $F:\X \to \Y$ is a restriction functor which preserves disjoint joins. Then,
  $F$ preserves inverse sums.
\end{lemma}
\begin{proof}
  In \X, consider the inverse sum over $A$ and $B$,
  \[
    \xymatrix{
      A\ar[r]^{i_0} &X\ar@/^9pt/[l]^{x_0} \ar@/_9pt/[r]_{x_1} & B \ar[l]_{i_1}
    }.
  \]
  The functor $F$ maps this as follows:
  \[
    \xymatrix@R+10pt@C+10pt{
      F(A)\ar[r]^{F(i_0)} &F(X)\ar@/^13pt/[l]^{F(x_0)} \ar@/_13pt/[r]_{F(x_1)} & F(B)\ar[l]_{F(i_1)}
    }.
  \]
  As $F$ is a restriction functor, we immediately have $F(x_0) = F(\inv{i_0}) = \inv{F(i_0)}$ and
  $F(x_1) = \inv{F(i_1)}$. Since $F$ preserves the disjoint join, we also have
  $\inv{F(i_0)}F(i_0) \cdperp\inv{F(i_1)}F(i_1)$ and
  $\inv{F(i_0)}F(i_0) \djoin \inv{F(i_1)}F(i_1) = 1$.

  Finally, as $F$ is a restriction functor, it preserves monics, hence $F(i_0)$ and $F(i_1)$ are
  both monic and therefore $F(X)$ is the inverse sum of $F(A)$ and $F(B)$.

\end{proof}
\begin{lemma}\label{lem:inverse_sum_maps_are_perp}
  Given \X an inverse sum category and maps $f:A \to C$ and $g:B\to D$ in \X. Then
  $\inv{i_0} f i_0 \cdperp \inv{i_1} g i_1 : \invsum{A}{B}\to \invsum{A}{B}$.
\end{lemma}
\begin{proof}
  Note that $\rst{\inv{i_0} f i_0} = \rst{\inv{i_0} f} \le \rst{\inv{i_0}}$ and similarly
  $\rst{\inv{i_1} g i_1} \le \rst{\inv{i_1}}$. Then, by \axiom{Dis}{3}, we have
  $\rst{\inv{i_0} f i_0} \cdperp \rst{\inv{i_1} g i_1}$.

  Then, as $\wrg{\inv{i_0} f i_0} \le \wrg{i_0}$ and $\wrg{\inv{i_1} g i_1} \le \wrg{i_1}$, this
  means we have $\wrg{\inv{i_0} f i_0} \cdperp \wrg{\inv{i_1} g i_1}$ and by
  Lemma~\ref{lem:disjointness_various}, this means $\inv{i_0} f i_0\cdperp \inv{i_1} g i_1$.
\end{proof}


\begin{lemma}\label{lem:inverse_sums_have_unique_maps_out}
  Given $\X$ is an inverse sum category. Denote the inverse sum of objects $A,B$ of $\X$ by
  $\invsum{A}{B}$. Then for objects $A,B$ and $X$ with maps $f:A\to X$ and $g:B\to X$ such that
  $\rg{f} \cdperp \rg{g}$, there exists a unique map $h$ making the following diagram commute.
  \[
    \xymatrix@R+10pt@C+28pt {
      A \ar[dr]^{f} \ar[d]_{i_0}
        \\
      \invsum{A}{B} \ar@{.>}[r]^{h}
        & X\text{.}\\
      B \ar[ur]_{g} \ar[u]_{i_1}
    }
  \]
  We use the notation $f\rgp g$ for the unique map $h$.
\end{lemma}
\begin{proof}
  As $\rg{f} \cdperp \rg{g}$ and $\rst{\inv{i_0}} \cdperp \rst{\inv{i_1}}$ we may form the map $h'
  = \inv{i_0}f \djoin \inv{i_1}g$. By its construction, $h'$ is a map from $\invsum{A}{B}$ to $X$
  which makes the diagram commute. Suppose now that both maps $v$ and $w$ are such maps. Then we
  have
  \[
    (\inv{i_0}i_0) v = (\inv{i_0}i_0) w \quad\text{ and }\quad(\inv{i_1}i_1) v = (\inv{i_1}i_1) w.
  \]
  As $\inv{i_0}i_0 \cdperp \inv{i_1}i_1$, by Lemmas~\ref{lem:disjointness_various} and
  \ref{lem:join_is_associative_and_commutative_monoid}, we know that $(\inv{i_0}i_0) v \cdperp
  (\inv{i_1}i_1) v$ and $(\inv{i_0}i_0) w \cdperp (\inv{i_1}i_1) w$ allowing us to form their
  respective disjoint joins. As the disjoint joins of equal maps remains equal, we have
  \begin{align*}
    (\inv{i_0}i_0) v \djoin (\inv{i_1}i_1) v &=     (\inv{i_0}i_0) w \djoin (\inv{i_1}i_1) w \\
    (\inv{i_0}i_0 \djoin \inv{i_1}i_1 )v &=     (\inv{i_0}i_0 \djoin \inv{i_1}i_1) w \\
    (1 )v &=     (1) w \\
        v &=  w.
  \end{align*}
\end{proof}

\begin{corollary}\label{cor:inverse_sums_have_unique_maps_in}
  Given $\X$ is an inverse sum category. Then for objects $A,B$ and $X$ with maps $f:X\to A$
  and $g:X\to B$ such that
  $\rst{f} \cdperp \rst{g}$, there exists a unique map $h$ making the following diagram commute.
  \[
    \xymatrix@R+10pt@C+28pt {
      &A \ar[d]_{i_0}
        \\
      X   \ar[dr]_{g} \ar[ur]^{f}  \ar@{.>}[r]^{h} & \invsum{A}{B}\\
      & B \ar[u]_{i_1}\text{.}
    }
  \]
  We use the notation $f\rstp g$ for the unique map $h$.
\end{corollary}
\begin{proof}
  This is simply the dual of Lemma~\ref{lem:inverse_sums_have_unique_maps_out}. The unique map $h$
  in this case is $f i_0 \djoin g i_1$.
\end{proof}

\begin{corollary}\label{cor:inverse_sums_have_unique_maps}
  Suppose $\X$ is an inverse sum category. Then for objects $A,B,C$ and $D$ with maps $f:A\to C$
  and $g:B\to D$, there exists a unique map $h$ making the following diagram commute.
  \[
    \xymatrix@R+10pt@C+28pt {
      A \ar[r]^{f} \ar[d]_{i_0}
        & C \ar[d]_{i_0} \\
      \invsum{A}{B} \ar@{.>}[r]^{h}
        & \invsum{C}{D}\\
      B \ar[r]_{g} \ar[u]_{i_1}
        & D \ar[u]_{i_1}\text{.}
    }
  \]
  We use the notation $\invsum{f}{g}$ for the map $h$.
\end{corollary}
\begin{proof}
  This follows directly from Lemma~\ref{lem:inverse_sums_have_unique_maps_out} by setting
  $X = \invsum{C}{D}$. The unique map in this case is $\inv{i_0} f i_0 \djoin \inv{i_1} g i_1$.
\end{proof}
\begin{lemma}\label{lem:functor_preserving_inverse_sums_preserves_joins}
  Suppose $\X$ and $\Y$ are inverse sum categories and $F:\X \to \Y$ is a restriction functor which
  preserves inverse sums. Then, $F$ preserves disjoint joins.
\end{lemma}
\begin{proof}
  By stating that $F$ preserves the inverse sum, we mean it preserves diagrams derived via the
  properties of the inverse sum, and specifically, it will preserve the diagrams of
  Lemma~\ref{lem:inverse_sums_have_unique_maps_out} and
  Corollaries~\ref{cor:inverse_sums_have_unique_maps_in} and
  \ref{cor:inverse_sums_have_unique_maps}.

  Suppose we are given $f, g: A \to B$ with $f \cdperp g$. In the inverse sum category, we know that
  $f \djoin g = (\rst{f}i_0 \djoin \rst{g} i_1) (\inv{i_0} f i_0 \djoin \inv{i_1} g i_1)
  (\inv{i_0} \rg{f} \djoin \inv{i_1} \rg{g})$, as this follows by:
  \begin{enumerate}
    \item Apply Corollary~\ref{cor:inverse_sums_have_unique_maps_in} to $\rst{f}$ and $\rst{g}$;
    \item then apply Corollary~\ref{cor:inverse_sums_have_unique_maps} to $f, g$;
    \item finally apply Lemma~\ref{lem:inverse_sums_have_unique_maps_out} to $\rg{f}, \rg{g}$.
  \end{enumerate}

  In the notation above, we have that $f\djoin g =
  (\rst{f} \rstp \rst{g})(\invsum{f}{g})(\rg{f} \rgp \rg{g})$. This gives us, as $F$ preserves
  the inverse sum:
  \begin{align*}
    F(f\djoin g) &= F(\rst{f}\rstp \rst{g})F(\invsum{f}{g})F(\rg{f} \rgp \rg{g})\\
    &=(F(\rst{f})\rstp F(\rst{g}))(\invsum{F(f)}{F(g)})(F(\rg{f}) \rgp F(\rg{g}))\\
    &=(\rst{F(f)}\rstp \rst{F(g)})(\invsum{F(f)}{F(g)})(\wrg{F(f)} \rgp \wrg{F(g)})\\
    &=F(f) \djoin F(g).
  \end{align*}

  The last line in the above is due to \Y being an inverse sum category as well.

\end{proof}


% subsection inverse_sums (end)

\subsection{Inverse sum tensor} % (fold)
\subsubsection{Inverse sum tensor definitions} % (fold)
\label{ssub:inverse_sum_tensor_definitions}
\label{sub:inverse_sum_tensor}
\begin{definition}\label{def:inverse_sum_tensor}
  An \emph{inverse sum tensor} in an inverse category \X with disjoint joins $\djoin$ based
  on a disjointness relation $\cdperp$ and a restriction zero
  is given by a tensor combined with two restriction monics, $\cpa$ and $\cpb$. The data for the
  tensor is:
 \begin{align*}
    \_ \+ \_ &: \X \times \X \to \X\ \ \text{(a restriction functor preserving disjoint joins)}\\
    0 &: \boldsymbol{1}\to \X \\
    \upl &: 0 \+ A \to A\\
    \upr &: A \+ 0 \to A\\
    a_\+ &: (A \+ B) \+ C \to A \+ (B \+ C) \\
    c_\+ &: A \+ B \to B \+ A\\
    \cpa &: A \to A \+ B\\
    \cpb &: B \to A \+ B
  \end{align*}
  where $\upl, \upr, a_\+, c_\+$ are all isomorphisms and the standard symmetric monoidal
  equations and coherence diagrams hold. The unit of the tensor, $0: \boldsymbol{1}\to \X$, is the
  restriction zero of the category. We specifically note that preserving disjoint joins means the
  tensor obeys the following two equations:
  \begin{align}
    f \cdperp g, \ h \cdperp k&\text{ implies } f\+ h \cdperp g \+ k
    \label{eq:invsum_preserve_perp}\\
    f \cdperp g, \ h \cdperp k &\text{ implies } (f\djoin g)\+(h\djoin k) = (f \+h )\djoin (g\+k).
    \label{eq:invsum_preserve_join}
  \end{align}

\end{definition}



% subsubsection inverse_sum_tensor_definitions (end)

\subsubsection{Disjointness and the inverse sum tensor} % (fold)
\label{ssub:disjointness_and_the_inverse_sum_tensor}
\begin{lemma}\label{lem:a_disjoint_sum_tensor_is_an_inverse_sum_tensor}
  Given an inverse category \X with a disjoint sum tensor $\+$ as in
  Definition~\ref{def:disjoint_sum_tensor}, then $\+$ is an inverse sum tensor.
\end{lemma}
\begin{proof}
  From the data of the disjoint sum tensor, the only thing remaining to show is that the tensor
  preserves the disjoint join.

  Suppose we have $f \tperp g$ and $\ h \tperp k$.
  From Lemma~\ref{lem:properties_of_tjleft_and_tjright}, item~\ref{lemitem:l_r_preserve_tensor}, we
  know both $(f\+h)\tjleft(g\+k)$ and $(f\+h)\tjright(g\+k)$ exist, hence $(f\+h)\tperp(g\+k)$.
  This shows condition~\eqref{eq:invsum_preserve_perp}.

  For condition~\eqref{eq:invsum_preserve_join}, we compute from the right hand side:
  \begin{align*}
    (f\+h)\tjoin(g\+k) &= (f\+h)\tjleft (g\+k) \wrg{(f\+h)}\tjright \wrg{(g\+k)}\\
    &= \left( (f\tjleft g)\+ (h \tjleft k) \right)
       \left( (\rg{f}\+\rg{h}) \tjright (\rg{g}\+\rg{k}) \right) \\
    &= \left( (f\tjleft g)\+ (h \tjleft k) \right)
       \left( (\rg{f}\tjright \rg{g}) \+ (\rg{h}\tjright\rg{k}) \right) \\
    &= \left( (f\tjleft g)(\rg{f}\tjright \rg{g})\right) \+
       \left((h \tjleft k) (\rg{h}\tjright\rg{k}) \right) \\
    &=  (f\tjoin g)\+(h \tjoin k).
  \end{align*}
  The second and third lines above again use Lemma~\ref{lem:properties_of_tjleft_and_tjright},
  item~\ref{lemitem:l_r_preserve_tensor}.

\end{proof}
% subsubsection disjointness_and_the_inverse_sum_tensor (end)

\subsubsection{Inverse sums and the inverse sum tensor} % (fold)
\label{ssub:inverse_sums_and_the_inverse_sum_tensor}

\begin{lemma}\label{lem:an_inverse_sum_tensor_gives_inverse_sums}
  If $\+$ is an inverse sum tensor in the inverse category \X, then $A\+B \cong A+B$, an inverse
  sum of $A$ and $B$.
\end{lemma}
\begin{proof}
  As $\+$ is a restriction functor from $\X\times\X$ to $\X$, this actually follows immediately
  from Lemma~\ref{lem:functor_preserving_joins_preserves_inverse_sums}. It may also be proven
  directly:

  Draw the inverse sum diagram:
  \[
    \xymatrix @C+60pt{
      A\ar[r]^{i_0=\inv{\upr} (1\+0)}
        & A\+B \ar@/^15pt/[l]^{x_0=(1\+0)\upr} \ar@/_15pt/[r]_{x_1=(0\+1)\upl}
          & B \ar[l]_{i_1=\inv{\upl} (0\+1)}
    }.
  \]
  Therefore, we have $\inv{i_0}i_0 =
  (1\+0)\upr \inv{\upr} (1\+0) = (1\+0)(1\+0) = (1\+0)$. Similarly, $\inv{i_1} i_1 = (0\+1)$.
  Since $0\cdperp 1$, we have $\inv{i_0}i_0 \cdperp \inv{i_1} i_1$.

  By the functorality of $\+$ and that it preserves disjoint joins, we have $(1\+0) \djoin (0 \+1)
  = (1\djoin 0 ) \+ (0\djoin 1) = 1 \+ 1 = 1_{A\+B}$. Hence $A\+ B$ is an inverse sum of $A$ and
  $B$ and by Lemma~\ref{lem:inverse_sum_is_unique} it is isomorphic to $A+B$.
\end{proof}

Conversely, we can show that given a tensor which produces inverse sums, that tensor will
be an inverse sum tensor.

\begin{lemma}\label{lem:inverse_sums_give_inverse_sum_tensors}
  Given an inverse category $\X$ with restriction zero, a disjointness relation $\cdperp$, a
  disjoint join $\djoin$ and a symmetric monoidal tensor $\+$, with natural restriction monics
  $\cpa: A\to A\+B$ and $\cpb: B \to A\+B$ such that $A\+B$ is an inverse sum under $\cpa$ and
  $\cpb$, then when $f,g :A \to B$ and $h,k:C\to D$ with $f \cdperp g$ and $h \cdperp k$, then $f
  \+h \cdperp g\+k$ and $(f\+h)\djoin (g\+k) = (f\djoin g)\+ (h\djoin k)$.
\end{lemma}
\begin{proof}
  Similarly, this follows immediately from
  Lemma~\ref{lem:functor_preserving_inverse_sums_preserves_joins}. We show it directly below:

  \[
    \xymatrix@R+10pt@C+28pt {
      A \ar[r]^{\rst{f}} \ar[d]_{\cpa} \ar@/^20pt/[rr]^f
        & A \ar[d]_{\cpa} \ar[r]^{f\djoin g} & B \ar[d]_{\cpa}\\
      A\+C \ar[r]^{\rst{f}\+\rst{h}}_{(1)}
        & A\+C  \ar[r]^{f\djoin g \+ h\djoin k}_{(2)} & B \+D \\
      C \ar[r]_{\rst{h}} \ar[u]_{\cpb} \ar@/_20pt/[rr]_h
        & C \ar[u]_{\cpb} \ar[r]_{h\djoin k} & D \ar[u]_{\cpb}
    }
  \]
  Consider $\inv{\cpa} \rst{f}\cpa$. As this is idempotent and we are in an inverse category, we
  know that $\inv{\cpa} \rst{f}\cpa = \rst{\inv{\cpa} \rst{f}\cpa} = \rst{\inv{\cpa}\rst{f}} =
  \wrg{\rst{f}\cpa}$. Similarly, $\inv{\cpb}\rst{h}\cpb = \wrg{\rst{h}\cpb}$. By \axiom{Dis}{5} and
  \axiom{Dis}{6}, we know that $\wrg{\rst{f}\cpa} \cdperp \wrg{\rst{g}\cpa}$ and $\wrg{\rst{h}\cpb}
  \cdperp \wrg{\rst{k}\cpb}$. Additionally, as shown in the proof of
  Lemma~\ref{lem:inverse_sum_is_unique}, we know $\wrg{\cpa}\cdperp \wrg{\cpb}$. Hence, by
  \axiom{Dis}{3}, we have $\wrg{\rst{x}\cpa} \cdperp \wrg{\rst{y}\cpb}$ for any maps $x,y$.

  Hence, we can form the map $\wrg{\rst{f}\cpa} \djoin \wrg{\rst{h}\cpb}$. Referring to the
  commutative diagram above, by Corollary~\ref{cor:inverse_sums_have_unique_maps} there is an unique
  map at location $(1)$ which makes the diagram commute --- currently given as
  $\rst{f}\+\rst{h}$. But, the map $\wrg{\rst{f}\cpa} \djoin \wrg{\rst{h}\cpb}$ also satisfies
  this, hence we have $\wrg{\rst{f}\cpa} \djoin \wrg{\rst{h}\cpb} = \rst{f}\+\rst{h}$. Similarly,
  $\wrg{\rst{g}\cpa} \djoin \wrg{\rst{k}\cpb} = \rst{g}\+\rst{k}$. But, by
  Lemma~\ref{lem:disjoint_joins_only_when_all_disjoint}, this means $\rst{f\+h} \cdperp \rst{g\+k}$.

  Using a similar argument based on the diagram

  \[
    \xymatrix@R+10pt@C+28pt {
      A \ar[r]^{f\djoin g} \ar[d]_{\cpa} \ar@/^20pt/[rr]^f
        & B \ar[r]^{f} \ar[d]_{\cpa}  & B \ar[d]_{\cpa}\\
      A\+C  \ar[r]^{f\djoin g \+ h\djoin k}_{(2)}
        & B\+D \ar[r]^{\rg{f}\+\rg{h}}_{(3)}   & B \+D \\
      C  \ar[r]_{h\djoin k} \ar[u]_{\cpb} \ar@/_20pt/[rr]_h
        & D\ar[r]_{\rg{h}} \ar[u]_{\cpb}  & D \ar[u]_{\cpb}
    }
  \]
  we can show $\wrg{f\+h} \cdperp \wrg{g\+k}$ and therefore $f\+h \cdperp g\+k$.

  This allows us to form the map $(f\+h) \djoin (g\+k)$. Once again, as the objects are inverse
  sums, the map at $(2)$ is unique.
  However, we see that both $f\djoin g \+ h\djoin k$ and $(f\+h) \djoin (g\+k)$ fulfill this
  requirement and hence they are equal.
\end{proof}


\begin{definition}\label{def:inverse_sum_tensor_category}
  An inverse category $\X$ with restriction zero, a disjointness relation $\cdperp$, a disjoint
  join $\djoin$ and an inverse sum tensor $\+$ is called an \emph{inverse sum tensor category}.
\end{definition}

\begin{corollary}\label{cor:inverse_sum_of_maps_is_disjoint_join}
  In an inverse sum tensor category, $f \+g$ is given by $\inv{i_0} f i_0 \djoin \inv{i_1} g i_1$.
\end{corollary}
\begin{proof}
  Recall that in the proof of Lemma~\ref{lem:inverse_sum_is_unique} that we showed $\rst{\inv{i_0}}
  \cdperp \rst{\inv{i_1}}$ and $\wrg{i_0}\perp \wrg{i_1}$. Hence, by \axiom{Dis}{7}, we know that
  $\inv{i_0} f i_0 \perp \inv{i_1} g i_1$ and we can therefore form the disjoint join.
\end{proof}


% subsubsection inverse_sums_and_the_inverse_sum_tensor (end)

\subsubsection{Matrices} % (fold)
\label{ssub:matrix_representation}
In this section, we will show that when given an inverse sum category \X, one can define a
type of matrix category based on $\X$. We will call this category $iMat(\X)$. Furthermore, we will
show that $iMat(\X)$ has is an inverse category and that $\X$ embeds within this category.

\begin{definition}\label{def:inverse_sum_matrix}
  A matrix of maps $[f_{i j}]$ in an inverse sum category \X which satisfy the condition:
  \begin{equation}
    \text{For each }i\text{, whenever }j \ne k\text{, then }f_{i j} \cdperp f_{i k}
    \label{eq:inverse_sum_matrix}
  \end{equation}
  is called an \emph{inverse sum matrix}.

  In the following we will use the notation $\cp{i}$ for the $i^{\text{th}}$ injection map of the
  inverse sum.

  Further, we define a monoid $\diamond$ of any two such matrices by
  $[h_{i k}] = [f_{i j}] \diamond [g_{j k}]$ where the element $h_{i k}$ of the matrix is given by:
  \[
    h_{i k} = \bigsqcup_j f_{i j} \cp{i} g_{j k}
  \]
  noting again that composition of maps is diagrammatic order. Of course, this is only defined when
  the domain of $g_{\jay k}$ is the inverse sum of some objects in $\X$ and the range of
  $f_{i \jay}$.

  We also define $\diamond$ for an inverse sum matrix $[f_{i j}]$ and a row (list) of objects
  $[A_i]$ in $\X$ by $[B_j] = [A_i]\diamond[f_{i\jay}]$ is given by
  \[
    B_j = \Sigma_i f_{i \jay} A_i
  \]
  where $\Sigma$ stands for the inverse sum operation of the category $\X$. Note this requires
  choosing a specific inverse sum for each pair of objects from the set of isomorphic inverse sums.
\end{definition}

\begin{definition}\label{def:inverse_matrix_category}
  Given an inverse sum category $\X$, we define $iMat(\X)$, the \emph{inverse matrix category} of
  \X as follows:
  \rcategory{Lists of the objects of $\X$.}{
    The inverse sum matrix $[f_{i\jay}]: [A_i] \to [B_j]$ where each individual map
    $f_{i\jay}:A_i \to B'_{i\jay}$ is a map in $\X$. The $B_j$ are the chosen inverse sums of the
    $B'_{i\jay}$. The result of $[f_{i\jay}] [A_i]$ is given by $[A_i]\diamond[f_{i\jay}]$.}{
    The inverse sum matrix $I$.}{
    Given $[f_{i\jay}]: [A_i] \to [B_\jay]$ and $[g_{j k}] : [B_\jay] \to [C_k]$, then
    $[f_{i\jay}][g_{j k}]: [A_i] \to [C_k]$ is defined as $[g_{j k}] \diamond [f_{i\jay}]$.}{
    We set $\rst{[f_{i\jay}]}$ to be $[f'_{i\jay}]$ where $f'_{i\jay} = 0$ when $i\neq \jay$
    and $f'_{i i} = \sqcup_\jay \rst{f_{i\jay}}$.}
  We will use the notation $\delta[d_1,d_2,\ldots,d_n]$ for a diagonal $n\times n$ matrix with
  entries along the diagonal of $[d_1,d_2,\ldots,d_n]$ and $\delta_j[d_j]$ for diagonal matrices
  where the $j,j$ entry is $d_j$.
  %TODO: Get the notation better
\end{definition}

\begin{lemma}\label{lem:inverse_matrix_category_is_a_restriction_category}
  When $\X$ is an inverse sum category, $iMat(\X)$ is a restriction category.
\end{lemma}
\begin{proof}
  We need to show the following:
  \begin{itemize}
    \item Composition is well defined and associative;
    \item The restriction is well defined.
  \end{itemize}
  Composition is well defined: Consider $[h_{i k}] = [f_{i j}] [g_{j k}] (= [g_{j k}] \diamond
  [f_{i\jay}])$ where
  \[
    [f_{i j}]: [A_1,\ldots, A_n] \to [B_1,\ldots, B_m] \text{ and }
    [g_{j k}]: [B_1,\ldots, B_m] \to [C_1,\ldots, C_\ell].
  \]
  By Definition~\ref{def:inverse_sum_matrix}, we know $h_{i k} = \bigsqcup_\jay f_{i \jay} g_{\jay
  k}$. Individually, for each $\jay$ we know the composition $f_{i \jay}\cp{i} g_{\jay k}$ is
  defined and is from $A_i$ to $C'_{\jay k}$. By the the stability and universality of $\perp$, we
  know $h_{i k}$ exists and by the definition of $\djoin$, we have each $h_{i k}: A_i \to C'_{\jay
  k}$ and hence composition is well-defined.

  Associativity of composition. We have
  \begin{align*}
    ([f_{i j}] [g_{j k}]) [h_{k \ell}]
        &= \left[(\bigsqcup_j f_{i j} \cp{i} g_{j k})\right] [h_{k \ell}]\\
        &=\left[ \bigsqcup_k (\bigsqcup_j f_{i j} \cp{i} g_{j k}) \cp{j} h_{k \ell}\right]\\
        &=\left[ \bigsqcup_j f_{i j} \cp{i} (\bigsqcup_k g_{j k}\cp{j} h_{k \ell})\right]\\
        &=[ f_{i j} ]  ([g_{j k}] [h_{k \ell}])
  \end{align*}

  For the restriction axioms:
  \begin{align*}
    \rone \quad& \rst{[f_{i\jay}]}[f_{i\jay}] =
      \begin{bmatrix}
        (\sqcup_j \rst{f_{1\jay}}) f_{11} & \cdots & (\sqcup_\jay \rst{f_{1 \jay}}) f_{1n}\\
        & \vdots & \\
        (\sqcup_\jay \rst{f_{m\jay}}) f_{m1} & \cdots & (\sqcup_\jay \rst{f_{m\jay}}) f_{m n}
      \end{bmatrix} = [f_{i\jay}].\\
  \end{align*}
  \begin{align*}
    \rtwo \quad & \rst{[f_{i\jay}]} \rst{g_{i\jay}} =  \rst{g_{i\jay}} \rst{[f_{i\jay}]}
      \text{ as diagonal matrices commute and }\djoin\text{ is also commutative.}\\
  \end{align*}
  \begin{align*}
    \rthree \quad & \rst{\rst{[f_i\jay]}[g_{\jay k}]} =
      \rst{\delta[\sqcup_\jay \rst{f_{1\jay}},\ldots,\sqcup_\jay \rst{f_{n \jay}}] [g_{\jay k}]}
      \\[10pt]
      & = \rst{\begin{bmatrix}
        \sqcup_\jay \rst{f_{1\jay}} g_{11} & \ldots & \sqcup_\jay \rst{f_{1 \jay}} g_{1 k}\\
        & \vdots & \\
        \sqcup_\jay \rst{f_{n\jay}} g_{n 1} & \ldots & \sqcup_\jay \rst{f_{n \jay}} g_{n k}\\
      \end{bmatrix}} \\[10pt]
      & = \delta[\sqcup_k (\rst{\sqcup_\jay (\rst{f_{1\jay}} g_{1k})}), \ldots,
                 \sqcup_k (\rst{\sqcup_\jay(\rst{f_{n\jay})} g_{n k}})]\\
      & = \delta[\sqcup_k (\sqcup_\jay(\rst{f_{1\jay}}) \rst{g_{1k}}), \ldots,
                 \sqcup_k (\sqcup_\jay(\rst{f_{n\jay}}) \rst{g_{n k}})]\\
      & = \delta[(\sqcup_\jay(\rst{f_{1\jay}}) \sqcup_k  \rst{g_{1k}}), \ldots,
                 (\sqcup_\jay(\rst{f_{n\jay}}) \sqcup_k  \rst{g_{n k}})]\\
      & = \rst{[f_i\jay]}\rst{[g_{\jay k}]}\\
  \end{align*}
  \begin{align*}
      \rfour \quad & [f_{i\jay}]\rst{[g_{\jay k}]} = [f_{i\jay}]\delta_\jay[\sqcup_k \rst{g_{\jay k}}]
        \\[10pt]
      & = \begin{bmatrix}
        f_{11}\sqcup_k \rst{g_{1 k}} &\ldots & f_{1n}\sqcup_k \rst{g_{n k}} \\
        & \vdots & \\
        f_{m1}\sqcup_k \rst{g_{1 k}} &\ldots & f_{m n}\sqcup_k \rst{g_{n k}}
      \end{bmatrix}\\[10pt]
      & = \begin{bmatrix}
        \sqcup_k f_{11}\rst{g_{1 k}} &\ldots & \sqcup_k f_{1n}\rst{g_{n k}} \\
        & \vdots & \\
        \sqcup_k f_{m1}\rst{g_{1 k}} &\ldots & \sqcup_k f_{m n}\rst{g_{n k}}
      \end{bmatrix}\\[10pt]
      & = \begin{bmatrix}
        \sqcup_k \rst{f_{11} g_{1 k}} f_{11} &\ldots & \sqcup_k \rst{f_{1n} g_{n k}} f_{1n}\\
        & \vdots & \\
        \sqcup_k \rst{f_{m1} g_{1 k}}f_{m1} &\ldots & \sqcup_k \rst{f_{m n} g_{n k}}f_{m n}
      \end{bmatrix}\\[10pt]
      & = \begin{bmatrix}
        \sqcup_\jay\sqcup_k \rst{f_{1\jay} g_{\jay k}} f_{11} &\ldots &
          \sqcup_\jay\sqcup_k \rst{f_{1\jay} g_{\jay k}} f_{1n}\\
        & \vdots & \\
        \sqcup_\jay\sqcup_k \rst{f_{m\jay} g_{\jay k}}f_{m1} &\ldots &
          \sqcup_\jay\sqcup_k \rst{f_{m \jay} g_{\jay k}}f_{m n}
      \end{bmatrix}\\[10pt]
      &= \rst{[f_{i\jay}][g_{\jay k}]} [f_{i\jay}].
  \end{align*}
\end{proof}

\begin{lemma}\label{lem:inverse_sum_category_maps_are_matrices}
  In an inverse sum tensor category, any map $f:A\+B \to C\+D$ may be represented in a matrix form.
  Composition of maps may be computed by multiplication of the matrices, with composition taking
  the place of base level multiplication and $djoin$ the place of addition.
\end{lemma}
\begin{proof}
  Recall from Lemma~\ref{lem:an_inverse_sum_tensor_gives_inverse_sums} that $A\+B$ and $C\+D$ are
  inverse sums. Referencing Definition~\ref{def:inverse_sum}, define $e_0 = \inv{i_0}i_0$ and $e_1
  = \inv{i_1}i_1$ and recall that $e_0 \cdperp e_1$, $e_0 \djoin e_1 = 1$. Then given a function
  $f:A\+B \to C\+D$ define
  \[ f_M =
     \begin{bmatrix}
       e_0 f e_0 & e_0 f e_1 \\
       e_1 f e_0 & e_1 f e_1
      \end{bmatrix}.
  \]
  Note first that since $e_0 \cdperp e_1$, the maps in the rows of $f_M$ are disjoint by the
  stability of the disjointness relation. Similarly, the maps in the columns are disjoint by
  universality. We have $e_0 f e_0 \djoin e_0 f e_1 = e_0 f$ and $e_1 f e_0 \djoin e_1 f e_1 = e_1
  f$. Each of these maps are disjoint by universality. Finally, $e_0 f \djoin e_1 f = (e_0 \djoin
  e_1) f = f$ and hence we may recover the initial map whenever we have a matrix of this from.
  We will call this computation the distinct join of $f_M$.

  Next, consider $f_M \times g_M$. As each $e_i$ is idempotent, this is
  \[ f_M \times g_M =
     \begin{bmatrix}
       e_0 f e_0 g e_0 \djoin e_0 f e_1 g e_0 & e_0 f e_0 g e_1 \djoin e_0 f e_1 g e_1 \\
       e_1 f e_0 g e_0 \djoin e_1 f e_1 g e_0 & e_1 f e_0 g e_1 \djoin e_1 f e_1 g e_1
      \end{bmatrix}
      =
     \begin{bmatrix}
       e_0 f g e_0  & e_0 f  g e_1  \\
       e_1 f  g e_0 & e_1  g f e_1
      \end{bmatrix}
  \]
  where the distinct joins are well defined due to the stability and universality of the join.
  We can see that the distinct join of $f_M \times g_M = f g$ and as such we have composition.

\end{proof}

In particular, we note that we may represent $f:A\to B$ by the matrix
\[
  \begin{bmatrix}
    1f1& 1f 0 \\ 0 f 1 & 0 f 0
  \end{bmatrix}
  =
  \begin{bmatrix}
    f& 0 \\ 0 & 0
  \end{bmatrix}
\]
as $A \cong A\+0$ and $B \cong B\+0$.

We now turn to examining a category of specialized matrices over an inverse sum category. In
general, the
matrix category $\imat{(\X)}$ will have objects that are lists of objects in $\X$,
$X=(X_1,\ldots,X_m)$. Maps between lists will be matrices $[f_{i j}]: (X_1,\ldots,X_m) \to
(Y_1,\ldots,Y_n)$. We will only consider maps whose matrices have disjoint rows, i.e., if
$[f_{i j}]$ is a matrix, it must have $f_{i j} \cdperp f_{i k}$ for all $i$ whenever $j \ne k$.




\begin{lemma}\label{lem:inverse_sums_means_mat_x_has_coproducts}
  If $\X$ is an inverse category with an inverse sum tensor, $\imat{(\X)}$, the category of
  inverse sum matrices over \X with composition as in Definition~\ref{def:inverse_sum_matrix},
  has sums.
\end{lemma}
\begin{proof}


\end{proof}



% subsubsection matrix_representation (end)







% A unique decomposition category is a category where any
% \[
%   h:A\+B \to C\+D
% \]
% is uniquely determined by:
% \begin{alignat*}{2}
%   \cpa  h  \cpa^{(-1)} &: A \to C & \quad \cpa  h  \cpb^{(-1)} &: A \to D\\
%   \cpb  h  \cpa^{(-1)} &: B \to C & \quad \cpb  h  \cpb^{(-1)} &: B \to D
% \end{alignat*}
% i.e., is writable as the matrix:
% \[
%   \begin{bmatrix}
%     \cpa  h  \cpa^{(-1)} & \cpa  h  \cpb^{(-1)}\\
%     \cpb  h  \cpa^{(-1)} &\quad \cpb  h  \cpb^{(-1)}
%   \end{bmatrix}:
%   A\+B \to C\+D.
% \]
% See \cite{haghverdi:2000phd}.
%
% The $0$ map in the category corresponds to the $0$-dimensional matrix,
% \[
%   0\xrightarrow{[]}B.
% \]
%
% Note also that as $\cpa$ and $\cpb$ are restriction maps, the following diagram commutes:
% \[
%   \xymatrix @C+10pt{
%     A \ar@{=}[d] \ar[r]^{\cpa} & A\+B \\
%     A \ar[r]_{\cpa} & A\+0 \ar[u]_{1+0}
%   }
% \]
%
% The codiagonal does not exist, except at the $0$ object. To see this, recall this is an inverse
% category and the codiagonal is a total map. Therefore, given $\nabla: A \+ A \to A$, then
% $\nabla^{(-1)} : A \to A \+ A$ and $\nabla \nabla^{(-1)} = 1_{A\+A}, \nabla^{(-1)} \nabla = 1_{A}$.

% subsection inverse_sum_tensor (end)

% section inverse_sums_categories (end)
\section{Completing a distributive inverse category} % (fold)
\label{sec:completing_a_distributive_inverse_category}
\subsection{Distributive restriction categories} % (fold)
\label{sub:distributive_restriction_categories}


\begin{definition}
  A Cartesian restriction category with a restriction zero and coproducts is called
  \emph{distributive} when there is an isomorphism $\rho$ such that
  \[
    A\times (B+C) \xrightarrow{\rho} (A\times B) +(A\times C).
  \]
\end{definition}

In a distributive inverse category, we lack:
\[
  A  \xrightarrow{!} 1
\]
and
\[
  A+A \xrightarrow{\nabla} A.
\]
% subsection distributive_restriction_categories (end)

\subsection{Distributive inverse categories} % (fold)
\label{sub:distributive_inverse_categories}


A \emph{Distributive inverse category} has a sum tensor and product tensor as defined in
sub-section~\ref{sub:distributive_restriction_categories} where the product distributes over the sum
in the following manner:
\[
  (A\* B) \+ (A \* C) \xrightarrow{f} A \* (B \+ C)
\]
where $f$ is some total function. As we are in an inverse category and $f$ is total, we have $f
\inv{f} = 1$. This also implies $1 = \inv{f} f$ where
\[
  A \* (B \+ C) \xrightarrow{\inv{f}}(A\* B) \+ (A \* C).
\]

Suppose we have an inverse category $\X$ with two tensors, $\*$ and $\+$, as described above and we
have:
\[
  \xymatrix @C=80pt @R=50pt{
    A \* (B \+ C) \ar[r]^{\inv{[1\* \cpa, 1\* \cpb]}} \ar@{=}[dr]
    & (A\* B) \+ (A \* C) \ar[d]^{[1\* \cpa, 1\* \cpb]} \\
    &A \* (B \+ C)
  }
\]
%TODO: Find example where this is not identity

\begin{lemma}
  The category \X\ as defined above is an inverse distributive category.
\end{lemma}
\begin{proof}
  We will show that the required function, $f$ is $[1\* \cpa, 1\* \cpb]$.

  \begin{equation}
    (A\* B) \+ (A \* C) \xrightarrow{[1\* \cpa, 1\* \cpb]} A \* (B \+ C) \label{eq:dist1}
  \end{equation}
  The map in \ref{eq:dist1} is total. To see this, consider the unique decomposition of the
  restriction of the map.
  \begin{align*}
    \restr{[1\* \cpa, 1\* \cpb]} &=
    \begin{bmatrix}
      \restr{1\* \cpa} & 0 \\
      0 & \restr{1\* \cpb}
    \end{bmatrix}\\
    &=
    \begin{bmatrix}
      \restr{1}\* \restr{\cpa} & 0 \\
      0 & \restr{1}\* \restr{\cpb}
    \end{bmatrix}\\
    &=
    \begin{bmatrix}
      1\* 1 & 0 \\
      0 & 1\* 1
    \end{bmatrix}\\
    &=
    \begin{bmatrix}
      1 & 0 \\
      0 & 1
    \end{bmatrix}\\
    &=1
  \end{align*}
\end{proof}
% subsection distributive_inverse_categories (end)
% section completing_a_distributive_inverse_category (end)
