%!TEX root = /Users/gilesb/UofC/thesis/phd-thesis/phd-thesis.tex
\chapter{Conclusions and future work} % (fold)
\label{cha:conclusions_and_future_work}

This thesis has studied inverse categories \cite{cockett2002:restcategories1}, providing conditions
for a tensor to act like a product --- the inverse product --- and similarly for a second
tensor to behave like a sum --- the disjoint sum.

The following are the main results of this thesis.

\begin{enumerate}
\item \textbf{Inverse Categories}

We showed in Proposition~\ref{prop:an_inverse_category_with_products_is_a_restriction_preorder} that
an inverse category with restriction products is a restriction preorder. Similarly, in
Proposition~\ref{prop:inverse_category_with_coproducts_is_pre-order}, an inverse category with
coproducts is a preorder.

\item \textbf{Discrete Inverse Categories}

We introduced the inverse product, Definition~\ref{def:inverse_product} and showed that this
provided meets for the inverse category in
Proposition~\ref{prop:discrete_inverse_category_has_meets}. We showed that the inverse subcategory
of a discrete Cartesian restriction category is a Discrete Inverse Category in
Lemma~\ref{lem:inv_x_is_a_discrete_inverse_category}.

Given a discrete inverse category $\X$, we constructed a category $\Xt$, the Cartesian Completion of
\X with the same objects and maps being equivalence classes of maps of $\X$. We showed this was a
restriction category in Lemmas~\ref{lem:xt_is_a_category} and \ref{lem:xt_is_a_restriction_category}
and in fact was a discrete Cartesian restriction category as shown in
Theorem~\ref{thm:xt_is_a_discrete_crc_when_x_is_an_inverse_category}.

Finally, in
Theorem~\ref{thm:discrete_inverse_categories_are_equivalent_to_discrete_restriction_categories} we
provided an equivalence functor between the category of discrete inverse categories and the category
of discrete Cartesian restriction categories.

\item \textbf{Disjointness Relations and Disjoint Joins}

In Definition~\ref{def:disjointness_relation} we defined what a disjointness relation is in an
inverse category, followed by Definition~\ref{def:disjointness_in_open_x} which defined disjointness
on the restriction idempotents of the inverse
category. Theorem~\ref{thm:open_disjointness_is_disjointness} shows that these two definitions are
equivalent, allowing us to define disjointness in whichever way is most convenient.

Disjoint joins are introduced with Definition~\ref{def:disjoint_join}, providing an analogue to
joins in a restriction category.

In Chapter~\ref{chap:disjoint_sum_tensors} we explore the properties a symmetric monoidal tensor
on an inverse category requires in order to generate a disjointness relation and a disjoint
join. These are referred to as disjoint sum tensors.

\item \textbf{Disjoint Sum Categories}

Disjoint sum categories are given by inverse categories that have a disjoint join tensor. We define
$\imatx$, a matrix category over the disjoint sum category \X in
Definition~\ref{def:inverse_matrix_category} and show that it is an disjoint sum category in
Theorem~\ref{thm:imatx_is_an_disjoint_sum_category}. Furthermore, we show
that an disjoint sum category \X is equivalent to a $\imatx$ in
Proposition~\ref{pro:x_and_imatx_are_equivalent}.


\item \textbf{Distributive Inverse Categories}

We define distributive inverse categories in Definition~\ref{def:distributive_inverse_category} and
show that distributivity of the inverse product over the disjoint join is equivalent to distributing
over an disjoint sum tensor in Lemma~\ref{lem:distributive_means_distribute_over_join}.

Then, in Theorem~\ref{thm:x_tilde_has_coproducts_if_x_is_inverse_distributive_category} we show that
Cartesian Completion turns a disjoint sum tensor into a coproduct. From this we conclude that $\Xt$
is a distributive restriction category in
Corollary~\ref{cor:xt_is_a_distributive_restriction_category}.

\item \textbf{Linkage to quantum computation}

In Chapter~\ref{chap:commutative_frobenius_algebras}, we review Frobenius algebras and how they
arrive in models of quantum computation. We then show that category of commutative Frobenius
algebras, CFrob(\X) based on a symmetric monoidal category \X is actually a discrete inverse category in
Theorem~\ref{thm:cfrob_is_a_discrete_inverse_category}. Furthermore, when the symmetric monoidal
category \X is an additive tensor category with zero maps and biproducts, then CFrob(\X) possesses a
disjointness relation, given by  $f\perp g \iff \Delta(f\* g)\inv{\Delta} = 0$. We then showed that
the biproduct of maps gives a  disjoint join and the biproduct of objects gives a disjoint sum.

\item \textbf{Inverse Turing Categories and Inverse PCAs}

Inverse Turing categories and inverse partial combinatory algebras are defined in
Definition~\ref{def:inverse_turing_category} and
Definition~\ref{def:inverse_partial_combinatory_algebra} respectively. First, we show in
Lemma~\ref{lem:discrete_turing_category_inverses_make_inverse_turing_category} that if we have a
discrete Turing category $\T$, then $\Inv{\T}$, the inverse subcategory of $\T$ is an inverse Turing
category. Then we show that the Cartesian Completion of an inverse Turing category gives a Turing
category. In Proposition~\ref{prop:inverse-pca-iff-pca} we show that a discrete inverse category \X
has an inverse PCA if and only if \Xt has a PCA. Furthermore, we discuss the category of computable
function based on a PCA in a Cartesian restriction category and show the conditions required for
that to be a discrete Cartesian restriction category in Lemma~\ref{lem:comp_a_is_discrete_cart}.
\end{enumerate}

Thus, we have shown that discrete inverse categories provide a link between standard computing
models (i.e., via inverse Turing categories to Turing categories) and quantum computing (commutative
Frobenius algebras are discrete inverse categories with a disjoint join).

\section{Future directions}
\label{sec:whither-next}

The obvious next step is to apply these results to existing reversible languages such as
\textbf{Inv} \cite{muetal04:injreversible}, and Theseus
\cite{james2013isomorphic,james2012information}. In each case, there do not appear to be any issues
in doing this work, it simply requires careful attention to the language definitions. Note that our
examples about the category of partial injective functions done throughout this thesis provide the
basis for the semantics of \textbf{Inv}.

A more interesting direction, in my opinion, would be to further investigate the work on reversible
CCS as in \cite{danos2004reversible} and \cite{phillips2006operational}. This could be combined with
the work of Chakraborty on MPL \cite{chakraborty2014} to describe the communication of multiple
processes and generalize that to a series of reversible processes.

A third direction would be to formalize types in reversible languages using the results of this
thesis as a basis. The obvious correspondences are from product types to the inverse product and
between sum types and the disjoint sum. From there, one would study the trace and explore infinite
and recursive types. Note that some of this work would follow naturally from exploring the semantics
of Theseus from \cite{james2013isomorphic,james2012information}.

The connection to quantum computing, as exampled in
Chapter~\ref{chap:commutative_frobenius_algebras} would benefit from further investigation. Although
we were able to create a discrete inverse category based on commutative Frobenius algebras, we have
not yet investigated the effect of performing the Cartesian Completion on CFrob(\X). As well, we
note that the bulk of current literature on semantics of quantum computing is based on finite
dimensional Hilbert spaces. Our construction on Frobenius algebras is not limited to the finite
dimensional case, and as such, may be of use when considering infinite dimensions.

% chapter conclusions_and_future_work (end)

%%% Local Variables:
%%% mode: latex
%%% TeX-master: "../phd-thesis"
%%% End:
