%!TEX root = /Users/gilesb/UofC/thesis/phd-thesis/phd-thesis.tex
\chapter{Conclusions and future work} % (fold)
\label{cha:conclusions_and_future_work}

This thesis has studied inverse categories \cite{cockett2002:restcategories1}, providing conditions
for a tensor to act in a product like way --- the inverse product --- and similarly for a second
tensor to behave like a sum --- the inverse sum.

The following are the main results of this thesis.

\begin{enumerate}
\item \textbf{Inverse Categories}

We showed in Proposition~\ref{prop:an_inverse_category_with_products_is_a_restriction_preorder} that
an inverse category with restriction products is a restriction pre-order. Similarly, in
Proposition~\ref{prop:inverse_category_with_coproducts_is_pre-order}, an inverse category with
coproducts is a pre-order.

\item \textbf{Discrete Inverse Categories}

We introduced the inverse product, Definition~\ref{def:inverse_product} and showed that this
provided meets for the inverse category in
Proposition~\ref{prop:discrete_inverse_category_has_meets}. We showed that the inverse sub-category
of a discrete Cartesian restriction category is a Discrete Inverse Category in
Lemma~\ref{lem:inv_x_is_a_discrete_inverse_category}. Commutative Frobenius Algebras are given as an
example of a discrete inverse category.

We then provided an equivalence relation on the maps of a discrete inverse category \X in
Definition~\ref{defn:xequivalence}. This allowed us to introduce the category $\Xt$ constructed from
$\X$ by using the equivalence relation to generate more maps between each object in
Definition~\ref{def:xt}. In Lemmas~\ref{lem:xt_is_a_category} and
\ref{lem:xt_is_a_restriction_category} we showed $\Xt$ is a restriction category and in
Theorem~\ref{thm:xt_is_a_discrete_crc_when_x_is_an_inverse_category} we showed it is in fact a
discrete Cartesian restriction category.

Finally, in
Theorem~\ref{thm:discrete_inverse_categories_are_equivalent_to_discrete_restriction_categories} we
provided an equivalence functor between the categories of discrete inverse categories and discrete
Cartesian restriction categories.

\item \textbf{Disjointness Relations and Disjoint Joins}

In Definition~\ref{def:disjointness_relation} we defined what a disjointness relation is in an
inverse category, followed by Definition~\ref{def:disjointness_in_open_x} which defined disjointness
on the restriction idempotents of the inverse
category. Theorem~\ref{thm:open_disjointness_is_disjointness} shows that these two definitions are
equivalent, allowing us to define disjointness in whichever way is most convenient.

Disjoint joins are introduced with Definition~\ref{def:disjoint_join}, providing a correspondence to
joins in a restriction category.

In Section~\ref{sec:tensors_for_disjointness} we explore the properties a symmetric monoidal tensor
on an inverse category requires in order to generate a disjointness relation and a disjoint
join. These are referred to as disjoint join tensors.

We show that in the category of commutative Frobenius algebras that $\Delta(f\* g)\inv{Delta} = 0$
gives a disjointness relation.

\item \textbf{Inverse Sum Categories}

Inverse sum categories are given by inverse categories that have a disjoint join tensor. We define
$\imatx$, a matrix category over the inverse sum category \X in
Definition~\ref{def:inverse_matrix_category} and show that it is an inverse sum category in
Theorem~\ref{thm:imatx_is_an_inverse_sum_category}. Furthermore, we show
that an inverse sum category \X is equivalent to a $\imatx$ in
Proposition~\ref{pro:x_and_imatx_are_equivalent}.


\item \textbf{Distributive Inverse Categories}

We define distributive inverse categories in Definition~\ref{def:distributive_inverse_category} and
show that distributivity of the inverse product over the disjoint join is equivalent to distributing
over an inverse sum tensor in Lemma~\ref{lem:distributive_means_distribute_over_join}.

Then, in Theorem~\ref{thm:x_tilde_has_coproducts_if_x_is_inverse_distributive_category} we show that
the $\wtf$ construction of Section~\ref{sub:the_restriction_category_hypxt} lifts an inverse sum
tensor into a coproduct. From this we conclude that $\Xt$ is a distributive restriction category in
Corollary~\ref{cor:xt_is_a_distributive_restriction_category}.
\end{enumerate}

\section{Whither next?}
\label{sec:whither-next}

The obvious next step is to apply these results to existing reversible languages such as
\textbf{Inv} \cite{muetal04:injreversible}, and the pair $\Pi^0$, \cite{james2013isomorphic} and $\Pi$,
\cite{james2012information}. In each case, there do not appear to be any issues in doing this work,
it simply requires careful attention to the language definitions.

A more interesting direction, in my opinion would be to further investigate the work on reversible
CCS as in \cite{danos2004reversible} and \cite{phillips2006operational}. This could be combined with
the work of Chakraborty on MPL \cite{chakraborty2014} to describe the communication of multiple
processes and generalize that to a series of reversible processes.

A third direction would be to formalize types in reversible languages using the results of this
thesis as a basis. The obvious correspondences are from product types to the inverse product and
between sum types and the inverse sum. From there, one would study the trace and explore infinite
and recursive types. Note that some of this work would follow naturally from exploring the semantics
of $\Pi^0$ and $\Pi$ from \cite{james2013isomorphic,james2012information}.


% chapter conclusions_and_future_work (end)

%%% Local Variables:
%%% mode: latex
%%% TeX-master: "../phd-thesis"
%%% End:
