%!TEX root = /Users/gilesb/UofC/thesis/phd-thesis/phd-thesis.tex
\chapter{Conclusions and future work} % (fold)
\label{cha:conclusions_and_future_work}

This thesis has studied inverse categories \cite{cockett2002:restcategories1}, providing conditions
for a tensor to act like a product --- the inverse product --- and similarly for a second
tensor to behave like a coproduct --- the disjoint sum.

The following are the main results of this thesis.

\begin{enumerate}
\item \textbf{Restriction Products and Coproducts in Inverse Categories}

We showed in Proposition~\ref{prop:an_inverse_category_with_products_is_a_restriction_preorder} that
an inverse category with restriction products is a restriction preorder. Similarly, in
Proposition~\ref{prop:inverse_category_with_coproducts_is_pre-order}, an inverse category with
coproducts is a preorder.

\item \textbf{Discrete Inverse Categories}

We introduced the inverse product, Definition~\ref{def:inverse_product} and showed that this
provided meets for the inverse category in
Proposition~\ref{prop:discrete_inverse_category_has_meets}. We showed that the inverse subcategory
of a discrete Cartesian restriction category is a discrete inverse category in
Lemma~\ref{lem:inv_x_is_a_discrete_inverse_category}.

Given a discrete inverse category $\X$, we constructed a category $\Xt$, the Cartesian Completion of
\X with the same objects as \X and maps being equivalence classes of maps of $\X$. We showed this was a
restriction category in Lemmas~\ref{lem:xt_is_a_category} and \ref{lem:xt_is_a_restriction_category}
and in fact was a discrete Cartesian restriction category as shown in
Theorem~\ref{thm:xt_is_a_discrete_crc_when_x_is_an_inverse_category}.

Finally, in
Theorem~\ref{thm:discrete_inverse_categories_are_equivalent_to_discrete_restriction_categories} we
provided an equivalence between the category of discrete inverse categories and the category
of discrete Cartesian restriction categories. This result, together with its consequences, is the
main theoretical contribution of this thesis.

\item \textbf{Disjointness Relations and Disjoint Joins}

In Definition~\ref{def:disjointness_relation} we defined what a disjointness relation is in an
inverse category, followed by Definition~\ref{def:disjointness_in_open_x} which defined disjointness
on the restriction idempotents of the inverse
category. Theorem~\ref{thm:open_disjointness_is_disjointness} shows that these two definitions are
equivalent, allowing us to define disjointness in whichever way is most convenient.

Disjoint joins are introduced with Definition~\ref{def:disjoint_join}, providing an analogue to
joins in a restriction category.

\item \textbf{Disjoint Sums}

Disjoint sums in an inverse category are given in Definition~\ref{def:disjoint_sum} and we show that
an inverse category having all disjoint sums has a symmetric monoidal tensor in
Proposition~\ref{prop:enough_disjoint_sums_make_a_symmetrict_monoidal_tensor}.

Disjointness tensors, Definition~\ref{def:disjointness_tensor}, are shown in
Lemma~\ref{lem:tensor_disjointness_is_disjointness} to provide enough structure to allow the
creation of a disjointness relation. However, to create a disjoint join requires a disjoint sum
tensor. Disjoint sum tensors are given in Definition~\ref{def:disjoint_sum_tensor} as additional
conditions on a disjointness tensor. Given a disjoint sum tensor,
Proposition~\ref{prop:disjointness_tensor_gives_disjoint_join} defines a disjoint join.

Section~\ref{sec:disjoint-sums-via-a-disjoint-sum-tensor} proves, in
Proposition~\ref{prop:a_disjoint_sum_tensor_gives_disjoint_sums} and
Proposition~\ref{prop:enough_disjoint_sums_make_a_disjoint_sum_tensor}, that a disjoint sum enables
the creation of a disjoint sum tensor and conversely, a disjoint sum tensor produces disjoint sums.

In Chapter~\ref{cha:matrix_categories}, Definition~\ref{def:inverse_matrix_category} gives us
$\imatx$, a matrix category over the inverse category \X with disjoint
joins. Theorem~\ref{thm:imatx_is_an_disjoint_sum_category} proves this is an inverse category with
disjoint sums. Lemma~\ref{lem:m_is_a_functor} shows that there is a functor $M$ from an inverse
category with disjoint joins to its matrix category.  The matrix construction gives an adjoint
between \DSum, the category of inverse categories with disjoint sums and \DJoin, the category of
inverse categories with disjoint joins. Furthermore,
Proposition~\ref{pro:x_and_imatx_are_equivalent} shows that an inverse category \X with disjoint
sums is equivalent to $\imatx$.


\item \textbf{Distributive Inverse Categories}

We define distributive inverse categories in Definition~\ref{def:distributive_inverse_category} and
show that distributivity of the inverse product over the disjoint join is equivalent to distributing
over a disjoint sum tensor in Lemma~\ref{lem:distributive_means_distribute_over_join}.

Then, in Theorem~\ref{thm:x_tilde_has_coproducts_if_x_is_inverse_distributive_category} we show that
Cartesian Completion turns a disjoint sum tensor into a coproduct. From this we conclude that $\Xt$
is a distributive restriction category in
Corollary~\ref{cor:xt_is_a_distributive_restriction_category}.

\item \textbf{Linkage to quantum computation}

Chapter~\ref{chap:commutative_frobenius_algebras} starts with a review of Frobenius algebras and how
they appear in models of quantum computation. Theorem~\ref{thm:cfrob_is_a_discrete_inverse_category}
shows the category of commutative Frobenius algebras, \CFrob, in a symmetric monoidal category \X
is actually a discrete inverse category. Furthermore, when the symmetric monoidal category \X is an
additive tensor category with zero maps and biproducts, then \CFrob is shown in
Lemma~\ref{lem:cfrobperp_is_a_disjointness_relation} to possess a disjointness relation, given by
$f\perp g \iff \Delta(f\* g)\inv{\Delta} =
0$. Proposition~\ref{prop:biproduct_is_the_disjoint_join_in_cfrobx} shows that the addition of maps
is a disjoint join in \CFrob and Proposition~\ref{prop:cfrobx_has_disjoint_sums} shows the
biproduct of objects is a disjoint sum.

\item \textbf{Inverse Turing Categories and Inverse PCAs}

Inverse Turing categories and inverse partial combinatory algebras are defined in
Definition~\ref{def:inverse_turing_category} and
Definition~\ref{def:inverse_partial_combinatory_algebra} respectively. First, we show in
Lemma~\ref{lem:discrete_turing_category_inverses_make_inverse_turing_category} that if we have a
discrete Turing category $\T$, then $\Inv{\T}$, the inverse subcategory of $\T$, is an inverse Turing
category. Then we show that the Cartesian Completion of an inverse Turing category gives a Turing
category. In Proposition~\ref{prop:inverse-pca-iff-pca} we show that a discrete inverse category \X
has an inverse PCA if and only if \Xt has a PCA. Furthermore, we discuss the category of computable
function based on a PCA in a Cartesian restriction category and show the conditions required for
that to be a discrete Cartesian restriction category in Lemma~\ref{lem:comp_a_is_discrete_cart}.
\end{enumerate}

This thesis demonstrates that discrete inverse categories provide a convenient intermediate link
between standard models of computing (i.e., via inverse Turing categories to Turing categories) and
quantum computing (via special commutative Frobenius algebras).

\section{Future directions}
\label{sec:whither-next}

The obvious next step is to use discrete inverse categories to provide a categorical semantics for
existing reversible languages such as \textbf{Inv} \cite{muetal04:injreversible}, and Theseus
\cite{james2013isomorphic,james2012information}. The examples of the category of partial injective
functions used throughout this thesis provides the basis for the semantics of \textbf{Inv}.

It would be interesting to understand how to formulate datatypes, higher order structures, etc. at the
inverse category level. The obvious correspondences are from product types to the inverse product as
well as between sum types and the disjoint sum. From there, one would study the trace and explore
infinite and recursive types. Note that some of this work would follow naturally from exploring the semantics
of Theseus from \cite{james2013isomorphic,james2012information}.

An interesting direction, in my opinion, would be to further investigate the work on reversible
CCS as in \cite{danos2004reversible} and \cite{phillips2006operational}. This could be combined with
the work of Chakraborty on MPL \cite{chakraborty2014} to describe the communication of multiple
processes and generalize to a series of reversible processes.


The connection to quantum computing, as exampled in
Chapter~\ref{chap:commutative_frobenius_algebras} would benefit from further investigation. As well,
the bulk of current literature on semantics of quantum computing is based on finite dimensional
Hilbert spaces. Our construction is not limited to the finite dimensional
case, and as such, may be of use when considering infinite dimensions. To pursue this area, one
would likely start with a consideration of Frobenius algebras in a linearly distributive category,
as described by Egger in \cite{egger2010frobenius}.

% chapter conclusions_and_future_work (end)

%%% Local Variables:
%%% mode: latex
%%% TeX-master: "../phd-thesis"
%%% End:
