%!TEX root = /Users/gilesb/UofC/thesis/phd-thesis/phd-thesis.tex

\chapter{Restriction categories} % (fold)
\label{chap:restriction_categories}


Restriction categories were introduced in
\cite{cockett2002:restcategories1,cockettlack2003:restcategories2,cockettlack2004:restcategories3}
as a way to give an algebraic treatment of partiality. We will introduce restriction
categories and provide a variety of results.

From this point forward in the thesis, we will use the symbol $\undef$ to mean a function is
undefined on some value or values. For example, if $f: \{1,2\} \to \{a,b,c\}$ where $f(1) = a$ and
$f(2)$ is not defined, then we write $f(2)$ is $\undef$.

\section{Definitions}
\label{sec:definitions}


\begin{definition}\label{def:restriction_category}
  A \emph{restriction category} is a category \X\ together with a \emph{restriction operator} on
  maps,
  \[
    \infer{\restr{f}:A\to A,}{f:A \to B}
  \]
  where $f$ is a map of \X\ and $A,B$ are objects of \X, such that the
  following four \emph{restriction identities} hold, whenever the
  compositions are defined:
  \begin{align*}
    &\rone\ \restr{f} f = f & &
    \rtwo\ \restr{g}  \restr{f} = \restr{f}  \restr{g}\\
    &\rthree\ \restr{\restr{f}  g} = \restr{f}   \restr{g} & &
    \rfour\  f \restr{h} = \restr{f h} f.
  \end{align*}
\end{definition}

\begin{definition}
  A \emph{restriction functor} is a functor which preserves the restriction. That is,
  given a functor $F: \X \to \Y$ with \X\  and \Y\ restriction categories,
  $F$ is a restriction functor if:
  \[
    F(\restr{f}) = \restr{F(f)}.
  \]
\end{definition}

Note that any map such that $r=\restr{r}$ is an idempotent, as $r r = \rst{r} r = r$.
Such a map is called a \emph{restriction idempotent}.

Here are some basic facts (see e.g., \cite{cockett2002:restcategories1} and
\cite{cockett-manes09-boolean-classical-rest-cats}) for restriction categories.

\begin{lemma}\label{lem:restrictionvarious}
  In a restriction category \X,
  \begin{multicols}{2}
    \begin{enumerate}[{(}i{)}]
      \item{}$\rst{f}$ is idempotent;
      \item{} $\rst{f g} = \rst{f g} \, \rst{f}$;\label{lemitem:rv_2}
      \item{} $\rst{f g} = \rst{f \rst{g}}$ ;\label{lemitem:rv_3}
      \item{} $\rst{\rst{f}} = \rst{f}$;
      \item{} $\rst{f}\,\rst{g} = \rst{\rst{f}\,\rst{g}}$;
      \item{} $f$ monic implies $\rst{f} = 1$;
      \item{} $f = \rst{g} f \implies \rst{g}\,\rst{f} = \rst{f}$.
      \\
    \end{enumerate}
  \end{multicols}
\end{lemma}
\begin{proof}
  \prepprooflist
  \begin{enumerate}[{(}i{)}]
    \item Using \rthree and then \rone, we see $\rst{f}\,\rst{f} = \rst{\rst{f} f} = \rst{f}$.
    \item Using \rone, \rthree and then \rtwo, $\rst{f g} = \rst{\rst{f} f g} = \rst{f}\,\rst{f g} =
      \rst{f g}\, f$.
    \item Using \ref{lemitem:rv_2}, \rthree and then \rfour, $\rst{f g} = \rst{f g}\,\rst{f} =
      \rst{\rst{f g} f} = \rst{f\rst{g}}$.
    \item By \ref{lemitem:rv_3}, $\rst{f} = \rst{1 f} = \rst{1 \rst{f}} = \rst{\rst{f}}$.
    \item Using \rthree, $\rst{\rst{f}\,\rst{g}} = \rst{f}\rst{\rst{g}} = \rst{f}\,\rst{g}$.
    \item By \rone $\rst{f} f = 1 f$, hence when $f$ is monic, $\rst{f} = 1$.
    \item $\rst{g}\rst{f} = \rst{\rst{g} f} = \rst{f}$.
  \end{enumerate}
\end{proof}

Note that by Lemma~\ref{lem:restrictionvarious}, all maps $\restr{f}$ are restriction idempotents
as $\rst{f}=\rst{\rst{f}}$.


\begin{definition}\label{def:total_map}
  A map $f:A\to B$ in a restriction category is said to be \emph{total} when
  $\rst{f} = 1_A$.
\end{definition}

\begin{lemma}\label{lem:total_maps_form_sub_category}
   The total maps in a restriction category form a subcategory  $Total(\X) \subseteq \X$.
\end{lemma}
\begin{proof}
  First, as the identity map $1$ is monic, by Lemma~\ref{lem:restrictionvarious}, we have $\rst{1} =
  1$ and therefore the identity map is in $Total(\X)$. If $f,g$ are composable maps in $Total(\X)$,
  then $\rst{f\,g} = \rst{f\rst{g}} = \rst{f} = 1$ and hence $f g$ is in $Total(\X)$. Therefore,
  $Total(\X)$ is a subcategory of $\X$.
\end{proof}
\begin{example}[\Par]\label{ex:par_is_a_restriction_category}
Continuing from Example~\ref{ex:category_par}, \Par is a restriction category. The restriction of $f:A\to B$ is:
\[
  \rst{f}(x) =
  \begin{cases}
    x&\text{if $f(x)$ is defined,}\\
    \undef&\text{if }f(x)\text{ is }\undef.
  \end{cases}
\]
In \Par, the
total maps correspond precisely to the functions that are defined on all elements of the domain.
\end{example}

\begin{example}[\rel]\label{ex:rel_is_not_a_restriction_category}
The category \rel from Example~\ref{ex:category_rel} is \emph{not} a restriction category with the
candidate restriction of $R=\{(a,b)\}$ being $\rst{R} = \{(a,a) | \exists b. (a,b) \in R\}$. The
axiom that fails is \rfour, as can be seen by setting $R=\{(1,1),(1,2)\}, S=\{(2,3)\}$. Then we
have $RS = \{(1,3)\}$, $\rst{RS} = \{(1,1)\}$ and therefore $\rst{RS} R = R$. However,
$R\rst{S} = R\{(2,2)\} = \{(1,2)\}$.
\end{example}

\begin{example}[\pinj]\label{ex:pinj_is_a_restriction_category}
From Example~\ref{ex:category_pinj}, we see \pinj is a restriction category and in fact is a
sub-restriction category of \Par. We will show the four restriction axioms:
\begin{align*}
  \rone\ & \rst{f}f = \{(x,z) | \exists x. (x,x) \in \rst{f} \text{ and } (x,z) \in f\} = \{(x,z) |
  (x,z) \in f\} = f,\\
  \rtwo\ & \rst{f}\rst{g} = \{(x,z) | \exists y. (x,y) \in \rst{f} \text{ and } (y,z) \in \rst{g}\} =
  \{(x,x) | (x,x) \in \rst{f} \text{ and } (x,x) \in \rst{g}\} = \rst{g}\rst{f},\\
  \rthree\ & \rst{\rst{f} g} = \rst{\{(x,y) | (x,x) \in \rst{f}, (x,y) \in g\}} = \{(x,x) | (x,x) in
  \rst{f}, (x,x) \in \rst{g} \} = \rst{f} \rst{g},\\
  \rfour\ & f \rst{g} = \{(x,y) | (x,y) \in f, (y,y) \in \rst{g}\} =  \{(x,y) | (x,y) \in f, \exists
  z. (y,z) \in g\} \\
  &=  \{(x,y) | (x,y) \in f, \exists z. (x,z) \in f g\} =  \{(x,y) | (x,y) \in f,
  (x,x) \in \rst{f g}\} = \rst{f g} f.
\end{align*}
\end{example}


\begin{example}[\topcatp]\label{ex:restriction_category_top}
  This is the category of topological spaces with partial functions.
  \rcategoryns{Topological spaces;}{Any partial function $f$, where $f$ is defined on some open subset
    of $\catdomain{f}$;}{The identity function;}{Function composition;}{The restriction of $f:A\to B$ is:
\[
  \rst{f}(x) =
  \begin{cases}
    x&\text{if $f(x)$ is defined,}\\
    \undef&\text{if }f(x)\text{ is }\undef.
  \end{cases}
\]
}
\end{example}

\begin{example}\label{ex:algebras-not-preserving-unit-as-restriction-cat}
  Given $\cS$ is a symmetric monoidal category, define $\specialcat{Copy}(\cS)$ as the category
  whose objects are the objects of $\cS$ which are commutative comonoids and maps are semigroup
  homomorphisms, i.e., maps which do not preserve the unit. As noted in
  \cite{cockett2002:restcategories1}, when there is a counit, $!:B\to I$ where $I$ is the unit of
  the tensor, this is a restriction category, where the restriction is given by
  \[
    \infer{A\xrightarrow{\rst{f}}A \definedas A \xrightarrow{\Delta} A\*A \xrightarrow{f\*1} B\* A
      \xrightarrow{!\*1} I\*A \xrightarrow{\usl} A.}{A\xrightarrow{f}B}
  \]

  We may also consider commutative monoids with semigroup morphisms,
  $\dual{\specialcat{Copy}(\cS)}$. These will have a corestriction operator, i.e., an operator
  $\rst{(\_)}$ on $f:A\to B$ with $\rst{f}:B\to B$ fulfilling the duals of the restriction axioms.

%  This is the dual of a counital copy category as defined in
%  \cite{cockettlack2004:restcategories3}. Theorem 5.2 from   \cite{cockettlack2004:restcategories3}
%  shows that counital copy categories have a restriction.

  For example, consider the category of commutative monoids in \Ab, the category of abelian
  categories. This is the same as $\dual{(\specialcat{Copy}(\dual{\Ab}))}$ and thus is a corestriction
  category. At the same time, this is the category of commutative rings with homomorphisms which do
  not preserve the unit.

  As another example, consider Frobenius algebras in \cS (defined below in
  \ref{def:frobeniusalgebra}) with coalgebra homomorphisms that do not preserve the unit. Again,
  this category will have a corestriction.
\end{example}

\section{Partial order enrichment} % (fold)
\label{sub:enrichment_in_restriction_categories}

We may use the restriction to define a partial order on the hom-sets of a restriction
category. Intuitively, we would think of a map $f$ being less than a map $g$ if $f$ is
defined on fewer elements than $g$ \emph{and} they agree where they are defined. This can be
expressed as:


\begin{definition}\label{def:restriction_category_hom_set_ordering}
  In a restriction category, for any two parallel maps  $f,g:A\to B$, define $f \le g$ if and only if
  $\restr{f} g = f$.
\end{definition}

\begin{lemma}\label{lem:restriction_cats_are_partial_order_enriched}
  Any restriction category \X is enriched in a category of partial orders under the ordering $\le$
  from  Definition~\ref{def:restriction_category_hom_set_ordering} and the following hold:
  \begin{multicols}{2}
    \begin{enumerate}[{(}i{)}]
      \item $f \le g \implies \restr{f} \le \restr{g}$;\label{lemitem:rst_ordering_2}
      \item $\rst{f g} \le \rst{f}$; \label{lemitem:rst_ordering_3}
      \item $f \le g \implies h f \le h g$;
      \item $f \le g \implies f h \le g h$;
      \item $f \le g$ and $\rst{f} = \rst{g}$ implies $f = g$;
      \item $f \le 1 \iff f = \restr{f}$;
      \item $f \le g$, $h \le k \implies f h \le g k$;
      \item $\rst{g}f = f$ implies $\rst{f} \le \rst{g}$.
    \end{enumerate}
  \end{multicols}
\end{lemma}
\begin{proof}
  First, we show the enrichment by showing $\le$ is a partial order on $\X(A,B)$. With
  $f,g,h:A\to  B$ parallel maps in \X, each of the requirements for a partial order is
  verified below:
  \begin{description}
    \itembf{Reflexivity:} $\restr{f} f = f$ and therefore, $ f \le f$.
    \itembf{Anti-Symmetry:} Given $\restr{f}g = f$ and $\restr{g}f = g$, it follows:
    \[
      f = \restr{f} f = \restr{\restr{f} g} f = \restr{f}\, \restr{g} f
      = \restr{g}\restr{f} f =  \restr{g} f = g.
    \]
    \itembf{Transitivity:} Given $f \le g$ and $g\le h$,
      \[
        \restr{f} h = \restr{\restr{f} g} h = \restr{f}\, \restr{g} h = \restr{f} g = f
      \]
    showing that $f \le h$.
  \end{description}

  We now show the rest of the claims.
  \setlist[enumerate,1]{leftmargin=1.2cm}
  \begin{enumerate}[{(}i{)}]
    \item The premise is that $\restr{f} g = f$. From this, $ \restr{f}\, \restr{g} =
      \restr{\restr{f} g} = \restr{f}$, showing $\restr{f} \le \restr{g}$.
    \item Computing, $\rst{\rst{fg}}\,\rst{f} = \rst{fg}\, \rst{f} = \rst{fg}$ where the last step is by
      Lemma~\ref{lem:restrictionvarious}\ref{lemitem:rv_2}.
    \item $\restr{h f} h g = h \restr{f} g = h f$  and therefore $h f \le h g$.
    \item $\restr{f} g = f$, this shows $\restr{f h} g h = \restr{\restr{f} g h} g h
      = \restr{f}\, \restr{g h} g h = \restr{f} g h = f h$ and therefore $f h \le g h$.
    \item $g = \rst{g} g = \rst{f} g = f$.
    \item As $f \le 1$ means precisely $\restr{f}1 = f$.
    \item $\rst{fh}gk = \rst{fh}\,\rst{f} g k = \rst{fh} f k = f \rst{h} k = f h$.
    \item Assuming $\rst{g} f = f$, we need to show $\rst{f}\, \rst{g} = \rst{f}$. Using \rtwo and
      then \rthree we have $\rst{f}\,\rst{g} = \rst{g}\rst{f}  = \rst{\rst{g} f}  = \rst{f}$.
      Hence, $\rst{f} \le \rst{g}$.
  \end{enumerate}
\end{proof}

In a restriction category \X, we will use the notation $\open{A}$ for the restriction idempotents
of $A$, an object of \X, that is, $\open{A} = \{x:A\to A| x = \rst{x}\}$. The notation $\open{A}$ was chosen
to be suggestive of open sets, as in \topcatp, see Example~\ref{ex:restriction_category_top}.

\begin{lemma}\label{lem:open_a_is_a_meet_semilattice}
  In a restriction category \X, $\open{A}$ is a meet semi-lattice --- a poset with a top element and
  binary meets.
\end{lemma}
\begin{proof}
  The top of the meet semi-lattice is $1_A$, under the ordering from
  Definition~\ref{def:restriction_category_hom_set_ordering}.
  The meet of any two idempotents is given by their composition.
\end{proof}

Let \stabLat be the category whose objects are meet semi-lattices and maps are stable homomorphisms,
that is, they preserve the meets but not necessarily the top. From Example 13 in
\cite{cockett2002:restcategories1}, this is a corestriction category, i.e. $\dual{\stabLat}$ is a
restriction category. We see that the operation $\mathcal{O}$ is a functor, $\mathcal{O}: \X \to
\dual{\stabLat}$.
%TODO - do we add this, what does it mean? ", and is the fundamental functor."

\section{Joins} % (fold)
\label{sub:joins_in_restriction_categories}

The restriction operator allows one to algebraically axiomatize the concept of ``domain of definition''
for a function. With that axiomatization, we may then consider other questions about the maps. In
this section, we consider when maps are identical on their common domain of definition. Two maps
having this property are called compatible.

\begin{definition}\label{def:compatible_maps}
  Two parallel maps $f,g:A \to B$ in a restriction category are \emph{compatible}, written as $f
  \compatible g$, when $\rst{f} g = \rst{g} f$.  A restriction category \X is a \emph{restriction
    preorder} when all parallel pairs of maps are compatible.
\end{definition}

\begin{example}[Compatibility in \Par]\label{ex:compatibility_in_par}
In the restriction category $\Par$, two maps, $f, g$ are compatible when
$(x,y) \in f$ and $(x,y')\in g$ implies that $y = y'$.
\end{example}

Given two compatible maps, $f,g:A\to B$, we now want to consider if we can create a map that
combines $f$ and $g$. Such a map needs to have certain properties:

\begin{definition}\label{def:joins}
  Given \R is a restriction category with zero maps, then \R is said to have
  \emph{joins}\cite{guox:thesis} whenever there is an operator $\join$
  \[
    \infer{A\xrightarrow{f\join g} B}
      {A\overset{f}{\underset{g}{\rightrightarrows}}B,\ f\compatible g}
  \]
  such that:
  \begin{enumerate}[{(}i{)}]
    \item $f \le f \join g$ and $g \le f \join g$,\label{defitem:join1}
    \item $\rst{f\join g} = \rst{f}\join \rst{g}$,\label{defitem:join2}
    \item $f,g \le h$ implies that $f\join g \le h$ and\label{defitem:join3}
    \item $h(f\join g) = h f \join h g$.\label{defitem:join4}
  \end{enumerate}
\end{definition}

\begin{example}[Joins in \Par]\label{ex:joins_in_par}
In the restriction category $\Par$, the join for two compatible maps is given by:
\[(f \join g) (x) = \begin{cases}
  f(x)( = g(x))& \text{ when both $f$ and $g$ are defined;}\\
  f(x)& \text{ when only $f$ is defined;}\\
  g(x)& \text{ when only $g$ is defined;}\\
  \undef & \text{ when both $f$ and $g$ are undefined.}
\end{cases}
\]
Note that the first line of the definition requires $f\compatible g$.

Showing that the conditions of Definition~\ref{def:joins} hold is straightforward. For example,
$\rst{f}(f\join g) = f(x)$ when $f$ is defined and is undefined otherwise, giving $f \le f\join g$
and similarly for $g \le f \join g$.
% We now must show each of the items in Definition~\ref{def:joins} hold:
%   \begin{enumerate}[{(}i{)}]
%     \item From its definition, we see the
%       \[
%         \rst{f}(f\join g) = \begin{cases}
%           f(x) & \text{when $f$ is defined,}\\
%           \undef & \text{otherwise.}
%           \end{cases}
%        \]
%        giving us $f \le f\join g$. Similarly, $g \le f\join g$.
%     \item $\rst{f\join g}(x) = x$ wherever $f$ or $g$ is defined, but that is the same as $\rst{f}\join \rst{g}$.
%     \item We are given $f,g \le h$. Calculating
%       \[
%         \rst{f\join g} h
%           = \begin{cases}
%             h(x) &\text{when $f$ or $g$ are defined,}\\
%             \undef &\text{otherwise.}
%             \end{cases}
%       \]
%       But since $\rst{f}h = f$ and $\rst{g}h = g$, this is the same as $f\join g$ giving $f\join g \le h$.
%     \item Calculating,
%       \[
%          h(f\join g) = \begin{cases}
%            hf(x)( = h g(x))& \text{ when $h$, $f$ and $g$ are defined;}\\
%            hf(x)& \text{ when only $h$ and $f$ are defined;}\\
%            h g(x)& \text{ when only $h$ and $g$ are defined;}\\
%            \undef & \text{ when $h$ or both $f$ and $g$ are undefined.}
%          \end{cases}
%        \]
%        This is the same as the definition of $h f \join h g$.
%   \end{enumerate}

\end{example}
\begin{example}[Joins in \topcatp]\label{ex:joins_in_top}
   Recall from Example~\ref{ex:restriction_category_top} that the map $f:A\to B$ is a continuous
   partial function on some open subset of $A$. $\rst{f}$ is the identity map on the open subset of
   $A$ where $f$ is defined, and as such, may be identified with that open subset.
   As the intersection of open subsets of $A$ is again
   an open subset of $A$, given $f,g:A\to B$, define
   \[
      f\join g(x) = \begin{cases}
        f(x) &\text{when }x\in  \rst{f}\intersection\rst{g},\\
        f(x) &\text{when }x\in  \rst{f}\setminus\rst{g},\\
        g(x) &\text{when }x\in  \rst{g}\setminus\rst{f},\\
        \undef &\text{otherwise.}
      \end{cases}
   \]
   Note this is similar to the definition of $\join$ in \Par and similar reasoning may be used to
   show it is a join.
\end{example}

When \R is a restriction category with joins, this means that $\R(X,Y)$ is now a join
semi-lattice. Joins are related to the coproduct by the following:

\begin{theorem}[Cockett-Guo]\label{lem:join_determines_coproduct}
  Given a restriction category $\R$ with joins, then
  \[
    A \xrightarrow{\cpa} C \xleftarrow{\cpb} B
  \]
  is a coproduct if and only if:
  \begin{enumerate}[{(}i{)}]
    \item $\cpa$ and $\cpb$ are restriction monics;
    \item $\rst{\icpa}\,\rst{\icpb} = \zeroob_{CC}$ (the zero map required by the definition of
      join) and
    \item $\rst{\icpa} \join \rst{\icpb} = 1_C$.
  \end{enumerate}
\end{theorem}
\begin{proof}
  See \cite{cockett-guo2007:joinrestrictioncats}.
\end{proof}

% section joins_in_restriction_categories (end)


\section{Meets} % (fold)
\label{sub:meets_in_restriction_categories}

\begin{definition}\label{def:meet_in_a_restriction_category}
  A restriction category has \emph{meets} if there is an operation $\meet$ on parallel maps:
  \[
    \infer{A\xrightarrow{f\meet g} B}
      {A\overset{f}{\underset{g}{\rightrightarrows}}B}
  \]
  such that $f\meet g \le f, f\meet g \le g, f\meet f = f, h (f\meet g) = h f \meet h g$.
\end{definition}

Meets were introduced in \cite{cockett-guo-hofstra-2012:range2}. Note that, in general,
$(f \meet g) h \ne f h \meet g h$. In fact equality only holds when $h$ is a partial monic, as in
Definition~\ref{def:partial_inverse_etc} below.
We give the following basic results on meets:

\begin{lemma}
  \label{lem:properties_of_meets_in_restriction_categories}
  In a restriction category \X with meets, where $f, g, h$ are maps in
  \X, the following are true:
  \setlist[enumerate,1]{leftmargin=1.2cm}
  \begin{enumerate}[{(}i{)}]
    \item $f\le g \text{ and } f \le h \iff f \le g\meet h$;
        \label{lemsub:properties_of_meets_one}
    \item $f\meet g = g \meet f$;\label{lemsub:properties_of_meets_two}
    \item $\restr{f\meet 1} = f \meet 1$;\label{lemsub:properties_of_meets_three}
    \item $(f \meet g) \meet h = f \meet (g \meet h)$;
    \item $r(f\meet g) = r f \meet g$ where $r=\rst{r}$ is a restriction idempotent;
    \item $(f\meet g)r = f r \meet g$ where $r=\rst{r}$ is a restriction idempotent;
    \item $\restr{f\meet g} \le \restr{f}$ (and therefore $\restr{f\meet g} \le \restr{g}$);
    \item $ (f \meet 1) f = f \meet 1$;
    \item $ e(e \meet 1) = e$ where $e$ is idempotent.
  %\item $ e \meet e' = e e'$
  \end{enumerate}
\end{lemma}
\begin{proof}
  \prepprooflist
  \setlist[enumerate,1]{leftmargin=1.2cm}
  \begin{enumerate}[{(}i{)}]
    \item $f\le g \text{ and } f \le h$ means precisely $f = \restr{f} g$ and $f = \restr{f} h$.
      Therefore,
      \[
        \restr{f} (g\meet h) =  \restr{f} g \meet \restr{f} h =  f\meet f = f
      \]
      and so $f \le g \meet h$. Conversely, given $f \le g\meet h$, we have
      $f = \restr{f} (g\meet h) = \restr{f} g \meet \restr{f} h \le \restr{f} g $. But
      $f \le \restr{f} g$ means $f = \restr{f}\,\restr{f} g = \restr{f}g$ and therefore
      $f \le g$. Similarly, $f \le h$.
    \item From \ref{lemsub:properties_of_meets_one}, as by definition, $f\meet g \le g$ and
      $f \meet g \le f$.
    \item $f\meet 1 = \restr{f\meet 1} (f \meet 1)= (\restr{f \meet 1} f ) \meet (\restr{f \meet 1})
      \le \restr{f \meet 1}$ from which the result follows. %def of $\le$, \rthree and \rone
    \item By definition and transitivity, $(f\meet g)\meet h \le f, g, h$ therefore by
      \ref{lemsub:properties_of_meets_one} $(f \meet g) \meet h \le f \meet (g \meet h)$. Similarly,
      $f \meet (g \meet h) \le(f \meet g) \meet h$ giving the equality.
    \item Given  $r f \meet g \le r f$, calculate:
      \[
        r f\meet g
        = \restr{r f\meet g} r f
        = \restr{r (r f\meet g)} f
        = \restr{r r f\meet r g} f
        = \restr{r (f\meet g)} f
        = r \restr{f\meet g} f
        = r (f\meet g).
      \]
    \item Using the previous point with the restriction idempotent $\restr{f r}$,
      \begin{equation*}
        \begin{split}
          f r \meet g
          = f \restr{r} \meet g   %%r rest id
          = \restr{f r }f \meet g  %% R.4
          = \restr{f r}(f\meet g)   %Pre
          = \restr{f r}\, \restr{f\meet g} f \\ % meet <=
          = \restr{f\meet g}\, \restr{f r} f % R2
          = \restr{f\meet g} f \restr{r}  %R4
          = (f\meet g) r. % meet, r rest id
        \end{split}
      \end{equation*}
    \item For the first claim,
      \[
        \restr{f\meet g}\, \restr{f} =\restr{\restr{f}(f\meet g)}\\
        =\restr{(\restr{f}f)\meet g} =\restr{f\meet g}.
      \]
      The second claim then follows by \ref{lemsub:properties_of_meets_two}.
    \item Given $ f \meet 1 \leq f$:
      \[
        f \meet 1 \leq f \iff  \restr{f \meet 1} f = f \meet 1 \iff  (f \meet 1) f = f \meet 1
      \]
      where the last step is by item \ref{lemsub:properties_of_meets_three} of this lemma.
    \item As $e$ is idempotent, $e (e\meet 1) = (e e \meet e) = e$.
  \end{enumerate}
\end{proof}

Additionally, when a restriction category has both meets and joins, we have:
\begin{lemma}\label{lem:meet_distributes_over_join}
  If \R is a meet restriction category with joins, then the meet distributes over the join, i.e.,
  \[
    h\meet(f\join g) = (h\meet f)\join (h\meet g).
  \]
\end{lemma}
\begin{proof}
  \begin{align*}
    h\meet(f\join g) &= \rst{(f\join g)} h\meet (f\join g)\\
    &= (\rst{f}\join \rst{g}) h\meet (f\join g)\\
    &= (\rst{f}(h\meet (f\join g))) \join (\rst{g}(h\meet (f\join g)))\\
    &= (h\meet \rst{f}(f\join g)) \join (h\meet \rst{g}(f\join g)))\\
    &= (h\meet (f\join g)) \join (h\meet (f\join g))).
  \end{align*}
\end{proof}

\begin{example}[Meets in \pinj and \Par]\label{ex:pinj_has_meets}
The restriction category \pinj has meets given by the intersection of the sets defining the
maps. First, we note that the hom-set ordering for \pinj is given by set inclusion. We immediately have
\begin{align*}
  f\intersection g &\subseteq f \\
  f\intersection g &\subseteq g\\
  f\intersection f & = f
\end{align*}
by the properties of sets and intersections. For the final requirement,
\begin{align*}
  h(f\intersection g) &= \{(x,z) | \exists y. (x,y) \in h, (y,z) \in f\intersection g\}\\
  &=   \{(x,z) | \exists y. (x,y)  \in h, (y,z) \in f, (y,z) \in g\}\\
  & = \{(x,z) | (x,z) \in h f, (x,z) \in h g\} = h f \intersection h g.
\end{align*}
Thus, intersection is a meet in \pinj.

Note that the calculations above apply immediately to \Par as well, therefore intersection is a
meet in \Par.

\end{example}
% section enrichment_and_meets (end)
\section{Partial monics and isomorphisms} % (fold)
\label{sub:restricted_monics_and_partial_isomorphisms}

Partial isomorphisms play a central role in this thesis. Below we present
some of their basic properties.

\begin{definition}\label{def:partial_inverse_etc}
  In a restriction category \X, a map $f$ may have some of the following properties:
  \begin{itemize}
    \item $f$ is a \emph{partial isomorphism} when there is a \emph{partial inverse}, written
      $\inv{f}$ with $f\inv{f} =\restr{f}$ and $\inv{f}f = \restr{\inv{f}}$;
    \item $f$ is a \emph{partial monic} if $h f = k f$ implies $h \restr{f} = k \restr{f}$;
%    \item $f$ is a \emph{partial section} if there exists an  $h$ such that $f h = \restr{f}$;
    \item $f$ is a \emph{restriction monic} if it is a section $s$ with a retraction
      $r$ such that $r s = \restr{r s}$.
  \end{itemize}
\end{definition}

For example, consider the following maps in \Par, $f_1, f_2, f_3:\{1,2\} \to \{a,b,c\}$ where
\[
  f_1(1) = a, f_1(2)=b;\quad  f_2(1) = a, f_2(2) \undef;\quad    f_3(1) = f_3(2)=a.
\]
Then, $f_1$ is a total partial isomorphism and a partial monic and a restriction monic. $f_2$  is a
partial isomorphism and is a partial monic but is not a restriction monic as it is not a
section, i.e., there is no map $\categorysection{f_2}$ such that $f_2\categorysection{f_2} =
1$. $f_3$ is none of the items in Definition~\ref{def:partial_inverse_etc}. Finally, in the
category \topcatp, we shall see in Example~\ref{ex:topcatp-not-discrete} that the diagonal
map, $\Delta: a\mapsto (a,a)$, does not have a partial inverse unless the topological space is
discrete. But as $\Delta$ is monic and total, it is a partial monic.

Note that restriction monic is a stronger notion than that of section. In fact, restriction monics
are the partial isomorphisms which are total.


\begin{lemma}
  \label{lem:rcs_partial_monic_section_inverse_properties}
  In a restriction category:
  \begin{enumerate}[{(}i{)}]
    \item $f,\ g$ partial monic implies $f g$ is partial monic;
%    \item $f$ a partial section implies $f$ is partial monic;
%    \item $f,\ g$ partial sections implies $f g$ is a partial section;
    \item The partial inverse of $f$, when it exists, is unique;
    \item If $f,\ g$ have partial inverses and $f\,g$ exists, then $f\,g$ has a partial inverse;
    \item A restriction monic $s$ is a partial isomorphism.
  \end{enumerate}
\end{lemma}
\begin{proof}
  \prepprooflist
  \begin{enumerate}[{(}i{)}]
    \item Suppose $h f g = k f g$. As $g$ is partial monic, $h f \restr{g} = k f \restr{g}$.
      Therefore:
      \begin{align*}
        h \restr{f g} f &= k \restr{f g} f &\rfour\\
        h \restr{f g}\,\restr{f} &= k \restr{f g}\, \restr{f} & f\text{partial monic}\\
        h \restr{f g}&= k \restr{f g} & \text{Lemma \ref{lem:restrictionvarious},
          \ref{lemitem:rv_2}.}
      \end{align*}
    % \item Suppose $g f = k f$. Then, $g\restr{f} = g f h = k f h = k \restr{f}$.
    % \item We have $f h = \restr{f}$ and $g h' = \restr{g}$. Therefore,
    %   \begin{align*}
    %     f g h' h &= f \restr{g} h & g \text{ partial section}\\
    %     &= \restr{f g} f h & \rfour\\
    %     &= \restr{f g}\, \restr{f} & f \text{ partial section}\\
    %     &= \restr{f}\, \restr{f g} & \rtwo\\
    %     &= \restr{\restr{f}f g} & \rthree\\
    %     &= \restr{f g} & \rone.
    %   \end{align*}
    \item Suppose both $\inv{f}$ and $f^{\diamond}$ are partial inverses of $f$. Then,
      \begin{multline*}
        \inv{f}
        = \restr{\inv{f}}\inv{f} %R.1
        =\inv{f}f\inv{f}  %Assumption, inverse is \inv{f}
        = \inv{f} \restr{f}   %Assumption, inverse is \inv{f}
        = \inv{f} f f^{\diamond}   %Assumption, inverse is f^{\diamond}
        = \inv{f} f \restr{f^{\diamond}} f^{\diamond}  \\ %R.1
        = \restr{\inv{f}}\restr{f^{\diamond}} f^{\diamond}   %Assumption, inverse is \inv{f}
        = \restr{f^{\diamond}}\restr{\inv{f}} f^{\diamond} %R.2
        = f^{\diamond} f \restr{\inv{f}}  f^{\diamond} %Assumption, inverse is f^{\diamond}
        = f^{\diamond} f \inv{f} f f^{\diamond} %Assumption, inverse is \inv{f}
        = f^{\diamond} f f^{\diamond} %Assumption, inverse is \inv{f} and R.1
        = f^{\diamond}. %Assumption, inverse is f^{\diamond}
      \end{multline*}
    \item For $f:A\to B,\ g:B\to C$ with partial inverses $\inv{f}$ and $\inv{g}$ respectively,
      the partial inverse of $f g$ is $\inv{g} \inv{f}$. Calculating $f g \inv{g} \inv{f}$
      using all the restriction identities:
      \[
        f g \inv{g} \inv{f} = f \restr{g} \inv{f} = \restr{f g} f \inv{f} =
        \restr{f g}\, \restr{f} = \restr{f}\, \restr{f g} = \restr{\restr{f} f g} = \restr{f g}.
      \]
      The calculation of $\inv{g} \inv{f} f g = \restr{\inv{g} \inv{f}}$ is similar.
    \item The partial inverse of $s$ is $\restr{r s}\,r$. First, note
      that $\restr{\restr{r s}\,r}
      = \restr{r s}\,\restr{r}
      = \restr{r}\, \restr{r s}
      = \restr{\restr{r}\,r s}
      = \restr{r s}$.
      Then, it follows that $(\restr{r s}\,r) s
      = r s\,= \restr{r s}
      = \restr{\restr{r s}r} $ and
      $s (\restr{r s}\,r)
      = s r \restr{s} %sr = 1 as r is the retraction of the section s
      = \restr{s}$.
  \end{enumerate}
\end{proof}

% section restricted_monics_sections_and_partial_isomorphisms (end)

\section{Range categories} % (fold)
\label{sub:range_categories}
Corresponding to Definition~\ref{def:restriction_category} for restriction, which axiomatizes the
concept of a domain of definition, we now introduce range categories
\cite{guox:thesis,cockett-guo-hofstra-2012:range,cockett-guo-hofstra-2012:range2}
which algebraically axiomatize the concept of the range for a function, in the presence of a
restriction. Note this is different from a corestriction category \Y, which has a single operator,
the corestriction, which is a restriction in \dual{\Y}. In general, the range is weaker than a
corestriction in that it may fail \axiom{R}{4}.

\begin{definition}\label{def:range_category}
  A restriction category \X is a \emph{range category} when it has an operator on all maps
  \[
    \infer{\rg{f}:B\to B}{f:A\to B}
  \]
  where the operator satisfies the following:
  \begin{align*}
    &\rrone\ \restr{\rg{f}} = \rg{f} & &
     \rrtwo\ f \rg{f} = f\\
    &\rrthree\ \wrg{f\rst{g}} = \rg{f} \rst{g} & &
     \rrfour\  \wrg{\rg{f}g} = \wrg{f g}
  \end{align*}
  whenever the compositions are defined.

\end{definition}

\begin{lemma}\label{lem:basic_range_category_properties}
  In a range category \X, the following hold:
  \begin{multicols}{2}
    \begin{enumerate}[{(}i{)}]
      \item $\rg{g}\rg{f} = \rg{f}\rg{g}$;
      \item $\rst{f}\rg{g} = \rg{g}\rst{f}$;
      \item $\wrg{f\rg{g}} = \rg{f}\rg{g}$;
      \item $\rg{f} = 1$ when $f$ is epic, hence $\rg{1} = 1$;
      \item $\rg{f}\rg{f} = \rg{f}$;
      \item $\rg{\rg{f}} = \rg{f}$;
      \item $\rg{\rst{f}} = \rst{f}$;
      \item $\rg{g}\wrg{f g} = \wrg{f g}$;
      \item $\wrg{\rg{f}\rg{g}} = \rg{f}\rg{g}$.
    \end{enumerate}
  \end{multicols}
\end{lemma}
\begin{proof}
  See, e.g., \cite{guox:thesis}.
\end{proof}

\begin{lemma}\label{lem:ordering_of_restriction_and_range}
  In a range category:
  \begin{multicols}{2}
    \begin{enumerate}[{(}i{)}]
      \item  $\wrg{h f} \le \rg{f}$; \label{lemitem:ordering_1}
      \item $f' \le f$ implies $\rg{f'} \le \rg{f}$. \label{lemitem:ordering_2}
    \end{enumerate}
  \end{multicols}
\end{lemma}
\begin{proof}
  \prepprooflist
  \begin{enumerate}[{(}i{)}]
    \item Noting that $\rst{\wrg{hf}} \rg{f} = \wrg{hf} \rg{f}  = \wrg{hf \rg{f}} = \wrg{h f}$,
      we see $\wrg{h f} \le \rg{f}$.
    \item Calculating $\rst{\rg{f'}} \rg{f} = \rg{f'} \rg{f} = \wrg{\rst{f'} f} \rg{f} =
      \wrg{\rst{f'} f \rg{f}} = \wrg{\rst{f'} f} = \rg{f'}$, we see $\rg{f'} \le \rg{f}$.
  \end{enumerate}
\end{proof}


Note that unlike restrictions, a range is a \emph{property} of a restriction category. To see
this, assume we have two ranges $\wrg{(\_)}$ and $\widetilde{(\_)}$. Then,
\[\rg{f}=\wrg{f \tilde{f}}=\rg{f} \tilde{f}=\tilde{f} \rg{f}=\widetilde{f \rg{f}}=\tilde{f}.\]
% section range_categories (end)

\begin{example}\label{ex:ranges}
   In \pinj, $\rg{f} = \{(y,y) | \exists x. (x,y) \in f\}$.
\end{example}

For a further example, see Section~\ref{sec:inverse_categories}.
\section{Split restriction categories} % (fold)
\label{sub:split_restriction_categories}

The Karoubi envelope of a restriction category, $\spl{E}{\X}$ as defined in
Definition~\ref{def:split_category} is a restriction category.

Note that for $f:(A,d)\to(B,e)$, by definition, in \X we have $f=d\uts f e$, giving
\[
  d\uts f = d(d\uts f e) = d\uts d\uts f e = d\uts f e =f\
  \text{ and }\  f e = (d\uts f e)e = d\uts f e\uts e = d\uts f e = f.
\]
When \X is a restriction category, there is an immediate candidate for a restriction in
$\spl{E}{\X}$. If $f\in\spl{E}{\X}$ is $e_1 f e_2$ in $\X$, then define $\restr{f}$ as
given by $e_1 \restr{f}$ in \X. Note that for $f:(A,d)\to(B,e)$, in \X we have:
\[
  d\uts\restr{f} = \restr{d\uts f} d = \restr{f} d.
\]

\begin{proposition}\label{prop:spleisarestrictioncat}
  If \X is a restriction category and $E$ is a set of idempotents, then
  the restriction as defined above makes $\spl{E}{\X}$ a restriction category.
\end{proposition}
\begin{proof}
  The restriction takes $f:(A,e_1)\to (B,e_2)$ to an endomorphism of $(A,e_1)$. The restriction
  is in $\spl{E}{\X}$ as
  \[
    e_1 (e_1\restr{f}) e_1 = e_1 \restr{f} e_1
    = \restr{e_1 f} e_1 e_1
    = \restr{e_1 f} e_1
    = e_1 \restr{ f}.
  \]

  Checking the 4 restriction axioms:
  \begin{align*}
    &[\text{{\bfseries R.1}}]\  e_1 \restr{f} f
    = e_1 f = f.\\
    %
    & [\text{{\textbf{R.2}}}]\  e_1\restr{g}  e_1\restr{f}
    = e_1 e_1\restr{g}  \restr{f} = e_1 e_1\restr{f}  \restr{g}
    = e_1\restr{f}  e_1\restr{g}.\\
    %
    & [\text{{\textbf{R.3}}}]\  e_1 (\restr{e_1 \restr{f}  g})
    =  \restr{e_1 e_1 \restr{f} g} e_1
    =  \restr{e_1 \restr{f} g} e_1
    =  e_1 \restr{\restr{f} g}
    = e_1 \restr{f}\restr{g}
    = e_{1}e_{1}\restr{f}\restr{g}
    = e_1 \restr{f}e_1\restr{g}.\\
    %
    &[\text{{\textbf{R.4}}}]\ f e_2 \restr{g}
    = \restr{ f e_2 g} f e_2
    = (\restr{e_1  f g} e_1)  f
    = e_1 \restr{  f  g}  f.
  \end{align*}
\end{proof}

Given this, provided all identity maps are in $E$, $\spl{E}{\X}$ is a
restriction category with $\X$ as a full sub-restriction category, via
the embedding defined by taking an object $A$ in \X to  the object $(A,1)$
in $\spl{E}{\X}$.


\begin{proposition}\label{pro:in_rc_x_with_meets_split_x_is_cong_to_split_r_x}
  In a restriction category \X with meets, let $R$ be the set of restriction idempotents.
  Then, $\spl{}{\X} \cong \spl{R}{\X}$. That is, splitting over all the idempotents is equivalent to
  splitting over just the restriction idempotents.
  Furthermore, $\spl{R}{\X}$ has meets.
\end{proposition}
\begin{proof}
  The proof first shows the equivalence of the two categories, then addresses the claim
  that $\spl{R}{\X}$ has meets.

  For equivalence, we require two functors,
  \[
    U:\spl{R}{\X}\to\spl{}{\X}\text{ and }V:\spl{}{\X}\to\spl{R}{\X},
  \]
  with:
  \begin{align}
    U V \cong I_{\spl{R}{\X}}\\
    V U \cong I_{\spl{}{\X}}.
  \end{align}


  $U$ is the standard inclusion functor. $V$ will take the object $(A,e)$ to
  $(A,e\meet 1)$ and the map $f:(A,e_1)\to (B,e_2)$ to $(e_1\meet 1)f $.

  $V$ is a functor as:
  \begin{description}
    \itembf{Well Defined:} If  $f:(A,e_1) \to (B,e_2)$, then
      $(e_1\meet 1) f $ is a map in \X from $A$ to $B$ and
      $ (e_1\meet 1)(e_1\meet 1) f  (e_2 \meet 1) =
      (e_1\meet 1) (f  e_2 \meet f ) = (e_1\meet 1) (f \meet f)= (e_1\meet 1) f$, therefore,
      $V(f):V((A,e_1)) \to V((B,e_2))$.
    \itembf{Identities:} $V(e) = (e\meet 1 ) e = e \meet 1$ by
      lemma \ref{lem:properties_of_meets_in_restriction_categories}.
    \itembf{Composition:} $V(f) V(g)
      = (e_1\meet 1 ) f (e_2 \meet 1) g
      = (e_1\meet 1 ) f e_2 (e_2 \meet 1) g
      = (e_1\meet 1 ) f  (e_2 \meet e_{2}) g
      = (e_1\meet 1 ) f e_2 g
      = (e_1\meet 1 ) f g
      = V(f g)$.
  \end{description}

  By Lemma \ref{lem:properties_of_meets_in_restriction_categories} $(e\meet 1)$
  is a restriction idempotent. Using this fact, the commutativity of restriction idempotents
  and Lemma~\ref{lem:properties_of_meets_in_restriction_categories}, the composite functor $U V$ is
  the identity on $\spl{R}{\X}$. This is because when $e$ is a restriction idempotent,
  $e = e (e\meet 1) = (e\meet 1) e = (e\meet 1)$.

  For the other direction,  note that for a particular idempotent $e:A\to A$,  this gives the
  maps $e:(A,e)\to(A,e\meet 1)$ and $e\meet 1 : (A,e\meet 1) \to (A,e)$, again by
  \ref{lem:properties_of_meets_in_restriction_categories}. These maps give the natural
  isomorphism between $I$ and $V U$ as
  \[
    \xymatrix{
      (A,e)\ar[r]^e \ar[dr]_{e} &(A,e\meet 1)\ar[d]^{e\meet 1}\\
      &(A,e)
    }\qquad \text{ and  }\qquad
    \xymatrix{
      (A,e\meet 1)\ar[r]^{e\meet 1} \ar[dr]_{e\meet 1} &(A,e)\ar[d]^{e}\\
      &(A,e\meet 1)
    }
  \]
  both commute. Therefore, $U V = I$ and $V U \cong I$, giving an equivalence of the categories.

  For the rest of this proof, functions in bold type, e.g., $\mbf$, are in $\spl{R}{\X}$.
  Functions in normal slanted type, e.g., $f$ are in \X.

  To show that $\spl{R}{\X}$ has meets,  designate the meet in $\spl{R}{\X}$ as \meetspl
  and define $\mbf \meetspl \mbg$ as the map given by the \X map $f \meet g$, where
  $\mbf,\mbg:(A,d)\to(B,e)$ in $\spl{R}{\X}$ and $f,g:A\to B$ in \X . This is
  a map in $\spl{R}{\X}$ as
  $d(f \meet g)e = (d\uts f \meet d\uts g) e = (f \meet g) e = (f e \meet g) = f\meet g$
  where the penultimate equality is by
  \ref{lem:properties_of_meets_in_restriction_categories}.
  By definition $\restr{\mbf \meetspl \mbg }$ is $d\restr{f\meet g}$.

  It is necessary to show \meetspl satisfies the four meet properties.
  \begin{itemize}
    \item{$\mbf\meetspl \mbg \le \mbf$: } We need to show
      $\rst{\mbf \meetspl \mbg} \mbf =  \mbf \meetspl \mbg$.  Calculating now in \X:
      \[
        d \rst{f \meet g} f= \rst{d(f\meet g)} d f  = \rst{d f \meet d g} d f
         = \rst{f \meet g} f  = f \meet g
      \]
      which is the definition of $\mbf \meetspl \mbg$.
    \item{$\mbf\meetspl \mbg \le \mbg$: } Similarly and once again calculating in \X,
      \[
        d \rst{f \meet g} g = \rst{d(f\meet g)} d g  = \rst{d f \meet d g} d g
         = \rst{f \meet g} g  = f \meet g
      \]
      which is the definition of $\mbf \meetspl \mbg$.
    \item{$\mbf\meetspl \mbf = \mbf$: } From the definition, this is $f \meet f = f$ which
      is just $ \mbf$.
    \item{$\mbh(\mbf\meetspl \mbg) = \mbh\mbf \meetspl \mbh\mbg$: }
      From the definition, this is given in \X by $ h (f \meet g) =
      h f \meet h g$ which in $\spl{R}{\X}$ is $\mbh\mbf \meetspl \mbh\mbg$.
  \end{itemize}
\end{proof}

Consider two objects $A, B$
in a restriction category where we have $m: A\to B$, $r:B \to A$ with $m r = 1_A$. In this case
$A$ is called a \emph{retract} of $B$, which we will write as $A\retract B$. As $m$ and $r$ need
not be unique, we will also write $A \retractmaps{m}{r} B$ when the specific section and retraction
are to be emphasized. Since $m$ is a section, it is a monic and therefore total. The map $r m$ is
idempotent on $B$ as $r m r m = r 1 m = r m$. $A$ is referred to as a \emph{splitting} of the
idempotent $r m$. Note there is no requirement that $r m = \rst{r m}$ when $m$ is simply monic.

% section split_restriction_categories (end)



\section{Partial map categories} % (fold)
\label{sub:partial_map_categories}

In \cite{cockett2002:restcategories1}, it is shown that split restriction categories are
equivalent to \emph{partial map categories}. The main definitions and results related to
partial map categories are given below.

\begin{definition}
  A collection $\Mstab$ of monics is \emph{a stable system of monics}
  when:
  \begin{enumerate}[{(}i{)}]
    \item it includes all isomorphisms;
    \item it is closed under composition;
    \item it is pullback stable.
  \end{enumerate}
\end{definition}

\emph{Stable} in this definition means that if $m:A\to B$ is in \Mstab, then for arbitrary
$b$ with codomain $B$, the pullback
\[
  \xymatrix{
    A'\ar[r]^a \ar[d]_{m'} &A\ar[d]^{m}\\
    B' \ar[r]_{b} & B
  }
\]
exists and $m' \in \Mstab$. A category that has a stable system of monics
is referred to as an \Mstab-category.

\begin{lemma}
  If $n m \in \Mstab$, a stable system of monics, and $m$ is monic, then $n \in \Mstab$.
\end{lemma}
\begin{proof}
  The commutative square
  \[
    \xymatrix{
      A\ar[d]_n \ar[r]^{1} &A\ar[d]^{n m}\\
      A' \ar[r]_{m} & B
    }
  \]
  is a pullback.
\end{proof}

Given a category \B and a stable system of monics $\Mstab$, the \emph{partial map category},
$\text{Par}(\B,\Mstab)$ is:
  \rcategoryequivns{$A\in\B$;}
    {$(m,f):A\to B$  with $m:A' \to A$ is in \Mstab and $f:A' \to B$ is a map in \B. i.e.,
      $\xymatrix @R-15pt @C-15pt{&A'\ar[dl]_{m} \ar[dr]^{f}\\A&&B}$;}
    {$1_A,1_A:A \to A$;}
    {via a pullback, $(m,f)(m',g) = (m'' m, f' g)$ where
      \[
        \xymatrix @C-15pt @R-15pt{
          &&A''\ar[dl]_{m''}\ar[dr]^{f'}\\
          &A'\ar[dl]_{m}\ar[dr]_{f}&\text{{\tiny (pb)}}&B'\ar[dl]^{m'}\ar[dr]^{g}\\
          A&&B&&C;
        }
      \]
    }
    {$\restr{(m,f)} = (m,m)$.}

For the maps, $(m,f) \sim (m',f')$ when there is an isomorphism $\gamma : A'' \to A'$
such that $\gamma m' = m$ and $\gamma f' = f$.

The proof that this is a restriction category is given by Proposition 3.1 in
\cite{cockett2002:restcategories1}.

In \cite{cockettlack2003:restcategories2}, it is shown that:
\begin{theorem}[Cockett-Lack]
  Every restriction category is a full subcategory of a partial map category.
\end{theorem}
% section partial_map_categories (end)
\section{Restriction products and Cartesian restriction categories} % (fold)
\label{sub:restriction_products_and_cartesian_restriction_categories}


Restriction categories have analogues of products and terminal objects.

\begin{definition}\label{def:restriction_product}
  In a restriction category \X, a \emph{restriction product}  of two objects $X, Y$ is an
  object $X\times Y$ equipped with \emph{total} projections
  $\pi_0:X\times Y\to X, \pi_1:X\times Y\to Y $ where:
  \begin{quote}
    $\forall f:Z\to X, g: Z\to Y, \quad \exists$ a unique $\<f,g\>:Z \to X\times Y$ such that
    \begin{itemize}
      \item $\<f,g\> \pi_0 \le f$,
      \item $\<f,g\> \pi_1 \le g$ and
      \item $\restr{\<f,g\>} = \restr{f}\, \restr{g} ( = \restr{g}\, \restr{f})$.
    \end{itemize}
  \end{quote}
\end{definition}

\begin{definition}\label{def:restriction_terminal_object}
  In a restriction category \X\, a \emph{restriction terminal object}
  is an object $\top$ such that for all objects $X$, there is a
  unique total map $!_X : X \to \top$ and the diagram
  \[
    \xymatrix @C=40pt @R=25pt{
      X \ar[r]^{\restr{f}} \ar[d]^{f} & X \ar[r]^{!_X}  &\top  \\
      Y \ar[urr]_{!_Y}
    }
  \]
  commutes. That is,  $f\, !_Y = \restr{f}\, !_X$. Note this implies
  that a restriction terminal object is unique up to a unique isomorphism.
\end{definition}

For example, in \Par, the restriction terminal object is the one object set, $\{*\}$. The product of
two sets is the standard Cartesian product, with $\pi_0$ mapping $(x,y) \mapsto x$ and
$\pi_1:(x,y)\mapsto y$. The product map $\<f,g\>:Z\to X\times Y$ is given as:
\[
  \<f,g\>(z) = \begin{cases}
    (x,y)&f(z) = x\text{ and } g(z) = y\\
    \undef&f(z) \undef\text{ or }g(y)\undef.
    \end{cases}
\]

\begin{definition}\label{def:cartesian_restriction_category}
  A restriction category \X\ is \emph{Cartesian} if it has all restriction products
  and a restriction terminal object.
\end{definition}

\section{Discrete Cartesian restriction categories}\label{sub:discrete_restriction_categories}

\begin{definition}\label{def:discrete_object_and_discrete_cartesian}
  An object $A$ in a Cartesian restriction category is \emph{discrete}
  when the diagonal map
  \[
    \Delta:A \to A \times A
  \]
  is a partial isomorphism.
  A  Cartesian restriction category where all objects are
  discrete is called a \emph{discrete} Cartesian restriction category.
\end{definition}


\begin{example}[\topcatp is not discrete]\label{ex:topcatp-not-discrete}
  In any topological space $T$, the only way that $\Delta$ can have a continuous inverse is when the
  topology is the discrete topology. This example is the motivating example for our terminology of
  discrete.

  The topology of $T\times T$ is generated by open sets $U\times V$ where $U,V$ are open sets of
  $T$. We see that $\Delta \intersection U\times V = \Delta \intersection (U\intersection V) \times
  (U\intersection V)$, so if $\Delta \subseteq \union_i U_i\times V_i$, then  $\Delta \subseteq
  \union_i (U_i\intersection V_i)\times( U_i\intersection V_i)$. Thus, any open cover of $\Delta$ has a
  subcover of the form $\union_i U_i\times U_i$. For $\inv{\Delta}$ to be a continuous
  map, that means the diagonal must be an open set. But if $\Delta$ is open, then $\Delta = \union_i
  U_i\times V_i$ if and only if $\Delta \subseteq \union_i (U_i\intersection V_i) \times (U_i \intersection
  V_I) \subseteq \union_i U_i \times U_i \subseteq \Delta$. So, $\Delta = \union_i U_i \times U_i$,
  which gives us $U_i \times U_i \subseteq \Delta$, but this can only happen when $U_i = \{x\}$, a
  singleton set.  Therefore, $T$ has the discrete topology.
\end{example}

\begin{example}[\Par is discrete]\label{ex:par_is_discrete}
  In \Par,
  \[
    \Delta: x \mapsto (x,x)\text{ and }\inv{\Delta}:(x,y)\mapsto\begin{cases}
      x& x = y,\\
      \undef & x\ne y.
    \end{cases}
  \]
  Thus, \Par is a discrete Cartesian restriction category.
\end{example}

Further examples of discrete and non-discrete Cartesian restriction categories are given at the end
of the section.

\begin{theorem}\label{thm:a_crc_is_discrete_iff_it_has_meets}
  A Cartesian restriction category \X is discrete if and only if it has meets.
\end{theorem}
\begin{proof}
  If \X has meets, then
  \[
    \Delta(\pi_0 \meet \pi_1) = \Delta\pi_0 \meet \Delta\pi_1 = 1\meet 1 = 1.
  \]
  As $\<\pi_0,\pi_1\>$ is identity,
  \begin{align*}
    \restr{\pi_0 \meet \pi_1} &= \restr{\pi_0 \meet \pi_1} \<\pi_0, \pi_1\> \\
    &=\<\rst{\pi_0 \meet \pi_1}\pi_0, \rst{\pi_0 \meet \pi_1}\pi_1\>\\
    &=\<\pi_0 \meet \pi_1,\pi_0 \meet \pi_1\>\\
    &=(\pi_0 \meet \pi_1 )\Delta
  \end{align*}
  and therefore, $\pi_0 \meet \pi_1$ is $\inv{\Delta}$.

  To show the other direction, we set $f\meet g = \<f,g\>\inv{\Delta}$.
  By the definition of the restriction product:
  \[
    f \meet g =  \<f,g\>\inv{\Delta} =\<f,g\>\inv{\Delta} \Delta \pi_0 =
      \<f,g\>\restr{\inv{\Delta}}\pi_0 \le \<f,g\>\pi_0 \le f.
  \]
  Then, substituting $\pi_1$ for $\pi_0$ above, this gives us $f \meet g \le g$.

  For the left distributive law,
  \[
    h(f \meet g) = h \<f,g\>\inv{\Delta} =  \<h f,h g\>\inv{\Delta} = h f \meet h g.
  \]
  The intersection of a map with itself is
  \[
    f\meet f = \<f,f\> \inv{\Delta} = (f \Delta) \inv{\Delta} = f \restr{\Delta} = f
  \]
  as $\Delta$ is total. This shows that $\meet$ as defined above is a meet for the
  Cartesian restriction category \X.

\end{proof}

\begin{definition}\label{def:graphic_map}
  In a Cartesian restriction category, a map $A\xrightarrow{f}B$ is called \emph{graphic} when the
  maps
  \[
    A\xrightarrow{\<f,1\>}B\times A\qquad \text{and}\qquad
    A\xrightarrow{\<\rst{f},1\>}A\times A
  \]
  have partial inverses. A Cartesian restriction category is \emph{graphic} when all of its maps
  are graphic.
\end{definition}

\begin{lemma}\label{lem:graphic_maps_are_closed_in_a_cartesian_restriction_category}
  In a Cartesian restriction category:
  \begin{enumerate}[{(}i{)}]
    \item Graphic maps are closed under composition;
    \item Graphic maps are closed under the restriction;
    \item An object is discrete if and only if its identity map is graphic.
  \end{enumerate}
\end{lemma}
\begin{proof}
  \prepprooflist
  \begin{enumerate}[{(}i{)}]
    \item To show closure, it is necessary to show for graphic maps $f:A\to B$ and $g:B\to C$ that
      $\<f g,1\>$ has a partial inverse. By Lemma
      \ref{lem:rcs_partial_monic_section_inverse_properties}, the uniqueness of the  partial inverse gives
      \[
        \inv{(\<f,1\> ( \<g,1\>\times 1))} = (\inv{\<g,1\>} \times 1) \inv{\<f,1\>} .
      \]
      By the definition of the restriction product, we have $\rst{\<f g,1\>} = \rst{f g}$. Additionally,
      a straightforward calculation shows that
        $\rst{\<f,1\>(\<g,1\> \times 1)} =
          \rst{\<f\<g,1\>, 1\>} = \rst{f \< g,1\>}
          = \rst{\<f g, f\>} = \rst{f g}\,\rst{f} = \rst{f g}
        $
      where the last equality is from Lemma~\ref{lem:restrictionvarious}.

    Consider the diagram
    \[
      \xymatrix @C+35pt @R+20pt{
        A \ar[r]^{\<f,1\>} \ar[drr]_{\<f g,1\>} &
           B \times A  \ar[r]^{\<g,1\> \times 1}
           &  C \times B \times A \\
        &&C \times A. \ar[u]_{1 \times \<f,1\>}
      }
    \]

    Thus,
    \begin{align*}
      \<f g,1\>  (1\times \<f,1\>) &( \inv{\<g,1\>}\times 1) \inv{\<f,1\>}\\
      &=\<f,1\>(\<g,1\>\times 1 ) (\inv{\<g,1\>}\times 1) \inv{\<f,1\>}\\
      &=\<f,1\> (\rst{g\times 1}) \inv{\<f,1\>}\\
      &=\rst{\<f,1\> (g\times 1)}  \<f,1\> \inv{\<f,1\>}\\
      &=\rst{\<f,1\> (g\times 1)}\,  \rst{\<f,1\>}\\
      &= \rst{\<f,1\>}\, \rst{\<f,1\>(g\times 1)}\\
      &= \rst{\<f,1\> (g\times 1)}\\
      &= \rst{\<f g,1\>}(=\restr{f g})\\
    \end{align*}
    showing that $1\times \<f,1\>  (\inv{\<g,1\>}\times 1 ) \inv{\<f,1\>}$ is
    a right inverse for $\<f g,1\>$.

    For the other direction, note that in general $\inv{(h k)} = \inv{k}\inv{h}$ and that
    we have $\<f g,1\> = \<f,1\> (\<g,1\>\times 1)  (1 \times \inv{\<f,1\>})$, thus
    $(1\times \<f,1\>)  (\inv{\<g,1\>}\times 1) \inv{\<f,1\>}$ will also be a left inverse and
    $\<f g,1\>$ is a partial isomorphism.

    \item This follows from the definition of graphic and that
       $\rst{\<f,1\>} = \rst{f} = \restr{\rst{f}} = \rst{\<\rst{f},1\>}$.

    \item Given a discrete object $A$, the identity on $A$ is graphic as $\<1,1\> = \Delta$
      and therefore $\inv{\<1,1\>} = \inv{\Delta}$. Conversely, if $\<1,1\> = \Delta$ has an
      inverse, $A$ is discrete by definition.
  \end{enumerate}
\end{proof}

\begin{lemma}\label{lem:a_discrete_crc_is_precisely_a_graphic_crc}
  A discrete Cartesian restriction category \D is precisely a graphic Cartesian restriction category.
\end{lemma}
\begin{proof}
  The requirement is that for all $f\in \arrows{\D}$ both $\<f,1\>$ and $\<\rst{f},1\>$ have partial
  inverses. For $\<f,1\>$ the inverse is $\rst{(1 \times f)\inv{\Delta}} \pi_1$.

  To show this, calculate  the two compositions. First,
  \[
    \<f,1\> \rst{1 \times f \inv{\Delta}} \pi_1 =
      \rst{\<f,f\> \inv{\Delta}}\<f,1\>\pi_1 % use R.4
    = \rst{f \Delta \inv{\Delta}}\<f,1\>\pi_1 % product
    = \rst{f}\<f,1\>\pi_1 % Delta total
    = \rst{f}.% product
  \]
  The other direction is:
  \begin{align*}
    \rst{(1 \times f)\inv{\Delta}} \pi_1 \<f,1\>
      &= \< \rst{(1 \times f)\inv{\Delta}} \pi_1 f ,
      \rst{(1 \times f)\inv{\Delta}}\pi_1 \>\\ %product definition
    &= \< \rst{(1 \times f)\inv{\Delta}} (1 \times f) \pi_1,
      \rst{(1 \times f)\inv{\Delta}}\pi_1 \>\\ %pi total, natural
    &= \< (1 \times f )\rst{\inv{\Delta}} \pi_1 ,
      \rst{(1 \times f)\inv{\Delta}}\pi_1 \>\\ %R.4
    &= \< (1 \times f) \rst{\inv{\Delta}} \pi_0 ,
      \rst{(1 \times f)\inv{\Delta}}\pi_1 \>\\ %below
    &= \< \rst{(1 \times f)\inv{\Delta}} (1 \times f) \pi_0,
      \rst{(1 \times f)\inv{\Delta}}\pi_1 \>\\ %R.4
  %  &= \< \rst{(1 \times f)\inv{\Delta}}\,
  %     \rst{(1 \times f)} \pi_0, \rst{(1 \times f)\inv{\Delta}}\pi_1 \>\\
    &= \< \rst{(1 \times f)\inv{\Delta}} \pi_0,
      \rst{(1 \times f)\inv{\Delta}}\pi_1 \>\\ %(a x b);pi0 = a
    &= \rst{(1 \times f)\inv{\Delta}} \< \pi_0, \pi_1 \>\\ % products
    &= \rst{(1 \times f)\inv{\Delta}}.
  \end{align*}
  The above follows in a discrete Cartesian restriction category, as we have
  \begin{equation*}
    \rst{\inv{\Delta}} \pi_1 = \inv{\Delta} \Delta \pi_1 = \inv{\Delta} = \inv{\Delta} \Delta \pi_0 = \rst{\inv{\Delta}} \pi_0.
  \end{equation*}

  For $\<\rst{f},1\>$, the inverse is $\rst{(1 \times \rst{f})\inv{\Delta}} \pi_1$. Similarly
  to above,
  \[
    \<\rst{f},1\> \rst{1 \times \rst{f} \inv{\Delta}} \pi_1 =
      \rst{\<\rst{f},\rst{f}\> \inv{\Delta}}\<\rst{f},1\>\pi_1 % use R.4
    = \rst{\rst{f} \Delta \inv{\Delta}}\<\rst{f},1\>\pi_1 % product
    = \rst{\rst{f}}\<\rst{f},1\>\pi_1 % Delta total
    = \rst{f}.% product
  \]
  The other direction follows the same pattern as for $\<f,1\>.$
\end{proof}

To conclude this section, we give a few examples of Cartesian restriction categories, of both the
discrete and non-discrete variety.

\begin{example}[Terminal object is discrete]\label{ex:terminal_object_is_discrete}
  In any Cartesian restriction category, the terminal object, $1$, is discrete as $1\times 1 \cong 1$.
\end{example}

\begin{example}[Semi-lattice is discrete]\label{ex:semi-lattice_is_discrete}
  As the product is the meet of the semi-lattice and $A\wedge A = A$, we have $\Delta = 1$ and,
  therefore, is always invertible. Note that a total discrete Cartesian restriction category must be a
  semi-lattice. Also, we see that any Cartesian restriction category which is a restriction preorder
  will also be discrete.
\end{example}

\begin{example}[Non-discrete Cartesian restriction categories]\label{ex:various_not_discrete}
Besides the example of \topcatp given at the beginning of this section, the following are not discrete:
  \begin{enumerate}[{(}i{)}]
  \item Any total non-trivial (i.e., not a semi-lattice) Cartesian category is not discrete.
  \item $\text{Par}(\X,\Mstab)$ is not discrete unless $\Delta: X\to X\times X$ is in \Mstab.
  \item $\dual{\stabLat}$ is not discrete.
  \end{enumerate}
  We will give the details for $\dual{\stabLat}$. Recall $\stabLat$ is the category of meet
  semi-lattices whose maps preserve the meet, but do not necessarily preserve the top, $\top$. The
  corestriction of $f:L_1\to L_2$ is given by $\rst{f}:L_2\to L_2$, $f(y) = y\wedge f(\top)$. Hence,
  the total maps are those which preserve the top element.

  The restriction product in $\dual{\stabLat}$ is given by the coproduct of the semi-lattices, which
  is also the product as the category has biproducts. In $\stabLat$, we have the maps
  \[
     in_0: L_0 \to L_0\times L_1\quad\text{ and }\quad in_1:L_2 \to L_1\times L_2
  \]
  where
  \[
     in_0: \ell\mapsto (\ell,\top) \quad\text{ and }in_1: m \mapsto (\top,m).
  \]
  Then, $\pi_i = \dual{in_i}$ are the projections in $\dual{\stabLat}$. Again considering
  $\stabLat$, for $f:L_0 \to L_2$, $g:L_1 \to L_2$, there is the map $[f,g]:L_0\times L_1 \to L_2$
  where $[f,g]:(\ell,m)\mapsto f(\ell)\wedge g(m)$. The product map in $\dual{\stabLat}$,
  $\<\dual{f},\dual{g}\>$ is $\dual{[f,g]}:L_2 \to L_0\times L_1$.

  Using the standard notation of $\nabla$ for the map $[1,1]$ in \stabLat, we see $\Delta:L_1\to
  L_1\times L_1$ in $\dual{\stabLat}$ is $\dual{\nabla}$. For this to have an inverse, we  need a
  map $h:L_1\to L_1\times L_1$ in \stabLat such that $\Delta \dual{h} = 1_{L_1}$, i.e.,  $h\nabla =
  1_{L_1}$ in \stabLat. In \stabLat, we may write $h$ as having two components, that is, $h(x) = (h_1(x),h_2(x))$.
  %As $\top = h(\top)[1,1] = h_1(\top)\wedge
  %h2(\top)$, we must have $h_1(\top) = h_2(\top) = \top$.

  As we know that $x\wedge y = \nabla(h(x\wedge y)$, we have that $h(\top) \wedge (x,y) = h (x\wedge
  y)$. But this gives us $(h_1(\top) \wedge x, h_2(\top) \wedge y) = (h_1(x\wedge y), h_2(x\wedge
  y))$.
  Therefore:
  \begin{align*}
    h_1(x\wedge y) &= h_1(\top)\wedge x\qquad\text{and}\\
    h_2(x\wedge y) &= h_2(\top)\wedge y.
  \end{align*}
  But then we have
  \[
     x\wedge y = h_1(x\wedge y) \wedge h_2(x\wedge y)  = h_1(\top)\wedge h_2(\top)\wedge x\wedge y
  \]
  which means that $h_1(\top)\wedge h_2(\top) = \top$, which gives $h_1(x\wedge y) = x$ and
  $h_2(x\wedge y) = y$. In any lattice where there is a non top element $p$, this produces a
  contradiction as $h_1(p) = h_1(p\wedge \top) = p$ and $h_1(p) = h_1(\top\wedge p) = \top$ which
  can only happen when $\top = p$. This shows that the only discrete object in $\dual{\stabLat}$ is
  the one element semi-lattice, $\{\top\}$.

  Thus, the map $h$ does not exist in general and $\dual{\stabLat}$ Is not discrete.
\end{example}


% section graphic_categories (end)

% chapter restriction_categories (end)

%%% Local Variables:
%%% mode: latex
%%% TeX-master: "../phd-thesis"
%%% End:
