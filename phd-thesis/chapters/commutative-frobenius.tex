%!TEX root = /Users/gilesb/UofC/thesis/phd-thesis/phd-thesis.tex
\chapter{Commutative Frobenius algebras and inverse categories} % (fold)
\label{chap:commutative_frobenius_algebras}


\section{The category of Commutative Frobenius Algebras} % (fold)
\label{sec:the_category_of_commutative_frobenius_algebras}
Dagger categories generalize the category of Hilbert spaces which is often used to model quantum
computation. These were introduced in \cite{abramsky04:catsemquantprot} as \emph{strongly compact
closed categories}, an additional structure on compact closed categories.

Before introducing dagger categories, we define compact closed
categories.



\begin{definition}\label{def:compactclosedcat}
A \emph{compact closed category} \cD{} is a symmetric monoidal category with tensor $\*$ where each
object $A$ has a dual $A^{*}$. Additionally, there must exist families of maps $\eta_{A}: I \to
A^{*} \* A$ (the \emph{unit}) and $\epsilon_{A}: A\*A^{*}\to I$ (the \emph{counit}) such that
\[
  \xymatrix@C+15pt{
    A \ar[r]^{u_{A}} \ar@{=}[d]  & A\*I \ar[r]^(.4){1\*\eta_{A}}
        & A\* (A^{*}\*A) \ar[d]^{a_{A,A^{*},A}} \\
    A & I\* A \ar[l]^{u_{A}^{-1}} & (A\* A^{*})\*A \ar[l]^(.6){\epsilon_{A}\*1}
    }\text{ and }
  \xymatrix@C+15pt{
    A^{*} \ar[r]^{u_{A^*}} \ar@{=}[d]  & I\*A^* \ar[r]^(.4){\eta_{A}\*1}
        & (A^{*}\* A)\*A^{*} \ar[d]^{a_{A^{*},A,A^{*}}^{-1}} \\
    A^* & A^*\*I \ar[l]^{u_{A^*}^{-1}} & A^{*}\*(A\*A^{*}) \ar[l]^(.6){1\*\epsilon_{A}}
    }
  \]
commute.
\end{definition}

Given a map $f:A\to B$ in a compact closed category,  define the map $f^{*}:B^{*} \to A^{*}$ as
\[
  \xymatrix@C+10pt{
    B^{*}\ar[r]^{u_{B^{*}}} \ar[d]_{f^{*}}& I\*B^{*} \ar[r]^{\eta_{A}\*1}
      & A^{*}\*A\*B^{*}\ar[d]^{1\*f\*1}\\
    A^{*}&    A^{*}\*I\ar[l]^{u_{A^{*}}^{-1}}  &   A^{*}\*B\*B^{*}\ar[l]^{1\*\epsilon_{B}}.
  }
\]


%!TEX root = /Users/gilesb/UofC/thesis/phd-thesis/phd-thesis.tex
\subsection{Dagger categories}\label{ssec:daggercategories}

Although dagger categories were introduced in the context of compact closed categories, the concept
of a dagger is definable independently. This was first done in \cite{selinger05:dagger}.

\begin{definition}\label{def:daggercat}
  A \emph{dagger} on a category $D$ is a functor $\dagger:\dual{\cD}\to \cD$, which is  involutive,
  that is, $\dgr{\dgr{f}} = f$ and which is the identity on objects. A \emph{dagger category} is a
  category that has a dagger.
\end{definition}

Typically, the dagger is written as a superscript on the morphism. So, if $f:A\to B$ is a map in
\cD, then $\dgr{f}:B\to A$ is a map in \cD{} and is called the \emph{adjoint} of $f$. A map where
$f^{-1} = \dgr{f}$ is called \emph{unitary}. A map $f:A\to A$ with $f=\dgr{f}$ is called
\emph{self-adjoint} or \emph{Hermitian}.

\begin{definition}\label{def:daggersmc}
  A \emph{dagger symmetric monoidal category} is a symmetric monoidal category \cD{} with a dagger
  operator such that:
  \begin{enumerate}[{(}i{)}]
    \item For all maps $f:A\to B$ and $g:C\to D$, $\dgr{(f\*g)} = \dgr{f}\*\dgr{g}:B\*D \to A\* C$;\label{defitem:dagger_smc_one}
    \item The monoid structure isomorphisms $a_{A,B,C}:(A\*B)\* C\to A\*(B\*C)$, $u^l_{A}:I\*A\to
      A$, $u^r_{A}:A\*I \to A$ and  $c_{A,B}:A\*B \to B\*A$ are unitary.\label{defitem:dagger_smc_two}
  \end{enumerate}
\end{definition}


\begin{definition}\label{def:daggercompact}
  A \emph{dagger compact closed category} \cD{} is a dagger symmetric monoidal category
  that is compact closed where the diagram
  \[
    \xymatrix @C+20pt @R+10pt{
      I \ar[r]^{\epsilon^{\dagger}_{A}} \ar[dr]_{\eta_{A}} &A\*A^{*}\ar[d]^{c_{A,A^{*}}}\\
      &A^{*}\* A
    }
  \]
  commutes for all  objects $A$ in \cD.
\end{definition}

\begin{lemma}\label{lemma:daggerbiproducts}
If \cD{} is a dagger category with biproducts, with injections $in_{1},in_{2}$ and projections
$p_{1},p_{2}$, then the following are equivalent:
\begin{enumerate}[{(}i{)}]
  \item $\dgr{p_{i}} = in_{i}, i=1,2$, \label{ldpdgrpisq}
  \item $\dgr{(f\biproduct g)} = \dgr{f}\biproduct \dgr{g}$ and $\dgr{\Delta} = \nabla$,\label{ldpddeltisnab}
  \item $\dgr{\<f,g\>} = [\dgr{f},\dgr{g}]$,\label{ldpdcopisprod}
  \item The map $[\dgr{p_{1}},\dgr{p_{2}}]: \dgr{A} \biproduct \dgr{B} \to \dgr{(A\biproduct B)}$ is
    the identity map.\label{ldpcommute}
%the below diagram commutes:
%  \[
%    \xymatrix @C+20pt @R+10pt{
%      \dgr{A} \biproduct \dgr{B} \ar[d]_{id} \ar[dr]^{[\dgr{p_{1}},\dgr{p_{2}}]}\\
%      A\biproduct B\ar[r]_{id}&\dgr{(A\biproduct B)}.
%    }
%  \]
\end{enumerate}
\end{lemma}
\begin{proof}
  \begin{description}
    \item[\ref{ldpdgrpisq}$\implies$\ref{ldpddeltisnab}] To show $\dgr{\Delta} = \nabla$,
    draw the product cone for $\Delta$,
    \[
      \xymatrix {
        &A \ar[d]^{\Delta} \ar[dr]^{id} \ar[dl]_{id}\\
        A
         & A\biproduct A \ar[l]^{p_{1}}  \ar[r]_{p_{2}}
         & A
      }
    \]
    and apply the dagger functor to it. As $\dgr{p_{i}} = in_{i}$, and $\dagger$ is identity on
    objects, this is now a coproduct diagram and therefore $\dgr{\Delta} = \nabla$.

    For $\dgr{(f\biproduct g)} = \dgr{f}\biproduct\dgr{g}$, start with the diagram defining
    $f\biproduct g$ as a product of the arrows:
    \[
      \xymatrix{
        A\ar[d]_{f}  & A\biproduct B \ar[l]_{p_{1}} \ar[r]^{p_{2}} \ar[d]^{f\biproduct g}&A \ar[d]^{g}\\
        C & C\biproduct D \ar[l]^{p_{1}} \ar[r]_{p_{2}}  & D.
      }
    \]
    Then, apply the dagger functor to this diagram. This is now the diagram defining the
    coproduct of maps and therefore $\dgr{(f\biproduct g)} = \dgr{f}\biproduct\dgr{g}$.
    \item[\ref{ldpddeltisnab}$\implies$\ref{ldpdcopisprod}] The calculation showing this is
      \begin{eqnarray*}
        &[\dgr{f},\dgr{g}] & = \nabla; (\dgr{f}\biproduct \dgr{g})\\
        & &=\dgr{\Delta}; (\dgr{f}\biproduct \dgr{g})\\
        & &=\dgr{\Delta}; \dgr{(f\biproduct g)}\\
        & & = \dgr{((f\biproduct g);\Delta)}\\
        & & = \dgr{\<f,g\>}.
      \end{eqnarray*}
    \item[\ref{ldpdcopisprod}$\implies$\ref{ldpcommute}]
      Under the assumption,
      \[
        [\dgr{p_{1}},\dgr{p_{2}}] = \dgr{\<p_{1},p_{2}\>}=\dgr{id}=id.
      \]
    \item[\ref{ldpcommute}$\implies$\ref{ldpdgrpisq}] As $[in_{1},in_{2}]:\dgr{A} \biproduct \dgr{B}
      \to \dgr{A} \biproduct \dgr{B} = id = [\dgr{p_{1}},\dgr{p_{2}}]$, we immediately have
      $\dgr{p_{1}} = in_{1}$ and $\dgr{p_{2}} = in_{2}$.
%
%Using the injections and under
%    the assumption, the following diagram commutes:
%      \[
%        \xymatrix @C+20pt @R+10pt{
%          \dgr{A} \biproduct \dgr{B} \ar[d]_{id} \ar[dr]^{[\dgr{p_{1}},\dgr{p_{2}}]}\ar[r]^{[in_{1},in_{2}]}
%            & \dgr{A} \biproduct \dgr{B} \ar[d]^{id}\\
%          A\biproduct B\ar[r]_{id}&\dgr{(A\biproduct B)}
%        }
%      \]
%      and therefore,
  \end{description}
\end{proof}

\begin{definition} \label{def:biproductdaggerccc}
  A \emph{biproduct dagger compact closed category} is a dagger compact closed category with
  biproducts where the conditions of lemma \ref{lemma:daggerbiproducts} hold.
\end{definition}
\subsection{Examples of dagger categories}

\begin{example}[\fdh]\label{ex:fdhilbert_is_dagger_category}
The category of finite dimensional Hilbert spaces is the motivating example for
the creation of the dagger and is, in fact, a biproduct dagger compact closed category. The
biproduct is the direct sum of Hilbert spaces and the tensor for compact closure is the standard
tensor of Hilbert spaces. The dual $H^{*}$ of a space $H$ is the space of all continuous linear
functions from $H$ to the base field. The dagger is defined via the adjoint as being the unique map
$\dgr{f}:B\to A$ such that $\<f a|b\> = \<a | \dgr{f} b\>$ for all $a\in A, b\in B$.
\end{example}

\begin{example}[\rel]\label{ex:rel_is_dagger_category}
The category \rel of sets and relations has the tensor $S\*T \definedas S\times T$ and the biproduct
$S\biproduct T \definedas S\disjointunion T$. This is compact closed under $A^{*} \definedas A$ and
the dagger is the relational converse. That is, if the relation
$R=\{(s,t)|s\in S, t\in T\}:S\to T$, then $\dgr{R}=R^*=\{(t,s)|(s,t)\in R\}$.
\end{example}

\begin{example}[Inverse categories]\label{ex:inverse_category_is_dagger_category}
An inverse category \X is also a dagger category when the dagger is defined as the partial inverse.
The unitary maps are the total maps. When the inverse category \X is also a
symmetric monoidal category where the monoid $\*$ is actually a restriction bi-functor, then \X is
a dagger symmetric monoidal category.

Requirement \ref{defitem:dagger_smc_one} of Definition~\ref{def:daggersmc}  is fulfilled, as
\[
  (f\*g) \inv{(f\*g)} = \rst{f\*g}=\rst{f} \*\rst{g} =
   f\inv{f} \* g \inv{g} = (f\*g) (\inv{f} \* \inv{g})
\]
and since the partial inverse of $f\*g$ is unique, $\inv{(f\*g)} = \inv{f} \* \inv{g}$.
Requirement \ref{defitem:dagger_smc_two} is that the structure isomorphisms are unitary. This is, of
course, true as each of them are isomorphisms, hence total and therefore unitary.
\end{example}
%%% Local Variables:
%%% mode: latex
%%% TeX-master: "../../phd-thesis"
%%% End:

\section{Frobenius Algebras} % (fold)
\label{sec:frobenius_algebras}
In their most general setting, Frobenius algebras are defined as a finite dimensional algebra
over a field together with a non-degenerate pairing operation. We will continue with the definitions
that make this precise.

\subsection{Frobenius algebra definitions} % (fold)
\label{sub:frobenius_algebra_definitions}


\begin{definition}\label{def:frobeniusalgebra}
  Given a symmetric monoidal category \cD, a \emph{Frobenius algebra} is an object $X$ of \cD and
  four maps, $\nabla :X\*X \to X$, $e: I \to X$, $\Delta: X\to X\* X$ and $\epsilon:X\to I$, with
  the conditions that $(X,\nabla,e)$ forms a commutative monoid, $(X,\Delta, \epsilon)$ forms a
  commutative comonoid and the diagram
  \[
    \xymatrix{
      X\*X \ar[rr]^{X\*\Delta} \ar[dd]_{\Delta \* X} \ar[dr]^{\nabla}
        && X\*X\*X \ar[dd]^{\nabla\*X}\\
      & X\ar[dr]^{\Delta}\\
      X\*X\*X\ar[rr]_{X\*\nabla}  && X\*X
    }
  \]
  commutes. The Frobenius algebra is \emph{special} when $\Delta \nabla = 1_{X}$ and
  \emph{commutative} when $\Delta c_{X,X} = \Delta$.
\end{definition}
\begin{definition}\label{def:daggerfrob}
  A Frobenius algebra in a dagger symmetric monoidal category where $\Delta = \dgr{\nabla}$ and
  $\epsilon=\dgr{u}$ is a $\dagger$\emph{-Frobenius algebra}.
\end{definition}
For an example of a $\dagger$-Frobenius algebra, consider a finite dimensional Hilbert space $H$
with an orthonormal basis $\{\ket{\phi_{i}}\}$ and define $\Delta:H\to H\*H: \ket{\phi_{i}}\mapsto
\ket{\phi_{i}} \* \ket{\phi_{i}}$ and $\epsilon : H\to \complex : \ket{\phi_{i}} \mapsto 1$. Then $(H,
\nabla=\dgr{\Delta}, u=\dgr{\epsilon}, \Delta, \epsilon)$ forms a commutative special
$\dagger$-Frobenius algebra.

% subsection frobenius_algebra_definitions (end)

\subsection{Bases and Frobenius Algebras} % (fold)
\label{sub:bases_and_frobenius_algebras}
In \cite{coeckeetal08:ortho}, Coecke et. al. provide an algebraic description of orthogonal bases in
finite dimensional Hilbert spaces. Additionally,  an orthonormal basis for such a space is
a special commutative $\dagger$-Frobenius algebra. To show the other direction, given a commutative
$\dagger$-Frobenius algebra, $(H,\nabla,u)$ and for each element $\alpha\in H$, define the right
action of $\alpha$ as $R_{\alpha}:=(id\*\alpha)\, \nabla:H\to H$. Note the use of the fact that
elements $\alpha\in H$ can be considered as linear maps $\alpha:\complex \to H:1\mapsto \ket{\alpha}$.
The dagger of a right action is also a right action, $\dgr{R_{\alpha}} = R_{\alpha'}$ where
$\alpha'= u\, \nabla\, (id\* \dgr{\alpha})$, which is a consequence of the Frobenius identities.

The $(\_)'$ construction is actually an involution:
\begin{eqnarray*}
  &(\alpha')' &= u \nabla (id \* \dgr{\alpha'}) \\
  && = u \nabla (id \* \dgr{(u \nabla (id \* \dgr{\alpha}))}\\
  && = u \nabla (id \* ( (id \* \alpha) \Delta \epsilon))\\
  && = (u \* \alpha) (\nabla \* id) (id \* \Delta) (id \*  \epsilon)\\
  && = (u \* \alpha) (id \* \Delta) (\nabla \* id) (id \*  \epsilon)\\
  && = (u \* \alpha)  (id \*  \epsilon)\\
  && = \alpha
\end{eqnarray*}

\begin{lemma}\label{lemma:cstaralgebra}
  Any $\dagger$-Frobenius algebra in \fdh is a $C^{*}$-algebra.
\end{lemma}
\begin{proof}
  The endomorphism monoid of \fdh(H,H) is a $C^{*}$-algebra. From the proceeding, we have
  \[
    H \cong \fdh(\complex,H) \cong R_{[\fdh(\complex,H)]}\subseteq\fdh(H,H).
  \]
  This inherits the algebra structure from \fdh(H,H). Furthermore, since any finite dimensional
  involution-closed sub-algebra of a $C^{*}$-algebra is also a $C^{*}$-algebra, this shows the
  $\dagger$-Frobenius algebra is a $C^{*}$-algebra.
\end{proof}

Using the fact that the involution preserving homomorphisms from a finite dimensional commutative
$C^{*}$-algebra to $\complex$ form a basis for the dual of the underlying vector space, write these
homomorphisms as $\dgr{\phi_{i}}:H \to \complex$. Then their adjoints, $\phi_{i}:\complex\to H$ will form a
basis for the space $H$. These are the copyable elements in $H$.

This, together with continued applications of the Frobenius rules and linear algebra allow the
authors to prove the following Theorem.
\begin{theorem}
  Every commutative $\dagger$-Frobenius algebra in \fdh determines an orthogonal basis consisting
  of its copyable elements. Conversely, every orthogonal basis $\{\ket{\phi_{i}}\}_{i}$ determines
  a commutative $\dagger$-Frobenius algebra via \[\Delta:H\to H\*H: \ket{\phi_{i}}\mapsto
  \ket{\phi_{i}} \* \ket{\phi_{i}}\qquad\epsilon : H\to \complex : \ket{\phi_{i}} \mapsto 1\] and these
  constructions are inverse to each other.
\end{theorem}

% subsection bases_and_frobenius_algebras (end)

\subsection{Quantum and classical data}\label{sec:quantumclassical}
In \cite{coecke08structures}, Coecke et.al. build on the results of \cite{coeckeetal08:ortho}
to start from a $\dagger$-symmetric monoidal category and construct the minimal machinery needed to
model quantum and classical computations. For the rest of this section, $\cD$ will be assumed to be
such a category, with $\*$ the monoid tensor and $I$ the unit of the monoid.

\begin{definition}\label{def:compact_structure}
  A compact structure on an object $A$ in the category $\cD$ is given by the object $A$, an object
  $A^{*}$ called its \emph{dual} and the maps $\eta:I \to A^{*}\* A$, $\epsilon: A\* A^{*} \to I$
  such that the diagrams
  \[
    \xymatrix@C+20pt{
      A^{*} \ar[dr]^{id} \ar[d]_{\eta\*A^{*}} \\
      A^{*} \*A\*A^{*}  \ar[r]_(.6){A^{*} \*\epsilon} & A^{*}
    }
    \text{ and }
    \xymatrix@C+20pt{
      A \ar[r]^(.4){A\*\eta} \ar[dr]_{id} & A\* A^{*}\* A \ar[d]^{\epsilon\*A}\\
      & A
    }
  \]
  commute.
\end{definition}

\begin{definition}\label{def:quantumstructure}
  A \emph{quantum structure} is an object $A$ and map $\eta:I\to A\*A$ such that
  $(A,A,\eta,\dgr{\eta})$ form a compact structure.
\end{definition}
Note that $A$ is self-dual in definition \ref{def:quantumstructure}.

This allows the creation of the category $\cD_{q}$ which has as objects quantum structures and maps
are the maps in $\cD$ between the objects in the quantum structures.

In the category $\cD_{q}$, it is now possible to define the upper and lower $*$ operations on maps,
such that $(f_{*})^{*}= (f^{*})_{*} = \dgr{f}$:
\begin{eqnarray*}
&f^{*} &:= (\eta_{A}\*1) (1 \* f\*1) (1\*\dgr{\eta}_{B}),\\
&f_{*} &:= (\eta_{B}\*1) (1 \* \dgr{f}\*1) (1\*\dgr{\eta}_{A}).
\end{eqnarray*}

Next, define a classical structure on \cD.
\begin{definition}\label{def:classicalstructure}
  A \emph{classical structure} in \cD{} is an object $X$ together with two maps, $\Delta :X \to X\* X$,
  $\epsilon:X\to I$ such that $(X,\dgr{\Delta},\dgr{\epsilon},\Delta,\epsilon)$ forms a special
  Frobenius algebra.
\end{definition}

As above, this allows us to define $\cD_{c}$, the category whose objects are the classical
structures of $\cD$. The maps in $\cD_{c}$ are given by the maps in $\cD$ between the
objects of the classical structure.

Note that a classical structure will induce a quantum structure, setting $\eta_{X}$ to be
$\dgr{\epsilon_{X}}\, \Delta_{X}$.


Later on, in \ref{sec:the_category_of_commutative_frobenius_algebras}, we will show that commutative
special Frobenius algebras possess a specialized inverse category structure.
% subsection quantum_and_classical_data (end)


% section frobenius_algebras (end)

%%% Local Variables:
%%% mode: latex
%%% TeX-master: "../../phd-thesis"
%%% End:


\subsection{\CFrob is an inverse category}\label{ssec:cfrob_x_is_an_inverse_category}
\begin{example}[Commutative separable Frobenius algebras\cite{kock04}]\label{example:commfrob}
  Let \X be a symmetric monoidal category and form \CFrob as follows: \paragraph{\textbf{Objects:}}
  Commutative separable Frobenius algebras, a quintuple $(A,\nabla,\eta,\Delta,\epsilon)$ where
  $A$ is an object of \X with the following maps:
  $\nabla :A\*A \to A$, $\eta:I\to A$, $\Delta : A \to A\*A$, $\epsilon : A \to I$ which are natural
  maps in \X, with $(A,\nabla,\eta)$ a monoid and $(A,\Delta,\epsilon)$ a comonoid. Additionally,
  these satisfy
  \[
    \xymatrix @C=10pt @R=20pt{
      A \* A \ar[dd]_{1\*\Delta} \ar[dr]^{\nabla}
        \ar[rr]^{\Delta \* 1} & &
        A \* (A \* A) \ar[dd]^{1 \* \nabla}\\
      & A \ar[dr]^{\Delta} & \\
      (A \* A) \* A \ar[rr]_{\nabla \* 1} & &
        A \* A\\
      &*!<0pt,-25pt>{\text{\textbf{Frobenius}}}
    }
  \]
  together with the additional property that $\Delta \nabla = 1$ (separable).

  \paragraph{\textbf{Maps:}} The maps of \X between the objects of \X which preserve multiplication ($\nabla$)
  and comultiplication ($\Delta$) but do not necessarily preserve the units.
  This means a map $f$ must satisfy the following commuting diagrams:
  \[
    \xymatrix@C+25pt{
      A \ar[d]_{\Delta} \ar[r]^{f} & B \ar[d]^{\Delta}\\
      A\*A \ar[r]_{f\*f} & B\* B
    }
    \text{ and }
    \xymatrix@C+25pt{
      A\*A \ar[d]_{\nabla} \ar[r]^{f\*f}& B\*B \ar[d]^{\nabla}\\
      A \ar[r]_{f} & B.
    }
  \]
\end{example}

\begin{lemma}\label{lem:cfrobx_is_an_inverse_category}
  When \X is a symmetric monoidal category, \CFrob is an inverse category.
\end{lemma}
\begin{proof}
  We need to show that \CFrob has restrictions and that each map has a partial inverse. We do
  this by exhibiting the partial inverse of a map.
  For $f:X \to Y$, define $\inv{f}$ as
  \[
    Y \xrightarrow{1\*\eta} Y\*X \xrightarrow{1\*\Delta}
      Y\*X\*X \xrightarrow{1\*f\*1} Y\*Y\*X \xrightarrow{\nabla\*1}
      Y\*X \xrightarrow{\epsilon\*1}X.
  \]
  As a string diagram, this looks like:
  \[
  \begin{tikzpicture}
    \path node at (-.5,3) (start) {}
    node at (0,2.5) [eta] (eta1) {}
    node at (0,2) [delta] (d) {}
    node at (-.25,1.5) [map] (f) {$\scriptstyle f$}
    node at (-.5,1) [nabla] (n1) {}
    node at (-.5,.5) [epsilon] (e1) {}
    node at (0,0) (end) {};
    \draw [] (start) to[out=270,in=125] (n1);
    \draw [] (eta1) to (d);
    \draw [] (d) to[out=305,in=90] (end);
    \draw [] (d) to[out=235,in=90] (f);
    \draw [] (f) to[out=270,in=55] (n1);
    \draw [-] (n1) to (e1);
  \end{tikzpicture}
  \ \raisebox{25pt}{\text{.}}
  \]

  In the following proofs, we also use the following two identities from \cite{kock04}:
  \begin{align}
    (1\*\eta)\nabla &= 1,\\
    \Delta(1\*\epsilon) &= 1.
  \end{align}
  Diagrammatically, this is:
  \[
    \begin{tikzpicture}
    \path   node at (.5,1) (start) {}
    node at (0,1) [eta] (eta1) {}
    node at (.25,.5) [nabla] (n1) {}
    node at (.25,0) (end) {};
    \draw [] (eta1) to[out=270,in=125] (n1);
    \draw [] (start) to[out=270,in=55] (n1);
    \draw [] (n1)   to (end);
  \end{tikzpicture}
  \ \raisebox{15pt}{\text{= }}
  \begin{tikzpicture}
    \path node at (0,1) (start) {}
    node at (0,0) (end) {};
    \draw [-] (start) to (end);
  \end{tikzpicture}
  \ \raisebox{15pt}{\text{=}}
  \begin{tikzpicture}
    \path node at (0,1) (start) {}
    node at (0,.5) [delta] (d1) {}
    node at (-.25,0) (end) {}
    node at (.25,0) [epsilon] (e1) {};
    \draw [] (start) to (d1);
    \draw [] (d1) to[out=305,in=90] (e1);
    \draw [] (d1) to[out=235,in=90] (end);
  \end{tikzpicture}
  \ \raisebox{15pt}{.}
  \]
  Note that when combined with the Frobenius identities, this allows transforms of the following
  types:
  \[
  \begin{tikzpicture}
    \path node at (0,1.5) (s1) {}
    node at (.75,1.5) (s2) {}
    node at (0,1) [delta] (d1) {}
    node at (.5,.5) [nabla] (n1) {}
    node at (0,0) (end) {}
    node at (.5,0) [epsilon] (e1) {};
    \draw [] (s1) to (d1);
    \draw [] (s2) to[out=270,in=55] (n1);
    \draw [] (d1) to[out=235,in=90] (end);
    \draw [] (d1) to[out=305,in=125] (n1);
    \draw [] (n1) to (e1);
  \end{tikzpicture}
  \raisebox{15pt}{$=$}
  \begin{tikzpicture}
    \path node at (0,1.5) (s1) {}
    node at (.5,1.5) (s2) {}
    node at (.25,1) [nabla] (n1) {}
    node at (.25,.5) [delta] (d1) {}
    node at (0,0) (end) {}
    node at (.5,0) [epsilon] (e1) {};
    \draw [] (s1) to[out=270,in=125] (n1);
    \draw [] (s2) to[out=270,in=55] (n1);
    \draw [] (d1) to[out=235,in=90] (end);
    \draw [] (d1) to[out=305,in=90] (e1);
    \draw [] (n1) to (d1);
  \end{tikzpicture}
  \raisebox{15pt}{$=$}
  \begin{tikzpicture}
    \path node at (0,1.5) (s1) {}
    node at (.5,1.5)  (s2) {}
    node at (.25,1) [nabla] (n1) {}
    node at (.25,0.5) (end) {};
    \draw [] (s1) to[out=270,in=125] (n1);
    \draw [] (s2) to[out=270,in=55] (n1);
    \draw [] (n1) to (end);
  \end{tikzpicture}
  \raisebox{15pt}{ and }
  \begin{tikzpicture}
    \path node at (0,1.5)  [eta](s1) {}
    node at (.75,1.5) (s2) {}
    node at (0,1) [delta] (d1) {}
    node at (.5,.5) [nabla] (n1) {}
    node at (0,0) (end) {}
    node at (.5,0) (e1) {};
    \draw [] (s1) to (d1);
    \draw [] (s2) to[out=270,in=55] (n1);
    \draw [] (d1) to[out=235,in=90] (end);
    \draw [] (d1) to[out=305,in=125] (n1);
    \draw [] (n1) to (e1);
  \end{tikzpicture}
  \raisebox{15pt}{$=$}
  \begin{tikzpicture}
    \path node at (0,1.5)  [eta] (s1) {}
    node at (.5,1.5) (s2) {}
    node at (.25,1) [nabla] (n1) {}
    node at (.25,.5) [delta] (d1) {}
    node at (0,0) (end) {}
    node at (.5,0)  (e1) {};
    \draw [] (s1) to[out=270,in=125] (n1);
    \draw [] (s2) to[out=270,in=55] (n1);
    \draw [] (d1) to[out=235,in=90] (end);
    \draw [] (d1) to[out=305,in=90] (e1);
    \draw [] (n1) to (d1);
  \end{tikzpicture}
  \raisebox{15pt}{$=$}
  \begin{tikzpicture}
    \path node at (.25,1.5) (s1) {}
    node at (.25,1) [delta] (d1) {}
    node at (0,0.5) (end) {}
    node at (.5,0.5) (end1) {};
    \draw [] (s1) to (d1);
    \draw [] (d1) to[out=255,in=90] (end);
    \draw [] (d1) to[out=305,in=90] (end1);
  \end{tikzpicture}
  \raisebox{15pt}{.}
  \]

  First, we must show that $\inv{f}$ is a map in the category, i.e., that $\Delta (\inv{f} \*
  \inv{f}) = \inv{f} \Delta$ and $(\inv{f} \* \inv{f})\nabla = \nabla \inv{f}$. We show this for
  $\Delta$ using string diagrams, starting from $\Delta(\inv{f} \*
  \inv{f})$. The proof for the preservation of $\nabla$ proceeds in a similar manner.
  \[
  \begin{tikzpicture}
    \path node at (0,3) (start) {}
    node at (-.75,2.5) [eta] (eta1) {}
    node at (0,2.5) [delta] (d0) {}
    node at (.75,2.5) [eta] (eta2) {}
    node at (-.75,2) [delta] (d) {}
    node at (.75,2) [delta] (d2) {}
    node at (-.5,1.5) [map] (f) {$\scriptstyle f$}
    node at (.5,1.5) [map] (f2) {$\scriptstyle f$}
    node at (-.25,1) [nabla] (n1) {}
    node at (.25,1) [nabla] (n2) {}
    node at (-.25,.5) [epsilon] (e1) {}
    node at (.25,.5) [epsilon] (e2) {}
    node at (-.75,0) (end) {}
    node at (.75,0) (end2) {};
    \draw [] (start) to (d0);
    \draw [] (eta1) to (d);
    \draw [] (eta2) to (d2);
    \draw (d0) to[out=235,in=55] (n1);
    \draw (d0) to[out=305,in=125] (n2);
    \draw [] (d) to[out=235,in=90] (end);
    \draw [] (d) to[out=305,in=90] (f);
    \draw [] (f) to[out=270,in=125] (n1);
    \draw [-] (n1) to (e1);
    \draw [] (d2) to[out=305,in=90] (end2);
    \draw [] (d2) to[out=235,in=90] (f2);
    \draw [] (f2) to[out=270,in=55] (n2);
    \draw [-] (n2) to (e2);
  \end{tikzpicture}
  \raisebox{45pt}{$=$}
  \begin{tikzpicture}
    \path node at (0,3.5) (start) {}
    node at (-.75,2.5) [eta] (eta1) {}
    node at (.75,3) [eta] (eta2) {}
    node at (-.75,2) [delta] (d) {}
    node at (.75,2.5) [delta] (d2) {}
    node at (-.5,1.5) [map] (f) {$\scriptstyle f$}
    node at (.5,2) [map] (f2) {$\scriptstyle f$}
    node at (-.25,1) [nabla] (n1) {}
    node at (.25,1.5) [nabla] (n2) {}
    node at (-.25,.5) [epsilon] (e1) {}
    node at (-.75,0) (end) {}
    node at (.75,0) (end2) {};
    \draw [] (start) to[out=270,in=125] (n2);
    \draw [] (eta1) to (d);
    \draw [] (eta2) to (d2);
    \draw [] (d) to[out=235,in=90] (end);
    \draw [] (d) to[out=305,in=90] (f);
    \draw [] (f) to[out=270,in=125] (n1);
    \draw [-] (n1) to (e1);
    \draw [] (d2) to[out=305,in=90] (end2);
    \draw [] (d2) to[out=235,in=90] (f2);
    \draw [] (f2) to[out=270,in=55] (n2);
    \draw [-] (n2) to[out=270,in=55] (n1);
  \end{tikzpicture}
  \raisebox{45pt}{$=$}
  \begin{tikzpicture}
    \path node at (-0.25,3.5) (start) {}
    node at (-.75,3) [eta] (eta1) {}
    node at (.75,3) [eta] (eta2) {}
    node at (-.75,2.5) [delta] (d) {}
    node at (.75,2.5) [delta] (d2) {}
    node at (-.5,2) [map] (f) {$\scriptstyle f$}
    node at (.5,2) [map] (f2) {$\scriptstyle f$}
    node at (0,1.5) [nabla] (n1) {}
    node at (0.1,1) [nabla] (n2) {}
    node at (0.1,.5) [epsilon] (e1) {}
    node at (-.75,0) (end) {}
    node at (.5,0) (end2) {};
    \draw [] (start) to[out=270,in=55] (n2);
    \draw [] (eta1) to (d);
    \draw [] (eta2) to (d2);
    \draw [] (d) to[out=235,in=90] (end);
    \draw [] (d) to[out=305,in=90] (f);
    \draw [] (f) to[out=270,in=125] (n1);
    \draw [-] (n1) to[out=270,in=125] (n2);
    \draw [] (d2) to[out=305,in=90] (end2);
    \draw [] (d2) to[out=235,in=90] (f2);
    \draw [] (f2) to[out=270,in=55] (n1);
    \draw [-] (n2) to (e1);
  \end{tikzpicture}
  \raisebox{45pt}{$=$}
  \begin{tikzpicture}
    \path node at (.75,3.5) (start) {}
    node at (-.25,3) [eta] (eta1) {}
    node at (.25,3) [eta] (eta2) {}
    node at (-.25,2.5) [delta] (d) {}
    node at (.25,2.5) [delta] (d2) {}
    node at (0,2) [nabla] (n1) {}
    node at (0,1.5) [map] (f) {$\scriptstyle f$}
    node at (0.25,1) [nabla] (n2) {}
    node at (0.25,.5) [epsilon] (e1) {}
    node at (-.5,0) (end) {}
    node at (.75,0) (end2) {};
    \draw [] (start) to[out=270,in=55] (n2);
    \draw [] (eta1) to (d);
    \draw [] (eta2) to (d2);
    \draw [] (d) to[out=235,in=90] (end);
    \draw [] (d) to[out=305,in=125] (n1);
    \draw [] (d2) to[out=305,in=90] (end2);
    \draw [] (d2) to[out=235,in=55] (n1);
    \draw [-] (n1) to[out=270,in=90] (f);
    \draw [] (f) to[out=270,in=125] (n2);
    \draw [-] (n2) to (e1);
  \end{tikzpicture}
  \raisebox{45pt}{$=$}
  \begin{tikzpicture}
    \path node at (.75,3.5) (start) {}
    node at (.25,3) [eta] (eta2) {}
    node at (.25,2.5) [delta] (d2) {}
    node at (-0.25,2) [delta] (d) {}
    node at (0,1.5) [map] (f) {$\scriptstyle f$}
    node at (0.25,1) [nabla] (n2) {}
    node at (0.25,.5) [epsilon] (e1) {}
    node at (-.5,0) (end) {}
    node at (.5,0) (end2) {};
    \draw [] (start) to[out=270,in=55] (n2);
    \draw [] (eta2) to (d2);
    \draw [] (d2) to[out=305,in=90] (end2);
    \draw [] (d2) to[out=235,in=90] (d);
    \draw [] (d) to[out=235,in=90] (end);
    \draw [] (d) to[out=305,in=90] (f);
    \draw [] (f) to[out=270,in=125] (n2);
    \draw [-] (n2) to (e1);
  \end{tikzpicture}
  \ \raisebox{45pt}{$=$}
  \begin{tikzpicture}
    \path node at (.75,3) (start) {}
    node at (0,2.5) [eta] (eta2) {}
    node at (0,2) [delta] (d2) {}
    node at (-0.25,1.5) [delta] (d) {}
    node at (.25,1.5) [map] (f) {$\scriptstyle f$}
    node at (0.5,1) [nabla] (n2) {}
    node at (0.5,.5) [epsilon] (e1) {}
    node at (-.5,0) (end) {}
    node at (0,0) (exit) {};
    \draw [] (start) to[out=270,in=55] (n2);
    \draw [] (eta2) to (d2);
    \draw [] (d2) to[out=235,in=90] (d);
    \draw [] (d2) to[out=305,in=90] (f);
    \draw [] (d) to[out=235,in=90] (end);
    \draw [] (d) to[out=305,in=90] (exit);
    \draw [] (f) to[out=270,in=125] (n2);
    \draw [-] (n2) to (e1);
  \end{tikzpicture}
  \raisebox{45pt}{$=\inv{f}\Delta$.}
  \]
  Thus, $\inv{f}$ is a map in the category whenever $f$ is.

  If $\inv{f}$ is truly a partial inverse, we may then define $\rst{f} = f \inv{f}$.
  Using Theorem 2.20 from \cite{cockett2002:restcategories1}, we need only show:
  \begin{align}
    \inv{(\inv{f})} &= f\label{eq:finvinv_is_f}\\
    f\inv{f}f &= f\label{eq:ffinvf_is_f}\\
    f\inv{f}g\inv{g} &=g\inv{g} f\inv{f}.\label{eq:ffinv_commutes_gginv}
  \end{align}
  Proof of Equation~\ref{eq:finvinv_is_f}: $\inv{(\inv{f})} =$
  \[
  \begin{tikzpicture}
    \path node at (-.75,4) (start) {}
    node at (0,3.5) [eta] (eta2) {}
    node at (0,3) [delta] (d2) {}
    node at (0,2.5) [eta] (eta1) {}
    node at (0,2) [delta] (d1) {}
    node at (-.25,1.5) [map] (f) {$\scriptstyle f$}
    node at (-.5,1) [nabla] (n1) {}
    node at (-.5,.5) [epsilon] (e1) {}
    node at (-.5,0) [nabla] (n2) {}
    node at (-.5,-.5) [epsilon] (e2) {}
    node at (.25,-1) (end) {};
    \draw [] (start) to[out=270,in=125] (n2);
    \draw [] (eta2) to (d2);
    \draw [] (d2) to[out=235,in=125] (n1);
    \draw [] (d2) to[out=305,in=90] (end);
    \draw [] (eta1) to (d1);
    \draw [] (d1) to[out=305,in=55] (n2);
    \draw [] (d1) to[out=235,in=90] (f);
    \draw [] (f) to[out=270,in=55] (n1);
    \draw [-] (n1) to (e1);
    \draw [-] (n2) to (e2);
  \end{tikzpicture}
  \ \raisebox{70pt}{\text{=}}
  \begin{tikzpicture}
    \path node at (.25,4) (start) {}
    node at (-.5,3.5) [eta] (eta2) {}
    node at (-.5,3) [delta] (d2) {}
    node at (0,2.5) [eta] (eta1) {}
    node at (0,2) [delta] (d1) {}
    node at (-.25,1.5) [map] (f) {$\scriptstyle f$}
    node at (-.5,1) [nabla] (n1) {}
    node at (-.5,.5) [epsilon] (e1) {}
    node at (0,0) [nabla] (n2) {}
    node at (0,-.5) [epsilon] (e2) {}
    node at (-.5,-1) (end) {};
    \draw [] (start) to[out=270,in=55] (n2);
    \draw [] (eta2) to (d2);
    \draw [] (d2) to[out=305,in=125] (n1);
    \draw [] (d2) to[out=235,in=90] (end);
    \draw [] (eta1) to (d1);
    \draw [] (d1) to[out=305,in=125] (n2);
    \draw [] (d1) to[out=235,in=90] (f);
    \draw [] (f) to[out=270,in=55] (n1);
    \draw [-] (n1) to (e1);
    \draw [-] (n2) to (e2);
  \end{tikzpicture}
  \ \raisebox{70pt}{\text{= }}
  \begin{tikzpicture}
    \path node at (.25,4) (start) {}
    node at (-.5,1) [eta] (eta2) {}
    node at (-.5,.5) [delta] (d2) {}
    node at (0,3.5) [eta] (eta1) {}
    node at (0,3) [delta] (d1) {}
    node at (-.25,1.5) [map] (f) {$\scriptstyle f$}
    node at (-.25,0) [nabla] (n1) {}
    node at (-.25,-.5) [epsilon] (e1) {}
    node at (.25,2.5) [nabla] (n2) {}
    node at (.25,2) [epsilon] (e2) {}
    node at (-.5,-1) (end) {};
    \draw [] (start) to[out=270,in=55] (n2);
    \draw [] (eta2) to (d2);
    \draw [] (d2) to[out=305,in=125] (n1);
    \draw [] (d2) to[out=235,in=90] (end);
    \draw [] (eta1) to (d1);
    \draw [] (d1) to[out=305,in=125] (n2);
    \draw [] (d1) to[out=235,in=90] (f);
    \draw [] (f) to[out=270,in=55] (n1);
    \draw [-] (n1) to (e1);
    \draw [-] (n2) to (e2);
  \end{tikzpicture}
  \ \raisebox{70pt}{\text{= }}
  \begin{tikzpicture}
    \path node at (.5,4) (start) {}
    node at (0,3.5) [eta] (eta1) {}
    node at (.25,3) [nabla] (n1) {}
    node at (.25,2.5) [delta] (d1) {}
    node at (.5,2) [epsilon] (e1) {}
    node at (0,1.5) [map] (f) {$\scriptstyle f$}
    node at (-.5,1) [eta] (eta2) {}
    node at (-.25,.5) [nabla] (n2) {}
    node at (-.25,0) [delta] (d2) {}
    node at (0,-.5) [epsilon] (e2) {}
    node at (-.5,-1) (end) {};
    \draw [] (start) to[out=270,in=55] (n1);
    \draw [] (eta1) to[out=270,in=125] (n1);
    \draw [] (n1) to (d1);
    \draw [] (d1) to[out=305,in=90] (e1);
    \draw [] (d1) to[out=235,in=90] (f);
    \draw [] (f) to[out=270,in=55] (n2);
    \draw [] (eta2) to[out=270,in=125] (n2);
    \draw [-] (n2) to (d2);
    \draw [] (d2) to[out=305,in=125] (e2);
    \draw [] (d2) to[out=235,in=90] (end);
  \end{tikzpicture}
  \ \raisebox{70pt}{\text{=}}
  \begin{tikzpicture}
    \path node at (.5,4) (start) {}
    node at (0,1.5) [map] (f) {$\scriptstyle f$}
    node at (-.5,-1) (end) {};
    \draw [] (start) to[out=270,in=90] (f);
    \draw [] (f) to[out=270,in=90] (end);
  \end{tikzpicture}
  \ \raisebox{70pt}{\text{= }$f$.}
  \]
  Proof of Equation~\ref{eq:ffinvf_is_f}: $f \inv{f} f =$
  \[
  \begin{tikzpicture}
    \path node at (-.5,3) (start) {}
    node at (0,2.5) [eta] (eta1) {}
    node at (0,2) [delta] (d) {}
    node at (-.75,1.5) [map] (f1) {$\scriptstyle f$}
    node at (-.25,1.5) [map] (f2) {$\scriptstyle f$}
    node at (.25,1.5) [map] (f3) {$\scriptstyle f$}
    node at (-.5,1) [nabla] (n1) {}
    node at (-.5,.5) [epsilon] (e1) {}
    node at (0,0) (end) {};
    \draw [] (start) to[out=270,in=90] (f1);
    \draw [] (f1) to [out=270,in=125] (n1);
    \draw [] (eta1) to (d);
    \draw [] (d) to[out=305,in=90] (f3);
    \draw [] (f3) to[out=270,in=90] (end);
    \draw [] (d) to[out=235,in=90] (f2);
    \draw [] (f2) to[out=270,in=55] (n1);
    \draw [-] (n1) to (e1);
  \end{tikzpicture}
  \ \raisebox{40pt}{$=$ }
  \begin{tikzpicture}
    \path node at (-.5,3) (start) {}
    node at (0,2.5) [eta] (eta1) {}
    node at (0,2) [delta] (d) {}
    node at (-.5,1.5) [nabla] (n1) {}
    node at (-.5,1) [map] (f2) {$\scriptstyle f$}
    node at (.25,1) [map] (f3) {$\scriptstyle f$}
    node at (-.5,.5) [epsilon] (e1) {}
    node at (0,0) (end) {};
    \draw [] (start) to [out=270,in=125] (n1);
    \draw (n1) to (f2);
    \draw (f2) to (e1);
    \draw [] (eta1) to (d);
    \draw [] (d) to[out=305,in=90] (f3);
    \draw [] (f3) to[out=270,in=90] (end);
    \draw [] (d) to[out=235,in=55] (n1);
  \end{tikzpicture}
  \ \raisebox{40pt}{$=$ }
  \begin{tikzpicture}
    \path node at (-.5,3) (start) {}
    node at (0,2.5) [eta] (eta1) {}
    node at (-.25,2) [nabla] (n1) {}
    node at (-.25,1.5) [delta] (d) {}
    node at (-.5,1) [map] (f2) {$\scriptstyle f$}
    node at (0,1) [map] (f3) {$\scriptstyle f$}
    node at (-.5,.5) [epsilon] (e1) {}
    node at (0,0) (end) {};
    \draw [] (start) to[out=270,in=125] (n1);
    \draw [] (eta1) to[out=270,in=55] (n1);
    \draw [] (n1) to (d);
    \draw [] (d) to[out=305,in=90] (f3);
    \draw [] (d) to[out=235,in=90] (f2);
    \draw (f2) to (e1);
    \draw [] (f3) to[out=270,in=90] (end);
  \end{tikzpicture}
  \ \raisebox{40pt}{\text{= }}
  \begin{tikzpicture}
    \path node at (-.5,3) (start) {}
    node at (-.25,1.5) [map] (f3) {$\scriptstyle f$}
    node at (-.25,1) [delta] (d) {}
    node at (-.5,.5) [epsilon] (e1) {}
    node at (0,0) (end) {};
    \draw [] (start) to[out=270,in=90] (f3);
    \draw [] (f3) to (d);
    \draw [] (d) to[out=305,in=90] (end);
    \draw [] (d) to[out=235,in=90] (e1);
  \end{tikzpicture}
  \ \raisebox{40pt}{\text{= }}
  \begin{tikzpicture}
    \path node at (-.5,3) (start) {}
    node at (-.25,1.5) [map] (f3) {$\scriptstyle f$}
    node at (0,0) (end) {};
    \draw [] (start) to[out=270,in=90] (f3);
    \draw [] (f3) to[out=270,in=90] (end);
  \end{tikzpicture}
  \ \raisebox{40pt}{$= f$.}
  \]
  Proof of Equation~\ref{eq:ffinv_commutes_gginv}:  $f\inv{f}g\inv{g} =$

  \[
  \begin{tikzpicture}
    \path node at (-.5,3) (start) {}
    node at (0,2.5) [eta] (eta1) {}
    node at (0,2) [delta] (d1) {}
    node at (1,2.5) [eta] (eta2) {}
    node at (1,2) [delta] (d2) {}
    node at (-.75,1.5) [map] (f1) {$\scriptstyle f$}
    node at (-.25,1.5) [map] (f2) {$\scriptstyle f$}
    node at (.25,1.5) [map] (g1) {$\scriptstyle g$}
    node at (.75,1.5) [map] (g2) {$\scriptstyle g$}
    node at (-.5,1) [nabla] (n1) {}
    node at (-.5,.5) [epsilon] (e1) {}
    node at (.5,1) [nabla] (n2) {}
    node at (.5,.5) [epsilon] (e2) {}
    node at (.75,0) (end) {};
    \draw [] (start) to[out=270,in=90] (f1);
    \draw [] (eta1) to (d1);
    \draw [] (eta2) to (d2);
    \draw [] (d1) to[out=235,in=90] (f2);
    \draw [] (d1) to[out=305,in=90] (g1);
    \draw [] (d2) to[out=235,in=90] (g2);
    \draw [] (d2) to[out=305,in=90] (end);
    \draw [] (f1) to [out=270,in=125] (n1);
    \draw [] (f2) to[out=270,in=55] (n1);
    \draw [] (g1) to[out=270,in=125] (n2);
    \draw [] (g2) to[out=270,in=55] (n2);
    \draw [-] (n1) to (e1);
    \draw [-] (n2) to (e2);
  \end{tikzpicture}
  \ \raisebox{40pt}{$=$ }
  \begin{tikzpicture}
    \path node at (-.5,3) (start) {}
    node at (0,2.5) [eta] (eta1) {}
    node at (0,2) [delta] (d1) {}
    node at (1,2.5) [eta] (eta2) {}
    node at (1,2) [delta] (d2) {}
    node at (-.5,1.5) [nabla] (n1) {}
    node at (.5,1.5) [nabla] (n2) {}
    node at (-.5,1) [map] (f1) {$\scriptstyle f$}
    node at (.5,1) [map] (g1) {$\scriptstyle g$}
    node at (-.5,.5) [epsilon] (e1) {}
    node at (.5,.5) [epsilon] (e2) {}
    node at (.75,0) (end) {};
    \draw [] (start) to[out=270,in=125] (n1);
    \draw [] (eta1) to (d1);
    \draw [] (eta2) to (d2);
    \draw [] (d1) to[out=235,in=55] (n1);
    \draw [] (d1) to[out=305,in=125] (n2);
    \draw [] (d2) to[out=235,in=55] (n2);
    \draw [] (d2) to[out=305,in=90] (end);
    \draw [-] (n1) to (f1);
    \draw [-] (n2) to (g1);
    \draw [] (f1) to (e1);
    \draw [] (g1) to (e2);
  \end{tikzpicture}
  \ \raisebox{40pt}{$=$ }
  \begin{tikzpicture}
    \path node at (-.25,3) (start) {}
    node at (0,2.5) [eta] (eta1) {}
    node at (-.25,2) [nabla] (n1) {}
    node at (-.25,1.5) [delta] (d1) {}
    node at (.5,1.5) [eta] (eta2) {}
    node at (-.5,1) [map] (f1) {$\scriptstyle f$}
    node at (.25,1) [nabla] (n2) {}
    node at (-.5,.5) [epsilon] (e1) {}
    node at (.25,.5) [delta] (d2) {}
    node at (0,0) [map] (g1) {$\scriptstyle g$}
    node at (0,-.5) [epsilon] (e2) {}
    node at (.25,-1) (end) {};
    \draw [] (start) to[out=270,in=125] (n1);
    \draw [] (eta1) to[out=270,in=55] (n1);
    \draw [] (n1) to (d1);
    \draw [] (eta2) to[out=270,in=55] (n2);
    \draw [] (d1) to[out=235,in=90] (f1);
    \draw [] (d1) to[out=305,in=125] (n2);
    \draw [] (f1) to (e1);
    \draw [] (n2) to (d2);
    \draw [] (d2) to[out=235,in=90] (g1);
    \draw [] (d2) to[out=305,in=90] (end);
    \draw [] (g1) to (e2);
  \end{tikzpicture}
  \ \raisebox{40pt}{$=$ }
  \begin{tikzpicture}
    \path node at (-.25,2.5) (start) {}
    node at (-.25,1.5) [delta] (d1) {}
    node at (-.5,.5) [map] (f1) {$\scriptstyle f$}
    node at (-.5,0) [epsilon] (e1) {}
    node at (.25,1) [delta] (d2) {}
    node at (0,.5) [map] (g1) {$\scriptstyle g$}
    node at (0,0) [epsilon] (e2) {}
    node at (.25,-.5) (end) {};
    \draw [] (start) to (d1);
    \draw [] (d1) to[out=235,in=90] (f1);
    \draw [] (d1) to[out=305,in=90] (d2);
    \draw [] (f1) to (e1);
    \draw [] (d2) to[out=235,in=90] (g1);
    \draw [] (d2) to[out=305,in=90] (end);
    \draw [] (g1) to (e2);
  \end{tikzpicture}
  \ \raisebox{40pt}{$=$ }
  \begin{tikzpicture}
    \path node at (-.25,2.5) (start) {}
    node at (.25,2) [delta] (d2) {}
    node at (-.25,1.5) [delta] (d1) {}
    node at (-.5,.5) [map] (f1) {$\scriptstyle f$}
    node at (-.5,0) [epsilon] (e1) {}
    node at (0,.5) [map] (g1) {$\scriptstyle g$}
    node at (0,0) [epsilon] (e2) {}
    node at (.25,-.5) (end) {};
    \draw [] (start) to[out=270,in=90] (d2);
    \draw [] (d2) to[out=235,in=90] (d1);
    \draw [] (d2) to[out=305,in=90] (end);
    \draw [] (d1) to[out=235,in=90] (f1);
    \draw [] (d1) to[out=305,in=90] (g1);
    \draw [] (f1) to (e1);
    \draw [] (g1) to (e2);
  \end{tikzpicture}
  \ \raisebox{40pt}{$=$ }
  \begin{tikzpicture}
    \path node at (-.25,2.5) (start) {}
    node at (.25,2) [delta] (d2) {}
    node at (-.25,1.5) [delta] (d1) {}
    node at (-.5,1) [map] (g1) {$\scriptstyle g$}
    node at (-.5,.5) [epsilon] (e2) {}
    node at (0,1) [map] (f1) {$\scriptstyle f$}
    node at (0,.5) [epsilon] (e1) {}
    node at (.25,0) (end) {};
    \draw [] (start) to[out=270,in=90] (d2);
    \draw [] (d2) to[out=235,in=90] (d1);
    \draw [] (d2) to[out=305,in=90] (end);
    \draw [] (d1) to[out=235,in=90] (g1);
    \draw [] (d1) to[out=305,in=90] (f1);
    \draw [] (f1) to (e1);
    \draw [] (g1) to (e2);
  \end{tikzpicture}
  \ \raisebox{40pt}{$= g\inv{g}f\inv{f}$ }
\]
where the last step is accomplished by reversing all the previous diagrammatic steps.
Hence, \CFrob is an inverse category.
\end{proof}

\begin{theorem}\label{thm:cfrob_is_a_discrete_inverse_category}
  When \X is a symmetric monoidal category, \CFrob is a discrete inverse category.
\end{theorem}
\begin{proof}
  Lemma~\ref{lem:cfrobx_is_an_inverse_category} shows \CFrob is an inverse category. We
  need to show the conditions of Definition~\ref{def:inverse_product} are met.

  First, we see that the tensor of $\X$ is a tensor in \CFrob. $A\*B$ is an object in \CFrob
  with $\Delta_{A\*B} = (\Delta_A\*\Delta_B)(1\*c_{\*}\*1)$,
  $\nabla_{A\*B} =  (1\*c_{\*}\*1)(\nabla_A\*\nabla_B)$,
  $\eta_{A\*B} = \Delta_I(\eta_A \* \eta_B)$, and
  $\epsilon_{A\*B} =  (\epsilon_A\*\epsilon_B)\nabla_I$.

  The map $\Delta : A \to A\*A$ is a map in \CFrob. To show it preserves $\Delta$, we need to
  show $\Delta_A \Delta_{A\*A} = \Delta_A (\Delta_A \* \Delta_A)$:
  \[
  \raisebox{20pt}{$\Delta_A \Delta_{A\*A} =$}
  \begin{tikzpicture}
    \path node at (0,1.5) (start) {}
    node at (0,1) [delta] (d0) {}
    node at (-.25,.5) [delta] (d1) {}
    node at (.25,.5) [delta] (d2) {}
    node at (-.35,0) (e1) {}
    node at (-.15,0) (e2) {}
    node at (.15,0) (e3) {}
    node at (.35,0) (e4) {};
    \draw [] (start) to[out=270,in=90] (d0);
    \draw [] (d0) to[out=235,in=90] (d1);
    \draw [] (d0) to[out=305,in=90] (d2);
    \draw [] (d1) to[out=235,in=90] (e1);
    \draw [] (d1) to[out=305,in=90] (e3);
    \draw [] (d2) to[out=235,in=90] (e2);
    \draw [] (d2) to[out=305,in=90] (e4);
  \end{tikzpicture}
  \raisebox{20pt}{$=$}
  \begin{tikzpicture}
    \path node at (0,2) (start) {}
    node at (-.25,1.5) [delta] (d0) {}
    node at (0,1) [delta] (d1) {}
    node at (.25,.5) [delta] (d2) {}
    node at (-.35,0) (e1) {}
    node at (-.15,0) (e2) {}
    node at (.15,0) (e3) {}
    node at (.35,0) (e4) {};
    \draw [] (start) to[out=270,in=90] (d0);
    \draw [] (d0) to[out=235,in=90] (e1);
    \draw [] (d0) to[out=305,in=90] (d1);
    \draw [] (d1) to[out=235,in=90] (e3);
    \draw [] (d1) to[out=305,in=90] (d2);
    \draw [] (d2) to[out=235,in=90] (e2);
    \draw [] (d2) to[out=305,in=90] (e4);
  \end{tikzpicture}
  \raisebox{20pt}{$=$}
  \begin{tikzpicture}
    \path node at (0,2) (start) {}
    node at (-.25,1.5) [delta] (d0) {}
    node at (.25,1) [delta] (d1) {}
    node at (0,.5) [delta] (d2) {}
    node at (-.35,0) (e1) {}
    node at (-.15,0) (e2) {}
    node at (.15,0) (e3) {}
    node at (.35,0) (e4) {};
    \draw [] (start) to[out=270,in=90] (d0);
    \draw [] (d0) to[out=235,in=90] (e1);
    \draw [] (d0) to[out=305,in=90] (d1);
    \draw [] (d1) to[out=235,in=90] (d2);
    \draw [] (d1) to[out=305,in=90] (e4);
    \draw [] (d2) to[out=235,in=90] (e3);
    \draw [] (d2) to[out=305,in=90] (e2);
  \end{tikzpicture}
  \raisebox{20pt}{$=$}
  \begin{tikzpicture}
    \path node at (0,2) (start) {}
    node at (-.25,1.5) [delta] (d0) {}
    node at (.25,1) [delta] (d1) {}
    node at (0,.5) [delta] (d2) {}
    node at (-.35,0) (e1) {}
    node at (-.15,0) (e2) {}
    node at (.15,0) (e3) {}
    node at (.35,0) (e4) {};
    \draw [] (start) to[out=270,in=90] (d0);
    \draw [] (d0) to[out=235,in=90] (e1);
    \draw [] (d0) to[out=305,in=90] (d1);
    \draw [] (d1) to[out=235,in=90] (d2);
    \draw [] (d1) to[out=305,in=90] (e4);
    \draw [] (d2) to[out=235,in=90] (e2);
    \draw [] (d2) to[out=305,in=90] (e3);
  \end{tikzpicture}
  \raisebox{20pt}{$=\Delta_A (\Delta_A \* \Delta_A).$}
  \]
  Note that in the last step, we simply reverse the various associativity steps used previously.

  To show that $\Delta$ preserves the $\nabla$, we must show that
  $(\Delta_A\*\Delta_A)\nabla_{A\*A} = \nabla_A \Delta_A$. Starting with $(\Delta_A\*\Delta_A)\nabla_{A\*A} =$
  \[
  \begin{tikzpicture}
    \path node at (0,1.5) (s1) {}
    node at (.5,1.5) (s2) {}
    node at (0,1) [delta] (d0) {}
    node at (.5,1) [delta] (d1) {}
    node at (0,.5) [nabla] (n0) {}
    node at (.5,.5) [nabla] (n1) {}
    node at (0,0) (e0) {}
    node at (.5,0) (e1) {};
    \draw [] (s1) to[out=270,in=90] (d0);
    \draw [] (s2) to[out=270,in=90] (d1);
    \draw [] (d0) to[out=235,in=125] (n0);
    \draw [] (d0) to[out=305,in=125] (n1);
    \draw [] (d1) to[out=235,in=55] (n0);
    \draw [] (d1) to[out=305,in=55] (n1);
    \draw [] (n0) to[out=270,in=90] (e0);
    \draw [] (n1) to[out=270,in=90] (e1);
  \end{tikzpicture}
  \raisebox{20pt}{$=$}
  \begin{tikzpicture}
    \path node at (0,2.5) (s1) {}
    node at (.5,2.5) (s2) {}
    node at (.5,2) [delta] (d1) {}
    node at (0,1.5) [nabla] (n0) {}
    node at (0,1) [delta] (d0) {}
    node at (0,.5) [nabla] (n1) {}
    node at (0,0) (e0) {}
    node at (.5,0) (e1) {};
    \draw [] (s1) to[out=270,in=125] (n0);
    \draw [] (s2) to[out=270,in=90] (d1);
    \draw [] (d1) to[out=235,in=55] (n0);
    \draw [] (d1) to[out=305,in=55] (n1);
    \draw [] (n0) to[out=270,in=90] (d0);
    \draw [] (d0) to[out=235,in=125] (n1);
    \draw [] (d0) to[out=305,in=90] (e1);
    \draw [] (n1) to[out=270,in=90] (e0);
  \end{tikzpicture}
  \raisebox{20pt}{$=$}
  \begin{tikzpicture}
    \path node at (0,2.5) (s1) {}
    node at (.25,2.5) (s2) {}
    node at (.25,2) [nabla] (n0) {}
    node at (.25,1.5) [delta] (d1) {}
    node at (0,1) [delta] (d0) {}
    node at (0,.5) [nabla] (n1) {}
    node at (0,0) (e0) {}
    node at (.5,0) (e1) {};
    \draw [] (s1) to[out=270,in=125] (n0);
    \draw [] (s2) to[out=270,in=55] (n0);
    \draw [] (n0) to[out=270,in=90] (d1);
    \draw [] (d1) to[out=235,in=90] (d0);
    \draw [] (d1) to[out=305,in=55] (n1);
    \draw [] (d0) to[out=235,in=125] (n1);
    \draw [] (d0) to[out=305,in=90] (e1);
    \draw [] (n1) to[out=270,in=90] (e0);
  \end{tikzpicture}
  \raisebox{20pt}{$=$}
  \begin{tikzpicture}
    \path node at (0,2.5) (s1) {}
    node at (.25,2.5) (s2) {}
    node at (.25,2) [nabla] (n0) {}
    node at (.25,1.5) [delta] (d1) {}
    node at (0,1) [delta] (d0) {}
    node at (0,.5) [nabla] (n1) {}
    node at (0,0) (e0) {}
    node at (.5,0) (e1) {};
    \draw [] (s1) to[out=270,in=125] (n0);
    \draw [] (s2) to[out=270,in=55] (n0);
    \draw [] (n0) to[out=270,in=90] (d1);
    \draw [] (d1) to[out=235,in=90] (d0);
    \draw [] (d1) to[out=305,in=90] (e1);
    \draw [] (d0) to[out=235,in=125] (n1);
    \draw [] (d0) to[out=305,in=55] (n1);
    \draw [] (n1) to[out=270,in=90] (e0);
  \end{tikzpicture}
  \raisebox{20pt}{$=$}
  \begin{tikzpicture}
    \path node at (0,2.5) (s1) {}
    node at (.5,2.5) (s2) {}
    node at (.25,2) [nabla] (n0) {}
    node at (.25,1.5) [delta] (d1) {}
    node at (0,1) (e0) {}
    node at (.5,1) (e1) {};
    \draw [] (s1) to[out=270,in=125] (n0);
    \draw [] (s2) to[out=270,in=55] (n0);
    \draw [] (n0) to[out=270,in=90] (d1);
    \draw [] (d1) to[out=235,in=90] (e0);
    \draw [] (d1) to[out=305,in=90] (e1);
  \end{tikzpicture}
  \raisebox{20pt}{$= \nabla_A \Delta_A$.}
  \]
  Note that the proof uses the ``special'' property in a non-trivial way.

  Thus, we have a $\Delta$ in \CFrob. As $\nabla = \inv{\Delta}$, the Frobenius requirement for
  the inverse product is immediately fulfilled. Commutativity, cocommutativity, associativity,
  coassociativity and the exchange rule all follow from the properties of the commutative Frobenius
  algebras and therefore \CFrob is a discrete inverse category.
\end{proof}

% section the_category_of_commutative_frobenius_algebras (end)

%%% Local Variables:
%%% mode: latex
%%% TeX-master: "../../phd-thesis"
%%% End:


\section{Disjointness in Frobenius Algebras}
\label{sec:disjointness_in_frobenius_algebras}
\begin{definition}\label{def:perp_in_cfrob}
  As shown in ..., $CFrob(\X)$ is a discrete inverse category. For $f,g:A\to B$, define $f\perp g$
  when
\[
\begin{tikzpicture}
\path node at (0,0) [nabla] (n1) {}
node at (0,2.5) (start) {}
node at (-.5,1) [map] (f) {$\scriptstyle f$}
node at (.5,1) [map] (g) {$\scriptstyle g$}
node at (0,2) [delta] (d) {};
\draw [] (d) to (start);
\draw [] (n1) to (0,-0.5);
\draw [] (d) to[out=305,in=90] (g);
\draw [] (d) to[out=235,in=90] (f);
\draw [-] (f) to[out=270,in=125] (n1);
\draw [-] (g) to[out=270,in=55] (n1);
\end{tikzpicture}
\ \raisebox{45pt}{$= 0.$}\
\]
\end{definition}

\begin{lemma}\label{lem:cfrobperp_is_a_disjointness_relation}
  The relation $\perp$ of Definition~\ref{def:perp_in_cfrob} is a disjointness relation.
\end{lemma}
\begin{proof}
We need to show the seven axioms of the disjointness relation hold. Note that we will show
\axiom{Dis}{6} early on as its result will be used in some of the other axiom proofs.\\
\axiom{Dis}{1}: For all $f:A\to B,\ f\cdperp 0$.\\
\[
\begin{tikzpicture}
\path node at (0,0) [nabla] (n1) {}
node at (0,2.5) (start) {}
node at (-.5,1) [map] (f) {$\scriptstyle f$}
node at (.5,1) [map] (z) {$\scriptstyle 0$}
node at (0,2) [delta] (d) {};
\draw [] (d) to (start);
\draw [] (n1) to (0,-0.5);
\draw [] (d) to[out=235,in=90] (f);
\draw [] (d) to[out=305,in=90] (z);
\draw [-] (f) to[out=270,in=125] (n1);
\draw [-] (z) to[out=270,in=55] (n1);
\end{tikzpicture}
\ \raisebox{45pt}{$=$}\
\begin{tikzpicture}
\path node at (0,0) [nabla] (n1) {}
node at (0,2.5) (start) {}
node at (-.5,1) [map] (f) {$\scriptstyle f$}
node at (.5,.75) [map] (z) {$\scriptstyle 0$}
node at (.5,1.25) [map] (r_z) {$\scriptstyle \rst{0}$}
node at (0,2) [delta] (d) {};
\draw [] (d) to (start);
\draw [] (n1) to (0,-0.5);
\draw [] (d) to[out=235,in=90] (f);
\draw [] (d) to[out=305,in=90] (r_z);
\draw [] (r_z) to (z);
\draw [-] (f) to[out=270,in=125] (n1);
\draw [-] (z) to[out=270,in=55] (n1);
\end{tikzpicture}
\ \raisebox{45pt}{$=$}\
\begin{tikzpicture}
\path node at (0,0) [nabla] (n1) {}
node at (0,2.75) [map] (t_z) {$\scriptstyle 0(=\rst{0})$}
node at (-.5,1) [map] (f) {$\scriptstyle f$}
node at (.5,1) [map] (z) {$\scriptstyle 0$}
node at (0,2) [delta] (d) {};
\draw [] (d) to (t_z);
\draw [] (n1) to (0,-0.5);
\draw [] (d) to[out=235,in=90] (f);
\draw [] (d) to[out=305,in=90] (z);
\draw [-] (f) to[out=270,in=125] (n1);
\draw [-] (z) to[out=270,in=55] (n1);
\end{tikzpicture}
\ \raisebox{45pt}{$= 0.$}
\]
\axiom{Dis}{6}: $f\cdperp g$ implies $\rst{f} \cdperp \rst{g}$ and $\rg{f}\cdperp\rg{g}$.\\
We will show the details of $\rst{f} \cdperp \rst{g}$, using $\rst{f} = f\inv{f}$ and the definition of
$\inv{f}$ as given in Theorem~\ref{thm:cfrob_is_a_discrete_inverse_category}. The proof of $\inv{f}f
= \rg{f} \perp \rg{g} = \inv{g}g$ is similar.
\[
\begin{tikzpicture}
\path node at (0,0) [nabla] (n1) {}
node at (0,2.5) (start) {}
node at (-.5,1) [map] (f) {$\scriptstyle \rst{f}$}
node at (.5,1) [map] (g) {$\scriptstyle \rst{g}$}
node at (0,2) [delta] (d) {};
\draw [] (d) to (start);
\draw [] (n1) to (0,-0.5);
\draw [] (d) to[out=305,in=90] (g);
\draw [] (d) to[out=235,in=90] (f);
\draw [-] (f) to[out=270,in=125] (n1);
\draw [-] (g) to[out=270,in=55] (n1);
\end{tikzpicture}
\ \raisebox{45pt}{$=$}\
\begin{tikzpicture}
    \path node at (0.5,3.5) [delta] (start) {}
    node at (0,2.5) [eta] (eta1) {}
    node at (0,2) [delta] (d) {}
    node at (-1.2,1.5) [map] (f1) {$\scriptstyle f$}
    node at (-.5,1.5) [map] (f) {$\scriptstyle f$}
    node at (-1,1) [nabla] (n1) {}
    node at (-1,.5) [epsilon] (e1) {}
    node at (2,2.5) [eta] (e_tag) {}
    node at (2,2) [delta] (d_g) {}
    node at (.8,1.5) [map] (g1) {$\scriptstyle g$}
    node at (1.5,1.5) [map] (g) {$\scriptstyle g$}
    node at (1,1) [nabla] (n_g) {}
    node at (1,.5) [epsilon] (e_g) {}
    node at (1,-.5) [nabla] (end) {};
    \draw [] (start) to[out=235,in=90] (f1);
    \draw [] (start) to[out=305,in=90] (g1);
    \draw [] (f1) to (n1);
    \draw [] (eta1) to (d);
    \draw [] (d) to (end);
    \draw [] (d) to (f);
    \draw [] (f) to (n1);
    \draw [] (n1) to (e1);
    \draw [] (g1) to (n_g);
    \draw [] (e_tag) to (d_g);
    \draw [] (d_g) to[out=305,in=55] (end);
    \draw [] (d_g) to (g);
    \draw [] (g) to (n_g);
    \draw [] (n_g) to (e_g);
\end{tikzpicture}
\ \raisebox{45pt}{$ = $}\
\begin{tikzpicture}
    \path node at (0,4) [delta] (start) {}
    node at (-1,3.5) [eta] (eta1) {}
    node at (1,3.5) [eta] (eta2) {}
    node at (-1,3) [delta] (d1) {}
    node at (1,3) [delta] (d2) {}
    node at (-.5,2.5) [nabla] (n1) {}
    node at (.5,2.5) [nabla] (n2) {}
    node at (-.5,2) [map] (f1) {$\scriptstyle f$}
    node at (.5,2) [map] (g2) {$\scriptstyle g$}
    node at (-.5,1.5) [epsilon] (e1) {}
    node at (.5,1.5) [epsilon] (e2) {}
    node at (0,.5) [nabla] (n) {};
    \draw [] (start) to (0,4.5);
    \draw [] (start) to (n1);
    \draw [] (start) to (n2);
    \draw [] (eta1) to (d1);
    \draw [] (eta2) to (d2);
    \draw [] (d1) to[out=235,in=125] (n);
    \draw [] (d1) to (n1);
    \draw [] (d2) to (n2);
    \draw [] (d2) to[out=305,in=55] (n);
    \draw [] (n1) to (f1);
    \draw [] (n2) to (g2);
    \draw [] (f1) to (e1);
    \draw [] (g2) to (e2);
    \draw [] (n)  to (0,0);
\end{tikzpicture}
\ \raisebox{45pt}{$=$}\
\begin{tikzpicture}
    \path node at (0,4) [delta] (start) {}
    node at (-.5,3) [delta] (d1) {}
    node at (.5,3) [delta] (d2) {}
    node at (-.25,2) [map] (f1) {$\scriptstyle f$}
    node at (.25,2) [map] (g2) {$\scriptstyle g$}
    node at (-.25,1.5) [epsilon] (e1) {}
    node at (.25,1.5) [epsilon] (e2) {}
    node at (0,0) [nabla] (n) {};
    \draw [] (start) to (0,4.5);
    \draw [] (start) to (d1);
    \draw [] (start) to (d2);
    \draw [] (d1) to[out=235,in=125] (n);
    \draw [] (d1) to (f1);
    \draw [] (d2) to (g2);
    \draw [] (d2) to[out=305,in=55] (n);
    \draw [] (f1) to (e1);
    \draw [] (g2) to (e2);
    \draw [] (n)  to (0,-.5);
\end{tikzpicture}
\ \raisebox{45pt}{$=$}
\]
\[
\begin{tikzpicture}
    \path node at (0,4) [delta] (start) {}
    node at (.5,3.5) [delta] (d2) {}
    node at (0,2.5) [delta] (d1) {}
    node at (-.25,2) [map] (f1) {$\scriptstyle f$}
    node at (.25,2) [map] (g2) {$\scriptstyle g$}
    node at (0,1.5) [nabla] (n12) {}
    node at (0,1) [epsilon] (e2) {}
    node at (0,0) [nabla] (n) {};
    \draw [] (start) to (0,4.5);
    \draw [] (start) to[out=235,in=125] (n);
    \draw [] (start) to (d2);
    \draw [] (d2) to[out=235,in=90] (d1);
    \draw [] (d1) to (f1);
    \draw [] (d1) to (g2);
    \draw [] (d2) to[out=305,in=55] (n);
    \draw [] (f1) to (n12);
    \draw [] (g2) to (n12);
    \draw [] (n12) to (e2);
    \draw [] (n)  to (0,-.5);
\end{tikzpicture}
\ \raisebox{45pt}{$=$}\
\begin{tikzpicture}
    \path node at (0,3) [delta] (start) {}
    node at (.5,2.5) [delta] (d2) {}
    node at (0,1.75) [map] (z) {$\scriptstyle 0$}
    node at (0,1) [epsilon] (e2) {}
    node at (0,0) [nabla] (n) {};
    \draw [] (start) to (0,3.5);
    \draw [] (start) to[out=235,in=125] (n);
    \draw [] (start) to (d2);
    \draw [] (d2) to[out=235,in=90] (z);
    \draw [] (d2) to[out=270,in=55] (n);
    \draw [] (z) to (e2);
    \draw [] (n)  to (0,-.5);
\end{tikzpicture}
\ \raisebox{45pt}{$=$}\
\begin{tikzpicture}
\path node at (0,0) [nabla] (n1) {}
node at (0,2.5) (start) {}
node at (.5,1) [map] (z) {$\scriptstyle 0$}
node at (0,2) [delta] (d) {};
\draw [] (d) to (start);
\draw [] (n1) to (0,-0.5);
\draw [] (d) to[out=305,in=90] (z);
\draw [] (d) to[out=235,in=125] (n1);
\draw [-] (z) to[out=270,in=55] (n1);
\end{tikzpicture}
\ \raisebox{45pt}{$= 0.$}
\]
\axiom{Dis}{2}: $f\cdperp g$ implies $\rst{f} g = 0$.\\
In this proof, we use the result of \axiom{Dis}{6}, i.e., that $\rst{f}\perp\rst{g}$.
\[
\raisebox{25pt}{
\begin{tikzpicture}
  \begin{pgfonlayer}{nodelayer}
    \node [style=map] (0) at (-2, 1) {$\scriptstyle \rst{f}$};
    \node [style=map] (1) at (-2, 0) {$\scriptstyle g$};
    \end{pgfonlayer}
    \begin{pgfonlayer}{edgelayer}
      \draw (0) to (1);
      \draw (0) to (-2,1.5);
      \draw (1) to (-2, -.5);
      \end{pgfonlayer}
\end{tikzpicture}
}
\ \raisebox{45pt}{$=$}\
\begin{tikzpicture}
  \begin{pgfonlayer}{nodelayer}
    \node [style=map] (0) at (-2, 3) {$\scriptstyle \rst{f}$};
    \node [style=map] (1) at (-2, 2.25) {$\scriptstyle g$};
    \node [style=delta] (2) at (-1.25, 4) {};
    \node [style=nabla] (3) at (-1.25, 1.25) {};
    \node [style=map] (4) at (-0.5, 3) {$\scriptstyle \rst{f}$};
    \node [style=map] (5) at (-0.5, 2.25) {$\scriptstyle g$};
    \end{pgfonlayer}
    \begin{pgfonlayer}{edgelayer}
      \draw (0) to (1);
      \draw (1) to[out=270,in=125] (3);
      \draw (5) to[out=270,in=55] (3);
      \draw (4) to (5);
      \draw (2) to[out=305,in=90] (4);
      \draw (2) to[out=235,in=90] (0);
      \draw (2) to (-1.25,4.5);
      \draw (3) to (-1.25,.75);
      \end{pgfonlayer}
\end{tikzpicture}
\ \raisebox{45pt}{$=$}\
\begin{tikzpicture}
  \begin{pgfonlayer}{nodelayer}
    \node [style=map] (0) at (-2, 3) {$\scriptstyle \rst{f}$};
    \node [style=map] (1) at (-2, 2.25) {$\scriptstyle g$};
    \node [style=delta] (2) at (-1.25, 4) {};
    \node [style=nabla] (3) at (-1.25, 1.25) {};
    \node [style=map] (4) at (-0.5, 2.25) {$\scriptstyle g$};
    \end{pgfonlayer}
    \begin{pgfonlayer}{edgelayer}
      \draw (0) to (1);
      \draw (1) to[out=270,in=125] (3);
      \draw (4) to[out=270,in=55] (3);
      \draw (2) to[out=235,in=90] (0);
      \draw (2) to[out=305,in=90] (4);
      \draw (3) to (-1.25,.75);
      \draw (2) to (-1.25,4.5);
      \end{pgfonlayer}
\end{tikzpicture}
\ \raisebox{45pt}{$=$}\
\begin{tikzpicture}
  \begin{pgfonlayer}{nodelayer}
    \node [style=map] (0) at (-2, 2.75) {$\scriptstyle \rst{f}$};
    \node [style=map] (1) at (-1.25, 0.75) {$\scriptstyle g$};
    \node [style=delta] (2) at (-1.25, 4) {};
    \node [style=nabla] (3) at (-1.25, 1.5) {};
    \node [style=map] (4) at (-0.5, 2.75) {$\scriptstyle \rst{g}$};
    \end{pgfonlayer}
    \begin{pgfonlayer}{edgelayer}
      \draw (4) to[out=270,in=55] (3);
      \draw (2) to[out=235,in=90] (0);
      \draw (2) to[out=305,in=90] (4);
      \draw (0) to[out=270,in=125] (3);
      \draw (3) to (1);
      \draw (1) to (-1.25,.25);
      \draw (2) to (-1.25,4.5);
      \end{pgfonlayer}
\end{tikzpicture}
\ \raisebox{45pt}{$=$}\
\raisebox{25pt}{
\begin{tikzpicture}
  \begin{pgfonlayer}{nodelayer}
    \node [style=map] (0) at (-1.25, 1.75) {$\scriptstyle 0$};
    \node [style=map] (1) at (-1.25, 0.75) {$\scriptstyle g$};
    \end{pgfonlayer}
    \begin{pgfonlayer}{edgelayer}
      \draw (0) to (1);
      \draw (1) to (-1.25,.25);
      \draw (0) to (-1.25,2.25);
      \end{pgfonlayer}
\end{tikzpicture}
}
\ \raisebox{45pt}{$=0$}
\]
\axiom{Dis}{3}: $f\cdperp g,\ f' \le f,\ g' \le g$ implies $f' \cdperp g'$.\\
\[
\begin{tikzpicture}
\path node at (0,0) [nabla] (n1) {}
node at (0,2.5) (start) {}
node at (-.5,1) [map] (f) {$\scriptstyle f'$}
node at (.5,1) [map] (g) {$\scriptstyle g'$}
node at (0,2) [delta] (d) {};
\draw [] (d) to (start);
\draw [] (n1) to (0,-0.5);
\draw [] (d) to[out=305,in=90] (g);
\draw [] (d) to[out=235,in=90] (f);
\draw [-] (f) to[out=270,in=125] (n1);
\draw [-] (g) to[out=270,in=55] (n1);
\end{tikzpicture}
\ \raisebox{45pt}{$=$}\
\begin{tikzpicture}
  \begin{pgfonlayer}{nodelayer}
    \node [style=map] (0) at (-2, 3) {$\scriptstyle \rst{f'}$};
    \node [style=map] (1) at (-2, 2.25) {$\scriptstyle f$};
    \node [style=delta] (2) at (-1.25, 4) {};
    \node [style=nabla] (3) at (-1.25, 1.25) {};
    \node [style=map] (4) at (-0.5, 3) {$\scriptstyle \rst{g'}$};
    \node [style=map] (5) at (-0.5, 2.25) {$\scriptstyle g$};
    \end{pgfonlayer}
    \begin{pgfonlayer}{edgelayer}
      \draw (0) to (1);
      \draw (1) to[out=270,in=125] (3);
      \draw (5) to[out=270,in=55] (3);
      \draw (4) to (5);
      \draw (2) to[out=305,in=90] (4);
      \draw (2) to[out=235,in=90] (0);
      \draw (2) to (-1.25,4.5);
      \draw (3) to (-1.25,.75);
      \end{pgfonlayer}
\end{tikzpicture}
\ \raisebox{45pt}{$=$}\
\begin{tikzpicture}
\path
node at (0,3) (start) {}
node at (0,2.5) [map] (fg) {$\scriptstyle \rst{f'}\,\rst{g'}$}
node at (0,2) [delta] (d) {}
node at (-.5,1) [map] (f) {$\scriptstyle f$}
node at (.5,1) [map] (g) {$\scriptstyle g$}
 node at (0,0) [nabla] (n1) {};
\draw [] (d) to (fg);
\draw [] (start) to (fg);
\draw [] (n1) to (0,-0.5);
\draw [] (d) to[out=305,in=90] (g);
\draw [] (d) to[out=235,in=90] (f);
\draw [-] (f) to[out=270,in=125] (n1);
\draw [-] (g) to[out=270,in=55] (n1);
\end{tikzpicture}
\ \raisebox{45pt}{$=$}\
\raisebox{25pt}{
\begin{tikzpicture}
\path
node at (0,1.5) (start) {}
node at (0,1) [map] (fg) {$\scriptstyle \rst{f'}\,\rst{g'}$}
node at (0,.5) [map] (z) {$\scriptstyle 0$};
\draw [] (z) to (fg);
\draw [] (start) to (fg);
\draw [] (z) to (0,0);
\end{tikzpicture}
}
\ \raisebox{45pt}{$=0$}
\]
\axiom{Dis}{4}: $f\cdperp g$ implies $g \cdperp f$.\\
This follows directly from the co-commutativity of $\Delta$.\\
\axiom{Dis}{5}: $f\cdperp g$ implies $h f \cdperp h g$.\\
This follows directly from the naturality of $\Delta$.\\
\axiom{Dis}{7}: $\rst{f}\cdperp \rst{g},\ \rg{h}\cdperp \rg{k}$ implies $f h \cdperp g k$.\\
\[
\begin{tikzpicture}
\path node at (0,0) [nabla] (n1) {}
node at (0,2.5) (start) {}
node at (-.5,1) [map] (fh) {$\scriptstyle f h$}
node at (.5,1) [map] (gk) {$\scriptstyle g k$}
node at (0,2) [delta] (d) {};
\draw [] (d) to (start);
\draw [] (n1) to (0,-0.5);
\draw [] (d) to[out=305,in=90] (gk);
\draw [] (d) to[out=235,in=90] (fh);
\draw [-] (fh) to[out=270,in=125] (n1);
\draw [-] (gk) to[out=270,in=55] (n1);
\end{tikzpicture}
\ \raisebox{45pt}{$=$}\
\begin{tikzpicture}
\path node at (0,0) [nabla] (n1) {}
node at (0,2.5) (start) {}
node at (-.5,1.5) [map] (rf) {$\scriptstyle \rst{f}$}
node at (-.5,1) [map] (fh) {$\scriptstyle f h$}
node at (-.5,.5) [map] (rngh) {$\scriptstyle \rg{h}$}
node at (.5,1.5) [map] (rg) {$\scriptstyle \rst{g}$}
node at (.5,1) [map] (gk) {$\scriptstyle g k$}
node at (.5,.5) [map] (rngk) {$\scriptstyle \rg{k}$}
node at (0,2) [delta] (d) {};
\draw [] (d) to (start);
\draw [] (n1) to (0,-0.5);
\draw [] (d) to[out=305,in=90] (rg);
\draw [] (d) to[out=235,in=90] (rf);
\draw (rf) to (fh);
\draw (fh) to (rngh);
\draw (rg) to (gk);
\draw (gk) to (rngk);
\draw [-] (rngh) to[out=270,in=125] (n1);
\draw [-] (rngk) to[out=270,in=55] (n1);
\end{tikzpicture}
\ \raisebox{45pt}{$=$}\
\begin{tikzpicture}
\path node at (0,0) [nabla] (n1) {}
node at (0,2.5) (start) {}
node at (-.5,1) [map] (fh) {$\scriptstyle f h$}
node at (-.5,.5) [map] (rnghrngk) {$\scriptstyle \rg{h}\rg{k}$}
node at (.5,1.5) [map] (rfrg) {$\scriptstyle \rst{f}\rst{g}$}
node at (.5,1) [map] (gk) {$\scriptstyle g k$}
node at (0,2) [delta] (d) {};
\draw [] (d) to (start);
\draw [] (n1) to (0,-0.5);
\draw [] (d) to[out=305,in=90] (rfrg);
\draw [] (d) to[out=235,in=90] (fh);
\draw (fh) to (rnghrngk);
\draw (rfrg) to (gk);
\draw [-] (rnghrngk) to[out=270,in=125] (n1);
\draw [-] (gk) to[out=270,in=55] (n1);
\end{tikzpicture}
\ \raisebox{45pt}{$=$}\
\begin{tikzpicture}
\path node at (0,0) [nabla] (n1) {}
node at (0,2.5) (start) {}
node at (-.5,1) [map] (fh) {$\scriptstyle f h$}
node at (-.5,.5) [map] (rnghrngk) {$\scriptstyle 0$}
node at (.5,1.5) [map] (rfrg) {$\scriptstyle 0$}
node at (.5,1) [map] (gk) {$\scriptstyle g k$}
node at (0,2) [delta] (d) {};
\draw [] (d) to (start);
\draw [] (n1) to (0,-0.5);
\draw [] (d) to[out=305,in=90] (rfrg);
\draw [] (d) to[out=235,in=90] (fh);
\draw (fh) to (rnghrngk);
\draw (rfrg) to (gk);
\draw [-] (rnghrngk) to[out=270,in=125] (n1);
\draw [-] (gk) to[out=270,in=55] (n1);
\end{tikzpicture}
\ \raisebox{45pt}{$=$}\
\begin{tikzpicture}
\path node at (0,0) [nabla] (n1) {}
node at (0,2.5) (start) {}
node at (-.5,1) [map] (fh) {$\scriptstyle 0$}
node at (.5,1) [map] (gk) {$\scriptstyle 0$}
node at (0,2) [delta] (d) {};
\draw [] (d) to (start);
\draw [] (n1) to (0,-0.5);
\draw [] (d) to[out=305,in=90] (gk);
\draw [] (d) to[out=235,in=90] (fh);
\draw [-] (fh) to[out=270,in=125] (n1);
\draw [-] (gk) to[out=270,in=55] (n1);
\end{tikzpicture}
\ \raisebox{45pt}{$= 0.$}\
\]
\end{proof}

%%% Local Variables:
%%% mode: latex
%%% TeX-master: "../../phd-thesis"
%%% End:


\section{Disjoint joins in \CFrob}
\label{sec:disjoint-joins-in-frobenius-algebras}

In the previous sections, we have shown that \CFrob is a discrete inverse category, with a
disjointness relation whenever \X is a symmetric monoidal category with zero maps.

We now show that if \X has biproducts and is an additive tensor category, i.e., a symmetric monoidal
category where the hom-sets are enriched in additive monoids, then \CFrob will have a disjoint
join. Moreover, in the following section, we shall show it possesses a disjoint sum. First, we
explicitly define additive tensor category:

\begin{definition}\label{def:additive-tensor-category}
  Suppose \X is a symmetric monoidal category with zero maps and the hom-sets are enriched in
  additive monoids. It is an \emph{additive tensor category}  when:
  \begin{itemize}
    \item $h(f+g)k = h f k + h g k$ for all $h:A\to B$, $f,g:B\to C$ and $k:C \to  D$;
    \item $(f+g)\*k = f\*k + g\* k$ for all $f,g:A\to B$ and  $k:C\to D$.
  \end{itemize}
\end{definition}

We begin by showing that given an additive tensor category, we may form an equivalent category which has
biproducts.

\begin{lemma}\label{lem:axb_is_a_biproduct}
  Suppose $\X$ is an additive symmetric monoidal category. If we have the diagram
  \[
      \xymatrix{
      A\ar[r]^{\sigma_1} &X\ar@/^9pt/[l]^{\pi_1} \ar@/_9pt/[r]_{\pi_2} & B \ar[l]_{\sigma_2}
    }
  \]
  with
  \[
    \sigma_1\pi_1 = 1_A,\quad\sigma_2\pi_2 = 1_B,\qquad \sigma_1\pi_2 = 0 = \sigma_2\pi_2
  \]
  and $\pi_1\sigma_1 + \pi_2\sigma_2 = 1$, then $X$ is a biproduct of $A$ and $B$.
\end{lemma}

\begin{corollary}\label{cor:functor_preserves_biproducts}
  If $F:\X\to\Y$ is an additively enriched functor, then $F$ preserves all biproducts.
\end{corollary}

\begin{corollary}\label{cor:tensor_preserves_biproducts}
  The functor $A\*\_:\X \to \X$ preserves biproducts.
\end{corollary}

Thus, if $\X$ is additively enriched, we may add biproducts by moving to the matrix category of
$\X$, defined as:
\category{Lists of the objects $[A_i]$ of $\X$}{Matrices of maps in $\X$, $[f_{i,
    j}]:[A_i]\to[B_j]$}{The diagonal matrix $I$ ($f_{i,i} = 1_{A_i}$ and $f_{i,j} = 0, i \ne
  j$)}{Matrix multiplication}

Now, let us consider \CFrob where \X is an additive tensor category with
biproducts. We know from Lemma~\ref{lem:cfrobperp_is_a_disjointness_relation} that
\[
\raisebox{40pt}{$f\perp g \iff$}\
\begin{tikzpicture}
\path node at (0,0) [nabla] (n1) {}
node at (0,2.5) (start) {}
node at (-.5,1) [map] (f) {$\scriptstyle f$}
node at (.5,1) [map] (g) {$\scriptstyle g$}
node at (0,2) [delta] (d) {};
\draw [] (d) to (start);
\draw [] (n1) to (0,-0.5);
\draw [] (d) to[out=305,in=90] (g);
\draw [] (d) to[out=235,in=90] (f);
\draw [-] (f) to[out=270,in=125] (n1);
\draw [-] (g) to[out=270,in=55] (n1);
\end{tikzpicture}
\ \raisebox{40pt}{$= 0.$}\
\]

We will now show that the biproduct is the disjoint join of any two disjoint maps. To do so, we
must show the biproduct of two disjoint maps is in the category \CFrob. We first give a lemma
about the biproduct of disjoint maps.

\begin{lemma}\label{lem:delta_disjoint_is_zero}
  Given \X is an additive tensor category, when $f,g$ are maps in \CFrob with  $f\perp g$, then
  $\Delta(f\*g) = 0$ and $(f\*g)\nabla = 0$.
\end{lemma}
\begin{proof}
  From
  Lemma~\ref{lem:properties_of_delta_and_tensor_in_a_discrete_inverse_category}~\ref{le:deltaefg}
  we have that $e\Delta (f\*g) = \Delta(e f\*g)$ and $(f\*g)\inv{\Delta}e = (f\*g e)\inv{\Delta}$ for
  $e$ a restriction idempotent. Thus, we have
  \[
    \Delta(f\*g) = \rst{f} \Delta(f\*g) = \Delta(f\*\rst{f} g) = \Delta(f \* 0) = \Delta 0 = 0,
  \]
  where $\rst{f}g = 0$ follows from the disjointness of $f,g$ by \axiom{Dis}{2} and the remaining
  equalities are due to \X being an additive tensor category. Dually, as $\nabla = \inv{\Delta}$, we
  have
  \[
    (f\*g)\nabla = (f\*g)\nabla\rg{f} = (f\*g\rg{f})\nabla = (f \* 0)\nabla = 0\nabla = 0.
  \]
\end{proof}

\begin{lemma}\label{lem:biproduct-is-in-cfrob-x}
  Given \X is an additive tensor category with biproducts, when $f,g$ are maps in \CFrob with
  $f\perp g$, then $f+g$ is a map in \CFrob.
\end{lemma}
\begin{proof}
  We must show $(f+g)\Delta = \Delta((f+g)\*(f+g))$ and $\nabla(f+g) = ((f+g)\*(f+g))\nabla$. As
  this is an additive tensor category and both $f,g$ are in \CFrob and using
  Lemma~\ref{lem:delta_disjoint_is_zero}, we have
  \begin{multline*}
    \Delta((f+g)\*(f+g)) = \Delta( f\*f + f\* g + g\* f + g\*g) = \\
    \Delta(f\*f) + \Delta(f\*g) + \Delta(g\*f) + \Delta(g\*g) = \\
    \Delta(f\*f) + \Delta(g\*g) = f\Delta + g \Delta = (f+g)\Delta.
  \end{multline*}
  Hence, the biproduct of $f,g$ preserves $\Delta$.
  Similarly, $f+g$ preserves $\nabla$:
  \begin{multline*}
    ((f+g)\*(f+g))\nabla =( f\*f + f\* g + g\* f + g\*g)\nabla = \\
    (f\*f)\nabla +(f\*g) \nabla+(g\*f)\nabla + (g\*g)\nabla = \\
    (f\*f)\nabla + (g\*g)\nabla =\nabla f + \nabla g  = \nabla(f+g).
  \end{multline*}
\end{proof}

\begin{proposition}\label{prop:biproduct_is_the_disjoint_join_in_cfrobx}
  Given \X is an  additive tensor category with biproduct $\+$, and $f\perp g$, then $f\djoin g
  \definedas f+g$ is a disjoint join.
\end{proposition}
\begin{proof}
  We need to show the four axioms of disjoint join from Definition~\ref{def:disjoint_join}.

  \axiom{DJ}{1} : $f \le f \djoin g$ and $g \le f \djoin g$.
  As $f+g = g+f$, we need only show the first part of the axiom. As $f\perp g$, we know $\rst{f}g=0$
  by the definition of disjointness and therefore we have:
  \[
    \rst{f}(f+g) = \rst{f}f + \rst{f} g = f + 0 = f
  \]
  and thus $f \le f \djoin g$ and \axiom{DJ}{1} is true.

  \axiom{DJ}{2}: $f \le h,\ g \le h$ and $f\perp g$ implies $f \djoin g \le h$.  We calculate
  \[
    \rst{f+g}h = (\rst{f}+\rst{g}) h = \rst{f}h + \rst{g} h = f + g
  \]
  giving us the required inequality and \axiom{DJ}{1} is true.

  \axiom{DJ}{3}: Disjoint join is stable, i.e., $h(f \djoin g) = h f \djoin h g$. This is immediate
  as $h(f+g) = hf + h g$.

  \axiom{DJ}{4}: $\cdperp \{f, g, h\} \iff f \perp (g\djoin h)$. Consider the $\Longleftarrow$
  direction first. We immediately have $g\perp h$ as we are able to form the disjoint join. By
  \axiom{DJ}{1}, we have both $g \le g \djoin h$ and $h \le g\djoin h$ and therefore by
  \axiom{Dis}{3}, $f \perp g$ and $f\perp h$. Thus the $\Longleftarrow$ direction is true.

  For the $\implies$ direction, we compute
  \[
  \begin{tikzpicture}
    \path node at (0,0) [nabla] (n1) {}
    node at (0,2.5) (start) {}
    node at (-.5,1) [map] (f) {$\scriptstyle f$}
    node at (.5,1) [map] (g) {$\scriptstyle g+h$}
    node at (0,2) [delta] (d) {};
    \draw [] (d) to (start);
    \draw [] (n1) to (0,-0.5);
    \draw [] (d) to[out=305,in=90] (g);
    \draw [] (d) to[out=235,in=90] (f);
    \draw [-] (f) to[out=270,in=125] (n1);
    \draw [-] (g) to[out=270,in=55] (n1);
  \end{tikzpicture}
  \ \raisebox{40pt}{$=$}\
  \begin{tikzpicture}
    \path node at (0,0) [nabla] (n1) {}
    node at (0,2.5) (start) {}
    node at (-.5,1) [map] (f) {$\scriptstyle f$}
    node at (.5,1) [map] (g) {$\scriptstyle g$}
    node at (0,2) [delta] (d) {};
    \draw [] (d) to (start);
    \draw [] (n1) to (0,-0.5);
    \draw [] (d) to[out=305,in=90] (g);
    \draw [] (d) to[out=235,in=90] (f);
    \draw [-] (f) to[out=270,in=125] (n1);
    \draw [-] (g) to[out=270,in=55] (n1);
  \end{tikzpicture}
  \raisebox{40pt}{$\,+\,$}
  \begin{tikzpicture}
    \path node at (0,0) [nabla] (n1) {}
    node at (0,2.5) (start) {}
    node at (-.5,1) [map] (f) {$\scriptstyle f$}
    node at (.5,1) [map] (g) {$\scriptstyle h$}
    node at (0,2) [delta] (d) {};
    \draw [] (d) to (start);
    \draw [] (n1) to (0,-0.5);
    \draw [] (d) to[out=305,in=90] (g);
    \draw [] (d) to[out=235,in=90] (f);
    \draw [-] (f) to[out=270,in=125] (n1);
    \draw [-] (g) to[out=270,in=55] (n1);
  \end{tikzpicture}
  \ \raisebox{40pt}{$= 0.$}\
  \]
  This gives us both directions and all of the axioms have been shown to be true, hence, the
  addition of maps is a disjoint join.
\end{proof}

\section{Disjoint sums in \CFrob}
\label{sec:disjoint-sums-in-cfrob-x}

Now that we have shown we have a disjoint join, our last remaining task is to show that the
biproduct in \X provides a disjoint sum. For the remainder of this section, we define the following
objects and maps:
\begin{align*}
    A\+B & \definedas A+B \text{ (the biproduct of $A$ and $B$),}\\
    \epsilon_{A+B} &\definedas [\epsilon_A,\epsilon_B]:A+B \to I,\\
    \eta_{A+B}&\definedas \<\eta_A,\eta_B\>:I\to A+B,\\
    \Delta_{A+B} & \definedas (\Delta_A i_1 + \Delta_B i_2):A+B\to A\*A+A\*B+B\*A+B\*B,\\
    \nabla_{A+B} & \definedas (\pi_1 \nabla_A + \pi_2 \nabla_B): (A\*A+A\*B)+(B\*A+B\*B) \to A+B.
\end{align*}
Note that we have $A\*A+A\*B+B\*A+B\*B \cong (A+B)\*(A+B)$ via the isomorphism $b = \<i_1\*i_1 |
i_1\*i_2 | i_2 \* i_1 | i_2\* i_2\>$.


\begin{lemma}\label{lem:biproduct-injections-and-projections-are-in-cfrob}
  Given \X is an additive tensor category with biproduct $+$, then the injection maps $i_1,i_2$ and
  projection maps $\pi_1,\pi_2$ of the biproduct are maps in \CFrob.
\end{lemma}
\begin{proof}
  We  must show each of the injections and projections preserve $\Delta$ and $\nabla$.

  Consider
  \[
     \xymatrix@C+40pt@R+10pt{
       A \ar[d]_{\Delta} \ar[rr]^{i_1} & &A+B \ar[d]^{\Delta_{A+B}(=\Delta_A i_1 + \Delta_B i_2)}\\
       A\*A \ar[rr]_{i_1\*i_1} \ar[dr]_(.425){i_1} & & (A+B)\*(A+B)\\
       &A\*A+A\*B \ar[r]_(.375){i_1} & A\*A+A\*B+B\*A+B\*B \ar[u]_{b}.
     }
  \]
  The outer arrows commute as $i_1(f+g) = f i_1$. The bottom half commutes due to the isomorphism
  hence the top rectangle commutes and $i_1$ preserves $\Delta$.  Similarly, $i_2$ preserves
  $\Delta$. By dualizing the diagram, we also have $\pi_1, \pi_2$ preserve $\nabla$.

  Next, we show the projections preserve $\nabla$,
  \[
    \xymatrix@C+40pt@R+10pt{
       A+B \ar[rr]_{\pi_1}\ar[d]_{\Delta}& & A\ar[d]^{\Delta_A}\\
        (A+B)\*(A+B)  \ar[rr]^{\pi_1\*\pi_1} \ar[d]_{b^{-1}}  & &A\*A\\
       {A\*A+A\*B+B\*A+B\*B}  \ar[r]^(.625){\pi_1}  &A\*A+A\*B \ar[ur]^(.575){\pi_1}
     }
  \]
  which, by the same argument as above, shows that $\Delta$ is preserved by $\pi_1$ and similarly by
  $\pi_2$. Once again, dualizing  this diagram gives us that $\nabla$ is preserved by $i_1$ and $i_2$.
\end{proof}

\begin{lemma}\label{prop:biproducts-of-frobenius-objects-are-frobenius}
  Suppose $A$ and $B$ are Frobenius algebras in \CFrob where \X, an additive tensor category, has
  the biproduct $+$.
  Then $A\+B$ is a Frobenius algebra and is therefore in \CFrob.
\end{lemma}
\begin{proof}

  To show $A\+B$ is a Frobenius algebra and therefore in \CFrob, we must show it is separable,
  the unit laws hold and the Frobenius condition hold for $A\+B$.

  For the requirement that it is separable:
  \begin{align*}
    \Delta_{A+B}\nabla_{A+B}&= (\Delta_A i_1 + \Delta_B i_2) (\pi_1 \nabla_A + \pi_2 \nabla_B)\\
    &=  (\Delta_A i_1 \pi_1 \nabla_A + \Delta_B i_2\pi_2 \nabla_B)\\
    &=  (\Delta_A \nabla_A + \Delta_B \nabla_B)\\
    &=  (1_A + 1_B) = 1_{A+B}.
  \end{align*}


  To show the comultiplication unit law,
  \[
     \xymatrix@C+20pt{
       A \ar[r]^{\Delta} \ar[d]_{i_1} & A\*A \ar[d]^{i_1\*i_1} \ar@{=}[dr] \\
       A+B \ar[r]^(.35){\Delta} \ar@{=}[ddr] &(A+B)\*(A+B) \ar[d]^{1\*\epsilon}
          & A\*A \ar[l]^(.35){i_1\*i_1}  \ar[d]^{1\*\epsilon}\\
       & (A+B)\*I \ar[d]^{\usl} & A\*I \ar[d]^{\usl} \ar[l]^{i_1\*1}\\
       & A+B & A. \ar[l]^{i_1}
     }
  \]
  The outer path shows the unit law for the Frobenius algebra $A$ and is what happens to the $A$
  component in the inner path. There is a similar diagram where the outer path $A$'s are replaced
  with a $B$ and the $i_1$ with $i_2$. As these outer paths commute, they show that the inner path
  commutes as each component of it commutes.

  For the Frobenius law, as we are in a commutative world, we need only show
  $\nabla \Delta = (1\*\Delta)(\nabla\*1)$:
  \[
  \xymatrix@C+17pt{
      &A\*A \ar@{=}[ld] \ar[rr]^{1\*\Delta} \ar[d]^{i_1\*i_1} && A\*A\*A  \ar[d]_{i_1\*i_1\*i_1} \ar@{=}[dr] \\
      A\*A \ar[r]^(.45){i_1\*i_1}  \ar[dr]_{\nabla}
         &\scriptstyle{(A+B)\*(A+B)} \ar[rr]^{1\*\Delta} \ar[dr]^{\nabla}
         &&\scriptstyle{(A+B)\*(A+B)\*(A+B)}  \ar[dd]^{\nabla\*1} & A\*A\*A  \ar[l]^(.4){i_1\*i_1\*i_1}  \ar[dd]^{\nabla\*1}\\
     & A \ar[r]^{i_1} \ar[ddrr]_{\Delta} & \scriptstyle{A+B} \ar[dr]^{\Delta} \\
     & & & \scriptstyle{(A+B)\*(A+B)} & A\*A  \ar[l]_(.45){i_1\*i_1} \ar@{=}[dl]\\
     & & & A\*A. \ar[u]_{i_1\*i_1}
    }
  \]
  By the same reasoning as the previous two arguments, the diagram commutes and $A+B$ is a Frobenius
  algebra.
\end{proof}

Now that we have that $A\+B$ is in \CFrob when, $A,B$ are in \CFrob, we can show that the
biproduct projections and injections are maps in \CFrob.


\begin{proposition}\label{prop:cfrobx_has_disjoint_sums}
  Suppose $A$ and $B$ are Frobenius algebras in \CFrob where \X, an additive tensor category, has
  the biproduct $+$.
  Then $A\+B$ as defined in Lemma~\ref{lem:biproduct-is-in-cfrob-x} is a disjoint sum in \CFrob.
\end{proposition}
\begin{proof}
  Next, we must give maps $i_1,i_2,x_1,x_2$  in \CFrob that satisfy the disjoint sum diagram,
  \begin{equation}
    \xymatrix@C+10pt{
      A\ar[r]^{i_1} &A+B\ar@/^9pt/[l]^{x_1} \ar@/_9pt/[r]_{x_2} & B \ar[l]_{i_2}
    }\label{dia:cfrob-disjoint-sum}
  \end{equation}
  where
  \begin{enumerate}[{(}i{)}]
    \item $i_1$ and $i_2$ are monic,
    \item $\inv{i_1} = \xa$ and $\inv{i_2} = \xb$, and
    \item $\inv{i_1}i_1 \perp \inv{i_2}i_2$ and $\inv{i_1}i_1 \djoin \inv{i_2}i_2 = 1_X$.
  \end{enumerate}

  By Lemma~\ref{lem:biproduct-injections-and-projections-are-in-cfrob}, setting $i_1,i_2$ to be the
  injections of the biproduct and $\xa,\xb$ to be the projections will immediately give us
  Diagram~\ref{dia:cfrob-disjoint-sum}, as all those maps are in \CFrob.

  For the three conditions, as $i_1$ and $i_2$ are total maps, they are monic in the inverse
  category. We know from above that $i_j\pi_j = 1$ and therefore $\inv{i_j} = \pi_j$. Additionally
  we know that $\pi_1i_1 + \pi_2i_2 = 1$, but as $+$ is the disjoint join, this shows the third
  condition is true and $A\+B$ is a disjoint sum.
\end{proof}


%%% Local Variables:
%%% mode: latex
%%% TeX-master: "../phd-thesis"
%%% End:
