%!TEX root = /Users/gilesb/UofC/thesis/phd-thesis/phd-thesis.tex

\chapter{Abstract Computability}\label{chap:abstract_computability}

%\section{Linear algebra} % (fold)
\label{sec:linear_algebra}

Quantum computation requires familiarity with the basics of linear algebra. This section will give
definitions of the terms used throughout this thesis.
\subsection{Basic definitions} % (fold)
\label{sub:basic_definitions}


The first definition needed is that of a \emph{vector space}.

\begin{definition}[Vector Space]
  Given a field $F$, whose elements will be referred to as scalars, a \emph{vector space} over $F$
  is a non-empty set $V$ with two operations, \emph{vector addition} and \emph{scalar
  multiplication}. \emph{Vector addition} is defined as ${+}:V\times V \to V$ and denoted as
  $\vc{v}+\vc{w}$ where $\vc{v},\vc{w}\in V$. The set $V$ must be an abelian group under $+$.
  \emph{Scalar multiplication} is defined as ${}:F\times V \to V$ and denoted as $c\vc{v}$ where
  $c\in F, \vc{v} \in V$. Scalar multiplication distributes over both vector addition and scalar
  addition and is associative. $F$'s multiplicative identity is an identity for scalar
  multiplication.

\end{definition}
The specific algebraic requirements are:
\begin{enumerate}
  \item{}$\forall \vc{u},\vc{v},\vc{w} \in V,\ (\vc{u} +\vc{v}) +\vc{w} =
    \vc{u}+ (\vc{v}+\vc{w})$;
  \item{}$\forall \vc{u},\vc{v} \in V,\ \vc{u} +\vc{v} =
    \vc{v}+ \vc{u}$;
  \item{}$\exists  \vc{0} \in V \mathrm{\ such\ that\ } \forall \vc{v} \in V,
    \vc{0} +\vc{v} =  \vc{v}$;
  \item{}$\forall \vc{u} \in V, \exists \vc{v} \in V \mathrm{\ such\ that\ }
    \vc{u}+ \vc{v} = \vc{0}$;
  \item{}$\forall \vc{u},\vc{v} \in V, c\in F,\
    c(\vc{u}+ \vc{v}) = c\vc{u} + c\vc{v}$;
  \item{}$\forall \vc{u} \in V, c,d\in F,\
    (c+d)\vc{u} = c\vc{u} + d\vc{u}$;
  \item{}$\forall \vc{u} \in V, c,d\in F,\
    (c d)\vc{u} = c(d\vc{u})$;
  \item{}$\forall \vc{u} \in V,\
    1\vc{u} = \vc{u}$.
\end{enumerate}

Examples of vector spaces over $F$ are: $F^{n\times m}$ -- the set of $n\times m$ matrices over
$F$; and $F^n$ -- the $n{-}$fold Cartesian product of $F$. $F^{n\times 1}$, the set of $n\times 1$
matrices over $F$ is also called the space of column vectors, while $F^{1\times n}$, the set of row
vectors. Often, $F^n$ is identified with $F^{n\times 1}$.


This thesis  shall identify $F^n$ with the column vector space over $F$.

\begin{definition}[Linearly independent]
  A subset of vectors $\{\vc{v}_i\}$ of the vector space $V$ is said to be \emph{linearly
  independent} when no finite linear combination of them, $\sum a_j\vc{v}_j$ equals \vc{0} unless
  all the $a_j$ are zero.

\end{definition}

\begin{definition}[Basis]
  A \emph{basis} of a vector space $V$ is a linearly independent subset of $V$ that generates $V$.
  That is, any vector $u \in V$ is a linear combination of the basis vectors.
\end{definition}

\begin{definition}\label{def:linear_map_of_vector_spaces}
  Given $V, W$ are vector spaces over $F$ with $v \in V$ and $s \in F$, then
  if $f:V \to W$ is a group homomorphism such that $f(v s) = f(v) s$, then we say $f$ is a
  \emph{linear map}. Furthermore, a map $f:V\times W \to X$ is called \emph{bilinear} when the
  map $f_v:W\to T$ and $f_w:V\to T$ are linear for each $v\in V$ and $w\in W$, where $f_v$ is the
  map obtained from $f$ by fixing $v\in V$ and $f_w$ is obtained from $f$ by fixing  $w\in W$.
\end{definition}

\begin{definition}\label{def:free_vector_space}
  Given a set $S$, the \emph{free vector space} of $S$ over a field $F$ is the abelian group
  of formal sums $\sum a_i s_i$ where the $s_i$ are the elements of $S$ and $a_i \in F$.
  Formal sums are independent of order. Addition is defined as $(\sum a_i s_i) + (\sum b_i s_i)$ is
  $(\sum (a_i + b_i) s_i)$.
\end{definition}

\begin{definition}\label{def:tensor_product_of_vector_spaces}
  Given vector spaces $V, W$ over the base field $F$, consider the free vector space of $V
  \times W = F(V\times W)$. Next, consider the subspace $T$ of $F(V\times W)$ generated by the
  following equations:
  \begin{align*}
    (v_1,w)+(v_2,w) & = (v_1+v_2,w)\\
    (v,w_1)+(v,w_2) & = (v, w_1+w_2)\\
    s(v,w) &= (s v,w)\\
    s(v,w) &= (v,s w),
  \end{align*}
  where $v,v_1,v_2 \in V$, $w,w_1,w_2 \in W$ and $s\in F$. Then the tensor product of
  $V$ and $W$, written $V\*W$ is $F(V\times W)/T$.
\end{definition}

Elements of the tensor product $V\*W$ are written as $v\*w$ and are the $T$-equivalence class of
$(v,w) \in V\times W$. If $\{v_i\}$ is a basis for $V$ and $\{w_j\}$ is a basis for $W$, then the
elements $\{v_i\*w_j\}$ form a basis for $V\*W$.
% subsection basic_definitions (end)
\subsection{Matrices} % (fold)
\label{sub:matrices}


As mentioned above, the set of $n\times m$ matrices over a field is a vector space. Additionally,
matrices compose and the tensor product of matrices is defined.

Matrix composition is defined as usual. That is, for $A = [a_{ij}] \in F^{m\times n}, B =
[b_{jk}]\in F^{n \times p}$:
  \[
    A \, B = \left[\left(\sum_{j}a_{ij}b_{jk}\right)_{ik}\right] \in F^{m \times p}.
  \]



\begin{definition}[Diagonal matrix]
  A \emph{diagonal matrix} is a matrix where the only non-zero entries are those where the column
  index equals the row index.
\end{definition}

The diagonal matrix $n\times n$ with only $1$'s on the diagonal is the identity for matrix
multiplication, and is designated by $I_n$.

\begin{definition}[Transpose]
  The \emph{transpose} of an $n\times m$ matrix $A=[a_{ij}]$ is an $m\times n$ matrix $A^{t}$ with
  the $i,j$ entry being $a_{ji}$.
\end{definition}

When the base field of a matrix is \complex, the complex numbers, the \emph{conjugate transpose}
(also called the \emph{adjoint}) of an $n\times m$ matrix $A=[a_{ij}]$ is defined as the $m\times
n$ matrix $A^{*}$ with the $i,j$ entry being $\overline{a}_{ji}$, where $\overline{a}$ is the
complex conjugate of $a\in\complex$.

When working with column vectors over \complex, note that $\vc{u} \in \complex^n \implies
\vc{u}^{*} \in \complex^{1\times n}$ and that $\vc{u}^{*}\times \vc{u} \in \complex^{1\times 1}$.
This thesis will use the usual identification of \complex{} with $\complex^{1\times1}$. A column
vector \vc{u} is called a \emph{unit vector} when $\vc{u}^{*}\times \vc{u} = 1$.

\begin{definition}[Trace]
  The \emph{trace}, $Tr(A)$ of a square matrix $A=[a_{ij}]$ is $\sum a_{ii}$.
\end{definition}

\subsubsection{Tensor Product} % (fold)
\label{ssub:tensor_product}


The tensor product of two matrices is the usual Kronecker product:
  \[
    U\otimes V =
    \begin{bmatrix}
      u_{11}V&u_{12}V & \cdots &u_{1m}V\\
      u_{21}V&u_{22}V & \cdots &u_{2m}V \\
      \vdots&\vdots&\ddots\\
      u_{n1}V&u_{n2}V & \cdots &u_{nm}V
    \end{bmatrix}
    =
    \begin{bmatrix}
      u_{11}v_{11}&\cdots&u_{12}v_{11} & \cdots& u_{1m}v_{1q} \\
      u_{11}v_{21}&\cdots&u_{12}v_{21} & \cdots& u_{1m}v_{2q} \\
      \vdots&\vdots&\vdots&\ddots \\
      u_{n1}v_{p1}&\cdots&u_{n2}v_{p1} & \cdots& u_{nm}v_{pq} \\
    \end{bmatrix}
  \]
% subsubsection tensor_product (end)

\subsubsection{Special matrices} % (fold)
\label{ssub:special_matrices}

When working with quantum values certain types of matrices over the complex numbers are of special
interest. These are:
\begin{description}
  \item[Unitary Matrix]: Any $n \times n$  matrix $A$ with $A A^{*} = I\ (= A^{*} A)$.
  \item[Hermitian Matrix]: Any  $n \times n$ matrix $A$ with $A=A^{*}$.
  \item[Positive Matrix]: Any Hermitian matrix $A$ in  $\complex^{n\times n}$
    where $\vc{u}^{*} A \vc{u} \ge 0$ for all vectors  $\vc{u}\in \complex^n$. Note
    that for any Hermitian matrix $A$ and vector $u$,  $\vc{u}^{*} A \vc{u}$ is real.
  \item[Completely Positive Matrix]: Any positive matrix $A$ in  $\complex^{n\times n}$
    where $I_m \otimes A$ is positive.
\end{description}
The matrix
  \[
    {\begin{singlespace}
      \begin{bmatrix}
        0&-i\\
        i&0
      \end{bmatrix}
    \end{singlespace}}
  \]
is an example of a matrix that is \emph{unitary}, \emph{Hermitian}, \emph{positive} and
\emph{completely positive}.


% subsubsection special_matrices (end)

\subsubsection{Superoperators} % (fold)
\label{ssub:superoperators}

A \emph{Superoperator} $S$ is a matrix over \complex{} with the following restrictions:
\begin{enumerate}
  \item{} $S$ is \emph{completely positive}. This implies that $S$ is positive as well.
  \item{} For all positive matrices $A$, $Tr(S\,A) \leq Tr(A)$.
\end{enumerate}
% subsubsection superoperators (end)
% subsection matrices (end)

% section linear_algebra (end)

%%% Local Variables:
%%% mode: latex
%%% TeX-master: "../../phd-thesis"
%%% End:


\section{Categories}
\label{sec:categories}

The information in this section is presented to introduce categories and fix notation for them. More
details of the definitions and lemmas presented here can be found in various books or online
resources for category theory, e.g., \cite{maclan97:categorieswrkmath}, \cite{cockett2009:ctcs},
\cite{barr:ctcs} and \cite{various:nlab}.

A category as a mathematical object can be defined in a variety of equivalent ways. As much of our
work will involve the exploration of partial and reversible maps, their domains and ranges, we
choose a definition that highlights the algebraic nature of these. Note that ranges are normally
referred to as co-domains in category theory and we will use the co-domain terminology in this
section.

\begin{definition}\label{def:category}
  A \emph{category} $\A$ is a directed graph consisting of objects $A_o$ and maps $A_m$. Each $f\in
  A_m$ has two associated objects in $A_o$, called the domain and co-domain. When $f$ has domain $X$
  and co-domain $Y$ we will write $f:X \to Y$. For $f, g \in A_m$, if $f:X\to Y$ and $g:Y \to Z$,
  there is a map called the \emph{composite} of $f$ and $g$, written $f g$,\footnote{Note that
    composition is written in diagrammatic order throughout this paper.} such that $f g:X \to Z$.
  For any $W \in A_o$ there is an \emph{identity} map $1_W:W \to W$. Additionally, these two axioms
  must hold:
  \begin{itemize}
    \item[\catone] for $f:X \to Y$, $1_X f = f = f 1_Y$,
    \item[\cattwo] given $f:X \to Y,\ g:Y \to Z$ and $h: Z\to W$, then $f (g h) = (f g) h$.
  \end{itemize}

\end{definition}

We may also consider the notion of containment between categories.

\begin{definition}\label{def:subcategories}
  Given the categories $\C$ and $\D$, we may say the following:
  \begin{enumerate}[{(}i{)}]
    \item \C is a \emph{sub-category} of \D when each object of \C is an object of \D and when each map
    of \C is a map of \D.
    \item \C is a \emph{full sub-category} of \D when it is a sub-category and given $A, B$ objects
      in $\C$ and $f:A \to B$ in \D, then $f$ is a map in $\C$.
  \end{enumerate}
\end{definition}

\subsection{Enrichment of categories} % (fold)
\label{sub:enrichement_of_categories}
\begin{definition}\label{def:hom-collection}
  If $\X$ is a category, then $\X(A,B)$ is called a \emph{hom-collection} of $\X$ and consists
  of all arrows $f$ with $D(f) = A$ and $C(f) = B$.
\end{definition}

In the case where the objects of a hom-collection of a category $\X$ are all sets, we call them
hom-sets. Alternatively, we say $\X$ is \emph{enriched} in sets. In general, categories may be enriched in a
variety of algebraic structures. For example a category may be enriched in groups, vector spaces or
monoids.

Specific types of enrichment may force a specific structure on a category. For example, if $\X$ is
enriched in sets of cardinality of 0 or 1, then $\X$ must be a pre-order.

\begin{example}
  A group $G$ may be considered as a one-object category $\mathbb{G}$, where the elements of the
  group are all the  maps between the single element. This category is (trivially) enriched in groups.
\end{example}
% subsection enrichment_of_categories (end)
\subsection{Examples of categories} % (fold)
\label{sub:examples_of_categories}
In this section, we will offer a few examples of categories.

We present four categories where the objects are the collection of sets.
\begin{example}\label{ex:category_sets}
The first is \sets, where the maps are given by all set functions.
\category{Sets}{Set functions}{The identity function}{Standard composition of functions}
\end{example}

\begin{example}\label{ex:category_par}
Our next example is \Par, where the maps are all partial set functions.
\category{Sets}{Partial set functions}{The identity function}{Standard composition of functions}
\end{example}

\begin{example}\label{ex:category_rel}
A third example, often of interest in quantum programming language semantics is \rel:
\category{All sets}{Relations: $R:X \to Y$}{$1_X = \{(x,x) | x \in X\}$}{$RS = \{(x,z) |\,
\exists y. (x,y) \in R$ and $(y,z)\in S\}$}
\end{example}

Note that \rel is enriched in posets, via set inclusion. \Par can be viewed as a sub-category of
\rel, with the same objects, but only allowing maps which are functions, i.e., if $(x,y), (x,y')
\in f$, then $y = y'$. \Par is also enriched in posets, via the same inclusion ordering as in \rel.

\begin{example}\label{ex:category_pinj}
Our final example based on sets is one that will be used throughout this thesis. The category \pinj consists of
the partial injective functions over sets. Similarly to \Par, it may be considered as a subcategory
of \rel. The maps $f,g$ (relations in \rel) in \pinj are defined as follows:
\begin{align}
   (x,y)\in f\text{ and }(x,y')\in f & \text{ implies } y = y',\label{eq:pinj_relation_is_a_function}\\
   (x,y)\in f\text{ and }(x',y)\in f & \text{ implies } x = x'.\label{eq:pinj_converse_relation_is_a_function}
\end{align}
\category{All sets}{Relations: $f:X \to Y$ which satisfy Equation~\ref{eq:pinj_relation_is_a_function} and
Equation~\ref{eq:pinj_converse_relation_is_a_function}.}{$1_X = \{(x,x) | x \in X\}$}{$f g = \{(x,z) |\,
\exists y. (x,y) \in f$ and $(y,z)\in g\}$}
\end{example}

\begin{example}[Matrix Category]\label{ex:matrix_category}
Given a rig $R$ (i.e., a ring minus negatives, e.g., the positive rationals), one may form the
category \textsc{Mat}($R$). For example, the category of matrices of natural numbers is:
\category{\nat}{$[r_{i j}]: n \to m$ where $[r_{i j}]$ is an $n \times m$ matrix over $R$}{
$I_n$}{Matrix multiplication}
\end{example}

\begin{example}[Dual Category]\label{ex:dual_category}
Given a category $\C$, we may form the \emph{dual} of $C$, written $C^{op}$ as the following
category:
\category{The objects of $\C$}{$f^{op}:B\to A$ in $\C^{op}$ when $f:A\to B$ in $\C$.}{
The identity maps of $\C$}{If $f g = h$ in $\C$, $g^{op} f^{op} = h^{op}$}
\end{example}

% subsection examples_of_categories (end)
\subsection{Properties of maps} % (fold)
\label{sub:properties_of_maps}
Many interesting properties of maps are generalizations of notions that have been found useful in
considering sets and functions. We present a few of these in a tabular format, together with their
categorical definition. Throughout Table~\ref{tab:properties_of_maps_in_categories}, $e,f,g$ are
maps in a category $C$ with $e:A \to A$ and $f,g:A \to B$.
\begin{table}[h!]
  \begin{center}
    \begin{tabular}{|p{1in}p{1in}p{3.73in}|}
      \hline
      {\bf Sets} & {\bf Categorical Property} & {\bf Definition}\\
      \hline
      \hline
      Injective & Monic & $f$ is monic whenever $h f = k f$ means that $h = k$.\\
      \hline
      Surjective & Epic & The dual notion to monic, $g$ is epic whenever $g h = g k$ means that $h = k$.
      A map that is both monic and epic is called \emph{bijic}.\\
      \hline
      Left~Inverse & Section & $f$ is a section when there is a map $f^*$ such that $f f^* = 1_A$. $f$
      is also referred to as the \emph{left inverse} of $f^*$.\\
      \hline
      Right Inverse & Retraction & $f$ is a retraction when there is a map $f_*$ such that $f_* f = 1_B$.
      $f$ is also referred to as the \emph{right inverse} of $f_*$. A map that is both a section and a
      retraction is called an \emph{isomorphism}.\\
      \hline
      Idempotent & Idempotent & An endomorphism $e$ is idempotent whenever $e e = e$.\\
      \hline
    \end{tabular}
  \end{center}
  \caption{Properties of Maps In Categories}
  \label{tab:properties_of_maps_in_categories}
\end{table}

We state without proof a number of properties about maps.

\begin{lemma}\label{lem:categorical_properties_of_maps}
  In a category \C,
  \begin{enumerate}[{(}i{)}]
    \item If $f,g$ are monic, then $f g$ is monic.
    \item If $f g$ is monic, then $f$ is monic.
    \item $f$ being a section means it is monic.
    \item $f, g$ sections implies that $f g$ is a section.
    \item $f g$ a section means $f$ is a section.
  \end{enumerate}
\end{lemma}


\begin{lemma}\label{lem:categorical_inverses_are_unique}
  If $f:A \to B$ is both a section and a retraction, then $f^* = f_*$, where $f^*$ and $f_*$ are as
  defined in Table~\ref{tab:properties_of_maps_in_categories}.
\end{lemma}

\begin{lemma}\label{lem:categorical_iso_is_epic_section}
  $f$ is an isomorphism if and only if it is an epic section.
\end{lemma}

Note there are corresponding properties for epics and retractions, obtained by dualizing the
statements of Lemma~\ref{lem:categorical_properties_of_maps} and
Lemma~\ref{lem:categorical_iso_is_epic_section}.

Suppose $f:A \to B$ is a retraction with left inverse $f_*:B \to A$. Note that $f f_*$ is idempotent
as $f f_* f f_* = f 1_B f_* = f f_*$. If we are given an idempotent $e$, we say $e$ is \emph{split}
if there is a retraction $f$ with $e = f f_*$.

In general, not all idempotents in a category will split. The following construction allows us to
create a category based on the original one in which all idempotents do split.

\begin{definition}\label{def:split_category}
  Given a category $\C$ and a set of idempotents $E$ of $\C$, we define \emph{Split${}_E$($\C$)},
  written as $\spl{E}{\C}$, as the following category:
  \category{$(A,e)$, where $A$ is an object of \C, $e:A\to A$ and $e\in E$.}
    {$f_{d,e}:(A,d)\to(B,e)$ is given by $f:A\to B$ in \C, where $f = d\uts f e$.}
    {The map $e_{e,e}$ for $(A,e)$.}
    {Inherited from \C.}
  When $E$ is the set of all idempotents in $\C$, we write $\spl{}{\C}$.
\end{definition}
This is the standard idempotent splitting construction, also known as the Karoubi
envelope.

\begin{lemma}\label{lem:split_category_splits_and_has_category}
  Given a category $\C$, then it is a full sub-category of $\spl{}{\C}$ and all idempotents split
  in  $\spl{}{\C}$.
\end{lemma}
\begin{proof}
  We identify each object $A$ in $\C$ with the object $(A,1)$ in $\spl{}{\C}$. The only maps between
  $(A,1)$ and $(B,1)$ in $\spl{}{\C}$ are the maps between $A$ and $B$ in $\C$, hence we have a
  full sub-category.

  Suppose we have the map $d_{e,e}: (A,e) \to (A,e)$ with $d d = d$, i.e., it is idempotent in $C$
  and $\spl{}{\C}$. In $\spl{}{\C}$, we have the maps $d_{e,d}:(A,e) \to (A,d)$ and $d_{d,e}:(A,d) \to
  (A,e)$ where $d_{d,e} d_{e,d} = d_{d,d} = 1_{(A,d)}$ and $d_{e,d} d_{d,e} = d_{e,e}$. Hence,
  it is a splitting of the map  $d_{e,e}$.
\end{proof}
% subsection properties_of_maps (end)

\subsection{Limits and colimits in categories} % (fold)
\label{sub:limits_and_colimits_in_categories}

We shall discuss only a few basic limits/colimits in categories. First we discuss initial
and terminal objects.

\begin{definition}\label{def:initial_object}
  An \emph{initial object} in a category $\C$ is an object which has exactly one map to each other
  object in the category. The dual notion is \emph{terminal object} which has exactly one map from
  each other object in the category.
\end{definition}


\begin{lemma}\label{lem:initial_objects_are_unique}
  Suppose $I,J$ are initial objects in $\C$. Then there is a unique isomorphism $i:I \to J$.
\end{lemma}
\begin{proof}
  First, note that by definition there is only one map from $I$ to $I$ --- which must be the
  identity map. As $I$ is initial there is a map $i: I \to J$. As $J$ is initial there is a map
  $j:J \to I$. But this means $i j : I \to I = 1$ and $j i : J \to J = 1$ and hence $i$ is the
  unique isomorphism from $I$ to $J$.
\end{proof}

Dually, we have the corresponding result of Lemma~\ref{lem:initial_objects_are_unique} for terminal
objects --- they are also unique up to a unique isomorphism.

In categories, we normally designate the initial object by $0$ and the terminal object by $1$.

We now turn to products and co-products.

\begin{definition}\label{def:categorical_product}
  Let $A,B$ be objects of the category $\C$. Then the object $A \times B$ is a \emph{product} of
  $A$ and $B$ when:
  \begin{itemize}
    \item There exist maps $\pi_0, \pi_1$ with $\pi_0:A\times B \to A$, $\pi_1:A\times B \to B$;
    \item Given an object $C$ with maps $f:C\to A$ and $g:C \to B$ there is a unique map
    $\<f,g\>$ such that the following diagram commutes:
    \[
      \xymatrix@C+15pt@R+25pt{
        &&&A\\
        C\ar[urrr]^{f} \ar[drrr]_{g}\ar@{.>}[rr]^{\<f,g\>} & &A\times B \ar[ur]_{\pi_0}\ar[dr]^{\pi_1}\\
        &&&B
      }
    \]
  \end{itemize}

\end{definition}

A co-product is the dual of a product.

\begin{definition}\label{def:categorical_coproduct}
  Let $A,B$ be objects of the category $\C$. Then the object $A + B$ is a \emph{coproduct} of
  $A$ and $B$ when:
  \begin{itemize}
    \item There exist maps $\cpa, \cpb$ with $\cpa:A \to A + B$, $\cpb:B \to A+ B$;
    \item Given an object $C$ with maps $h:A\to C$ and $k:B \to C$ there is a unique map
    $[f,g]$ such that the following diagram commutes:
    \[
      \xymatrix@C+15pt@R+25pt{
        A \ar[drr]^h \ar[d]_{\cpa} \\
        A +B \ar@{.>}[rr]^{[f,g]} && C\\
        B \ar[urr]_k \ar[u]^{\cpb}
      }
    \]
  \end{itemize}
\end{definition}

It is possible for an object to both a limit and a co-limit at the same time.
\begin{definition}\label{def:categorical_zero}
  Given a category $\C$, any object that is both a terminal and initial object is called a
  \emph{zero object}. This object is labelled $\zeroob$.
\end{definition}

Note that any category with a zero object has a special map, $\zeroob_{A,B}$
between any two objects $A,B$ of the category given by: $\zeroob_{A,B} : A \to \mathbf{0} \to B$.

\begin{definition}\label{def:categorical_biproduct}
  In a category \C, given any two objects, $A,B$, if $A\times B$ is isomorphic to $A+B$,  it is
  referred to as the \emph{biproduct}.
\end{definition}

\begin{definition}\label{def:finite_biproducts}
A category \cD{} is said to have \emph{finite biproducts} when it has a zero object $\mathbf{0}$
and when each pair of objects $A,B$ have a biproduct $A\+B$.
\end{definition}

Note that a category with finite biproducts is enriched in commutative monoids. If $f,g:A\to
B$, define $f+g:A\to B$ as $\<id_{A}, id_{A}\>\, (f\+g)\, [id_{B},id_{B}]$. The unit for the
addition is $\zeroob_{A,B}$. Throughout this thesis, when working in a category with finite
biproducts, $\<id, id\>$ will be designated by $\Delta$ and $[id,id]$ will be designated by
$\nabla$.

% subsection limits_and_colimits_in_categories (end)

\subsection{Functors and natural transformations} % (fold)
\label{ssub:functors_and_natural_transformations}

\begin{definition}\label{def:functor}
  A map $F:\X \to \Y$ between categories, as in Definition~\ref{def:category}, is called a
  \emph{functor}, provided it satisfies the following:
  \begin{itemize}
    \item[\axiom{F}{1}] $F(D(f)) = D(F(f))$ and $F(C(f)) = C(F(f))$;
    \item[\axiom{F}{2}] $F(f g) = F(f)F(g)$;
  \end{itemize}
\end{definition}

\begin{lemma}\label{lem:cat_is_a_category}
  The collection of categories and functors form the category \cat.
\end{lemma}
\begin{proof}
  \category{Categories.}{Functors.}{The identity functor which takes a map to the same map.}{
  $F G(x) = F(G(x))$ which is clearly associative.}
\end{proof}


Functors with two arguments, e.g., $F:\sets \times \sets \to \sets$ which satisfy \axiom{F}{1} and
\axiom{F}{2} for each argument independently are called \emph{bi-functors}.


We will often restrict ourselves to specific classes of functors which either \emph{preserve} or
\emph{reflect} certain characteristics of the domain category or co-domain category. To be more
precise, we provide some definitions.

\begin{definition}\label{def:diagram_in_a_category}
  A \emph{diagram} in a category is a collection of objects and maps between those objects
  which satisfy categorical composition rules. More precisely: Given a category $\S$, a diagram
  in a category $\C$ of \emph{shape} $\S$ is a functor $D:\S \to \C$.
\end{definition}

In practice, diagrams are pictorially represented by drawing the objects and the maps between them.

\begin{definition}\label{def:property_of_a_diagram}
  A \emph{property} of a diagram $D$, written $P(D)$ is a logical relation expressed using the
  objects and maps of the diagram $D$.
\end{definition}

\begin{example}
  $P(f:A \to B) = \exists h : B \to A. h f = 1_A$ expresses that $f$ is a retraction.
\end{example}

\begin{definition}\label{def:functor_preserving_a_property}
  A functor $F$ \emph{preserves} the property $P$ over maps $f_i$ and objects $A_j$ when
  $P(f_1,\ldots,f_n, A_1,\ldots,A_m)$ implies $P(F(f_1),\ldots,F(f_n), F(A_1),\ldots,F(A_m))$.
\end{definition}

\begin{definition}\label{def:functor_reflects_a_property}
  A functor $F$ \emph{reflects} the property $P$ over maps $f_i$ and objects $A_j$ when
  $P(F(f_1),\ldots,F(f_n), F(A_1),\ldots,F(A_m))$ implies $P(f_1,\ldots,f_n, A_1,\ldots,A_m)$.
\end{definition}

For example, all functors preserve the properties of being an idempotent or a retraction or section,
but in general, not the property of being monic.

A functor $F:\C \to \D$ induces a map between hom-objects in $\C$ and hom-objects in $\D$. For
each object $A,B$ in $\C$ we have the map:
\[
  F_{AB} : \C(A,B) \to \D(F(A),F(B)).
\]

\begin{definition}\label{def:full_functor_faithful_functor}
  Given a functor $F:\C \to \D$, we say:
  \begin{itemize}
    \item $F$ is \emph{faithful} when for all $A,B$, $F_{AB}$ is an injective function;
    \item $F$ is \emph{full} when  for all $A,B$, $F_{AB}$ is an surjective function.
  \end{itemize}
\end{definition}



\begin{definition}\label{def:natural_transformation}
  Given functors $F,G:\X \to \Y$, a \emph{natural transformation} $\alpha:F \natto G$ is a collection
  of maps in $\Y$, $\alpha_X : F(X) \to G(X)$, indexed by the objects of $\X$ such that for all
  $f:X_1 \to X_2$ in $\X$ the following diagram in $\Y$ commutes:
  \[\xymatrix @R+10pt @C+10pt{
      F(X_1) \ar[r]^{F(f)} \ar[d]_{\alpha_{X_1}} & F(X_2) \ar[d]^{\alpha_{X_2}}\\
      G(X_1) \ar[r]_{G(f)} &  G(X_2)
    }
  \]
\end{definition}
% subsection functors_and_natural_transformations (end)

\subsection{Categories with additional structure} % (fold)
\label{sub:categories_with_additional_structure}

\begin{definition}\label{symmetricmonoidalcat}
  A \emph{symmetric monoidal category} \cD{} is a category equipped with a monoid $\+$ (a bi-functor
  $\+:\cD \times \cD \to \cD$) together with three families of natural isomorphisms:
  $a_{A,B,C}:A\*(B\*C) \to (A\*B)\*C$, $u_{A}:A\to A\+I$ and $c_{A,B}:A\+B \to B\+ A$, which satisfy
  specific coherence diagrams. The isomorphisms are referred to as the \emph{structure isomorphisms}
  for the symmetric monoidal category. $I$ is the unit of the monoid.
  A symmetric monoidal category where each of $a_{A,B,C}$, $u_{A}$ and $c_{A,B}$ are actually
  identity maps is called a \emph{strict symmetric monoidal category}.
\end{definition}
For details on the coherence diagrams, please see e.g., \cite{barr:ctcs} or
\cite{maclan97:categorieswrkmath}. The essence of the coherence diagrams is that any diagram
composed solely of the structure isomorphisms will commute.

\begin{definition}\label{def:compactclosedcat}
A \emph{compact closed category} \cD{} is a symmetric monoidal category with monoid $\*$ where each
object $A$ has a dual $A^{*}$. Additionally, there must exist families of maps $\eta_{A}: I \to
A^{*} \* A$ (the \emph{unit}) and $\epsilon_{A}: A\*A^{*}\to I$ (the \emph{counit}) such that
\[
  \xymatrix@C+20pt{
    A \ar[r]^{u_{A}} \ar@{=}[d]  & A\*I \ar[r]^{1\*\eta_{A}}
        & A\* (A^{*}\*A) \ar[d]^{a_{A,A^{*},A}} \\
    A & I\* A \ar[l]^{u_{A}^{-1}} & (A\* A^{*})\*A \ar[l]^{\*\epsilon_{B}\*1}
    }
  \]
commutes and so does the similar one based on $A^{*}$.
\end{definition}

Given a map $f:A\to B$ in a compact closed category,  define the map $f^{*}:B^{*} \to A^{*}$ as
\[
  \xymatrix@C+10pt{
    B^{*}\ar[r]^{u_{B^{*}}} \ar[d]_{f^{*}}& I\*B^{*} \ar[r]^{\eta_{A}\*1}
      & A^{*}\*A\*B^{*}\ar[d]^{1\*f\*1}\\
    A^{*}&    A^{*}\*I\ar[l]^{u_{A^{*}}^{-1}}  &   A^{*}\*B\*B^{*}\ar[l]^{1\*\epsilon_{B}}.
  }
\]


% subsection categories_with_additional_structure (end)

% section categories (end)
\section{Restriction categories} % (fold)
\label{sec:restriction_categories}


Restriction categories were introduced in
 \cite{cockett2002:restcategories1} as a way to give an algebraic axiomatization of partial maps.
\begin{definition}\label{def:restriction_category}
  A \emph{restriction category} is a category \X\ together with a \emph{restriction operator} on
  maps,
  \[
    \infer{\restr{f}:A\to A,}{f:A \to B}
  \]
  where $f$ is a map of \X\ and $A,B$ are objects of \X, such that the
  following four \emph{restriction identities} hold, whenever the
  compositions are defined:
  \begin{align*}
    &\rone\ \restr{f} f = f & &
    \rtwo\ \restr{g}  \restr{f} = \restr{f}  \restr{g}\\
    &\rthree\ \restr{\restr{f}  g} = \restr{f}   \restr{g} & &
    \rfour\  f \restr{g} = \restr{f g} f.
  \end{align*}
\end{definition}

\begin{definition}
  A \emph{restriction functor} is a functor which preserves the restriction. That is,
  given a functor $F: \X \to \Y$ with \X\  and \Y\ restriction categories,
  $F$ is a restriction functor if:
  \[
    F(\restr{f}) = \restr{F(f)}.
  \]
\end{definition}

\begin{definition}\label{def:restriction_idempotent}
Note that any map such that $r=\restr{r}$ is an idempotent, as $r r = \rst{r} r = r$.
Such a map is called a \emph{restriction idempotent}.
\end{definition}

Here are some basic facts for restriction categories. The proof may be found in
\cite{cockett2002:restcategories1} and \cite{cockett-manes09-boolean-classical-rest-cats}.
\begin{lemma}\label{lem:restrictionvarious}
  In a restriction category \X,
  \begin{multicols}{2}
    \begin{enumerate}[{(}i{)}]
      \item{}$\rst{f}$ is idempotent;
      \item{} $\rst{f g} = \rst{f g} \, \rst{f}$;\label{lemitem:rv_2}
      \item{} $\rst{f g} = \rst{f \rst{g}}$ ;\label{lemitem:rv_3}
      \item{} $\rst{\rst{f}} = \rst{f}$;
      \item{} $\rst{f}\,\rst{g} = \rst{\rst{f}\,\rst{g}}$;
      \item{} $f$ monic implies $\rst{f} = 1$;
      \item{} $f = \rst{g} f \implies \rst{g}\,\rst{f} = \rst{f}$.
      \\
    \end{enumerate}
  \end{multicols}
\end{lemma}

Note that by Lemma~\ref{lem:restrictionvarious}, all maps $\restr{f}$ are restriction idempotents
as $\rst{f}=\rst{\rst{f}}$.


\begin{definition}\label{def:total_map}
  A map $f:A\to B$ in a restriction category is said to be \emph{total} when
  $\rst{f} = 1_A$. The total maps in a restriction category form a sub-category
  $Total(\X) \subseteq \X$.
\end{definition}

\begin{example}[\Par]\label{ex:par_is_a_restriction_category}
Continuing from Example~\ref{ex:category_par}, \Par is a restriction category. The restriction of $f:A\to B$ is:
\[
  \rst{f}(x) =
  \begin{cases}
    x&\text{if $f(x)$ is defined,}\\
    \uparrow&\text{if }f(x)\text{ is }\uparrow.
  \end{cases}
\]
In \Par, the
total maps correspond precisely to the functions that are defined on all elements of the domain.
\end{example}

\begin{example}[\rel]\label{ex:rel_is_a_restriction_category}
The category \rel from Example~\ref{ex:category_rel} is a restriction category with the restriction
of $R=\{(a,b)\}$ being $\rst{R} = \{(a,a) | \exists b. (a,b) \in R\}$.
\end{example}

\begin{example}[\pinj]\label{ex:pinj_is_a_restriction_category}
From Example~\ref{ex:category_pinj}, we see \pinj is a restriction category and in fact is a
sub-restriction category of \rel. We will show the four restriction axioms:
\begin{align*}
  \rone\ & \rst{f}f = \{(x,z) | \exists x. (x,x) \in \rst{f} \text{ and } (x,z) \in f\} = \{(x,z) |
  (x,z) \in f\} = f,\\
  \rtwo\ & \rst{f}\rst{g} = \{(x,z) | \exists y. (x,y) \in \rst{f} \text{ and } (y,z) \in \rst{g}\} =
  \{(x,x) | (x,x) \in \rst{f} \text{ and } (x,x) \in \rst{g}\} = \rst{g}\rst{f},\\
  \rthree\ & \rst{\rst{f} g} = \rst{\{(x,y) | (x,x) \in \rst{f}, (x,y) \in g\}} = \{(x,x) | (x,x) in
  \rst{f}, (x,x) \in \rst{g} \} = \rst{f} \rst{g},\\
  \rfour\ & f \rst{g} = \{(x,y) | (x,y) \in f, (y,y) \in \rst{g}\} =  \{(x,y) | (x,y) \in f, \exists
  z. (y,z) \in g\} \\
  &=  \{(x,y) | (x,y) \in f, \exists z. (x,z) \in f g\} =  \{(x,y) | (x,y) \in f,
  (x,x) \in \rst{f g}\} = \rst{f g} f.
\end{align*}
\end{example}

\subsection{Enrichment and meets} % (fold)
\label{sub:enrichment_and_meets}

We may use the restriction to define a partial order on the hom-sets of a restriction
category. Intuitively, we would think of a map $f$ being less than a map $g$ if $f$ is
defined on fewer elements than $g$ \emph{and} they agree where they are defined. This Sub-Section
will bring precision to the above intuition.

\begin{definition}\label{def:restriction_category_hom_set_ordering}
  In a restriction category, for any two parallel maps  $f,g:A\to B$, define $f \le g$ iff
  $\restr{f} g = f$.

\end{definition}


\begin{lemma}\label{lem:restriction_cats_are_partial_order_enriched}
  In a restriction category \X:
  \begin{multicols}{2}
    \begin{enumerate}[{(}i{)}]
      \item  $\le$ from Definition~\ref{def:restriction_category_hom_set_ordering}
        is a partial order on each hom-set;
      \item $f \le g \implies \restr{f} \le \restr{g}$;\label{lemitem:rst_ordering_2}
      \item $\rst{f g} \le \rst{f}$; \label{lemitem:rst_ordering_3}
      \item $f \le g \implies h f \le h g$;
      \item $f \le g \implies f h \le g h$;
      \item $f \le g$ and $\rst{f} = \rst{g}$ implies $f = g$;
      \item $f \le 1 \iff f = \restr{f}$;
      \item $\rst{g}f = f$ implies $\rst{f} \le \rst{g}$.
    \end{enumerate}
  \end{multicols}
\end{lemma}
\begin{proof}
  \prepprooflist
  \setlist[enumerate,1]{leftmargin=1.2cm}
  \begin{enumerate}[{(}i{)}]
    \item With $f,g,h$ parallel maps in \X, each of the requirements for a partial order is
    verified below:
    \begin{description}
      \itembf{Reflexivity:} $\restr{f} f = f$ and therefore, $ f \le f$.
      \itembf{Anti-Symmetry:} Given $\restr{f}g = f$ and $\restr{g}f = g$, it follows:
        \[
          f = \restr{f} f = \restr{\restr{f} g} f = \restr{f}\, \restr{g} f
          = \restr{g}\restr{f} f =  \restr{g} f = g.
        \]
      \itembf{Transitivity:} Given $f \le g$ and $g\le h$,
        \[
          \restr{f} h = \restr{\restr{f} g} h = \restr{f}\, \restr{g} h = \restr{f} g = f
        \]
        showing that $f \le h$.
    \end{description}
    \item The premise is that $\restr{f} g = f$. From this, $ \restr{f}\, \restr{g} =
      \restr{\restr{f} g} = \restr{f}$, showing $\restr{f} \le \restr{g}$.
    \item $\restr{h f} h g = h \restr{f} g = h f$  and therefore $h f \le h g$.
    \item $\restr{f} g = f$, this shows $\restr{f h} g h = \restr{\restr{f} g h} g h
      = \restr{f}\, \restr{g h} g h = \restr{f} g h = f h$ and therefore $f h \le g h$.
    \item $g = \rst{g} g = \rst{f} g = f$.
    \item As $f \le 1$ means precisely $\restr{f}1 = f$.
    \item Assuming $\rst{g} f = f$, we need to show $\rst{f}\, \rst{g} = \rst{f}$.
      \begin{align*}
        \rst{f}\,\rst{g} &= \rst{g}\rst{f} & \rtwo \\
        & = \rst{\rst{g} f} & \rthree \\
        & = \rst{f} & \text{Assumption}.
      \end{align*}
      Hence, $\rst{f} \le \rst{g}$.
  \end{enumerate}
\end{proof}

Lemma \ref{lem:restriction_cats_are_partial_order_enriched} shows that restriction
categories are enriched in partial orders.

In a restriction category \X, we will use the notation $\open{A}$ for the restriction idempotents
of $A\in \ob{\X}$. $\open{A} = \{x:A\to A| x = \rst{x}\}$. The notation $\open{A}$ was chosen
to be suggestive of open sets.

\begin{lemma}\label{lem:open_a_is_a_meet_semilattice}
  In a restriction category \X, $\open{A}$ is a meet semi-lattice.
\end{lemma}
\begin{proof}
  The top of the meet semi-lattice is $1_A$, under the ordering from
  Definition~\ref{def:restriction_category_hom_set_ordering}.
  The join of any two idempotents is given by their composition.
\end{proof}

\begin{definition}
  A restriction category has \emph{meets} if there is an operation $\cap$ on parallel maps:
  \[
    \infer{A\xrightarrow{f\cap g} B}
      {A\overset{f}{\underset{g}{\rightrightarrows}}B}
  \]
  such that $f\cap g \le f, f\cap g \le g, f\cap f = f, h (f\cap g) = h f \cap h g$.
\end{definition}

Meets were introduced in \cite{cockett-guo-hofstra-2012:range2}.
The following are basic results on meets:

\begin{lemma}
  \label{lem:properties_of_meets_in_restriction_categories}
  In a restriction category \X with meets, where $f, g, h$ are maps in
  \X, the following are true:
  \setlist[enumerate,1]{leftmargin=1.2cm}
  \begin{enumerate}[{(}i{)}]
    \item $f\le g \text{ and } f \le h \iff f \le g\cap h$;
        \label{lemsub:properties_of_meets_one}
    \item $f\cap g = g \cap f$;\label{lemsub:properties_of_meets_two}
    \item $\restr{f\cap 1} = f \cap 1$;\label{lemsub:properties_of_meets_three}
    \item $(f \cap g) \cap h = f \cap (g \cap h)$;
    \item $r(f\cap g) = r f \cap g$ where $r=\rst{r}$ is a restriction idempotent;
    \item $(f\cap g)r = f r \cap g$ where $r=\rst{r}$ is a restriction idempotent;
    \item $\restr{f\cap g} \le \restr{f}$ (and therefore $\restr{f\cap g} \le \restr{g}$);
    \item $ (f \cap 1) f = f \cap 1$;
    \item $ e(e \cap 1) = e$ where $e$ is idempotent.
  %\item $ e \cap e' = e e'$
  \end{enumerate}
\end{lemma}
\begin{proof}
  \prepprooflist
  \setlist[enumerate,1]{leftmargin=1.2cm}
  \begin{enumerate}[{(}i{)}]
    \item $f\le g \text{ and } f \le h$ means precisely $f = \restr{f} g$ and $f = \restr{f} h$.
      Therefore,
      \[
        \restr{f} (g\cap h) =  \restr{f} g \cap \restr{f} h =  f\cap f = f
      \]
      and so $f \le g \cap h$. Conversely, given $f \le g\cap h$, we have
      $f = \restr{f} (g\cap h) = \restr{f} g \cap \restr{f} h \le \restr{f} g $. But
      $f \le \restr{f} g$ means $f = \restr{f}\,\restr{f} g = \restr{f}g$ and therefore
      $f \le g$. Similarly, $f \le h$.
    \item From \ref{lemsub:properties_of_meets_one}, as by definition, $f\cap g \le g$ and
      $f \cap g \le f$.
    \item $f\cap 1 = \restr{f\cap 1} (f \cap 1)= (\restr{f \cap 1} f ) \cap (\restr{f \cap 1})
      \le \restr{f \cap 1}$ from which the result follows. %def of $\le$, \rthree and \rone
    \item By definition and transitivity, $(f\cap g)\cap h \le f, g, h$ therefore by
      \ref{lemsub:properties_of_meets_one} $(f \cap g) \cap h \le f \cap (g \cap h)$. Similarly,
      $f \cap (g \cap h) \le(f \cap g) \cap h$ giving the equality.
    \item Given  $r f \cap g \le r f$, calculate:
      \[
        r f\cap g
        = \restr{r f\cap g} r f
        = \restr{r (r f\cap g)} f
        = \restr{r r f\cap r g} f
        = \restr{r (f\cap g)} f
        = r \restr{f\cap g} f
        = r (f\cap g).
      \]
    \item Using the previous point with the restriction idempotent $\restr{f r}$,
      \begin{equation*}
        \begin{split}
          f r \cap g
          = f \restr{r} \cap g   %%r rest id
          = \restr{f r }f \cap g  %% R.4
          = \restr{f r}(f\cap g)   %Pre
          = \restr{f r}\, \restr{f\cap g} f \\ % meet <=
          = \restr{f\cap g}\, \restr{f r} f % R2
          = \restr{f\cap g} f \restr{r}  %R4
          = (f\cap g) r. % meet, r rest id
        \end{split}
      \end{equation*}
    \item For the first claim,
      \[
        \restr{f\cap g}\, \restr{f} =\restr{\restr{f}(f\cap g)}\\
        =\restr{(\restr{f}f)\cap g} =\restr{f\cap g}.
      \]
      The second claim then follows by \ref{lemsub:properties_of_meets_two}.
    \item Given $ f \cap 1 \leq f$:
      \[
        f \cap 1 \leq f \iff  \restr{f \cap 1} f = f \cap 1 \iff  (f \cap 1) f = f \cap 1
      \]
      where the last step is by item \ref{lemsub:properties_of_meets_three} of this lemma.
    \item As $e$ is idempotent, $e (e\cap 1) = (e e \cap e) = e$.
  \end{enumerate}
\end{proof}

\begin{example}[Meets in \pinj]\label{ex:pinj_has_meets}
The restriction category \pinj has meets given by the intersection of the sets defining the
maps. First, we note that the hom-set ordering for \pinj is given by set inclusion. Denoting
intersection by $\Cap$, we immediately have
\begin{align*}
  f\Cap g &\subseteq f \\
  f\Cap g &\subseteq g\\
  f\Cap f & = f
\end{align*}
by the properties of sets and intersections. For the final requirement,
\begin{align*}
  h(f\Cap g) &= \{(x,z) | \exists y. (x,y) \in h, (y,z) \in f\Cap g\}\\
  &=   \{(x,z) | \exists y. (x,y)  \in h, (y,z) \in f, (y,z) \in g\}\\
  & = \{(x,z) | (x,z) \in h f, (x,z) \in h g\} = h f \Cap h g.
\end{align*}
Thus, intersection is a meet in \pinj.
\end{example}
% subsection enrichment_and_meets (end)
\subsection{Partial monics, sections and isomorphisms} % (fold)
\label{sub:restricted_monics_sections_and_partial_isomorphisms}

Partial isomorphisms play a central role in this thesis. Below we present
some of their basic properties.

\begin{definition}
  For maps $f$ in a restriction category \X:
  \begin{itemize}
    \item $f$ is a \emph{partial isomorphism} when there is a \emph{partial inverse}, written
      $\inv{f}$ with $f\inv{f} =\restr{f}$ and $\inv{f}f = \restr{\inv{f}}$;
    \item $f$ is a \emph{partial monic} if $h f = k f \implies h \restr{f} = k \restr{f}$;
    \item $f$ is a \emph{partial section} if there exists an  $h$ such that $f h = \restr{f}$;
    \item $f$ is a \emph{restriction monic} if it is a section $s$ with a retraction
      $r$ such that $r s = \restr{r s}$.
  \end{itemize}
\end{definition}

Note that restriction monic is a stronger notion than that of monic. Consider two objects $A, B$
in a restriction category where we have $m: A\to B$, $r:B \to A$ with $m r = 1_A$. In this case
$A$ is called a \emph{retract} of $B$, which we will write as $A\retract B$. As $m$ and $r$ need
not be unique, we will also write $(m,r) A \retract B$ when the specific section and retraction
are to be emphasized. Since $m$ is a section,
it is a monic and therefore total. The map $r m$ is idempotent on $B$ as $r m r m = r 1 m = r m$.
$A$ is referred to as a \emph{splitting} of the idempotent $r m$. Note there is no requirement that
$r m = \rst{r m}$ when $m$ is simply monic.

\begin{lemma}
  \label{lem:rcs_partial_monic_section_inverse_properties}
  In a restriction category:
  \begin{enumerate}[{(}i{)}]
    \item $f,\ g$ partial monic implies $f g$ is partial monic;
    \item $f$ a partial section implies $f$ is partial monic;
    \item $f,\ g$ partial sections implies $f g$ is a partial section;
    \item The partial inverse of $f$, when it exists, is unique;
    \item If $f,\ g$ have partial inverses and $f\,g$ exists, then $f\,g$ has a partial inverse;
    \item A restriction monic $s$ is a partial isomorphism.
  \end{enumerate}
\end{lemma}
\begin{proof}
  \prepprooflist
  \begin{enumerate}[{(}i{)}]
    \item Suppose $h f g = k f g$. As $g$ is partial monic, $h f \restr{g} = k f \restr{g}$.
      Therefore:
      \begin{align*}
        h \restr{f g} f &= k \restr{f g} f &\rfour\\
        h \restr{f g}\,\restr{f} &= k \restr{f g}\, \restr{f} & f\text{partial monic}\\
        h \restr{f g}&= k \restr{f g} & \text{Lemma \ref{lem:restrictionvarious},
          \ref{lemitem:rv_2}.}
      \end{align*}
    \item Suppose $g f = k f$. Then, $g\restr{f} = g f h = k f h = k \restr{f}$.
    \item We have $f h = \restr{f}$ and $g h' = \restr{g}$. Therefore,
      \begin{align*}
        f g h' h &= f \restr{g} h & g \text{ partial section}\\
        &= \restr{f g} f h & \rfour\\
        &= \restr{f g}\, \restr{f} & f \text{ partial section}\\
        &= \restr{f}\, \restr{f g} & \rtwo\\
        &= \restr{\restr{f}f g} & \rthree\\
        &= \restr{f g} & \rone.
      \end{align*}
    \item Suppose both $\inv{f}$ and $f^{\diamond}$ are partial inverses of $f$. Then,
      \begin{multline*}
        \inv{f}
        = \restr{\inv{f}}\inv{f} %R.1
        =\inv{f}f\inv{f}  %Assumption, inverse is \inv{f}
        = \inv{f} \restr{f}   %Assumption, inverse is \inv{f}
        = \inv{f} f f^{\diamond}   %Assumption, inverse is f^{\diamond}
        = \inv{f} f \restr{f^{\diamond}} f^{\diamond}  \\ %R.1
        = \restr{\inv{f}}\restr{f^{\diamond}} f^{\diamond}   %Assumption, inverse is \inv{f}
        = \restr{f^{\diamond}}\restr{\inv{f}} f^{\diamond} %R.2
        = f^{\diamond} f \restr{\inv{f}}  f^{\diamond} %Assumption, inverse is f^{\diamond}
        = f^{\diamond} f \inv{f} f f^{\diamond} %Assumption, inverse is \inv{f}
        = f^{\diamond} f f^{\diamond} %Assumption, inverse is \inv{f} and R.1
        = f^{\diamond}. %Assumption, inverse is f^{\diamond}
      \end{multline*}
    \item For $f:A\to B,\ g:B\to C$ with partial inverses $\inv{f}$ and $\inv{g}$ respectively,
      the partial inverse of $f g$ is $\inv{g} \inv{f}$. Calculating $f g \inv{g} \inv{f}$
      using all the restriction identities:
      \[
        f g \inv{g} \inv{f} = f \restr{g} \inv{f} = \restr{f g} f \inv{f} =
        \restr{f g}\, \restr{f} = \restr{f}\, \restr{f g} = \restr{\restr{f} f g} = \restr{f g}.
      \]
      The calculation of $\inv{g} \inv{f} f g = \restr{\inv{g} \inv{f}}$ is similar.
    \item The partial inverse of $s$ is $\restr{r s}\,r$. First, note
      that $\restr{\restr{r s}\,r}
      = \restr{r s}\,\restr{r}
      = \restr{r}\, \restr{r s}
      = \restr{\restr{r}\,r s}
      = \restr{r s}$.
      Then, it follows that $(\restr{r s}\,r) s
      = r s\,= \restr{r s}
      = \restr{\restr{r s}r} $ and
      $s (\restr{r s}\,r)
      = s r \restr{s} %sr = 1 as r is the retraction of the section s
      = \restr{s}$.
  \end{enumerate}
\end{proof}

\begin{definition}\label{def:inverse_category}
  A restriction category in which every map is a partial
  isomorphism is called an \emph{inverse category}.
\end{definition}

\begin{lemma}
  \label{lem:inverse_idempotents_are_restriction_idempotents}
  In an inverse category, all idempotents are restriction idempotents.
\end{lemma}
\begin{proof}
  Given an idempotent $e$,
  \[
    \rst{e} = e\inv{e} = e e \inv{e} = e \rst{e} = \rst{e e} e = \rst{e} e = e.
  \]
\end{proof}

\begin{example}\label{ex:pinj_is_an_inverse_category}
  The category \pinj is an inverse category. For any map $f$, $\inv{f} = \{(y,x) | (x,y) \in
  f\}$. Note that $\inv{f}$ is a map in $\pinj$ due to the two dual conditions on maps as given in
  Example~\ref{ex:category_pinj}.

  On the other hand, \Par is not an inverse category. For example, let $A=\{1,2\},\ B=\{1\}$ and
  $f=\{(1,1),(2,1)\}$ in \Par. The restriction of $f$ is $\rst{f} = \{(1,1),(2,2)\} = 1_A$. There is
  no partial function $g:B\to A$ such that $f g = 1_A$.
\end{example}
% subsection restricted_monics_sections_and_partial_isomorphisms (end)

\subsection{Range categories} % (fold)
\label{sub:range_categories}
Corresponding to Definition~\ref{def:restriction_category} for restriction, which axiomatizes the
concept of a domain of definition, we now introduce range categories
\cite{guox:thesis,cockett-guo-hofstra-2012:range,cockett-guo-hofstra-2012:range2}
which algebraically axiomatize the concept of the range for a function.

\begin{definition}\label{def:range_category}
  A restriction category \X is a \emph{range category} when it has an operator on all maps
  \[
    \infer{\rg{f}:B\to B}{f:A\to B}
  \]
  where the operator satisfies the following:
  \begin{align*}
    &\rrone\ \restr{\rg{f}} = \rg{f} & &
     \rrtwo\ f \rg{f} = f\\
    &\rrthree\ \wrg{f\rst{g}} = \rg{f} \rst{g} & &
     \rrfour\  \wrg{\rg{f}g} = \wrg{f g}
  \end{align*}
  whenever the compositions are defined.

\end{definition}

\begin{lemma}\label{lem:basic_range_category_properties}
  In a range category \X, the following hold:
  \begin{multicols}{2}
    \begin{enumerate}[{(}i{)}]
      \item $\rg{g}\rg{f} = \rg{f}\rg{g}$;
      \item $\rst{f}\rg{g} = \rg{g}\rst{f}$;
      \item $\wrg{f\rg{g}} = \rg{f}\rg{g}$;
      \item $\rg{f} = 1$ when $f$ is epic, hence $\rg{1} = 1$;
      \item $\rg{f}\rg{f} = \rg{f}$;
      \item $\rg{\rg{f}} = \rg{f}$;
      \item $\rg{\rst{f}} = \rst{f}$;
      \item $\rg{g}\wrg{f g} = \wrg{f g}$;
      \item $\wrg{\rg{f}\rg{g}} = \rg{f}\rg{g}$.
    \end{enumerate}
  \end{multicols}
\end{lemma}
\begin{proof}
  See, e.g., \cite{guox:thesis}.
\end{proof}

\begin{lemma}\label{lem:ordering_of_restriction_and_range}
  In a range category:
  \begin{multicols}{2}
    \begin{enumerate}[{(}i{)}]
      \item  $\wrg{h f} \le \rg{f}$; \label{lemitem:ordering_1}
      \item $f' \le f$ implies $\rg{f'} \le \rg{f}$. \label{lemitem:ordering_2}
    \end{enumerate}
  \end{multicols}
\end{lemma}
\begin{proof}
  \prepprooflist
  \begin{enumerate}[{(}i{)}]
    \item Noting that $\rst{\wrg{hf}} \rg{f} = \wrg{hf} \rg{f}  = \wrg{hf \rg{f}} = \wrg{h f}$,
      we see $\wrg{h f} \le \rg{f}$.
    \item Calculating $\rst{\rg{f'}} \rg{f} = \rg{f'} \rg{f} = \wrg{\rst{f'} f} \rg{f} =
      \wrg{\rst{f'} f \rg{f}} = \wrg{\rst{f'} f} = \rg{f'}$, we see $\rg{f'} \le \rg{f}$.
  \end{enumerate}
\end{proof}

\begin{remark}
  Note that unlike restrictions, a range is a \emph{property} of a restriction category. To see
  this, assume we have two ranges $\wrg{(\_)}$ and $\widetilde{(\_)}$. Then,
  \[\rg{f}=\wrg{f \tilde{f}}=\rg{f} \tilde{f}=\tilde{f} \rg{f}=\widetilde{f \rg{f}}=\tilde{f}.\]
\end{remark}
\begin{lemma}\label{lem:inverse_categories_are_range_categories}
  An inverse category \X is a range category, where $\rg{f} = \inv{f}f = \rst{\inv{f}}$.
\end{lemma}
\begin{proof}
  \prepprooflist
  \setlist[enumerate,1]{leftmargin=1.5cm}
  \begin{enumerate}
    \item[\rrone] $\restr{\rg{f}} = \rst{\rst{\inv{f}}} = \rst{\inv{f}} = \rg{f}$;
    \item[\rrtwo] $f \rg{f} = f \rst{\inv{f}} = f \inv{f} f = \rst{f} f = f$;
    \item[\rrthree] $\wrg{f\rst{g}} = \rst{\inv{(f\rst{g})}} = \rst{\inv{\rst{g}} \inv{f}} =
      \rst{\rst{g} \inv{f}} =
      \rst{g} \rst{\inv{f}} = \rst{\inv{f}} \rst{g} =\rg{f} \rst{g}$;
    \item[\rrfour]  $\wrg{\rg{f}g} = \rst{\inv{(\rst{\inv{f}} g)}} =
      \rst{\inv{g}\inv{\rst{\inv{f}}}} = \rst{\inv{g} \rst{\inv{f}}} =
      \rst{\inv{g} \inv{f}} = \rst{\inv{(f g)}} = \wrg{f g}$
  \end{enumerate}
\end{proof}
% subsection range_categories (end)

\begin{example}\label{ex:ranges}
   In \pinj, $\rg{f} = \{(y,y) | \exists x. (x,y) \in f\}$.
\end{example}
\subsection{Split restriction categories} % (fold)
\label{sub:split_restriction_categories}

The Karoubi envelope, $\spl{E}{\X}$ as defined in Definition~\ref{def:split_category}
is also a restriction category when $\X$ is a restriction category.

Note that for $f:(A,d)\to(B,e)$, by definition, in \X we have $f=d\uts f e$, giving
\[
  d\uts f = d(d\uts f e) = d\uts d\uts f e = d\uts f e =f\
  \text{ and }\  f e = (d\uts f e)e = d\uts f e\uts e = d\uts f e = f.
\]
When \X is a restriction category, there is an immediate candidate for a restriction in
$\spl{E}{\X}$. If $f\in\spl{E}{\X}$ is $e_1 f e_2$ in $\X$, then define $\restr{f}$ as
given by $e_1 \restr{f}$ in \X. Note that for $f:(A,d)\to(B,e)$, in \X we have:
\[
  d\uts\restr{f} = \restr{d\uts f} d = \restr{f} d.
\]

\begin{proposition}\label{prop:spleisarestrictioncat}
  If \X is a restriction category and $E$ is a set of idempotents, then
  the restriction as defined above makes $\spl{E}{\X}$ a restriction category.
\end{proposition}
\begin{proof}
  The restriction takes $f:(A,e_1)\to (B,e_2)$ to an endomorphism of $(A,e_1)$. The restriction
  is in $\spl{E}{\X}$ as
  \[
    e_1 (e_1\restr{f}) e_1 = e_1 \restr{f} e_1
    = \restr{e_1 f} e_1 e_1
    = \restr{e_1 f} e_1
    = e_1 \restr{ f}.
  \]

  Checking the 4 restriction axioms:
  \begin{align*}
    &[\text{{\bfseries R.1}}]\  \llbracket\restr{f}f\rrbracket = e_1 \restr{f} f
    = e_1 f = \llbracket f\rrbracket.\\
    %
    & [\text{{\textbf{R.2}}}]\ \llbracket\restr{g}\restr{f}\rrbracket =
    e_1\restr{g}  e_1\restr{f}
    = e_1 e_1\restr{g}  \restr{f} = e_1 e_1\restr{f}  \restr{g}
    = e_1\restr{f}  e_1\restr{g}  = \llbracket\restr{f}\restr{g}\rrbracket.\\
    %
    & [\text{{\textbf{R.3}}}]\ \llbracket\restr{\restr{f} g } \equiv
    e_1 \restr{e_1 \restr{f}  g}
    =  \restr{e_1 e_1 \restr{f} g} e_1
    =  \restr{e_1 \restr{f} g} e_1
    =  e_1 \restr{\restr{f} g}
    = e_1 \restr{f}\restr{g}
    = e_{1}e_{1}\restr{f}\restr{g}
    = e_1 \restr{f}e_1\restr{g}
    = \llbracket\restr{f}\, \restr{g}\rrbracket.\\
    %
    &[\text{{\textbf{R.4}}}]\  \llbracket f \restr{g} \rrbracket =
     e_1f e_2 \restr{g}
    = \restr{e_1 f e_2 g} e_1 f e_2
    = \restr{e_1 e_1 f e_2 g} e_1 e_1 f e_2 \\
    & \qquad \qquad \qquad \qquad \qquad \qquad \qquad \qquad \qquad \quad
    = e_1 \restr{ e_1 f e_2 g} e_1 f e_2
    = e_1 \restr{f g} e_1 f e_2
    = \llbracket\restr{f g} f\rrbracket.
  \end{align*}
\end{proof}

Given this, provided all identity maps are in $E$, $\spl{E}{\X}$ is a
restriction category with $\X$ as a full sub-restriction category, via
the embedding defined by taking an object $A$ in \X to  the object $(A,1)$
in $\spl{E}{\X}$.  Furthermore, the property of being an inverse category is
preserved by splitting.

\begin{lemma}\label{lem:the_idempotent_splitting_of_an_inverse_category_is_an_inverse_category}
  When \X is an inverse category, $\spl{E}{X}$ is an inverse category.
\end{lemma}
\begin{proof}
  The inverse of $f:(A,e_1)\to(B,e_2)$   in \spl{E}{\X} is $e_2\inv{f}e_1$ as
  \[
    \llbracket f \inv{f} \rrbracket = e_1 f e_2 e_2 \inv{f} e_1
    = e_1 e_1 f e_2 \inv{f} e_1
    = e_1 f  \inv{f} e_1
    = e_1 e_1 \restr{f} e_1
    = e_1 \restr{f}
    = \llbracket\restr{f}\rrbracket
  \]
  and
  \begin{multline*}
    \llbracket \inv{f} f\rrbracket=
    e_2 \inv{f} e_1 e_1 f e_2
    = e_2 \inv{f} e_1 f e_2 e_2
    = e_2 \inv{f} f  e_2\\
    = e_2 e_2 \restr{\inv{f}}  e_2
    = e_2 \restr{\inv{f}}
    = \llbracket\restr{\inv{f}}\rrbracket.
  \end{multline*}

\end{proof}

\begin{proposition}\label{pro:in_rc_x_with_meets_split_x_is_cong_to_split_r_x}
  In a restriction category \X, with meets, let $R$ be the set of restriction idempotents.
  Then, $\spl{}{\X} \cong \spl{R}{\X}$, where \spl{}{\X} is the split of \X over all idempotents.
  Furthermore, $\spl{R}{\X}$ has meets.
\end{proposition}
\begin{proof}
  The proof below first shows the equivalence of the two categories, then addresses the claim
  that $\spl{R}{\X}$ has meets.

  For equivalence, we require two functors,
  \[
    U:\spl{R}{\X}\to\spl{}{\X}\text{ and }V:\spl{}{\X}\to\spl{R}{\X},
  \]
  with:
  \begin{align}
    U V \cong I_{\spl{R}{\X}}\\
    V U \cong I_{\spl{}{\X}}.
  \end{align}


  $U$ is the standard inclusion functor. $V$ will take the object $(A,e)$ to
  $(A,e\cap 1)$ and the map $f:(A,e_1)\to (B,e_2)$ to $(e_1\cap 1)f $.

  $V$ is a functor as:
  \begin{description}
    \itembf{Well Defined:} If  $f:(A,e_1) \to (B,e_2)$, then
      $(e_1\cap 1) f $ is a map in \X from $A$ to $B$ and
      $ (e_1\cap 1)(e_1\cap 1) f  (e_2 \cap 1) =
      (e_1\cap 1) (f  e_2 \cap f ) = (e_1\cap 1) (f \cap f)= (e_1\cap 1) f$, therefore,
      $V(f):V((A,e_1)) \to V((B,e_2))$.
    \itembf{Identities:} $V(e) = (e\cap 1 ) e = e \cap 1$ by
      lemma \ref{lem:properties_of_meets_in_restriction_categories}.
    \itembf{Composition:} $V(f) V(g)
      = (e_1\cap 1 ) f (e_2 \cap 1) g
      = (e_1\cap 1 ) f e_2 (e_2 \cap 1) g
      = (e_1\cap 1 ) f  (e_2 \cap e_{2}) g
      = (e_1\cap 1 ) f e_2 g
      = (e_1\cap 1 ) f g
      = V(f g)$.
  \end{description}

  Recalling from Lemma \ref{lem:properties_of_meets_in_restriction_categories}, $(e\cap 1)$
  is a restriction idempotent. Using this fact, the commutativity of restriction idempotents
  and the general idempotent identities from
  \ref{lem:properties_of_meets_in_restriction_categories}, the composite functor $U V$ is
  the identity on $\spl{r}{\X}$ as when $e$ is a restriction idempotent,
  $e = e (e\cap 1) = (e\cap 1) e = (e\cap 1)$.

  For the other direction,  note that for a particular idempotent $e:A\to A$,  this gives the
  maps $e:(A,e)\to(A,e\cap 1)$ and $e\cap 1 : (A,e\cap 1) \to (A,e)$, again by
  \ref{lem:properties_of_meets_in_restriction_categories}. These maps give the natural
  isomorphism between $I$ and $V U$ as
  \[
    \xymatrix{
      (A,e)\ar[r]^e \ar[dr]_{e} &(A,e\cap 1)\ar[d]^{e\cap 1}\\
      &(A,e)
    }\qquad \text{ and  }\qquad
    \xymatrix{
      (A,e\cap 1)\ar[r]^{e\cap 1} \ar[dr]_{e\cap 1} &(A,e)\ar[d]^{e}\\
      &(A,e\cap 1)
    }
  \]
  both commute. Therefore, $U V = I$ and $V U \cong I$, giving an equivalence of the categories.

  For the rest of this proof, functions in bold type, e.g., $\mbf$, are in $\spl{R}{\X}$.
  Functions in normal slanted type, e.g., $f$ are in \X.

  To show that $\spl{R}{\X}$ has meets,  designate the meet in $\spl{R}{\X}$ as \capspl
  and define $\mbf \capspl \mbg$ as the map given by the \X map $f \cap g$, where
  $\mbf,\mbg:(A,d)\to(B,e)$ in $\spl{R}{\X}$ and $f,g:A\to B$ in \X . This is
  a map in $\spl{R}{\X}$ as
  $d(f \cap g)e = (d\uts f \cap d\uts g) e = (f \cap g) e = (f e \cap g) = f\cap g$
  where the penultimate equality is by
  \ref{lem:properties_of_meets_in_restriction_categories}.
  By definition $\restr{\mbf \capspl \mbg }$ is $d\restr{f\cap g}$.

  It is necessary to show \capspl satisfies the four meet properties.
  \begin{itemize}
    \item{$\mbf\capspl \mbg \le \mbf$: } We need to show
      $\rst{\mbf \capspl \mbg} \mbf =  \mbf \capspl \mbg$.  Calculating now in \X:
      \begin{align*}
        d \rst{f \cap g} f&= \rst{d(f\cap g)} d f\\
        & = \rst{d f \cap d g} d f \\
        & = \rst{f \cap g} f \\
        & = f \cap g
      \end{align*}
      which is the definition of $\mbf \capspl \mbg$.
    \item{$\mbf\capspl \mbg \le \mbg$: } Similarly and once again calculating in \X,
      \begin{align*}
        d \rst{f \cap g} g&= \rst{d(f\cap g)} d g\\
        & = \rst{d f \cap d g} d g \\
        & = \rst{f \cap g} g \\
        & = f \cap g
      \end{align*}
      which is the definition of $\mbf \capspl \mbg$.
    \item{$\mbf\capspl \mbf = \mbf$: } From the definition, this is $f \cap f = f$ which
      is just $ \mbf$.
    \item{$\mbh(\mbf\capspl \mbg) = \mbh\mbf \capspl \mbh\mbg$: }
      From the definition, this is given in \X by $ h (f \cap g) =
      h f \cap h g$ which in $\spl{R}{\X}$ is $\mbh\mbf \capspl \mbh\mbg$.
  \end{itemize}
\end{proof}
% subsection split_restriction_categories (end)



\subsection{Partial Map Categories} % (fold)
\label{sub:partial_map_categories}

In \cite{cockett2002:restcategories1}, it is shown that split restriction categories are
equivalent to \emph{partial map categories}. The main definitions and results related to
partial map categories are given below.

\begin{definition}
  A collection $\Mstab$ of monics is \emph{a stable system of monics}
  when:
  \begin{enumerate}[{(}i{)}]
    \item it includes all isomorphisms;
    \item it is closed under composition;
    \item it is pullback stable.
  \end{enumerate}
\end{definition}

\emph{Stable} in this definition means that if $m:A\to B$ is in \Mstab, then for arbitrary
$b$ with co-domain $B$, the pullback
\[
  \xymatrix{
    A'\ar[r]^a \ar[d]_{m'} &A\ar[d]^{m}\\
    B' \ar[r]_{b} & B
  }
\]
exists and $m' \in \Mstab$. A category that has a stable system of monics
is referred to as an \Mstab-category.

\begin{lemma}
  If $n m \in \Mstab$, a stable system of monics, and $m$ is monic, then $n \in \Mstab$.
\end{lemma}
\begin{proof}
  The commutative square
  \[
    \xymatrix{
      A\ar[d]_n \ar[r]^{1} &A\ar[d]^{n m}\\
      A' \ar[r]_{m} & B
    }
  \]
  is a pullback.
\end{proof}

Given a category \C and a stable system of monics, the \emph{partial map category},
$\text{Par}(\C,\Mstab)$ is:
  \rcategoryequiv{$A\in\C$}
    {$(m,f):A\to B$  with $m:A' \to A$ is in \Mstab and $f:A' \to B$ is a map in \C. i.e.,
      $\xymatrix @R-15pt @C-15pt{&A'\ar[dl]_{m} \ar[dr]^{f}\\A&&B}$.}
    {$1_A,1_A:A \to A$}
    {via a pullback, $(m,f)(m',g) = (m'' m, f' g)$ where
      \[
        \xymatrix @C-15pt @R-15pt{
          &&A''\ar[dl]_{m''}\ar[dr]^{f'}\\
          &A'\ar[dl]_{m}\ar[dr]_{f}&\text{{\tiny (pb)}}&B'\ar[dl]^{m'}\ar[dr]^{g}\\
          A&&B&&C
        }
      \]
    }
    {$\restr{(m,f)} = (m,m)$}

For the maps, $(m,f) \sim (m',f')$ when there is an isomorphism $\gamma : A'' \to A'$
such that $\gamma m' = m$ and $\gamma f' = f$.

In \cite{cockettlack2003:restcategories2}, it is shown that:
\begin{theorem}[Cockett-Lack]
  Every restriction category is a full sub-category of a partial map category.
\end{theorem}
% subsection partial_map_categories (end)
\subsection{Restriction products and Cartesian restriction categories} % (fold)
\label{sub:restriction_products_and_cartesian_restriction_categories}


Restriction categories have analogues of products and terminal objects.

\begin{definition}
  In a restriction category \X, a \emph{restriction product}  of two objects $X, Y$ is an
  object $X\times Y$ equipped with \emph{total} projections
  $\pi_0:X\times Y\to X, \pi_1:X\times Y\to Y $ where:
  \begin{quote}
    $\forall f:Z\to X, g: Z\to Y,\quad \exists$ a unique $\<f,g\>:Z \to X\times Y$ such that
    \begin{itemize}
      \item $\<f,g\> \pi_0 \le f$,
      \item $\<f,g\> \pi_1 \le g$ and
      \item $\restr{\<f,g\>} = \restr{f}\, \restr{g} ( = \restr{g}\, \restr{f})$.
    \end{itemize}
  \end{quote}
\end{definition}

\begin{definition}
  In a restriction category \X\, a \emph{restriction terminal object}
  is an object $\top$ such that for all objects $X$, there is a
  unique total map $!_X : X \to \top$ and the diagram
  \[
    \xymatrix @C=40pt @R=25pt{
      X \ar[r]^{\restr{f}} \ar[d]^{f} & X \ar[r]^{!_X}  &\top  \\
      Y \ar[urr]_{!_Y}
    }
  \]
  commutes. That is,  $f\, !_Y = \restr{f}\, !_X$. Note this implies
  that a restriction terminal object is unique up to a unique isomorphism.
\end{definition}

\begin{definition}
  A restriction category \X\ is \emph{Cartesian} if it has all restriction products
  and a restriction terminal object.
\end{definition}

\begin{definition}
  An object $A$ in a Cartesian restriction category is \emph{discrete}
  when the diagonal map
  \[
    \Delta:A \to A \times A
  \]
  is a partial isomorphism.
\end{definition}


% subsection restriction_products_and_cartesian_restriction_categories (end)

\subsection{Discrete Restriction Categories} % (fold)
\label{sub:discrete_restriction_categories}

\begin{definition}\label{def:discrete_restriction_category}
  A  Cartesian restriction category where all objects are
  discrete is called a \emph{discrete restriction category}.
\end{definition}

\begin{theorem}\label{thm:a_crc_is_discrete_iff_it_has_meets}
  A Cartesian restriction category \X is discrete if and only if it has meets.
\end{theorem}
\begin{proof}
  If \X has meets, then
  \[
    \Delta(\pi_0 \cap \pi_1) = \Delta\pi_0 \cap \Delta\pi_1 = 1\cap 1 = 1.
  \]
  As $\<\pi_0,\pi_1\>$ is identity,
  \begin{align*}
    \restr{\pi_0 \cap \pi_1} &= \restr{\pi_0 \cap \pi_1} \<\pi_0, \pi_1\> \\
    &=\<\rst{\pi_0 \cap \pi_1}\pi_0, \rst{\pi_0 \cap \pi_1}\pi_1\>\\
    &=\<\pi_0 \cap \pi_1,\pi_0 \cap \pi_1\>\\
    &=(\pi_0 \cap \pi_1 )\Delta
  \end{align*}
  and therefore, $\pi_0 \cap \pi_1$ is $\inv{\Delta}$.

  To show the other direction, we set $f\cap g = \<f,g\>\inv{\Delta}$.
  By the definition of the restriction product:
  \[
    f \cap g =  \<f,g\>\inv{\Delta} =\<f,g\>\inv{\Delta} \Delta \pi_0 =
      \<f,g\>\restr{\inv{\Delta}}\pi_0 \le \<f,g\>\pi_0 \le f.
  \]
  Then, substituting $\pi_1$ for $\pi_0$ above, this gives us $f \cap g \le g$.

  For the left distributive law,
  \[
    h(f \cap g) = h \<f,g\>\inv{\Delta} =  \<h f,h g\>\inv{\Delta} = h f \cap h g.
  \]
  The intersection of a map with itself is
  \[
    f\cap f = \<f,f\> \inv{\Delta} = (f \Delta) \inv{\Delta} = f \restr{\Delta} = f
  \]
  as $\Delta$ is total. This shows that $\cap$ as defined above is a meet for the
  Cartesian restriction category \X.

\end{proof}

\begin{definition}\label{def:graphic_map}
  In a Cartesian restriction category, a map $A\xrightarrow{f}B$ is called \emph{graphic} when the
  maps
  \[
    A\xrightarrow{\<f,1\>}B\times A\qquad \text{and}\qquad
    A\xrightarrow{\<\rst{f},1\>}A\times A
  \]
  have partial inverses. A Cartesian restriction category is \emph{graphic} when all of its maps
  are graphic.
\end{definition}

\begin{lemma}\label{lem:graphic_maps_are_closed_in_a_cartesian_restriction_category}
  In a Cartesian restriction category:
  \begin{enumerate}[{(}i{)}]
    \item Graphic maps are closed under composition;
    \item Graphic maps are closed under the restriction;
    \item An object is discrete if and only if its identity map is graphic.
  \end{enumerate}
\end{lemma}
\begin{proof}
  \prepprooflist
  \begin{enumerate}[{(}i{)}]
    \item To show closure, it is necessary to show that $\<f g,1\>$ has a partial inverse.
      By Lemma \ref{lem:rcs_partial_monic_section_inverse_properties}, the uniqueness of the
      partial inverse gives
      \[
        \inv{(\<f,1\> ; \<g,1\>\times 1)} = \inv{\<g,1\>} \times 1 ; \inv{\<f,1\>} .
      \]
      By the definition of the restriction product, we have $\rst{\<f g,1\>} = \rst{f g}$. Additionally,
      a straightforward calculation shows that
        $\rst{\<f,1\>;\<g,1\> \times 1} =
          \rst{\<f\<g,1\>, 1\>} = \rst{f ;\< g,1\>}
          = \rst{\<f;g, f\>} = \rst{f g}\,\rst{f} = \rst{f g}
        $
      where the last equality is by \rtwo, \rthree and finally \rone.

    Consider the diagram
    \[
      \xymatrix @C+35pt @R+20pt{
        A \ar[r]^{\<f,1\>} \ar[drr]_{\<f g,1\>} &
           B \times A  \ar[r]^{\<g,1\> \times 1}
           &  C \times B \times A \\
        &&C \times A. \ar[u]_{1 \times \<f,1\>}
      }
    \]

    Thus,
    \begin{align*}
      \<f g,1\>  (1\times \<f,1\>) &( \inv{\<g,1\>}\times 1) \inv{\<f,1\>}\\
      &=\<f,1\>(\<g,1\>\times 1 ) (\inv{\<g,1\>}\times 1) \inv{\<f,1\>}\\
      &=\<f,1\> (\rst{g\times 1}) \inv{\<f,1\>}\\
      &=\rst{\<f,1\> (g\times 1)}  \<f,1\> \inv{\<f,1\>}\\
      &=\rst{\<f,1\> (g\times 1)}\,  \rst{\<f,1\>}\\
      &= \rst{\<f,1\>}\, \rst{\<f,1\>(g\times 1)}\\
      &= \rst{\<f,1\> (g\times 1)}\\
      &= \rst{\<f g,1\>}(=\restr{f g})\\
    \end{align*}
    showing that $1\times \<f,1\>  (\inv{\<g,1\>}\times 1 ) \inv{\<f,1\>}$ is
    a right inverse for $\<f g,1\>$.

    For the other direction, note that in general $\inv{(h k)} = \inv{k}\inv{h}$ and that
    we have $\<f g,1\> = \<f,1\> (\<g,1\>\times 1)  (1 \times \inv{\<f,1\>})$, thus
    $(1\times \<f,1\>)  (\inv{\<g,1\>}\times 1) \inv{\<f,1\>}$ will also be a left inverse and
    $\<f g,1\>$ is a restriction isomorphism.

    \item This follows from the definition of graphic and that
       $\rst{\<f,1\>} = \rst{f} = \restr{\rst{f}}$.

    \item Given a discrete object $A$, the map $1_A$ is graphic as $\<1_A,1\> = \Delta$
      and therefore $\inv{\<1,1\>} = \inv{\Delta}$. Conversely, if $\<1_A,1\>$ has an inverse,
      then $\Delta = \<1_A,1\>$ has that same inverse and therefore the object is discrete.
  \end{enumerate}
\end{proof}

\begin{lemma}\label{lem:a_discrete_crc_is_precisely_a_graphic_crc}
  A discrete restriction category is precisely a graphic Cartesian restriction category.
\end{lemma}
\begin{proof}
  The requirement is that $\<f,1\>$ (and $\<\rst{f},1\>$) each have partial inverses. For
  $\<f,1\>$, the inverse is $\rst{(1 \times f)\inv{\Delta}} \pi_1$.

  To show this, calculate  the two compositions. First,
  \[
    \<f,1\> \rst{1 \times f \inv{\Delta}} \pi_1 =
      \rst{\<f,f\> \inv{\Delta}}\<f,1\>\pi_1 % use R.4
    = \rst{f \Delta \inv{\Delta}}\<f,1\>\pi_1 % product
    = \rst{f}\<f,1\>\pi_1 % Delta total
    = \rst{f}.% product
  \]
  The other direction is:
  \begin{align*}
    \rst{(1 \times f)\inv{\Delta}} \pi_1 \<f,1\>
      &= \< \rst{(1 \times f)\inv{\Delta}} \pi_1 f ,
      \rst{(1 \times f)\inv{\Delta}}\pi_1 \>\\ %product definition
    &= \< \rst{(1 \times f)\inv{\Delta}} (1 \times f) \pi_1,
      \rst{(1 \times f)\inv{\Delta}}\pi_1 \>\\ %pi total, natural
    &= \< (1 \times f )\rst{\inv{\Delta}} \pi_1 ,
      \rst{(1 \times f)\inv{\Delta}}\pi_1 \>\\ %R.4
    &= \< (1 \times f) \rst{\inv{\Delta}} \pi_0 ,
      \rst{(1 \times f)\inv{\Delta}}\pi_1 \>\\ %below
    &= \< \rst{(1 \times f)\inv{\Delta}} (1 \times f) \pi_0,
      \rst{(1 \times f)\inv{\Delta}}\pi_1 \>\\ %R.4
  %  &= \< \rst{(1 \times f)\inv{\Delta}}\,
  %     \rst{(1 \times f)} \pi_0, \rst{(1 \times f)\inv{\Delta}}\pi_1 \>\\
    &= \< \rst{(1 \times f)\inv{\Delta}} \pi_0,
      \rst{(1 \times f)\inv{\Delta}}\pi_1 \>\\ %(a x b);pi0 = a
    &= \rst{(1 \times f)\inv{\Delta}} \< \pi_0, \pi_1 \>\\ % products
    &= \rst{(1 \times f)\inv{\Delta}}
  \end{align*}
  The above follows in a discrete restriction category, as we have
  \begin{equation*}
    \rst{\inv{\Delta}} \pi_1 = \inv{\Delta} \Delta \pi_1 = \inv{\Delta} = \inv{\Delta} \Delta \pi_0 = \rst{\inv{\Delta}} \pi_0.
  \end{equation*}

  For $\<\rst{f},1\>$, the inverse is $\rst{(1 \times \rst{f})\inv{\Delta}} \pi_1$. Similarly
  to above,
  \[
    \<\rst{f},1\> \rst{1 \times \rst{f} \inv{\Delta}} \pi_1 =
      \rst{\<\rst{f},\rst{f}\> \inv{\Delta}}\<\rst{f},1\>\pi_1 % use R.4
    = \rst{\rst{f} \Delta \inv{\Delta}}\<\rst{f},1\>\pi_1 % product
    = \rst{\rst{f}}\<\rst{f},1\>\pi_1 % Delta total
    = \rst{f}.% product
  \]
  The other direction follows the same pattern as for $\<f,1\>.$
\end{proof}
% subsection graphic_categories (end)

% section restriction_categories (end)
%!TEX root = /Users/gilesb/UofC/thesis/phd-thesis/phd-thesis.tex
\subsection{Dagger categories}\label{ssec:daggercategories}

Although dagger categories were introduced in the context of compact closed categories, the concept
of a dagger is definable independently. This was first done in \cite{selinger05:dagger}.

\begin{definition}\label{def:daggercat}
  A \emph{dagger} on a category $D$ is a functor $\dagger:\dual{\cD}\to \cD$, which is  involutive,
  that is, $\dgr{\dgr{f}} = f$ and which is the identity on objects. A \emph{dagger category} is a
  category that has a dagger.
\end{definition}

Typically, the dagger is written as a superscript on the morphism. So, if $f:A\to B$ is a map in
\cD, then $\dgr{f}:B\to A$ is a map in \cD{} and is called the \emph{adjoint} of $f$. A map where
$f^{-1} = \dgr{f}$ is called \emph{unitary}. A map $f:A\to A$ with $f=\dgr{f}$ is called
\emph{self-adjoint} or \emph{Hermitian}.

\begin{definition}\label{def:daggersmc}
  A \emph{dagger symmetric monoidal category} is a symmetric monoidal category \cD{} with a dagger
  operator such that:
  \begin{enumerate}[{(}i{)}]
    \item For all maps $f:A\to B$ and $g:C\to D$, $\dgr{(f\*g)} = \dgr{f}\*\dgr{g}:B\*D \to A\* C$;\label{defitem:dagger_smc_one}
    \item The monoid structure isomorphisms $a_{A,B,C}:(A\*B)\* C\to A\*(B\*C)$, $u^l_{A}:I\*A\to
      A$, $u^r_{A}:A\*I \to A$ and  $c_{A,B}:A\*B \to B\*A$ are unitary.\label{defitem:dagger_smc_two}
  \end{enumerate}
\end{definition}


\begin{definition}\label{def:daggercompact}
  A \emph{dagger compact closed category} \cD{} is a dagger symmetric monoidal category
  that is compact closed where the diagram
  \[
    \xymatrix @C+20pt @R+10pt{
      I \ar[r]^{\epsilon^{\dagger}_{A}} \ar[dr]_{\eta_{A}} &A\*A^{*}\ar[d]^{c_{A,A^{*}}}\\
      &A^{*}\* A
    }
  \]
  commutes for all  objects $A$ in \cD.
\end{definition}

\begin{lemma}\label{lemma:daggerbiproducts}
If \cD{} is a dagger category with biproducts, with injections $in_{1},in_{2}$ and projections
$p_{1},p_{2}$, then the following are equivalent:
\begin{enumerate}[{(}i{)}]
  \item $\dgr{p_{i}} = in_{i}, i=1,2$, \label{ldpdgrpisq}
  \item $\dgr{(f\biproduct g)} = \dgr{f}\biproduct \dgr{g}$ and $\dgr{\Delta} = \nabla$,\label{ldpddeltisnab}
  \item $\dgr{\<f,g\>} = [\dgr{f},\dgr{g}]$,\label{ldpdcopisprod}
  \item The map $[\dgr{p_{1}},\dgr{p_{2}}]: \dgr{A} \biproduct \dgr{B} \to \dgr{(A\biproduct B)}$ is
    the identity map.\label{ldpcommute}
%the below diagram commutes:
%  \[
%    \xymatrix @C+20pt @R+10pt{
%      \dgr{A} \biproduct \dgr{B} \ar[d]_{id} \ar[dr]^{[\dgr{p_{1}},\dgr{p_{2}}]}\\
%      A\biproduct B\ar[r]_{id}&\dgr{(A\biproduct B)}.
%    }
%  \]
\end{enumerate}
\end{lemma}
\begin{proof}
  \begin{description}
    \item[\ref{ldpdgrpisq}$\implies$\ref{ldpddeltisnab}] To show $\dgr{\Delta} = \nabla$,
    draw the product cone for $\Delta$,
    \[
      \xymatrix {
        &A \ar[d]^{\Delta} \ar[dr]^{id} \ar[dl]_{id}\\
        A
         & A\biproduct A \ar[l]^{p_{1}}  \ar[r]_{p_{2}}
         & A
      }
    \]
    and apply the dagger functor to it. As $\dgr{p_{i}} = in_{i}$, and $\dagger$ is identity on
    objects, this is now a coproduct diagram and therefore $\dgr{\Delta} = \nabla$.

    For $\dgr{(f\biproduct g)} = \dgr{f}\biproduct\dgr{g}$, start with the diagram defining
    $f\biproduct g$ as a product of the arrows:
    \[
      \xymatrix{
        A\ar[d]_{f}  & A\biproduct B \ar[l]_{p_{1}} \ar[r]^{p_{2}} \ar[d]^{f\biproduct g}&A \ar[d]^{g}\\
        C & C\biproduct D \ar[l]^{p_{1}} \ar[r]_{p_{2}}  & D.
      }
    \]
    Then, apply the dagger functor to this diagram. This is now the diagram defining the
    coproduct of maps and therefore $\dgr{(f\biproduct g)} = \dgr{f}\biproduct\dgr{g}$.
    \item[\ref{ldpddeltisnab}$\implies$\ref{ldpdcopisprod}] The calculation showing this is
      \begin{eqnarray*}
        &[\dgr{f},\dgr{g}] & = \nabla; (\dgr{f}\biproduct \dgr{g})\\
        & &=\dgr{\Delta}; (\dgr{f}\biproduct \dgr{g})\\
        & &=\dgr{\Delta}; \dgr{(f\biproduct g)}\\
        & & = \dgr{((f\biproduct g);\Delta)}\\
        & & = \dgr{\<f,g\>}.
      \end{eqnarray*}
    \item[\ref{ldpdcopisprod}$\implies$\ref{ldpcommute}]
      Under the assumption,
      \[
        [\dgr{p_{1}},\dgr{p_{2}}] = \dgr{\<p_{1},p_{2}\>}=\dgr{id}=id.
      \]
    \item[\ref{ldpcommute}$\implies$\ref{ldpdgrpisq}] As $[in_{1},in_{2}]:\dgr{A} \biproduct \dgr{B}
      \to \dgr{A} \biproduct \dgr{B} = id = [\dgr{p_{1}},\dgr{p_{2}}]$, we immediately have
      $\dgr{p_{1}} = in_{1}$ and $\dgr{p_{2}} = in_{2}$.
%
%Using the injections and under
%    the assumption, the following diagram commutes:
%      \[
%        \xymatrix @C+20pt @R+10pt{
%          \dgr{A} \biproduct \dgr{B} \ar[d]_{id} \ar[dr]^{[\dgr{p_{1}},\dgr{p_{2}}]}\ar[r]^{[in_{1},in_{2}]}
%            & \dgr{A} \biproduct \dgr{B} \ar[d]^{id}\\
%          A\biproduct B\ar[r]_{id}&\dgr{(A\biproduct B)}
%        }
%      \]
%      and therefore,
  \end{description}
\end{proof}

\begin{definition} \label{def:biproductdaggerccc}
  A \emph{biproduct dagger compact closed category} is a dagger compact closed category with
  biproducts where the conditions of lemma \ref{lemma:daggerbiproducts} hold.
\end{definition}
\subsection{Examples of dagger categories}

\begin{example}[\fdh]\label{ex:fdhilbert_is_dagger_category}
The category of finite dimensional Hilbert spaces is the motivating example for
the creation of the dagger and is, in fact, a biproduct dagger compact closed category. The
biproduct is the direct sum of Hilbert spaces and the tensor for compact closure is the standard
tensor of Hilbert spaces. The dual $H^{*}$ of a space $H$ is the space of all continuous linear
functions from $H$ to the base field. The dagger is defined via the adjoint as being the unique map
$\dgr{f}:B\to A$ such that $\<f a|b\> = \<a | \dgr{f} b\>$ for all $a\in A, b\in B$.
\end{example}

\begin{example}[\rel]\label{ex:rel_is_dagger_category}
The category \rel of sets and relations has the tensor $S\*T \definedas S\times T$ and the biproduct
$S\biproduct T \definedas S\disjointunion T$. This is compact closed under $A^{*} \definedas A$ and
the dagger is the relational converse. That is, if the relation
$R=\{(s,t)|s\in S, t\in T\}:S\to T$, then $\dgr{R}=R^*=\{(t,s)|(s,t)\in R\}$.
\end{example}

\begin{example}[Inverse categories]\label{ex:inverse_category_is_dagger_category}
An inverse category \X is also a dagger category when the dagger is defined as the partial inverse.
The unitary maps are the total maps. When the inverse category \X is also a
symmetric monoidal category where the monoid $\*$ is actually a restriction bi-functor, then \X is
a dagger symmetric monoidal category.

Requirement \ref{defitem:dagger_smc_one} of Definition~\ref{def:daggersmc}  is fulfilled, as
\[
  (f\*g) \inv{(f\*g)} = \rst{f\*g}=\rst{f} \*\rst{g} =
   f\inv{f} \* g \inv{g} = (f\*g) (\inv{f} \* \inv{g})
\]
and since the partial inverse of $f\*g$ is unique, $\inv{(f\*g)} = \inv{f} \* \inv{g}$.
Requirement \ref{defitem:dagger_smc_two} is that the structure isomorphisms are unitary. This is, of
course, true as each of them are isomorphisms, hence total and therefore unitary.
\end{example}
%%% Local Variables:
%%% mode: latex
%%% TeX-master: "../../phd-thesis"
%%% End:

%%!TEX root = /Users/gilesb/UofC/thesis/phd-thesis/phd-thesis.tex
\section{Semantics of quantum computation}% (fold)
\label{sec:semanticsquantum}
\subsection{Semantics of QPL}\label{sec:semanticsqpl}
\subsubsection{QPL Basics}\label{sec:qplbasics}
In \cite{selinger04:qpl} Dr. Selinger provides a denotational semantics for a quantum programming,
QPL, with the slogan of ``quantum data with classical control''. This slogan refers to the semantic
representation described in the paper, where explicit classical branching based on a classical
value is described by a \bit value with specific probabilities of being 0 or 1.

QPL is defined via a collection of functional flowchart components, where ``functional''
specifically means that each flowchart is a function from its inputs to its outputs. These
components describe the basic operations on \bits and \qbits. Edges between the components
represent the data (\bits and \qbits). These edges are labelled with a typing context and annotated
with a tuple of density matrices, describing the probability distribution of the classical data and
the state of the quantum data. In the case of purely classical data, this annotation will be a
tuple of probabilities, whereas in the case of purely quantum data, it will be a single density
matrix.

\begin{figure}[ht]
\[
  \xymatrix{
    *+<10pt>{\text{\bf Allocate bit}}
    \ar[d]^{\Gamma=A}\\
    *+[F-]{\text{new bit }b :=0}
    \ar[d]^{b:\text{\bf bit},\ \Gamma=(A,0)}\\
    {}
  }\hskip-1em
  \xymatrix{
    *+<10pt>{\text{\bf Assign bit}}
    \ar[d]^{b:\text{\bf bit},\ \Gamma=(A,B)}\\
    *+[F-]{b :=0}
    \ar[d]^{b:\text{\bf bit},\ \Gamma=(A+B,0)}\\
    {}
  }
  \xymatrix{
    *+<10pt>{\text{\bf }}
    \ar[d]^{b:\text{\bf bit},\ \Gamma=(A,B)}\\
    *+[F-]{b :=1}
    \ar[d]^{b:\text{\bf bit},\ \Gamma=(0,A+B)}\\
    {}
  }
\]
\[
  \xymatrix{
    *+<10pt>{\text{\bf Discard bit}}
    \ar[d]^{b:\text{\bf bit},\ \Gamma=(A,B)}\\
    *+[F-]{\text{discard }b}
    \ar[d]^{b:\text{\bf bit},\ \Gamma=A+B}\\
    {}
  }
  \xymatrix@!C@C=-2em@R+0ex{
     {\text{\bf Branch}} & {}
     \\
     &   *+{\text{\bf branch}~ b}
     \branchbox{a}{.35}{2ex}{2.5ex}
     \ar"a0";[dl]_<(0.2){\bf 0}_<>(.7){b:\text{\bf bit},\ \Gamma=(A,0)}
     \ar"a1";[dr]^<(0.2){\bf 1}^<>(.7){b:\text{\bf bit},\ \Gamma=(0,B)}
     \ar[u];"au"^<>(.5){b:\text{\bf bit},\ \Gamma=(A,B)}\\
     {}& &
  }
\]
\caption{Classical flowcharts}\label{fig:classicalflow}
\end{figure}


\begin{figure}[ht]
\[
  \xymatrix@!C@C=-2em@R+0ex{
    {} \ar[dr]_{\Gamma=A}   &*+<10pt>{\text{\bf Merge}} & {}  \ar[dl]^{\Gamma=B}
     \\
     & *{\circ} \ar[d]^<>(.5){\Gamma=A+B}
     \\
     &
   }
   \xymatrix{
    *+<10pt>{\text{\bf Initial}}\\
    *{\circ} \ar[d]^{\Gamma=0}\\
    {}
  }
  \xymatrix{
    *+<10pt>{\text{\bf Permute}} \ar[d]_{b_{i}:\text{bit},q_{j}:\text{qbit},\Gamma=A_{i}}\\
    *+[F-]{\text{permute} \phi} \ar[d]_{b_{\phi(i)}:\text{bit},q_{\phi(j)}:\text{qbit},\Gamma=A_{\phi(i)}}\\
    {}
  }
\]
\caption{General flowcharts}\label{fig:generalflow}
\end{figure}

\begin{figure}[ht]
\[
  \xymatrix{
    *+<10pt>{\text{\bf Allocate qubit}}
    \ar[d]^{\Gamma=A}\\
    *+[F-]{\text{new qubit }q :=0}
    \ar[d]^{q:\text{\bf qubit},\ \Gamma=\qsmatss{A}{0}{0}{0}}\\
    {}
  }\hskip-2em
  \xymatrix{
    *+<10pt>{\text{\bf Unitary Transform}}
    \ar[d]^{\vec{q}:\text{\bf qubit},\ \Gamma=A}\\
    *+[F-]{q *= S}
    \ar[d]^{\vec{q}:\text{\bf qubit},\ \Gamma=(S \* I) A (S \* I)^{*}}\\
    {}
  }
\]
\[
  \xymatrix@!C@C=-2em@R+0ex{
    {\text{\bf Discard qubit}}
    \ar[d]^{q:\text{\bf qubit},\ \Gamma=\qsmatss{A}{B}{C}{D}}\\
    *+[F-]{\text{discard } q}
    \ar[d]^{\Gamma=A+D}\\
    {}
  }
\]
\[
  \xymatrix@!C@C=-2em@R+0ex{
    {\text{\bf Measure}} & {}
    \\
    &   *+{\text{\bf measure}~ q}
    \branchbox{a}{.35}{2ex}{2.5ex}
    \ar"a0";[dl]_<(0.2){\bf 0}_<>{q:\text{\bf qubit},\ \Gamma=\qsmatss{A}{0}{0}{0}}
    \ar"a1";[dr]^<(0.2){\bf 1}^<>{q:\text{\bf qubit},\ \Gamma=\qsmatss{0}{0}{0}{D}}
    \ar[u];"au"^<>(.5){q:\text{\bf qubit},\ \Gamma=\qsmatss{A}{B}{C}{D}}\\
    {}& &
  }
\]
\caption{Quantum flowcharts}\label{fig:quantumflow}
\end{figure}

In figure \ref{fig:classicalflow}, the annotation $\Gamma$ consists of a tuple of probabilities,
with $n$ \bits requiring $2^{n}$ probabilities for their description. In figure
\ref{fig:quantumflow}, $\Gamma$ will consist of a density matrix of size $2^{m}\times 2^{m}$ for
$m$ \qubits. Note also that in figure \ref{fig:quantumflow}, the notation $\vec{q}$ indicates an
ordered set of \qubits.

In QPL, the classical operations consist of: \emph{Allocate bit, Assignment, Discard bit} and
\emph{Branch}. The quantum operations are:\emph{Allocate qubit, Unitary Transform, Discard qubit}
and \emph{Measure}. The operations applicable to both types of data are \emph{Merge, Initial} and
\emph{Permute}. These are found in Figure \ref{fig:generalflow}.

When components are combined, the type annotation $\Gamma$ consists of a tuple of density matrices.
Flowchart components must be combined so that they are connected via edges with identical typing
judgements. Flowcharts may have arbitrary numbers of input and output edges. By convention,
component flow is from the top down and programs are read is the same manner.

The semantics of a component is the function that calculates the matrix tuple(s) of the output
edges when given the matrix tuple of the input edges. Each of these functions is linear and
preserves adjoints. They also preserve positivity and the sum of the traces of the output edges
equals the sum of the traces of the input edges, which can be viewed as the probability of leaving
a fragment is the same as the probability of entering a fragment.

\subsubsection{Looping, subroutines and recursion}
In the flow chart representation of QPL, looping occurs when one edge is connected to a component
above the component originating the edge. Subroutines are represented by boxes with double left and
right lines. A subroutine may have multiple input and output edges and is considered shorthand for
the flowchart making up the subroutine. For example, see figure \ref{fig:procandloop}, where the
subroutine $Proc1$ accepts two \qubits $q,r$ as input and produces either two \qubits $c,d$ or a
single \qubit $q$. In the case when the output is the single \qubit $q$, the output is looped back
to be merged with the original input.

\begin{figure}
\[
  \xymatrix@!C@C=-4em@R-.7ex{
    *[.]{} \ar[dr]_<>(.5){q:\text{\bf qubit}}
    &&
    *[.]{}
    \\
    &*{\circ} \ar[d]^<>(.5){q:\text{\bf qubit}}
    \\
    &*+[F-]{\text{\bf new qubit}~r:={\bf 0}}\ar[d]^<>(.5){q,r:\text{\bf qubit}}
    \\
    &*+{\text{\bf input}~q,r}
    \\
    *+{\text{\bf output1}~c,d}
    \ar[d]^<>(.5){c,d:\text{\bf qubit}}
    &&
    *+{\text{\bf output2}~a}
    \procbox{[ll].[].[ul]}{Proc1}
    \ar`d[drr]`[r]`[uuuu]_<>(.5){a:\text{\bf qubit}}`^dl[uuul][uuul]
    &&
    *[.]{}
    \\
    *[.]{}
    &&
    *[.]{}
    &&
    *[.]{}
  }
\]
\caption{Example of a subroutine and loop}\label{fig:procandloop}
\end{figure}


\begin{figure}
\[
  \xymatrix@!C@C=1em@R-.7ex{
    *[.]{}
    &&\\
    *[.]{} \ar[dr]_<>(.5){n \text{ inputs}}
    &&
    \\
    &*+<10pt>[F-]{X}\ar[dl]_<>(.5){m \text{ outputs}}
     \ar`d[dr]`^u`[u]`^dl[][]
    &&
    *[.]{}
    \\
    *[.]{}
    &&
    *[.]{}
    &&
    *[.]{}
  } \hskip-6em
  \xymatrix@C=7pt@R=5pt@H+2pt{
    & \ar[dd]_{A} & & \\
    &  & & *{\circ}\ar[d] \\
    &\save [].[rr]!C="b1"*[F] \frm{}\restore \ar`d[dl]`[dddd]^{F_{11}(A)}[ldddd] & *+{X}
      & \ar[ddd]^{F_{21}(A)}\\
    & & &\\
      & *{\circ} \ar[d]& & \\
    &\save [].[rr]!C="b2"*[F] \frm{}\restore \ar`d[dl][dl] & *+{X} &\ar[ddd]^{F_{22}F_{21}(A)}\\
    *{\circ} \ar[ddd]_{F_{11}(A)+F_{12}F_{21}(A)}& & & \\
     & *{\circ} \ar[d]& & \\
    &\save [].[rr]!C="b2"*[F] \frm{}\restore   \ar`d[dl][dl] & *+{X}
      &\ar@{-}[dd]^{F_{22}F_{22}F_{21}(A)}\\
    *{\circ} \ar@{. }[d]& & &\\
    \ar[d]_{G(A)} & & & *{\vdots}\\
    & & &
  }
\]
\caption{Unwinding a loop}\label{fig:loopunwinding}
\end{figure}


Semantics of looping is based on ``infinite unwinding''. It is interesting to note this is similar
to the method used in \cite{kozen-semanticsprobabilistic} where a \texttt{while} program
construction is unwound to
\begin{quote}
  \texttt{if} $\neg B$ \texttt{then} $I$ \texttt{else} $S$; \texttt{while} $B$ \texttt{do} $S$
  \texttt{od fi};
\end{quote}
and the semantics of \texttt{while} is given as a fixpoint $W$ of the equation $W=e_{\neg B} +
(e_{B} ; S ; W)$. Referring to figure \ref{fig:loopunwinding}, the input to $X$ is $n+k$ density
matrices, the output is $m+k$ density matrices, where the $k$ matrices partake in the loop. This
can be written as $F(A,C) = (B,D)$ with $A=(A_{1},\ldots,A_{n})$, $B=(B_{1},\ldots,B_{m})$, where
$F$ is the linear function giving the semantics of $X$. This allows creation of four component
functions, where $F(A,0) = (F_{11}(A),F_{21}(A))$ and $F(0,C) = (F_{12}(C),F_{22}(C))$.

Following the right hand side of figure \ref{fig:loopunwinding}, the state of the edges at the end
are given by
\begin{equation}
  G(A) = F_{11}(A)+\sum_{i=0}^{\infty}F_{12}(F^{i}_{22}(F_{21}(A)))\label{eqn:loopunwinding}
\end{equation}
I will show later that this is a convergent sum.

The semantics of subroutines without recursion is the same as ``in-lining'' the subroutine at the
place of its call. The first requirement for this is that the program handle renaming of variables
as the formal parameters of the subroutine may have different names than the calling parameters.
For \bits, renaming $b$ to $c$ may be accomplished by the fragment
\begin{quote}
  \textbf{new bit $c:=0$;\\  branch $b$ 0>\{ \} 1>\{$c:=1$\} ;\\
   discard $b$}.
\end{quote}
For \qbits, renaming $q$ to $r$ is done by the fragment
\begin{quote}
  \textbf{new qubit $p :=0$; \\
  q,p *=CNOT; \\
  p,q *=CNOT; \\
  discard q}.
\end{quote}
The second requirement is that the program needs to be able to extend the semantic context. That
is, suppose that a subroutine $X$ is defined with input typing $\Gamma$, with semantic function
$F$. This means that starting with $\Gamma=A$, applying the subroutine $X$ gives us the typing and
context $\Gamma'=F(A)$. Inlining a subroutine requires that the addition of an arbitrary number of
\bits and \qbits to the context and be able to derive the semantics. But, since each of the
components of flow charts and looping are linear functions, this is a straightforward induction on
the structure of the subroutine. The proof for one of the cases is below.
\begin{lemma}[Context extension]\label{lemma:contextextension}
Given a subroutine $X$ in context $\Gamma=A$ with semantics $F$ (i.e., applying $X$ to $\Gamma=A$
gives $\Gamma'=F(A)$),
\begin{itemize}
  \item{} The result of $X$ in context $b:$bit, $\Gamma=(A,B)$ is $b:$bit, $\Gamma'=(F(A),F(B))$.
  \item{} The result of $X$ in context $q:$qubit, $\Gamma=\qsmat{A}{B}{C}{D}$ is $q:$qubit, $\Gamma'=\qsmat{F(A)}{F(B)}{F(C)}{F(D)}$.
\end{itemize}
\end{lemma}
\begin{proof}
  \textbf{Case $X=$``allocate bit''}
  The semantics of allocate bit is $F(A) = (A,0)$, where the number of $0$ density matrices is the same as the number
  of density matrices in $A$. When the additional context is a \bit, the starting context is $x:$bit$,\Gamma=(A,B)$, where again the number and
  dimensions of density matrices in $A$ and $B$ agree. After applying $X$,
  $b:$bit,$x:$bit$\Gamma'=(A,B,0,0)$. Next, permute $x$ and $b$, to retain $x$ in the correct order and get
  $x:$bit,$b:$bit$\Gamma'=(A,0,B,0) = ((A,0),(B,0) = (F(A),F(B))$.

  When the additional context is a \qubit, the starting context is $x:$qubit$,\Gamma=\qsmat{A}{B}{C}{D}$. After allocation of a bit,
  \begin{eqnarray*}
    &b:\text{bit},x:\text{qubit},\Gamma&=(\qsmat{A}{B}{C}{D}, \qsmat{0}{0}{0}{0}) \\
    &&= \qsmat{(A,0)}{(B,0)}{(C,0)}{(D,0)}\\
    &&=\qsmat{F(A)}{F(B)}{F(C)}{F(D)}.
  \end{eqnarray*}
\end{proof}

As the semantics of each of the components of the flowchart language are linear functions, the
technique used in the example case in Lemma \ref{lemma:contextextension} is applicable for each of
these flowchart components. Furthermore, as the semantics are compositional, this extends to
looping.

For recursive subroutines, a variant of the infinite unwinding is used. A recursive subroutine is
one that calls itself in some way. If we have a subroutine $X$, let $X(Y)$ be the flowchart defined
as $X$ with the recursive call to itself replaced with a call to $Y$. As the semantics are
compositional, there is some function $\Theta$ such that give the semantics of $X(Y)$ from the
semantics of $Y$. Let $Y_{0}$ be a non-terminating program and define $Y_{i}$ by the equation
$Y_{i+1} = X(Y_{i})$. Denote the semantics of $Y_{i}$ by $F_{i}$. Then $F_{0} = 0$ and $F_{i+1} =
\Theta(F_{i})$. From this, define
\begin{equation}\label{eqn:semrecursion}
  X =\lim_{i\to\infty}F_{i}.
\end{equation}
The existence of this limit will be discussed below in section \ref{sec:catsemanticsofqpl}.

An important point to note here is that the semantics of an arbitrary $G$ may actually reduce the
trace, that is, there may be a non-zero chance the program will not terminate.

\subsubsection{Categorical semantics of QPL}\label{sec:catsemanticsofqpl}
While the exposition above referred to the functions described in the flowchart components as the
semantics of the program, this section will give a formal definition of the categorical semantics.

\begin{definition}[Signature]\label{def:signature}
  A \emph{signature} is a list of positive non-zero integers, $\sigma=n_{1},\ldots,n_{s}$ which is
  associated with the complex vector space $V_{\sigma}=\C^{n_{1}\times
  n_{1}}\times\cdots\times\C^{n_{s}\times n_{s}}$
\end{definition}
Designate the elements of $V_{\sigma}$ by tuples of matrices, $A=(A_{1},\ldots,A_{s})$. The trace
of $A$ will be the sum of the traces of the tuple matrices. A will be said to have a specific
property when all matrices in the tuple have that property, e.g., positive, hermitian.

From the above, now define the category $V$ with objects being signatures $\sigma$ and maps from
$\sigma$ to $\tau$ being any complex linear function from $V_{\sigma}$ to $V_{\tau}$. Define $\+$
by concatenation of signatures. Then, $\+$ is both a product and coproduct in $V$. The co-pair map
$[F,G]:\sigma\+\sigma' \to \tau$ is defined as $[F,G](A,B) = F(A)+F(B)$, while the pairing map
$\<F,G\>:\sigma \to \tau\+\tau'$ is defined as $\<F,G\>(A)=(FA,GA)$. Additionally, define the
tensor $\*$ on $\sigma=n_{1},\ldots,n_{s}$ and $\tau=m_{1},\ldots,m_{t}$ as \[\sigma\*\tau =
n_{1}m_{1},n_{1}m_{2},\ldots,n_{s},m_{t}.\] This tensor, together with the unit $I=1$ makes $V$ a
symmetric monoidal category. Note that it is also distributive with $\tau\*(\sigma \+\sigma') =
(\tau\*\sigma)\+(\tau\*\sigma')$.

As $V$ is equivalent to the category of finite dimensional vector spaces, we will need to restrict
the morphisms to those that can occur as programs in QPL. $V$ has too many morphisms, for instance
the signature $1,1$ (which will be designated as \bit) is isomorphic to the signature $2$ (which
will be designated as \qbit).

\begin{definition}[Superoperator]
  Given $F:V_{\sigma}\to V_{\tau}$, define:
  \begin{itemize}
    \item $F$ as \emph{positive} if $F(A)$ is positive for all positive $A$;
    \item $F$ as \emph{completely positive} if $id_{\rho}\*F:V_{\rho\*\sigma}\to V_{\rho\*\tau}$ is
      positive for all $\rho$;
    \item $F$ as a \emph{superoperator} if it is completely positive and $\text{tr }F(A)\le$~tr~$A$
      for all positive $A$.
  \end{itemize}
\end{definition}
The definition of a superoperator is trace \emph{non-increasing} rather than trace
\emph{preserving} due to the possibility of non-termination in programs.

Considering superoperators in the category $V$, there a number of properties that hold. It is
immediate to see that an identity map is a superoperator and that compositions of superoperators
are again superoperators. The canonical injections $i_{1}:\sigma\to \sigma\+\tau$ and
$i_{2}:\tau\to \sigma\+\tau$ are superoperators. The remain properties of interest are detailed in
the following lemma.
\begin{lemma}\label{lemma:superoperator}
In the category $V$, the following hold:
\begin{enumerate}
  \item  If $F:\sigma\to\tau$ and $G:\sigma'\to\tau$ are superoperators, so is
  $[F,G]:\sigma\+\sigma'\to\tau$.
  \item If $F:\sigma\to\sigma'$ and $G:\tau\to\tau'$ are superoperators, then so are $F\+G$ and
    $F\*G$.
  \item if $id_{\nu}\*F$ is positive for all one element signatures $\nu$, then $F$ is completely
    positive.
  \item Given $U$, a unitary $n\times n$ matrix, then $F:n \to n$ defined as $F(A) = UAU^{*}$ is a
    superoperator.
  \item If $T_{1},T_{2}$ are $n\times n$ matrices such that $T_{1}^{*}T_{1} + T_{2}^{*}T_{2} = I$,
    then $F:n\to n,n$ defined as $F(A) = (T_{1}AT_{1}^{*}, T_{2}AT_{2}^{*})$ is a superoperator.
\end{enumerate}
\end{lemma}
\begin{proof}
  For statement 1, as $tr\ (F(A)+F(B)) = (tr\ F(A))+(tr\ F(B)) \le tr\ A + tr\ B = tr\ (A,B)$, note
  that $[F,G]$ satisfies the trace condition. Secondly, because of distributivity $id_{\rho}\*[F,G]
  = [id_{\rho}\*F,id_{\rho}\*G]$ and the complete positivity follows. The first assertion of
  statement 2 follows in a similar manner. As $F\*G = (F\*id_{\tau}) (id_{\sigma'}\*G)$, note that
  each element of the composition is a superoperator, hence so is $F\*G$. In statement 3, note that
  any signature $\nu$ of length $n$ is equal to a coproduct of $n$ single element signatures,
  $\nu_{1}\+\cdots\+\nu_{n}$. Then using distributivity, $id_{\nu}\*F = (id_{\nu_{1}}\*F)\+\cdots\+
  (id_{\nu_{1}}\*F)$ which by assumption and statement 2 is positive. Hence, $F$ is completely
  positive. For statement 4, as $U$ is unitary, it is immediate that $F$ is positive and preserves
  the trace. Note also that $(id_{n} \* F)(A) = (I\*U)A(I\*U)^{*}$ where $I$ is the $n\times n$
  identity matrix. However, $I\*U$ is also a unitary matrix, therefore $(id_{n} \* F)$ is positive
  for all $n$ and by the previous point, $F$ is completely positive and therefore a superoperator.
  For the last statement, by construction $F$ preserves both positivity and trace and by a similar
  argument to the previous point, it is a superoperator.
\end{proof}

At this point there is now sufficient machinery to define the category $Q$ which will be used for
the categorical semantics for QPL. Define $Q$ as the subcategory of $V$ having the same objects,
but only superoperators as morphisms. By lemma \ref{lemma:superoperator}, this is a valid
subcategory, which inherits $\+$ as a coproduct. It is not a product as the diagonal morphism
increases the trace and is therefore not a superoperator.

$Q$ is also a CPO enriched category. First, note that superoperators send density matrices to
density matrices (positive hermitian matrices with trace $\le 1$). Designating $D_{\sigma}$ to be
the subset of density matrix tuples contained in $V_{\sigma}$, then for any superoperator $F$, it
can be restricted to the density matrices. This restricted function preserves the L\"owner order
from definition \ref{def:lownerorder} and it preserves the least upper bounds of sequences. From
this, given signatures $\sigma$ and $\tau$, define a partial order on $Q(\sigma,\tau)$ by $F\le G$
when $\forall \nu, A\in D_{\nu\*\sigma}: (id_{\nu} \* F)(A) \le (id_{\nu} \* G)(A)$.

\begin{lemma}\label{lemma:qiscpoenriched}
The poset $Q(\sigma,\tau)$ is a complete partial order. Composition, co-pairing and tensor are
Scott-continuous and therefore $Q$ is CPO-enriched.
\end{lemma}

As $Q$ is a CPO enriched category, it is possible to define a monoidal trace over the coproduct
monoid. Recall that if $F:\sigma \+ \tau \to \sigma' \+ \tau$, then $tr\ F$ is a map, $\sigma \to
\sigma'$. Given such an $F$, construct the trace as follows:
\begin{itemize}
  \item Define $T_{0}$ as the constant zero function.
  \item Define $T_{i+1} = F ; [id_{\sigma'},i_{2} H_{i}]: \sigma\+ \tau \to \sigma'$.
\end{itemize}
Then, $T_{0}\le T_{1}$ as 0 is the least element in the partial order. $T_{i}\le T_{i+1}$ for all
$i$ as all the categorical operations are monotonic due to the CPO enrichment. Therefore, now
define $T = \vee_{i} T_{i}:\sigma\+ \tau \to \sigma'$. Finally, define $tr\ F = i_{1}; T: \sigma
\to \sigma'$.

This trace construction may be compared to the loop semantics construction in section
\ref{sec:qplbasics}. For $F$ as above, we may decompose it into components
$F_{11}:\sigma\to\sigma'$, $F_{21}:\sigma\to\tau$, $F_{12}:\tau\to\sigma'$ and
$F_{22}:\tau\to\tau$. This gives us
\begin{eqnarray*}
  &T_{0}(A,0) &= 0,\\
  &T_{1}(A,0) &= F_{11}(A),\\
  &T_{2}(A,0) &= F_{11}(A) + F_{12}F_{21}(A),\\
  &&\vdots
\end{eqnarray*}
which brings us to
\begin{equation*}
  (Tr\ F)(A) = T(A,0) = F_{11}(A) +\sum_{i=0}^{\infty}F_{12}(F_{22}^{i}(F_{21}(A))).
\end{equation*}
This is the same construction as equation (\ref{eqn:loopunwinding}) and will be used for the
interpretation of loops. In particular, this justifies the convergence of the infinite sum in that
equation.

At this point, we now have the information required to give an interpretation of the quantum flow
charts of QPL in the category $Q$. There are two types, \interp{\bit{}}$= 1,1$ and
\interp{\qubit{}}$=2$. The interpretation of basic operations is given in Table \ref{tab:interp}.
\begin{table*}[ht]
  \caption{Interpretation of QPL operations}
  \begin{center}
    \begin{eqnarray*}
      \interp{\text{new bit }b:=0} &=newbit:I\to \bit: & a\mapsto (a,0)\\
      \interp{\text{discard }b} &=discardbit:\bit\to I: & (a,b)\mapsto a+b\\
      \interp{b:=0} &=set_{0}:\bit\to\bit:&(a,b)\mapsto(a+b,0)\\
      \interp{b:=1} &=set_{1}:\bit\to\bit:&(a,b)\mapsto(0,a+b)\\
      \interp{\text{branch }b}&=branch: \bit\to\bit\+\bit:&(a,b)\mapsto(a,0,0,b)\\
      \interp{\text{new qbit }q:=0} &=newqbit:I\to \qbit:
        & a\mapsto \begin{pmatrix}a&0\\0&0\end{pmatrix}\\
      \interp{\text{discard }q} &=discardqbit:\qbit\to I:
        & \begin{pmatrix}a&b\\c&d\end{pmatrix}\mapsto a+d\\
      \interp{\vec{q}*=U} &=unitary_{U}:\qbit^{n}\to\qbit^{n}:&A\mapsto UAU^{*}\\
      \interp{\text{measure }q}&=measure: \qbit\to\qbit\+\qbit :
        &\begin{pmatrix}a&b\\c&d\end{pmatrix}\mapsto(\begin{pmatrix}a&0\\0&0\end{pmatrix},
        \begin{pmatrix}0&0\\0&d\end{pmatrix})\\
      \interp{\text{merge}} &= merge:I\+I \to I : &(a,b)\mapsto (a+b)\\
      \interp{\text{initial}}&=initial: 0 \to I : &() \mapsto 0\\
      \interp{\text{permute }\Phi} &= permute_{\Phi}: \+_{i}A_{i} \to \+_{i}A_{\Phi(i)}
    \end{eqnarray*}
  \end{center}
  \label{tab:interp}
\end{table*}

Additionally, if a type context $\Gamma$ is $x_{i}:T_{i}$, then
$\interp{\Gamma}=\*_{i}\interp{A_{i}}$. If $\bar{\Theta}$ is a list of typing context,
$\Theta_{i}$, then $\interp{\bar{\Theta}} = \+_{i}\interp{\Theta_{i}}$. For the various types of
composite flowcharts, the interpretation is as follows:
\begin{itemize}
  \item If the context $\Gamma$ is added to flowchart $A$, producing $B$, then
    $\interp{B} = \interp{A}\*\interp{\Gamma}$.
  \item If the outputs of flowchart $A$ are connected to the inputs of $B$, giving flowchart $C$,
    then $\interp{C}=\interp{A} ; \interp{B}$.
  \item If flowchart $C$ is made up of parallel flowcharts $A$ and $B$, then
    $\interp{C} = \interp{A}\+\interp{B}$.
  \item if flowchart $C$ is a loop on flowchart $A$, then $\interp{C} = tr\,(\interp{X})$.
\end{itemize}

For procedures, it is necessary to consider abstract variable flowcharts with specified types,
$R_{i}:\overline{\Theta_{i}} \to \overline{\Theta'_{i}}$. With these variable flowcharts allowed,
these may be interpreted in a specified environment $\kappa$ which maps the $R_{i}$ to specific
morphisms of $Q$ with the appropriate type. Then, $\interp{R_{i}}_{\kappa} = \kappa{R_{i}}$ and if
$A$ is a flow chart using $R_{i}$, its interpretation relative to $\kappa$ may be built up
inductively via the operations above, giving a function $\Omega_{A}$ which will map the
environments to a specific map in $Q$.

For recursion, consider the recursive subroutine defined as $P=T(P)$ for a flowchart $T$. Then
$\Omega_{T}:Q(\sigma,\tau)\to Q(\sigma,\tau)$ will be a Scott-continuous function. In this case,
$\interp{Y}$ will be the least fixed point of $\Omega_{T}$. There is an increasing sequence for all
$i\ge0$, $S_{i} \le S_{i+1}$ given by $S_{0} = 0$ and $S_{i+1} = \Omega_{T}(S_{i})$. This gives the
interpretation of $P$ as \[\interp{P} = \vee_{i} S_{i} = \lim_{i} S_{i}.\] This corresponds to
equation (\ref{eqn:semrecursion}) above, which shows this is the correct interpretation for
recursive procedures and since $Q$ is a CPO enriched category with a least point, therefore this
limit exists and therefore the the limit in equation (\ref{eqn:semrecursion}) will exist.


\subsubsection{Conclusions for QPL}\label{sec:conclusionsqpl}
\paragraph{Data types}
As can be seen by the proceeding pages, creating the categorical machinery for a semantic
interpretation of QPL is quite detailed. The paper \cite{selinger04:qpl} goes on to prove
soundness, completeness and provides some alternative syntaxes for QPL. The subject of structured
types is discussed briefly. Tuple types $(\Gamma,\Lambda)$ may be constructed as
$\interp{\Gamma}\*\interp{\Lambda}$. This immediately provides types such as fixed length classical
or quantum integers, characters, and so forth. Sum types can similarly be added as $\Gamma \+
\Lambda$, noting that the ``choice'' between the two types remains classical. The one primary
weakness in the type system is not allowing structured recursive types such as \type{List}. This
weakness is addressed in a follow on paper, \cite{huet2007}. In this paper, the major change is
that rather than restricting to a tuple of integers for the objects of $V$ and $Q$, they consider
arbitrary families of integers as the objects, defining a new category $Q^{\infty}$. Definitions
such as positive, Hermitian, the L\"owner order and trace follow in a straightforward manner, as
does the definition $V_{\sigma} = \prod_{i\in|\sigma|}\C^{\sigma_{i}\times\sigma_{i}}$, noting this
is now an infinite product. Note that in the infinite dimensional case there is no canonical basis
for $V_{\sigma}$ and therefore no canonical isomorphism between $V_{\sigma\*\tau}$ and
$V_{\sigma}\*V_{\tau}$. To rectify this, the authors refine the allowed morphisms in the category
$Q^{\infty}$. First, the define the category $\overline{Q^{\infty}}$ as having infinite signatures
as objects, but maps $f:\sigma\to\tau$ are maps $f:D_{\sigma}\to D_{\tau}$ ($D_{\sigma}$ are the
density matrix tuples of $V_{\sigma}$). These $f$ are called \emph{positive operators}. They are
required to extend to linear maps $\overline{f}: V_{\sigma} \to V_{\tau}$ and be continuous for the
L\"owner order. These maps are definable as a matrix of maps over finite dimensional spaces
$f_{ij}:\C^{\sigma_{i}\times\sigma_{i}}\to\C^{\tau_{i}\times\tau_{i}}$, calling this the
\emph{operator matrix}.

One can now define the tensor of two positive operators $f,g$ by tensoring their respective
operator matrices. Then, following the finite case, the positive operator $f:\sigma\to\tau$ with
operator matrix $F$ is a superoperator, if when $ID_{\gamma}$ is the operator matrix for the
identity on the signature $\gamma$, $ID_{\gamma} \* F$ is a positive operator. The infinite case
superoperators follow the desired properties as in the finite case and the category $Q^{\infty}$ is
defined as the category with objects being infinite signatures and morphisms these superoperators.

From this, the authors show that any endofunctor definable via an ``arithmetic'' equation involving
the coproduct and tensor will give rise to a data type in the category. In particular, one can
define \qubit lists as \[QList = 1 \+ (\qbit \* QList)\] and trees of \qbits as \[QTree =
\qbit\+(QTree \* QTree).\]

\paragraph{Quantum communication}
QPL makes no attempts to handle communication or transmission of quantum data. This will not be
addressed in this thesis.

\paragraph{Higher order functions}
QPL is defined as a functional language. One of the expectations of modern functional languages is
that programs themselves are first class objects, that is, they may be operated on by the program.
Typical uses are partial evaluation and passing a subroutine of a specified type for use by another
subroutine. In quantum computation, the primary issue with this in how does one guarantee the
no-cloning, no-erasing rules with respect to quantum data. Work on a quantum lambda calculus,
\cite{tonder03:qcsemantics,valiron:thesis}, has attempted to address this, albeit primarily with
operational rather than denotational semantics. In, \cite{selinger04:towardssemantics}, the author
explores the use of cones rather than vector spaces to create a denotational semantics, but finds
that the candidates fail to provide the correct answer over the base types. We will not be
considering the higher-order issues further in this paper.

\subsection{Semantics of pure quantum computations}\label{sec:puresemantics}
In \cite{abramsky04:catsemquantprot}, the authors approach the creation of a categorical semantics
for quantum computation independently of a specific language. Rather, they use finitary quantum
mechanics as their reference point.

Finitary quantum mechanics consists of the following:
\begin{enumerate}
  \item The system's state space is represented by a finite dimensional Hilbert space $H$.
    \label{lis:qfm1}
  \item The basic type of the system is that of \qubit --- 2-dimensional Hilbert space --- with the
    computational basis $\{\kz, \ko\}$.\label{lis:qfm2}
  \item Compound systems are tensor products of the components. This is what enables
    \emph{entanglement} as the general form of the system $H\*J$ where $H$ and $J$ are Hilbert
    spaces is
    \[
      \sum_{i=1}^{n}\alpha_{i} (u_{i} \* v_{i})
    \]
    where $u_{i}$ is a basis element of $H$ and $v_{i}$ is a basis element of $J$.\label{lis:qfm3}
  \item The basic transforms are \emph{unitary transformations}. \label{lis:qfm4}
  \item The measurements performable are \emph{self-adjoint} (hermitian) operators - with two
    sub-steps:\label{lis:qfm5}
    \begin{enumerate}
      \item The actual act of measurement. (Preparation).\label{lis:qfm5a}
      \item The communication of the results of the measurement. (Observation).\label{lis:qfm5b}
    \end{enumerate}
\end{enumerate}
The above definition does allow for the possibility of mixed states, as described in section
\ref{sec:density}, but for the remainder of this section, it is assumed both steps of the
measurement are carried out, resulting in pure states only.

\cite{abramsky04:catsemquantprot} gives the interpretation of finitary quantum mechanics in the
context of a biproduct dagger compact closed category, \cD.
\begin{description}
  \item[\ref{lis:qfm1}.] An $n-$dimensional state space $S$ is an object of \cD,
    together with a unitary isomorphism $base_{A}:\+^{n}I\to A$.
  \item[\ref{lis:qfm2}.] A \qubit is a 2 dimensional state space $Q$ with the computational basis
    $base_{Q}:I\+I \to Q$.
  \item[\ref{lis:qfm3}.] Compound systems $A,B$ are described by $A\*B$ and
    $base_{A\*B} = \phi (base_{A}\*base_{B})$ where $\phi:\+^{nm}I \cong(\+^{n}I)\*(\+^{m}I)$ is
    the isomorphism obtained by repeated application of distributivity isomporphisms.
  \item[\ref{lis:qfm4}.] The basic transformations are unitary transformations, i.e., $f$, where
    $\dgr{f} = f^{-1}$.
  \item[\ref{lis:qfm5a}.] A preparation is a morphism $P:I \to A$ which has a corresponding unitary
    morphism $f_{P}:\+^{n}I\to\+^{n}I$ and
    \[
      \xymatrix{
        I \ar[r]^{P} \ar[d]_{i_{1}}& A\\
        \+^{n}I \ar[r]_{f_{P}} & \+^{n}I \ar[u]_{base_{A}}
      }
    \]
    commutes.
  \item[\ref{lis:qfm5b}.] An observation  is an isomorphism $O = \+^{n}O_{i}$ with components
    $O_{i}:A \to I$ which has an unitary automorphism $f_{O}:\+^{n}I\to\+^{n}I$ such that
    \[
      \xymatrix{
        A \ar[r]^{O_{i}} & I\\
        \+^{n}I \ar[r]_{f_{O}}  \ar[u]_{base_{A}} & \+^{n}I \ar[u]_{p_{i}}
      }
    \]
    commutes for all $i=1,\ldots,n$. The observational branches are the individual $O_{i}:A \to I$.
\end{description}
Additionally, the biproduct $\+$ represents distinct branches resulting from measurement.
Accordingly, any operation on a biproduct must be an explicit biproduct, that is $f:A\+B\to C\+D$
will be $f_{1}\+f_{2}$ with $f_{1}:A\to C$ and $f_{2}:B\to D$.

The authors go on to show how this interpretation is sufficient to model quantum teleportation,
logic gate teleportation and entanglement swapping.


\subsection{Complete positivity}\label{sec:completepositivity}
Given a $\dagger$-compact closed category, it is possible to construct its category of completely
positive maps.

\begin{definition}[Positive map]\label{def:positivemap}
  A map $f:A\to A$ in a dagger category is called \emph{positive} if there is an object $B$ and a
  map $g:A\to B$ with $f = g \dgr{g}$
\end{definition}

\begin{definition}[Trace]\label{def:tracecp}
  For $f:A\to A$ in a compact closed category, its \emph{trace} is defined as $tr\, f:I\to I =
  \eta_{A} ; c_{A^{*},A} ; (f\*A^{*}) ; \epsilon$.
\end{definition}

The following lemma gives some properties of positive maps:

\begin{lemma}\label{lemma:positivemaps}
  In any biproduct dagger compact closed category, the following hold:
  \begin{enumerate}
    \item{} $f$ positive $\implies$ $h f \dgr{h}$ is positive for all maps $h$.
    \item{} $id_{A}$ is positive.
    \item If $f:A\to A$ and $g:B\to B$ are positive, so are $f\*g$ and $f\+g$.
    \item $0_{A,A}$ is positive. If $f,g:A\to A$ is positive, so is $f+g$.
    \item $f$ positive $\implies$ $\dgr{f}=f$.
    \item $f$ positive $\implies$ $f^{*}$ and $tr\ f$ are positive.
    \item $f,g:A\to A$ positive $\implies$ $tr (g\,f)$ is positive.
  \end{enumerate}
\end{lemma}
\begin{proof}
  The first six items follow immediately from the definitions and how structure is preserved for
  $(\_)^{\dagger}$. For item 6, note that $g = h\, \dgr{h}$ and $tr(g\,f) = tr(\dgr{h}\,f\,h)$
  which is positive by points 1 and 5.%FIXME - why
\end{proof}

\begin{definition}\label{def:name}
  In a compact closed category, the \emph{name} of a map $f:A\to B$ is the map $\ulcorner f
  \urcorner:I \to A^{*} \* B$ defined as $\eta_{A}; (1\*f)$. This is also called the \emph{matrix} of
  $f$.
\end{definition}

In the case of a positive map $f$, $\ulcorner f \urcorner$ is referred to as a \emph{positive
matrix}.

\begin{definition}\label{def:completelypositive}
  In a dagger compact closed category, a map $f:A^{*}\*A \to B^{*}\* B$ is \emph{completely positive}
  if for all objects $C$ and all positive matrices $f: I \to C^{*} \* A^{*} \* A \* C$ the morphism
  $g ; (1\*f\*1):I \to C^{*} \* B^{*}\* B \* C$ is a positive matrix.
\end{definition}

This now allows us to define the CPM construction.

\begin{definition}\label{def:cpmconstruction}
  Given a dagger compact closed category $\cD$, define \specialcat{CPM(d)} as the category with the
  same objects as $\cD$, and a map $f:A\to B$ in \specialcat{CPM(d)} is a completely positive map
  $f:A^{*}\*A \to B^{*}\* B$ in \cD.
\end{definition}

\specialcat{CPM(d)} is also a dagger compact closed structure, inheriting its tensor from \cD.
There is a functor $F:\cD \to \specialcat{CPM(d)}$ defined as $F(A) = A$ on objects and $F(f)=
f_{*}\*f$ on maps. The image of the structure maps under $F$ are structure maps for
\specialcat{CPM(d)}. The dagger of a map $f$ is the same as its dagger in \cD.

\subsubsection{Biproduct completion}\label{sec:biproduct}
When the \specialcat{CPM} construction is applied to a biproduct dagger compact closed category, it
will not in general retain biproducts. However, it will be monoid enriched by lemma
\ref{lemma:positivemaps}. This allows us to create the biproduct completion.

The biproduct completion of a category \cD, which is enriched in commutative monoids is the
category $\cD^{\+}$ which has as objects finite sequences $\<A_{1},\ldots,A_{n}\>$ where $n\ge 0$.
The morphisms of $\cD^{\+}$ are matrices of the morphisms of \cD. Application and composition of
morphisms is via matrix multiplication. The functor $F(A) = \<A\>$, $F(f)=[f]$ is an embedding of
\cD{} in $\cD^{\+}$. If \cD{} is compact closed and the tensor is linear (i.e., interacts with the
enrichment in a linear fashion), then $\cD^{\+}$ is also compact closed.

Furthermore, if \cD{} is a dagger category and the dagger is linear, then $\cD^{\+}$ will be a
dagger category. The dagger of a map $(f_{i,j})$ in $\cD^{\+}$ is $(\dgr{(f_{j,i})})$.

This gives us the following theorem:

\begin{theorem}\label{theorem:biproductcompletion}
Given \cD, a biproduct dagger compact closed category, \cpm{d} is enriched in commutative monoids
as a dagger compact closed category. Therefore, it is possible to construct its biproduct
completion, \bcpm{d}.
\end{theorem}

Note that the canonical embedding from above, $F$, while it preserves the dagger compact closed
structure, it does \emph{not} preserve biproducts.
% section
\section{Frobenius Algebras} % (fold)
\label{sec:frobenius_algebras}
In their most general setting, Frobenius algebras are defined as a finite dimensional algebra
over a field together with a non-degenerate pairing operation. We will continue with the definitions
that make this precise.

\subsection{Frobenius algebra definitions} % (fold)
\label{sub:frobenius_algebra_definitions}


\begin{definition}\label{def:frobeniusalgebra}
  Given a symmetric monoidal category \cD, a \emph{Frobenius algebra} is an object $X$ of \cD and
  four maps, $\nabla :X\*X \to X$, $e: I \to X$, $\Delta: X\to X\* X$ and $\epsilon:X\to I$, with
  the conditions that $(X,\nabla,e)$ forms a commutative monoid, $(X,\Delta, \epsilon)$ forms a
  commutative comonoid and the diagram
  \[
    \xymatrix{
      X\*X \ar[rr]^{X\*\Delta} \ar[dd]_{\Delta \* X} \ar[dr]^{\nabla}
        && X\*X\*X \ar[dd]^{\nabla\*X}\\
      & X\ar[dr]^{\Delta}\\
      X\*X\*X\ar[rr]_{X\*\nabla}  && X\*X
    }
  \]
  commutes. The Frobenius algebra is \emph{special} when $\Delta \nabla = 1_{X}$ and
  \emph{commutative} when $\Delta c_{X,X} = \Delta$.
\end{definition}
\begin{definition}\label{def:daggerfrob}
  A Frobenius algebra in a dagger symmetric monoidal category where $\Delta = \dgr{\nabla}$ and
  $\epsilon=\dgr{u}$ is a $\dagger$\emph{-Frobenius algebra}.
\end{definition}
For an example of a $\dagger$-Frobenius algebra, consider a finite dimensional Hilbert space $H$
with an orthonormal basis $\{\ket{\phi_{i}}\}$ and define $\Delta:H\to H\*H: \ket{\phi_{i}}\mapsto
\ket{\phi_{i}} \* \ket{\phi_{i}}$ and $\epsilon : H\to \complex : \ket{\phi_{i}} \mapsto 1$. Then $(H,
\nabla=\dgr{\Delta}, u=\dgr{\epsilon}, \Delta, \epsilon)$ forms a commutative special
$\dagger$-Frobenius algebra.

% subsection frobenius_algebra_definitions (end)

\subsection{Bases and Frobenius Algebras} % (fold)
\label{sub:bases_and_frobenius_algebras}
In \cite{coeckeetal08:ortho}, Coecke et. al. provide an algebraic description of orthogonal bases in
finite dimensional Hilbert spaces. Additionally,  an orthonormal basis for such a space is
a special commutative $\dagger$-Frobenius algebra. To show the other direction, given a commutative
$\dagger$-Frobenius algebra, $(H,\nabla,u)$ and for each element $\alpha\in H$, define the right
action of $\alpha$ as $R_{\alpha}:=(id\*\alpha)\, \nabla:H\to H$. Note the use of the fact that
elements $\alpha\in H$ can be considered as linear maps $\alpha:\complex \to H:1\mapsto \ket{\alpha}$.
The dagger of a right action is also a right action, $\dgr{R_{\alpha}} = R_{\alpha'}$ where
$\alpha'= u\, \nabla\, (id\* \dgr{\alpha})$, which is a consequence of the Frobenius identities.

The $(\_)'$ construction is actually an involution:
\begin{eqnarray*}
  &(\alpha')' &= u \nabla (id \* \dgr{\alpha'}) \\
  && = u \nabla (id \* \dgr{(u \nabla (id \* \dgr{\alpha}))}\\
  && = u \nabla (id \* ( (id \* \alpha) \Delta \epsilon))\\
  && = (u \* \alpha) (\nabla \* id) (id \* \Delta) (id \*  \epsilon)\\
  && = (u \* \alpha) (id \* \Delta) (\nabla \* id) (id \*  \epsilon)\\
  && = (u \* \alpha)  (id \*  \epsilon)\\
  && = \alpha
\end{eqnarray*}

\begin{lemma}\label{lemma:cstaralgebra}
  Any $\dagger$-Frobenius algebra in \fdh is a $C^{*}$-algebra.
\end{lemma}
\begin{proof}
  The endomorphism monoid of \fdh(H,H) is a $C^{*}$-algebra. From the proceeding, we have
  \[
    H \cong \fdh(\complex,H) \cong R_{[\fdh(\complex,H)]}\subseteq\fdh(H,H).
  \]
  This inherits the algebra structure from \fdh(H,H). Furthermore, since any finite dimensional
  involution-closed sub-algebra of a $C^{*}$-algebra is also a $C^{*}$-algebra, this shows the
  $\dagger$-Frobenius algebra is a $C^{*}$-algebra.
\end{proof}

Using the fact that the involution preserving homomorphisms from a finite dimensional commutative
$C^{*}$-algebra to $\complex$ form a basis for the dual of the underlying vector space, write these
homomorphisms as $\dgr{\phi_{i}}:H \to \complex$. Then their adjoints, $\phi_{i}:\complex\to H$ will form a
basis for the space $H$. These are the copyable elements in $H$.

This, together with continued applications of the Frobenius rules and linear algebra allow the
authors to prove the following Theorem.
\begin{theorem}
  Every commutative $\dagger$-Frobenius algebra in \fdh determines an orthogonal basis consisting
  of its copyable elements. Conversely, every orthogonal basis $\{\ket{\phi_{i}}\}_{i}$ determines
  a commutative $\dagger$-Frobenius algebra via \[\Delta:H\to H\*H: \ket{\phi_{i}}\mapsto
  \ket{\phi_{i}} \* \ket{\phi_{i}}\qquad\epsilon : H\to \complex : \ket{\phi_{i}} \mapsto 1\] and these
  constructions are inverse to each other.
\end{theorem}

% subsection bases_and_frobenius_algebras (end)

\subsection{Quantum and classical data}\label{sec:quantumclassical}
In \cite{coecke08structures}, Coecke et.al. build on the results of \cite{coeckeetal08:ortho}
to start from a $\dagger$-symmetric monoidal category and construct the minimal machinery needed to
model quantum and classical computations. For the rest of this section, $\cD$ will be assumed to be
such a category, with $\*$ the monoid tensor and $I$ the unit of the monoid.

\begin{definition}\label{def:compact_structure}
  A compact structure on an object $A$ in the category $\cD$ is given by the object $A$, an object
  $A^{*}$ called its \emph{dual} and the maps $\eta:I \to A^{*}\* A$, $\epsilon: A\* A^{*} \to I$
  such that the diagrams
  \[
    \xymatrix@C+20pt{
      A^{*} \ar[dr]^{id} \ar[d]_{\eta\*A^{*}} \\
      A^{*} \*A\*A^{*}  \ar[r]_(.6){A^{*} \*\epsilon} & A^{*}
    }
    \text{ and }
    \xymatrix@C+20pt{
      A \ar[r]^(.4){A\*\eta} \ar[dr]_{id} & A\* A^{*}\* A \ar[d]^{\epsilon\*A}\\
      & A
    }
  \]
  commute.
\end{definition}

\begin{definition}\label{def:quantumstructure}
  A \emph{quantum structure} is an object $A$ and map $\eta:I\to A\*A$ such that
  $(A,A,\eta,\dgr{\eta})$ form a compact structure.
\end{definition}
Note that $A$ is self-dual in definition \ref{def:quantumstructure}.

This allows the creation of the category $\cD_{q}$ which has as objects quantum structures and maps
are the maps in $\cD$ between the objects in the quantum structures.

In the category $\cD_{q}$, it is now possible to define the upper and lower $*$ operations on maps,
such that $(f_{*})^{*}= (f^{*})_{*} = \dgr{f}$:
\begin{eqnarray*}
&f^{*} &:= (\eta_{A}\*1) (1 \* f\*1) (1\*\dgr{\eta}_{B}),\\
&f_{*} &:= (\eta_{B}\*1) (1 \* \dgr{f}\*1) (1\*\dgr{\eta}_{A}).
\end{eqnarray*}

Next, define a classical structure on \cD.
\begin{definition}\label{def:classicalstructure}
  A \emph{classical structure} in \cD{} is an object $X$ together with two maps, $\Delta :X \to X\* X$,
  $\epsilon:X\to I$ such that $(X,\dgr{\Delta},\dgr{\epsilon},\Delta,\epsilon)$ forms a special
  Frobenius algebra.
\end{definition}

As above, this allows us to define $\cD_{c}$, the category whose objects are the classical
structures of $\cD$. The maps in $\cD_{c}$ are given by the maps in $\cD$ between the
objects of the classical structure.

Note that a classical structure will induce a quantum structure, setting $\eta_{X}$ to be
$\dgr{\epsilon_{X}}\, \Delta_{X}$.


Later on, in \ref{sec:the_category_of_commutative_frobenius_algebras}, we will show that commutative
special Frobenius algebras possess a specialized inverse category structure.
% subsection quantum_and_classical_data (end)


% section frobenius_algebras (end)

%%% Local Variables:
%%% mode: latex
%%% TeX-master: "../../phd-thesis"
%%% End:


%%% Local Variables:
%%% mode: latex
%%% TeX-master: "../phd-thesis"
%%% End:
