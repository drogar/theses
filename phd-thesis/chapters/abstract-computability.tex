%!TEX root = /Users/gilesb/UofC/thesis/phd-thesis/phd-thesis.tex

\chapter{Abstract Computability}\label{chap:abstract_computability}
\section{Categories} % (fold)
\label{sec:categories}

A category as a mathematical object can be defined in a variety of equivalent ways. As much of our
work will involve the exploration of partial and reversible maps, their domains and ranges, we
choose a definition that highlights the algebraic nature of these. Note that ranges are normally
referred to as co-domains in category theory and we will use the co-domain terminology in this
section.

\begin{definition}\label{def:category}
  A \emph{category} $\A$ is a collection of maps together with two functions, $D$ and $C$, from
  $\A$ to $\A$ and a partial associative composition of maps (written by juxtaposing maps), such
  that:
  \begin{itemize}
    \item[\catone] $D(f) f$ is defined and equals $f$,
    \item[\cattwo] $f C(f)$ is defined and equals $f$,
    \item[\catthree] $f g$ is defined iff $C(f) = D(g)$ and $D(f g) = D(f)$ and $C(f g) = C(g)$,
    \item[\catfour] $(f g) h = f (g h)$ whenever either side is defined,
    \item[\catfive] $D(C(x)) = C(x)$, $C(D(x)) = D(x)$ and $C,D$ are both idempotent.
  \end{itemize}
\end{definition}

Another definition, often used in introducing categories, is given next.

\begin{definition}\label{def:category_alt}
  A \emph{category} $\A$ is a directed graph consisting of objects $A_o$ and maps $A_m$. Each $f\in
  A_m$ has two associated objects in $A_o$, called the domain and co-domain. When $f$ has domain $X$
  and co-domain $Y$ we will write $f:X \to Y$. For $f, g \in A_m$, if $f:X\to Y$ and $g:Y \to Z$,
  there is a map called the \emph{composite} of $f$ and $g$, written $fg$ such that $fg:X \to Z$.
  For any $W \in A_o$ there is an \emph{identity} map $1_W:W \to W$. Additionally, these two axioms
  must hold:
  \begin{itemize}
    \item[\axiom{C'}{1}] for $f:X \to Y$, $1_X f = f = f 1_Y$,
    \item[\axiom{C'}{2}] given $f:X \to Y,\ g:Y \to Z$ and $h: Z\to W$, then $f (g h) = (f g) h$.
  \end{itemize}

\end{definition}

\begin{lemma}\label{lem:category_is_category_alt}
  A category as defined in Definition~\ref{def:category} is equivalent to a category as defined
  in Definition~\ref{def:category_alt} and vice versa.
\end{lemma}
\begin{proof}
  Assume $\A$ is as in Definition~\ref{def:category}. Then:
  \begin{itemize}
    \item Set $A_o$ to the collection of all  $D(f)$ and $C(f)$;
    \item Set $A_m$ to all the maps in $\A$.
  \end{itemize}
  The domain of any map $f \in A_m$ is $D(f)$
  and the co-domain is $C(f)$. By \catthree, for $f:X\to Y$ and $g:Y \to Z$ the composite $fg$ is
  defined. The identity map of the object $D(f)$ is the map $D(f)$ and the identity map of the
  object $C(f)$ is $C(f)$. By \catfive, we see \axiom{C'}{1} is satisfied. By \catfour, we see
  \axiom{C'}{2} is satisfied. Therefore, $\A$ satisfies Definition~\ref{def:category_alt}.

  Conversely, assume $\Z$ is as in Definition~\ref{def:category_alt}, with the
  collection of maps, $Z_m$. For each $f : A \to B \in Z_m$, set $D(f) = 1_A$ and $C(f) = 1_B$. By
  the definition of the identity maps and \axiom{C'}{1}, we see \catone, \cattwo and \catfive are
  all satisfied. From the composition requirements on $\Z$ and \axiom{C'}{2}, it follows that
  \catfour is satisfied. For \catthree, assume $f g$ is defined. Then for some $A,B,C \in Z_o$,
  $f:A \to B$ and $g:B \to C$. This gives us $1_B = C(f) = D(g)$, $1_A = D(f g) = D_f$ and $1_B =
  C(f g) = C(g)$. Next, assume we have $C(f) = D(g)$, $D(f g) = D(f)$ and $C(f g) = C(g)$. This
  tells us the co-domain of $f$ is some object $B$ which is also the domain of $g$, hence we may
  form the composition $f g$ which will have domain $A$, the domain of $f$ and co-domain $C$, the
  co-domain of $g.$
\end{proof}

We have shown the two definitions are equivalent. It will be convenient to reference either
definition and manner of referring to a category throughout this thesis. We will use
whichever definition seems the most appropriate to use at any point.

We may also consider the notion of containment between categories.

\begin{definition}\label{def:subcategories}
  Given the categories $\C$ and $\D$, we may say the following:
  \begin{enumerate}[{(}i{)}]
    \item \C is a \emph{sub-category} of \D when each object of \C is an object of \D and when each map
    of \C is a map of \D.
    \item \C is a \emph{full sub-category} of \D when it is a sub-category and given $A, B$ objects
      in $\C$ and $f:A \to B$ in \D, then $f$ is a map in $\C$.
  \end{enumerate}
\end{definition}

\subsection{Enrichment of categories} % (fold)
\label{sub:enrichement_of_categories}
\begin{definition}\label{def:hom-collection}
  If $\X$ is a category, then $\X(A,B)$ is called a \emph{hom-collection} of $\X$ and consists
  of all arrows $f$ with $D(f) = A$ and $C(f) = B$.
\end{definition}

In the case where the hom-objects of a category $\X$ are all sets, we call them hom-sets. Additionally,
we say $\X$ is \emph{enriched} in \sets. We may extend this to any mathematical structure, e.g.,
enriched in partial orders, enriched in groups, etc..


Specific types of enrichment may force a specific structure on a category. For example, if $\X$ is
enriched in sets of cardinality of 0 or 1, then $\X$ must be a pre-order.


% subsection enrichment_of_categories (end)
\subsection{Examples of categories} % (fold)
\label{sub:examples_of_categories}
In this section, we will offer a few examples of categories. As Definition~\ref{def:category_alt}
tends to be a more succinct way to present the data of a category, this section will give the
examples in terms of objects and maps rather than the ``object-free'' definition.


\subsubsection{Categories based on \protect{\sets}} % (fold)
\label{ssub:categories_based_on_sets}
There are three primary categories of interest to us where the objects are the collection of sets.
The first is \sets, where the maps are given by all set functions. The second is \Par, where the
maps are all partial maps. In each case, the standard definition of functions suffices to ensure
identities, compositions and associativity are all satisfied. Domain and co-domain are given by the
domain and range respectively.

A third example, often of interest in quantum programming language semantics is \rel:
\category{Sets}{Relations: $R:X \to Y$}{$1_X = \{(x,x) | x \in X\}$}{$RS = \{(x,z) |
\exists y, (x,y) \in R$ and $(y,z)\in S\}$}

Note that \rel is enriched in posets, via set inclusion. \Par can be viewed as a sub-category of
\rel, with the same objects, but only allowing maps which are functions, i.e., if $(x,y), (x,y')
\in R$, then $y = y'$. \Par is also enriched in posets, via the same inclusion ordering as in \rel.

% sub-subsection categories_based_on_sets (end)
\subsubsection{Matrix categories} % (fold)
\label{ssub:matrix_categories}
Given a rig $R$ (i.e., a ring minus negatives, e.g., the positive rationals), one may form the
category \specialcat{Mat}($R$).
\category{\nat}{$[r_{ij}]: n \to m$ where $[r_{ij}]$ is an $n \times m$ matrix over $R$}{
$I_n$}{Matrix multiplication}
% sub-subsection matrix_categories (end)

\subsubsection{Dual categories} % (fold)
\label{ssub:dual_categories}

Given a category $\C$, we may form the \emph{dual} of $C$, written $C^{op}$ as the following
category:
\category{The objects of $\C$}{$f^{op}:B\to A$ in $\C^{op}$ when $f:A\to B$ in $\C$.}{
The identity maps of $\C$}{If $f g = h$ in $\C$, $g^{op} f^{op} = h^{op}$}

% sub-subsection dual_categories (end)
% subsection examples_of_categories (end)
\subsection{Properties of maps} % (fold)
\label{sub:properties_of_maps}
Many interesting properties of maps are generalizations of notions that have been found useful in
considering sets and functions. We present a few of these in a tabular format, together with their
categorical definition. Throughout the table, $e,f,g$ are maps in a category $C$ with
$e:A \to A$ and $f,g:A \to B$.
\\[14pt]

\begin{tabular}{|p{1in}p{1in}p{3.73in}|}
\hline
{\bf Sets} & {\bf Categorical Property} & {\bf Definition}\\
\hline
\hline
Injective & Monic & $f$ is monic whenever $h f = k f$ means that $h = k$.\\
\hline
Surjective & Epic & The dual notion to monic, $g$ is epic whenever $g h = g k$ means that $h = k$.
A map that is both monic and epic is called \emph{bijic}.\\
\hline
Left~Inverse & Section & $f$ is a section when there is a map $f^*$ such that $f f^* = 1_A$. $f$
is also referred to as the \emph{left inverse} of $f^*$.\\
\hline
Right Inverse & Retraction & $f$ is a retraction when there is a map $f_*$ such that $f_* f = 1_B$.
$f$ is also referred to as the \emph{right inverse} of $f_*$. A map that is both a section and a
retraction is called an \emph{isomorphism}.\\
\hline
Idempotent & Idempotent & An endomap $e$ is idempotent whenever $e e = e$.\\
\hline
\end{tabular}\\[14pt]

We state without proof a number of properties about maps.

\begin{lemma}\label{lem:categorical_properties_of_maps}
  In a category \C,
  \begin{enumerate}[{(}i{)}]
    \item If $f,g$ are monic, then $f g$ is monic.
    \item If $f g$ is monic, then $f$ is monic.
    \item $f$ being a section means it is monic.
    \item $f, g$ sections implies that $f g$ is a section.
    \item $f g$ a section means $f$ is a section.
  \end{enumerate}
\end{lemma}


\begin{lemma}\label{lem:categorical_inverses_are_unique}
  If $f:A \to B$ is both a section and a retraction, then $f^* = f_*$.
\end{lemma}

\begin{lemma}\label{lem:categorical_iso_is_epic_section}
  $f$ is an isomorphism if and only if it is an epic section.
\end{lemma}

Note there are corresponding properties for epics and retractions, obtained by dualizing the
statements of Lemma~\ref{lem:categorical_properties_of_maps} and
Lemma~\ref{lem:categorical_iso_is_epic_section}.

Suppose $f:A \to B$ is a retraction with left inverse $f_*:B \to A$. Note that $f f_*$ is idempotent
as $f f_* f f_* = f 1_B f_* = f f_*$. If we are given an idempotent $e$, we say $e$ is \emph{split}
if there is a retraction $f$ with $e = f f_*$.

In general, not all idempotents in a category will split. The following construction allows us to
create a category based on the original one in which all idempotents do split.

\begin{definition}\label{def:split_category}
  Given a category $\C$ we define \emph{Split($\C$)} as the following category:
  \category{$(A,e)$, where $A$ is an object of \C, $e:A\to A$ and $e\in E$.}
    {$f_{d,e}:(A,d)\to(B,e)$ is given by $f:A\to B$ in \C, where $f = d\uts f e$.}
    {The map $e_{e,e}$ for $(A,e)$.}
    {Inherited from \C.}
\end{definition}

\begin{lemma}\label{lem:split_category_splits_and_has_category}
  Given a category $\C$, then it is a full sub-category of Split($\C$) and all idempotents split
  in Split($\C$).
\end{lemma}
\begin{proof}
  We identify each object $A$ in $\C$ with the object $(A,1)$ in Split($\C$). The only maps between
  $(A,1)$ and $(B,1)$ in Split($\C$) are the maps between $A$ and $B$ in $\C$, hence we have a
  full sub-category.

  Suppose we have the map $d_{e,e}: (A,e) \to (A,e)$ with $d d = d$, i.e., it is idempotent in $C$
  and Split($\C$). In Split($\C$), we have the map $d_{e,d}:(A,e) \to (A,d)$ and $d_{d,e}:(A,d) \to
  (A,e)$ where $d_{d,e} d_{e,d} = d_{d,d} = 1_{(A,d)}$ and $d_{e,d} d_{d,e} = d_{e,e}$, hence
  it is a splitting of the map  $d_{e,e}$.
\end{proof}
% subsection properties_of_maps (end)

\subsection{Limits and colimits in categories} % (fold)
\label{sub:limits_and_colimits_in_categories}

We shall discuss only a few basic limits/colimits in categories. First we discuss initial
and terminal objects.

\begin{definition}\label{def:initial_object}
  An \emph{initial object} in a category $\C$ is an object which has exactly one map to each other
  object in the category. The dual notion is \emph{terminal object} which has exactly one map from
  each other object in the category.
\end{definition}


\begin{lemma}\label{lem:initial_objects_are_unique}
  Suppose $I,J$ are initial objects in $\C$. Then there is a unique isomorphism $i:I \to J$.
\end{lemma}
\begin{proof}
  First, note that by definition there is only one map from $I$ to $I$ --- which must be the
  identity map. As $I$ is initial there is a map $i: I \to J$. As $J$ is initial there is a map
  $j:J \to I$. But this means $ij : I \to I = 1$ and $j i : J \to J = 1$ and hence $i$ is the
  unique isomorphism from $I$ to $J$.
\end{proof}

Dually, we have the corresponding result of Lemma~\ref{lem:initial_objects_are_unique} for terminal
objects --- they are also unique up to a unique isomorphism.

In categories, we normally designate the initial object by $0$ and the terminal object by $1$.

We now turn to products and co-products.

\begin{definition}\label{def:categorical_product}
  Let $A,B$ be objects of the category $\C$. Then the object $A \times B$ is a \emph{product} of
  $A$ and $B$ when:
  \begin{itemize}
    \item There exist maps $\pi_0, \pi_1$ with $\pi_0:A\times B \to A$, $\pi_1:A\times B \to B$;
    \item Given an object $C$ with maps $f:C\to A$ and $g:C \to B$ there exists a unique map
    $\<f,g\>$ such that the following diagram commutes:
    \[
      \xymatrix@C+15pt@R+25pt{
        &&&A\\
        C\ar[urrr]^{f} \ar[drrr]_{g}\ar@{.>}[rr]^{\<f,g\>} & &A\times B \ar[ur]_{\pi_0}\ar[dr]^{\pi_1}\\
        &&&B
      }
    \]
  \end{itemize}

\end{definition}
% subsection limits_and_colimits_in_categories (end)

\subsection{Functors and natural transformations} % (fold)
\label{ssub:functors_and_natural_transformations}

\begin{definition}\label{def:functor}
  A map $F:\X \to \Y$ between categories (as in Definition~\ref{def:category} is called a
  \emph{functor}, provided it satisfies the following:
  \begin{itemize}
    \item[\axiom{F}{1}] $F(D(f)) = D(F(f))$ and $F(C(f)) = C(F(f))$;
    \item[\axiom{F}{2}] $F(f g) = F(f)F(g)$;
  \end{itemize}
\end{definition}

\begin{lemma}\label{lem:cat_is_a_category}
  The collection of categories and functors form the category \cat.
\end{lemma}
\begin{proof}
  \category{Categories.}{Functors.}{The identity functor which takes a map to the same map.}{
  $FG(x) = F(G(x))$ which is clearly associative.}
\end{proof}

We will often restrict ourselves to specific classes of functors which either \emph{preserve} or
\emph{reflect} certain characteristics of the domain category or co-domain category. To be more
precise, we provide some definitions.

\begin{definition}\label{def:diagram_in_a_category}
  A \emph{diagram} in a category is a collection of objects and maps between those objects
  which satisfy categorical composition rules. More precisely: Given a category $\S$, a diagram
  in a category $\C$ of \emph{shape} $\S$ is a functor $D:\S \to \C$.
\end{definition}

In practice, diagrams are pictorially represented by drawing the objects and the maps between them.

\begin{definition}\label{def:property_of_a_diagram}
  A \emph{property} of a diagram $D$, written $P(D)$ is a logical relation expressed using the
  objects and maps of the diagram $D$.
\end{definition}

\begin{example}
  $P(f:A \to B) = \exists h : B \to A. h f = 1_A$ expresses that $f$ is a retraction.
\end{example}

\begin{definition}\label{def:functor_preserving_a_property}
  A functor $F$ \emph{preserves} the property $P$ over maps $f_i$ and objects $A_j$ when
  $P(f_1,\ldots,f_n, A_1,\ldots,A_m)$ implies $P(F(f_1),\ldots,F(f_n), F(A_1),\ldots,F(A_m))$.
\end{definition}

\begin{definition}\label{def:functor_reflects_a_property}
  A functor $F$ \emph{reflects} the property $P$ over maps $f_i$ and objects $A_j$ when
  $P(F(f_1),\ldots,F(f_n), F(A_1),\ldots,F(A_m))$ implies $P(f_1,\ldots,f_n, A_1,\ldots,A_m)$.
\end{definition}

For example, all functors preserve the properties of being an idempotent or a retraction or section,
but in general, not the property of being monic.

A functor $F:\C \to \D$ induces a map between hom-objects in $\C$ and hom-objects in $\D$. For
each object $A,B$ in $\C$ we have the map:
\[
  F_{AB} : \C(A,B) \to \D(F(A),F(B)).
\]

\begin{definition}\label{def:full_functor_faithful_functor}
  Given a functor $F:\C to \D$, we say:
  \begin{itemize}
    \item $F$ is \emph{faithful} when for all $A,B$, $F_{AB}$ is an injective function;
    \item $F$ is \emph{full} when  for all $A,B$, $F_{AB}$ is an surjective function.
  \end{itemize}
\end{definition}



\begin{definition}\label{def:natural_transformation}
  Given functors $F,G:\X \to \Y$, a \emph{natural transformation} $\alpha:F \natto G$ is a collection
  of maps in $\Y$, $\alpha_X : F(X) \to G(X)$, indexed by the objects of $\X$ such that for all
  $f:X_1 \to X_2$ in $\X$ the following diagram in $\Y$ commutes:
  \[\xymatrix @R+10pt @C+10pt{
      F(X_1) \ar[r]^{F(f)} \ar[d]_{\alpha_{X_1}} & F(X_2) \ar[d]^{\alpha_{X_2}}\\
      G(X_1) \ar[r]_{G(f)} &  G(X_2)
    }
  \]
\end{definition}
% subsection functors_and_natural_transformations (end)

\subsection{Categories with additional structure} % (fold)
\label{sub:categories_with_additional_structure}

\begin{definition}[Symmetric Monoidal Category]\label{symmetricmonoidalcat}
  A \emph{symmetric monoidal category} \cD{} is a category equipped with a monoid $\+$ (a bi-functor
  $\+:\cD \times \cD \to \cD$) together with three families of natural isomorphisms:
  $a_{A,B,C}:A\*(B\*C) \to (A\*B)\*C$, $u_{A}:A\to A\+I$ and $c_{A,B}:A\+B \to B\+ A$, which satisfy
  specific coherence diagrams. The isomorphisms are referred to as the \emph{structure isomorphisms}
  for the symmetric monoidal category. $I$ is the unit of the monoid.
\end{definition}
For details on the coherence diagrams, please see e.g., \cite{barr:ctcs} or
\cite{maclan97:categorieswrkmath}. The essence of the coherence diagrams is that any diagram
composed solely of the structure isomorphisms will commute.

\begin{definition}[Compact Closed Category]\label{def:compactclosedcat}
A \emph{compact closed category} \cD{} is a symmetric monoidal category with monoid $\*$ where each
object $A$ has a dual $A^{*}$ and there exist families of maps $\eta_{A}: I \to A^{*} \* A$ (the
\emph{unit}) and $\epsilon_{A}: A\*A^{*}\to I$ (the \emph{counit}) such that
\[
  \xymatrix@C+20pt{
    A \ar[r]^{u_{A}} \ar@{=}[d]  & A\*I \ar[r]^{1\*\eta_{A}}
        & A\* (A^{*}\*A) \ar[d]^{a_{A,A^{*},A}} \\
    A & I\* A \ar[l]^{u_{A}^{-1}} & (A\* A^{*})\*A \ar[l]^{\*\epsilon_{B}\*1}
    }
  \]
commutes and so does the similar one based on $A^{*}$.
\end{definition}

Given a map $f:A\to B$ in a compact closed category,  define the map $f^{*}:B^{*} \to A^{*}$ as
\[
  \xymatrix@C+10pt{
    B^{*}\ar[r]^{u_{B^{*}}} \ar[d]_{f^{*}}& I\*B^{*} \ar[r]^{\eta_{A}\*1}
      & A^{*}\*A\*B^{*}\ar[d]^{1\*f\*1}\\
    A^{*}&    A^{*}\*I\ar[l]^{u_{A^{*}}^{-1}}  &   A^{*}\*B\*B^{*}\ar[l]^{1\*\epsilon_{B}}
  }
\]


% subsection categories_with_additional_structure (end)

% section categories (end)
\section{Restriction categories} % (fold)
\label{sec:restriction_categories}


Restriction categories were introduced in
 \cite{cockett2002:restcategories1} as a convenient axiomatization of partial maps.
\begin{definition}\label{def:restriction_category}
  A \emph{restriction category} is a category \X\ together with a \emph{restriction operator} on
  maps:
  \[
    \infer{\restr{f}:A\to A}{f:A \to B}
  \]
  where $f$ is a map of \X\ and $A,B$ are objects of \X, such that the
  following four \emph{restriction identities} hold, whenever the
  compositions\footnote{Note that composition is
  written in diagrammatic order throughout this paper.} are defined.
  \begin{align*}
    &\rone\ \restr{f} f = f & &
    \rtwo\ \restr{g}  \restr{f} = \restr{f}  \restr{g}\\
    &\rthree\ \restr{\restr{f}  g} = \restr{f}   \restr{g} & &
    \rfour\  f \restr{g} = \restr{f g} f
  \end{align*}
\end{definition}

\begin{definition}
  A \emph{restriction functor} is a functor which preserves the restriction. That is,
  given a functor $F: \X \to \Y$ with \X\  and \Y\ restriction categories,
  $F$ is a restriction functor if:
  \[
    F(\restr{f}) = \restr{F(f)}.
  \]
\end{definition}

Any map such that $r=\restr{r}$ is an idempotent, as $\rst{r}\rst{r} = \rst{\rst{r} r} = \rst{r}$,
and is called a \emph{restriction idempotent}. All maps $\restr{f}$ are restriction idempotents as
$\rst{f}=\rst{\rst{f}}$.

Here are some basic facts for restriction categories as originally
shown in \cite{cockett2002:restcategories1} pp 4-5:
\begin{lemma}\label{lem:restrictionvarious}
  In a restriction category \X,
  \begin{multicols}{2}
    \begin{enumerate}[{(}i{)}]
      \item{}$\rst{f}$ is idempotent;
      \item{} $\rst{f g} = \rst{f g} \, \rst{f}$;\label{lemitem:rv_2}
      \item{} $\rst{f g} = \rst{f \rst{g}}$ ;\label{lemitem:rv_3}
      \item{} $\rst{\rst{f}} = \rst{f}$;
      \item{} $\rst{f}\,\rst{g} = \rst{\rst{f}\,\rst{g}}$;
      \item{} $f$ monic implies $\rst{f} = 1$;
      \item{} $f = \rst{g} f \implies \rst{g}\,\rst{f} = \rst{f}$.
      \\
    \end{enumerate}
  \end{multicols}
\end{lemma}

\begin{definition}\label{def:total_map}
  A map $f:A\to B$ in a restriction category is said to be \emph{total} when
  $\rst{f} = 1_A$. The total maps in a restriction category form a sub-category
  $Total(\X) \subseteq \X$.
\end{definition}


An example of a restriction category is \Par, the category defined via:
\rcategory{Sets.}{Partial set functions.}{The identity function.}{%
Normal set function composition.}{The restriction of $f:A\to B$ is:
\[
  \rst{f}(x) =
  \begin{cases}
    x&\text{if $f(x)$ is defined,}\\
    \uparrow&\text{if }f(x)\text{ is }\uparrow.
  \end{cases}
\]
}
In \Par, the
total maps correspond precisely to the functions that are defined on all elements of the domain.


\subsection{Enrichment and meets} % (fold)
\label{sub:enrichment_and_meets}

\begin{definition}\label{def:restriction_category_hom_set_ordering}
  In any restriction category, for any two maps  $f,g:A\to B$, define $f \le g$ iff
  $\restr{f} g = f$.

\end{definition}


\begin{lemma}\label{lem:restriction_cats_are_partial_order_enriched}
  In a restriction category \X:
  \begin{multicols}{2}
    \begin{enumerate}[{(}i{)}]
      \item  $\le$ from Definition~\ref{def:restriction_category_hom_set_ordering}
        is a partial order on each hom-set;
      \item $f \le g \implies \restr{f} \le \restr{g}$;\label{lemitem:rst_ordering_2}
      \item $\rst{f g} \le \rst{f}$; \label{lemitem:rst_ordering_3}
      \item $f \le g \implies h f \le h g$;
      \item $f \le g \implies f h \le g h$;
      \item $f \le g$ and $\rst{f} = \rst{g}$ implies $f = g$;
      \item $f \le 1 \iff f = \restr{f}$;
      \item $\rst{g}f = f$ implies $\rst{f} \le \rst{g}$.
    \end{enumerate}
  \end{multicols}
\end{lemma}
\begin{proof}
  \prepprooflist
  \setlist[enumerate,1]{leftmargin=1.2cm}
  \begin{enumerate}[{(}i{)}]
    \item With $f,g,h$ parallel maps in \X, each of the requirements for a partial order is
    verified below:
    \begin{description}
      \itembf{Reflexivity:} $\restr{f} f = f$ and therefore, $ f \le f$.
      \itembf{Anti-Symmetry:} Given $\restr{f}g = f$ and $\restr{g}f = g$, it follows:
        \[
          f = \restr{f} f = \restr{\restr{f} g} f = \restr{f}\, \restr{g} f
          = \restr{g}\restr{f} f =  \restr{g} f = g.
        \]
      \itembf{Transitivity:} Given $f \le g$ and $g\le h$,
        \[
          \restr{f} h = \restr{\restr{f} g} h = \restr{f}\, \restr{g} h = \restr{f} g = f
        \]
        showing that $f \le h$.
    \end{description}
    \item The premise is that $\restr{f} g = f$. From this, $ \restr{f}\, \restr{g} =
      \restr{\restr{f} g} = \restr{f}$, showing $\restr{f} \le \restr{g}$.
    \item $\restr{h f} hg = h \restr{f} g = h f$  and therefore $h f \le hg$.
    \item $\restr{f} g = f$, this shows $\restr{f h} g h = \restr{\restr{f} g h} g h
      = \restr{f}\, \restr{g h} g h = \restr{f} g h = f h$ and therefore $f h \le g h$.
    \item $g = \rst{g} g = \rst{f} g = f$.
    \item As $f \le 1$ means precisely $\restr{f}1 = f$.
    \item Assuming $\rst{g} f = f$, we need to show $\rst{f}\, \rst{g} = \rst{f}$.
      \begin{align*}
        \rst{f}\,\rst{g} &= \rst{g}\rst{f} & \rtwo \\
        & = \rst{\rst{g} f} & \rthree \\
        & = \rst{f} & \text{Assumption}.
      \end{align*}
      Hence, $\rst{f} \le \rst{g}$.
  \end{enumerate}
\end{proof}

Lemma \ref{lem:restriction_cats_are_partial_order_enriched} shows that restriction
categories are enriched in partial orders.

In a restriction category \X, we will use the notation $\open{A}$ for the restriction idempotents
of $A\in \ob{\X}$. $\open{A} = \{x:A\to A| x = \rst{x}\}$. The notation $\open{A}$ was chosen
to be suggestive of open sets.

\begin{lemma}\label{lem:open_a_is_a_meet_semilattice}
  In a restriction category \X, $\open{A}$ is a meet semi-lattice.
\end{lemma}
\begin{proof}
  The top of the meet semi-lattice is $1_A$, under the ordering from
  Definition~\ref{def:restriction_category_hom_set_ordering}.
  The join of any two idempotents is given by their composition.
\end{proof}

\begin{definition}
  A restriction category has \emph{meets} if there is an operation $\cap$ on parallel maps:
  \[
    \infer{A\xrightarrow{f\cap g} B}
      {A\overset{f}{\underset{g}{\rightrightarrows}}B}
  \]
  such that $f\cap g \le f, f\cap g \le g, f\cap f = f, h (f\cap g) = h f \cap hg$.
\end{definition}

Meets were introduced in \cite{cockett-guo-hofstra-2012:range2}.
The following are basic results on meets:

\begin{lemma}
  \label{lem:properties_of_meets_in_restriction_categories}
  In a restriction category \X with meets, where $f, g, h$ are maps in
  \X, the following are true:
  \setlist[enumerate,1]{leftmargin=1.2cm}
  \begin{enumerate}[{(}i{)}]
    \item $f\le g \text{ and } f \le h \iff f \le g\cap h$;
        \label{lemsub:properties_of_meets_one}
    \item $f\cap g = g \cap f$;\label{lemsub:properties_of_meets_two}
    \item $\restr{f\cap 1} = f \cap 1$;\label{lemsub:properties_of_meets_three}
    \item $(f \cap g) \cap h = f \cap (g \cap h)$;
    \item $r(f\cap g) = r f \cap g$ where $r=\rst{r}$ is a restriction idempotent;
    \item $(f\cap g)r = f r \cap g$ where $r=\rst{r}$ is a restriction idempotent;
    \item $\restr{f\cap g} \le \restr{f}$ (and therefore $\restr{f\cap g} \le \restr{g}$);
    \item $ (f \cap 1) f = f \cap 1$;
    \item $ e(e \cap 1) = e$ where $e$ is idempotent.
  %\item $ e \cap e' = e e'$
  \end{enumerate}
\end{lemma}
\begin{proof}
  \prepprooflist
  \setlist[enumerate,1]{leftmargin=1.2cm}
  \begin{enumerate}[{(}i{)}]
    \item $f\le g \text{ and } f \le h$ means precisely $f = \restr{f} g$ and $f = \restr{f} h$.
      Therefore,
      \[
        \restr{f} (g\cap h) =  \restr{f} g \cap \restr{f} h =  f\cap f = f
      \]
      and so $f \le g \cap h$. Conversely, given $f \le g\cap h$, we have
      $f = \restr{f} (g\cap h) = \restr{f} g \cap \restr{f} h \le \restr{f} g $. But
      $f \le \restr{f} g$ means $f = \restr{f}\,\restr{f} g = \restr{f}g$ and therefore
      $f \le g$. Similarly, $f \le h$.
    \item From \ref{lemsub:properties_of_meets_one}, as by definition, $f\cap g \le g$ and
      $f \cap g \le f$.
    \item $f\cap 1 = \restr{f\cap 1} (f \cap 1)= (\restr{f \cap 1} f ) \cap (\restr{f \cap 1})
      \le \restr{f \cap 1}$ from which the result follows. %def of $\le$, \rthree and \rone
    \item By definition and transitivity, $(f\cap g)\cap h \le f, g, h$ therefore by
      \ref{lemsub:properties_of_meets_one} $(f \cap g) \cap h \le f \cap (g \cap h)$. Similarly,
      $f \cap (g \cap h) \le(f \cap g) \cap h$ giving the equality.
    \item Given  $r f \cap g \le r f$, calculate:
      \[
        r f\cap g
        = \restr{r f\cap g} r f
        = \restr{r (r f\cap g)} f
        = \restr{r r f\cap r g} f
        = \restr{r (f\cap g)} f
        = r \restr{f\cap g} f
        = r (f\cap g).
      \]
    \item Using the previous point with the restriction idempotent $\restr{f r}$,
      \begin{equation*}
        \begin{split}
          f r \cap g
          = f \restr{r} \cap g   %%r rest id
          = \restr{f r }f \cap g  %% R.4
          = \restr{f r}(f\cap g)   %Pre
          = \restr{f r}\, \restr{f\cap g} f \\ % meet <=
          = \restr{f\cap g}\, \restr{f r} f % R2
          = \restr{f\cap g} f \restr{r}  %R4
          = (f\cap g) r. % meet, r rest id
        \end{split}
      \end{equation*}
    \item For the first claim,
      \[
        \restr{f\cap g}\, \restr{f} =\restr{\restr{f}(f\cap g)}\\
        =\restr{(\restr{f}f)\cap g} =\restr{f\cap g}.
      \]
      The second claim then follows by \ref{lemsub:properties_of_meets_two}.
    \item Given $ f \cap 1 \leq f$:
      \[
        f \cap 1 \leq f \iff  \restr{f \cap 1} f = f \cap 1 \iff  (f \cap 1) f = f \cap 1
      \]
      where the last step is by item \ref{lemsub:properties_of_meets_three} of this lemma.
    \item As $e$ is idempotent, $e (e\cap 1) = (e e \cap e) = e$.
  \end{enumerate}
\end{proof}
% subsection enrichment_and_meets (end)

\subsection{Range categories} % (fold)
\label{sub:range_categories}
Corresponding to Definition~\ref{def:restriction_category} for restriction, which axiomatizes the
concept of a domain of definition, we now introduce range categories which algebraically axiomatize
the concept of the range for a function.

\begin{definition}\label{def:range_category}
  A restriction category \X is a \emph{range category} when it has an operator on all maps
  \[
    \infer{\rg{f}:B\to B}{f:A\to B}
  \]
  where the operator satisfies the following:
  \begin{align*}
    &\rrone\ \restr{\rg{f}} = \rg{f} & &
     \rrtwo\ f \rg{f} = f\\
    &\rrthree\ \wrg{f\rst{g}} = \rg{f} \rst{g} & &
     \rrfour\  \wrg{\rg{f}g} = \wrg{f g}
  \end{align*}
  whenever the compositions are defined.

\end{definition}

\begin{lemma}\label{lem:basic_range_category_properties}
  In a range category \X, the following hold:
  \begin{multicols}{2}
    \begin{enumerate}[{(}i{)}]
      \item $\rg{g}\rg{f} = \rg{f}\rg{g}$;
      \item $\rst{f}\rg{g} = \rg{g}\rst{f}$;
      \item $\wrg{f\rg{g}} = \rg{f}\rg{g}$;
      \item $\rg{f} = 1$ when $f$ is epic, hence $\rg{1} = 1$;
      \item $\rg{f}\rg{f} = \rg{f}$;
      \item $\rg{\rg{f}} = \rg{f}$;
      \item $\rg{\rst{f}} = \rst{f}$;
      \item $\rg{g}\wrg{f g} = \wrg{f g}$;
      \item $\wrg{\rg{f}\rg{g}} = \rg{f}\rg{g}$.
    \end{enumerate}
  \end{multicols}
\end{lemma}
\begin{proof}
  See, e.g., \cite{guox:thesis}.
\end{proof}

\begin{lemma}\label{lem:ordering_of_restriction_and_range}
  In a range category:
  \begin{multicols}{2}
    \begin{enumerate}[{(}i{)}]
      \item  $\wrg{h f} \le \rg{f}$; \label{lemitem:ordering_1}
      \item $f' \le f$ implies $\rg{f'} \le \rg{f}$. \label{lemitem:ordering_2}
    \end{enumerate}
  \end{multicols}
\end{lemma}
\begin{proof}
  \prepprooflist
  \begin{enumerate}[{(}i{)}]
    \item Noting that $\rst{\wrg{hf}} \rg{f} = \wrg{hf} \rg{f}  = \wrg{hf \rg{f}} = \wrg{h f}$,
      we see $\wrg{h f} \le \rg{f}$.
    \item Calculating $\rst{\rg{f'}} \rg{f} = \rg{f'} \rg{f} = \wrg{\rst{f'} f} \rg{f} =
      \wrg{\rst{f'} f \rg{f}} = \wrg{\rst{f'} f} = \rg{f'}$, we see $\rg{f'} \le \rg{f}$.
  \end{enumerate}
\end{proof}

\begin{remark}
  Note that unlike restrictions, a range is a \emph{property} of a restriction category. To see
  this, assume we have two ranges $\wrg{(\_)}$ and $\widetilde{(\_)}$. Then,
  \[\rg{f}=\wrg{f \tilde{f}}=\rg{f} \tilde{f}=\tilde{f} \rg{f}=\widetilde{f \rg{f}}=\tilde{f}.\]
\end{remark}
\begin{lemma}\label{lem:inverse_categories_are_range_categories}
  An inverse category \X is a range category, where $\rg{f} = \inv{f}f = \rst{\inv{f}}$.
\end{lemma}
\begin{proof}
  \prepprooflist
  \setlist[enumerate,1]{leftmargin=1.5cm}
  \begin{enumerate}
    \item[\rrone] $\restr{\rg{f}} = \rst{\rst{\inv{f}}} = \rst{\inv{f}} = \rg{f}$;
    \item[\rrtwo] $f \rg{f} = f \rst{\inv{f}} = f \inv{f} f = \rst{f} f = f$;
    \item[\rrthree] $\wrg{f\rst{g}} = \rst{\inv{(f\rst{g})}} = \rst{\inv{\rst{g}} \inv{f}} =
      \rst{\rst{g} \inv{f}} =
      \rst{g} \rst{\inv{f}} = \rst{\inv{f}} \rst{g} =\rg{f} \rst{g}$;
    \item[\rrfour]  $\wrg{\rg{f}g} = \rst{\inv{(\rst{\inv{f}} g)}} =
      \rst{\inv{g}\inv{\rst{\inv{f}}}} = \rst{\inv{g} \rst{\inv{f}}} =
      \rst{\inv{g} \inv{f}} = \rst{\inv{(f g)}} = \wrg{f g}$
  \end{enumerate}
\end{proof}
% subsection range_categories (end)

\subsection{Partial monics, sections and isomorphisms} % (fold)
\label{sub:restricted_monics_sections_and_partial_isomorphisms}

Partial isomorphisms play a central role in this thesis. Below we present
some of their basic properties.

\begin{definition}
  For maps $f$ in a restriction category \X:
  \begin{itemize}
    \item $f$ is a \emph{partial isomorphism} when there is a \emph{partial inverse}, written
      $\inv{f}$ with $f\inv{f} =\restr{f}$ and $\inv{f}f = \restr{\inv{f}}$;
    \item $f$ is a \emph{partial monic} if $h f = k f \implies h \restr{f} = k \restr{f}$;
    \item $f$ is a \emph{partial section} if there exists an  $h$ such that $f h = \restr{f}$;
    \item $f$ is a \emph{restriction monic} if it is a section $s$ with a retraction
      $r$ such that $r s = \restr{r s}$.
  \end{itemize}
\end{definition}

Note that restriction monic is a stronger notion than that of monic. Consider two objects $A, B$
in a restriction category where we have $m: A\to B$, $r:B \to A$ with $m r = 1_A$. In this case
$A$ is called a \emph{retract} of $B$, which we will write as $A\retract B$. As $m$ and $r$ need
not be unique, we will also write $(m,r) A \retract B$ when the specific section and retraction
are to be emphasized. Since $m$ is a section,
it is a monic and therefore total. $r m$ is of course, an idempotent on $B$. $A$ is referred to as
a splitting of the idempotent $r m$. Note there is no requirement that $r m = \rst{r m}$ if $m$ is
simply monic.

\begin{lemma}
  \label{lem:rcs_partial_monic_section_inverse_properties}
  In a restriction category:
  \begin{enumerate}[{(}i{)}]
    \item $f,\ g$ partial monic implies $f g$ is partial monic;
    \item $f$ a partial section implies $f$ is partial monic;
    \item $f,\ g$ partial sections implies $f g$ is a partial section;
    \item The partial inverse of $f$, when it exists, is unique;
    \item If $f,\ g$ have partial inverses and $f\,g$ exists, then $f\,g$ has a partial inverse;
    \item A restriction monic $s$ is a partial isomorphism.
  \end{enumerate}
\end{lemma}
\begin{proof}
  \prepprooflist
  \begin{enumerate}[{(}i{)}]
    \item Suppose $h f g = k f g$. As $g$ is partial monic, $h f \restr{g} = k f \restr{g}$.
      Therefore:
      \begin{align*}
        h \restr{f g} f &= k \restr{f g} f &\rfour\\
        h \restr{f g}\,\restr{f} &= k \restr{f g}\, \restr{f} & f\text{partial monic}\\
        h \restr{fg}&= k \restr{fg} & \text{Lemma \ref{lem:restrictionvarious},
          \ref{lemitem:rv_2}}
      \end{align*}
    \item Suppose $g f = k f$. Then, $g\restr{f} = g f h = k f h = k \restr{f}$.
    \item We have $f h = \restr{f}$ and $g h' = \restr{g}$. Therefore,
      \begin{align*}
        f g h' h &= f \restr{g} h & g \text{ partial section}\\
        &= \restr{f g} f h & \rfour\\
        &= \restr{f g}\, \restr{f} & f \text{ partial section}\\
        &= \restr{f}\, \restr{f g} & \rtwo\\
        &= \restr{\restr{f}f g} & \rthree\\
        &= \restr{f g} & \rone
      \end{align*}
    \item Suppose both $\inv{f}$ and $f^*$ are partial inverses of $f$. Then,
      \begin{multline*}
        \inv{f}
        = \restr{\inv{f}}\inv{f} %R.1
        =\inv{f}f\inv{f}  %Assumption, inverse is \inv{f}
        = \inv{f} \restr{f}   %Assumption, inverse is \inv{f}
        = \inv{f} f f^*   %Assumption, inverse is f^*
        = \inv{f} f \restr{f^*} f^*  \\ %R.1
        = \restr{\inv{f}}\restr{f^*} f^*   %Assumption, inverse is \inv{f}
        = \restr{f^*}\restr{\inv{f}} f^* %R.2
        = f^* f \restr{\inv{f}}  f^* %Assumption, inverse is f^*
        = f^* f \inv{f} f f^* %Assumption, inverse is \inv{f}
        = f^* f f^* %Assumption, inverse is \inv{f} and R.1
        = f^* %Assumption, inverse is f^*
      \end{multline*}
    \item For $f:A\to B,\ g:B\to C$ with partial inverses $\inv{f}$ and $\inv{g}$ respectively,
      the partial inverse of $f g$ is $\inv{g} \inv{f}$. Calculating $f g \inv{g} \inv{f}$
      using all the restriction identities:
      \[
        f g \inv{g} \inv{f} = f \restr{g} \inv{f} = \restr{f g} f \inv{f} =
        \restr{f g}\, \restr{f} = \restr{f}\, \restr{f g} = \restr{\restr{f} f g} = \restr{f g}.
      \]
      The calculation of $\inv{g} \inv{f} f g = \restr{\inv{g} \inv{f}}$ is similar.
    \item The partial inverse of $s$ is $\restr{r s}\,r$. First, note
      that $\restr{\restr{r s}\,r}
      = \restr{r s}\,\restr{r}
      = \restr{r}\, \restr{r s}
      = \restr{\restr{r}\,r s}
      = \restr{r s}$.
      Then, it follows that $(\restr{r s}\,r) s
      = r s\,= \restr{r s}
      = \restr{\restr{r s}r} $ and
      $s (\restr{r s}\,r)
      = s r \restr{s} %sr = 1 as r is the retraction of the section s
      = \restr{s}$.
  \end{enumerate}
\end{proof}


A restriction category in which every map is a partial
isomorphism is called an \emph{inverse category}.

An interesting property of inverse categories:

\begin{lemma}
  \label{lem:inverse_idempotents_are_restriction_idempotents}
  In an inverse category, all idempotents are restriction idempotents.
\end{lemma}
\begin{proof}
  Given an idempotent $e$,
  \[
    \rst{e} = e\inv{e} = e e \inv{e} = e \rst{e} = \rst{e e} e = \rst{e} e = e.
  \]
\end{proof}
% subsection restricted_monics_sections_and_partial_isomorphisms (end)
\subsection{Split restriction categories} % (fold)
\label{sub:split_restriction_categories}

The split restriction category, $\spl{E}{\X}$  is defined as:
\category{$(A,e)$, where $A$ is an object of \X, $e:A\to A$ and $e\in E$.}
  {$f:(A,d)\to(B,e)$ is given by $f:A\to B$ in \X, where $f = d\uts f e$.}
  {The map $e$ for $(A,e)$.}
  {inherited from \X.}
This is the standard idempotent splitting construction, also known as the Karoubi
envelope.

Note that for $f:(A,d)\to(B,e)$, by definition, in \X we have $f=d\uts f e$, giving
\[
  d\uts f = d(d\uts f e) = d\uts d\uts f e = d\uts f e =f\
  \text{ and }\  f e = (d\uts f e)e = d\uts f e\uts e = d\uts f e = f.
\]
When \X is a restriction category, there is an immediate candidate for a restriction in
$\spl{E}{\X}$. If $f\in\spl{E}{\X}$ is $e_1 f e_2$ in $\X$, then define $\restr{f}$ as
given by $e_1 \restr{f}$ in \X. Note that for $f:(A,d)\to(B,e)$, in \X we have:
\[
  d\uts\restr{f} = \restr{d\uts f} d = \restr{f} d.
\]
\begin{proposition}\label{prop:spleisarestrictioncat}
  If \X is a restriction category and $E$ is a set of idempotents, then
  the restriction as defined above makes $\spl{E}{\X}$ a restriction category.
\end{proposition}
\begin{proof}
  The restriction takes $f:(A,e_1)\to (B,e_2)$ to an endomorphism of $(A,e_1)$. The restriction
  is in $\spl{E}{\X}$ as
  \[
    e_1 (e_1\restr{f}) e_1 = e_1 \restr{f} e_1
    = \restr{e_1 f} e_1 e_1
    = \restr{e_1 f} e_1
    = e_1 \restr{ f}.
  \]

  Checking the 4 restriction axioms:
  \begin{align*}
    &[\text{{\bfseries R.1}}]\  \llbracket\restr{f}f\rrbracket = e_1 \restr{f} f
    = e_1 f = \llbracket f\rrbracket.\\
    %
    & [\text{{\textbf{R.2}}}]\ \llbracket\restr{g}\restr{f}\rrbracket =
    e_1\restr{g}  e_1\restr{f}
    = e_1 e_1\restr{g}  \restr{f} = e_1 e_1\restr{f}  \restr{g}
    = e_1\restr{f}  e_1\restr{g}  = \llbracket\restr{f}\restr{g}\rrbracket.\\
    %
    & [\text{{\textbf{R.3}}}]\ \llbracket\restr{\restr{f} g } \equiv
    e_1 \restr{e_1 \restr{f}  g}
    =  \restr{e_1 e_1 \restr{f} g} e_1
    =  \restr{e_1 \restr{f} g} e_1
    =  e_1 \restr{\restr{f} g}
    = e_1 \restr{f}\restr{g}
    = e_{1}e_{1}\restr{f}\restr{g}
    = e_1 \restr{f}e_1\restr{g}
    = \llbracket\restr{f}\, \restr{g}\rrbracket.\\
    %
    &[\text{{\textbf{R.4}}}]\  \llbracket f \restr{g} \rrbracket =
     e_1f e_2 \restr{g}
    = \restr{e_1 f e_2 g} e_1 f e_2
    = \restr{e_1 e_1 f e_2 g} e_1 e_1 f e_2 \\
    & \qquad \qquad \qquad \qquad \qquad \qquad \qquad \qquad \qquad \quad
    = e_1 \restr{ e_1 f e_2 g} e_1 f e_2
    = e_1 \restr{f g} e_1 f e_2
    = \llbracket\restr{f g} f\rrbracket.
  \end{align*}
\end{proof}

Given this, provided all identity maps are in $E$, $\spl{E}{\X}$ is a
restriction category with $\X$ as a full sub-restriction category, via
the embedding defined by taking an object $A$ in \X to  the object $(A,1)$
in $\spl{E}{\X}$.  Furthermore, the property of being an inverse category is
preserved by splitting.

\begin{lemma}\label{lem:the_idempotent_splitting_of_an_inverse_category_is_an_inverse_category}
  When \X is an inverse category, $\spl{E}{X}$ is an inverse category.
\end{lemma}
\begin{proof}
  The inverse of $f:(A,e_1)\to(B,e_2)$   in \spl{E}{\X} is $e_2\inv{f}e_1$ as
  \[
    \llbracket f \inv{f} \rrbracket = e_1 f e_2 e_2 \inv{f} e_1
    = e_1 e_1 f e_2 \inv{f} e_1
    = e_1 f  \inv{f} e_1
    = e_1 e_1 \restr{f} e_1
    = e_1 \restr{f}
    = \llbracket\restr{f}\rrbracket
  \]
  and
  \begin{multline*}
    \llbracket \inv{f} f\rrbracket=
    e_2 \inv{f} e_1 e_1 f e_2
    = e_2 \inv{f} e_1 f e_2 e_2
    = e_2 \inv{f} f  e_2\\
    = e_2 e_2 \restr{\inv{f}}  e_2
    = e_2 \restr{\inv{f}}
    = \llbracket\restr{\inv{f}}\rrbracket.
  \end{multline*}

\end{proof}

\begin{proposition}\label{pro:in_rc_x_with_meets_split_x_is_cong_to_split_r_x}
  In a restriction category \X, with meets, let $R$ be the set of restriction idempotents.
  Then, $\spl{}{\X} \cong \spl{R}{\X}$(where \spl{}{\X} is the split of \X over all idempotents).
  Furthermore, $\spl{R}{\X}$ has meets.
\end{proposition}
\begin{proof}
  The proof below first shows the equivalence of the two categories, then addresses the claim
  that $\spl{R}{\X}$ has meets.

  For equivalence, we require two functors,
  \[
    U:\spl{R}{\X}\to\spl{}{\X}\text{ and }V:\spl{}{\X}\to\spl{R}{\X},
  \]
  with:
  \begin{align}
    U V \cong I_{\spl{R}{\X}}\\
    V U \cong I_{\spl{}{\X}}.
  \end{align}


  $U$ is the standard inclusion functor. $V$ will take the object $(A,e)$ to
  $(A,e\cap 1)$ and the map $f:(A,e_1)\to (B,e_2)$ to $(e_1\cap 1)f $.

  $V$ is a functor as:
  \begin{description}
    \itembf{Well Defined:} If  $f:(A,e_1) \to (B,e_2)$, then
      $(e_1\cap 1) f $ is a map in \X from $A$ to $B$ and
      $ (e_1\cap 1)(e_1\cap 1) f  (e_2 \cap 1) =
      (e_1\cap 1) (f  e_2 \cap f ) = (e_1\cap 1) (f \cap f)= (e_1\cap 1) f$, therefore,
      $V(f):V((A,e_1)) \to V((B,e_2))$.
    \itembf{Identities:} $V(e) = (e\cap 1 ) e = e \cap 1$ by
      lemma \ref{lem:properties_of_meets_in_restriction_categories}.
    \itembf{Composition:} $V(f) V(g)
      = (e_1\cap 1 ) f (e_2 \cap 1) g
      = (e_1\cap 1 ) f e_2 (e_2 \cap 1) g
      = (e_1\cap 1 ) f  (e_2 \cap e_{2}) g
      = (e_1\cap 1 ) f e_2 g
      = (e_1\cap 1 ) f g
      = V(f g)$.
  \end{description}

  Recalling from Lemma \ref{lem:properties_of_meets_in_restriction_categories}, $(e\cap 1)$
  is a restriction idempotent. Using this fact, the commutativity of restriction idempotents
  and the general idempotent identities from
  \ref{lem:properties_of_meets_in_restriction_categories}, the composite functor $U V$ is
  the identity on $\spl{r}{\X}$ as when $e$ is a restriction idempotent,
  $e = e (e\cap 1) = (e\cap 1) e = (e\cap 1)$.

  For the other direction,  note that for a particular idempotent $e:A\to A$,  this gives the
  maps $e:(A,e)\to(A,e\cap 1)$ and $e\cap 1 : (A,e\cap 1) \to (A,e)$, again by
  \ref{lem:properties_of_meets_in_restriction_categories}. These maps give the natural
  isomorphism between $I$ and $V U$ as
  \[
    \xymatrix{
      (A,e)\ar[r]^e \ar[dr]_{e} &(A,e\cap 1)\ar[d]^{e\cap 1}\\
      &(A,e)
    }\qquad \text{ and  }\qquad
    \xymatrix{
      (A,e\cap 1)\ar[r]^{e\cap 1} \ar[dr]_{e\cap 1} &(A,e)\ar[d]^{e}\\
      &(A,e\cap 1)
    }
  \]
  both commute. Therefore, $U V = I$ and $V U \cong I$, giving an equivalence of the categories.

  For the rest of this proof, the bolded functions, e.g., $\mbf$ are in $\spl{R}{\X}$.
  Italic functions, e.g., $f$ are in \X.

  To show that $\spl{R}{\X}$ has meets,  designate the meet in $\spl{R}{\X}$ as \capspl
  and define $\mbf \capspl \mbg$ as the map given by the \X map $f \cap g$, where
  $\mbf,\mbg:(A,d)\to(B,e)$ in $\spl{R}{\X}$ and $f,g:A\to B$ in \X . This is
  a map in $\spl{R}{\X}$ as
  $d(f \cap g)e = (d\uts f \cap d\uts g) e = (f \cap g) e = (f e \cap g) = f\cap g$
  where the penultimate equality is by
  \ref{lem:properties_of_meets_in_restriction_categories}.
  By definition $\restr{\mbf \capspl \mbg }$ is $d\restr{f\cap g}$.

  It is necessary to show \capspl satisfies the four meet properties.
  \begin{itemize}
    \item{$\mbf\capspl \mbg \le \mbf$: } We need to show
      $\rst{\mbf \capspl \mbg} \mbf =  \mbf \capspl \mbg$.  Calculating now in \X:
      \begin{align*}
        d \rst{f \cap g} f&= \rst{d(f\cap g)} d f\\
        & = \rst{df \cap dg} df \\
        & = \rst{f \cap g} f \\
        & = f \cap g
      \end{align*}
      which is the definition of $\mbf \capspl \mbg$.
    \item{$\mbf\capspl \mbg \le \mbg$: } Similarly and once again calculating in \X,
      \begin{align*}
        d \rst{f \cap g} g&= \rst{d(f\cap g)} d g\\
        & = \rst{df \cap dg} dg \\
        & = \rst{f \cap g} g \\
        & = f \cap g
      \end{align*}
      which is the definition of $\mbf \capspl \mbg$.
    \item{$\mbf\capspl \mbf = \mbf$: } From the definition, this is $f \cap f = f$ which
      is just $ \mbf$.
    \item{$\mbh(\mbf\capspl \mbg) = \mbh\mbf \capspl \mbh\mbg$: }
      From the definition, this is given in \X by $ h (f \cap g) =
      h f \cap h g$ which in $\spl{R}{\X}$ is $\mbh\mbf \capspl \mbh\mbg$.
  \end{itemize}
\end{proof}
% subsection split_restriction_categories (end)



\subsection{Partial Map Categories} % (fold)
\label{sub:partial_map_categories}

In \cite{cockett2002:restcategories1}, it is shown that split restriction categories are
equivalent to \emph{partial map categories}. The main definitions and results related to
partial map categories are given below.

\begin{definition}
  A collection $\Mstab$ of monics is \emph{a stable system of monics}
  when:
  \begin{enumerate}[{(}i{)}]
    \item it includes all isomorphisms;
    \item it is closed under composition;
    \item it is pullback stable.
  \end{enumerate}
\end{definition}

\emph{Stable} in this definition means that if $m:A\to B$ is in \Mstab, then for arbitrary
$b$ with co-domain $B$, the pullback
\[
  \xymatrix{
    A'\ar[r]^a \ar[d]_{m'} &A\ar[d]^{m}\\
    B' \ar[r]_{b} & B
  }
\]
exists and $m' \in \Mstab$. A category that has a stable system of monics
is referred to as an \Mstab-category.

\begin{lemma}
  If $nm \in \Mstab$, a stable system of monics, and $m$ is monic, then $n \in \Mstab$.
\end{lemma}
\begin{proof}
  The commutative square
  \[
    \xymatrix{
      A\ar[d]_n \ar[r]^{1} &A\ar[d]^{nm}\\
      A' \ar[r]_{m} & B
    }
  \]
  is a pullback.
\end{proof}

Given a category \C and a stable system of monics, the \emph{partial map category},
$\text{Par}(\C,\Mstab)$ is:
  \rcategoryequiv{$A\in\C$}
    {$(m,f):A\to B$  with $m:A' \to A$ is in \Mstab and $f:A' \to B$ is a map in \C. i.e.,
      $\xymatrix @R-15pt @C-15pt{&A'\ar[dl]_{m} \ar[dr]^{f}\\A&&B}$.}
    {$1_A,1_A:A \to A$}
    {via a pullback, $(m,f)(m',g) = (m'' m, f' g)$ where
      \[
        \xymatrix @C-15pt @R-15pt{
          &&A''\ar[dl]_{m''}\ar[dr]^{f'}\\
          &A'\ar[dl]_{m}\ar[dr]_{f}&\text{{\tiny (pb)}}&B'\ar[dl]^{m'}\ar[dr]^{g}\\
          A&&B&&C
        }
      \]
    }
    {$\restr{(m,f)} = (m,m)$}

For the maps, $(m,f) \sim (m',f')$ when there is an isomorphism $\gamma : A'' \to A'$
such that $\gamma m' = m$ and $\gamma f' = f$.

In \cite{cockettlack2003:restcategories2}, it is shown that:
\begin{theorem}[Cockett-Lack]
  Every restriction category is a full sub-category of a partial map category.
\end{theorem}
% subsection partial_map_categories (end)
\subsection{Restriction products and Cartesian restriction categories} % (fold)
\label{sub:restriction_products_and_cartesian_restriction_categories}


Restriction categories have analogues of products and terminal objects.

\begin{definition}
  In a restriction category \X\, a \emph{restriction product}  of two objects $X, Y$ is an
  object $X\times Y$ equipped with \emph{total} projections
  $\pi_0:X\times Y\to X, \pi_1:X\times Y\to Y $ where:
  \begin{quote}
    $\forall f:Z\to X, g: Z\to Y,\quad \exists$ a unique $\<f,g\>:Z \to X\times Y$ such that
    \begin{itemize}
      \item $\<f,g\> \pi_0 \le f$,
      \item $\<f,g\> \pi_1 \le g$ and
      \item $\restr{\<f,g\>} = \restr{f}\, \restr{g} ( = \restr{g}\, \restr{f})$.
    \end{itemize}
  \end{quote}
\end{definition}

\begin{definition}
  In a restriction category \X\, a \emph{restriction terminal object}
  is an object $\top$ such that $\forall X$, there is a
  unique total map $!_X : X \to \top$ and the diagram
  \[
    \xymatrix @C=40pt @R=25pt{
      X \ar[r]^{\restr{f}} \ar[d]^{f} & X \ar[r]^{!_X}  &\top  \\
      Y \ar[urr]_{!_Y}
    }
  \]
  commutes. That is,  $f\, !_Y = \restr{f}\, !_X$. Note this implies
  that a restriction terminal object is unique up to a unique isomorphism.
\end{definition}

\begin{definition}
  A restriction category \X\ is \emph{Cartesian} if it has all restriction products
  and a restriction terminal object.
\end{definition}

\begin{definition}
  An object $A$ in a Cartesian restriction category is \emph{discrete}
  when the diagonal map,
  \[
    \Delta:A \to A \times A
  \]
  is a partial isomorphism.

  A Cartesian restriction category is \emph{discrete} when every object is discrete.
\end{definition}

\begin{theorem}\label{thm:a_crc_is_discrete_iff_it_has_meets}
  A Cartesian restriction category \X is discrete if and only if it has meets.
\end{theorem}
\begin{proof}
  If \X has meets, then
  \[
    \Delta(\pi_0 \cap \pi_1) = \Delta\pi_0 \cap \Delta\pi_1 = 1\cap 1 = 1
  \]
  and as $\<\pi_0,\pi_1\>$ is identity,
  \begin{align*}
    \restr{\pi_0 \cap \pi_1} &= \restr{\pi_0 \cap \pi_1} \<\pi_0, \pi_1\> \\
    &=\<\rst{\pi_0 \cap \pi_1}\pi_0, \rst{\pi_0 \cap \pi_1}\pi_1\>\\
    &=\<\pi_0 \cap \pi_1,\pi_0 \cap \pi_1\>\\
    &=(\pi_0 \cap \pi_1 )\Delta
  \end{align*}
  and therefore, $\pi_0 \cap \pi_1$ is $\inv{\Delta}$.

  For the other direction, set $f\cap g = \<f,g\>\inv{\Delta}$.
  By the definition of the restriction product:
  \[
    f \cap g =  \<f,g\>\inv{\Delta} =\<f,g\>\inv{\Delta} \Delta \pi_0 =
      \<f,g\>\restr{\inv{\Delta}}\pi_0 \le \<f,g\>\pi_0 \le f
  \]
  Similarly, substituting $\pi_1$ for $\pi_0$ above, this gives $f \cap g \le g$.
  For the left distributive law,
  \[
    h(f \cap g) = h \<f,g\>\inv{\Delta} =  \<h f,h g\>\inv{\Delta} = h f \cap h g
  \]
  and finally an intersection of a map with itself is
  \[
    f\cap f = \<f,f\> \inv{\Delta} = (f \Delta) \inv{\Delta} = f \restr{\Delta} = f
  \]
  as $\Delta$ is total. This shows that $\cap$ as defined above is a meet for the
  Cartesian restriction category \X.

\end{proof}

We shall refer to a Cartesian restriction category in which every object is
discrete as simply a discrete restriction category.
% subsection restriction_products_and_cartesian_restriction_categories (end)

\subsection{Discrete Categories} % (fold)
\label{sub:discrete_categories}


In a Cartesian restriction category, a map $A\xrightarrow{f}B$ is
called \emph{graphic} when the maps
\[
  A\xrightarrow{\<f,1\>}B\times A\qquad \text{and}\qquad
  A\xrightarrow{\<\rst{f},1\>}A\times A
\]
have partial inverses. A Cartesian restriction category is \emph{graphic} when all of its maps
are graphic.
\begin{lemma}\label{lem:graphic_maps_are_closed_in_a_cartesian_restriction_category}
  In a Cartesian restriction category:
  \begin{enumerate}[{(}i{)}]
    \item Graphic maps are closed under composition;
    \item Graphic maps are closed under the restriction;
    \item An object is discrete if and only if its identity map is graphic.
  \end{enumerate}
\end{lemma}
\begin{proof}
  \prepprooflist
  \begin{enumerate}[{(}i{)}]
    \item To show closure, it is necessary to show that $\<f g,1\>$ has a partial inverse.
      By Lemma \ref{lem:rcs_partial_monic_section_inverse_properties}, the uniqueness of the
      partial inverse gives
      \[
        \inv{(\<f,1\> ; \<g,1\>\times 1)} = \inv{\<g,1\>} \times 1 ; \inv{\<f,1\>} .
      \]
      By the definition of the restriction product, $\rst{\<f g,1\>} = \rst{f g}$. Additionally,
      a straightforward calculation shows that
        $\rst{\<f,1\>;\<g,1\> \times 1} =
          \rst{\<f\<g,1\>, 1\>} = \rst{f ;\< g,1\>}
          = \rst{\<f;g, f\>} = \rst{f g}\,\rst{f} = \rst{f g}
        $
      where the last equality is by \rtwo, \rthree and finally \rone.

    Consider the diagram
    \[
      \xymatrix @C+35pt @R+20pt{
        A \ar[r]^{\<f,1\>} \ar[drr]_{\<f g,1\>} &
           B \times A  \ar[r]^{\<g,1\> \times 1}
           &  C \times B \times A \\
        &&C \times A \ar[u]_{1 \times \<f,1\>}
      }
    \]

    From this:
    \begin{align*}
      \<f g,1\>  (1\times \<f,1\>) ( \inv{\<g,1\>}\times 1) \inv{\<f,1\>}
      &=\<f,1\>(\<g,1\>\times 1 ) (\inv{\<g,1\>}\times 1) \inv{\<f,1\>}\\
      &=\<f,1\> (\rst{g\times 1}) \inv{\<f,1\>}\\
      &=\rst{\<f,1\> (g\times 1)}  \<f,1\> \inv{\<f,1\>}\\
      &=\rst{\<f,1\> (g\times 1)}  \rst{\<f,1\>}\\
      &= \rst{\<f,1\>} \rst{\<f,1\>(g\times 1)}\\
      &= \rst{\<f,1\> (g\times 1)}\\
      &= \rst{\<f g,1\>}(=\restr{f g})\\
    \end{align*}
    showing that $1\times \<f,1\>  (\inv{\<g,1\>}\times 1 ) \inv{\<f,1\>}$ is
    a right inverse for $\<f g,1\>$.

    For the other direction, note that in general $\inv{h k} = \inv{k}\inv{h}$ and that
    we have $\<f g,1\> = \<f,1\> (\<g,1\>\times 1)  (1 \times \inv{\<f,1\>})$, thus
    $(1\times \<f,1\>)  (\inv{\<g,1\>}\times 1) \inv{\<f,1\>}$ will also be a left inverse and
    $\<f g,1\>$ is a restriction isomorphism.

    \item This follows from the definition of graphic and that
       $\rst{\<f,1\>} = \rst{f} = \restr{\rst{f}}$.

    \item Given a discrete object $A$, the map $1_A$ is graphic as $\<1_A,1\> = \Delta$
      and therefore $\inv{\<1,1\>} = \inv{\Delta}$. Conversely, if $\<1_A,1\>$ has an inverse,
      then $\Delta = \<1_A,1\>$ has that same inverse and therefore the object is discrete.
  \end{enumerate}
\end{proof}

\begin{lemma}\label{lem:a_discrete_crc_is_precisely_a_graphic_crc}
  A discrete restriction category is precisely a graphic Cartesian restriction category.
\end{lemma}
\begin{proof}
  The requirement is that $\<f,1\>$ (and $\<\rst{f},1\>$) each have partial inverses. For
  $\<f,1\>$, the inverse is $\rst{(1 \times f)\inv{\Delta}} \pi_1$.

  To show this, calculate  the two compositions. First,
  \[
    \<f,1\> \rst{1 \times f \inv{\Delta}} \pi_1 =
      \rst{\<f,f\> \inv{\Delta}}\<f,1\>\pi_1 % use R.4
    = \rst{f \Delta \inv{\Delta}}\<f,1\>\pi_1 % product
    = \rst{f}\<f,1\>\pi_1 % Delta total
    = \rst{f}.% product
  \]
  The other direction is:
  \begin{align*}
    \rst{(1 \times f)\inv{\Delta}} \pi_1 \<f,1\>
      &= \< \rst{(1 \times f)\inv{\Delta}} \pi_1 f ,
      \rst{(1 \times f)\inv{\Delta}}\pi_1 \>\\ %product definition
    &= \< \rst{(1 \times f)\inv{\Delta}} (1 \times f) \pi_1,
      \rst{(1 \times f)\inv{\Delta}}\pi_1 \>\\ %pi total, natural
    &= \< (1 \times f )\rst{\inv{\Delta}} \pi_1 ,
      \rst{(1 \times f)\inv{\Delta}}\pi_1 \>\\ %R.4
    &= \< (1 \times f) \rst{\inv{\Delta}} \pi_0 ,
      \rst{(1 \times f)\inv{\Delta}}\pi_1 \>\\ %below
    &= \< \rst{(1 \times f)\inv{\Delta}} (1 \times f) \pi_0,
      \rst{(1 \times f)\inv{\Delta}}\pi_1 \>\\ %R.4
  %  &= \< \rst{(1 \times f)\inv{\Delta}}\,
  %     \rst{(1 \times f)} \pi_0, \rst{(1 \times f)\inv{\Delta}}\pi_1 \>\\
    &= \< \rst{(1 \times f)\inv{\Delta}} \pi_0,
      \rst{(1 \times f)\inv{\Delta}}\pi_1 \>\\ %(a x b);pi0 = a
    &= \rst{(1 \times f)\inv{\Delta}} \< \pi_0, \pi_1 \>\\ % products
    &= \rst{(1 \times f)\inv{\Delta}}
  \end{align*}
  The one tricky step is to realize
  \begin{align*}
    \rst{\inv{\Delta}} \pi_1
      &= \inv{\Delta} \Delta \pi_1\\
      &= \inv{\Delta}\\
      &= \inv{\Delta} \Delta \pi_0\\
      &= \rst{\inv{\Delta}} \pi_0
  \end{align*}

  For $\<\rst{f},1\>$, the inverse is $\rst{(1 \times \rst{f})\inv{\Delta}} \pi_1$. Similarly
  to above,
  \[
    \<\rst{f},1\> \rst{1 \times \rst{f} \inv{\Delta}} \pi_1 =
      \rst{\<\rst{f},\rst{f}\> \inv{\Delta}}\<\rst{f},1\>\pi_1 % use R.4
    = \rst{\rst{f} \Delta \inv{\Delta}}\<\rst{f},1\>\pi_1 % product
    = \rst{\rst{f}}\<\rst{f},1\>\pi_1 % Delta total
    = \rst{f}.% product
  \]
  The other direction follows the same pattern as for $\<f,1\>.$
\end{proof}
% subsection graphic_categories (end)

% section restriction_categories (end)
%!TEX root = /Users/gilesb/UofC/thesis/phd-thesis/phd-thesis.tex
\chapter{Turing categories and PCAs} % (fold)
\label{chap:turing_categories}

In this chapter, we review the definition and properties of a Turing category and partial
combinatory algebras
\cite{cockett-hostra08-intro-to-turing,cockett2010:categories-and-computability}. Because of
the theorems of the earlier chapters, we will be able to transfer these ideas in a straightforward
way from discrete Cartesian restriction categories to discrete inverse categories.  Inverse Turing
categories are defined below and correspond to Turing categories using
Theorem~\ref{thm:inverse_turing_category_gives_a_turing_category}: This  provides the link between
reversible computation and standard models of computation as promised in the introduction.

As noted in the introduction, Bennett\cite{bennett:1973reverse} showed how a reversible Turing
machine can emulate a standard Turing machine. As Turing machines can perform the applications of a
partial combinatory algebra, we have a link between inverse Turing categories through Turing
categories to reversible Turing machines.

\section{Turing categories}
\label{sec:turing_category_definitions}
Turing categories provide a categorical formulation for
computability and includes partial combinatory algebras, the partial lambda calculus, and various
other models as given in
\cite{cockett-hostra08-intro-to-turing}.

\begin{definition}[Turing category]\label{def:turing_category}
  Given \X is a Cartesian restriction category:
  \begin{enumerate}[{(}i{)}]
    \item For a map $\txy: A \times X \to Y$, a map $f:B\times X \to Y$ \emph{admits a $\txy$-index}
      when there is a total $\name{f}:B\to A$ such that
      \[
        \xymatrix@C+10pt@R+10pt{
          A\times X \ar[r]^{\txy} & Y \\
          B\times X \ar@{.>}[u]^{\name{f}\times 1_X} \ar[ur]_f
        }
      \]
      commutes.\label{defitem:turing_admit_txy_index}
    \item A map $\txy: A \times X \to Y$ is called a \emph{universal application} if all
      $f:B\times X \to Y$ admit a $\txy$-index.\label{defitem:turing_universal_application}
    \item If $A$ is an object in $\X$ such that for every pair of objects $X,Y$ in \X there is
      a universal application $\tau:A\times X \to Y$, then $A$ is called a \emph{Turing object}.
    \item A Cartesian restriction category that contains a Turing object is called a
      \emph{Turing category}.
  \end{enumerate}
\end{definition}

Note there is no requirement in the definition for the map $\name{f}$ to be unique. When $\name{f}$ is unique
for a specific $\txy$, then that $\txy$ is called \emph{extensional}. In the case where the object
$B$ is the terminal object, then the map $\name{f}$ is a point of $A$ (with $f = (\name{f} \times 1)\txy$) and
$\name{f}$ is referred to as a \emph{code} of $f$.

\begin{example}\label{ex:turing-category-kleene}
  This example is due to Cockett and Hofstra\cite{cockett-hostra08-intro-to-turing}.

  We start with a ``suitable'' enumeration of partial recursive functions $f:\nat\to\nat$. Based on
  the fact that functions such as these can be described by Turing machines, and that Turing
  machines may be enumerated as $\{\phi_0,\phi_1,\ldots\}$, each of these functions can be coded
  into a single number. This may be extended to partial recursive functions of $n$ variables which
  may similarly be enumerated, $\{\phi_0^{(n)},\phi_1^{(n)},\ldots\}$. When $f$ is given by $\phi_e$
  we say $e$ is a \emph{code} for $f$.

  Two facts we will need about the family $\phi_m^{(n)}$:
  \begin{itemize}
    \item \textbf{Universal Functions:} There are partial recursive functions such that for each $n > 0$,
      \[ \Phi^{(n)}(e,x_1,\ldots,x_n) = \phi_e(x_1,\ldots,x_n) \]
      which are called the \emph{universal functions}.
    \item \textbf{Parameter Theorem:}There are primitive recursive functions $S_m^n$ for each $n,m >
      0$ such that:
      \[ \Phi^{(n+m)}(e,x_1,\ldots,x_m,u_1,\ldots,u_n) = \Phi^{(m)}(S_m^n(e,x_1,\ldots,x_m),u_1,\ldots,u_n).\]
  \end{itemize}

  Suppose we choose such an enumeration and use it for the \emph{Kleene-application} on the natural
  numbers, i.e., $\nat\times\nat\xrightarrow{\bullet}\nat$ will be defined as
  \[
    n\bullet x = \phi_n(x).
  \]

  Now, consider the category with objects being the finite powers of $\nat$ and a map $\nat^k \to
  \nat^m$ is an $m$-tuple of partial recursive function of $k$ variables. When $k$ is zero, the map
  simply picks a specific $m$-tuple of $\nat$. We denote this category by $\compn$.
  Then, $\compn$ is a Turing category with $\N$ being a Turing object, using Kleene-application
  as the application map. We know $\N$ is isomorphic to $\N\times \N$ and hence we have $\N\times \N
  \retract \N$. Application is guaranteed to be partial recursive by the universal functions item
  above and the weak universal property of $\bullet$ is a result of the Parameter Theorem.
\end{example}

\begin{definition}\label{def:turing_structure}
  Given $\T$ is a Turing category and $A$ is an object of \T,
  \begin{enumerate}[{(}i{)}]
    \item If $\Upsilon=\{\txy: A\times X \to Y | X,Y \in \T_o\}$, then $\Upsilon$ is called an
      \emph{applicative family} for $A$.
    \item An applicative family $\Upsilon$ is called \emph{universal for $A$} when each $\txy$ is
      a universal application. This is also referred to as a \emph{Turing structure} on $A$.
    \item A pair $(A,\Upsilon)$ where $\Upsilon$ is universal for $A$ is called a \emph{Turing
      structure} on \T.
  \end{enumerate}
\end{definition}

\begin{lemma}\label{lem:turing_object_is_retractable}
  If \T is a Turing category with Turing object $T$, then every object $B$ in \T is a retract of
  $T$.
\end{lemma}
\begin{proof}
  As $T$ is a Turing object, we have a diagram for $\tur{1}{B}$ and $\pi_0:B\times 1 \to B$:
  \[
    \xymatrix@C+10pt@R+10pt{
      T\times 1 \ar[r]^{\tur{1}{B}} & B \\
      B\times 1. \ar@{.>}[u]^{\name{\pi_0}\times 1_1} \ar[ur]_{\pi_0}
    }
  \]
  Note we also have $u_r:B\to B\times 1$ is an
  isomorphism and therefore we have $1_X = u_r \pi_0 = (u_r (\name{\pi_0}\times 1)) \tur{1}{X}$. Hence, we
  have $B \retractmaps{u_r (\name{\pi_0}\times 1)}{\tur{1}{X}} T$.
\end{proof}

This allows for various recognition criteria for Turing categories.

\begin{theorem}\label{thm:turing_recognition}
  A Cartesian restriction category \D is a Turing category if and only if $\D$ has an object $T$
  for which every other object of \D is a retract and $T$ has a universal self-application map
  $\bullet$, written as $T\times T \xrightarrow{\ \bullet\ }T$.
\end{theorem}
\begin{proof}
  The ``only if'' portion follows immediately from setting $T$ to be the Turing object of $\D$ and
  $\bullet = \tur{T}{T}$.

  For the ``if'' direction, we need to construct the family of universal applications
  $\txy : T\times X \to Y$ for each pair of objects $X,Y$ in \D.

  Let us choose pairs of maps that witness the retractions of $X, Y$ of $T$, that is:
  \[
    X \retractmaps{m_X}{r_X} T \quad\text{and}\quad Y\retractmaps{m_Y}{r_Y} T.
  \]
  Define $\txy = (1_T\times m_X ) \bullet r_Y$. Suppose we are given $f:B\times X \to Y$. Consider
  \[
    \xymatrix@C+10pt@R+10pt{
      T \times X \ar[r]^{1_T\times m_X} & T\times T \ar[r]^{\bullet} & T \ar[r]^{r_Y} & Y \\
      B \times X \ar[r]^{1_B\times m_X} \ar[u]^{h\times 1_X}
        & B \times T \ar@{.>}[u]^{h\times 1_T} \ar[r]^{1_B \times r_X}
        & B \times X \ar[ur]^{f} \ar[u]^{f m_Y}
      }
  \]
  where $h$ is the index for the composite map $(1_B \times r_X) f m_Y$. The middle square commutes
  as $\bullet$ is a universal application for $T,T$. The right triangle commutes as $m_Y r_Y =1$.
  The left square commutes as each composite is $h \times m_X$. Noting that the bottom path from
  $B\times X$ to $Y$ is $(1_B \times m_X)(1_B \times r_X)f = f$ and the top path from $T\times X$ to
  $Y$ is our definition of $\txy$, this means $f$ admits the $\txy$-index $h$.
\end{proof}

Note that different splittings (choices of $(m,r)$ pairs) would lead to different $\txy$ maps. In
fact there is no requirement that this is the only way to create a universal applicative family
for $T$.

There is another criteria that also gives a Turing category:

\begin{lemma}\label{lem:t_t_to_t_gives_a_turing_category}
  A Cartesian restriction category \T is a Turing category if:
  \begin{enumerate}[{(}i{)}]
  \item $\T$ has an object $T$ for which every other object of \T is a retract;
  \item $T\times T$ has a map  $T\times T \xrightarrow{\ \circ\ }T$ and for all
    $f:T\to T$ there exists an element, $\code{f}:1\to T$ (which is total) such that
      \[
        \xymatrix{
          T\times T \ar[r]^{\circ} & T \\
          T \ar[ur]_{f} \ar[u]^{\<!\code{f},1\>}
        }
      \]
    is a commutative diagram.
  \end{enumerate}
\end{lemma}
\begin{proof}
  We need only show that $T$ has a universal self-application map and then use
  Theorem~\ref{thm:turing_recognition}.

  $T$ having a universal self-application map, $\bullet$, means for every map $f:B\times T \to T$ there is a
  map, $\name{f}:B\to T$ such that
  \[
    \xymatrix{
      T\times T \ar[r]^{\bullet} & T \\
      B\times T \ar[ur]_{f} \ar[u]^{\name{f}\times1}
    }
  \]
  commutes.

  Let $T\times T\retractmaps{m}{r} T$. Then, consider
  \[
    \xymatrix@C+15pt{
     T\times T \ar[r]^{r\times 1} & T\times T\times T \ar[r]^{1\times m} &T\times T \ar[rr]^{\circ}
       && T \\
     T\times T\times T \ar[u]^{m\times 1} \ar@{=}[ur]\\
     T\times T \ar@{.>}[u]^{\<\code{(r f)},\pi_0,\pi_1\>} \ar@{.>}[uur]_{\<\code{(r f)},\pi_0,\pi_1\>}
       \ar[rr]_{m} \ar@{=}@/_20pt/[rrr]
       & & T \ar[r]_{r} \ar@{.>}[uu]_{\<\code{(r f)},1\>} &T\times T. \ar[uur]_{f}
    }
  \]
  \\[10pt]
  The rightmost quadrilateral commutes by assumption of this lemma. The middle quadrilateral
  commutes due to the properties of the product map and $\pi_0$ and $\pi_1$. The top left triangle
  commutes as $m r = 1$ and the remaining triangle has the same map on both dotted lines.

  Thus, we may conclude that $\bullet \definedas (r \times 1)(1\times m) \circ$ and
  $\name{f}\definedas \<!\code{f},1\> m$ satisfy the requirements of
  Theorem~\ref{thm:turing_recognition} and therefore $T$ is a Turing object in a Turing category.
\end{proof}
\section{Inverse Turing categories}
\label{sec:inverse_turing_categories}
Now, we define inverse Turing categories. The idea is that an inverse Turing category should be a discrete
inverse category \X such that $\Xt$ is a Turing category. A concrete description of this is
developed below.

\begin{definition}\label{def:inverse_turing_category}
  A discrete inverse category \X is an \emph{inverse Turing category} when there is a universal
  object $T$ (i.e., every $B\in\X_o$ is a retract of $T$) in
  \X with a map $\diamond :T\*T \to T\*T$ such that for every map $f:T \to T\*T$ there is a total map
  $\iname{f}: I\to T$ and a map $h_f:T\*T \to T\*T$ with $h_f \in \dmap{T}$ such that $f \xequiv{h_f}
  \inv{\usl}(\iname{f}\*1)\diamond$, i.e., the diagram
  \begin{equation}
    \xymatrix @C=15pt @R=15pt{
      & & T \* T \ar@{.>}[dddd]^{h_f}\\
      &T\*T \ar[ur]^{\diamond} & & \\
      T \ar[ur]^(.3){\inv{\usl}(\iname{f}\*1)} \ar[ddrr]_{f} \\
      & & & \\
      && T\*T
    }\label{dia:inverse-turing-category-code}
  \end{equation}
  commutes.
\end{definition}

First, we observe:

\begin{lemma}\label{lem:discrete_turing_category_inverses_make_inverse_turing_category}
  When $\T$ is a discrete Turing category then $\Inv{\T}$ is an inverse Turing category.
\end{lemma}
\begin{proof}
  By Lemma~\ref{lem:inv_x_is_a_discrete_inverse_category}, we know that $\Inv{\T}$ is a discrete
  inverse category. Thus, all that remains is to show:
  \begin{enumerate}[{(}i{)}]
    \item There is a map $\diamond:T\*T \to T\*T$ in $\Inv{\T}$;
    \item for a  map $f$ in $\Inv{\T}$, there is another map $\iname{f}$ which makes
      Diagram~\ref{dia:inverse-turing-category-code} commute.
  \end{enumerate}

  Given $f$ is invertible and we are in a Turing category, we know that we have the diagram
      \[
        \xymatrix{
          T\times T \ar[r]^{\circ} & T \\
          T \ar[ur]_{f} \ar[u]^{\<!\code{f},1\>}
        }
      \]
  in $\T$.   We know that the  map $\<\iname{f},1\>$ is invertible by
  \ref{lem:a_discrete_crc_is_precisely_a_graphic_crc} as we are in a discrete Cartesian restriction
  category.

  Expressing this in $\Inv{\T}$, for some $h'_f \in \dmap{T}$ we have:
  \begin{equation}
    \xymatrix @C=15pt @R=15pt{
      & & T \* T\*C \ar@{.>}[dddd]^{h'_f}\\
      &T\*T \ar[ur]^{\circ} & & \\
      T \ar[ur]^(.3){\inv{\usl}(\code{f}\*1)} \ar[ddrr]_{f} \\
      & & & \\
      && T\*D.
    }\label{dia:pre-inverse-turing}
  \end{equation}

  Recall from Lemma~\ref{lem:delta_nabla_maps_are_closed} that $\dmap{T}$ is closed under
  composition and that $1\*f\in \dmap{T}$ for any $f$. In particular, for $f:T\to T\*D$ in
  $\Inv{\T}$, as $D\retract T$, we have $f$ is in the same equivalence class as $f(1\*m_D)$. We see
  this as $\rst{f} = \rst{f(1\*m_D)}$, $1\*m_D\in \dmap{T}$ and the diagram
  \[
    \xymatrix@R-10pt{
      &T\*D \ar@{.>}[dd]^{1\*m_D}\\
      T \ar[ur]^{f} \ar[dr]_{f(1\*m_D)}& \\
      &T\*T
    }
  \]
  obviously commutes.

  We use this and the fact that we have both $T\*C \retract T$
  and $D \retract T$, to add to   Diagram~\ref{dia:pre-inverse-turing}:
  \[
    \xymatrix @C=15pt @R=15pt{
      & & & T \* T \ar@{.>}[d]^{1\*r_{T\*C}}\\
      & & T \* T \* C \ar@{.>}[dddd]^{h'_f} \ar[ur]^{1\*m_{T\*C}} \ar@{=}[r] &T\*T\*C \ar@{.>}[dddd]^{h'_f} \\
      &T\*T \ar[ur]^{\circ} & & \\
      T \ar[ur]^(.3){\inv{\usl}(\iname{f(1\*m_D)}\*1)} \ar[ddrr]_{f} \\
      & & & \\
      && T\*D \ar[dr]_{1\*m_D} &  T\*D \ar[d]^{1\*m_D}\\
      &&&T\*T.
    }
  \]
  But this is the required diagram for an inverse Turing category with $h_f = (1\*r_{T\*C}) h'_f (1\*m_D)$,
  and with $\iname{f(1\*m_D)} = \code{f}$ and  with $\diamond = \circ(1\* m_{T\*C})$. Therefore
  $\Inv{\T}$ is an inverse Turing category.
\end{proof}

We know that applying the Cartesian Completion to \X, an inverse Turing category, results in \Xt, a
discrete Cartesian restriction category. Moreover, if $A\retractmaps{m_A}{r_A} T$ in \X, then $A
\retractmaps{m_A\inv{\usr}}{r_A\inv{\usr}} T$ in \Xt and hence $T$ will remain universal in
\Xt. Hence, we have the basic requirements for a Turing category as specified in
Theorem~\ref{thm:turing_recognition} and Lemma~\ref{lem:t_t_to_t_gives_a_turing_category}. All that
remains to be shown is that we have a self-application map and a code for each map $f:1\to T$ as in
Lemma~\ref{lem:t_t_to_t_gives_a_turing_category}.

\begin{theorem}\label{thm:inverse_turing_category_gives_a_turing_category}
  When $\X$ is an inverse Turing category, $\Xt$ is a Turing category.
\end{theorem}
\begin{proof}
  From the discussion, we need to specify the self-application map $\circ:T\times T \to T$ and
  $\code{f}:1 \to T$ in \Xt.

  The diagram of Definition~\ref{def:inverse_turing_category}, when raised to $\Xt$
  translates to:
  \[
    \xymatrix@C+25pt@R+10pt{
      T \times T \ar[r]^{(\diamond, T)} &T \\
      T. \ar[u]^{\<\iname{f},1\>} \ar[ur]_{(f, T)}
    }
  \]
  But this corresponds exactly to the requirement of
  Lemma~\ref{lem:t_t_to_t_gives_a_turing_category} with $\circ = (\diamond,T)$ and $\code{(f,T)} =
  \<\name{f},1\>$.  Finally, noting that $T$ is universal in \X, if we have $(f,B):T\to T$ in \Xt,
  where $B\retractmaps{m_B}{r_B} T$ in \X, we recall that $(f,B) \xequiv{} (f(1\*m_B),T)$ in
  \X. Therefore, it may be  written as above and we therefore have shown that $\Xt$ is a Turing
  category.
\end{proof}

\section{Partial combinatory algebras}
\label{sec:partial_combinatory_algebras}

In a Cartesian restriction category, for any operation $f:A\times A\to A$ define $\multiapp{f}{n}$
for $n\ge 1$  recursively by:
\begin{enumerate}[{(}i{)}]
  \item $\multiapp{f}{1} = f$,
  \item $\multiapp{f}{n+1} = (f\times 1) \multiapp{f}{n}$.
\end{enumerate}

\begin{definition}\label{def:partial_combinatory_algebra}
  A Cartesian restriction category has a \emph{partial combinatory algebra} when it has an object
  $A$ together with:
  \begin{enumerate}[{(}i{)}]
  \item A partial map $\bullet:A\times A \to A$,\label{defitem:pca-1}
  \item two total elements $1\xrightarrow{k}A$ and $1\xrightarrow{s}{A}$ which satisfy\label{defitem:pca-2}
    \[
      \xymatrix@C+25pt{
        A\times A\times A \ar[r]^(.6){(\bullet\times 1)\bullet} & A\\
        A\times A \ar[u]^{k\times1\times1} \ar[ur]_{\pi_1}
      }\quad
      \xymatrix{
        A\times A\times A\times A \ar[r]^(.6){\multibullet{3}}&A\\
        &A\times A \ar[u]_{\bullet}\\
        A\times A\times A \ar[uu]^{s\times1\times1\times1} \ar[r]_(.4){\theta_A'}
          & (A\times A) \times (A\times A), \ar[u]_{\bullet\times\bullet}
      }
    \]
    \item $A\times A \xrightarrow{s\times1\times1} A\times A\times A \xrightarrow{\bullet^2} A$ is total.\label{defitem:pca-3}
  \end{enumerate}
  In the above $\theta' = (1\times1\times\Delta)(1\times c \times 1)a$ where $a$ sets the
  parenthesis as in the diagram.
\end{definition}

Of course, this is more familiarly given equationally by:
\[
   (k\bullet x)\bullet y = x \qquad ((s\bullet x)\bullet y) \bullet z = (x\bullet z) \bullet
   (y\bullet z).
\]
These are the equations of a combinatory algebra where partiality is not considered. As we have
partiality, we also add the requirement that $s\bullet x\bullet y$ is a total map for any $x,y$.

Note that if we have a Turing object $T$ in a Cartesian restriction category, it is a partial
combinatory algebra. All we need to do is to actually define the element $k$ and $s$ by using the
commuting diagrams of Definition~\ref{def:partial_combinatory_algebra}.

Now, we want to consider what are the conditions required for an inverse category \X such that $\Xt$
has a partial combinatory algebra.

In a discrete inverse category, we define the notation $\imultiapp{f}{n}$. For any operation $f:A\* A\to A\*A$
define $\imultiapp{f}{n}$ recursively by:
\begin{enumerate}[{(}i{)}]
\item
  $\imultiapp{f}{1}:A\*A \to A\*A =f =\ $\raisebox{-12pt}{
  \begin{tikzpicture}
    \node [style=nothing] (s1) at (0,1) {};
    \node [style=nothing] (s2) at (.5,1) {};
    \node [style=map] (bullet1) at (.25,.5) {$\scriptstyle f$};
    \node [style=nothing] (e1) at (0,0) {};
    \node [style=nothing] (e2) at (.5,0) {};
    \draw [] (s1) to[out=270,in=125] (bullet1);
    \draw [] (s2) to[out=270,in=55] (bullet1);
    \draw [] (bullet1) to[out=235,in=90] (e1);
    \draw [] (bullet1) to[out=305,in=90] (e2);
  \end{tikzpicture}}.
\item
  $\imultiapp{f}{n+1}:A\*(\*_{n}A)\*A \to A\*(\*_{n+1}A) =\ $\raisebox{-45pt}{
  \begin{tikzpicture}
    \node [style=nothing] (s1) at (-0.25,3.5) {};
    \node [style=nothing] (s1a) at (0.25,3.5) {};
    \node [style=nothing] (topdots) at (.5,3.5) {$\scriptstyle \cdots$};
    \node [style=nothing] (s2) at (.75,3.5) {};
    \node [style=nothing] (s3) at (1,3.5) {};
    \node [style=tensor] (t0) at (.5,3) {$\scriptstyle \*$};
    \node [style=map] (bullet1) at (.25,2.25) {$\scriptstyle \imultiapp{f}{n}$};
    \node [style=map] (bullet2) at (.25,1) {$\scriptstyle f$};
    \node [style=tensor] (t1) at (.75,.5) {$\scriptstyle \*$};
    \node [style=nothing] (e1) at (0,0) {};
    \node [style=nothing] (e2) at (.75,0) {};
    \draw [] (s1) to[out=270,in=125] (bullet1);
    \draw [] (s1a) to[out=270,in=125] (t0);
    \draw [] (s2) to[out=270,in=55] (t0);
    \draw [] (s3) to[out=270,in=55] (bullet2);
    \draw (t0) to[out=270,in=55] (bullet1);
    \draw [] (bullet1) to[out=235,in=125] (bullet2);
    \draw [] (bullet1) to[out=305,in=55] (t1);
    \draw [] (bullet2) to[out=305,in=125] (t1);
    \draw [] (bullet2) to[out=235,in=90] (e1);
    \draw (t1) to (e2);
  \end{tikzpicture}}.
\end{enumerate}

\begin{definition}\label{def:inverse_partial_combinatory_algebra}
  A discrete inverse category \X has an \emph{inverse partial combinatory algebra} when there is an
   object $A$ in \X with a map $A\*A  \xrightarrow{\bullet} A\*A$ and two total elements:
  \[
      1\xrightarrow{k}A \qquad 1\xrightarrow{s}{A}
  \]
  and maps
    $h_k:A\*A\*A\to A\*A, h_s:A\*A\*A\*A\to A\*A\*A\*A$  in $\dmap{A}$ which satisfy the following
    three axioms:\\
    \axiom{iCPA}{1}
    \[
      \xymatrix@C+15pt@R-10pt{
         && A\*A\*A \ar@{.>}[dddd]^{h_k}\\
        &A\* A\* A \ar[ur]^{\imultibullet{2}} \\
        A\*A \ar[ur]^(.4){\inv{\usl}(k\*1\*1)\ \ } \ar[ddrr]_{1} \\
         && & \\
        && A\*A.
      }
    \]
    \axiom{iCPA}{2}
    \[
      \xymatrix@R-10pt{
        &&& A\* A \*A\*A\ar@{.>}[dddddd]^{h_s} \\
        &&A\* A\* A\* A
          \ar[ur]^{\imultibullet{3}}\\
        && \\
        A\* A\* A \ar[uurr]^{\inv{\usl}(s\*1)} \ar[dr]_(.4){\theta_A'}\\
        &(A\* A) \* (A\* A) \ar[dr]_{(\bullet\*\bullet)(1\*c\*1)\ \ } \\
        & &A\* A\* A\*A\ar[dr]_{(\bullet\*1)}\\
        &&&A\*A\*A\*A.\\
      }
    \]
    \axiom{iCPA}{3} $I\*A\* A \xrightarrow{s\*1\*1} A\* A\* A \xrightarrow{\imultibullet{2}} A\*A\*A$ is total.
\end{definition}

\begin{proposition}\label{prop:inverse-pca-iff-pca}
  A discrete inverse category \X has an inverse partial combinatory algebra if and only if $\Xt$ has
  a partial combinatory algebra.
\end{proposition}
\begin{proof}
  When we have a discrete inverse category \X with an inverse partial combinatory algebra, we see
  immediately the map $\bullet:A\*A\to A\*A$ in \X becomes the map $(\bullet,A):A\times A \to A$,
  satisfying \ref{defitem:pca-1} of Definition~\ref{def:partial_combinatory_algebra}. The
  commutative diagrams \axiom{iCPA}{1} and \axiom{iCPA}{2}, when lifted to \Xt, become the diagrams
  for a partial combinatory algebra as given in \ref{defitem:pca-2}, where $(k\inv{\usl}, I)$ and
  $(s\inv{\usl}, I)$ are the $k,s$ of the partial combinatory algebra. Finally, the totality
  requirement, \axiom{iCPA}{3}, gives \ref{defitem:pca-3} of the partial combinatory algebra
  definition.

  Hence, we have shown that an inverse partial combinatory algebra in \X gives a partial combinatory
  algebra in \Xt.

  For the reverse, when we have a partial combinatory algebra over $A$ in $\Xt$, a discrete Cartesian
  restriction category, by Lemma~\ref{lem:a_discrete_crc_is_precisely_a_graphic_crc} we know that
  the map $\<\bullet,1\>$ is invertible and hence is in $\X$. The two maps $\<k,1\>$ and $\<s,1\>$
  are also invertible and therefore are in $\X$.

  Given this, the diagrams of the partial combinatory algebra in \Xt translate directly to the
  \axiom{iCPA}{1} and \axiom{iCPA}{2} where $\bullet$ in \X is the invertible map $\<\bullet,1\>$.

  The totality  of $s \multibullet{2}$ in \Xt then immediately gives us \axiom{iCPA}{3}, the
  totality of $s \imultibullet{2}$ in \X.


\end{proof}

However, we can simplify the definition of an inverse partial combinatory algebra when $A$ is
powerful in \X. (Here, powerful means that $1\retract A$, $A\*A\retract A$, $A\*A\*A\retract A$,
$\ldots$). Note that if $A$ is a partial combinatory algebra in $\Xt$, that guarantees it is
powerful in \Xt. Assuming the retractions are $A^n \retractmaps{m_n}{r_n} A$, we have $m_j r_j =
1$ and $r_j m_j = \rst{r_j m_j}$ for each $j$. Thus, each of the maps are partial inverses in
$\Xt$ and therefore in $\X$. Thus, $A$ is a powerful object in $\X$.


We redefine the meaning of the notation $\multiapp{f}{n}$ when in a discrete inverse
category.

In a discrete inverse category, we redefine the notation $\multiapp{f}{n}$. For any map $f:A\* A\to
A\*A$ where $A$ is a powerful object define $\multiapp{f}{n}$ recursively by:
\begin{enumerate}[{(}i{)}]
\item
  $\multiapp{f}{1}:A\*A \to A\*A =f =\
  \raisebox{-12pt}{\begin{tikzpicture}
    \node [style=nothing] (s1) at (0,1) {};
    \node [style=nothing] (s2) at (.5,1) {};
    \node [style=map] (bullet1) at (.25,.5) {$\scriptstyle f$};
    \node [style=nothing] (e1) at (0,0) {};
    \node [style=nothing] (e2) at (.5,0) {};
    \draw [] (s1) to[out=270,in=125] (bullet1);
    \draw [] (s2) to[out=270,in=55] (bullet1);
    \draw [] (bullet1) to[out=235,in=90] (e1);
    \draw [] (bullet1) to[out=305,in=90] (e2);
  \end{tikzpicture}}$;
\item
  $\multiapp{f}{n+1}:\*_{n+2}A \to A\*A =\
  \raisebox{-35pt}{\begin{tikzpicture}
    \node [style=nothing] (s1) at (0,2.5) {};
    \node [style=nothing] (s2) at (.5,2.5) {};
    \node [style=nothing] (s3) at (1,2.5) {};
    \node [style=map] (bullet1) at (.25,1.75) {$\scriptstyle \multiapp{f}{n}$};
    \node [style=map] (bullet2) at (.25,1) {$\scriptstyle f$};
    \node [style=map] (r1) at (.65,.4) {$\scriptstyle m_2$};
    \node [style=nothing] (e1) at (0,0) {};
    \node [style=nothing] (e2) at (.65,0) {};
    \draw [] (s1) to[out=270,in=125] (bullet1);
    \draw [] (s2) to[out=270,in=55] (bullet1);
    \draw [] (s3) to[out=270,in=55] (bullet2);
    \draw [] (bullet1) to[out=235,in=125] (bullet2);
    \draw [] (bullet1) to[out=305,in=55] (r1);
    \draw [] (bullet2) to[out=305,in=125] (r1);
    \draw [] (bullet2) to[out=235,in=90] (e1);
    \draw [] (r1) to[out=270,in=90] (e2);
  \end{tikzpicture}}$.
\end{enumerate}

\begin{lemma}\label{lem:powerful_inverse_partial_combinatory_algebra}
  Suppose a discrete inverse category \X has a inverse partial combinatory algebra over $A$ and $A$
  is  a powerful object in \X, with $\*^n A \retractmaps{m_n}{r_n} A$. Then \axiom{iCPA}{1},
  \axiom{iCPA}{2} and \axiom{iCPA}{3} may be simplified to:\\
  \axiom{iCPA$'$}{1}
    \[
      \xymatrix@C+15pt{
         && A\*A \ar@{.>}[dddd]^{h'_k}\\
        &A\* A\* A \ar[ur]^{\multibullet{2}} & & \\
        A\* A \ar[ur]^{\inv{\usl}(k\*1\*1)\ \ } \ar[ddrr]_{1} && &\\
         && & \\
        && A\*A. &
      }
    \]
  \axiom{iCPA$'$}{2}
    \[
      \xymatrix{
        &&& A\* A \ar@{.>}[dddddd]^{h'_s} \\
        & &  && \\
        &A\* A\* A\* A
          \ar[uurr]^{\multibullet{3}}\\
        A\* A\* A \ar[ur]^{\inv{\usl}(s\*1)\,} \ar[dr]_(.4){\theta_A'}\\
        &(A\* A) \* (A\* A) \ar[dr]|{(\bullet\*\bullet)(1\*c\*1)(1\*1\*m_2)} \\
        & &A\* A\* A\ar[dr]_{(\bullet\*1)(1\*m_2)\ }& &\\
        &&&A\*A. \\
      }
    \]
   \axiom{iCPA$'$}{3} $I\*A\* A \xrightarrow{s\*1\*1} A\* A\* A \xrightarrow{\multibullet{2}} A\*A$
   is total.
\end{lemma}

\section{Computable functions}
\label{sec:computable_functions}

Given a partial combinatory algebra, $A$, in a Cartesian restriction category, one can form \compa,
the category of computable partial functions generated by $A$. These are the maps with an index:
\[
  \xymatrix@C+25pt{
    A\times (\times_n A) \ar[r]^(.6){\multibullet{n}} & A \\
    (\times_n A). \ar[u]^{\name{f}\times 1} \ar[ur]_f
  }
\]
We would like \compa to be a discrete Turing category so that $\Inv{\compa}$ is an inverse Turing
category by
Lemma~\ref{lem:discrete_turing_category_inverses_make_inverse_turing_category}. Unfortunately, there
is no guarantee that \compa is a discrete Turing category.  However, we can define conditions
so that it is true:

\begin{definition}\label{def:discrete_pca}
  Given a discrete object $A$ in a Cartesian restriction category, $A$ has a \emph{discrete partial
    combinatory algebra} when:
  \begin{enumerate}[{(}i{)}]
  \item $A$ has a partial combinatory algebra;
  \item there exists $e:1\to A$, a total element, such that
    \[
      \xymatrix@C+25pt{
        A\times A \times A\ar[r]^(.6){\multibullet{2}} & A \\
        A\times A  \ar[u]^{e\times 1} \ar[ur]_{\inv{\Delta}}
      }
    \]
    meaning there is a code for $\inv{\Delta}$.
  \end{enumerate}
\end{definition}

We immediately have:
\begin{lemma}\label{lem:comp_a_is_discrete_cart}
  When $A$ has a discrete partial combinatory algebra, then \compa is a discrete Cartesian
  restriction category.
\end{lemma}
By Lemma~\ref{lem:discrete_turing_category_inverses_make_inverse_turing_category}, this means that
$\Inv{\compa}$ is an inverse Turing category. Note it is still the case that there can be a map of
\compa which is invertible in \Xt (i.e., is in \X), but is \emph{not} invertible in \compa.

% chapter turing_categories (end)

%%% Local Variables:
%%% mode: latex
%%% TeX-master: "../phd-thesis"
%%% End:

%%% Local Variables:
%%% mode: latex
%%% TeX-master: "../phd-thesis"
%%% End:
