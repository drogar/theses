%!TEX root = /Users/gilesb/UofC/thesis/phd-thesis/phd-thesis.tex
\chapter{Disjointness in Inverse Categories} % (fold)
\label{cha:disjointness_in_inverse_categories}

In this chapter we will first explore coproducts in restrictions categories, showing, similar to
product that this is too restrictive a notion for inverse categories. We will then turn our
attention to defining a ``disjointness'' relation between parallel maps in an inverse category and a
``disjoint join'' of disjoint maps. We will conclude the chapter showing how a tensor with specific
conditions can give rise to both a disjointness relation and a disjoint join.



\section{Coproducts in restriction categories} % (fold)
\label{sec:coproducts_in_restriction_categories}
\subsection{Restriction Coproducts} % (fold)
\label{sub:restriction_coproducts}

Restriction categories may also have coproducts and initial objects.
\begin{definition}
  In a restriction category \X, a coproduct is a \emph{restriction coproduct} when the embeddings
  $\cpa$ and $\cpb$ are total.
\end{definition}

\begin{lemma}\label{lem:restriction_coproduct_is_restriction_functor}
  The  coproduct in a restriction category satisfies:
  \begin{enumerate}[{(}i{)}]
    \item $\restr{f+g} = \restr{f} + \restr{g}$ which means $+$ is a restriction functor.
    \item $\nabla:A+A\to A$ is total.
    \item $?:0 \to A$ is total, where $0$ is the initial object in the restriction category.
  \end{enumerate}
\end{lemma}
\begin{proof}
  \prepprooflist
  \begin{enumerate}[{(}i{)}]
    \item \textbf{$+$ is a restriction functor.}
      Consider the diagram:
      \[
        \xymatrix @C=30pt @R=35pt{
          A \ar[rr]_{f} \ar[dr]_{\cpa} && A' \ar[dr]^{\cpa'}\\
          &A+B \ar@{.>}[rr]^{f+g} & & A' + B'\\
          B \ar[rr]_{g} \ar[ur]^{\cpb} && B'. \ar[ur]_{\cpb'}
        }
      \]
      In order to show $\restr{f+g} = \restr{f} + \restr{g}$, it suffices to show that
      $\cpa\restr{f+g} = \cpa(\restr{f} + \restr{g}) = \restr{f}\cpa$.
      \begin{align*}
        \cpa\restr{f+g} &= \restr{\cpa(f+g)}\cpa & \rfour\\
        &= \restr{f \cpa'} \cpa &\text{coproduct diagram}\\
        &= \restr{f \restr{\cpa'}} \cpa
          & \text{Lemma }\refitem{lem:restrictionvarious}{lemitem:rv_3}\\
        &= \restr{f} \cpa & \cpa'\text{ total}.
      \end{align*}
    \item   \textbf{$\nabla:A+A\to A$ is total.}
      By the definition of $\nabla$ ($=\<1 | 1 \>$) and the co-product, the following diagram
      commutes,
      \[
        \xymatrix @C=30pt @R=35pt{
          &A + A \ar[d]_{\nabla}\\
          A \ar@{=}[r] \ar[ur]^{\cpa}&A \ar@{=}[r] & A \ar[ul]_{\cpb}
        }
      \]
      resulting in:
      \[
         \cpa \restr{\nabla}  = \restr{\cpa \nabla} \cpa = \restr{1} \cpa = \cpa.
      \]
      Similarly, $\cpb \restr{\nabla} = \cpb$, hence, the restriction of $\nabla$ is $1$ and
      therefore $\nabla$ is total.
    \item  \textbf{$\why:0 \to A$ is total.}
      This follows from
      \[
        \xymatrix @C=30pt @R=35pt{
          0 \ar[dr]_{\why} \ar[r]^{\cpb} &A + 0 \\
          & A \ar@{=}[u]
        }
      \]
      so $\why$ can be defined as the total coproduct injection.
  \end{enumerate}



\end{proof}

Recall that when an object is both initial and terminal, it is referred to as a zero object and
denoted as $0$. This gives rise to the zero map $0_{A,B}: A \to 0 \to B$ between any two objects.

\begin{definition}\label{def:restriction_zero}
  Given a restriction category \X with a zero object, then $0$ is a \emph{restriction zero} when
  for each object $A$ in \X we have $\restr{0_{A,A}} = 0_{A,A}$.
\end{definition}

\begin{lemma}[Cockett-Lack]\label{lem:restriction_zero_equivalences}
  For a restriction category \X, the following are equivalent:
  \begin{enumerate}[{(}i{)}]
    \item \X has a restriction zero;
    \item \X has an initial object $0$ and terminal object $1$ and each initial map $z_A$ is a
      restriction monic;
    \item \X has a terminal object $1$ and each terminal map $t_A$ is a restriction retraction.
  \end{enumerate}
\end{lemma}
% subsection coproducts (end)




\subsection{Inverse categories with restriction coproducts} % (fold)
\label{sub:inverse_categories_with_restriction_coproducts}


\begin{proposition}\label{prop:inverse_category_with_coproducts_is_pre-order}
  An inverse category \X with restriction coproducts is a pre-order.
\end{proposition}
\begin{proof}
  By Lemma~\ref{lem:restriction_coproduct_is_restriction_functor}, we know $\nabla$ is total and
  therefore $\nabla\inv{\nabla} = 1$. From the coproduct diagrams, we have $\cpa\nabla =1$ and
  $\cpb\nabla = 1$. But this gives us $\inv{\nabla}\inv{\cpa} = \inv{(\cpa\nabla)} = 1$ and
  similarly $\inv{\nabla}\inv{\cpb} =1$. Hence, $\inv{\nabla} = \cpa$ \emph{and} $\inv{\nabla} =
  \cpb$.

  This means for parallel maps $f,g:A \to B$, we have
  \[
    f = \cpa \<f|g\> = \inv{\nabla} \<f|g\> = \cpb\<f|g\> = g
  \]
  and therefore \X is a pre-order.
\end{proof}
% subsection inverse_categories_with_restriction_coproducts (end)
% section coproducts_in_restriction_categories (end)

\section{Disjointness in an inverse category} % (fold)
\label{sec:disjointness_in_an_inverse_category}
In the following, we will add two related structures to an inverse category with a restriction
zero and zero maps. This structure is meant to be evocative of the \emph{join}\cite{guox:thesis}
in a  restriction category. We will first re-iterate some of the basic definitions and
lemmas about joins.

\subsection{Disjointness relations} % (fold)
\label{sub:disjointness_relations}
In this subsection, we will define a disjointness relationship between maps and
explore alternate characterizations of this relation on the restriction idempotents of objects.

From this point forward in the thesis, we will work with a number of relations and operations on
parallel pairs of maps. Suppose we have a relation $\lozenge$ between maps $f,g:B\to C$, i.e., $f
\lozenge g$. We will refer to $\lozenge$ as \emph{stable} whenever given a $h:A \to B$, then $h f
\lozenge h g$. We will refer to $\lozenge$ as \emph{universal} whenever given a $k:C \to D$, then $f
k \lozenge g k$.


\begin{definition}\label{def:disjointness_relation}
  In an inverse category \X with zero maps, the relation $\cdperp$ between two parallel
  maps $f, g:A \to B$ is called a \emph{disjointness relation} when it satisfies the following
  properties:
  \begin{align*}
    \axiom{Dis}{1}\quad &\text{ For all }f:A\to B,\ f\cdperp 0;\\
    \axiom{Dis}{2}\quad &f\cdperp g\text{ implies }\rst{f} g = 0; \\
    \axiom{Dis}{3}\quad &f\cdperp g,\ f' \le f,\ g' \le g\text{ implies }f' \cdperp g';\\
    \axiom{Dis}{4}\quad &f\cdperp g\text{ implies }g \cdperp f; \text{ (Symmetric)}\\
    \axiom{Dis}{5}\quad &f\cdperp g\text{ implies }h f \cdperp h g; \text{ (Stable)}\\
    \axiom{Dis}{6}\quad &f\cdperp g\text{ implies }\rst{f} \cdperp \rst{g}
      \text{ and }\rg{f}\cdperp\rg{g}; \text{ (Closed under range and restriction)}\\
    \axiom{Dis}{7}\quad &\rst{f}\cdperp \rst{g},\ \rg{h}\cdperp \rg{k}\text{ implies }f h
      \cdperp g k.\text{ (Determined by restriction/range.)}
  \end{align*}
  When $f\cdperp g$, we will say $f$ is \emph{disjoint} from $g$.
\end{definition}

\begin{lemma}\label{lem:disjointness_equivalent_axioms}
  In Definition~\ref{def:disjointness_relation}, provided we retain $\axiom{Dis}{1-5}$, we may
  replace $\axiom{Dis}{6}$ and $\axiom{Dis}{7}$ by:
  \begin{align*}
    \axiom{Dis}{6'}\quad &f\cdperp g\text{ if and only if }\rst{f} \cdperp \rst{g}
      \text{ and }\rg{f}\cdperp\rg{g}.
  \end{align*}
\end{lemma}
\begin{proof}
  Given $\axiom{Dis}{6}$ and $\axiom{Dis}{7}$, the \emph{only if} direction of $\axiom{Dis}{6'}$ is
  immediate. To show the \emph{if} direction, assume $\rst{f} \cdperp \rst{g}$ and
  $\rg{f}\cdperp\rg{g}$. This also means that $\rst{\rst{f}} \cdperp \rst{\rst{g}}$. Then, by
  \axiom{Dis}{7}, $\rst{f} f \cdperp \rst{g}g$ and therefore $f \cdperp g$.

  Conversely, assume we are given \axiom{Dis}{6'}. Then, \axiom{Dis}{6} follows immediately. To show
  \axiom{Dis}{7}, assume we have $\rst{f}\cdperp \rst{g},\ \rg{h}\cdperp \rg{k}$. As
  $\rst{f h} \le \rst{f}$ and $\rst{g k}\le\rst{g}$, by \axiom{Dis}{3}, we know that
  $\rst{f h} \cdperp \rst{g k}$. Similarly, $\wrg{f h} \le \rg{h}$ and $\wrg{g k} \le \rg{k}$,
  giving us $\wrg{f h}\cdperp \wrg{g k}$. Then, from \axiom{Dis}{6'} we may conclude
  $f h \cdperp g k$, showing \axiom{Dis}{7} holds.
\end{proof}
\begin{lemma}\label{lem:disjointness_various}
  In an inverse category \X with $\cdperp$ a disjointness relation:
  \begin{enumerate}[{(}i{)}]
    \item $f \cdperp g$ if and only if $\inv{f}\cdperp \inv{g}$; \label{lemitem:djv_inverses}
    \item $f \cdperp g$ implies $f h \cdperp g h$ (Universal);\label{lemitem:djv_universal}
    \item $f \cdperp g$ implies $f\rg{g} = 0$; \label{lemitem:djv_disjoint_composition_is_0}
    \item if $m,n$ are monic, then $f m \cdperp g n$ implies $\rst{f} \perp \rst{g}$;
      \label{lemitem:djv_monic_implies}
    \item if $m,n$ are monic, then $\inv{m} f \cdperp \inv{n}g$ implies $\rg{f} \perp \rg{g}$.
      \label{lemitem:djv_inv_monic_implies}
  \end{enumerate}
\end{lemma}
\begin{proof}
  \prepprooflist
  \begin{enumerate}[{(}i{)}]
    \item Assume $f \cdperp g$. By \axiom{Dis}{6}, we have $\rst{f} \cdperp \rst{g}$ and
      $\rg{f}\cdperp\rg{g}$. Since $\rg{f} = \rst{\inv{f}}$ and $\rst{f} = \wrg{\inv{f}}$, this
      means $\rst{\inv{f}} \cdperp \rst{\inv{g}}$ and $\wrg{\inv{f}}\cdperp\wrg{\inv{g}}$. By
      \axiom{Dis}{6'} from Lemma~\ref{lem:disjointness_equivalent_axioms}, we have
      $\inv{f} \cdperp \inv{g}$. The converse follows with  a similar argument.
    \item Assume $f \cdperp g$. By the previous item, we have $\inv{f}\cdperp\inv{g}$. By
      \axiom{Dis}{5}, $\inv{h}\inv{f}\cdperp\inv{h}\inv{g}$, giving us $\inv{(f h)} \cdperp
      \inv{(g h)}$. By \ref{lemitem:djv_inverses}, we now have $f h \cdperp g h$.
    \item Assume $f \cdperp g$. From \ref{lemitem:djv_inverses} and reflexivity, we know that
      $\inv{g} \cdperp \inv{f}$ and therefore $\rst{\inv{g}}\inv{f} = \rg{g}\inv{f}= 0$. However, in
      an inverse category, $\inv{0} = 0$ and therefore $0 = \inv{(\rg{g}\inv{f})} = f \inv{\rg{g}} =
      f \rg{g}$.
    \item Assume $f m \cdperp g n$ where $m, n$ are monic. By \axiom{Dis}{6}, this gives us
      $\rst{f m} \cdperp \rst{g n}$. By Lemma~\ref{lem:restrictionvarious},
      $\rst{f m} = \rst{f \rst{m}} = \rst{f 1} = \rst{f}$ and therefore $\rst{f} \cdperp \rst{g}$.
    \item By assumption, we have $\inv{m} f \cdperp  \inv{n}g$ and therefore $\inv{f}m \cdperp
      \inv{g} n$. By \ref{lemitem:djv_monic_implies}, this means $\rst{\inv{f}} \cdperp
      \rst{\inv{g}}$ and hence $\rg{f} \cdperp \rg{g}$.
  \end{enumerate}
\end{proof}

We may define the disjointness relation via its action in \open{a}.

\begin{definition}\label{def:disjointness_in_open_x}
  Given an inverse category \X, a relation $\ocdperpsub{A} \subseteq \open{A}^2$ for each $
  A \in \mathrm{ob}(\X)$, is an \emph{open disjointness} relation when for all $e, e' \in \open{A}$
  \begin{align*}
    \axiom{$\mathcal{O}$dis}{1}\quad &1 \ocdperpsub{A} 0; \\
    \axiom{$\mathcal{O}$dis}{2}\quad &e \ocdperpsub{A} e' \text{ implies }e' \ocdperpsub{A} e;\\
    \axiom{$\mathcal{O}$dis}{3}\quad &e \ocdperpsub{A} e' \text{ implies }e e' = 0;\\
    \axiom{$\mathcal{O}$dis}{4}\quad &e \ocdperpsub{A} e' \text{ implies }\rst{f e} \ocdperpsub{B}
      \rst{f e'}\text{ for all }f:B \to A;\\
    \axiom{$\mathcal{O}$dis}{5}\quad &e \ocdperpsub{A} e' \text{ implies }\wrg{e g} \ocdperpsub{C}
      \wrg{e' g}\text{ for all }g:A \to C; \\
    \axiom{$\mathcal{O}$dis}{6}\quad &e \ocdperpsub{A} e',\ e_1 \le e,\ e_1' \le e'
      \text{ implies }e_1 \ocdperpsub{A} e_1'.
  \end{align*}
\end{definition}

We will normally write $\ocdperp$ rather than $\ocdperpsub{A}$ where the object is clear.

\begin{proposition}\label{prop:disjointness_is_open_disjointness}
  If $\cdperp$ is a disjointness relation in $\X$, it is an open disjointness
  relation on the restriction idempotents.
\end{proposition}
\begin{proof}
  \prepprooflist
  \setlist[enumerate,1]{leftmargin=1.75cm}
  \begin{enumerate}
    \item[\axiom{$\mathcal{O}$dis}{1}] This follows immediately from \axiom{Dis}{1} by taking
      $f = 1$.
    \item[\axiom{$\mathcal{O}$dis}{2}] Reflexivity follows directly from  \axiom{Dis}{4}.
    \item[\axiom{$\mathcal{O}$dis}{3}] By \axiom{Dis}{2}, $0 = \rst{e} e' = e e'$.
    \item[\axiom{$\mathcal{O}$dis}{4}] Given $e \cdperp e'$, we have $f e \cdperp f e'$ by
      \axiom{Dis}{5}. Then, by \axiom{Dis}{6} we may conclude $\rst{f e} \cdperp \rst{f e'}$.
    \item[\axiom{$\mathcal{O}$dis}{5}] This follows from the above item, using $\inv{g}$ for $f$.
      This means we have $\rst{\inv{g}e} \cdperp \rst{\inv{g}e'}$. But this gives us $\rst{\inv{(e
      g)}} \cdperp \rst{\inv{(e' g)}}$. Recalling from
      Lemma~\ref{lem:inverse_categories_are_range_categories} that $\rg{k} = \rst{\inv{k}}$, we may
      conclude $\wrg{e g} \cdperp \wrg{e' g}$.
    \item[\axiom{$\mathcal{O}$dis}{6}] Assuming $e \cdperp e'$ and $e_1 \le e,\ e'_1 \le e'$, by
      \axiom{Dis}{3}, $e_1 \cdperp e'_1$.
  \end{enumerate}
  Therefore, $\cdperp$ acts as an open disjointness relation on $\open{A}^2$.

\end{proof}

\begin{definition}\label{def:extended_open_disjointness}
  If $\ocdperp$ is an open disjointness relation in $\X$, then we may define a relation
  $\perpab \in \X(A,B)^2$ by
  \[
    \infer{f\perpab g}{f,g:A\to B,\ \rst{f}\ocdperp\rst{g},\ \rg{f}\ocdperp\rg{g}}.
  \]
  We call $\perpab$ the disjointness relation generated by $\ocdperp$.
\end{definition}

\begin{proposition}\label{prop:extended_disjointness_is_a_disjointness_relation}
  If $\cdperp$ is the disjointness relation generated by $\ocdperp$ in \X, then
  $\cdperp$ is a disjointness relation in \X.
\end{proposition}
\begin{proof}
  \prepprooflist
  \setlist[enumerate,1]{leftmargin=1.5cm}
  \begin{enumerate}
    \item[\axiom{Dis}{1}] We need to show $f \cdperp 0$ for any $f$. We know that $1 \ocdperp 0$
      and therefore $\rst{f} \ocdperp 0$ and $\rg{f} \ocdperp 0$, as $\rst{f}\le 1$ and $\rg{f}\le
      1$. This gives us $f \cdperp 0$.
    \item[\axiom{Dis}{2}] Assume $f \cdperp g$, i.e., $\rst{f}\ocdperp \rst{g}$. Then, $\rst{f}g =
      \rst{f}\rst{g}g = 0 g = 0$.
    \item[\axiom{Dis}{3}] We are given $f \cdperp g$, $f' \le f$ and $g' \le g$. By
      Lemma~\refitem{lem:restriction_cats_are_partial_order_enriched}{lemitem:rst_ordering_2}
      $\rst{f'} \le \rst{f}$ and $\rst{g'} \le \rst{g}$. Then, by \axiom{$\mathcal{O}$dis}{6}, as
      $\rst{f} \ocdperp \rst{g}$ we have $\rst{f'} \ocdperp \rst{g'}$. By
      Lemma~\refitem{lem:ordering_of_restriction_and_range}{lemitem:ordering_2}, we have $\wrg{f'}
      \le \rg{f}$ and $\wrg{g'} \le \rg{g}$. Then, by \axiom{$\mathcal{O}$dis}{6}, as $\rg{f}
      \ocdperp \rg{g}$ we have $\wrg{f'} \ocdperp \wrg{g'}$. This means $f' \cdperp g'$.
    \item[\axiom{Dis}{4}] Reflexivity of $\cdperp$ follows immediately from the reflexivity of
      $\ocdperp$.
    \item[\axiom{Dis}{5}] Assume $f \cdperp g$, i.e., $\rst{f}\ocdperp \rst{g}$ and $\rg{f}
      \ocdperp \rg{g}$. Then we have $\rst{hf}\ocdperp \rst{h g}$ by \axiom{$\mathcal{O}$dis}{4}. By
      Lemma~\refitem{lem:ordering_of_restriction_and_range}{lemitem:ordering_1} we have $\wrg{h f}
      \le \rg{f}$ and $\wrg{h g} \le \rg{g}$. Therefore we have $\wrg{h f}\ocdperp \wrg{h g}$ by
      \axiom{$\mathcal{O}$dis}{6} and therefore $h f \cdperp h g$.
    \item[\axiom{Dis}{6}] This follows directly from definition
      \ref{def:extended_open_disjointness}.
    \item[\axiom{Dis}{7}] We assume $\rst{f} \cdperp \rst{g}$ and $\rg{h} \cdperp \rg{k}$. By
      definition \ref{def:extended_open_disjointness} we have $\rst{f} \ocdperp \rst{g}$ and
      $\rg{h} \ocdperp \rg{k}$. By
      Lemma~\refitem{lem:restriction_cats_are_partial_order_enriched}{lemitem:rst_ordering_3}, we
      have $\rst{f h} \le \rst{f}$ and $\rst{g k} \le \rst{g}$. Therefore, $\rst{f h} \ocdperp
      \rst{g k}$ by \axiom{$\mathcal{O}$dis}{6}. By
      Lemma~\refitem{lem:ordering_of_restriction_and_range}{lemitem:ordering_1}, $\wrg{f h} \le
      \rg{h}$ and $\wrg{g k} \le \rg{k}$, giving us $\wrg{f h}\ocdperp \wrg{g k}$ also by
      \axiom{$\mathcal{O}$dis}{6}. This means $f h \cdperp g k$.
  \end{enumerate}
\end{proof}
\begin{theorem}\label{thm:open_disjointness_is_disjointness}
   To give a disjointness relation $\perp$ on $\X$ is to give an open disjointness relation
   $\ocdperp$ on \X.
\end{theorem}
\begin{proof}
  Suppose we are given the disjointness relation $\perp$. By
  Proposition~\ref{prop:disjointness_is_open_disjointness}, this is an open disjointness relation
  on each of the sets of idempotents, $\open{A}$. We will label that relation $\ocdperp$.

  Use Definition~\ref{def:extended_open_disjointness} to create an extended
  disjointness relation based on $\ocdperp$, signify it by $\eocdperp$.
  By Proposition~\ref{prop:extended_disjointness_is_a_disjointness_relation}, $\eocdperp$ is
  a disjointness relation on $\X$.

  Assume $f\perp g$. We have $\rst{f}\ocdperp\rst{g}$ and $\rg{f}\ocdperp\rg{g}$ by
  \axiom{Dis}{6} and Proposition~\ref{prop:disjointness_is_open_disjointness}. Then, from
  Definition~\ref{def:extended_open_disjointness}, we have $f \eocdperp g$.

  Assume $f \eocdperp g$. This means we must have had $\rst{f}\ocdperp\rst{g}$ and
  $\rg{f}\ocdperp\rg{g}$
  by Definition~\ref{def:extended_open_disjointness} and therefore $\rst{f}\perp\rst{g}$ and
  $\rg{f}\perp\rg{g}$. By Proposition~\ref{lem:disjointness_various}, we have $f \perp g$.


  Now, suppose we are given the open disjointness relation $\ocdperp$. Similar to above, we can
  construct the extended disjointness relation $\perp$ by
  Definition~\ref{def:extended_open_disjointness}. From the disjointness relation $\perp$, we have
  the open disjointness relation $\overline{\perp}$ by
  Lemma~\ref{prop:disjointness_is_open_disjointness}.

  Assume $e \ocdperp e'$. As this means both $\rst{e} \ocdperp \rst{e'}$ and
  $\rg{e} \ocdperp \wrg{e'}$, we have $e \perp e'$. By
  Proposition~\ref{prop:disjointness_is_open_disjointness} this means $e \overline{\perp} e'$.

  If we are given that $e \overline{\perp} e'$, then we know that $e \perp e'$ by
  Proposition~\ref{prop:disjointness_is_open_disjointness}. From
  Definition~\ref{def:extended_open_disjointness}, this requires that $\rst{e} \ocdperp \rst{e'}$
  and $\rg{e} \ocdperp \wrg{e'}$, but that just means $e \ocdperp e'$.
\end{proof}

Note that while we have worked with binary disjointness throughout this section, one may extend
the concept to lists of maps simply by considering disjointness pairwise. I.e., we have
$\cdperp [f_1,f_2,\ldots,f_n]$ if and only if $f_i \cdperp f_k$ whenever $i \ne j$.

Disjointness is additional structure on a restriction category, i.e., it is possible to have more
than one disjointness relation on the category.
\begin{example}[\pinj has a disjointness relation]\label{ex:pinj_has_a_disjointness_relation}
  Consider the inverse category \pinj, introduced in Example~\ref{ex:pinj_is_a_restriction_category}
  and Example~\ref{ex:pinj_is_a_discrete_inverse_category}.
  We note the restriction zero is the empty set, $\emptyset$ with initial and terminal maps being
  $\emptyset$ and therefore $0_{A,B} = \emptyset$.

  We may define the disjointness relation $\perp$ by $f \perp g$ if and only if
  $\rst{f}\intersection\rst{g}=\emptyset$ and $\rg{f}\intersection\rg{g} = \emptyset$. It is then
  reasonably straightforward to verify \axiom{Dis}{1} through \axiom{Dis}{7}. For example, take
  \axiom{Dis}{7}:
  \begin{proof}
    We are given $\rst{f}\cdperp\rst{g}$ and $\rg{h}\cdperp\rg{k}$. This means
    \[
      \rst{f}\intersection\rst{g}=\emptyset \text{ and }\rg{h}\intersection\rg{k} = \emptyset.
    \]
    As discussed in Example~\ref{ex:pinj_has_meets}, we have
    $\rst{m n} \subseteq \rst{m}$ and that $\wrg{m n} \subseteq \rg{n}$. Hence we have
    \begin{align*}
      \rst{f h}\intersection\rst{g k}&\subseteq \rst{f}\intersection \rst{g}= \emptyset\\
      \wrg{f h}\intersection\wrg{g k}&\subseteq \rg{h}\intersection \rg{k}= \emptyset.
    \end{align*}
    Therefore, $f h \cdperp g k$.
  \end{proof}

  We may define a different disjointness relation, $\perp'$, on the same restriction category.
  Define $f \perp' g$ if and only if one of $f$ or $g$ is the restriction 0, $\emptyset$. As
  $0 = \rst{0} = \rg{0} = h 0 = 0 k$, all of the seven disjointness axioms are easily verifiable.
\end{example}

Although disjointness is additional structure on a restriction category, one can use the
disjointness structure of a base category (or categories) to define a disjointness structure
on derived categories, such as the product category.

\begin{lemma}\label{lem:disjointness_is_derivable_on_a_product_category}
  If $\X$ and $\Y$ are inverse categories with restriction zeros and respective disjointness
  relations $\perp$ and $\perp'$, then we may construct a disjointness relation
  $\perp_{\times}$ on $\X\times\Y$.
\end{lemma}
\begin{proof}
  Recall that product categories are defined component-wise. These definitions extend to the
  restriction, the inverse and the restriction zero. That is:
  \begin{itemize}
    \item If $(f,g)$ is a map in $\X\times\Y$, then $\inv{(f,g)} = (\inv{f}, \inv{g})$;
    \item If $(f,g)$ is a map in $\X\times\Y$, then $\rst{(f,g)} = (\rst{f}, \rst{g})$;
    \item The map $(0_X,0_Y)$ is the restriction zero in $\X\times\Y$.
  \end{itemize}

  Following this pattern, for $(f,g)$ and $(h,k)$ maps in $\X\times\Y$, $(f,g) \perp_{\times}(h,k)$
  iff $f\perp h$ and $g\perp' k$.

  Verifying the disjointness axioms is straightforward, we show axioms 2 and 5. Proofs of the
  others are similar.
  {
  \setlist[itemize,1]{leftmargin=1.5cm}
  \begin{itemize}
    \item [\axiom{Dis}{2}]: Given $(f,g)\perp_{\times}(h,k)$, we have
      $\rst{(f,g)}(h,k) =  (\rst{f},\rst{g})(h,k) = (\rst{f} h, \rst{g} k) = (0, 0) = 0$.
    \item [\axiom{Dis}{5}]: We are given $(f,g)\perp_{\times}(h,k)$. Consider the map $z = (x,y)$
      in $\X \times \Y$. We know that $x f \perp x h$ and $y g \perp y k$, therefore we have
      $z(f,g) = (x f, y g) \perp_{\times} (x h, y k) = z(h,k)$.
  \end{itemize}
  }
\end{proof}
% subsection disjointness_relations (end)

\section{Disjoint joins} % (fold)
\label{sec:disjoint_joins}

We now consider additional structure on the inverse category, dependent upon the disjointness
relation.
\begin{definition}\label{def:disjoint_join}
  An \emph{inverse category with disjoint joins} is an inverse category $\X$, with a restriction
  $0$, a disjointness relation $\perp$ and a binary operator on disjoint parallel maps:
  \[
    \infer{f \djoin g : A \to B}{f: A\to B,\ g: A\to B,\ f \perp g}
  \]
  where the following hold:
  \begin{align*}
    \axiom{DJ}{1}\quad & f \le f \djoin g\text{ and }g \le f \djoin g; \\
    \axiom{DJ}{2}\quad & f \le h,\ g \le h\text{ and }f\perp g\text{ implies }f \djoin g \le h;\\
    \axiom{DJ}{3}\quad & h(f \djoin g) = h f \djoin h g. \text{ (Stable)} \\
    \axiom{DJ}{4}\quad & \cdperp [f, g, h]\text{ if and only if }f \perp (g\djoin h).\\
  \end{align*}
  The binary operator, $\djoin$, is referred to as the \emph{disjoint join}.
\end{definition}

Note that \axiom{DJ}{1} with \axiom{DJ}{2} immediately gives us that there is only one disjoint
join given a specific disjointness relation.
\begin{lemma}\label{lem:disjoint_join_is_unique}
  Suppose \X in an inverse category with disjoint joins, with the join $\djoin$ and that it
  has a second disjoint join, $\altjoin$. Then $f \djoin g = f \altjoin g$ for all maps $f,g$ in
  \X.
\end{lemma}
\begin{proof}
  The first axiom tells us:
  \[
    f,g \le f\djoin g\text{ and }f,g \le f \altjoin g.
  \]
  Using the second axiom, we may therefore conclude $f \djoin g \le f \altjoin g$ and
  $f \altjoin g \le f\djoin g$, hence $f \djoin g = f \altjoin g$.
\end{proof}
\begin{lemma}\label{lem:join_is_associative_and_commutative_monoid}
  In an inverse category with disjoint joins, the disjoint join respects the restriction and is
  universal. Additionally, it is a partial associative and
  commutative operation, with identity $0$. That is, the following hold:
  \begin{enumerate}[{(}i{)}]
    \item $\rst{f\djoin g} = \rst{f} \djoin \rst{g}$;
    \item $(f \djoin g)k = f k \djoin g k$ (Universal);
    \item $f \perp g,\ g\perp h,\ f\perp h$ implies that $(f\djoin g)\djoin h = f\djoin(g\djoin
      h)$; \label{lemitem:associative_join}
    \item $f \perp g$ implies $f \djoin g = g \djoin f$; \label{lemitem:commutative_join}
    \item $f \djoin 0 = f$. \label{lemitem:identity_for_join}
  \end{enumerate}
\end{lemma}
\begin{proof}
  \prepprooflist
  \begin{enumerate}[{(}i{)}]
    \item As $\rst{f}, \rst{g} \le \rst{f\djoin g}$, we immediately have
      $\rst{f} \djoin \rst{g} \le \rst{f \djoin g}$. To show the other direction, consider
      \begin{align*}
        \rst{f}(\rst{f} \djoin \rst{g}) (f \djoin g)
        &= (\rst{f}\,\rst{f} \djoin \rst{f}\rst{g})(f \djoin g) &\text{\axiom{DJ}{3}}\\
        &= \rst{f} (f\djoin g) &\text{Lemma~\ref{lem:restrictionvarious}, \axiom{Dis}{2}}\\
        &= f.
      \end{align*}
      Hence, we have $f \le (\rst{f} \djoin \rst{g}) (f \djoin g)$ and similarly, so is $g$. By
      \axiom{DJ}{2} and that $\rst{f} \djoin \rst{g}$ is a restriction idempotent, we then have
      \[
        f \djoin g \le (\rst{f} \djoin \rst{g}) (f \djoin g) \le f \djoin g
      \]
      and therefore $f \djoin g = (\rst{f} \djoin \rst{g}) (f \djoin g)$. By
      Lemma~\ref{lem:restriction_cats_are_partial_order_enriched}, $\rst{f\djoin g} \le \rst{f}
      \djoin \rst{g}$ and so $\rst{f\djoin g} = \rst{f}\djoin \rst{g}$.
    \item First consider when $f, g$ and $k$ are restriction idempotents, say $e_0, e_1 $ and $e_2$.
      Then, we have $(e_0 \djoin e_1)e_2 = e_2(e_0 \djoin e_1) = e_2 e_0 \djoin e_2 e_1 =
      e_0 e_2 \djoin e_1 e_2$.
      Next, note that for general $f,g,h$, we have $f k \djoin g k \le (f\djoin g) k$ as both
      $f k, g k \le (f\djoin g) k$.
      By Lemma~\ref{lem:restriction_cats_are_partial_order_enriched}, we need only show that their
      restrictions are equal:
      \begin{align*}
        \rst{(f\djoin g)k} &= \rst{\rst{f\djoin g}(f\djoin g) k } & \rone \\
        & = \rst{f\djoin g}\, \rst{(f\djoin g)k } & \rthree\\
        & = (\rst{f}\djoin \rst{g}) \rst{(f\djoin g)k }  & \text{previous item}\\
        & = \rst{f}\,\rst{(f\djoin g)k } \djoin \rst{g}\, \rst{(f\djoin g)k }
          &\text{idempotent universal}\\
        & = \rst{\rst{f} (f\djoin g)k} \djoin \rst{\rst{g} (f\djoin g)k} & \rthree\\
        & = \rst{f k} \djoin \rst{g k}\\
        & = \rst{f k \djoin g k}.
      \end{align*}
      Therefore, as the restrictions are equal, we have shown $(f\djoin g)k = f k \djoin g k$.
    \item \emph{Associativity}: Note that \axiom{DJ}{4} shows that both sides of the equation
      exist. To show they are equal, we show that they are less than or equal to each other.

      From the definitions, we know that $f\djoin g, h \le (f\djoin g)\djoin h$, which also means
      $f, g \le (f\djoin g)\djoin h$. Similarly, $g\djoin h \le (f\djoin g)\djoin h $ and then $f
      \djoin (g\djoin h)\le (f\djoin g)\djoin h$. Conversely, $f,g,h \le f \djoin (g\djoin h)$ and
      therefore $(f\djoin g)\djoin h \le f \djoin (g\djoin h)$ and both sides are equal.
    \item \emph{Commutativity}: Note first that both $f$ and $g$ are less than or equal to both
      $f\djoin g$ and $g \djoin f$, by \axiom{DJ}{1}. By \axiom{DJ}{2}, we have $f \djoin g \le
      g\djoin f$ and $g\djoin f \le f \djoin g$ and we may conclude $f \djoin g = g \djoin f$.
    \item \emph{Identity}: By \axiom{DJ}{1}, $f \le f \djoin 0$. As $0 \le f$ and $f \le f$, by
      \axiom{DJ}{2}, $f \djoin 0 \le f$ and we have $f = f \djoin 0$.
  \end{enumerate}
\end{proof}

Note that the previous lemma and proof of associativity allows a simple inductive argument which
shows that having binary disjoint joins extends to disjoint joins of an arbitrary
finite collection of disjoint maps.

We will write $[f_i]$ to signify a list of maps, where each $f_i: A \to B$. For disjointness,
$\cdperp [f_i]$ will mean that $f_j \cdperp f_k$ where $j\ne k$ and $f_j, f_k \in [f_i]$. Finally,
$\djoin [f_i]$ will mean the disjoint join of all maps $f_i$, i.e.,
$f_1 \djoin f_2 \djoin \cdots \djoin f_n$.

\begin{lemma}\label{lem:arbitrary_disjoint_joins}
  In an inverse category with disjoint joins, $\cdperp [f_i]$ if and only if
  $\djoin [f_i]$ is defined.
\end{lemma}
\begin{proof}
  Using \axiom{Dj}{4}, proceed as in the proof of
  Lemma~\refitem{lem:join_is_associative_and_commutative_monoid}{lemitem:associative_join},
  inducting on $n$.
\end{proof}


\begin{lemma}\label{lem:disjoint_joins_only_when_all_disjoint}
  Given $\X$ is an inverse category with a disjoint join, then if $f_i, g_j : A \to B$ and $\cdperp
  [f_i]$ and $\cdperp [g_j]$, then $\djoin [f_i] \cdperp \djoin [g_j]$ if and only $f_i \cdperp
  g_j$ for all $i,j$.
\end{lemma}
\begin{proof}
  Assume $\djoin [f_i] \cdperp \djoin [g_j]$. By \axiom{Dj}{4} and associativity, we have
  $\djoin [f_i] \cdperp g_j$ for each $j$. Using the reflexivity of $\cdperp$,
  \axiom{Dj}{4} and associativity, we have $f_i \cdperp g_j$ for each $i$ and $j$.

  Assume $f_i \cdperp g_j$ for each $i$ and $j$. Then by \axiom{Dj}{4} and associativity, $f_i
  \cdperp \djoin [g_j]$ for each $i$. Applying \axiom{Dj}{4} again, we have $\djoin [f_i] \cdperp
  \djoin [g_j]$.
\end{proof}

Following the same method as in the previous section, we show that the product of two
inverse categories with disjoint joins has a disjoint join.
\begin{lemma}\label{lem:disjoint_join_is_in_product_category}
  Given $\X, \Y$ are inverse categories with disjoint joins, $\djoin$ and $\djoin'$ respectively,
  then the category $\X \times \Y$ is an inverse category with disjoint joins.
\end{lemma}
\begin{proof}
  From Lemma~\ref{lem:disjointness_is_derivable_on_a_product_category}, we know $\X\times\Y$ has a
  disjointness relation that is defined point-wise.  We therefore define $\djoin_{\times}$ the
  disjoint join on $\X\times\Y$ by
  \begin{equation}
    (f,g)\djoin_{\times}(h,k) = (f \djoin h, g \djoin' k). \label{eq:disjoint_join_on_product}
  \end{equation}
  We now prove each of the axioms in Definition~\ref{def:disjoint_join} hold.
  \setlist[itemize,1]{leftmargin=1.5cm}
  \begin{itemize}
    \item [\axiom{DJ}{1}] From Equation~\ref{eq:disjoint_join_on_product}, we see that since
      $f,h \le f \djoin h$ and $g,k \le g \djoin' k$, we have $(f,g) \le (f,g)\djoin_{\times}(h,k)$
      and  $(h,k) \le (f,g)\djoin_{\times}(h,k)$.
    \item [\axiom{DJ}{2}] Suppose $(f,g) \le (x,y)$, $(h,k) \le (x,y)$ and $(f,g) \perp_{\times}
      (h,k)$. Then regarding it point-wise, we have $(f,g)\djoin_{\times}(h,k) = (f \djoin h, g
      \djoin' k) \le (x,y)$.
    \item [\axiom{DJ}{3}] $(x,y)\left((f,g)\djoin_{\times}(h,k)\right) = (x(f \djoin h), y(g
      \djoin' k)) = (x f \djoin x h, y g \djoin' y k) = (x f,y g)\djoin_{\times}(x h,y k) =
      ((x,y)(f,g))\djoin_{\times}((x,y)(h,k))$.
    \item [\axiom{DJ}{4}] Given $\perp_{\times}\!\![(f,g),(h,k),(x,y)]$, we know $f \perp (h\djoin
      x)$ and $g \perp' (k\djoin' y)$. Hence, $(f,g) \perp_{\times}((h,k)\djoin_{\times}(x,y))$. The
      opposite direction is similar.
  \end{itemize}
\end{proof}

\begin{example}[\pinj has a disjoint join]\label{ex:pinj_has_a_disjoint_join}
  Continuing from Example~\ref{ex:pinj_has_a_disjointness_relation}, we show that \pinj has disjoint
  joins. If $f = \{(a,b)\}$ and $g=\{(a',b'\}$ are disjoint parallel maps in \pinj from $A$ to $B$,
  define $f\djoin g \doteq \{(a'',b'') | (a'',b'') \in f\text{ or }(a'',b'') \in g\}$, i.e., the
  union of $f$ and $g$.

  This is still a partial injective map, due to the requirement of disjointness. Recall that $f\perp
  g$ means that $\rst{f}\intersection \rst{g} = \emptyset$ and $\rg{f}\intersection\rg{g} =
  \emptyset$ and that the respective meets will also be $\emptyset$. The empty meet of the
  restrictions means that $f\djoin g$ is still a partial function,  as each $a''$ will appear only
  once. The empty meet of the ranges gives us that $f\djoin g$ is injective, because each $b''$ is
  unique.

  The axioms for disjoint joins all hold:
  \begin{enumerate}
    \item[\axiom{DJ}{1}] By construction, both $f$ and $g$ are less than $f\djoin g$.
    \item[\axiom{DJ}{2}] $f \le h,\ g \le h$ means that $h$ must contain all of the $(a,b)\in f$ and
      $(a',b') \in g$ and therefore $f\djoin g \le h$.
    \item[\axiom{DJ}{3}] Suppose $h:C\to A = \{(c,\dot{a})\}$. Then
      \begin{align*}
        h(f \djoin g) &= \{(c,\dot{b}) |(\exists a,\dot{a}.\dot{a} = a, (a,\dot{b}) \in f,
        (c,\dot{a}) \in h) \\
        & \qquad\qquad \text{ or }(\exists a',\dot{a}.\dot{a} = a', (a',\dot{b}) \in g, (c,\dot{a}) \in h)\}\\
        & = \{(c,\dot{b}) |\exists a, \dot{a}.\dot{a} = a, (a,\dot{b}) \in f, (c,\dot{a}) \in h\} \bigcup\\
        & \qquad\qquad\{(c,\dot{b}) | \exists a',\dot{a}.\dot{a} = a', (a',\dot{b}) \in g, (c,\dot{a}) \in h\}\\
        &= h f \djoin hg
      \end{align*}
    \item[\axiom{DJ}{4}] Suppose $\cdperp [f, f', f'']$, $f=\{(a,b)\}, f'=\{(a',b')\},
      f''=\{(a'',b'')\}$. Then the set $\{a\}$ does not intersect either $\{a'\}$ nor $\{a''\}$ and
      similarly for the sets $\{b\}, \{b'\}$ and $\{b''\}'$. Thus we have $f \perp (g\djoin h)$. The
      reverse direction is argued similarly.
  \end{enumerate}
\end{example}
% section disjoint_joins (end)

\section{Monoidal Tensors for disjointness} % (fold)
\label{sec:tensors_for_disjointness}

When we have a monoidal tensor $\+$ on $\X$, an inverse category with a restriction zero,
there are certain conditions where it is possible to define disjointness based upon the action of the
tensor. We are assuming the following naming for the standard monoidal tensor
isomorphisms:
\begin{align*}
   \upl &: 0 \+ A \to A\\
   \upr &: A \+ 0 \to A\\
   a_\+ &: (A \+ B) \+ C \to A \+ (B \+ C) \\
   c_\+ &: A \+ B \to B \+ A.
 \end{align*}
We also require the tensor isomorphisms above be natural.

\begin{definition}\label{def:disjointness_tensor}
  Given we have an inverse category \X with restriction zero and a symmetric monoidal tensor
  $\+$, the tensor $\+$ is a \emph{disjointness tensor} when:
  \begin{enumerate}[{(}i{)}]
    \item It is a restriction functor --- i.e., $\_ \+ \_ : \X \times \X \to \X$.
    \item The unit is the restriction zero. ($0 : \boldsymbol{1}\to \X$ picks out the restriction
    zero in \X).
    \item Define $\cpa = \inv{\upr}(1\+0):A\to A\+B$ and $\cpb = \inv{\upl}(0\+1): A\to B\+A$.
      $\cpa$ and $\cpb$ must be jointly epic. That is, if $\cpa f = \cpa g$ and $\cpb f = \cpb g$, then
      $f = g$.
    \item Define $\scpa := (1\+0)\upr: A\+B \to A$ and $\scpb := (0\+1)\upl:A\+B \to B$.
      $\scpa$  and $\scpb$ must be jointly monic. That is, whenever $f\scpa = g \scpa$ and
      $f\scpb = g\scpb$ then $f = g$.
  \end{enumerate}
\end{definition}

\begin{example}[\pinj has a disjointness tensor]\label{ex:pinj_has_disjointness_tensor}
  In \pinj, the disjoint union, $\disjointunion$, is a disjointness tensor. We will designate elements of
  the disjoint union as pairs of the elements of the original sets and the order in the disjoint
  join. That is, when
  \[
     A=\{a\}, B=\{b\}, \text{ then } A\disjointunion B = \{(x,n) | n\in\{1,2\},n =1 \implies x \in A, n = 2
     \implies x\in B\}.
  \]
  Setting $\+$ as $\disjointunion$, we  have the identity for the tensor is $\emptyset$. The action
  of the tensor on maps $f:A\to C = \{(a,c)\}$, $g:B\to D = \{(b,d)\}$ is given by:
  \[
    f\+g:A\+B\to C\+D = \{((x,n),(v,m)) | (x,v)\in f \text{ or } (x,v) \in g\}.
  \]
  From the definition of the disjoint union, we see that the associativity mapand commutativity tensor
  structure  maps ($a_\+, c_\+$) are identity maps. The unit maps are as follows:
  \begin{align*}
    \upl &:0\+A \to A, &(a,A) \mapsto a\\
    \upr &:A\+0 \to A, &(a,A) \mapsto a\\
    a_\+ &: (A\+B)\+C, &((a,1),1) \mapsto (a,1)\\
     &  &((b,2),1) \mapsto ((b,1),2)\\
     &  &(c,2) \mapsto ((c,1),2)\\
    c_\+ &: A\+B \to B\+ A & (a,1) \mapsto (a,2)\\
    & & (a,2) \mapsto (a,1)
  \end{align*}

  The map $\cpa = \inv{\upr}(1\+0)$ is given by the mapping $a\in A\mapsto (a,1) \in
  A\+B$. Similarly, $\cpb = \inv{\upl}(0\+1)$ is given by the mapping $a\in A\mapsto (a,2) \in
  B\+A$. We immediately see $\cpa$ and $\cpb$ are jointly epic. Similarly, $\icpa$ and $\icpb$ are
  jointly monic.
\end{example}

\begin{lemma}\label{lem:zero_plus_zero_is_zero}
  Given an inverse category \X with restriction zero and disjointness tensor $\+$, then the map
  $0\+0: A\+B \to C\+D$ is the map $0: A\+B \to C\+D$.
\end{lemma}
\begin{proof}
  Recall the zero map factors through the restriction zero, i.e. $0:A \to B$ is the same as
  saying $A\xrightarrow{!} 0 \xrightarrow{\why}B$. Additionally, as objects, $0\+0 \cong 0$ ---
  the restriction zero.

  Therefore the map $0\+0: A\+B \to C\+D$ is writable as
  \[
    A\+B \xrightarrow{!\+!}0\+0  \xrightarrow{\why\+\why} C\+D,
  \]
  which may then be rewritten as
  \[
    A\+B \xrightarrow{!\+!}0\+0 \xrightarrow{\upl} 0
      \xrightarrow{\inv{\upl}} 0\+0 \xrightarrow{\why\+\why} C\+D.
  \]

  But by the properties of the restriction zero, $(!\+!) \upl = !$ and $\inv{\upl}(\why\+\why) = \why$
  and therefore the map $0\+0: A\+B \to C\+D$ is the same as the map $0: A\+B \to C\+D$.
\end{proof}


\begin{lemma}\label{lem:disjointness_tensor_has_injections}
  Given an inverse category \X with a restriction zero and a disjointness tensor, the map $\cpa$ is
  natural in the left component and $\cpb$ is natural in the right, up to isomorphism. This means:
  \[
    \cpa (f\+g) = f \cpa \quad\text{and}\quad \cpb (f\+g) = g \cpb.
  \]
\end{lemma}
\begin{proof}
  For the left and right naturality, we see:
  \[
    \cpa (f\+g) = \inv{\upr} (1\+0) (f\+g) = \inv{\upr} (f \+ 0) =
      f \inv{\upr} (1\+0) = f \cpa,
  \]
  and
  \[
    \cpb (f\+g) = \inv{\upl} (0\+1) (f\+g) = \inv{\upl} (0 \+ g) =
      g \inv{\upl} (0\+1) = g \cpb.
  \]
\end{proof}

\begin{lemma}\label{lem:tensor_identities}
  Given an inverse category \X with restriction zero and disjointness tensor $\+$,
  $\scpa = \icpa$ and $\scpb = \icpb$ and the following identities hold:
  \begin{enumerate}[{(}i{)}]
    \item $\scp{i} \cp{i} = \rst{\scp{i}}$ and $\cp{i}\scp{i} = \rst{\cp{i}} = 1$;\label{lemitem:disjointness_tensor_identities_1}
    \item $\rst{\scpa} \scpb = 0$ and $\rst{\scpb} \scpa = 0$;\label{lemitem:disjointness_tensor_identities_2}
    \item $\cpb \scpa = 0$, $\cpb \rst{\scpa} = 0$,  $\cpa \scpb = 0$ and
      $\cpa \rst{\scpb} = 0$;\label{lemitem:disjointness_tensor_identities_3}
    \item the maps $\cpa$ and $\cpb$ are monic.\label{lemitem:disjointness_tensor_identities_4}
  \end{enumerate}
\end{lemma}
\begin{proof}
  For \ref{lemitem:disjointness_tensor_identities_1}, recalling that the restriction zero is its own
  partial inverse, we see that
  \[
    \icpa = \inv{(\inv{\upr}(1\+0))}  = \inv{(1\+0)}\upr = (1\+0)\upr = \scpa.
  \]
  Similarly,
  \[
    \icpb = \inv{(\inv{\upl}(0\+1))} = (0\+1)\upl = \scpb.
  \]

  Hence, we may calculate the restriction of $\cpa$,
  \begin{equation*}
    \cpa \scpa   = \inv{\upr}(1\+0) (1\+0)\upr
      = (\inv{\upr}(1\+0))\upr = 1 \inv{\upr}\upr= 1.
  \end{equation*}
  The calculation for $\scpb$ and $\cpb$ is analogous.

  For \ref{lemitem:disjointness_tensor_identities_2}, to show $\rst{\scpa} \scpb = 0$ and
  $\rst{\scpb} \scpa = 0$,
  \begin{align*}
    \rst{\scpa} \scpb  &= \rst{(1\+0)\upr}(0\+1)\upl\\
      &= \rst{1\+0} (0 \+ 1)\upl\\
      &= (1\+0) (0\+1)\upl\\
      &= (0\+0)\upl  = 0,
  \end{align*}
  and
  \begin{align*}
    \rst{\scpb} \scpa  &= \rst{(0\+1)\upl}(1\+0)\upr \\
     &= (0\+1) (1\+0)\upr\\
     &= (0\+0)\upr\\
     &= 0.
 \end{align*}

  We show \ref{lemitem:disjointness_tensor_identities_3}, $\cp{i} \scp{j} = 0$, $\cp{i}
  \rst{\scp{j}} = 0$ when $i\ne j$,
  \begin{align*}
    \cpa \scpb &= (\inv{\upr}(1\+0))(0\+1) \upl\\
       &= \inv{\upr}(0\+0)\upl\\
       &= 0
  \end{align*}
  and
  \begin{align*}
    \cpb \scpa &= (\inv{\upl}(0\+1))(1\+0) \upr\\
      &= \inv{\upl}(0\+0)\upr\\
       &= 0.
  \end{align*}
  As $\rst{\scpa} = 1\+0$ and $\rst{\scpb} = 0 \+1$, we see the other two identities hold as well.

  Finally, to prove \ref{lemitem:disjointness_tensor_identities_4}, we first show $\cpa$ is
  monic. Suppose $f\cpa = g\cpa$. Therefore we must have
  \[
    f = f (\cpa \icpa) = (f \cpa) \icpa = (g \cpa) \icpa = g (\cpa \icpa) = g.
  \]
  The proof that $\cpb$ is monic follows via a similar argument.
\end{proof}

As we have shown that $\scp{i} = \icp{i}$, we will prefer the explicit notation of $\icp{i}$ for the
remainder of this chapter and in Chapter~\ref{cha:inverse_sum_categories}.

\begin{corollary}\label{cor:split_plus_tensor}
  In an inverse category \X with a restriction zero and disjointness tensor, the following
  identities hold:
  \begin{multicols}{2}
    \begin{enumerate}[{(}i{)}]
      \item $\cpa (f \+g) \icpa = f$;
      \item $\cpa (f\+ g) \icpb = 0$;
      \item $\cpb (f \+g) \icpa = 0$;
      \item $\cpb (f\+ g) \icpb = g$.
    \end{enumerate}
  \end{multicols}
  Additionally, if $t$ is a map such that for $i \in \{1,2\}$,
  \[
    \cp{i} t \icp{j} =
    \begin{cases}
      t_i& : \quad i \neq j\\
      0 & : \quad i = j,
    \end{cases}
  \]
  then $t = t_1 \+ t_2$.
\end{corollary}
\begin{proof}
  The calculations for $f\+g$ follow from Lemma~\ref{lem:disjointness_tensor_has_injections} and
  Lemma~\ref{lem:tensor_identities}. For example, $\cpa (f\+g) \icpa = f \cpa \icpa = f$.

  For the second claim, note that we have $\cpa (t \icpa) = t_1 = \cpa ( t_1 \+ t_2)\icpa$ and
  $\cpb  (t \icpa) = 0 = \cpb ( t_1 \+ t_2)\icpa$, hence $t \icpa = ( t_1 \+ t_2)\icpa$. Similarly,
  we see $t\icpb = ( t_1 \+ t_2)\icpb$ and therefore $t = t_1 \+ t_2$.
\end{proof}

\begin{definition}\label{def:up_triangle_and_down_triangle}
  In an inverse category \X with a restriction zero and disjointness tensor, we define two partial
  operations on pairs of arrows in \X to another arrow in \X. First, for arrows
  $f:A \to B$ and $g:A \to C$, we define $f\tjdown g$ as
  being the map that makes Diagram~\ref{eq:tjdowndefinition} below commute, when it exists.
  \begin{equation}
    \xymatrix@C+0pt @R+20pt{
      &&A \ar[dll]_{f} \ar[drr]^{g} \ar@{.>}[d]|-{f\tjdown g}\\
      B&&B\+C \ar[ll]^{\icpa} \ar[rr]_{\icpb} && C
      }\label{eq:tjdowndefinition}
  \end{equation}
  Then for $h:B \to A$, $k: C \to A$, $h\tjup k$ is that map that makes
  Diagram~\ref{eq:tjupdefinition} commute, if it exists.
  \begin{equation}
    \xymatrix@C+0pt @R+20pt{
      B\ar[rr]^{\cpa} \ar[drr]_{h} &&B\+C   \ar@{.>}[d]|-{h\tjup k}
        && C \ar[ll]_{\cpb} \ar[dll]^{k}\\
      &&A
      }\label{eq:tjupdefinition}
  \end{equation}
  Due to $\icpa$ and $\icpb$ being jointly monic, $f\tjdown g$ is unique when it exists.
  Similarly, as $\cpa$ and $\cpb$ are jointly epic, $f\tjup g$ is unique when it exists.
\end{definition}
\begin{example}[\pinj]\label{ex:pinj_has_triangle_maps_when_restriction_is_disjoint}
  Continuing from Example~\ref{ex:pinj_has_a_disjoint_join}, we see that $f\tjdown g$ can only exist
  when $\rst{f} \meet \rst{g} = 0$, as it must be a set function, i.e., $f\tjdown g$ of some element
  $a$ must be either $(b,1)$ when  $f(a) = b\in B$ or $(c,2)$ when $g(a) = c\in C$.

  Similarly, $h\tjup k$ can only exist when $\rg{h} \meet \rg{k} = 0$.
\end{example}

The following lemma explores the behaviour of the two operations $\tjdown$ and $\tjup$:

\begin{lemma}\label{lem:properties_of_tjdown_and_tjup}
  Given \X is an inverse category with a restriction zero and a disjointness tensor $\+$ then the
  following relations hold for $\tjdown$ and $\tjup$:
   \begin{enumerate}[{(}i{)}]
    \item If $f \tjdown g$ exists, then $g \tjdown f$ exists. If $f \tjup g$ exists, then
    $g \tjup f$ exists. \label{lemitem:l_r_commute}
    \item $f \tjdown 0$ and $f \tjup 0$ always exist. \label{lemitem:l_r_zero_exists}
    \item When $f\tjdown g$ exists, $\rst{f}(f\tjdown g) = f \tjdown 0$, $\rst{f}g = 0$,
      $\rst{g}(f \tjdown g) = 0 \tjdown g$ and $\rst{g}f = 0$.
      \label{lemitem:l_r_rst_orthogonal_to_zero}
    \item Dually to the previous item, when $f\tjup g$ exists,
      $(f\tjup g)\rg{f} = f \tjup 0$, $g\rg{f} = 0$, $(f \tjup g)\rg{g} = 0 \tjup g$ and
      $f\rg{g} = 0$.\label{lemitem:l_r_rg_orthogonal_to_zero}
    \item When $f\tjdown g$ exists, $f\tjdown g (h\+k) = f h \tjdown g k$.
      \label{lemitem:l_r_l_pull_right}
    \item Dually, when $f \tjup g$ exists, $(h \+ k) f \tjup g = h f \tjup k g$.
      \label{lemitem:l_r_r_pull_left}
    \item When $f \tjdown g$ exists, then $h(f \tjdown g) = h f \tjdown h g$ and when $f \tjup g$
      exists, $(f \tjup g)h = f h \tjup g h$.
      \label{lemitem:l_r_l_universal_r_stable}
    \item If $\rst{f} \tjdown \rst{g}$ exists, then $\rst{f} \tjup \rst{g}$ exists and is
      the partial inverse of $\rst{f} \tjdown \rst{g}$.
      \label{lemitem:l_r_rst_inverses}
    \item If $f\tjdown g$ exists and $f' \le f$, $g' \le g$, then $f' \tjdown g'$ exists.
      \label{lemitem:l_r_less_than}
    \item When $f\tjup g$ exists, $(f\tjup g) \inv{(f\tjup g)} = \rst{f}\+\rst{g}$.
      \label{lemitem:l_r_rst_is_natural}
    \item Given $f\tjdown g$ and $h\tjdown k$ exist, then
      $(f\+h)\tjdown(g\+k) = (f\tjdown g)\+ (h\tjdown k)$. Dually, the existence of $f\tjup g$
      and $h\tjup k$ implies $(f\+h)\tjup(g\+k) = (f\tjup g)\+ (h\tjup k)$.
      \label{lemitem:l_r_preserve_tensor}
  \end{enumerate}
\end{lemma}
\begin{proof}
  \prepprooflist
  \begin{enumerate}[{(}i{)}]
    \item $g \tjdown f = (f \tjdown g)\comp $ and $g \tjup f = \comp(f\tjup g)$.
    \item Consider $f \cpa$. Then $f \cpa \icpa = f$ and
      $f \cpa \icpb = f 0 = 0$. Hence, $f \cpa = f \tjdown 0$.

      Consider $\icpa f$. Then $\cpa \icpa f = f$ and
      $\cpb \icpa f = 0 f = 0$ and therefore $\icpa f = (f\tjup 0)$.
    \item Using Lemma~\ref{lem:tensor_identities}
      \[
        \rst{f}g = \rst{(f\tjdown g)\icpa} (f\tjdown g)\icpb =
          (f\tjdown g) \rst{\icpa} \icpb = 0.
      \]
      Similarly, $\rst{g}f = f\tjdown g \rst{\icpb}\icpa = 0$.

      Recall that $\icpa$ and $\icpb$ are jointly monic. We have
      $\rst{f}(f \tjdown g) \icpa = \rst{f} f = f = (f\tjdown 0) \icpa$ and
      $\rst{f}(f \tjdown g) \icpb = \rst{f}g = 0= (f\tjdown 0) \icpb$.
      Therefore, $\rst{f}(f \tjdown g) = f \tjdown 0$. Similarly,
      $\rst{g}(f \tjdown g) = 0 \tjdown g$.
    \item Using Lemma~\ref{lem:tensor_identities}
      \begin{multline*}
        g\rg{f} = \cpb(f\tjup g)\wrg{(\cpa(f\tjup g))}  =
          \cpb(f\tjup g) \rst{\inv{(f\tjup g)}\icpa} =\\
          \cpb(f\tjup g) \rst{\inv{(f\tjup g)}\rst{\icpa}} =
          \rst{\cpb \rst{(f\tjup g)} \rst{\icpa}}\cpb(f\tjup g) = \\
          \rst{\cpb \rst{\icpa}\rst{(f\tjup g)} }\cpb(f\tjup g) =
          \rst{0}\cpb(f\tjup g) = 0
      \end{multline*}
      Similarly, $f\rg{g} =  0$.

      Recall that $\cpa$ and $\cpb$ are jointly epic. We have
      $\cpa(f \tjup g)\rg{f} = f\rg{f} = f = \cpa(f\tjup 0) $ and
      $\cpb(f \tjup g)\rg{f} = g\rg{f} = 0= \cpb(f\tjup 0)$.
      Therefore, $(f \tjup g)\rg{f} = f \tjup 0$. Similarly,
      $(f \tjup g) \rg{g} = 0 \tjup g$.
    \item Calculating, we have
      \[
        f \tjdown g (h \+ k) \icpa = f\tjdown g \icpa h = f h
      \]
      and
      \[
        f \tjdown g (h \+ k) \icpb = f\tjdown g \icpb k = g k,
      \]
      which means that
      $f \tjdown g (h \+ k) = f h \tjdown g k$ by the joint monic property of $\icpa$, $\icpb$.
    \item The proof for this is dual to \ref{lemitem:l_r_l_pull_right}, and depends on the joint
      epic property of $\cpa$ and $\cpb$.
    \item We are given $f\tjdown g$ exists, therefore $f = (f\tjdown g)\icpa$ and
      $g = (f\tjdown g)\icpb$. But this means $hf = h(f\tjdown g)\icpa$ and
      $h g = h(f\tjdown g)\icpb$, from which we may conclude $hf \tjdown h g = h(f\tjdown g)$ by
      the fact that $\icpa$ and $\icpb$ are jointly monic. The proof of
      $(f\tjup g)h = f h \tjup g h$ is similar.
    \item We are given $\rst{f} = \rst{f}\tjdown\rst{g} \icpa$. Therefore,
      \[
        \rst{f} = \inv{\rst{f}} = \inv{\icpa}\inv{(\rst{f}\tjdown\rst{g})}
          = \cpa \inv{(\rst{f}\tjdown\rst{g})}.
      \]
      Similarly, $\rst{g} = \cpb \inv{(\rst{f}\tjdown\rst{g})}$. But this means
      $\inv{(\rst{f}\tjdown\rst{g})} = \rst{f}\tjup\rst{g}$.
    \item Note that from \ref{lemitem:l_r_l_pull_right}, we know that
      $f \tjdown g  = \rst{f} \tjdown \rst{g} (f\+g)$. We are given $f' \le f$ and $g' \le g$. This
      gives us $\rst{f'}f =f'$, $\rst{g'}g =g'$, $\rst{f'}\,\rst{f} =\rst{f'}$ and
      $\rst{g'}\,\rst{g} =\rst{g'}$. Consider the map
      $\rst{f} \tjdown \rst{g} (\rst{f'}\+\rst{g'})(f\+g)$. Calculating, we see
      \begin{align*}
        \rst{f} \tjdown \rst{g} (\rst{f'}\+\rst{g'})(f\+g)
          &= \rst{f} \tjdown \rst{g} (\rst{f'}\+\rst{g'})(\rst{f'}\+\rst{g'})(f\+g) \\
          & = \rst{f} \tjdown \rst{g} (\rst{f'}\+\rst{g'})(f'\+g') \\
          & = \rst{f}\,\rst{f'} \tjdown \rst{g} \rst{g'} (f'\+g') \\
          & = \rst{f'}\,\rst{f} \tjdown \rst{g'} \rst{g} (f'\+g') \\
          & = \rst{f'} \tjdown \rst{g'}  (f'\+g') \\
          & = f'\tjdown g'.
      \end{align*}
    \item From our diagram for $\tjup$, we know:
      \begin{align*}
        \inv{f} &= \inv{(f\tjup g)} \icpa\text{ and}\\
        \inv{g} &= \inv{(f\tjup g)} \icpb.
      \end{align*}
      As well, we know that $\cpa(f\tjup g) = f$ and
      $\cpa(f\tjup g) = g$.
      Therefore, we have:
      \[
         \cpa(f\tjup g)\inv{(f\tjup g)} \icpa = \rst{f} \text{ and  }
          \cpb(f\tjup g)\inv{(f\tjup g)} \icpb = \rst{g}.
      \]
      As $f\tperp g$, we know that $f\inv{g} = f \rg{g} \inv{g} = 0 \inv{g} = 0$ and therefore,

      \[
         \cpa(f\tjup g)\inv{(f\tjup g)} \icpb = 0 \text{ and  }
          \cpb(f\tjup g)\inv{(f\tjup g)} \icpa = 0.
      \]
      By Corollary~\ref{cor:split_plus_tensor} this means
      $(f\tjup g)\inv{(f\tjup g)} = \rst{f} \+ \rst{g}$.
    \item As $(f\tjdown g)\+ (h\tjdown k)\icpa = (f\tjdown g)$ and
      $(f\tjdown g)\+ (h\tjdown k) \icpb = (h\tjdown k)$, we see that
      $(f\tjdown g)\+ (h\tjdown k)$ satisfies the diagram for $(f\+h)\tjdown(g\+k)$. Dually, as
      $\cpa (f\tjup g)\+ (h\tjup k) = (f\tjup g)$ and
      $\cpb(f\tjup g)\+ (h\tjup k) = (h\tjup k)$, $(f\tjup g)\+ (h\tjup k)$ satisfies
      the diagram for $(f\+h)\tjup(g\+k)$.
  \end{enumerate}
\end{proof}

% \begin{definition}\label{def:general_triangle}
%   In an inverse category \X with a restriction zero and disjointness tensor, we define two partial
%   operations on pairs of arrows in \X to another arrow in \X. First, for arrows
%   $f:A \to B$ and $g:A \to C$, suppose we have arrows $f':A  \to B\+X$ and $g': A \to C\+X$
%   such that $f'\icpa =f$, $g'\icpa = g$ and $f'\icpb = g' \icpb$. Then, we say $f\gtjdown g$ if
%   regardless of the choice of $f',\ g'$ and $X$, there exists a map $\alpha$ that makes
%   diagram \ref{eq:gtjdowndefinition} below commute.
%   \begin{equation}
%     \xymatrix@C+0pt @R+20pt{
%       B\+X&&B\+C\+X \ar[ll]_{\icp{0,2}} \ar[rr]^{\icp{1,2}} && C\+X\\
%       &&A \ar[ull]^{g'} \ar[urr]_{f'} \ar@{.>}[u]|-{\alpha}
%       }\label{eq:gtjdowndefinition}
%   \end{equation}
%
%   Similarly, for $h:B \to A$, $k: C \to A$, and $h': A\+X \to C$, $k': B\+X \to C$, with $h = \cpa
%   h'$ and $k = \cpb k'$, then $h\gtjup k$ if regardless of the choice of $h',\ k'$ and $X$, there
%   exists a map $\beta$ that makes diagram \ref{eq:gtjupdefinition} commute.
%   \begin{equation}
%     \xymatrix@C+0pt @R+20pt{
%       &&A\\
%       B\+X\ar[rr]_{\amalg_{0,2}} \ar[urr]^{h'} &&B\+C \+X  \ar@{.>}[u]|-{\beta}
%         && C\+X \ar[ll]^{\amalg_{1,2}} \ar[ull]_{k'}
%       }\label{eq:gtjupdefinition}
%   \end{equation}
%   In the above diagrams, $\icp{0,2} = (1\+0\+1)(\upr\+1)$, $\amalg_{0,2} =
%   (\inv{\upr}\+1)(1\+0\+1)$, $\icp{1,2}=(0\+1\+1)(\upl\+1)$ and
%   $\cp{1,2} = (\inv{\upl}\+1)(0\+1\+1)$.
% \end{definition}
% \begin{lemma}\label{lem:general_triangle_implies_triangle}
%   In an inverse category $\X$ with a restriction zero and a disjointness tensor, with maps
%   $f:A\to B$, $g: A \to C$, $h: B \to A$ and $k: C \to A$. Then $f\gtjdown g$
%   (respectively $h \gtjup k$) implies that $f\tjdown g$ (respectively $h \tjup k$) exists.
% \end{lemma}
% \begin{proof}
%   Suppose $f \gtjdown g$. Then set $f' = f \inv{\upr}$, $g' = g \inv{\upr}$ and $X = 0$. Then
%   $\alpha: A \to B\+C\+0$ exists and we can see that setting $f\tjdown g = \alpha \upr$ makes the
%   first diagram of Definition~\ref{def:up_triangle_and_down_triangle} commute. In detail,
%   \[
%     \xymatrix@C+0pt @R+10pt{
%       B&&B\+C \ar[ll]_{\icpa} \ar[rr]^{\icpb} && C\\
%       &B\+0\ar[ul]^{\upl}&B\+C\+0 \ar[l]_{\icp{0,2}} \ar[r]^{\icp{1,2}} \ar[u]|-{\upl}
%         & C\+0\ar[ur]_{\upl}\\
%       &&A \ar[ul]^{g \inv{\upl}} \ar[ur]_{f\inv{\upl}} \ar@{.>}[u]|-{\alpha}
%       }
%   \]
%   where the lower triangle commutes by assumption and the upper quadrilaterals commute due to the
%   coherence diagrams of the tensor.
%
%   Similarly, setting $h' = \inv{\upr}h$ and $k' = \inv{\upr}k$ gives us $\beta:B\+C\+0 \to A\+0$ and
%   we set $f\tjup = \beta \upr.$
% \end{proof}

\begin{lemma}\label{lem:tensor_disjointness_is_disjointness}
  Define $f \tperp g$ ($f$ is tensor disjoint to $g$) when $f, g: A\to B$ and both $f\tjdown g$ and
  $f\tjup g$ exist.  If \X is an inverse category with a restriction zero and a disjointness tensor
  $\+$ then the relation $\tperp$ is a disjointness relation.
\end{lemma}
\begin{proof}
  We need to show that $\tperp$ satisfies the disjointness axioms. We will use \axiom{Dis}{6'} in
  place of \axiom{Dis}{6} and \axiom{Dis}{7} as discussed in
  Lemma~\ref{lem:disjointness_equivalent_axioms}.
  \setlist[itemize,1]{leftmargin=1.5cm}
  \begin{itemize}
    \item[\axiom{Dis}{1}] We must show $f \tperp 0$. This follows immediately from
      Lemma~\ref{lem:properties_of_tjdown_and_tjup}, item \ref{lemitem:l_r_zero_exists}.
    \item[\axiom{Dis}{2}] Show $f \tperp g$ implies $\rst{f}g = 0$. This is a direct consequence of
      Lemma~\ref{lem:properties_of_tjdown_and_tjup}, item
      \ref{lemitem:l_r_rst_orthogonal_to_zero}.
    \item[\axiom{Dis}{3}] We require $f\tperp g$, $f' \le f$, $g' \le g$ implies $f' \tperp g'$.
      From Lemma~\ref{lem:properties_of_tjdown_and_tjup}, item \ref{lemitem:l_r_less_than}, we
      immediately have $f' \tjdown g'$ exists. Using a similar argument to the proof of this item,
      we also have $f' \tjup g'$ exists and hence $f' \tperp g'$.
    \item[\axiom{Dis}{4}] Commutativity of $\tperp$ follows from the symmetry of the two required
    diagrams, see Lemma~\ref{lem:properties_of_tjdown_and_tjup}, item
      \ref{lemitem:l_r_commute}.
    \item[\axiom{Dis}{5}] Show that if $f\tperp g$ then $h f \tperp h g$ for any map $h$.
      By  Lemma~\ref{lem:properties_of_tjdown_and_tjup}, item
      \ref{lemitem:l_r_l_universal_r_stable}, we know that $h f \tjdown h g$ exists.
      By item \ref{lemitem:l_r_r_pull_left},  $(h f)\tjup (h g) = (h \+ h) (f\tjup g)$
      and therefore $h f \tperp h g$.
    \item[\axiom{Dis}{6'}] We need to show $f\tperp g$ if and only if $\rst{f} \tperp \rst{g}$ and
      $\rg{f}\tperp \rg{g}$. This follows directly from
      Lemma~\ref{lem:properties_of_tjdown_and_tjup}, items
      \ref{lemitem:l_r_l_pull_right} and       \ref{lemitem:l_r_r_pull_left}, which give us
      $f\tjdown g = \rst{f}\tjdown \rst{g} (f\+g)$ and
      $f\tjup g = (f \+ g)\rg{f}\tjup \rg{g}$, where the equalities hold if either side of
      the equation exists.
  \end{itemize}
\end{proof}

\begin{example}[$\tperp$ in \pinj]\label{ex:tensor_perp_in_pinj}
  Referring to Example~\ref{ex:pinj_has_triangle_maps_when_restriction_is_disjoint}, we noted
  $f\tjdown g$ exists when $\rst{f} \meet \rst{g} = 0$ and that $f \tjup g$ exists when $\rg{f} \meet
  \rg{g} = 0$. But this agrees with our initial definition of disjointness $(\perp)$ in \pinj from
  Example~\ref{ex:pinj_has_a_disjointness_relation} and hence we have that $\tperp$ is the same
  relation as $\perp$ in \pinj.
\end{example}

The operations $\tjdown$ and $\tjup$ are sufficient to define a disjointness relation
on an inverse category. However, when we wish to extend this to a disjoint join, we need to prove
$\axiom{DJ}{4}$, That is, we need to show that $\tperp [f,g,h]$ implies $f \tperp (g \tjoin h)$.

Therefore, we add one more assumption regarding our tensor in order to define disjointness.

\begin{definition}\label{def:disjoint_sum_tensor}
  Let \X be an inverse category with a disjointness tensor $\+$ and a restriction zero. Consider the
  commutative diagrams \ref{dia:tensor_complete_left} and \ref{dia:tensor_complete_right}.
  \begin{align}
    &\xymatrix@C+5pt@R+10pt{
      A \ar@/_/[ddr]_f \ar@/^/[drr]^g \ar@{.>}[dr]|-{\alpha} \\
        & X\+Y\+Z \ar[d]^{\icp{1,2}} \ar[r]_{\icp{1,3}} & X\+Z \ar[d]^{\icpa} \\
        & X\+Y \ar[r]_{\icpa} & X
    } \label{dia:tensor_complete_left}\\
    &\xymatrix@C+5pt@R+10pt{
        &&A \\
         X\+Y \ar[r]_{\cp{1,2}} \ar@/^/[urr]^h &X\+Y\+Z \ar@{.>}[ur]|-{\beta} \\
         X \ar[u]^{\cpa} \ar[r]_{\cpa} & X\+Z \ar[u]^{\cp{1,3}} \ar@/_/[uur]_k
    } \label{dia:tensor_complete_right}
  \end{align}
  Then
  $\+$ is a \emph{disjoint sum tensor} when the following two conditions hold:
  \begin{itemize}
    \item $\alpha$ exists if and only if  $f \icpb \tjdown g \icpb$ exists;
    \item $\beta$ exists if and only if $\cpb h \tjup \cpb k$ exists.
  \end{itemize}

\end{definition}

\begin{example}[In \pinj, $\+$ is a disjoint sum tensor]\label{ex:pinj_has_disjoint_sum_tensor}
  In \pinj, Diagram~\ref{dia:tensor_complete_left} means that $f$ and $g$ must agree on those
  elements of $A$ that map to $(x,1)$ in either $X\+Y$ or $X\+Z$. The statement that
  $f \icpb  \tjdown g \icpb$ exists means that if $f(a) = (y,2)$, then $g(a)$ must be undefined and
  vice versa. In such a case $\alpha$ exists and is defined as:
  \[
     \alpha(a) = \begin{cases}
       (x,1) & f(a) = (x,1)\in X\+Y \text{ and }g(a) = (x,1) in X\+Z\\
       (y,2) & f(a) = (y,2)\in X\+Y \text{ and }g(a) \uparrow\\
       (z,3) & g(a) = (z,2)\in X\+Z \text{ and }f(a) \uparrow.
       \end{cases}
 \]
 Conversely, if $\alpha$ exists, $\alpha(a)$ must be one of $(x,1), (y,2)$ or $(z,3)$. As
 $f\icpb = \alpha\icp{1,2}\icpb$ and $g\icpb = \alpha\icp{1,3}\icpb$, we see this immediately
 requires that $\rst{f\icpb}\meet\rst{\icpb} =0$ and therefore $f \icpb  \tjdown g \icpb$ exists.

 The reasoning for Diagram~\ref{dia:tensor_complete_left} is similar.
\end{example}
\begin{lemma}\label{lem:complete_disjointness_means_multiple_disjoints}
  Let \X be an inverse category with a disjoint sum tensor as in
  Definition~\ref{def:disjoint_sum_tensor} and we are given $f,g,h:A\to B$ with
  $\tperp[f,g,h]$. Then both $f \tjdown (g \tjdown h)$ and $f\tjup(g\tjup h)$ exist.
\end{lemma}
\begin{proof}
  As all the maps are disjoint, we know the maps $\tjdown$ and $\tjup$ exist for each pair.
  Consider the diagram
  \[
    \xymatrix@C+5pt@R+10pt{
      A \ar@/_/[ddr]_{g\tjdown h} \ar@/^/[drr]^{g\tjdown f} \ar@{.>}[dr]|-{\alpha} \ar[rr]^{f}
        & & B \\
        & B\+B\+B \ar[d]^{\icp{0,1}} \ar[r]_{\icp{0,2}} & B\+B \ar[d]^{\icpa} \ar[u]_{\icpb} \\
        & B\+B \ar[r]_{\icpa} & B
    }
  \]
  where we claim $\alpha = (g\tjdown h)\tjdown f$.

  The lower part of the diagram commutes as it fulfills the conditions of
  Definition~\ref{def:disjoint_sum_tensor}. The upper rightmost triangle of the diagram commutes by
  the definition of $g\tjdown f$. Noting that $\icp{0,1}:B\+B\+B \to B\+B$ is the same map as
  $\icpa:(B\+B)\+B\to (B\+B)$ and $\icp{0,2} \icpb:B\+B\+B \to B\+B \to B$ is the same map as
  $\icpb:(B\+B)\+B\to B$, we see $\alpha$ does make the $\tjdown$ diagram for $g\tjdown h$ and $f$
  commute. Therefore by Lemma~\ref{lem:properties_of_tjdown_and_tjup}, $f\tjdown(g\tjdown h)$
  exists and is equal to $\alpha \com{\+\,\{01,2\}}$.

  A dual diagram and corresponding reasoning shows $f\tjup(g\tjup h)$ exists.
\end{proof}

\begin{lemma}\label{lem:tjdown_and_tjup_associate}
  In an inverse category $\X$ with a disjoint sum tensor, when $\tperp [f,g,h]$, then:
  \begin{enumerate}[{(}i{)}]
    \item   $f \tjdown (g \tjdown h) = ((f \tjdown g) \tjdown h) \assocp$ and both exist,
    \item   $f \tjup (g \tjup h) = ((f \tjup g) \tjup h) \assocp$ and both exist.
  \end{enumerate}
\end{lemma}
\begin{proof}
  Consider the diagram
  \begin{equation}
    \xymatrix@C+5pt@R+10pt{
      A \ar@/_/[ddr]_{f\tjdown g} \ar@/^/[drr]^{f\tjdown h} \ar@{.>}[dr]|-{\alpha} \ar[rr]^{h}
        & & B \\
        & B\+B\+B \ar[d]^{\icp{0,1}} \ar[r]_{\icp{0,2}} & B\+B \ar[d]^{\icpa} \ar[u]_{\icpb} \\
        & B\+B \ar[r]_{\icpa} & B
    }\label{dia:alpha_plus_h}
  \end{equation}
  which gives us $\alpha = (f \tjdown g) \tjdown h: A \to (B\+B)\+B$ and
  $\alpha \assocp :A \to B\+(B\+B)$. Next consider the diagram
  \begin{equation}
    \xymatrix@C+5pt@R+10pt{
      A \ar@/_/[ddr]_{g\tjdown h} \ar@/^/[drr]^{g\tjdown f} \ar@{.>}[dr]|-{\gamma} \ar[rr]^{f}
        & & B \\
        & B\+B\+B \ar[d]^{\icp{0,1}} \ar[r]_{\icp{0,2}} & B\+B \ar[d]^{\icpa} \ar[u]_{\icpb} \\
        & B\+B \ar[r]_{\icpa} & B
    }\label{dia:gamma_plus_f}
  \end{equation}
  which gives us $\gamma \comp = f \tjdown (g\tjdown h): A \to B\+(B\+B)$.

  Note from Diagrams~\ref{dia:alpha_plus_h} and \ref{dia:gamma_plus_f} we have
  \begin{align*}
    \gamma \comp  \icp{0}         =\, &f  = \alpha \assocp\icpa\\
    \gamma \comp  \icp{1} \icp{0} =\, &g  = \alpha \assocp\icpb\icpa\\
    \gamma \comp  \icp{1} \icpb   =\, & h = \alpha \assocp\icpb\icpb.
  \end{align*}
  By the assumption that $\icpa, \icpb$ are jointly monic, we have
  $\alpha = \gamma \comp  \assocp$. Therefore $f \tjdown (g \tjdown h) = (f \tjdown g) \tjdown h$,
  up to the associativity isomorphism.
\end{proof}
\begin{definition}\label{def:tensor_disjoint_join}
  Let \X be an inverse category with a disjointness tensor and restriction zero. Assume we have two
  maps $f,g: A \to B$ with $f\tperp g$. Then define the \emph{tensor disjoint join} of $f$ and $g$
  as the map $f \tjoin g = \rst{f}\tjdown\rst{g} (f\+g) \rg{f}\tjup\rg{g}$.
\end{definition}


For reference, the map $f \tjoin g$ may be visualized as follows:
\[
  \xymatrix@C+20pt @R+10pt{
    &A
      & B  \ar[dr]^{\rg{f}} \ar[d]_{\cpa}\\
    A \ar[ur]^{\rst{f}} \ar@{.>}[r]^{\rst{f}\tjdown \rst{g}} \ar[dr]_{\rst{g}}
      & A\+A \ar[u]_{\icpa} \ar[d]^{\icpb}
        \ar[r]^{f\+g}
       &B\+B \ar@{.>}[r]^{\rg{f}\tjup\rg{g}} & B.\\
    & A & B\ar[ur]_{\rg{g}} \ar[u]^{\cpb}
  }
\]

Using Lemma~\ref{lem:properties_of_tjdown_and_tjup}, we may rewrite this in a variety of
equivalent ways:
\begin{align*}
  f \tjoin g &= \rst{f}\tjdown\rst{g} (f\+g) \rg{f}\tjup\rg{g} \\
  &= f \tjdown g \rg{f}\tjup\rg{g}\\
  & = \rst{f}\tjdown\rst{g}  f\tjup g\\
  & = f \tjdown g (\inv{f}\+\inv{g})f\tjup g.
\end{align*}

In particular, note that $\rst{f}\tjoin \rst{g} = (\rst{f} \tjdown \rst{g})
(\rst{f}\tjup\rst{g})$ as $\rg{\rst{g}} = \rst{g}$.

\begin{example}[Tensor join in \pinj]\label{ex:tensor_join_in_pinj}
  We will use the third equality of those above for $f\tjoin g$, i.e., $\rst{f}\tjdown\rst{g}
  f\tjup g$.

  We have:
  \begin{equation}
    \rst{f}\tjdown\rst{g}(a) =
    \begin{cases}
      (a,1) & \rst{f}(a) = a, \rst{g}\undefined\\
      (a,2) & \rst{g}(a) = a, \rst{f}\undefined
    \end{cases}\label{eq:definition_of_rf_tjdown_rg_in_pinj}
  \end{equation}
  and
  \begin{equation}
    f\tjup g((a,n)) =
    \begin{cases}
      f(a)  & n = 1\\
      g(a)  & n=2.
    \end{cases}\label{eq:definition_of_f_tjup_g_in_pinj}
  \end{equation}
  Combining Equation~\ref{eq:definition_of_rf_tjdown_rg_in_pinj} with
  Equation~\ref{eq:definition_of_f_tjup_g_in_pinj} then gives us the same definition as that of
  $\djoin$ as given in Example~\ref{ex:pinj_has_a_disjoint_join}.
\end{example}
\begin{lemma}\label{lem:tensor_disjoint_join_properties}
  Let \X be an inverse category with a disjointness tensor and restriction zero. Let \X have the
  maps $f,g: A \to B$ with $f\tperp g$. Then $\tjoin$ has the following properties.
  \begin{enumerate}[{(}i{)}]
    \item For all maps $h:A \to B$, $\rst{f}h \tjoin \rst{g}h = (\rst{f}\tjoin \rst{g})h$.
      \label{lemitem:tdj_rst_universal}
    \item $\rst{f}\tjoin \rst{g} = \rst{f\tjoin g}$. \label{lemitem:tdj_rst_is_rst}
  \end{enumerate}
\end{lemma}
\begin{proof}
  \prepprooflist
  \begin{enumerate}[{(}i{)}]
    \item By Lemma~\ref{lem:disjointness_various}, item \ref{lemitem:djv_universal}, we know that
    $\rst{f}h \tperp \rst{g} h$, hence we can form $\rst{f}h \tjoin \rst{g}h$.
    Also, noting that
      \[
        h \wrg{\rst{f} h} = h \rst{\inv{h}\rst{f}} = \rst{h \inv{h} \rst{f}} h
          = \rst{\rst{h}\rst{f}} h = \rst{f} \rst{h}h = \rst{f} h,
      \]
      we may then calculate from the left hand side as follows:
      \begin{align*}
        \rst{f}h \tjoin \rst{g}h
          & = (\rst{f}h \tjdown \rst{g} h) (\wrg{\rst{f}h}\tjup\wrg{\rst{g} h})\\
          & = (\rst{f} \tjdown \rst{g} ) (h\wrg{\rst{f}h}\tjup h\wrg{\rst{g} h})\\
          & = (\rst{f} \tjdown \rst{g} ) (\rst{f}h\tjup \rst{g} h)\\
          & = (\rst{f} \tjdown \rst{g} ) (\rst{f}\tjup \rst{g} )h\\
          & = (\rst{f} \tjoin \rst{g} ) h.
      \end{align*}
    \item
    Using Lemma~\ref{lem:properties_of_tjdown_and_tjup}, item~\ref{lemitem:l_r_rst_is_natural},
    we can compute:
    \begin{align*}
      \rst{f\tjoin g} & = f \tjoin g \inv{(f\tjoin g)} \\
      & = \left((\rst{f} \tjdown \rst{g})(f\tjup g)\right)
             \left(\inv{(f\tjdown g)} \inv{(\rst{f}\tjdown \rst{g})}\right)\\
      & = \rst{f}\tjdown\rst{g}(f\tjdown g) \inv{(f\tjdown g)} \rst{f}\tjup \rst{g}\\
      & = \rst{f}\tjdown\rst{g}(\rst{f}\+ \rst{g})\rst{f}\tjup \rst{g}\\
      & = \rst{f}\tjdown\rst{g}\rst{f}\tjup \rst{g}\\
      & = \rst{f}\tjoin\rst{g}.
    \end{align*}
  \end{enumerate}
\end{proof}


\begin{proposition}\label{prop:disjointness_tensor_gives_disjoint_join}
  Let \X be an inverse category with a disjoint sum tensor and restriction zero. Assume we have two
  maps $f,g$ with $f\tperp g$. Then the tensor disjoint join map $f \tjoin g$ from
  Definition~\ref{def:tensor_disjoint_join} is a disjoint join.
\end{proposition}
\begin{proof}
  \setlist[itemize,1]{leftmargin=1.5cm}
  \begin{itemize}
    \item [\axiom{DJ}{1}] We must show $f, g \le f \tjoin g$.
      \begin{align*}
        \rst{f}\,(\rst{f}\tjdown\rst{g})  f\tjup g
          &= (\rst{f}\tjdown\rst{g}) \icpa (\rst{f}\tjdown\rst{g} ) f\tjup g \\
        &=\rst{(\rst{f}\tjdown\rst{g}) \icpa} (\rst{f}\tjdown\rst{g}) f\tjup g  \\
        &=(\rst{f}\tjdown\rst{g}) \rst{\icpa} f\tjup g  \\
        &=(\rst{f}\tjdown\rst{g}) \icpa \cpa f\tjup g \\
        &=((\rst{f}\tjdown\rst{g}) \icpa) (\cpa (f \tjup g)) \\
        &= \rst{f} f \\
        &=f.
      \end{align*}
      Thus, we see $f \le f \tjoin g$. Showing $g \le f \tjoin g$ proceeds in the same manner.
    \item [\axiom{DJ}{2}] We must show that $f \le h,\ g\le h$ and $f\tperp g$ implies
      $f \tjoin g \le h$.
      \begin{align*}
        \rst{f \tjoin g} \,h & = \rst{\rst{f} h \tjoin \rst{g} h} \,h\\
        & = \rst{(\rst{f} \tjoin \rst{g})h} \,h\\
        & = \rst{\rst{(\rst{f} \tjoin \rst{g})}h} \,h\\
        & = \rst{\rst{(\rst{f} \tjoin \rst{g})}h}\,(\rst{f} \tjoin \rst{g})h\\
        & = \rst{(\rst{f} \tjoin \rst{g})h}\,(\rst{f} \tjoin \rst{g})h\\
        & = (\rst{f}\tjoin \rst{g})h\\
        & = (\rst{f}h\tjoin \rst{g}h)\\
        & = (f\tjoin g).
      \end{align*}
    \item [\axiom{DJ}{3}] We must show stability of $\tjoin$, i.e., that
      $h(f\tjoin g) = h f \tjoin h g$.

      \begin{align*}
        h (f\tjoin g) &= h ((\rst{f}\tjdown\rst{g})  (f \tjup g))\\
        &= (h\rst{f}\tjdown h\rst{g}) (f \tjup g)\\
        &= (\rst{h f}h\tjdown \rst{h g}h)  (f \tjup g)\\
        &= (\rst{h f}\tjdown \rst{h g})(h\+h)  (f \tjup g)\\
        &= (\rst{h f}\tjdown \rst{h g}) (h f \tjup h g)\\
        &= h f \tjoin h g.
      \end{align*}


    \item [\axiom{DJ}{4}] We need to show $\tperp [f,g,h]$ if and only if $f \tperp (g \tjoin h)$.
      For the right to left implication, note that the existence of $g\tjoin h$ implies $g \tperp
      h$. We also know $g, h \le g\tjoin h$ by item 1 of this lemma. This gives us that $f \tperp
      g$ and $f \tperp h$, hence $\tperp [f,g,h]$.

      For the left to right implication, we use
      Lemma~\ref{lem:complete_disjointness_means_multiple_disjoints}. As we have $\tperp [f,g,h]$,
      we  know $f\tjdown(g\tjdown h)$ and $f\tjup(g\tjup h)$ exist.

      Recall that $g\tjoin h = (g\tjdown h)(\rg{g}\tjup\rg{h})$. Then the map
      \[
        A\xrightarrow{f\tjdown(g\tjdown h)} B\+B\+B\xrightarrow{1\+(\rg{g}\tjup\rg{h})} B\+B
      \]
      makes the diagram for $f \tjdown (g\tjoin h)$ commute.

      Recalling that  $g\tjoin h =
      (\rst{g}\tjdown \rst{h})(g\tjup h)$, we also see that
      \[
        A\+A\xrightarrow{1\+(\rst{g}\tjdown\rst{h})}A\+A\+A\xrightarrow{f\tjup(g\tjup h)}B
      \]
      provides the witness map for $f \tjup(g\tjoin h)$ and hence $f \tperp (g \tjoin h)$.
  \end{itemize}

\end{proof}
% subsection disjoint_join_via_a_monoidal_tensor (end)
% section tensors_for_disjointness (end)

\chapter{Sums in Frobenius Algebras} % (fold)
\label{cha:sums_in_frobenius_algebras}

\begin{tikzpicture}
 [ node distance=.1mm, inner sep=0.5mm,
   delta/.style={shape=isosceles triangle, isosceles triangle apex angle=70, draw, shape border
     rotate=90, minimum size=4mm},
   nabla/.style={shape=isosceles triangle, isosceles triangle apex angle=70, draw, shape border rotate=270, minimum size=4mm}]
\path node at (0,0) [nabla] (n1) {}
node at (0,2.5) (start) {}
node at (-.5,1) [shape=square,draw] (fh) {$\scriptstyle f h$}
node at (.5,1) [shape=square,draw] (gk) {$\scriptstyle g k$}
node at (0,2) [delta] (d) {};
\draw [] (d) to (start);
\draw [] (n1) to (0,-0.5);
\draw [] (d) to[out=305,in=90] (gk);
\draw [] (d) to[out=235,in=90] (fh);
\draw [-] (fh) to[out=270,in=125] (n1);
\draw [-] (gk) to[out=270,in=65] (n1);
\end{tikzpicture}
% chapter sums_in_frobenius_algebras (end)
%%% Local Variables:
%%% mode: latex
%%% TeX-master: "../phd-thesis"
%%% End:


%%% Local Variables:
%%% mode: latex
%%% TeX-master: "../phd-thesis"
%%% End:
