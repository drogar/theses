%!TEX root = /Users/gilesb/UofC/thesis/phd-thesis/phd-thesis.tex
\chapter{Disjointness in Inverse Categories} % (fold)
\label{cha:disjointness_in_inverse_categories}

This chapter explores coproduct like structure in inverse categories. It starts by showing,
that similar to the product, having coproducts is too restrictive a notion for inverse categories: An
inverse category with a coproduct is a preorder.
Nonetheless it is possible to define coproduct like structures in an inverse category. To introduce
this structure we define a ``disjointness'' relation between parallel maps of an inverse category
and whence a ``disjoint join'' for disjoint maps. The  next chapter will then show how a
tensor satisfying certain specific conditions gives rise to both a disjointness relation and a
disjoint join. Such a tensor provides the replacement for ``coproducts'' in an inverse category.



\section{Coproducts in inverse categories} % (fold)
\label{sec:coproducts_in_inverse_categories}

A restriction category can have coproducts and an initial object. For example, \Par (sets and
partial functions) has coproducts.

\begin{definition}\label{def:restriction_coproduct}
  In a restriction category \R, a coproduct is a \emph{restriction coproduct} when the embeddings
  $\cpa$ and $\cpb$ are total.
\end{definition}
\begin{lemma}\label{lem:restriction_coproduct_is_restriction_functor}
  A restriction coproduct $+$ in $\R$ satisfies:
  \begin{enumerate}[{(}i{)}]
    \item $\restr{f+g} = \restr{f} + \restr{g}$ which means $+$ is a restriction functor.
    \item $\nabla:A+A\to A$ is total.
    \item $?:0 \to A$ is total, where $0$ is the initial object in the restriction category.
  \end{enumerate}
\end{lemma}
\begin{proof}
  \prepprooflist
  \begin{enumerate}[{(}i{)}]
    \item \textbf{$+$ is a restriction functor.}
      Consider the diagram:
      \[
        \xymatrix @C=30pt @R=35pt{
          A \ar[rr]^{f} \ar[dr]_{\cpa} && A' \ar[dr]^{\cpa'}\\
          &A+B \ar@{.>}[rr]^{f+g} & & A' + B'\\
          B \ar[rr]_{g} \ar[ur]^{\cpb} && B'. \ar[ur]_{\cpb'}
        }
      \]
      In order to show $\restr{f+g} = \restr{f} + \restr{g}$, it suffices to show that
      $\cpa\restr{f+g} = \cpa(\restr{f} + \restr{g}) = \restr{f}\cpa$.
      \begin{align*}
        \cpa\restr{f+g} &= \restr{\cpa(f+g)}\cpa & \rfour\\
        &= \restr{f \cpa'} \cpa &\text{coproduct diagram}\\
        &= \restr{f \restr{\cpa'}} \cpa
          & \text{Lemma }\refitem{lem:restrictionvarious}{lemitem:rv_3}\\
        &= \restr{f} \cpa & \cpa'\text{ total}.
      \end{align*}
    \item   \textbf{$\nabla:A+A\to A$ is total.}
      By the definition of $\nabla$ ($=\<1 | 1 \>$) and the coproduct, the following diagram
      commutes,
      \[
        \xymatrix @C=30pt @R=35pt{
          &A + A \ar[d]_{\nabla}\\
          A \ar@{=}[r] \ar[ur]^{\cpa}&A \ar@{=}[r] & A \ar[ul]_{\cpb}
        }
      \]
      resulting in:
      \[
         \cpa \restr{\nabla}  = \restr{\cpa \nabla} \cpa = \restr{1} \cpa = \cpa.
      \]
      Similarly, $\cpb \restr{\nabla} = \cpb$, hence, the restriction of $\nabla$ is $1$ and
      therefore $\nabla$ is total.
    \item  \textbf{$\why:0 \to A$ is total.}
      This follows from
      \[
        \xymatrix @C=30pt @R=35pt{
          0 \ar[dr]_{\why} \ar[r]^{\cpb} &A + 0 \\
          & A \ar@{=}[u]
        }
      \]
      so $\why$ can be defined as the total coproduct injection.
  \end{enumerate}



\end{proof}

Recall that when an object is both initial and terminal, it is referred to as a zero object and
denoted as $\zeroob$. This gives rise to the zero map $\zeroob_{A,B}: A \to \zeroob \to B$ between any two objects.
Note that for all $f:D\to A, g:B\to C$, we have $f\zeroob_{A,B} g = \zeroob_{D,C}$ as this factors through
the zero object.

\begin{definition}\label{def:restriction_zero}
  Given a restriction category \X with a zero object, then $\zeroob$ is a \emph{restriction zero} when
  for each object $A$ in \X, $\restr{\zeroob_{A,A}} = \zeroob_{A,A}$.
\end{definition}

\begin{lemma}[Cockett-Lack\cite{cockettlack2004:restcategories3}, Lemma 2.7]\label{lem:restriction_zero_equivalences}
  For a restriction category \X, the following are equivalent:
  \begin{enumerate}[{(}i{)}]
    \item \X has a restriction zero;
    \item \X has an initial object $0$ and terminal object $1$ and each initial map $z_A$ is a
      restriction monic;
    \item \X has a terminal object $1$ and each terminal map $t_A$ is a restriction retraction.
  \end{enumerate}
\end{lemma}
% subsection coproducts (end)




\subsection{Inverse categories with restriction coproducts} % (fold)
\label{sub:inverse_categories_with_restriction_coproducts}


\begin{proposition}\label{prop:inverse_category_with_coproducts_is_pre-order}
  An inverse category \X with restriction coproducts is a preorder.
\end{proposition}
\begin{proof}
  By Lemma~\ref{lem:restriction_coproduct_is_restriction_functor}, $\nabla$ is total and
  therefore $\nabla\inv{\nabla} = 1$. From the coproduct diagrams, $\cpa\nabla =1$ and
  $\cpb\nabla = 1$. But this gives $\inv{\nabla}\inv{\cpa} = \inv{(\cpa\nabla)} = 1$ and
  similarly $\inv{\nabla}\inv{\cpb} =1$. Hence, $\inv{\nabla} = \cpa$ \emph{and} $\inv{\nabla} =
  \cpb$.

  This means for parallel maps $f,g:A \to B$,
  \[
    f = \cpa [f,g] = \inv{\nabla} [f,g] = \cpb [f,g] = g
  \]
  and therefore \X is a preorder.
\end{proof}
% subsection inverse_categories_with_restriction_coproducts (end)
% section coproducts_in_restriction_categories (end)

\section{Disjointness in an inverse category} % (fold)
\label{sec:disjointness_in_an_inverse_category}
This section and the next first add a relation, \emph{disjointness}, and then an operation,
\emph{disjoint join}, on parallel maps in  an inverse category with a restriction zero and zero
maps. The disjoint join is evocative of the join as defined in
Section~\ref{sub:joins_in_restriction_categories}. This section will begin by defining disjointness
on parallel maps and then show this is equivalent to a definition of disjointness on the restriction
idempotents.

From this point forward in the thesis, we will work with a number of relations and operations on
parallel pairs of maps. Suppose there is a relation $\lozenge$ between maps $f,g:B\to C$, i.e., $f
\lozenge g$. Then, $\lozenge$ will be referred to as \emph{stable} whenever given $h:A \to B$, then $h f
\lozenge h g$. As well, $\lozenge$ will be referred to as \emph{universal} whenever given $k:C \to D$, then $f
k \lozenge g k$.


\begin{definition}\label{def:disjointness_relation}
  In an inverse category \X with zero maps, the relation $\cdperp$ between two parallel
  maps $f, g:A \to B$ is called a \emph{disjointness relation} when it satisfies the following
  properties:
  \begin{align*}
    \axiom{Dis}{1}\quad &\text{ For all }f:A\to B,\ f\cdperp 0;&\text{(Zero is disjoint to all maps)}\\
    \axiom{Dis}{2}\quad &f\cdperp g\text{ implies }\rst{f} g = 0; &\text{(Disjoint maps have no intersection)}\\
    \axiom{Dis}{3}\quad &f\cdperp g,\ f' \le f,\ g' \le g\text{ implies }f' \cdperp
    g';&\text{(Disjointness is down closed)}\\
    \axiom{Dis}{4}\quad &f\cdperp g\text{ implies }g \cdperp f; &\text{(Symmetric)}\\
    \axiom{Dis}{5}\quad &f\cdperp g\text{ implies }h f \cdperp h g; &\text{(Stable)}\\
    \axiom{Dis}{6}\quad &f\cdperp g\text{ implies }\rst{f} \cdperp \rst{g}
      \text{ and }\rg{f}\cdperp\rg{g}; &\text{(Closed under range and restriction)}\\
    \axiom{Dis}{7}\quad &\rst{f}\cdperp \rst{g},\ \rg{h}\cdperp \rg{k}\text{ implies }f h
      \cdperp g k &\text{(Determined by restriction/range).}
  \end{align*}
  When $f\cdperp g$, we will say $f$ is \emph{disjoint} from $g$.
\end{definition}

\begin{lemma}\label{lem:disjointness_equivalent_axioms}
  In \ref{def:disjointness_relation}, given $\axiom{Dis}{1-5}$, the axioms $\axiom{Dis}{6}$ and
  $\axiom{Dis}{7}$ may be replaced by:
  \begin{align*}
    \axiom{Dis}{6'}\quad &f\cdperp g\text{ if and only if }\rst{f} \cdperp \rst{g}
      \text{ and }\rg{f}\cdperp\rg{g}.
  \end{align*}
\end{lemma}
\begin{proof}
  Given $\axiom{Dis}{6}$ and $\axiom{Dis}{7}$, the \emph{only if} direction of $\axiom{Dis}{6'}$ is
  immediate. To show the \emph{if} direction, assume $\rst{f} \cdperp \rst{g}$ and
  $\rg{f}\cdperp\rg{g}$. This also means that $\rst{\rst{f}} \cdperp \rst{\rst{g}}$. Then, by
  \axiom{Dis}{7}, $\rst{f} f \cdperp \rst{g}g$ and therefore $f \cdperp g$, showing $\axiom{Dis}{6}$
  and $\axiom{Dis}{7}$ imply $\axiom{Dis}{6'}$.

  Conversely, assume \axiom{Dis}{6'}. Then, \axiom{Dis}{6} follows immediately. To show
  \axiom{Dis}{7}, assume $\rst{f}\cdperp \rst{g},\ \rg{h}\cdperp \rg{k}$. As
  $\rst{f h} \le \rst{f}$ and $\rst{g k}\le\rst{g}$, by \axiom{Dis}{3}, it follows that
  $\rst{f h} \cdperp \rst{g k}$. Similarly, $\wrg{f h} \le \rg{h}$ and $\wrg{g k} \le \rg{k}$,
  giving us $\wrg{f h}\cdperp \wrg{g k}$. Then, from \axiom{Dis}{6'} we may conclude
  $f h \cdperp g k$, showing \axiom{Dis}{7} holds.
\end{proof}
\begin{lemma}\label{lem:disjointness_various}
  In an inverse category \X with $\cdperp$ a disjointness relation:
  \begin{enumerate}[{(}i{)}]
    \item $f \cdperp g$ if and only if $\inv{f}\cdperp \inv{g}$; \label{lemitem:djv_inverses}
    \item $f \cdperp g$ implies $f h \cdperp g h$ (Universal);\label{lemitem:djv_universal}
    \item $f \cdperp g$ implies $f\rg{g} = 0$; \label{lemitem:djv_disjoint_composition_is_0}
    \item if $m,n$ are monic, then $f m \cdperp g n$ implies $\rst{f} \perp \rst{g}$;
      \label{lemitem:djv_monic_implies}
    \item if $m,n$ are monic, then $\inv{m} f \cdperp \inv{n}g$ implies $\rg{f} \perp \rg{g}$.
      \label{lemitem:djv_inv_monic_implies}
  \end{enumerate}
\end{lemma}
\begin{proof}
  \prepprooflist
  \begin{enumerate}[{(}i{)}]
    \item Assume $f \cdperp g$. By \axiom{Dis}{6}, $\rst{f} \cdperp \rst{g}$ and
      $\rg{f}\cdperp\rg{g}$. Since $\rg{f} = \rst{\inv{f}}$ and $\rst{f} = \wrg{\inv{f}}$, this
      means $\rst{\inv{f}} \cdperp \rst{\inv{g}}$ and $\wrg{\inv{f}}\cdperp\wrg{\inv{g}}$. By
      \axiom{Dis}{6'} from Lemma~\ref{lem:disjointness_equivalent_axioms},
      $\inv{f} \cdperp \inv{g}$. The converse follows with  a similar argument.
    \item Assume $f \cdperp g$. By \ref{lemitem:djv_inverses}, $\inv{f}\cdperp\inv{g}$. By
      \axiom{Dis}{5}, $\inv{h}\inv{f}\cdperp\inv{h}\inv{g}$, giving $\inv{(f h)} \cdperp
      \inv{(g h)}$. Applying \ref{lemitem:djv_inverses}, we now have $f h \cdperp g h$.
    \item Assume $f \cdperp g$. From \ref{lemitem:djv_inverses} and reflexivity, we know that
      $\inv{g} \cdperp \inv{f}$ and therefore $\rst{\inv{g}}\inv{f} = \rg{g}\inv{f}= 0$. However, in
      an inverse category, $\inv{0} = 0$ and therefore $0 = \inv{(\rg{g}\inv{f})} = f \inv{\rg{g}} =
      f \rg{g}$.
    \item Assume $f m \cdperp g n$ where $m, n$ are monic. By \axiom{Dis}{6}, this gives
      $\rst{f m} \cdperp \rst{g n}$. By Lemma~\ref{lem:restrictionvarious},
      $\rst{f m} = \rst{f \rst{m}} = \rst{f 1} = \rst{f}$ and therefore $\rst{f} \cdperp \rst{g}$.
    \item By assumption, we have $\inv{m} f \cdperp  \inv{n}g$ and therefore $\inv{f}m \cdperp
      \inv{g} n$. By \ref{lemitem:djv_monic_implies}, this means $\rst{\inv{f}} \cdperp
      \rst{\inv{g}}$ and hence $\rg{f} \cdperp \rg{g}$.
  \end{enumerate}
\end{proof}

We may equivalently define a disjointness relation by a relation on the restriction idempotents, \open{A}.

\begin{definition}\label{def:disjointness_in_open_x}
  Given an inverse category \X, a relation $\ocdperpsub{A} \subseteq \open{A}\times\open{A}$ for
  each $A \in \mathrm{ob}(\X)$, is an \emph{open disjointness} relation when for all $e, e' \in
  \open{A}$
  \begin{align*}
    \axiom{$\mathcal{O}$dis}{1}\quad &1 \ocdperpsub{A} 0; &\text{(Zero disjoint to identity)}\\
    \axiom{$\mathcal{O}$dis}{2}\quad &e \ocdperpsub{A} e' \text{ implies }e' \ocdperpsub{A} e;
    &\text{(Symmetric)}\\
    \axiom{$\mathcal{O}$dis}{3}\quad &e \ocdperpsub{A} e' \text{ implies }e e' = 0; &\text{(Disjoint implies no intersection)}\\
    \axiom{$\mathcal{O}$dis}{4}\quad &e \ocdperpsub{A} e' \text{ implies }\rst{f e} \ocdperpsub{B}
      \rst{f e'}\text{ for all }f:B \to A; &\text{(Stable)}\\
    \axiom{$\mathcal{O}$dis}{5}\quad &e \ocdperpsub{A} e' \text{ implies }\wrg{e g} \ocdperpsub{C}
      \wrg{e' g}\text{ for all }g:A \to C; &\text{(Universal)}\\
    \axiom{$\mathcal{O}$dis}{6}\quad &e \ocdperpsub{A} e',\ e_1 \le e,\ e_1' \le e'
      \text{ implies }e_1 \ocdperpsub{A} e_1' &\text{(Down closed)}.
  \end{align*}
\end{definition}

We will normally write $\ocdperp$ rather than $\ocdperpsub{A}$ where the object is clear.

\begin{proposition}\label{prop:disjointness_is_open_disjointness}
  If $\cdperp$ is a disjointness relation in $\X$, then $\ocdperp = \cdperp \intersection
  (\union_{A\in\X} \open{A}\times \open{A})$, the restriction of $\cdperp$ to the restriction
  idempotents, is an open disjointness relation.
\end{proposition}
\begin{proof}
  \prepprooflist
  \setlist[enumerate,1]{leftmargin=1.75cm}
  \begin{enumerate}
    \item[\axiom{$\mathcal{O}$dis}{1}] This follows immediately from \axiom{Dis}{1} by taking
      $f = 1$.
    \item[\axiom{$\mathcal{O}$dis}{2}] Reflexivity follows directly from  \axiom{Dis}{4}.
    \item[\axiom{$\mathcal{O}$dis}{3}] By \axiom{Dis}{2}, $0 = \rst{e} e' = e e'$.
    \item[\axiom{$\mathcal{O}$dis}{4}] Given $e \cdperp e'$, this means $f e \cdperp f e'$ by
      \axiom{Dis}{5}. Then, \axiom{Dis}{6} gives a conclusion of $\rst{f e} \cdperp \rst{f e'}$.
    \item[\axiom{$\mathcal{O}$dis}{5}] This follows from the above item, using $\inv{g}$ for $f$.
      This means $\rst{\inv{g}e} \cdperp \rst{\inv{g}e'}$. But this gives $\rst{\inv{(e
      g)}} \cdperp \rst{\inv{(e' g)}}$. Recalling from
      Lemma~\ref{lem:inverse_categories_are_range_categories} that $\rg{k} = \rst{\inv{k}}$, the
      conclusion is $\wrg{e g} \cdperp \wrg{e' g}$.
    \item[\axiom{$\mathcal{O}$dis}{6}] Assuming $e \cdperp e'$ and $e_1 \le e,\ e'_1 \le e'$, by
      \axiom{Dis}{3}, $e_1 \cdperp e'_1$.
  \end{enumerate}
  Therefore, $\ocdperp = \cdperp \intersection (\union_{A\in\X} \open{A}\times \open{A})$ acts as an
  open disjointness relation on $\open{A}^2$.

\end{proof}

For the converse, if $\ocdperp$ is an open disjointness relation in $\X$, then define a relation
$\perpab \subseteq \union_{A,B\in\X} \X(A,B)\times \X(A,B)$ on parallel maps by
\[
  f\perpab g \iff  \rst{f}\ocdperp\rst{g} \text{ and } \rg{f}\ocdperp\rg{g}.
\]
Note the use of the $\rg{f}\ocdperp\rg{g}$. By \rrone, $\rst{\rg{f}} = \rg{f}$ and
therefore it is a restriction idempotent in $\open{B}$.
The relation $\perpab$ is a disjointness relation:

\begin{proposition}\label{prop:extended_disjointness_is_a_disjointness_relation}
  If $\ocdperp$ is an open disjointness relation in \X, then
  \[
    f\perpab g \iff  \rst{f}\ocdperp\rst{g}  \text{ and }  \rg{f}\ocdperp\rg{g}
  \]
  is a disjointness relation in \X.
\end{proposition}
\begin{proof}
  \prepprooflist
  \setlist[enumerate,1]{leftmargin=1.5cm}
  \begin{enumerate}
    \item[\axiom{Dis}{1}] This requires showing $f \cdperp 0$ for any $f$. The assumption is that
      $1 \ocdperp 0$  and therefore $\rst{f} \ocdperp 0$ and $\rg{f} \ocdperp 0$, as $\rst{f}\le 1$
      and $\rg{f}\le 1$. This gives $f \cdperp 0$.
    \item[\axiom{Dis}{2}] Assume $f \cdperp g$, i.e., $\rst{f}\ocdperp \rst{g}$. Then, $\rst{f}g =
      \rst{f}\rst{g}g = 0 g = 0$.
    \item[\axiom{Dis}{3}] This axiom assumes $f \cdperp g$, $f' \le f$ and $g' \le g$. By
      Lemma~\refitem{lem:restriction_cats_are_partial_order_enriched}{lemitem:rst_ordering_2}
      $\rst{f'} \le \rst{f}$ and $\rst{g'} \le \rst{g}$. Then,  $\rst{f'} \ocdperp \rst{g'}$ by
      \axiom{$\mathcal{O}$dis}{6}, as  $\rst{f} \ocdperp \rst{g}$. By
      Lemma~\refitem{lem:ordering_of_restriction_and_range}{lemitem:ordering_2}, both $\wrg{f'}
      \le \rg{f}$ and $\wrg{g'} \le \rg{g}$ and by \axiom{$\mathcal{O}$dis}{6}, $\wrg{f'} \ocdperp
      \wrg{g'}$ as $\rg{f} \ocdperp \rg{g}$. This means $f' \cdperp g'$.
    \item[\axiom{Dis}{4}] Reflexivity of $\cdperp$ follows immediately from the reflexivity of
      $\ocdperp$.
    \item[\axiom{Dis}{5}] Assume $f \cdperp g$, i.e., $\rst{f}\ocdperp \rst{g}$ and $\rg{f}
      \ocdperp \rg{g}$. By \axiom{$\mathcal{O}$dis}{4}, $\rst{hf}\ocdperp \rst{h g}$. By
      Lemma~\refitem{lem:ordering_of_restriction_and_range}{lemitem:ordering_1}, $\wrg{h f}
      \le \rg{f}$ and $\wrg{h g} \le \rg{g}$. Therefore $\wrg{h f}\ocdperp \wrg{h g}$ by
      \axiom{$\mathcal{O}$dis}{6} and therefore $h f \cdperp h g$.
    \item[\axiom{Dis}{6}] Assume $f\perpab g$ which gives both $\rst{f}\ocdperp\rst{g}$ and
      $\rg{f}\ocdperp\rg{g}$. By Lemma~\ref{lem:basic_range_category_properties} for any map
      $h$, $\rg{\rst{h}} = \rst{h}$ and by Lemma~\ref{lem:restrictionvarious} we have $\rst{\rst{h}}
      = \rst{h}$. Thus, we have both $\rst{\rst{f}} \ocdperp \rst{\rst{g}}$ and $\rg{\rst{f}}
      \ocdperp \rg{\rst{g}}$ and therefore $\rst{f} \perpaa \rst{g}$. Similarly for any map $h$,
      \rrone gives $\rst{\rg{h}} = \rg{h}$ and Lemma~\ref{lem:basic_range_category_properties} gives
      $\rg{\rg{h}} = \rg{h}$. This means $\rst{\rg{f}} \ocdperp \rst{\rg{g}}$ and $\rg{\rg{f}}
      \ocdperp \rg{\rg{g}}$ which gives $\rg{f} \perpvv{B}{B} \rg{g}$.
    \item[\axiom{Dis}{7}] Start with the assumption $\rst{f} \cdperp \rst{g}$ and $\rg{h} \cdperp
      \rg{k}$, which gives $\rst{f} \ocdperp \rst{g}$ and $\rg{h} \ocdperp \rg{k}$. By
      Lemma~\refitem{lem:restriction_cats_are_partial_order_enriched}{lemitem:rst_ordering_3}, both
      $\rst{f h} \le \rst{f}$ and $\rst{g k} \le \rst{g}$. Therefore, $\rst{f h} \ocdperp
      \rst{g k}$ by \axiom{$\mathcal{O}$dis}{6}. By
      Lemma~\refitem{lem:ordering_of_restriction_and_range}{lemitem:ordering_1}, $\wrg{f h} \le
      \rg{h}$ and $\wrg{g k} \le \rg{k}$, giving us $\wrg{f h}\ocdperp \wrg{g k}$ also by
      \axiom{$\mathcal{O}$dis}{6}. This means $f h \cdperp g k$.
  \end{enumerate}
\end{proof}
\begin{theorem}\label{thm:open_disjointness_is_disjointness}
   To give a disjointness relation $\perp$ on $\X$ is to give an open disjointness relation
   $\ocdperp$ on \X.
\end{theorem}
\begin{proof}
  This requires showing there is a bijection between disjointness relations and open disjointness
  relations. That is, give a disjointness relation $\perp$, it generates an open disjointness
  relation, $\ocdperp$. We then need to show that the disjointness relation generated from
  $\ocdperp$ is in fact $\perp$.

  Starting with the disjointness relation $\perp$, by
  Proposition~\ref{prop:disjointness_is_open_disjointness}, this is an open disjointness $\ocdperp =
  \perp \intersection (\union_{A\in\X}\open{A}\times\open{A})$.

  By Proposition~\ref{prop:extended_disjointness_is_a_disjointness_relation}, $\perpab$ defined by
  \[
    f\perpab g \iff  \rst{f}\ocdperp\rst{g}  \text{ and }  \rg{f}\ocdperp\rg{g}.
  \]
  is a disjointness relation on $\X$.

  Assume $f\perp g$. By \axiom{Dis}{6} and Proposition~\ref{prop:disjointness_is_open_disjointness},
  $\rst{f}\ocdperp\rst{g}$ and $\rg{f}\ocdperp\rg{g}$. Then by its definition, we have $f \perpab g$.

  Assume $f \perpab g$, which required that both $\rst{f}\ocdperp\rst{g}$ and
  $\rg{f}\ocdperp\rg{g}$. Therefore $\rst{f}\perp\rst{g}$ and  $\rg{f}\perp\rg{g}$. By
  Proposition~\ref{lem:disjointness_various}, $f \perp g$.

  We have shown $f \perp g \iff f\perpab g$. We may also conclude that if we started with an open
  disjointness relation $\ocdperp$ and used it to construct $\perpab$, then the relation $\perpab
  \intersection (\union_{A\in\X}\open{A}\times\open{A})$ would again be $\ocdperp$.

  Hence there is a bijection between disjointness relations and open disjointness relations on
  an inverse category $\X$.
\end{proof}


Disjointness is additional structure on a restriction category, i.e., it is possible to have more
than one disjointness relation on the category. To see this, consider the trivial disjointness
relation, where $f \perp_0 g$ if and only if $f = 0$ or $g = 0$. As $\rst{0} = 0 = \rg{0}$, $f \le 0
\iff f = 0$ and $h 0 =0$ for any map $h$, axioms \axiom{Dis}{1} through \axiom{Dis}{7} are
immediately satisfied. $\perp_0$ is definable on any inverse category with zero maps, hence is
definable on \pinj. However, \pinj does have a more useful disjointness relation:
\begin{example}[\pinj has a disjointness relation]\label{ex:pinj_has_a_disjointness_relation}
  Consider the inverse category \pinj, introduced in Example~\ref{ex:pinj_is_a_restriction_category}
  and Example~\ref{ex:pinj_is_a_discrete_inverse_category}.
  Note the restriction zero is the empty set, $\emptyset$, with initial and terminal maps being
  $\emptyset$ and therefore $\zeroob_{A,B} = \emptyset$.

  Define the disjointness relation $\perp$ by $f \perp g$ if and only if
  $\rst{f}\intersection\rst{g}=\emptyset$ and $\rg{f}\intersection\rg{g} = \emptyset$. It is then
  reasonably straightforward to verify \axiom{Dis}{1} through \axiom{Dis}{7}. For example, take
  \axiom{Dis}{7}:
  \begin{proof}
    We are given $\rst{f}\cdperp\rst{g}$ and $\rg{h}\cdperp\rg{k}$. This means
    \[
      \rst{f}\intersection\rst{g}=\emptyset \text{ and }\rg{h}\intersection\rg{k} = \emptyset.
    \]
    As discussed in Example~\ref{ex:pinj_has_meets}, both
    $\rst{m n} \subseteq \rst{m}$ and  $\wrg{m n} \subseteq \rg{n}$. Hence,
    \begin{align*}
      \rst{f h}\intersection\rst{g k}&\subseteq \rst{f}\intersection \rst{g}= \emptyset\\
      \wrg{f h}\intersection\wrg{g k}&\subseteq \rg{h}\intersection \rg{k}= \emptyset.
    \end{align*}
    Therefore, $f h \cdperp g k$.
  \end{proof}

  Alternatively, define $f \perp_0 g$ on \pinj as well: $f \perp_0 g$ if and only if one of $f$
  or $g$ is the restriction $\zeroob$, $\emptyset$. As $\zeroob = \rst{\zeroob} = \rg{\zeroob} = h
  \zeroob = \zeroob k$, all of the seven disjointness axioms are easily verifiable. This shows that
  disjointness is additional structure on a restriction category.
\end{example}

Although disjointness is additional structure, one can use the disjointness structure of the base
categories to define a disjointness structure on the product category.

\begin{lemma}\label{lem:disjointness_is_derivable_on_a_product_category}
  If $\X$ and $\Y$ are inverse categories with restriction zeros and respective disjointness
  relations $\perp$ and $\perp'$, then there is a disjointness relation
  $\perp_{\times}$ on $\X\times\Y$.
\end{lemma}
\begin{proof}
  The restriction, the inverse and the restriction zero on the product category are defined pointwise.
  \begin{itemize}
    \item If $(f,g)$ is a map in $\X\times\Y$, then $\inv{(f,g)} = (\inv{f}, \inv{g})$;
    \item If $(f,g)$ is a map in $\X\times\Y$, then $\rst{(f,g)} = (\rst{f}, \rst{g})$;
    \item The map $(\zeroob_X,\zeroob_Y)$ is the restriction zero in $\X\times\Y$.
  \end{itemize}

  Following this pattern, for $(f,g)$ and $(h,k)$ maps in $\X\times\Y$, $(f,g) \perp_{\times}(h,k)$
  iff $f\perp h$ and $g\perp' k$.

  Verifying the disjointness axioms is straightforward, we show axioms 2 and 5. Proofs of the
  others are similar.
  {
  \setlist[itemize,1]{leftmargin=1.5cm}
  \begin{itemize}
    \item [\axiom{Dis}{2}]: Given $(f,g)\perp_{\times}(h,k)$,
      $\rst{(f,g)}(h,k) =  (\rst{f},\rst{g})(h,k) = (\rst{f} h, \rst{g} k) = (\zeroob, \zeroob) = \zeroob$.
    \item [\axiom{Dis}{5}]: The assumption is $(f,g)\perp_{\times}(h,k)$. For the map $z = (x,y)$
      in $\X \times \Y$, \axiom{Dis}{5} in the base categories gives $x f \perp x h$ and $y g \perp
      y k$. Thus
      \[
        z(f,g) = (x f, y g) \perp_{\times} (x h, y k) = z(h,k).
      \]
  \end{itemize}
  }
\end{proof}
% subsection disjointness_relations (end)

\section{Disjoint joins} % (fold)
\label{sec:disjoint_joins}

This section will now consider additional structure on the inverse category, the disjoint join,
dependent upon the disjointness relation. Although we have only considered with binary disjointness
up to this point, extending disjointness to sets of maps is done by considering disjointness
pairwise: $\cdperp \{f_1,f_2,\ldots,f_n\} \definedas f_i \cdperp f_k$ whenever $i \ne j$.

\begin{definition}\label{def:disjoint_join}
  An inverse category with $\X$ with a restriction $0$ and a disjointness relation $\perp$ has
  \emph{disjoint joins} when there is a binary operator on disjoint parallel maps:
  \[
    \infer{f \djoin g : A \to B}{f: A\to B,\ g: A\to B,\ f \perp g}
  \]
  such that the following hold:
  \begin{align*}
    \axiom{DJ}{1}\quad & f \le f \djoin g\text{ and }g \le f \djoin g; \\
    \axiom{DJ}{2}\quad & f \le h,\ g \le h\text{ and }f\perp g\text{ implies }f \djoin g \le h;\\
    \axiom{DJ}{3}\quad & h(f \djoin g) = h f \djoin h g. \text{ (Stable)} \\
    \axiom{DJ}{4}\quad & \cdperp \{f, g, h\}\text{ if and only if }f \perp (g\djoin h).\\
  \end{align*}
  The binary operator, $\djoin$, is called the \emph{disjoint join}.
\end{definition}

Given a specific disjointness relation on a category, there is only one disjoint join:
\begin{lemma}\label{lem:disjoint_join_is_unique}
  Suppose \X in an inverse category with a disjointness relation $\perp$, then if  $\djoin$ and
  $\altjoin$ are disjoint joins for $\perp$ then $\djoin = \altjoin$.
\end{lemma}
\begin{proof}
  \axiom{DJ}{1} gives us:
  \[
    f,g \le f\djoin g\text{ and }f,g \le f \altjoin g.
  \]
  By \axiom{DJ}{2}, $f \djoin g \le f \altjoin g$ and
  $f \altjoin g \le f\djoin g$, hence $f \djoin g = f \altjoin g$.
\end{proof}
\begin{lemma}\label{lem:join_is_associative_and_commutative_monoid}
  In an inverse category with disjoint joins, the disjoint join respects the restriction and is
  universal. Additionally, it is a partial associative and
  commutative operation, with identity $\zeroob$. That is, the following hold:
  \begin{enumerate}[{(}i{)}]
    \item $\rst{f\djoin g} = \rst{f} \djoin \rst{g}$;
    \item $(f \djoin g)k = f k \djoin g k$ (Universal);
    \item $f \perp g,\ g\perp h,\ f\perp h$ implies that $(f\djoin g)\djoin h = f\djoin(g\djoin
      h)$; \label{lemitem:associative_join}
    \item $f \perp g$ implies $f \djoin g = g \djoin f$; \label{lemitem:commutative_join}
    \item $f \djoin \zeroob = f$. \label{lemitem:identity_for_join}
  \end{enumerate}
\end{lemma}
\begin{proof}
  \prepprooflist
  \begin{enumerate}[{(}i{)}]
    \item As $\rst{f}, \rst{g} \le \rst{f\djoin g}$, \axiom{DJ}{2} gives
      $\rst{f} \djoin \rst{g} \le \rst{f \djoin g}$. To show the other direction, consider
      \begin{align*}
        \rst{f}(\rst{f} \djoin \rst{g}) (f \djoin g)
        &= (\rst{f}\,\rst{f} \djoin \rst{f}\rst{g})(f \djoin g) &\text{\axiom{DJ}{3}}\\
        &= \rst{f} (f\djoin g) &\text{Lemma~\ref{lem:restrictionvarious}, \axiom{Dis}{2}}\\
        &= f.
      \end{align*}
      Hence, $f \le (\rst{f} \djoin \rst{g}) (f \djoin g)$ and similarly, so is $g$. By
      \axiom{DJ}{2} and $\rst{f} \djoin \rst{g}$ being a restriction idempotent,
      \[
        f \djoin g \le (\rst{f} \djoin \rst{g}) (f \djoin g) \le f \djoin g
      \]
      and therefore $f \djoin g = (\rst{f} \djoin \rst{g}) (f \djoin g)$. By
      Lemma~\ref{lem:restriction_cats_are_partial_order_enriched}, $\rst{f\djoin g} \le \rst{f}
      \djoin \rst{g}$ and so $\rst{f\djoin g} = \rst{f}\djoin \rst{g}$.
    \item First consider when $f, g$ and $k$ are restriction idempotents, say $e_0, e_1 $ and $e_2$.
      Then, $(e_0 \djoin e_1)e_2 = e_2(e_0 \djoin e_1) = e_2 e_0 \djoin e_2 e_1 =
      e_0 e_2 \djoin e_1 e_2$.
      Next, note that for general $f,g,h$, by \axiom{DJ}{2} $f k \djoin g k \le (f\djoin g) k$ as both
      $f k, g k \le (f\djoin g) k$.
      By Lemma~\ref{lem:restriction_cats_are_partial_order_enriched}, we need only show that their
      restrictions are equal:
      \begin{align*}
        \rst{(f\djoin g)k} &= \rst{\rst{f\djoin g}(f\djoin g) k } & \rone \\
        & = \rst{f\djoin g}\, \rst{(f\djoin g)k } & \rthree\\
        & = (\rst{f}\djoin \rst{g}) \rst{(f\djoin g)k }  & \text{previous item}\\
        & = \rst{f}\,\rst{(f\djoin g)k } \djoin \rst{g}\, \rst{(f\djoin g)k }
          &\text{idempotent universal}\\
        & = \rst{\rst{f} (f\djoin g)k} \djoin \rst{\rst{g} (f\djoin g)k} & \rthree\\
        & = \rst{f k} \djoin \rst{g k}\\
        & = \rst{f k \djoin g k}.
      \end{align*}
      Therefore, as the restrictions are equal, $(f\djoin g)k = f k \djoin g k$.
    \item \emph{Associativity}: Note that \axiom{DJ}{4} shows that both sides of the equation
      exist.
      From \axiom{DJ}{1} $f\djoin g, h \le (f\djoin g)\djoin h$, which also means
      $f, g \le (f\djoin g)\djoin h$. Similarly, $g\djoin h \le (f\djoin g)\djoin h $ and then $f
      \djoin (g\djoin h)\le (f\djoin g)\djoin h$. Conversely, $f,g,h \le f \djoin (g\djoin h)$ and
      therefore $(f\djoin g)\djoin h \le f \djoin (g\djoin h)$ and both sides are equal.
    \item \emph{Commutativity}: Note first that both $f$ and $g$ are less than or equal to each of
      $f\djoin g$ and $g \djoin f$, by \axiom{DJ}{1}. By \axiom{DJ}{2}, $f \djoin g \le
      g\djoin f$ and $g\djoin f \le f \djoin g$ and thus $f \djoin g = g \djoin f$.
    \item \emph{Identity}: By \axiom{DJ}{1}, $f \le f \djoin \zeroob$. As $\zeroob \le f$ and $f \le
      f$, by \axiom{DJ}{2}, $f \djoin \zeroob \le f$ and therefore $f = f \djoin \zeroob$.
  \end{enumerate}
\end{proof}

Note that the previous lemma and proof of associativity allows a simple inductive argument which
shows that having binary disjoint joins extends to disjoint joins of an arbitrary
finite collection of disjoint maps.

Recalling our notation for disjointness of a set of maps,
$\djoin \{f_i\}$ will mean the disjoint join of all maps $f_i$, i.e.,
$f_1 \djoin f_2 \djoin \cdots \djoin f_n$.

\begin{lemma}\label{lem:arbitrary_disjoint_joins}
  In an inverse category $\X$ with disjoint joins,
  \begin{enumerate}[{(}i{)}]
    \item $\cdperp \{f_i\}$ if and only if  $\djoin \{f_i\}$ is
      defined,\label{lemitem:arbitrary_disjoint_joins_1}
    \item if $f_i, g_j : A \to B$ and $\cdperp_{i\in I} \{f_i\}$ and $\cdperp_{j\in J} \{g_j\}$,
      then $\djoin_{i\in I} \{f_i\} \cdperp \djoin_{j\in J} \{g_j\}$ if and only $f_i \cdperp g_j$
      for all $i\in I$ and $j\in J$. \label{lemitem:arbitrary_disjoint_joins_2}
  \end{enumerate}
\end{lemma}
\begin{proof}
  For \ref{lemitem:arbitrary_disjoint_joins_1}, using \axiom{Dj}{4}, proceed as in the proof of
  Lemma~\refitem{lem:join_is_associative_and_commutative_monoid}{lemitem:associative_join},
  inducting on $n$.

  To show \ref{lemitem:arbitrary_disjoint_joins_2}, first assume $\djoin \{f_i\} \cdperp \djoin
  \{g_j\}$. By \axiom{Dj}{4} and associativity, $\djoin \{f_i\} \cdperp g_j$ for each
  $j$. Using the reflexivity of $\cdperp$, \axiom{Dj}{4} and associativity, $f_i \cdperp
  g_j$ for each $i$ and $j$.

  Next, assume $f_i \cdperp g_j$ for each $i$ and $j$. Then by \axiom{Dj}{4} and associativity, $f_i
  \cdperp \djoin \{g_j\}$ for each $i$. Applying \axiom{Dj}{4} again, $\djoin \{f_i\} \cdperp
  \djoin \{g_j\}$.
\end{proof}

Clearly the product of two inverse categories with disjoint joins has a disjoint join:
\begin{lemma}\label{lem:disjoint_join_is_in_product_category}
  Given $\X, \Y$ are inverse categories with disjoint joins, $\djoin$ and $\djoin'$ respectively,
  then the category $\X \times \Y$ is an inverse category with disjoint joins.
\end{lemma}
\begin{proof}
  From Lemma~\ref{lem:disjointness_is_derivable_on_a_product_category}, $\X\times\Y$ has a
  disjointness relation that is defined point-wise.  Define $\djoin_{\times}$ the
  disjoint join on $\X\times\Y$ by
  \begin{equation}
    (f,g)\djoin_{\times}(h,k) = (f \djoin h, g \djoin' k). \label{eq:disjoint_join_on_product}
  \end{equation}
  It remains to prove each of the axioms in Definition~\ref{def:disjoint_join} hold.
  \setlist[itemize,1]{leftmargin=1.5cm}
  \begin{itemize}
    \item [\axiom{DJ}{1}] From Equation~\ref{eq:disjoint_join_on_product},  since
      $f,h \le f \djoin h$ and $g,k \le g \djoin' k$, both $(f,g) \le (f,g)\djoin_{\times}(h,k)$
      and  $(h,k) \le (f,g)\djoin_{\times}(h,k)$.
    \item [\axiom{DJ}{2}] Suppose $(f,g) \le (x,y)$, $(h,k) \le (x,y)$ and $(f,g) \perp_{\times}
      (h,k)$. Then regarding it point-wise, $(f,g)\djoin_{\times}(h,k) = (f \djoin h, g
      \djoin' k) \le (x,y)$.
    \item [\axiom{DJ}{3}] Calculating,
      \begin{multline*}
        (x,y)\left((f,g)\djoin_{\times}(h,k)\right) = (x(f \djoin h), y(g  \djoin' k)) = \\
        (x f \djoin x h, y g \djoin' y k) = (x f,y g)\djoin_{\times}(x h,y k) =\\
        ((x,y)(f,g))\djoin_{\times}((x,y)(h,k)).
      \end{multline*}
  \item [\axiom{DJ}{4}] Given $\perp_{\times}\!\![(f,g),(h,k),(x,y)]$,  both $f \perp (h\djoin
      x)$ and $g \perp' (k\djoin' y)$. Hence, $(f,g) \perp_{\times}((h,k)\djoin_{\times}(x,y))$. The
      opposite direction is similar.
  \end{itemize}
\end{proof}

\begin{example}[\pinj has a disjoint join]\label{ex:pinj_has_a_disjoint_join}
  Continuing from Example~\ref{ex:pinj_has_a_disjointness_relation}, we show that \pinj has disjoint
  joins. If $f = \{(a,b)\}$ and $g=\{(a',b'\}$ are disjoint parallel maps in \pinj from $A$ to $B$,
  define $f\djoin g \definedas \{(a'',b'') | (a'',b'') \in f\text{ or }(a'',b'') \in g\}$, i.e., the
  union of $f$ and $g$.

  This is still a partial injective map, due to the requirement of disjointness. Recall that $f\perp
  g$ means that $\rst{f}\intersection \rst{g} = \emptyset$ and $\rg{f}\intersection\rg{g} =
  \emptyset$ and that the respective meets will also be $\emptyset$. The empty meet of the
  restrictions means that $f\djoin g$ is still a partial function,  as each $a''$ will appear only
  once. The empty meet of the ranges gives us that $f\djoin g$ is injective, because each $b''$ is
  unique.

  The axioms for disjoint joins all hold:
  \begin{enumerate}
    \item[\axiom{DJ}{1}] By construction, both $f$ and $g$ are less than $f\djoin g$.
    \item[\axiom{DJ}{2}] $f \le h,\ g \le h$ means that $h$ must contain all of the $(a,b)\in f$ and
      $(a',b') \in g$ and therefore $f\djoin g \le h$.
    \item[\axiom{DJ}{3}] Suppose $h:C\to A = \{(c,\dot{a})\}$. Then
      \begin{align*}
        h(f \djoin g) &= \{(c,\dot{b}) |(\exists a,\dot{a}.\dot{a} = a, (a,\dot{b}) \in f,
        (c,\dot{a}) \in h) \\
        & \qquad\qquad \text{ or }(\exists a',\dot{a}.\dot{a} = a', (a',\dot{b}) \in g, (c,\dot{a}) \in h)\}\\
        & = \{(c,\dot{b}) |\exists a, \dot{a}.\dot{a} = a, (a,\dot{b}) \in f, (c,\dot{a}) \in h\} \bigcup\\
        & \qquad\qquad\{(c,\dot{b}) | \exists a',\dot{a}.\dot{a} = a', (a',\dot{b}) \in g, (c,\dot{a}) \in h\}\\
        &= h f \djoin h g
      \end{align*}
    \item[\axiom{DJ}{4}] Suppose $\cdperp [f, f', f'']$, $f=\{(a,b)\}, f'=\{(a',b')\},
      f''=\{(a'',b'')\}$. Then the set $\{a\}$ does not intersect either $\{a'\}$ nor $\{a''\}$ and
      similarly for the sets $\{b\}, \{b'\}$ and $\{b''\}'$. Thus we have $f \perp (g\djoin h)$. The
      reverse direction is argued similarly.
  \end{enumerate}
\end{example}
% section disjoint_joins (end)


%%% Local Variables:
%%% mode: latex
%%% TeX-master: "../phd-thesis"
%%% End:
