%!TEX root = /Users/gilesb/UofC/thesis/phd-thesis/phd-thesis.tex
\chapter{\protect{$D[\omega]$} based \protect{$\dagger$}-categories} % (fold)
\label{cha:d_omega_based_dagger_categories}
\section{Toy quantum semantics} % (fold)
\label{sec:toy_quantum_semantics}

% section toy_quantum_semantics (end)
\section{Introduction to synthesis} % (fold)
\label{sec:introduction_to_synthesis}
An important problem in quantum information theory is the decomposition of arbitrary unitary
operators into gates from some fixed universal set {\cite{neilsen2000:QuantumComputationAndInfo}}.
Depending on the operator to be decomposed, this may either be done exactly or to within some given
accuracy $\epsilon$; the former problem is known as {\em exact synthesis} and the latter as {\em
approximate synthesis} {\cite{Kliuchnikov-et-al}}.

% section introduction_to_synthesis (end)
\section{Algebraic background} % (fold)
\label{sec:algebraic_background}

We first introduce some notation and terminology, primarily following {\cite{Kliuchnikov-et-al}}.
Recall that $\N$ is the set of natural numbers including 0, and $\Z$ is the ring of
integers. We write $\Zb=\Z/2\Z$ for the ring of integers modulo 2. Let $\D$ be the ring
of {\em dyadic fractions}, defined as $\D = \Z[\frac12] = \s{\frac{a}{2^n}\mid a\in\Z, n\in\N}$.

Let $\omega = e^{i\pi/4} = (1+i)/\sqrt{2}$. Note that $\omega$ is an 8th root of unity satisfying
$\omega^2=i$ and $\omega^4=-1$. We will consider three different rings related to $\omega$:

\begin{definition}
  Consider the following rings. Note that the first two are subrings of the complex numbers, and
  the third one is not:
  \begin{itemize}
    \item $\Dw = \s{a\omega^3+b\omega^2+c\omega+d \mid a,b,c,d\in\D}$.
    \item $\Zw = \s{a\omega^3+b\omega^2+c\omega+d \mid a,b,c,d\in\Z}$.
    \item $\Zbw = \s{p\omega^3+q\omega^2+r\omega+s \mid p,q,r,s\in\Zb}$.
  \end{itemize}
  Note that the ring $\Zbw$ only has 16 elements. The laws of addition and multiplication are
  uniquely determined by the ring axioms and the property $\omega^4=1\ (\mod 2)$. We call the
  elements of $\Zbw$ {\em residues} (more precisely, residue classes of $\Zw$ modulo 2).
\end{definition}

\begin{remark}
  The ring $\Dw$ is the same as the ring $\Z[\frac1{\sqrt{2}},i]$. However, as already pointed out
  in {\cite{Kliuchnikov-et-al}}, the formulation in terms of $\omega$ is far more convenient
  algebraically.
\end{remark}

\begin{remark}
  The ring $\Zw$ is also called the {\em ring of algebraic integers} of $\Dw$. It has an intrinsic
  definition, i.e., one that is independent of the particular presentation of $\Dw$. Namely, a
  complex number is called an {\em algebraic integer} if it is the root of some polynomial with
  integer coefficients and leading coefficient 1. It follows that $\omega$, $i$, and $\sqrt{2}$ are
  algebraic integers, whereas, for example, $1/\sqrt{2}$ is not. The ring $\Zw$ then consists of
  precisely those elements of $\Dw$ that are algebraic integers.
\end{remark}

\subsection{Conjugate and norm} % (fold)
\label{sub:conjugate_and_norm}
\begin{remark}[Complex conjugate and norm]
  Since $\Dw$ and $\Zw$ are subrings of the complex numbers, they inherit the usual notion of
  complex conjugation. We note that $\omega\da = -\omega^3$. This yields the following formula:
  \begin{equation}\label{eqn-adjoint}
    (a\omega^3+b\omega^2+c\omega+d)\da = -c\omega^3-b\omega^2-a\omega+d.
  \end{equation}
  Similarly, the sets $\Dw$ and $\Zw$ inherit the usual norm from the complex numbers. It is given
  by the following explicit formula, for $t=a\omega^3+b\omega^2+c\omega+d$:
  \begin{equation}\label{eqn-cplx-norm}
    \snorm{t}^2 = t\da t = (a^2+b^2+c^2+d^2) + (cd+bc+ab-da) \sqrt{2}.
  \end{equation}
\end{remark}

\begin{definition}[Weight]
  For $t\in\Dw$ or $t\in\Zw$, the {\em weight} of $t$ is denoted $\sweight{t}$, and is given by:
  \begin{equation}\label{eqn-weight}
    \sweight{t}^2 = a^2+b^2+c^2+d^2.
  \end{equation}
\end{definition}

Note that the square of the norm is valued in $\D[\sqrt{2}]$, whereas the square of the weight is
valued in $\D$. We also extend the definition of norm and weight to vectors in the obvious way: For
$u = (u_\jay)_\jay$, we define
\[
  \norm{u}^2 = \sum_\jay\norm{u_\jay}^2
  \quad\mbox{and}\quad
  \weight{u}^2 = \sum_\jay\weight{u_\jay}^2.
\]

\begin{lemma}\label{lem-ring-norm}
  Consider a vector $u\in\Dw^n$. If $\norm{u}^2$ is an integer, then $\weight{u}^2=\norm{u}^2$.
\end{lemma}

\begin{proof}
  Any $t\in\D[\sqrt{2}]$ can be uniquely written as $t=a+b\sqrt{2}$, where $a,b\in\D$. We can call
  $a$ the {\em dyadic part} of $t$. Now the claim is obvious, because $\weight{u}^2$ is exactly the
  dyadic part of $\norm{u}^2$.
\end{proof}

% subsection conjugate_and_norm (end)

\subsection{Denominator exponents} % (fold)
\label{sub:denominator_exponents}
\begin{definition}
  Let $t\in\Dw$. A natural number $k\in\N$ is called a {\em denominator exponent} for $t$ if
  $\sqrt{2}^k t \in \Zw$. It is obvious that such $k$ always exists. The least such $k$ is called
  the {\em least denominator exponent} of $t$.

  More generally, we say that $k$ is a denominator exponent for a vector or matrix if it is a
  denominator exponent for all of its entries. The least denominator exponent for a vector or
  matrix is therefore the least $k$ that is a denominator exponent for all of its entries.
\end{definition}

\begin{remark}
  Our notion of least denominator exponent is almost the same as the ``smallest denominator
  exponent'' of {\cite{Kliuchnikov-et-al}}, except that we do not permit $k<0$.
\end{remark}

% subsection denominator_exponents (end)

\subsection{Residues} % (fold)
\label{sub:residues}
\begin{remark}
  The ring $\Zbw$ is not a subring of the complex numbers; rather, it is a quotient of the ring
  $\Zw$. Indeed, consider the {\em parity function} $\parity{()}:\Z\to\Zb$, which is the unique
  ring homomorphism. It satisfies $\parity{a}=0$ if $a$ is even and $\parity{a}=1$ if $a$ is odd.
  The parity map induces a surjective ring homomorphism $\rho:\Zw\to\Zbw$, defined by
  \[
    \rho(a\omega^3+b\omega^2+c\omega+d) =
    \parity{a}\omega^3+\parity{b}\omega^2+\parity{c}\omega+\parity{d}.
  \]
  We call $\rho$ the {\em residue map}, and we call $\rho(t)$ the {\em residue} of $t$.
\end{remark}

\begin{convention}
  Since residues will be important for the constructions of this thesis, we introduce a shortcut
  notation, writing each residue $p\omega^3+q\omega^2+r\omega+s$ as a string of binary digits
  $pqrs$.
\end{convention}

What makes residues useful for our purposes is that many important operations on $\Zw$ are
well-defined on residues. Here, we say that an operation $f:\Zw\to\Zw$ is {\em well-defined on
residues} if for all $t,s$, $\rho(t)=\rho(s)$ implies $\rho(f(t))=\rho(f(s))$.

For example, two operations that are obviously well-defined on residues are complex conjugation,
which takes the form $(pqrs)\da = rqps$ by (\vref{eqn-adjoint}), and multiplication by $\omega$,
which is just a cyclic shift $\omega(pqrs)=qrsp$. Table~\vref{tab-residue} shows two other important
operations on residues, namely multiplication by $\sqrt{2}$ and the squared norm.

%TODO - add column of a+a^t / \sqrt(2) or just make a remark that generators follow this
\begin{table}
  \[ \begin{array}{c|c|c} \rho(t) & \rho(\sqrt{2}\,t) & \rho(t\da t)\\\hline
    0000 & 0000 & 0000 \\
    0001 & 1010 & 0001 \\
    0010 & 0101 & 0001 \\
    0011 & 1111 & 1010 \\

    0100 & 1010 & 0001 \\
    0101 & 0000 & 0000 \\
    0110 & 1111 & 1010 \\
    0111 & 0101 & 0001 \\
  \end{array}\qquad
  \begin{array}{c|c|c} \rho(t) & \rho(\sqrt{2}\,t) & \rho(t\da t)\\\hline
    1000 & 0101 & 0001 \\
    1001 & 1111 & 1010 \\
    1010 & 0000 & 0000 \\
    1011 & 1010 & 0001 \\

    1100 & 1111 & 1010 \\
    1101 & 0101 & 0001 \\
    1110 & 1010 & 0001 \\
    1111 & 0000 & 0000 \\
  \end{array}
  \]
  \caption{Some operations on residues}\label{tab-residue}
\end{table}

\begin{definition}[$k$-Residue]
  Let $t\in\Dw$ and let $k$ be a (not necessarily least) denominator exponent for $t$. The {\em
  $k$-residue of $t$}, in symbols $\rho_k(t)$, is defined to be
  \[
    \rho_k(t) = \rho(\sqrt{2}^k t).
  \]
\end{definition}

\begin{definition}[Reducibility]
  We say that a residue $x\in\Zbw$ is {\em reducible} if it is of the form $\sqrt{2}\,y$, for some
  $y\in\Zbw$. Moreover, we say that $x\in\Zbw$ is {\em twice reducible} if it is of the form $2y$,
  for some $y\in\Zbw$.
\end{definition}

\begin{lemma}\label{lem-reducible}
  For a residue $x$, the following are equivalent:
  \begin{enumerate}\alphalabels
    \item $x$ is reducible;
    \item $x\in\s{0000,0101,1010,1111}$;
    \item $\sqrt{2}\,x = 0000$;
    \item $x\da x=0000$.
  \end{enumerate}
  Moreover, $x$ is twice reducible iff $x=0000$.
\end{lemma}

\begin{proof}
  By inspection of Table~\vref{tab-residue}.
\end{proof}

\begin{lemma}\label{lem-reducible2}
  Let $t\in\Zw$. Then $t/2\in\Zw$ if and only if $\rho(t)$ is twice reducible, and
  $t/\sqrt{2}\in\Zw$ if and only if $\rho(t)$ is reducible.
\end{lemma}

\begin{proof}
  The first claim is trivial, as $\rho(t)=0000$ if and only if all components of $t$ are even. For
  the second claim, the left-to-right implication is also trivial: assume $t'=t/\sqrt{2}\in\Zw$.
  Then $\rho(t) = \rho(\sqrt{2}\,t')$, which is reducible by definition. Conversely, let $t\in\Zw$
  and assume that $\rho(t)$ is reducible. Then $\rho(t)\in\s{0000, 0101, 1010, 1111}$, and it can
  be seen from Table~\vref{tab-residue} that $\rho(\sqrt{2}\,t)=0000$. Therefore, $\sqrt{2}\,t$ is
  twice reducible by the first claim; hence $t$ is reducible.
\end{proof}

\begin{corollary}\label{cor-reducible3}
  Let $t\in\Dw$ and let $k>0$ be a denominator exponent for $t$. Then $k$ is the least denominator
  exponent for $t$ if and only if $\rho_k(t)$ is irreducible.
\end{corollary}

\begin{proof}
  Since $k$ is a denominator exponent for $t$, we have $\sqrt{2}^k t \in \Zw$. Moreover, $k$ is
  least if and only if $\sqrt{2}^{k-1} t\not\in\Zw$. By Lemma~\vref{lem-reducible2}, this is the
  case if and only if $\rho(\sqrt{2}^k t)=\rho_k(t)$ is irreducible.
\end{proof}

\begin{lemma}\label{lem-divisiblebyroot2}
  For all $a$ in $\Zw$, $a+a^t$ is divisible by $\sqrt{2}$ in $\Zw$.
\end{lemma}
\begin{proof}
  It is sufficient to show this on the generators $\{1,\omega,\omega^2,\omega^3\}$.
  \begin{enumerate}
    \item $1+1 = 2$ and $\frac{2}{\sqrt{2}} = \sqrt{2} = \omega-\omega^3$.
    \item $\omega+\omega^t = \omega-\omega^3 = \sqrt{2}$.
    \item $\omega^2+(\omega^2)^t = \omega^2 - \omega^2 = 0$.
    \item $\omega^3+(\omega^3)^t = \omega^3 - \omega = -\sqrt{2}$.
  \end{enumerate}
\end{proof}

\begin{definition}
  The notions of residue, $k$-residue, reducibility, and twice-reducibility all extend in an
  obvious componentwise way to vectors and matrices. Thus, the residue $\rho(u)$ of a vector or
  matrix $u$ is obtained by taking the residue of each of its entries, and similar for
  $k$-residues. Also, we say that a vector or matrix is reducible if each of its entries is
  reducible, and similarly for twice-reducibility.
\end{definition}


\begin{example}\label{exa-k-residue}
  Consider the matrix
  \[ \small U \,{=}\, \frac{\small 1}{\small \sqrt{2}^3}\!\footnotesize\zmatrix{cccc}{
    -\omega^3+\omega -1
			& \omega^2+\omega+1
				& \omega^2
					& -\omega\\
    \omega^2+\omega
			& -\omega^3+\omega^2
				& -\omega^2-1
					& \omega^3+\omega \\
    \omega^3+\omega^2
			& -\omega^3-1
				& 2\omega^2
					&0 \\
    -1
			& \omega
				& 1
					& -\omega^3+2\omega
    }.
  \]
  It has least denominator exponent $3$. Its $3$-, $4$-, and $5$-residues are:
  \[
    \begin{split}
    &
    \rho_3(U) = \zmatrix{cccc}{
      1011 & 0111 & 0100 & 0010\\
      0110 & 1100 & 0101 & 1010\\
      1100 & 1001 & 0000 & 0000\\
      0001 & 0010 & 0001 & 1000
    },
    \\&
    \rho_4(U) = \zmatrix{cccc}{
      1010 & 0101 & 1010 & 0101\\
      1111 & 1111 & 0000 & 0000\\
      1111 & 1111 & 0000 & 0000\\
      1010 & 0101 & 1010 & 0101
    },\quad
    \rho_5(U) = 0.\end{split}
  \]
\end{example}

% subsection residues (end)


% section algebraic_background (end)
\section{Exact synthesis of single qubit operators} % (fold)
\label{sec:exact_synthesis_of_single_qubit_operators}


Matsumoto and Amano {\cite{MA08}} showed that every single-qubit Clifford+$T$ operator can be
uniquely written in the following form, which we call the {\em Matsumoto-Amano normal form:}
\begin{equation}\label{eqn-ma}
 (T\mid\emptyseq)\,(HT\mid SHT)^*\,{\cC}.
\end{equation}
Here, we have used the syntax of regular expressions {\cite{regexp}} to denote a set of sequences
of operators. The symbol $\varepsilon$ denotes the empty sequence (more precisely, the singleton
set containing just the empty sequence); if $\cL$ and $\cK$ are two sets of sequences, then
$\cL\mid \cK$ denotes their union; $\cL \cK$ denotes the set $\s{st\mid s\in\cL, t\in\cK}$; $\cL^*$
denotes the set $\s{s_1\ldots s_n \mid n\geq 0; s_1,\ldots,s_n\in\cL}$; and $\cC$ denotes any
Clifford operator. In words, the Matsumoto-Amano representation of an operator consists of a
Clifford operator, followed by any number of {\em syllables} of the form $HT$ or $SHT$, followed by
an optional syllable $T$. (We follow the usual convention of multiplying operators right-to-left,
so when we say one operator ``follows'' another, we mean that it appears to its left).

The most important properties of the Matsumoto-Amano decomposition are:
\begin{itemize}
  \item Existence: all single-qubit Clifford+$T$ operators can be written in Matsumoto-Amano
    normal form (moreover, there is an efficient algorithm for converting the operator to normal
    form);

  \item Uniqueness: no operator can be written in Matsumoto-Amano normal form in more than one way;

  \item $T$-optimality: of all the possible exact decompositions of a given operator into the
    Clifford+$T$ set of gates, the Matsumoto-Amano normal form contains the smallest possible number
    of $T$-gates.
\end{itemize}
It is perhaps less well-known that the uniqueness proof given by Matsumoto and Amano yields an
efficient {\em algorithm} for $T$-optimal exact single-qubit synthesis. One may contrast this, for
example, with the recent algorithm by Kliuchnikov et al.~{\cite{Kliuchnikov-et-al}}, which is
efficient, but only asymptotically $T$-optimal. The purpose of this note is to give a detailed
presentation of the algorithmic content of Matsumoto and Amano's result. Along the way, we also
simplify Matsumoto and Amano's proofs, and we give an intrinsic characterization of the
Clifford+$T$ subgroup of $SO(3)$.

\subsection{Existence} % (fold)
\label{sub:existence}
In the following, we will often speak of sequences of operators. For our purposes, a sequence is
just an $n$-tuple. We write $st$ for the operation of concatenating two sequences, and we write
$\emptyseq$ for the empty sequence. We identify a 1-tuple $(A)$ with the operator $A$ itself. If
$s=\tuple{A_1,\ldots,A_n}$ is a sequence of operators, we write $\sem{s}=A_1\cdots A_n$ for the
product of the operators in the sequence; naturally, $\sem{\emptyseq}=I$. Note that the notation is
ambiguous; for example, depending on the context, $SHT$ may denote either the sequence
$\tuple{S,H,T}$ of 3 operators, or their product, which is a single operator. To alleviate the
ambiguity, we assume that everything is a sequence by default, and we write $s\seqeq t$ if two
sequences are equal as tuples, and $s=t$ if they are equal as operators, i.e., if $\sem{s}=\sem{t}$.

\begin{remark}
  Consider any operator $A$ in Matsumoto-Amano normal form. If $\lambda$ is any unit scalar, then
  $\lambda A$ can clearly also be written in Matsumoto-Amano normal form with the same $T$-count,
  namely by multiplying $\lambda$ into the rightmost Clifford operator. Moreover, if $A$ can be
  {\em uniquely} written in Matsumoto-Amano normal form, then the same is true for $\lambda A$.
  Therefore, nothing is added or lost to the Matsumoto-Amano normal form whether one allows
  arbitrary global phases, a suitable discrete set of global phases (for example, powers of
  $e^{i\pi/4}$), or whether one works modulo global phase. Since it is convenient to work modulo
  global phase, we do so in the remainder; however, this does not restrict the generality of the
  results.
\end{remark}

\begin{definition}
  Let $\sC$ denote the Clifford group on one qubit, modulo global phases. This group has 24
  elements. Let $\sS$ be the 8-element subgroup spanned by $S$ and $X$. Let $\sCp = \sC\sm\sS$. Let
  $\sH = \s{I,H,SH}$ and $\sHp = \s{H,SH}$.
\end{definition}

\begin{lemma}\label{lem-SH}
  The following hold:
  \begin{align}
    \sC ~&=~ \sH\sS,\label{eqn-CHS}\\
    \sCp ~&=~ \sHp\sS,\label{eqn-CSHS}\\
    \sS\sHp ~&\seq~ \sHp\sS,\label{eqn-SHHS}\\
    \sS T ~&=~ T \sS,\label{eqn-STTS}\\
    T \sS T ~&=~ \sS.\label{eqn-TSTS}
  \end{align}
\end{lemma}

\begin{proof}
  Since $\sS$ is an 8-element subgroup of $\sC$, it has three left cosets. They are $\sS$, $H\sS$,
  and $SH\sS$. Since $\sC$ is the disjoint union of these cosets, (\vref{eqn-CHS}) and
  (\vref{eqn-CSHS}) immediately follow. For (\vref{eqn-SHHS}), first notice that $\sS S=\sS$, and
  therefore $\sS\sHp = \sS H\cup \sS SH = \sS H$. Since $\sS H$ is a non-trivial right coset of
  $\sS$, it follows that $\sS H \seq \sC\sm\sS = \sCp$. Combining these facts with
  (\vref{eqn-CSHS}), we have (\vref{eqn-SHHS}). Finally, the equations (\vref{eqn-STTS}) and
  (\vref{eqn-TSTS}) are trivial consequences of the equations $ST=TS$, $XT=TXS$, and $TT=S$.
\end{proof}

\begin{proposition}[Matsumoto and Amano {\cite{MA08}}]\label{prop-ma}
  Every single-qubit Clifford+$T$ operator can be written in Matsumoto-Amano normal form.
\end{proposition}

\begin{proof}
  Let $M$ be a single-qubit Clifford+$T$ operator. Clearly, $M$ can be written as
  \begin{equation}\label{eqn-M}
    M = C_n\;T\;C_{n-1}\;\cdots\;C_1\;T\;C_0,
  \end{equation}
  for some $n\geq 0$, where $C_0,\ldots,C_n\in\sC$. First note that if $C_i\in\sS$ for any
  $i\in\s{1,\ldots,n-1}$, then we can immediately use (\vref{eqn-TSTS}) to replace $TC_iT$ by a
  single Clifford operator. This yields a shorter expression of the form (\vref{eqn-M}) for $M$. We
  may therefore assume without loss of generality that $C_i\not\in\sS$ for $i=1,\ldots,n-1$. If
  $n=0$, then $M$ is a Clifford operator, and there is nothing to show. Otherwise, we have
  \begin{align}
    M ~&\in~ \sC\;T\;\sCp\;\cdots\;\sCp\;T\;\sC &&\mbox{ by (\ref{eqn-M})}\\
    &=~ \sH\sS\;T\;\sHp\sS\;\cdots\;\sHp\sS\;T\;\sC &&\mbox{ by
      (\ref{eqn-CHS}) and (\ref{eqn-CSHS})}\\
    &\seq~ \sH\;T\;\sHp\;\cdots\;\sHp\;T\;\sC &&\mbox{ by
      (\ref{eqn-SHHS}) and (\ref{eqn-STTS}).}\label{eqn-HTHpTC}
  \end{align}
  Note how, in the last step, the relations (\vref{eqn-SHHS}) and (\vref{eqn-STTS}) were used to move
  all occurrences of $\sS$ to the right, where they were absorbed into the final $\sC$. It is now
  trivial to see that every element of (\vref{eqn-HTHpTC}) can be written in Matsumoto-Amano normal
  form, finishing the proof.
\end{proof}

\begin{corollary}\label{cor-efficient}
  There exists a linear-time algorithm for symbolically reducing any sequence of Clifford+$T$
  operators to Matsumoto-Amano normal form. More precisely, this algorithm runs in time at most
  $O(n)$, where $n$ is the length of the input sequence.
\end{corollary}

\begin{proof}
  The proof of Proposition~\vref{prop-ma} already contains an algorithm for reducing any sequence of
  Clifford+$T$ operators to Matsumoto-Amano normal form. However, in the stated form, it is perhaps
  not obvious that the algorithm runs in linear time. Indeed, a naive implementation of the first
  step would require up to $n$ searches of the entire sequence for a term of the form $T\sS T$,
  which can take time $O(n^2)$.

  One obtains a linear time algorithm from the following observation: if $M$ is already in
  Matsumoto-Amano normal form, and $A$ is either a Clifford operator or $T$, then $MA$ can be
  reduced to Matsumoto-Amano normal form in constant time. This is trivial when $A$ is a Clifford
  operator, because it will simply be absorbed into the rightmost Clifford operator of $M$. In the
  case where $A=T$, a simple case distinction shows that at most the rightmost 5 elements of $MA$
  need to be updated. The normal form of a sequence of operators $A_1A_2\ldots A_n$ can now be
  computed by starting with $M=I$ and repeatedly right-multiplying by $A_1,\ldots,A_n$, reducing to
  normal form after each step.
\end{proof}

% subsection existence (end)

\subsection{$T$-Optimality} % (fold)
\label{sub:_t_optimality}
\begin{corollary}
  Let $M$ be an operator in the Clifford+$T$ group, and assume that $M$ can be written with
  $T$-count $n$. Then there exists a Matsumoto-Amano normal form for $M$ with $T$-count at most $n$.
\end{corollary}

\begin{proof}
  This is an immediately consequence of the proof of Proposition~\ref{prop-ma}, because the
  reduction from Equation~(\ref{eqn-M}) to Equation~(\ref{eqn-HTHpTC}) does not increase the
  $T$-count.
\end{proof}

% subsection _t_optimality (end)

\subsection{Uniqueness} % (fold)
\label{sub:uniqueness}
\begin{theorem}[Matsumoto and Amano {\cite{MA08}}]\label{thm-ma}
  If $M$ and $N$ are two different Matsumoto-Amano normal forms, then they describe different
  operators.
\end{theorem}

Recall that each single-qubit unitary operator (modulo global phase) can be uniquely represented as
a rotation on the Bloch sphere, or equivalently, as an element of $SO(3)$, the real orthogonal
$3\times 3$ matrices with determinant 1. The relationship between an operator $U\in U(2)$ and its
Bloch sphere representation $\hat U\in SO(3)$ is given by
\begin{equation}\label{eqn-bloch-conversion}
  \hat U\zmatrix{c}{x\\y\\z} ~=~ \zmatrix{c}{x'\\y'\\z'}
  \quad\Longleftrightarrow\quad
  U\,(xX+yY+zZ)\;U\da ~=~ x'X + y'Y + z'Z,
\end{equation}
where $X$, $Y$, and $Z$ are the Pauli operators. The Bloch sphere representations of the operators
$H$, $S$, and $T$ are:
\begin{equation}\label{bloch-generators}
  \hat H = \zmatrix{ccc}{0&0&1\\0&-1&0\\1&0&0}, \quad
  \hat S = \zmatrix{ccc}{0&-1&0\\1&0&0\\0&0&1}, \quad
  \hat T = \frac{1}{\sqrt{2}}\zmatrix{ccc}{1&-1&0\\1&1&0\\0&0&\sqrt{2}}.
\end{equation}
\begin{remark}\label{rem-bloch}
  The Bloch sphere representation of any scalar is the identity matrix. The Bloch sphere
  representations of the 24 Clifford operators (modulo phase) are precisely those elements of
  $SO(3)$ that can be written with matrix entries in $\s{-1,0,1}$; these are exactly the 24
  symmetries of the cube $\s{(x,y,z)\mid -1\leq x,y,z\leq 1}$.
\end{remark}

\begin{definition}
  Recall that $\N$ denotes the natural numbers including 0; $\Z$ denotes the integers; and $\Z_2$
  denotes the integers modulo 2. We define three subrings of the real numbers:
  \begin{itemize}
    \item $\D = \Z[\frac12] = \s{\frac{a}{2^n}\mid a\in\Z,
        n\in\N}$. This is the ring of {\em dyadic fractions}.
    \item $\Zs = \s{a+b\sqrt 2 \mid a,b\in\Z}$. This is the ring of {\em
        quadratic integers} with radicand 2.
    \item $\Ds = \Z[\frac1{\sqrt2}] = \s{r+s\sqrt 2 \mid r,s\in\D}$.
  \end{itemize}
  We will also need a fourth ring, which is not a subring of the real numbers.
  \begin{itemize}
    \item $\Zss = \s{a+b\sqrt 2 \mid a,b\in\Z_2}$.
  \end{itemize}
  Note that the ring $\Zss$ has only 4 elements; they are residue classes modulo 2 of the ring
  $\Z[\sqrt 2]$. For brevity, we refer to the elements of $\Zss$ as {\em residues}.
\end{definition}


\begin{definition}[Residue\footnote{I don't think we need residue here,
  which is good as it will then be used only in section 7, the $U(2)$ case}
  and parity]
  Consider the unique ring homomorphism $\Z\to\Z_2$, mapping $a\in\Z$ to $\bar a\in\Z_2$, where
  $\bar a=0$ if $a$ is even and $\bar a=1$ if $a$ is odd. This induces a surjective ring
  homomorphism $\rho:\Zs\to\Zss$, defined by $\rho(a+b\sqrt{2}) = \bar a + \bar b\sqrt{2}$. For any
  given $x\in\Zs$, we refer to $\rho(x)$ as the {\em residue of $x$}.

  Moreover, consider the ring homomorphism $p:\Zs\to\Z_2$ given by $p(a+b\sqrt{2}) = \bar a$. We
  refer to $p(x)$ as the {\em parity} of $x$.
\end{definition}

\begin{definition}[Denominator exponent]
  For every element $q\in\Ds$, there exists some natural number $k\geq 0$ such that
  $\sqrt{2}^kq\in\Zs$, or equivalently, such that $q$ can be written as $\frac{x}{\sqrt{2}^k}$, for
  some quadratic integer $x$. Such $k$ is called a {\em denominator exponent} for $q$. The least
  such $k$ is called the {\em least denominator exponent} of $q$.

  More generally, we say that $k$ is a denominator exponent for a vector or matrix if it is a
  denominator exponent for all of its entries. The least denominator exponent for a vector or
  matrix is therefore the least $k$ that is a denominator exponent for all of its entries.
\end{definition}

\begin{definition}[$k$-parity]
  Let $k$ be a denominator exponent for $q\in\Ds$. We define the {\em $k$-residue} of $q$, in
  symbols $\rho_k(q)\in \Zss$, and the {\em $k$-parity} of $q$, in symbols $p_k(q)\in\Z_2$, by
  \[
    \rho_k(q) = \rho(\sqrt{2}^k q),
    \hspace{0.5in}
    p_k(q) = p(\sqrt{2}^k q).
  \]
  The $k$-residue and $k$-parity of a vector or matrix are defined componentwise.
\end{definition}

% ......................................................................
\begin{figure}
  \[
  \xymatrix{
    *+{\makebox[0in][r]{Start:~}\zmatrix{ccc}{1&0&0 \\ 0&1&0 \\ 0&0&1}}
    \ar[d]^<>(.5){\displaystyle T}_<>(.5){k\pp}
    \ar@<1.5ex>@(dr,r)[]+R_<>(.53){\displaystyle \sC}
    \\
    *+{\zmatrix{ccc}{1&1&0 \\ 1&1&0 \\ 0&0&0}}
    \ar@<-0.5ex>@/_/[d]_<>(.5){\displaystyle H}
    \\
    *+{\zmatrix{ccc}{0&0&0 \\ 1&1&0 \\ 1&1&0}}
    \ar[r]_<>(.5){\displaystyle S}
    \ar@<-0.5ex>@/_/[u]_<>(.6){\displaystyle T}_<>(.25){k\pp}
    &
    *+{\zmatrix{ccc}{1&1&0 \\ 0&0&0 \\ 1&1&0}}
    \ar@<-2ex>@/_4ex/[ul]_<>(.5){\displaystyle T}^<>(.5){k\pp}
  }
  \]
  \caption{The action of Matsumoto-Amano normal forms on
    $k$-parities. All matrices are written modulo the right action
    of the Clifford group, i.e., modulo a permutation of the
    columns.}\label{fig-so3-action}
\end{figure}
% ......................................................................

\begin{remark}
  Let $C$ be any Clifford operator, and $\hat C$ its Bloch sphere representation. As noted above,
  the matrix entries of $\hat C$ are in $\s{-1,0,1}$; it follows that $\hat C$ has denominator
  exponent 0. In particular, it follows that multiplication by $\hat C$ is a well-defined operation
  on parity matrices: for any $3\times 3$-matrix $U$ with entries in $\Zss$, we define $U\bullet C
  := U\cdot p(\hat C)$. This defines a right action of the Clifford group on the set of parity
  matrices.
\end{remark}

\begin{definition}\label{def-simg}
  If $G$ is any subgroup of the Clifford group, we define $\sim_G$ the be the equivalence relation
  induced by this right action, i.e., for parity matrices $U,V$, we write $U\sim_G V$ if there
  exists some $C\in G$ such that $V=U\bullet G$. In case $G=\sC$ is the entire Clifford group,
  $U\sim_{\sC} V$ holds if and only if $U$ and $V$ differ by a permutation of columns.
\end{definition}

\begin{lemma}\label{lem-ma}
  Let $M$ be a Matsumoto-Amano normal form, and $\hat M\in SO(3)$ the Bloch sphere operator of $M$.
  Let $k$ be the least denominator exponent of $\hat M$. Then exactly one of the following holds:\rm
  \begin{itemize}
    \item $k=0$, and $M$ is a Clifford operator.
    \item $k>0$, $p_k(\hat M)\sim_{\sC} \zmatrix{ccc}{1&1&0 \\ 1&1&0 \\
        0&0&0}$, and the leftmost syllable in $M$ is $T$.
    \item $k>0$, $p_k(\hat M)\sim_{\sC} \zmatrix{ccc}{0&0&0 \\ 1&1&0 \\ 1&1&0}$, and the
      leftmost syllable in $M$ is $HT$.
    \item $k>0$, $p_k(\hat M)\sim_{\sC} \zmatrix{ccc}{1&1&0 \\ 0&0&0 \\
        1&1&0}$, and the leftmost syllable in $M$ is $SHT$.
  \end{itemize}
  Moreover, the $T$-count of $M$ is equal to $k$.
\end{lemma}

\begin{proof}
  By induction on the length of the Matsumoto-Amano normal form $M$. Figure~\vref{fig-so3-action}
  shows the action of Matsumoto-Amano operators on parity matrices. Each vertex represents a
  $\sim_{\sC}$-equivalence class of $k$-parities. The vertex labelled ``Start'' represents the
  empty Matsumoto-Amano normal form, i.e., the identity operator. Each arrow represents left
  multiplication by the relevant operator, i.e., a Clifford operator, $T$, $H$, or $S$. Thus, each
  Matsumoto-Amano normal form, read from right to left, gives rise to a unique path in the graph of
  Figure~\vref{fig-so3-action}. The label $k\pp$ on an arrow indicates that the least denominator
  exponent increases by $1$. The claims of the lemma then immediately follow from
  Figure~\vref{fig-so3-action}.
\end{proof}

\begin{proof}[Proof of Theorem~\vref{thm-ma}]
  This is an immediate consequence of Lemma~\ref{lem-ma}. Indeed, suppose that $M$ and $N$ are two
  Matsumoto-Amano normal forms describing the same Bloch sphere operator $U$. We show that $M=N$ by
  induction on the length of $M$. Let $k$ be the least denominator exponent of $U$. If $k=0$, then
  by Lemma~\ref{lem-ma}, both $M$ and $N$ are Clifford operators; since they describe the same
  Bloch sphere operator, they differ only by a phase.\footnote{Todo: fix the treatment of phases}
  If $k>0$, then by Lemma~\ref{lem-ma}, the Matsumoto-Amano normal forms $M$ and $N$ have the same
  leftmost operator (either $T$, $H$, or $S$), and the claim follows by induction hypothesis.
\end{proof}

% subsection uniqueness (end)

\subsection{The Matsumoto-Amano decomposition algorithm} % (fold)
\label{sub:the_matsumoto_amano_decomposition_algorithm}
As an immediate consequence of Lemma~\vref{lem-ma}, we can an efficient algorithm for calculating
the Matsumoto-Amano normal form of any Clifford+$T$ operator, given as a matrix.

\begin{theorem}\label{thm:ma-decomposition}
  Let $U\in SO(3)$ be the Bloch sphere representation of some Clifford+$T$ operator. Let $k$ be the
  least denominator exponent of $U$. Then the Matsumoto-Amano normal form $M$ of $U$ can be
  efficiently computed with $O(k)$ arithmetic operations.
\end{theorem}

\begin{proof}
  By assumption, $U$ is the Bloch sphere representation of some Clifford+$T$ operator. Let $M$ be
  the unique Matsumoto-Amano normal form of this operator\footnote{Todo: treat phase correctly}.
  Note that, by Lemma~\vref{lem-ma}, the $T$-count of $M$ is $k$. We compute $M$ recursively. If
  $k=0$, then by Lemma~\vref{lem-ma}, $M$ is a Clifford operator; it can be determined from the
  matrix $U$ in constant time. If $k>0$, we compute $p_k(U)$, which must be one of the three cases
  listed in Lemma~\vref{lem-ma}. This determines whether the leftmost syllable of $M$ is $T$, $HT$,
  or $SHT$. Let $N$ be this syllable, so that $M=NM'$, for some Matsumoto-Amano normal form $M'$.
  Then $M'$ can be recursively computed as the Matsumoto-Amano normal form of $U'=\hat N\inverse U$;
  moreover, since $M'$ has $T$-count $k-1$, the recursion terminates after $k$ steps. Since each
  induction step only requires a constant number of arithmetic operations, the total number of
  operations is $O(k)$.
\end{proof}

% subsection the_matsumoto_amano_decomposition_algorithm (end)

\subsection{A characterization of Clifford+$T$ on the Bloch sphere} % (fold)
\label{sub:a_characterization_of_clifford_t_on_the_bloch_sphere}
\begin{lemma}\label{lem:dot_product_is_simpler_in_ds}
  Let $U\in SO(3)$ be an orthogonal matrix with entries in $\Ds$. Let $k$ be a denominator exponent
  of $U$, and let $v_1,v_2,v_3$ be the columns of $U$, with
  \[v_\jay = \frac{1}{\sqrt{2}^k}\left(\begin{matrix}
      a_\jay+b_\jay\sqrt{2} \\
      c_\jay+d_\jay\sqrt{2} \\
      e_\jay+f_\jay\sqrt{2}
    \end{matrix}\right),\]
  for $a_\jay,\ldots,f_\jay\in\Z$.
  Then for all $\jay,\ell\in\s{1,2,3}$,
  \begin{equation}\label{eqn-abba}
    a_\jay b_\ell+b_\jay a_\ell+
            c_\jay d_\ell+d_\jay c_\ell+e_\jay f_\ell+f_\jay e_\ell = 0
  \end{equation}
  and
  \begin{equation}\label{eqn-aacc}
    a_\jay a_\ell+c_\jay c_\ell+e_\jay e_\ell+
    2(b_\jay b_\ell+d_\jay d_\ell+f_\jay f_\ell)
    = 2^k\<v_\jay,v_\ell\>.
  \end{equation}
  In particular, we have, for all $\jay\in\s{1,2,3}$,
  \begin{equation}\label{eqn-abcd}
    a_\jay b_\jay+
    c_\jay d_\jay+e_\jay f_\jay = 0
  \end{equation}
  and
  \begin{equation}\label{eqn-a2c2}
    a_\jay^2 + c_\jay^2 + e_\jay^2 + 2(b_\jay^2 + d_\jay^2 + f_\jay^2) = 2^k.
  \end{equation}
\end{lemma}

\begin{proof}
  Computing the inner product, we have
  \begin{multline}  \label{eqn-dotprod_of_vl_and_vj}
    \<v_\jay,v_\ell\> =\\
    \frac{1}{2^k}
    \left(
      a_\jay a_\ell+c_\jay c_\ell+e_\jay e_\ell+
      2(b_\jay b_\ell+d_\jay d_\ell+f_\jay f_\ell) \right. \\
      \left.
      +\sqrt{2}(a_\jay b_\ell+b_\jay a_\ell+
      c_\jay d_\ell+d_\jay c_\ell+e_\jay f_\ell+f_\jay e_\ell)
    \right).
  \end{multline}
  Since $U\da U=I$, we have $\<v_\jay,v_\jay\>=1$, and $\<v_\jay,v_\ell\> = 0$ when $\ell \ne
  \jay$. Therefore, the coefficient of $\sqrt{2}$ in equation \vref{eqn-dotprod_of_vl_and_vj} must
  be zero, proving \vref{eqn-abba} and \vref{eqn-aacc}. Equations {\vref{eqn-abcd}} and
  {\vref{eqn-a2c2}} immediately follow by letting $\jay=\ell$.
\end{proof}

\begin{remark}
  In Lemma~\vref{lem:dot_product_is_simpler_in_ds}, we have worked with columns $v_\jay$ of the
  matrix $U$. But since $U$ is orthogonal, the analogous properties also hold for the rows of $U$.
\end{remark}

\begin{lemma}\label{lem:k0-clifford}
  Let $U\in SO(3)$ be an orthogonal matrix with entries in $\Ds$, and with least denominator
  exponent $k=0$. Then $U$ the Bloch sphere representation of some Clifford operator.
\end{lemma}

\begin{proof}
  Let $v_j$ be any column of $U$, with the notation of
  Lemma~\vref{lem:dot_product_is_simpler_in_ds}. By {\vref{eqn-a2c2}}, we have
  $a_\jay^2+c_\jay^2+e_\jay^2+2(b_\jay^2+d_\jay^2+f_\jay^2)=1$. Since each summand is a positive
  integer, we must have $b_\jay,d_\jay,f_\jay = 0$, and exactly one of $a_\jay$, $c_\jay$ or
  $e_\jay=\pm1$, for each $\jay=1,2,3$. Therefore, all the matrix entries are in $\s{-1,0,1}$, and
  the claim follows by Remark~\vref{rem-bloch}.
\end{proof}

\begin{lemma}\label{lem:parity_u_gives_a_row_in_the_algorithm}
  Let $U\in SO(3)$ be an orthogonal matrix with entries in $\Ds$, and let $k$ be the least
  denominator exponent of $U$. If $k=0$, then $p_k(U)\sim_{\sC} M_1$. If $k>0$, then
  $p_k(U)\sim_{\sC} M$ for some $M\in\s{M_T,M_H,M_S}$, where
  \[
     M_1 = \zmatrix{ccc}{1&0&0 \\ 0&1&0 \\ 0&0&1},\quad
     M_T = \zmatrix{ccc}{1&1&0 \\ 1&1&0 \\ 0&0&0},\quad
     M_H = \zmatrix{ccc}{0&0&0 \\ 1&1&0 \\ 1&1&0},\quad
     M_S = \zmatrix{ccc}{1&1&0 \\ 0&0&0 \\ 1&1&0}.
  \]
\end{lemma}
\begin{proof}
  First consider the case $k=0$. Let $v_\jay$ be any column of $U$, with the notation of
  Lemma~\vref{lem:dot_product_is_simpler_in_ds}. By {\vref{eqn-a2c2}}, we have
  $a_\jay^2+c_\jay^2+e_\jay^2+2(b_\jay^2+d_\jay^2+f_\jay^2)=1$. Since each summand is a positive
  integer, we must have $b_\jay,d_\jay,f_\jay = 0$, and exactly one of $a_\jay$, $c_\jay$ or
  $e_\jay=\pm1$, for each $\jay=1,2,3$. Noting that the columns of $U$ are orthogonal, we see that
  $p_k(U)\sim_{\sC}M_1$.

  Now consider the case $k>0$. Let $v_\jay$ be any row or column of $U$, with the notation of
  Lemma~\vref{lem:dot_product_is_simpler_in_ds}. By {\vref{eqn-a2c2}}, it follows that $a_\jay^2 +
  c_\jay^2 + e_\jay^2$ is even, and therefore an even number of $a_\jay$, $c_\jay$, and $e_\jay$
  have parity $1$. Therefore, each row or column of $p_k(U)$ has an even number of $1$'s. Moreover,
  since $k$ is the least denominator exponent of $U$, $p_k(U)$ has at least one non-zero entry.
  Modulo a permutation of columns, this leaves exactly four possibilities for $p_k(U)$:
  \[
    (a)~ \zmatrix{ccc}{1&1&0 \\ 1&1&0 \\ 0&0&0},\quad
    (b)~ \zmatrix{ccc}{0&0&0 \\ 1&1&0 \\ 1&1&0},\quad
    (c)~ \zmatrix{ccc}{1&1&0 \\ 0&0&0 \\ 1&1&0},\quad
    (d)~ \zmatrix{ccc}{1&1&0 \\ 1&0&1 \\ 0&1&1}.
   \]
   In cases (a)--(c), we are done. Case (d) is impossible because it implies that
   $a_1a_2+c_1c_2+e_1e_2$ is odd, contradicing the fact that it is even by {\vref{eqn-aacc}}.
\end{proof}

\begin{lemma}\label{lem:applying_a_t_symbol_decreases_k}
  Let $U \in SO(3)$ be an orthogonal matrix with entries in $\Ds$, and with least denominator
  exponent $k > 0$. Then there exists $N\in\s{T, HT, SHT}$ such that the least denominator exponent
  of $\hat N\inverse U$ is $k-1$.
\end{lemma}

\begin{proof}
  By Lemma~\vref{lem:parity_u_gives_a_row_in_the_algorithm}, we know that $p_k(U)\sim_{\sC} M$, for
  some $M\in\{M_T, M_H, M_S\}$. We consider each of these cases.
  \begin{enumerate}
    \item \label{lemparitycase:applyt}$p_k(U) \sim_{\sC} M_T$.
      By assumption, $U$ has two columns $v$ with $p_k(v) = (1,1,0)^T$.
      Let
      \[ v =
      \frac{1}{\sqrt{2}^k}\zmatrix{c}{a+b\sqrt{2}\\c+d\sqrt{2}\\e+f\sqrt{2}}
      \]
      be any such column. By {\vref{eqn-abcd}},
      we have $ab+cd+ef = 0$. Since $\bar e = 0$, we have
      $\bar a \bar b+ \bar c \bar d = 0$. Since $\bar a = \bar c = 1$,
      we can conclude $\bar b+\bar d = 0$.
      Applying $\hat{T}^{-1}$ to $v$, we compute:
      \begin{multline*}
        \hat{T}^{-1}v = \frac{1}{\sqrt{2}^{k+1}}
        \begin{bmatrix}
          c+a &+& (d+b)\sqrt{2} \\
          c-a &+& (d-b)\sqrt{2} \\
          e\sqrt{2} &+& 2f
        \end{bmatrix}
        = \frac{1}{\sqrt{2}^{k-1}}
        \begin{bmatrix}
          \frac{c+a}{2} &+& \frac{d+b}{\sqrt{2}} \\
          \frac{c-a}{2} &+& \frac{d-b}{\sqrt{2}} \\
          \frac{e}{\sqrt{2}} &+& f
        \end{bmatrix}\\
        = \frac{1}{\sqrt{2}^{k-1}}
        \begin{bmatrix}
          a' &+& b'{\sqrt{2}} \\
          c' &+& d'{\sqrt{2}} \\
          f &+ & e'{\sqrt{2}}
        \end{bmatrix}
      \end{multline*}
      where $a' = \frac{c+a}{2}, b' = \frac{d+b}{2}, c' =
      \frac{c-a}{2}, d' = \frac{d-b}{2}$ and $e' = \frac{e}{2}$ are
      all integers. Hence, $k-1$ is a denominator exponent of
      $\hat{T}^{-1}v$. Moreover, since $a'+c'=c$ is odd, one of $a'$
      and $c'$ is odd, proving that $k-1$ is the least denominator
      exponent of $\hat{T}^{-1}v$.

      Now consider the third column $w$ of $U$, where $p_k(w) =
      (0,0,0)^T$. Then $k-1$ is a denominator exponent of $w$, so that
      $k$ is a denominator exponent for $\hat{T}^{-1}w$. Let
      \[
        p_k(\hat{T}^{-1}w) = \zmatrix{c}{x\\y\\z}.
      \]
      As the least denominator exponent of the other two column of
      $p_k(\hat{T}^{-1}U)$ is $k-1$, we have
      \[
        p_k(\hat{T}^{-1}U) \sim_{\sC}
        \begin{bmatrix} 0&0&x \\0&0&y \\0&0&z\end{bmatrix}.
      \]
      But $\hat{T}^{-1}U$ is orthogonal, so by {\vref{eqn-a2c2}},
      applied to each row of $\hat{T}^{-1}U$, we conclude that
      $x=y=z=0$. It follows that the least denominator exponent of
      $\hat{T}^{-1}U$ is $k-1$.
    \item \label{lemparitycase:applyh}$p_k(U) \sim_{\sC} M_H$.
      In this case, $p_k(\hat{H}^{-1}U) \sim_{\sC} p(\hat{H}^{-1}M_H) = M_T$.
      We then continue as in case \vref{lemparitycase:applyt}.
    \item \label{lemparitycase:applys}$p_k(U) \sim_{\sC} M_S$.
      In this case, $p_k(\hat{H}\inverse\hat{S}^{-1}U) \sim_{\sC} p(\hat{H}\inverse\hat{S}^{-1}M_S) = M_T$.
      We then continue as in case \vref{lemparitycase:applyt}.\qedhere
  \end{enumerate}
\end{proof}

Combining Lemmas~\vref{lem:k0-clifford} and
{\vref{lem:parity_u_gives_a_row_in_the_algorithm}}, we easily get the
following result:

\begin{theorem}\label{thm:completness_of_algorithm}
  Let $U \in SO(3)$ be an orthogonal matrix. Then $U$ is the Bloch
  sphere representation of some Clifford+$T$ operator $M$ if and only
  if the entries of $U$ are in the ring $\Ds$.
\end{theorem}

\begin{proof}
  The ``only if'' direction is trivial, since all the generators of
  the Clifford+$T$ group have this property (see
  {\ref{bloch-generators}}). To prove the ``if'' direction, let $k$
  be the least denominator exponent of $U$. We proceed by induction on
  $k$. If $k=0$, by Lemma~\ref{lem:k0-clifford}, $U$ is the Bloch
  sphere representation of some Clifford operator, and therefore a
  Clifford+$T$ operator. If $k>0$, then by
  Lemma~\ref{lem:parity_u_gives_a_row_in_the_algorithm}, we can write
  $U=\hat NU'$, where $N\in\s{T,HT,SHT}$ and $U'$ has least
  denominator exponent $k-1$. By induction hypothesis, $U'$ is a
  Clifford+$T$ operator, and therefore so is $U$.
\end{proof}

\begin{remark}
  Combining this result with the algorithm of Theorem~\vref{thm:ma-decomposition}, we have a
  linear-time algorithm for computing the Matsumoto-Amano normal form of any unitary operator $U\in
  SO(3)$ with entries in $\Ds$.
\end{remark}

\begin{corollary}[Kliuchnikov et al. {\cite{Kliuchnikov-et-al}}]\label{cor-u2}
  Let $U\in U(2)$ be a unitary matrix. Then $U$ is a Clifford+$T$ operator if and only if the
  matrix entries of $U$ are in the ring $\D[\sqrt{2},i]$.
\end{corollary}

\begin{proof}
  Again, the ``only if'' direction is trivial, as it is true for the generators. For the ``if''
  direction, it suffices to note that, by {\vref{eqn-bloch-conversion}}, whenever $U$ takes its
  entries in $\D[\sqrt{2},i]$, then $\hat U$ takes its entries in $\Ds$.\footnote{Todo: treat phase
  correctly; also introduce the ring $\D[\sqrt{2},i]$ at some appropriate time}
\end{proof}

Corollary~\vref{cor-u2} was first proved by Kliuchnikov et al.~{\cite{Kliuchnikov-et-al}}, using a direct method
(i.e., not going via the Bloch sphere representation). It is interesting to note that
Theorem~\vref{thm:completness_of_algorithm} is stronger than Corollary~\vref{cor-u2}, in the sense
that the Theorem obviously implies the Corollary, whereas it is not a priori obvious that the
Corollary implies the Theorem.
% subsection a_characterization_of_clifford_t_on_the_bloch_sphere (end)
\subsection{Alternative normal forms} % (fold)
\label{sub:alternative_normal_forms}

With the exception of the left-most and right-most gates, the Matsumoto-Amano normal form uses
syllables of the form $HT$ and $SHT$. It is of course possible to use different sets of syllables
instead.

\subsubsection{$E$-$T$ normal form} % (fold)
\label{ssub:_e_t_normal_form}
Consider the Clifford operator
\[
  E = HS^3\omega^3 = \frac{1}{2}\zmatrix{cc}{-1+i & 1+i \\ -1+i & -1-i}.
\]
It has the following properties:
\[
  E^3 = I,
  \quad EXE\inverse = Y,
  \quad EYE\inverse = Z,
  \quad EZE\inverse = X.
\]
The operator $E$ serves as a convenient operator for switching between the $X$-, $Y$-, and
$Z$-bases. On the Bloch sphere, it represents a rotation by 120 degrees about the axis $(1,1,1)^T$:
\[
  \hat E = \zmatrix{ccc}{0&0&1\\1&0&0\\0&1&0}.
\]

The operators $E$ and $E^2$ have properties analogous to $H$ and $SH$. Specifically, if we let $\sH
= \s{I,E,E^2}$ and $\sHp = \s{E,E^2}$, then the properties of Lemma~\ref{lem-SH} are satisfied. The
proofs of Proposition~\ref{prop-ma} and Corollary~\ref{cor-efficient} only depend on these
properties, and the uniqueness proof (Theorem~\vref{thm-ma}) also goes through without significant
changes. We therefore have:

\begin{proposition}[$E$-$T$ normal form]
  Every single-qubit Clifford+$T$ operator can be uniquely written in the form
  \begin{equation}\label{eqn-et}
    (T\mid\emptyseq)\,(ET\mid E^2T)^*\,{\cC}.
  \end{equation}
  Moreover, this normal form has minimal $T$-count, and there exists a linear-time algorithm for
  symbolically reducing any sequence of Clifford+$T$ operators to this normal form.
\end{proposition}

% subsubsection _e_t_normal_form (end)

\subsubsection{$T_x$-$T_y$-$T_z$ normal form} % (fold)
\label{ssub:_t_x_t_y_t_z_normal_form}
It is plain to see that every syllable of the $E$-$T$ normal form (except perhaps the first or last
one) consists of a 45 degree $z$-rotation, followed by a basis change that rotates either the $x$-
or $y$-axis into the $z$-position. Abstracting away from these basis changes, the entire normal
form can therefore be regarded as a sequence of 45-degree rotations about the $x$-, $y$-, and
$z$-axes. More precisely, let us define variants of the $T$-gate that rotate about the three
different axes:
\[
 \begin{array}{l}
    T_x = ETE^2, \\
    T_y = E^2TE, \\
    T_z = T.
  \end{array}
\]
Using the commutativities $ET_x = T_yE$, $ET_y = T_zE$, and $ET_z = T_xE$, it is then clear that
every expression of the form (\vref{eqn-et}) can be uniquely rewritten as a sequence of $T_x$,
$T_y$, and $T_z$ rotations, with no repeated symbol, followed by a Clifford operator. This can be
easily proved by induction, but is best seen in an example:
\[
  \begin{array}{rcl}
    TETETE^2TEC
    &=& T_zET_zET_zE^2T_zEC \\
    &\rightarrow& T_zT_xE^2T_zE^2T_zEC \\
    &\rightarrow& T_zT_xT_yE^4T_zEC \\
    &\rightarrow& T_zT_xT_yET_zEC \\
    &\rightarrow& T_zT_xT_yT_xE^2C \\
    &\rightarrow& T_zT_xT_yT_xC'. \\
  \end{array}
\]
We have:
\begin{proposition}[$T_x$-$T_y$-$T_z$ normal form]
  Every single-qubit Clifford+$T$ operator can be uniquely written in the form
  \[
    T_{r_1}T_{r_2}\ldots T_{r_n} C,
  \]
  where $n\geq 0$, $r_1,\ldots,r_n\in\s{x,y,z}$, and $r_i\neq r_{i+1}$ for all $i\leq n-1$. We
  define the $T$-count of such an expression to be $n$; then this normal form has minimal
  $T$-count. Moreover, there exists a linear-time algorithm for symbolically reducing any sequence
  of Clifford+$T$ operators to this normal form.
\end{proposition}

The $T_x$-$T_y$-$T_z$ normal form is, in a sense, the most ``canonical'' one of the normal forms
considered here; it also explains why $T$-count is an appropriate measure of the size of a
Clifford+$T$ operator. In a physical quantum computer with error correction, there is in general no
reason to expect the $T_z$ gate to be more privileged than the $T_x$ or $T_y$ gates; one may
imagine that it would be efficient for a quantum computer to provide all three $T$-gates as
primitive logical operations.
% subsubsection _t_x_t_y_t_z_normal_form (end)

\subsubsection{Bocharov-Svore normal form} % (fold)
\label{ssub:bocharov_svore_normal_form}
Bocharov and Svore {\cite[Proposition 1]{BS}} consider the following normal form for single-qubit
Clifford+$T$ circuits:
\begin{equation}\label{eqn-bs1}
  (H\mid\emptyseq)\,(TH\mid SHTH)^*\,{\cC}.
\end{equation}
This normal form is not unique; for example, $H.H$ and $I$ are two different normal forms denoting
the same operator, as are $SHTH.Z$ and $H.SHTH$. (Here we have used a dot to delimit syllables;
this is for readability only). Recall that two regular expressions are {\em equivalent} if they
define the same set of strings. Using laws of regular expressions, we can equivalently rewrite
(\vref{eqn-bs1}) as
\begin{equation}\label{eqn-bs2}
  ((\epsilon\mid T\mid SHT) (HT\mid HSHT)^*\,H{\cC}) ~\mid~ \cC.
\end{equation}
Since $H\cC$ is just a redundant way to write a Clifford operator, we can simplify it to $\cC$;
moreover, in this case, $\epsilon\cC$ and $\cC$ are the same, so (\vref{eqn-bs2}) simplifies to
\begin{equation}\label{eqn-bs3}
    (\epsilon\mid T\mid SHT) (HT\mid HSHT)^*\,{\cC}.
\end{equation}
Moreover, since $SHT=HSHT.X$, any expression starting with $SHT$ can be rewritten as one starting
with $HSHT$, so the $SHT$ syllable is redundant and we can eliminate it:
\begin{equation}\label{eqn-bs4}
    (\epsilon\mid T) (HT\mid HSHT)^*\,{\cC}.
\end{equation}
Let us say that an operator is in {\em Bocharov-Svore normal form} if it is written in the form
(\vref{eqn-bs4}). This version of the Bocharov-Svore normal form is indeed unique; note that it is
almost the same as the Matsumoto-Amano normal form, except that the syllable $SHT$ has been
replaced by $HSHT$. Since the set $\sH=\s{I,H,HSH}$ satisfies Lemma~\vref{lem-SH}, existence,
uniqueness, $T$-optimality, and efficiency are proved in the same way as for the Matsumoto-Amano
and $E$-$T$ normal forms.

Bocharov and Svore {\cite[Prop.2]{BS}} also consider a second normal form, which has Clifford
operators on both sides, but the first four interior syllables restricted to $TH$:
\begin{equation}
  \cC\,(\epsilon\mid TH\mid (TH)^2 \mid (TH)^3 \mid (TH)^4(TH\mid SHTH)^*)\,\cC
\end{equation}
However, this normal form is not at all unique; for instance, $Z.TH$ and $TH.X$ denote the same
operator, as do $YS.TH.TH$ and $TH.TH.X\omega$.

% subsubsection bocharov_svore_normal_form (end)
% subsection alternative_normal_forms (end)

\subsection{Matsumoto-Amano normal forms and U(2)} % (fold)
\label{sub:matsumoto_amano_normal_forms_and_u_2_}

\begin{example}\label{exa-k-residue-for-single}
  Consider the matrix
  \[ U \,{=}\, \frac{\scriptstyle 1}{\scriptstyle \sqrt{2}^3}\!\scriptscriptstyle\zmatrix{cc}{
    \omega^2+\omega         & -2\omega^3+\omega^2+\omega \\
     \omega^3 -2\omega^2-1  & -\omega^3+1
    }.
  \]
  It has least denominator exponent $3$. Its $3$-, $4$-, and
  $5$-residues are:
  \begin{equation*}
    \rho_3(U) = \zmatrix{cc}{
      0110 & 0110\\
      1001 & 1001
    },\quad
    \rho_4(U) = \zmatrix{cc}{
      1111 & 1111\\
      1111 & 1111
    },\quad
    \rho_5(U) = 0.
  \end{equation*}
\end{example}

\begin{figure}
  \begin{tikzpicture}[node distance=1cm and 1.15cm]
    \node (filler) {};

    \node (r1c1) [below=of filler]
          {\umatrix{0001}{0000}{0000}{0001}};

    \node [xshift=-4mm,yshift=-1.5mm] at (r1c1.west) {Start:};

    \draw [->] ([xshift=-7mm] r1c1.north) arc[start angle=0, end angle=270, x radius=4mm, y radius=4mm];
    \node at ([xshift=.15cm, yshift=.55cm] r1c1.north west) {$\scriptscriptstyle \omega,S,X$};

    \node [xshift=-.3cm] at (r1c1.south east) {$\scriptscriptstyle H=T=2k=0$};

    \node (r2c1) [below=of r1c1]
          {\umatrix{0001}{0001}{0001}{0001}}
          \leftrightfrom{H}{k++}{r1c1};
    \node [xshift=-.2cm] at (r2c1.south east) {$\scriptscriptstyle H+1=T+2=2k=2$};

    \node (r2c2) [right=of r2c1]
          {\umatrix{0001}{0001}{0100}{0100}}
          edge [<-] node[above] {$\scriptstyle S$} (r2c1);
    \node [xshift=-.2cm] at (r2c2.south east) {$\scriptscriptstyle H+1=T+2=2k=2$};

    \node (r2c3) [right=of r2c2]
          {\umatrix{0001}{0010}{0001}{0010}};
    \node [xshift=-.2cm] at (r2c3.south east) {$\scriptscriptstyle H+1=T+1=2k=2$};

    \node (r2c4) [right=of r2c3]
          {\umatrix{0001}{0010}{0100}{1000}}
          edge [<-] node[above] {$\scriptstyle S$} (r2c3);
    \node [xshift=-.2cm] at (r2c4.south east) {$\scriptscriptstyle H+1=T+1=2k=2$};

    \node (r1c3) [above=of r2c3]
          {\umatrix{0001}{0000}{0000}{0010}}
          edge [<-] node[above ]{$\scriptstyle T$} (r1c1)
          \leftrightto{H}{k++}{r2c3};
    \node [xshift=-.3cm] at (r1c3.south east) {$\scriptscriptstyle H=T-1=2k=0$};

    \node (r3c1) [below=of r2c1]
          {\umatrix{0001}{0001}{0010}{0010}}
          edge [<-] node[left] {$\scriptstyle T$} (r2c1);
    \node [xshift=-.2cm] at (r3c1.south east) {$\scriptscriptstyle H+1=T+1=2k=2$};

    \node (r3c2) [right=of r3c1]
          {\umatrix{0001}{0001}{1000}{1000}}
          edge [<-] node[left] {$\scriptstyle T$} (r2c2);
    \node [xshift=-.2cm] at (r3c2.south east) {$\scriptscriptstyle H+1=T+1=2k=2$};

    \node (r3c3) [right=of r3c2]
          {\umatrix{0001}{0010}{0010}{0100}}
          edge [<-] node[left] {$\scriptstyle T$} (r2c3);
    \node [xshift=-.2cm] at (r3c3.south east) {$\scriptscriptstyle H+1=T=2k=2$};

    \node (r3c4) [right=of r3c3]
          {\umatrix{0001}{0010}{1000}{0001}}
          edge [<-] node[left] {$\scriptstyle T$} (r2c4);
    \node [xshift=-.2cm] at (r3c4.south east) {$\scriptscriptstyle H+1=T=2k=2$};

    \node (r4c1) [below=of r3c1]
          {\umatrix{0011}{0011}{0011}{0011}}
          \leftrightfrom{H}{k++}{r3c1}
          \starportfrom{H}{k++}{r3c2};
    \node [xshift=-.2cm] at (r4c1.south east) {$\scriptscriptstyle H+2=T+3=2k\ge4$};

    \node (r4c2) [right=of r4c1]
          {\umatrix{0011}{0011}{1100}{1100}}
          edge [<-] node[above] {$\scriptstyle S$} (r4c1);
    \node [xshift=-.2cm] at (r4c2.south east) {$\scriptscriptstyle H+2=T+3=2k\ge4$};

    \node (r4c3) [right=of r4c2]
          {\umatrix{0011}{0110}{0011}{0110}}
          \leftrightfrom{H}{k++}{r3c3}
          \starportfrom{H}{k++}{r3c4};
    \node [xshift=-.2cm] at (r4c3.south east) {$\scriptscriptstyle H+2=T+2=2k\ge4$};

    \node (r4c4) [right=of r4c3]
          {\umatrix{0011}{0110}{1100}{1001}}
          edge [<-] node[above] {$\scriptstyle S$} (r4c3);
    \node [xshift=-.2cm] at (r4c4.south east) {$\scriptscriptstyle H+2=T+2=2k\ge4$};

    \node (r5c1) [below=of r4c1]
          {\umatrix{0011}{0011}{0110}{0110}}
          edge [<-] node[left] {$\scriptstyle T$} (r4c1);
    \node [xshift=-.2cm] at (r5c1.south east) {$\scriptscriptstyle H+2=T+2=2k\ge4$};

    \node (r5c2) [right=of r5c1]
          {\umatrix{0011}{0011}{1001}{1001}}
          edge [<-] node[left] {$\scriptstyle T$} (r4c2);
    \node [xshift=-.2cm] at (r5c2.south east) {$\scriptscriptstyle H+2=T+2=2k\ge4$};

    \node (r5c3) [right=of r5c2]
          {\umatrix{0011}{0110}{0110}{1100}}
          edge [<-] node[left] {$\scriptstyle T$} (r4c3);
    \node [xshift=-.2cm] at (r5c3.south east) {$\scriptscriptstyle H+2=T+1=2k\ge4$};

    \node (r5c4) [right=of r5c3]
          {\umatrix{0011}{0110}{1001}{0011}}
          edge [<-] node[left] {$\scriptstyle T$} (r4c4);
    \node [xshift=-.2cm] at (r5c4.south east) {$\scriptscriptstyle H+2=T+1=2k\ge4$};

    \node (r6c1) [below=of r5c1]
          {\umatrix{0101}{0101}{0101}{0101}}
          \leftrightfrom{H}{k++}{r5c1}
          \starportfrom{H}{k++}{r5c2};
    \node [xshift=-.2cm] at (r6c1.south east) {$\scriptscriptstyle H+3=T+4=2k\ge6$};


    \node (r6c3) [below=of r5c3]
          {\umatrix{0101}{1010}{0101}{1010}}
          \leftrightfrom{H}{k++}{r5c3}
          \starportfrom{H}{k++}{r5c4};
    \node [xshift=-.2cm] at (r6c3.south east) {$\scriptscriptstyle H+3=T+3=2k\ge6$};

    \node (r7c1) [below=of r6c1]
          {\umatrix{0001}{1110}{1110}{0001}}
          edge [<-] node[left] {\footnotesize Reduce}  node[right] {$\scriptstyle k--$} (r6c1);
    \node [xshift=-.2cm] at (r7c1.south east) {$\scriptscriptstyle H+1=T+2=2k\ge4$};

    \node (r7c2) [right=of r7c1]
          {\umatrix{0001}{1110}{1011}{0100}}
          edge [<-] node[above] {$\scriptstyle S$} (r7c1);
    \node [xshift=-.2cm] at (r7c2.south east) {$\scriptscriptstyle H+1=T+2=2k\ge4$};

    \node (r7c3) [right=of r7c2]
          {\umatrix{0001}{1101}{1110}{0010}}
          edge [<-] node[left] {\footnotesize Reduce}  node[right] {$\scriptstyle k--$} (r6c3);
    \node [xshift=-.2cm] at (r7c3.south east) {$\scriptscriptstyle H+1=T+1=2k\ge4$};

    \node (r7c4) [right=of r7c3]
          {\umatrix{0001}{1101}{1011}{1000}}
          edge [<-] node[above] {$\scriptstyle S$} (r7c3);
    \node [xshift=-.2cm] at (r7c4.south east) {$\scriptscriptstyle H+1=T+1=2k\ge4$};


    \node (r8c1) [below=of r7c1]
          {\umatrix{0001}{1110}{1101}{0010}}
          edge [<-] node[left] {$\scriptstyle T$} (r7c1);
    \node [xshift=-.2cm] at (r8c1.south east) {$\scriptscriptstyle H+1=T+1=2k\ge4$};

    \node (r8c2) [right=of r8c1]
          {\umatrix{0001}{1110}{0111}{1000}}
          edge [<-] node[left] {$\scriptstyle T$} (r7c2);
    \node [xshift=-.2cm] at (r8c2.south east) {$\scriptscriptstyle H+1=T+1=2k\ge4$};

    \node (r8c3) [right=of r8c2]
          {\umatrix{0001}{1101}{1101}{0100}}
          edge [<-] node[left] {$\scriptstyle T$} (r7c3);
    \node [xshift=-.2cm] at (r8c3.south east) {$\scriptscriptstyle H+1=T=2k\ge4$};

    \node (r8c4) [right=of r8c3]
          {\umatrix{0001}{1101}{0111}{0001}}
          edge [<-] node[left] {$\scriptstyle T$} (r7c4);
    \node [xshift=-.2cm] at (r8c4.south east) {$\scriptscriptstyle H+1=T=2k\ge4$};

    \node (r9c1) [below=of r8c1]
          {\umatrix{0011}{0011}{0011}{0011}}
          \leftrightfrom{H}{k++}{r8c1}
          \starportfrom{H}{k++}{r8c2};
    \node [xshift=-.2cm] at (r9c1.south east) {$\scriptscriptstyle H+2=T+3=2k\ge6$};

    \draw [->, thick, double] ([xshift=.2cm] r9c1.west) to
            [out=180,in=270] ([xshift=-1.6cm,yshift=.5cm]  r9c1) to
            ([xshift=-1.6cm,yshift=-.5cm]  r4c1) to
            [out=90,in=180] ([xshift=.2cm]  r4c1.west);

    \node (r9c3) [below=of r8c3]
          {\umatrix{0011}{0110}{0011}{0110}}
          \leftrightfrom{H}{k++}{r8c3}
          \starportfrom{H}{k++}{r8c4};
    \node [xshift=-.2cm] at (r9c3.south east) {$\scriptscriptstyle H+2=T+2=2k\ge6$};

    \draw [->, thick, double] ([xshift=.2cm] r9c3.west) to
            [out=180,in=270] ([xshift=-1.6cm,yshift=.5cm]  r9c3) to
            ([xshift=-1.6cm,yshift=-.5cm]  r4c3) to
            [out=90,in=180] ([xshift=.2cm]  r4c3.west);
  \end{tikzpicture}
  \caption{Transitions of residue matrices in U2 when applying the
  Matsumoto-Amano algorithm}
  \label{fig:u2-transitions}
\end{figure}

\begin{definition}
  Let $\sW$ be the 64-element subgroup of the Clifford group in U(2) spanned by $S, X$ and
  $\omega$. Then $\sim_\sW$ defined by right multiplication by $\sW$ is an equivalence relation on
  the residue matrices of operators in the Clifford+$T$ group.
\end{definition}
\begin{remark}
  The equivalence relation $\sim_\sW$ is characterized by the following operations:
  \begin{enumerate}
    \item ``Rotating'' all of the entries in the matrix by 1,2 or 3 positions. This corresponds to
      multiplication by a power of $\omega$.
    \item Swapping the two columns. The corresponds to the right action of $X$.
    \item ``Rotating'' the entries of the second column by two positions. This corresponds to the
      right action of $S$.
  \end{enumerate}
\end{remark}
\begin{remark}
  The residue matrices in figure \vref{fig:u2-transitions} are modulo $\sim_\sW$.
\end{remark}
\begin{lemma}\label{lem-columncannotbe1s}
  Let $k\ge2$ and $b,d \in \Zw$. Then $\frac{1}{\sqrt{2}^k}\begin{pmatrix}1+2b\\1+2d\end{pmatrix}$
  is not a unit vector.
\end{lemma}
\begin{proof}
  Suppose otherwise, then
  \begin{align*}
    2^k = &1+2(b+b^t) +4bb^t +\\
          &1+2(d+d^t) +4dd^t,
  \end{align*}
  so $2 = 2^k - 2(b+b^t) - 2(d+d^t) - 4(bb^t + dd^t)$. By lemma \vref{lem-divisiblebyroot2}, the
  right hand side of this is divisible in $\Zw$ by $2 \sqrt{2}$, while the left hand side is not.
  Thus we have a contradiction.
\end{proof}
\begin{lemma}\label{lem-reduction}
  Given a unitary matrix $U \in \Dw^{2\times2}$ with least denominator exponent $k\ge2$, such that:
  \begin{enumerate}
    \item $\rho_{k+1}(U) = \umatrix{0101}{0101}{0101}{0101}$ and
    \item $\rho_k(H U) = \umatrix{0011}{0011}{0110}{0110}$,
  \end{enumerate}
  then, $\rho_k(U) = \umatrix{0010}{1101}{1101}{0010}$
\end{lemma}
\begin{proof}
  Referencing Table \vref{tab-residue}, we see the first condition limits the possible choices for
  the entries of $\rho_k(U)$ to the set $\{0010, 0111, 1000, 1101\}$. The second condition implies
  that $\rho_{k+1}(H U)$ is reducible and in fact that each entry is $1111$. This means each column
  of $U$ must be either $[0010, 1101]^t, [1101, 0010]^t, [0111,1000]^t$ or $[1000,0111]^t$. As we
  are considering equivalence classes, we can assume without loss of generality, that the columns
  are in $\{[0010, 1101]^t, [1101, 0010]^t\}$. But by Lemma \vref{lem-columncannotbe1s}, we can not
  have a row like $[0010, 0010]$, therefore $U = \umatrix{0010}{1101}{1101}{0010}$.
\end{proof}
\begin{corollary}
  In Figure \vref{fig:u2-transitions}, the transitions labelled with ``Reduce'' are correct.
\end{corollary}
\begin{proof}
  For the left most reduce, we directly apply Lemma \vref{lem-reduction} and then note that
  \[\umatrix{0001}{1110}{1110}{0001}\sim_\sW\umatrix{0010}{1101}{1101}{0010}.\] For the rightmost
  reduce, the same argument as shown in Lemma \vref{lem-reduction} can be applied to the
  preconditions of this reduce, giving us a similar result.
\end{proof}


% subsection matsumoto_amano_normal_forms_and_u_2_ (end)
% section exact_synthesis_of_single_qubit_operators (end)


\section{Exact synthesis of multi-qubit operators} % (fold)
\label{sec:exact_synthesis_of_multi_qubit_operators}
Here, we focus on the problem of exact synthesis for $n$-qubit operators, using the Clifford+$T$
universal gate set. Recall that the Clifford group on $n$ qubits is generated by the Hadamard gate
$H$, the phase gate $S$, the controlled-not gate, and the scalar $\omega=e^{i\pi/4}$ (one may allow
arbitrary unit scalars, but it is not convenient for our purposes to do so). It is well-known that
one obtains a universal gate set by adding the non-Clifford operator $T$
{\cite{neilsen2000:QuantumComputationAndInfo}}.

\begin{equation}\label{eqn-generators}
  \begin{array}{c}
  \displaystyle
  \omega = e^{i\pi/4},\quad
  H = \frac{1}{\sqrt{2}}\zmatrix{cc}{1&1\\1&-1},\quad
  S = \zmatrix{cc}{1&0\\0&i},\\\\[-1.5ex]
  \displaystyle
  {\it CNOT} = \zmatrix{cccc}{1&0&0&0\\0&1&0&0\\0&0&0&1\\0&0&1&0},\quad
  T = \zmatrix{cc}{1&0\\0&e^{i\pi/4}}.
\end{array}
\end{equation}

In addition to the Clifford+$T$ group on $n$ qubits, as defined above, we also consider the
slightly larger group of Clifford+$T$ operators ``with ancillas''. We say that an $n$-qubit
operator $U$ is a Clifford+$T$ operator {\em with ancillas} if there exists $m\geq 0$ and a
Clifford+$T$ operator $U'$ on $n+m$ qubits, such that $U'(\ket{\phi}\otimes\ket{0})=
(U\ket{\phi})\otimes\ket{0}$ for all $n$-qubit states $\ket{\phi}$.

Kliuchnikov, Maslov, and Mosca {\cite{Kliuchnikov-et-al}} showed that a single-qubit operator $U$
is in the Clifford+$T$ group if and only if all of its matrix entries belong to the ring
$\Z[\frac1{\sqrt{2}},i]$. They also showed that the Clifford+$T$ groups ``with ancillas'' and
``without ancillas'' coincide for $n=1$, but not for $n\geq 2$. Moreover, Kliuchnikov et
al.~conjectured that for all $n$, an $n$-qubit operator $U$ is in the Clifford+$T$ group with
ancillas if and only if its matrix entries belong to $\Z[\frac1{\sqrt{2}},i]$. They also
conjectured that a single ancilla qubit is always sufficient in the representation of a
Clifford+$T$ operator with ancillas. This section of the thesis will prove these conjectures. In
particular, this yields an algorithm for exact Clifford+$T$ synthesis of $n$-qubit operators. We
also obtain a characterization of the Clifford+$T$ group on $n$ qubits without ancillas.

It is important to note that, unlike in the single-qubit case, the circuit synthesized here are not
in any sense canonical, and very far from optimal. Thus, the question of {\em efficient} synthesis
is not addressed here.

\subsection{Decomposition into two-level matrices} % (fold)
\label{sub:decomposition_into_two_level_matrices}
Recall that a {\em two-level matrix} is an $n\times n$-matrix that acts non-trivially on at most
two vector components {\cite{neilsen2000:QuantumComputationAndInfo}}. If
\[
  U=\zmatrix{cc}{a&b\\c&d}
\]
is a $2\times 2$-matrix and $\jay\neq\ell$, we write $U\level{\jay,\ell}$ for the two-level
$n\times n$-matrix defined by
\[
  U\level{\jay,\ell}=\begin{blockarray}{cccccc}
    &\matindex{\cdots}& \matindex{\jay} &\matindex{\cdots}& \matindex{\ell} &\matindex{\cdots} \\
    \begin{block}{c(c|c|c|c|c)}
      \matindex{\vdots} & \bigI &&&& \\\cline{2-6}
      \matindex{\jay} & & a && b \\\cline{2-6}
      \matindex{\vdots} & && \bigI && \\\cline{2-6}
      \matindex{\ell} & & c && d &  \\\cline{2-6}
      \matindex{\vdots} &&&&& \bigI \\
    \end{block}
  \end{blockarray}\,,
\]
and we say that $U\level{\jay,\ell}$ is a two-level matrix {\em of type $U$}. Similarly, if $a$ is
a scalar, we write $a\level{\jay}$ for the one-level matrix
\[
  a\level{\jay} = \begin{blockarray}{cccc}
    &\matindex{\cdots}& \matindex{\jay} &\matindex{\cdots} \\
    \begin{block}{c(c|c|c)}
      \matindex{\vdots} & \bigI && \\\cline{2-4}
      \matindex{\jay} & & a & \\\cline{2-4}
      \matindex{\vdots} & && \bigI\\
    \end{block}
  \end{blockarray}\,,
\]
and we say that $a\level{\jay}$  is a one-level matrix {\em of type $a$}.

\begin{lemma}[Row operation]\label{lem-row}
  Let $u=(u_1,u_2)^T\in\Dw^2$ be a vector with denominator exponent $k>0$ and $k$-residue
  $\rho_k(u)=(x_1,x_2)$, such that $x_1\da x_1 = x_2\da x_2$. Then there exists a sequence of
  matrices $U_1,\ldots,U_h$, each of which is $H$ or $T$, such that $v=U_1\cdots U_hu$ has
  denominator exponent $k-1$, or equivalently, $\rho_k(v)$ is defined and reducible.
\end{lemma}

\begin{proof}
  It can be seen from Table~\vref{tab-residue} that $x_1\da x_1$ is either $0000$, $1010$, or $0001$.
  \begin{itemize}
  \item Case 1: $x_1\da x_1 = x_2\da x_2 = 0000$. In this case, $\rho_k(u)$ is already reducible,
    and there is nothing to show.
  \item Case 2: $x_1\da x_1 = x_2\da x_2 = 1010$. In this case, we know from
    Table~\vref{tab-residue} that $x_1,x_2\in\s{0011, 0110, 1100, 1001}$. In particular, $x_1$ is a
    cyclic permutation of $x_2$, say, $x_1=\omega^m x_2$. Let $v=HT^mu$. Then
    \[
      \begin{split}
        \rho_k(\sqrt{2}\,v) &=
        \rho_k(\zmatrix{cc}{1&1\\1&-1}\zmatrix{cc}{1&0\\0&\omega^m}\zmatrix{c}{u_1\\u_2})
        \\&= \rho_k\zmatrix{c}{u_1+\omega^m u_2\\ u_1-\omega^m u_2} \\&=
        \zmatrix{c}{x_1+\omega^m x_2\\x_1-\omega^m x_2} =
        \zmatrix{c}{0000\\0000}.
      \end{split}
    \]
    This shows that $\rho_k(\sqrt{2}\,v)$ is twice reducible; therefore, $\rho_k(v)$ is defined and
    reducible as claimed.
  \item Case 3: $x_1\da x_1 = x_2\da x_2 = 0001$. In this case, we know from Table~\vref{tab-residue}
    that $x_1,x_2\in\s{0001,0010,0100,1000}\cup\s{0111,1110,1101,1011}$. If both $x_1,x_2$ are in
    the first set, or both are in the second set, then $x_1$ and $x_2$ are cyclic permutations of
    each other, and we proceed as in case 2. The only remaining cases are that $x_1$ is a cyclic
    permutation of $0001$ and $x_2$ is a cyclic permutation of $0111$, or vice versa. But then
    there exists some $m$ such that $x_1+\omega^mx_2=1111$. Letting $u'=HT^mu$, we have
    \[
      \begin{split}
        \rho_k(\sqrt{2}\,u') &=
        \rho_k(\zmatrix{cc}{1&1\\1&-1}\zmatrix{cc}{1&0\\0&\omega^m}\zmatrix{c}{u_1\\u_2})
        \\&= \rho_k\zmatrix{c}{u_1+\omega^m u_2\\ u_1-\omega^m u_2} \\&=
        \zmatrix{c}{x_1+\omega^m x_2\\x_1-\omega^m x_2} =
        \zmatrix{c}{1111\\1111}.
      \end{split}
    \]
    Since this is reducible, $u'$ has denominator exponent $k$. Let $\rho_k(u')=(y_1,y_2)$. Because
    $\sqrt{2}\,y_1=\sqrt{2}\,y_2=1111$, we see from Table~\vref{tab-residue} that
    $y_1,y_2\in\s{0011,0110,1100,1001}$ and $y_1\da y_1=y_2\da y_2=1010$. Therefore, $u'$ satisfies
    the condition of case 2 above. Proceeding as in case 2, we find $m'$ such that $v=HT^{m'}
    u'=HT^{m'} HT^mu$ has denominator exponent $k-1$. This finishes the proof.\qedhere
  \end{itemize}
\end{proof}

\begin{lemma}[Column lemma]\label{lem-column}
  Consider a unit vector $u\in\Dw^n$, i.e., an $n$-dimensional column vector of norm 1 with entries
  from the ring $\Dw$. Then there exist a sequence $U_1,\ldots,U_h$ of one- and two-level unitary
  matrices of types $X$, $H$, $T$, and $\omega$ such that $U_1\cdots U_hu = e_1$, the first
  standard basis vector.
\end{lemma}

\begin{proof}
  The proof is by induction on $k$, the least denominator exponent of $u$. Let
  $u=(u_1,\ldots,u_n)^T$.
  \begin{itemize}
  \item Base case. Suppose $k=0$. Then $u\in\Zw^n$. Since by assumption $\norm{u}^2=1$, it follows
    by Lemma~\vref{lem-ring-norm} that $\weight{u}^2=1$. Since $u_1,\ldots,u_n$ are elements of
    $\Zw$, their weights are non-negative integers. It follows that there is precisely one $\jay$
    with $\weight{u_\jay}=1$, and $\weight{u_\ell}=0$ for all $\ell\neq \jay$. Let
    $u'=X\level{1,\jay}u$ if $\jay\neq 1$, and $u'=u$ otherwise. Now $u'_1$ is of the form
    $\omega^{-m}$, for some $m\in\s{0,\ldots,7}$, and $u'_\ell=0$ for all $\ell\neq 1$. We have
    $\omega^m\level{1}u'=e_1$, as desired.
  \item Induction step. Suppose $k>0$. Let $v=\sqrt{2}^ku\in\Zw^n$, and let $x=\rho_k(u) =
    \rho(v)$. From $\norm{u}^2=1$, it follows that $\norm{v}^2 = v_1\da v_1+\ldots+v_n\da v_n = 2^k$.
    Taking residues of the last equation, we have
    \begin{equation}
      x_1\da x_1+\ldots+x_n\da x_n = 0000.
    \end{equation}
    It can be seen from Table~\vref{tab-residue} that each summand $x_\jay\da x_\jay$ is either
    $0000$, $0001$, or $1010$. Since their sum is $0000$, it follows that there is an even number
    of $\jay$ such that $x_\jay\da x_\jay=0001$, and an even number of $\jay$ such that $x_\jay\da
    x_\jay=1010$.

    We do an inner induction on the number of irreducible components of $x$. If $x$ is reducible,
    then $u$ has denominator exponent $k-1$ by Corollary~\vref{cor-reducible3}, and we can apply the
    outer induction hypothesis. Now suppose there is some $\jay$ such that $x_\jay$ is irreducible;
    then $x_\jay\da x_\jay\neq 0000$ by Lemma~\vref{lem-reducible}. Because of the evenness property
    noted above, there must exist some $\ell\neq \jay$ such that $x_\jay\da x_\jay=x_\ell\da
    x_\ell$. Applying Lemma~\vref{lem-row} to $u'=(u_\jay,u_\ell)^T$, we find a sequence $\vec U$ of
    row operations of types $H$ and $T$, making $\rho_k(\vec Uu')$ reducible. We can lift this to a
    two-level operation $\vec U\level{\jay,\ell}$ acting on $u$; thus $\rho_k(\vec
    U\level{\jay,\ell}u)$ has fewer irreducible components than $x=\rho_k(u)$, and the inner
    induction hypothesis applies.\qedhere
  \end{itemize}
\end{proof}

\begin{lemma}[Matrix decomposition]\label{lem-matrix-decomposition}
  Let $U$ be a unitary $n\times n$-matrix with entries in $\Dw$. Then there exists a sequence
  $U_1,\ldots,U_h$ of one- and two-level unitary matrices of types $X$, $H$, $T$, and $\omega$ such
  that $U=U_1\cdots U_h$.
\end{lemma}

\begin{proof}
  Equivalently, it suffices to show that there exist one- and two-level unitary matrices
  $V_1,\ldots,V_h$ of types $X$, $H$, $T$, and $\omega$ such that $V_h\cdots V_1U=I$. This is an
  easy consequence of the column lemma, exactly as in e.g. {\cite[Sec.~4.5.1]{neilsen2000:QuantumComputationAndInfo}}.
  Specifically, first use the column lemma to find suitable one- and two-level row operations
  $V_1,\ldots,V_{h_1}$ such that the leftmost column of $V_{h_1}\cdots V_1 U$ is $e_1$. Because
  $V_{h_1}\cdots V_1 U$ is unitary, it is of the form
  \[
    \zmatrix{c|c}{1&0\\\hline \rule{0mm}{2.1ex}0&U'}.
  \]
  Now recursively find row operations to reduce $U'$ to the identity matrix.
\end{proof}

\begin{example}\label{exa-decomp}
  We will decompose the matrix $U$ from Example~\ref{exa-k-residue}. We start with the first column
  $u$ of $U$:
	\[
    \begin{array}{c}
    \displaystyle
    u=\frac{1}{\sqrt{2}^3}\zmatrix{c}{
    -\omega^3+\omega -1\\
    \omega^2+\omega \\
    \omega^3+\omega^2\\
    -1
    }, \\\\[-1.5ex]
    \rho_3(u) = \zmatrix{c}{
    1011\\
    0110\\
    1100\\
    0001
    },\quad \rho_3(u_\jay\da u_\jay) = \zmatrix{c}{
    0001\\
    1010 \\
    1010\\
    0001
    }.
    \end{array}
  \]
	Rows 2 and 3 satisfy case 2 of Lemma~\vref{lem-row}. As they are not aligned, first apply
  $T_{[2,3]}^3$ and then $H_{[2,3]}$. Rows 1 and 4 satisfy case 3. Applying $H_{[1,4]} T_{[1,4]}^2$,
  the residues become $\rho_3(u'_1)=0011$ and $\rho_3(u'_4)=1001$, which requires applying $H_{[1,4]}
  T_{[1,4]}$. We now have
	\[
    \begin{array}{@{}c@{}}
      \displaystyle
      H_{[1,4]} T_{[1,4]} H_{[1,4]} T_{[1,4]}^2 H_{[2,3]} T_{[2,3]}^3 u=
  	  v=\frac{1}{\footnotesize\sqrt{2}^2}\small\zmatrix{c}{
      0 \\
  		0 \\
      \omega^2{+}\omega  \\
      -\omega {+}1
      }, \\\\[-1.5ex]
      \rho_2(v) = \zmatrix{c}{
      0000\\
      0000\\
      0110\\
      0011
      },\quad \rho_2(v_\jay\da v_\jay) = \zmatrix{c}{
      0000\\
      0000\\
      1010\\
      1010
      }.
    \end{array}
  \]
	Rows 3 and 4 satisfy case 2, while rows 1 and 2 are already reduced. We reduce rows 3 and 4 by
  applying $H_{[3,4]} T_{[3,4]}$. Continuing, the first column is completely reduced to $e_1$ by
  further applying $\omega_{[1]}^7 X_{[1,4]} H_{[3,4]} T_{[3,4]}^3$. The complete decomposition of
  $u$ is therefore given by
	\[
    \begin{split}
      W_1={}&\omega_{[1]}^7  X_{[1,4]} H_{[3,4]} T_{[3,4]}^3H_{[3,4]} T_{[3,4]}\\&
      H_{[1,4]} T_{[1,4]} H_{[1,4]} T_{[1,4]}^2 H_{[2,3]} T_{[2,3]}^3.
    \end{split}
    \]
	Applying this to the original matrix $U$, we have $W_1U=$
	\[
    \small \frac{\small 1}{\small\sqrt{2}^3}\footnotesize\zmatrix{cccc}{
    \sqrt{2}^3             & 0      & 0                           & 0\\
    0 & \omega^3{-}\omega^2{+}\omega{+}1  & {-}\omega^2{-}\omega{-}1          & \omega^2\\
    0 & 0                           & \omega^3{+}\omega^2{-}\omega{+}1  & \omega^3{+}\omega^2{-}\omega{-}1\\
    0 & \omega^3{+}\omega^2{+}\omega{+}1 & \omega^2                    & \omega^3{-}\omega^2{+}1
    }.
  \]
  Continuing with the rest of the columns, we find $W_2 = \omega_{[2]}^6 H_{[2,4]} T_{[2,4]}^3
  H_{[2,4]} T_{[2,4]}$, $W_3 = \omega_{[3]}^4 H_{[3,4]} T_{[3,4]}^3 H_{[3,4]}$, and
  $W_4=\omega_{[4]}^5$. We then have $U = W_1\da\, W_2\da\, W_3\da\, W_4\da$, or explicitly:
  \[
    \begin{split}
      U ={}&  T_{[2,3]}^5 H_{[2,3]} T_{[1,4]}^6 H_{[1,4]} T_{[1,4]}^7 H_{[1,4]}\\&
    					T_{[3,4]}^7 H_{[3,4]}
    					T_{[3,4]}^5 H_{[3,4]}
                                            X_{[1,4]} \omega_{[1]}\\&
      T_{[2,4]}^7 H_{[2,4]} T_{[2,4]}^5 H_{[2,4]} \omega_{[2]}^2
      H_{[3,4]} T_{[3,4]}^5 H_{[3,4]} \omega_{[3]}^4
      \omega_{[4]}^3.
    \end{split}
  \]
\end{example}


% subsection decomposition_into_two_level_matrices (end)

\subsection{Main result} % (fold)
\label{sub:main_result}
Consider the ring $\Z[\frac1{\sqrt{2}},i]$, consisting of complex numbers of the form
\[
  \frac{1}{2^n} (a + bi + c\sqrt{2} + di\sqrt{2}),
\]
where $n\in\N$ and $a,b,c,d\in\Z$. Our goal is to prove the following theorem, which was
conjectured by Kliuchnikov et al.~{\cite{Kliuchnikov-et-al}}:
\begin{theorem}\label{thm-main}
  Let $U$ be a unitary $2^n\times 2^n$ matrix. Then the following are equivalent:
\begin{enumerate}\alphalabels
  \item[(a)] $U$ can be exactly represented by a quantum circuit over the Clifford+$T$ gate set,
    possibly using some finite number of ancillas that are initialized and finalized in state $\ket0$.
  \item[(b)] The entries of $U$ belong to the ring $\Z[\frac1{\sqrt{2}},i]$.
\end{enumerate}
Moreover, in (a), a single ancilla is always sufficient.
\end{theorem}
\begin{proof}
  First note that, since all the elementary Clifford+$T$ gates, as shown in (\vref{eqn-generators}),
  take their matrix entries in $\Dw=\Z[\frac1{\sqrt{2}},i]$, the implication (a) $\implies$ (b) is
  trivial. For the converse, let $U$ be a unitary $2^n\times 2^n$ matrix with entries from $\Dw$.
  By Lemma~\vref{lem-matrix-decomposition}, $U$ can be decomposed into one- and two-level matrices
  of types $X$, $H$, $T$, and $\omega$. It is well-known that each such matrix can be further
  decomposed into controlled-not gates and multiply-controlled $X$, $H$, $T$, and $\omega$-gates,
  for example using Gray codes {\cite[Sec.~4.5.2]{neilsen2000:QuantumComputationAndInfo}}. But all of these gates have
  well-known exact representations in Clifford+$T$ with ancillas, see e.g. {\cite[Fig.~4(a) and
  Fig.~9]{AMMR12}} (and noting that a controlled-$\omega$ gate is the same as a $T$-gate). This
  finishes the proof of (b) $\implies$ (a).

  The final claim that needs to be proved is that a circuit for $U$ can always be found using at
  most one ancilla. It is already known that for $n>1$, an ancilla is sometimes necessary
  {\cite{Kliuchnikov-et-al}}. To show that a single ancilla is sufficient, in light of the above
  decomposition, it is enough to show that the following can be implemented with one ancilla:
  \begin{enumerate}\alphalabels
  \item a multiply-controlled $X$-gate;
  \item a multiply-controlled $H$-gate;
  \item a multiply-controlled $T$-gate.
  \end{enumerate}
    We first recall from {\cite[Fig.~4(a)]{AMMR12}} that a
    singly-controlled Hadamard gate can be decomposed into Clifford+$T$
    gates with no ancillas:
  \[
    \m{\begin{tikzqcircuit}[scale=0.5]
      \grid{2}{0,1};
      \controlled{\tqgate{$H$}}{1,0}{1};
    \end{tikzqcircuit}}
  =\!\!\!
    \m{\begin{tikzqcircuit}[scale=0.5]
      \grid{11}{0,1};
      \tqgate{$S$}{1,0};
      \tqgate{$H$}{2.5,0};
      \tqgate{$T$}{4,0};
      \controlled{\notgate}{5.5,0}{1};
      \widegate{$T\da$}{.55}{7,0};
      \tqgate{$H$}{8.5,0};
      \widegate{$S\da$}{.55}{10,0};
    \end{tikzqcircuit}.}
  \]
  We also recall that an $n$-fold controlled $iX$-gate can be represented using $O(n)$ Clifford+$T$
  gates with no ancillas. Namely, for $n=1$, we have
  \[
    \m{\begin{tikzqcircuit}[scale=0.5]
        \grid{2}{1,2};
        \controlled{\widegate{$iX$}{.65}}{1,1}{2};
      \end{tikzqcircuit}}
      =\!\!\!
      \m{\begin{tikzqcircuit}[scale=0.5]
        \grid{3.5}{1,2};
        \tqgate{$S$}{1,2};
        \controlled{\notgate}{2.5,1}{2};
      \end{tikzqcircuit},}
  \]
  and for $n\geq 2$, we can use
  \[
      \m{\begin{tikzqcircuit}[scale=0.5]
      \grid{2}{1,2,3,4,5};
      \draw(0.4,2.7) node[anchor=center]{$\vdots$};
      \draw(0.4,4.7) node[anchor=center]{$\vdots$};
      \controlled{\widegate{$iX$}{.65}}{1,1}{2,3,4,5};
      \draw(1.6,2.7) node[anchor=center]{$\vdots$};
      \draw(1.6,4.7) node[anchor=center]{$\vdots$};
    \end{tikzqcircuit}}
    =\!\!\!
    \m{\begin{tikzqcircuit}[scale=0.5]
      \gridx{0.1}{13.5}{1,2,3,4,5};
      \tqgate{$H$}{1.1,1};
      \widegate{$T\da$}{.55}{2.5,1};
      \controlled{\notgate}{3.75,1}{2,3};
      \tqgate{$T$}{5,1};
      \controlled{\notgate}{6.25,1}{4,5};
      \widegate{$T\da$}{.55}{7.5,1};
      \controlled{\notgate}{8.75,1}{2,3};
      \tqgate{$T$}{10,1};
      \controlled{\notgate}{11.25,1}{4,5};
      \tqgate{$H$}{12.5,1};
      \draw(3.25,2.7) node[anchor=center]{$\vdots$};
      \draw(4.25,2.7) node[anchor=center]{$\vdots$};
      \draw(5.75,4.7) node[anchor=center]{$\vdots$};
      \draw(6.75,4.7) node[anchor=center]{$\vdots$};
      \draw(8.25,2.7) node[anchor=center]{$\vdots$};
      \draw(9.25,2.7) node[anchor=center]{$\vdots$};
      \draw(10.75,4.7) node[anchor=center]{$\vdots$};
      \draw(11.75,4.7) node[anchor=center]{$\vdots$};
    \end{tikzqcircuit},}
  \]
  with further decompositions of the multiply-controlled not-gates as in
  {\cite[Lem.~7.2]{Barenco-etal-1995}} and {\cite[Fig.~4.9]{neilsen2000:QuantumComputationAndInfo}}.
  We then obtain the following representations for (a)--(c), using only one ancilla:
  \[
  (a)
    \m{\begin{tikzqcircuit}[scale=0.47]
      \grid{2}{0.5,2,3};
      \draw(0.4,2.7) node[anchor=center]{$\vdots$};
      \controlled{\tqgate{$X$}}{1,0.5}{2,3};
      \draw(1.6,2.7) node[anchor=center]{$\vdots$};
    \end{tikzqcircuit}}
    =\!\!
    \m{\begin{tikzqcircuit}[scale=0.47]
      \grid{8.5}{0,2,3};
      \gridx{1}{7.5}{1};
      \draw(1,2.7) node[anchor=center]{$\vdots$};
      \init{$0$}{1,1};
      \controlled{\widegate{$iX$}{.75}}{2.5,1}{2,3};
      \controlled{\tqgate{$X$}}{4.25,0}{1};
      \controlled{\widegate{$-iX$}{.75}}{6,1}{2,3};
      \term{$0$}{7.5,1};
      \draw(7.5,2.7) node[anchor=center]{$\vdots$};
    \end{tikzqcircuit}}
  \]\[
  (b)
    \m{\begin{tikzqcircuit}[scale=0.47]
      \grid{2}{0.5,2,3};
      \draw(0.4,2.7) node[anchor=center]{$\vdots$};
      \controlled{\tqgate{$H$}}{1,0.5}{2,3};
      \draw(1.6,2.7) node[anchor=center]{$\vdots$};
    \end{tikzqcircuit}}
    =\!\!
    \m{\begin{tikzqcircuit}[scale=0.47]
      \grid{8.5}{0,2,3};
      \gridx{1}{7.5}{1};
      \draw(1,2.7) node[anchor=center]{$\vdots$};
      \init{$0$}{1,1};
      \controlled{\widegate{$iX$}{.75}}{2.5,1}{2,3};
      \controlled{\tqgate{$H$}}{4.25,0}{1};
      \controlled{\widegate{$-iX$}{.75}}{6,1}{2,3};
      \term{$0$}{7.5,1};
      \draw(7.5,2.7) node[anchor=center]{$\vdots$};
    \end{tikzqcircuit}}
  \]\[
  (c)
    \m{\begin{tikzqcircuit}[scale=0.47]
      \grid{2}{0.5,2,3};
      \draw(0.4,2.7) node[anchor=center]{$\vdots$};
      \controlled{\tqgate{$T$}}{1,0.5}{2,3};
      \draw(1.6,2.7) node[anchor=center]{$\vdots$};
    \end{tikzqcircuit}}
    =\!\!\!
    \m{\begin{tikzqcircuit}[scale=0.47]
      \grid{8.5}{1,2,3};
      \gridx{1}{7.5}{0};
      \draw(1,2.7) node[anchor=center]{$\vdots$};
      \init{$0$}{1,0};
      \controlled{\widegate{$iX$}{.75}}{2.5,0}{1,2,3};
      \tqgate{$T$}{4.25,0}{1};
      \controlled{\widegate{$-iX$}{.75}}{6,0}{1,2,3};
      \term{$0$.}{7.5,0};
      \draw(7.5,2.7) node[anchor=center]{$\vdots$};
    \end{tikzqcircuit}}
  \]
\end{proof}

\begin{remark}
  The fact that one ancilla is always sufficient in Theorem~\vref{thm-main} is primarily of
  theoretical interest. In practice, one may assume that on most quantum computing architectures,
  ancillas are relatively cheap. Moreover, the use of additional ancillas can significantly reduce
  the size and depth of the generated circuits (see e.g.~\cite{Selinger-toffoli}).
\end{remark}

% subsection main_result (end)

\subsection{The no-ancilla case} % (fold)
\label{sub:the_no_ancilla_case}
\begin{lemma}\label{lem-det1}
  Under the hypotheses of Theorem~\vref{thm-main}, assume that $\det U=1$. Then $U$ can be exactly
  represented by a Clifford+$T$ circuit with no ancillas.
\end{lemma}

\begin{proof}
  This requires only minor modifications to the proof of Theorem~\vref{thm-main}. First observe that
  when an operator of the form $HT^m$ was used in the proof of Lemma~\vref{lem-row}, we can
  instead use $T^{-m}(iH)T^m$ without altering the rest of the argument. In the base case of
  Lemma~\vref{lem-column}, the operator $X\level{1,\jay}$ can be replaced by $iX\level{1,\jay}$.
  Also, in the base case of Lemma~\vref{lem-column}, whenever $n\geq 2$, the operator
  $\omega\level{1}$ can be replaced by $W\level{1,2}$, where
  \[
    W = \zmatrix{cc}{\omega&0\\0&\omega^{-1}}.
  \]
  Therefore, the decomposition of Lemma~\vref{lem-matrix-decomposition} can be performed so as to
  yield only two-level matrices of types
  \begin{equation}\label{eqn-det1}
    iX,\quad T^{-m}(iH)T^m,\quad \mbox{and}\ W,
  \end{equation}
  plus at most one one-level matrix of type $\omega^m$. But since all two-level matrices of types
  (\vref{eqn-det1}), as well as $U$ itself, have determinant 1, it follows that $\omega^m = 1$. We
  finish the proof by observing that the multiply-controlled operators of types (\vref{eqn-det1})
  possess ancilla-free Clifford+$T$ representations, with the latter two given by
  \[
    \m{\begin{tikzqcircuit}[scale=0.47]
        \grid{5}{1,2,3};
        \draw(1,2.7) node[anchor=center]{$\vdots$};
        \controlled{\widegate{$T^{-m}(iH)T^m$}{2}}{2.5,1}{2,3};
        \draw(4,2.7) node[anchor=center]{$\vdots$};
      \end{tikzqcircuit}}
    =\!\!
    \m{\begin{tikzqcircuit}[scale=0.47]
        \grid{14.9}{1,2,3};
        \draw(1,2.7) node[anchor=center]{$\vdots$};
        \widegate{$T^m$}{.55}{1,1};
        \tqgate{$S$}{2.5,1};
        \tqgate{$H$}{4,1};
        \tqgate{$T$}{5.5,1};
        \controlled{\widegate{$iX$}{.75}}{7.25,1}{2,3};
        \widegate{$T\da$}{.55}{9,1};
        \tqgate{$H$}{10.5,1};
        \widegate{$S\da$}{.55}{12,1};
        \widegate{$T^{-m}$}{.75}{13.75,1};
        \draw(13.9,2.7) node[anchor=center]{$\vdots$};
      \end{tikzqcircuit}}
  \]\[
  \m{\begin{tikzqcircuit}[scale=0.47]
      \grid{2}{1,2,3};
      \draw(0.4,2.7) node[anchor=center]{$\vdots$};
      \controlled{\widegate{$W$}{.6}}{1,1}{2,3};
      \draw(1.6,2.7) node[anchor=center]{$\vdots$};
    \end{tikzqcircuit}}
  =\!\!
  \m{\begin{tikzqcircuit}[scale=0.47]
      \grid{7.5}{1,2,3};
      \draw(0.4,2.7) node[anchor=center]{$\vdots$};
      \controlled{\widegate{$iX$}{.75}}{1.25,1}{2,3};
      \tqgate{$T$}{3,1};
      \controlled{\widegate{$-iX$}{.75}}{4.75,1}{2,3};
      \widegate{$T\da$}{.55}{6.5,1};
      \draw(7.1,2.7) node[anchor=center]{$\vdots$};
    \end{tikzqcircuit}}
  \]
\end{proof}

As a corollary, we obtain a characterization of the $n$-qubit Clifford+$T$ group (with no ancillas)
for all $n$:

\begin{corollary}\label{cor-noancilla}
  Let $U$ be a unitary $2^n\times 2^n$ matrix. Then the following are equivalent:
\begin{enumerate}\alphalabels
\item[(a)] $U$ can be exactly represented by a quantum
  circuit over the Clifford+$T$ gate set on $n$ qubits with no ancillas.
\item[(b)] The entries of $U$ belong to the ring
  $\Z[\frac1{\sqrt{2}},i]$, and:
  \begin{itemize}
    \item $\det U=1$, if $n\geq 4$;
    \item $\det U\in\s{-1,1}$, if $n=3$;
    \item $\det U\in\s{i,-1,-i,1}$, if $n=2$;
    \item $\det U\in\s{\omega,i,\omega^3,-1,\omega^5,-i,\omega^7,1}$,
      if $n\leq 1$.
    \end{itemize}
  \end{enumerate}
\end{corollary}

\begin{proof}
  For (a) $\implies$ (b), it suffices to note that each of the generators of the Clifford+$T$ group,
  regarded as an operation on $n$ qubits, satisfies the conditions in (b). For (b) $\implies$ (a), let
  us define for convenience $d_0=d_1=\omega$, $d_2=i$, $d_3=-1$, and $d_n=1$ for $n\geq 4$. First
  note that for all $n$, the Clifford+$T$ group on $n$ qubits (without ancillas) contains an
  element $D_n$ whose determinant is $d_n$, namely $D_n=I$ for $n\geq 4$, $D_3=T\otimes I\otimes
  I$, $D_2=T\otimes I$, $D_1=T$, and $D_0=\omega$. Now consider some $U$ satisfying (b). By
  assumption, $\det U=d_n^m$ for some $m$. Let $U'=UD_n^{-m}$, then $\det U'=1$. By
  Lemma~\vref{lem-det1}, $U'$, and therefore $U$, is in the Clifford+$T$ group with no ancillas.
\end{proof}

\begin{remark}
  Note that the last condition in Corollary~\vref{cor-noancilla}, namely that $\det U$ is a power of
  $\omega$ for $n\leq 1$, is of course redundant, as this already follows from $\det
  U\in\Z[\frac1{\sqrt{2}},i]$ and $|\det U|=1$. We stated the condition for consistency with the
  case $n\geq 2$.
\end{remark}

\begin{remark}
  The situation of Theorem~\vref{thm-main} and Corollary~\vref{cor-noancilla} is analogous to the
  case of classical reversible circuits. It is well-known that the not-gate, controlled-not gate,
  and Toffoli gate generate all classical reversible functions on $n\leq 3$ bits. For $n\geq 4$
  bits, they generate exactly those reversible boolean functions that define an {\em even
  permutation} of their inputs (or equivalently, those that have determinant 1 when viewed in
  matrix form) {\cite{Musset97}}; the addition of a single ancilla suffices to recover all boolean
  functions.
\end{remark}

% subsection the_no_ancilla_case (end)

\subsection{Complexity} % (fold)
\label{sub:complexity}
The proof of Theorem~\vref{thm-main} immediately yields an algorithm, albeit not a very efficient
one, for synthesizing a Clifford+$T$ circuit with ancillas from a given operator $U$. We estimate
the size of the generated circuits.

We first estimate the number of (one- and two-level) operations generated by the matrix
decomposition of Lemma~\vref{lem-matrix-decomposition}. The row operation from Lemma~\vref{lem-row}
requires only a constant number of operations. Reducing a single $n$-dimensional column from
denominator exponent $k$ to $k-1$, as in the induction step of Lemma~\vref{lem-column}, requires
$O(n)$ operations; therefore, the number of operations required to reduce the column completely is
$O(nk)$.

Now consider applying Lemma~\ref{lem-matrix-decomposition} to an $n\times n$-matrix with least
denominator exponent $k$. Reducing the first column requires $O(nk)$ operations, but unfortunately,
it may {\em increase} the least denominator exponent of the rest of the matrix, in the worst case,
to $3k$. Namely, each row operation of Lemma~\ref{lem-row} potentially increases the denominator
exponent by $2$, and any given row may be subject to up to $k$ row operations, resulting in a
worst-case increase of its denominator exponent from $k$ to $3k$ during the reduction of the first
column. It follows that reducing the second column requires up to $O(3(n-1)k)$ operations, reducing
the third column requires up to $O(9(n-2)k)$ operations, and so on. Using the identity
$\sum_{j=0}^{n-1}3^j(n-j) = (3^{n+1}-2n-3)/4$, this results in a total of $O(3^nk)$ one- and
two-level operations for Lemma~\ref{lem-matrix-decomposition}.

In the context of Theorem~\vref{thm-main}, we are dealing with $n$ qubits, i.e., a $2^n\times
2^n$-operator, which therefore decomposes into $O(3^{2^n}k)$ two-level operations. Using one
ancilla, each two-level operation can be decomposed into $O(n)$ Clifford+$T$ gates, resulting in a
total gate count of $O(3^{2^n}\!nk)$ elementary Clifford+$T$ gates.

% subsection complexity (end)

% section exact_synthesis_of_multi_qubit_operators (end)
% chapter synthesis_of_quantum_operations (end)
