
\section{Suitability of the semantics} % (fold)
\label{sec:suitability_of_the_semantics}
The previous section presented two alternate views of semantics for quantum computation. That of
\cite{selinger04:qpl} which models a specific language designed to write quantum
programs, and \cite{abramsky04:catsemquantprot} which models finitary quantum mechanics directly.

Additionally, in section \ref{sec:biproduct}, given a biproduct dagger compact closed category, we
create a category of completely positive maps based on it. Starting with the category $\fdh$, now
construct \bcpm{\fdh}. Recalling $V$ from section \ref{sec:catsemanticsofqpl}, \bcpm{\fdh} is
equivalent to the subcategory of $V$ which has only the completely positive maps. Recall that $Q$,
the category used to interpret QPL further restricted the maps, requiring them to be trace
non-increasing. So, there are the following inclusions: $Q\subseteq \bcpm{\fdh} \subseteq V$. The
language QPL can therefore obviously be interpreted inside \bcpm{\fdh}.

Referring to section \ref{sec:puresemantics}, the interpretation of finitary quantum mechanics
presented there was intended to axiomatize \fdh. The maps for initialization, $base_{Q}:I\+I \to Q$
and measurement $meas:Q\to I\+I$ are given as isomorphisms. If we were to compose them
$Q\xrightarrow{meas}I\+I\xrightarrow{base_{Q}} Q$, this would be the identity function. However,
this is prevented in their interpretation by insisting any map $f$ occurring in an interpretation
that has $A\+B$ as its domain must be decomposable to $f_{1}\+f_{2}$. Therefore, it is not legal to
apply $base_{Q}$ after a measure.

However the axiomatization given in \cite{abramsky04:catsemquantprot} is applicable to \bcpm{\fdh}
with a few changes as noted in \cite{selinger05:dagger}. Instead of the map $base_{Q}$ being an
isomorphism, it needs to only be a retraction
\[
  \xymatrix{
    I\+I \ar[r]^{base_{Q}} \ar[dr]_{id} & Q \ar[d]^{meas}\\
    & I\+I
    }.
\]

\subsection{Applicability to quantum algorithms}\label{sec:applicability}
In this section, we present a few examples of using the semantics on specific quantum algorithms.
For simple initial example, consider ``coin flipping'', that is, initialize a \qbit, apply the
Hadamard transform and measure it. In QPL, this can be accomplished by the program:

\begin{Verbatim}[numbers=left,numbersep=6pt,firstnumber=0]

newqbit q:=0;
q *= H;
measure q
 |0> => {discard q;}
 |1> => {discard q;};
\end{Verbatim}

Using the interpretations of Table \ref{tab:interp}, the semantics and annotations at the
completion of each line are:
\begin{eqnarray*}
  0 & &\Gamma=1\\
  1 & q:\text{qbit} & \Gamma=\begin{pmatrix}1&0\\0&0\end{pmatrix}\\
  2 & q:\text{qbit} & \Gamma=\begin{pmatrix}.5&.5\\.5&.5\end{pmatrix}\\
  3 & q:\text{qbit} \+ q:\text{qbit} & \Gamma=\left( \begin{pmatrix}.5&0\\0&0\end{pmatrix},
    \begin{pmatrix}0&0\\0&.5\end{pmatrix}\right)\\
  4 & I \+ q:\text{qbit} & \Gamma=\left( .5, \begin{pmatrix}0&0\\0&.5\end{pmatrix}\right)\\
  5 & I\+I (=\text{bit}) & \Gamma=(.5,.5)
\end{eqnarray*}

QPL is able to encode the quantum teleportation algorithm, but it should be noted that the spatial
components (i.e, separating the entangled pair) of the algorithm are not expressible in the
language. Essentially, this is because the teleport algorithm to consists of four stages:
\begin{description}
  \item[Preparation] Creation of an entangled pair of \qubits $a$ and $b$ which are then physically
    separated
  \item[Measurement] Alice, the holder of $a$ applies transforms to $a$ and $n$ a new \qbit and
    measures them both.
  \item[Communication] Alice sends Bob, the holder of $b$ two \bits of information, based the the
    measurements of the previous step.
  \item[Application] Bob, depending on the \bits received in the previous stage conditionally
    applies transforms to $b$.
\end{description}

In a physical implementation, this would consist of three programs (one each for Preparation,
Measurement and Application) along with the actual communication. QPL has no features for actual
input, output or communication, so we must idealize this to a single program. In the program,
assume the \qbit $n$ to be transferred is available at stage $0$ and has the density matrix $T
=\begin{pmatrix}a&b\\c&d\end{pmatrix}$.

\begin{Verbatim}[numbers=left,numbersep=6pt,firstnumber=0]

newqbit b:=0;
newqbit a:=0;
a *= H;
a,b *= CNot;
permute a,b,n -> n,a,b;
n,a *= CNot;
n *= H;
permute n,a,b -> a,n,b;
measure a
      |0> => {discard a;}
      |1> => {discard a;
              b *= Not};
merge;
measure n
      |0> => {discard n;}
      |1> => {discard n;
              b * = Z};
merge;
\end{Verbatim}
In the interests of space, I will only show the first few stages. The number in the first column
matches the annotation to the line of the program it corresponds to, after execution of that line.
\begin{eqnarray*}
  0 & n:\text{qbit} & T\\
  1 & b,n:\text{qbit} & \begin{pmatrix}1&0\\0&0\end{pmatrix} \*T\\
  3 & a,b,n:\text{qbit} & \begin{pmatrix}.5&.5\\.5&.5\end{pmatrix} \*
    \begin{pmatrix}1&0\\0&0\end{pmatrix} \* T\\
  4 & a,b,n:\text{qbit} & \begin{pmatrix}.5&0&0&.5\\0&0&0&0\\0&0&0&0\\.5&0&0&.5\end{pmatrix} \* T\\
  6 & n,a,b:\text{qbit} &
    \begin{pmatrix}a/2&0&0&a/2&0&b/2&b/2&0\\0&0&0&0&0&0&0&0\\0&0&0&0&0&0&0&0\\
        a/2&0&0&a/2&0&b/2&b/2&0\\0&0&0&0&0&0&0&0\\c/2&0&0&c/2&0&d/2&d/2&0\\
          c/2&0&0&c/2&0&d/2&d/2&0\\0&0&0&0&0&0&0&0\end{pmatrix}\\
  & \vdots &
\end{eqnarray*}
\section{Relations to other semantics}
\subsection{Probabilistic semantics}
As presented in \cite{kozen-semanticsprobabilistic}, a probabilistic programming language may be
interpreted as linear operators between Banach spaces. Assume $X$ is some domain of values and $M$
is the collection of measurable sets associated with it.
%Denote by $(X^{n}, M^{(n)})$ the Cartesian product of $n$ copies of the measurable space
%$(X,M)$ The semantics for a program $P$ map distributions $\delta$ on
%$(X^{n}, M^{(n)})$ to the distribution $P(\delta)$ over $(X^{n}, M^{(n)})$.
These linear operators form a Banach space with the \emph{uniform norm} and pointwise addition and
scalar multiplication. In this interpretation of probabilistic programs, isotone operators $T$
(where $\mu \le \nu \implies T(\mu) \le T(\nu)$) allow the definition of an ordering of the space
of programs and will give a least fixed point semantics. \cite{kozen-semanticsprobabilistic} gives
denotations of five elements of probabilistic programming: \emph{Simple Assignment}; \emph{Random
Assignment}; \emph{Composition}; \emph{Conditional} and \emph{While Loop}.

As every Hilbert space is a Banach space (as the inner product generates a norm), it should be
possible to model Kozen's semantics in the category $Q$ used for QPL, or in $\bcpm{\fdh}$.
Specifically, one would have to ensure that the interpretation mapped \emph{simple assignment} to a
measurable function and would have to provide specific operators for the various allowed random
distributions in \emph{random assignment}. As an example, the ``coin flip'' program presented
\ref{sec:applicability} generates random bit values.

\subsection{Reversible semantics}
In \cite{gilescockett09}, we initiate the exploration of semantics of reversibility with the
introduction of discrete inverse categories, an inverse category with a tensor, which is a natural,
coassociative, cocommutative, Frobenius bifunctor with a natural ``duplication'' map $\Delta:A\to
A\*A$.

As seen in \cite{coeckeetal08:ortho,coecke08structures}, $\dagger$-Frobenius algebras in
$\dagger$-symmetric monoidal categories give rise to classical structures, and to quantum
structures (See sections \ref{sec:basesandfrob} and \ref{sec:quantumclassical} and definitions
\ref{def:quantumstructure} and \ref{def:classicalstructure}).

Conversley, in \cite{gilescockett09}, we show
\begin{theorem}
  Given \X is a symmetric monoidal category, define the category $CFrob(\X)$. $CFrob(\X)$ has
  objects the commutative Frobenius algebras $X$ where $X$ is an object of $\X$ and $X$ is a
  $k$-algebra for some field $k$. The maps of $CFrob(\X)$ are the multiplication and
  comultiplication preserving homomorphisms. Then, $CFrob(\X)$ forms a discrete inverse category.
\end{theorem}
