
\section{Basic quantum computation} % (fold)
\label{sec:basic_quantum_computation}

\subsection{Quantum bits} % (fold)
\label{sub:quantum_bits}

Quantum computation deals with operations on \qubit{}s. A \qubit{} is typically represented in the
literature on quantum computation as a complex linear combination of \ket{0} and \ket{1},
respectively identified with (1,0) and (0,1) in $\C^2$. Because of the identification of the basis
vectors, any \qubit{} can be identified with a non-zero vector in $\C^2$. In standard quantum
computation, the important piece of information in a \qubit{} is its direction rather than
amplitude. In other words, given $q=\alpha\ket{0}+\beta\ket{1}$ and
$q'=\alpha'\ket{0}+\beta'\ket{1}$ where $\alpha = \gamma\alpha'$ and $\beta = \gamma\beta'$, then
$q$ and $q'$ represent the same quantum state.

A \qubit{} that has either $\alpha$ or $\beta$ zero is said to be in a \emph{classical state}. Any
other combination of values is said to be a \emph{superposition}.

\Vref{sec:quantum_circuits} will introduce quantum circuits which act on \qubits{}. This section
will have some forward references to circuits to illustrate points introduced here.

% subsection quantum_bits (end)

\subsection{Quantum entanglement} % (fold)
\label{sub:quantum_entanglement}

Consider what happens when working with a pair of \qubit{}s, $p$ and $q$. This can be considered as
the a vector in $\C^4$ and written as
\begin{equation}
  \alpha_{00}\ket{00}+\alpha_{01}\ket{01}+\alpha_{10}\ket{10}+
  \alpha_{11}\ket{11}.\label{eq:appTwoqubits}
\end{equation}
In the case where $p$ and $q$ are two independent \qubit{}s, with $p=\alpha\ket{0}+\beta\ket{1}$
and $q=\gamma\ket{0}+\delta\ket{1}$,
\begin{equation}
  p \otimes q = \alpha\gamma\ket{00}+\alpha\delta\ket{01}+\beta\gamma\ket{10}+\beta\delta\ket{11}
\end{equation}
where $p \otimes q$ is the standard tensor product of $p$ and $q$ regarded as vectors. There are
states of two \qubits{} that cannot be written as a tensor product. As an example, the state
\begin{equation}
  \frac{1}{\sqrt{2}}\ket{00} + \frac{1}{\sqrt{2}}\ket{11}\label{eq:appEntangledqubits}
\end{equation}
is not a tensor product of two distinct \qubit{}s. In this case the two \qubit{}s are said to be
\emph{entangled}.

% subsection quantum_entanglement (end)

\subsection{Quantum gates} % (fold)
\label{sub:quantum_gates}

\emph{Quantum gates} operate on \qubit{}s. These gates are conceptually similar to logic gates in
the classical world. In the classical world the only non-trivial single \bit{} gate is the Not gate
which sends $0$ to $1$ and $1$ to $0$. However, there are infinitely many non-trivial quantum gates.

An $n{-}\qubit{}$ quantum gate is represented by a $2^n \times 2^n$ matrix. A necessary and
sufficient condition for such a matrix to be a quantum gate is that it is \emph{unitary}.

The entanglement of two \qubits{}, $p$ and $q$, is accomplished by applying a \Had{} transformation
to $p$ followed by a \nottr{} applied to $q$ controlled by $p$. The circuit in
\vref{qc:appEntangle} shows how to entangle two \qubit{}s that start with an initial state of
$\ket{00}$. See %todo {reference for lqpl teleportation} %\vref{fig:stackTeleportation} for how
this can be done in \lqpl.

A list of some common gates, together with their usual quantum circuit representation is given in
the next section in \vref{tab:qgatesAndRep}.
% subsection quantum_gates (end)

\subsection{Measurement} % (fold)
\label{sub:measuerment}

The other allowed operation on a \qubit{} or group of \qubit{}s is measurement. When a \qubit{} is
measured it assumes only one of two possible values, either \ket{0} or \ket{1}. Given
\begin{equation} q=\alpha\ket{0}+\beta\ket{1}\label{eq:appSinglequbit} \end{equation} where
$|\alpha|^2+|\beta|^2 = 1$, then measuring $q$ will result in \ket{0} with probability $|\alpha|^2$
and \ket{1} with probability $|\beta|^2$. Once a \qubit{} is measured, re-measuring will always
produce the same value.

In multi-\qubit{} systems the order of measurement does not matter. If $p$ and $q$ are as in
\vref{eq:appTwoqubits}, let us suppose measuring $p$ gives \ket{0}. The measure will result in that
value with probability $|\alpha_{00}|^2 + |\alpha_{01}|^2$, after which the system collapses to the
state:
\begin{equation}
  \alpha_{00}\ket{00} + \alpha_{01}\ket{01}\label{eq:appCollapsedFirstBit}
\end{equation}

Measuring the second \qubit{}, $q$, will give \ket{0} with probability $|\alpha_{00}|^2$ or \ket{1}
with probability $|\alpha_{01}|^2$.

Conversely, if $q$ was measured first and gave us \ket{0} (with a probability of $|\alpha_{00}|^2 +
|\alpha_{10}|^2$) and then $p$ was measured, $p$ will give us \ket{0} with probability
$|\alpha_{00}|^2$ or \ket{1} with probability $|\alpha_{10}|^2$.

Thus, when measuring both $p$ and $q$, the probability of getting \ket{0} from both measures is
$|\alpha_{00}|^2$, regardless of which \qubit{} is measured first.

Considering states such as in \vref{eq:appEntangledqubits}, measuring either \qubit{} would actually
force the other \qubit{} to the same value. This type of entanglement is used in many quantum
algorithms such as quantum teleportation.
% subsection measurement (end)

\subsection{Mixed states} % (fold)
\label{sub:mixed_states}

The notion of \emph{mixed states} refers to an outside observer's knowledge of the state of a
quantum system. Consider a $1$ \qubit{} system
\begin{equation}
  \nu = \alpha\ket{0} + \beta\ket{1}.\label{eq:appOnequbitSystem}
\end{equation}

If $\nu$ is measured but the results of the measurement are not examined, the state of the system
is either \ket{0} or \ket{1} and is no longer in a superposition. This type of state is written as:
\begin{equation}
  \nu = |\alpha|^2\{\ket{0}\} + |\beta|^2\{\ket{1}\}.\label{eq:appMixedState}
\end{equation}

An external (to the state) observer knows that the state of $\nu$ is as expressed in
\vref{eq:appMixedState}. Since the results of the measurement were not examined, the exact state
($0$ or $1$) is unknown. Instead, a probability is assigned as expressed in the equation. Thus, if
the \qubit{} $\nu$ is measured and the results are not examined, $\nu$ can be treated as a
probabilistic \bit{} rather than a \qubit.

% subsection mixed_states (end)

\subsection{Density matrix notation} % (fold)
\label{sub:density_matrix_notation}

The state of any quantum system of \qubit{}s may be represented via a \emph{density matrix}. In
this notation, given a \qubit{} $\nu$, the coefficients of \ket{0} and \ket{1} form a column vector
$u$. Then the density matrix corresponding to $\nu$ is $uu^{*}$. If $\nu=\alpha\ket{0} +
\beta\ket{1}$,
\begin{equation}
  \begin{singlespace}
    \nu =
    \begin{pmatrix}
      \alpha \\
      \beta
    \end{pmatrix}
    \begin{pmatrix}
      \overline{\alpha} & \overline{\beta}
    \end{pmatrix}
    =
    \begin{pmatrix}
      \alpha\overline{\alpha} & \alpha\overline{\beta}\\
      \beta\overline{\alpha} & \beta\overline{\beta}
    \end{pmatrix}.
  \end{singlespace}
\end{equation}

When working with mixed states the density matrix of each component of the mixed state is added.
For example, the mixed state shown in \vref{eq:appMixedState} would be represented by the density
matrix

\begin{equation}
  \begin{singlespace}
    |\alpha|^2
    \begin{pmatrix}
      1&0 \\
      0&0
    \end{pmatrix}
    +
    |\beta|^2
    \begin{pmatrix}
    0&0 \\
    0&1
    \end{pmatrix}
    =
    \begin{pmatrix}
      |\alpha|^2&0 \\
      0&|\beta|^2
    \end{pmatrix}.\label{eq:appDensityOfMixedState}
  \end{singlespace}
\end{equation}

Note that since the density matrix of mixed states is a linear combination of other density
matrices, it is possible to have two different mixed states represented by the same density matrix.

The advantage of this notation is that it becomes much more compact for mixed state systems.
Additionally, scaling issues are handled by insisting the density matrix has a trace = 1. During a
general quantum computation, as we shall see, the trace can actually fall below 1 indicating that
the computation is not everywhere total.

% subsection density_matrix_notation (end)

\subsection{Gates and density matrices} % (fold)
\label{sub:gates_and_density_matrices}

When considering a \qubit{} $q$ as a column vector and a unitary transform $T$ as a matrix, the
result of applying the transform $T$ to $q$ is the new vector $T q$. The density matrix of the
original \qubit{} is given by $q\, q^{*}$, while the density matrix of the transformed \qubit{} is
$(T q) ( T q)^{*}$, which equals $ T (q q^{*}) T^{*}$. Thus, when a \qubit{} $q$ is represented by
a density matrix $A$, the formula for applying the transform $T$ to $q$ is $T A T^{*}$.
% subsection gates_and_density_matrices (end)


% section basic_quantum_computation (end)





\section{Quantum circuits} % (fold)
\label{sec:quantum_circuits}

\subsection{Contents of quantum circuits} % (fold)
\label{sub:contents_of_quantum_circuits}


Currently a majority of quantum algorithms are defined and documented using \emph{quantum
circuits}. These are wire-type diagrams with a series of \qubit{}s input on the left of the diagram
and output on the right. Various graphical elements are used to describe quantum gates,
measurement, control and classical \bits.

\subsubsection{Gates and \qubit{}s} % (fold)
\label{ssub:gates_and_qubits}

The simplest circuit is a single wire with no action:
\[
  \Qcircuit @C=1em @R=.7em {
    & \qw& \qw_x&\qw\\
  }
\]
The next simplest circuit is one \qubit{} and one gate. The \qubit{} is represented by a single
wire, while the gate is represented by a box with its name, $G$, inside it. This is shown in the
circuit in \vref{qc:simpleqcircuit}. In general, the name of the wire which is input to the gate
$G$ may be different from the name of $G$'s output wire. Circuit diagrams may also contain constant
components as input to gates as in the circuit in \vref{qc:cnotcircuit}.

\begin{figure}[htbp]
  \centerline{%
    \Qcircuit @C=1em @R=.7em {
      &\qw_x& \gate{G}  & \qw& \qw_y\\
    }
  }
  \caption{Simple single gate circuit}\label{qc:simpleqcircuit}
\end{figure}


\begin{figure}[htbp]
  \centerline{%
    \Qcircuit @C=1em @R=.7em {
      &\qw_x & \gate{H} &\qw_{\quad x'} & \ctrl{1} & \qw_{\ x'}\\
      &\qw_y  & \qw & \qw & \targ & \qw_{\ y'}\\
    }
  }
  \caption{Entangling two \qubit{}s.}
  \label{qc:appEntangle}
\end{figure}


\begin{figure}[htbp]
  \centerline{%
    \Qcircuit @C=1em @R=.7em {
      \lstick{\ket{1}} & \targ & \rstick{\ket{0}} \qw \\
      \lstick{\ket{1}}  & \ctrl{-1} & \rstick{\ket{1}} \qw \\
    }
  }
  \caption{\cnot{} of $\ket{1}$ and $\ket{1}$}
  \label{qc:cnotcircuit}
\end{figure}

Future diagrams will drop the wire labels except when they are important to the concept under
discussion.

Controlled gates, where the gate action depends upon another \qubit, are shown by attaching a wire
between the wire of the control \qubit{} and the controlled gate. The circuit in
\vref{qc:appEntangle} shows two \qubit{}s, where a \Had{} is applied to the top \qubit{}, followed
by a \cnot{} applied to the second \qubit{}. In circuits, the control \qubit{} is on the vertical
wire with the solid dot. This is then connected via a horizontal wire to the gate being controlled.

A list of common gates, their circuits and corresponding matrices is given in
\vref{tab:qgatesAndRep}.

\begin{table}
  \centerline{
    \begin{tabular}{|l|c|l|}
      \hline
      \text{\textbf{Gate}}&\text{\textbf{Circuit}}&\text{\textbf{Matrix}}\\
      \hline
      & & \\
      Not ($X$) &
      $\Qcircuit @C=1em @R=.7em {
      & \targ & \qw
      }$ & $
      \begin{bmatrix}
        0&1\\
        1&0
      \end{bmatrix}$ \\
      & & \\\hline
      & &  \\
      $Z$ &
      $\Qcircuit @C=1em @R=.7em {
        & \gate{Z} & \qw
      }$ & $
      \begin{bmatrix}
        1&0\\
        0&-1
      \end{bmatrix}$ \\
      & & \\\hline
      & &  \\
      Hadamard &
      $\Qcircuit @C=1em @R=.7em {
        & \gate{H} & \qw
      }$ & $
      \frac{1}{\sqrt{2}}
      \begin{bmatrix}
        1&1\\
        1&-1
      \end{bmatrix}$ \\ & & \\\hline
      & & \\
      Swap &
      $\Qcircuit @C=1em @R=.7em {
        & \qswap & \qw \\
        & \qswap \qwx & \qw
      }$ & $
      \begin{bmatrix}
        1&0&0&0\\
        0&0&1&0\\
        0&1&0&0\\
        0&0&0&1
      \end{bmatrix}$ \\
      & & \\\hline
      & & \\
      Controlled-Not &
      $\Qcircuit @C=1em @R=.7em {
        & \ctrl{1} & \qw \\
        & \targ & \qw
      }$ & $
      \begin{bmatrix}
        1&0&0&0\\
        0&1&0&0\\
        0&0&0&1\\
        0&0&1&0
      \end{bmatrix}$ \\
      & & \\\hline
      & &  \\
      Toffoli &
      $\Qcircuit @C=1em @R=.7em {
        & \ctrl{1} & \qw \\
        & \ctrl{1} & \qw \\
        & \targ & \qw
      }$ & $
      \begin{bmatrix}
        1&0&0&0&0&0&0&0\\
        0&1&0&0&0&0&0&0\\
        0&0&1&0&0&0&0&0\\
        0&0&0&1&0&0&0&0\\
        0&0&0&0&1&0&0&0\\
        0&0&0&0&0&1&0&0\\
        0&0&0&0&0&0&0&1\\
        0&0&0&0&0&0&1&0
      \end{bmatrix}$ \\
      & & \\\hline
    \end{tabular}
  } % end centerline
  \caption{Gates, circuit notation and matrices}\label{tab:qgatesAndRep}
\end{table}

% subsubsection gates_and_qubits (end)

\subsubsection{Measurement} % (fold)
\label{ssub:measurement}

Measurement is used to transform the quantum data to classical data so that it may be then used in
classical computing (e.g. for output). The act of measurement is placed at the last part of the
quantum algorithm in many circuit diagrams and is sometimes just implicitly considered to be there.

While there are multiple notations used for measurement in quantum circuit diagrams, this thesis
will standardize on the \emph{D-box} style of measurement as shown in \vref{qc:dboxmeasure}.

%\begin{figure}[htbp]
%\centerline{%
%\Qcircuit @C=1em @R=.7em {
% & \gate{H} &\ctrl{1} & \meter & \cw &\rstick{\text{meter}}\\
% & \ctrl{1} & \targ & \measure{\mbox{bit}}& \cw  &\rstick{\text{oval}}\\
% & \gate{V} &\qw & \measuretab{M}  & &\rstick{\text{tab}}\\
% & \gate{W} & \qw & \measureD{\chi}& &\rstick{\text{D-box}}
% }}
%\caption{Examples of different measure notations}
%\label{qc:manymeasures}
%\end{figure}

\begin{figure}[htbp]
  \centerline{%
    \Qcircuit @C=1em @R=.7em {
      & \gate{W} & \qw & \measureD{\chi}& \cw & &\rstick{\text{D-box}}
    }
  }
  \caption{Measure notation in quantum circuits}
  \label{qc:dboxmeasure}
\end{figure}

A measurement may have a double line leaving it, signifying a \bit, or nothing, signifying a
destructive measurement.

Operations affecting multiple \qubit{}s at the same time are shown by extending the gate or measure
box to encompass all desired wires. In the circuit in \vref{qc:multibox}, the gate $U$ applies to
all of the first three \qubits{} and the measurement applies to the first two \qubits{}.
\begin{figure}[htbp]
  \centerline{%
    \Qcircuit @C=1em @R=.7em {
      & \multigate{2}{U} & \multimeasureD{1}{\mbox{bits}} & \cw \\
      & \ghost{U} &  \ghost{\mbox{bits}} &\cw \\
      & \ghost{U} & \qw &\qw
    }
  }
  \caption{Examples of multi-\qubit{} gates and measures}
  \label{qc:multibox}
\end{figure}

% subsubsection measurement (end)

\subsubsection{$0$-control and control by \protect{\bits}} % (fold)
\label{ssub:_0_control_and_control_by_bits}


The examples above have only shown control based upon a \qubit{} being \ket{1}. Circuits also allow
control on a \qubit{} being \ket{0} and upon classical values. These forms of control are illustrated
by the circuit
in \vref{qc:zctrlAndClassicalControl} with four \qubit{}s $(r_1, r_2, p$ and $q)$.

At $g_1$, a \Had{} is $1-$controlled by $r_2$ and is applied to each of $r_1$ and $p$. This is
followed in column $g_2$ with the \nottr{} transform applied to $r_2$ being $0-$controlled by $r_1$
. In the same column, a $Z$ gate is $0-$controlled by $q$ and applied to $p$. $p$ and $q$ are then
measured in column $g_3$ and their corresponding classical values are used for control in $g_4$. In
$g_4$, the $U_R$ gate is applied to both $r_1$ and $r_2$, but only when the measure result of $p$
is $0$ and the measure result of $q$ is $1$.

\begin{figure}[htbp]
  \centerline{%
    \Qcircuit @C=1em @R=.7em {
      &\qw_{r_1} & \gate{H} &\ctrlo{1} & \qw & \qw & \multigate{1}{U_R} & \qw \\
      &\qw_{r_2} & \ctrl{-1} \qwx[1] &\targ & \qw  & \qw & \ghost{U_R} & \qw \\
      &\qw_{p} & \gate{H} & \gate{Z} &\qw & \multimeasureD{1}{M_{p,q}} & \controlo \cw \cwx \\
      &\qw_{q} & \qw & \ctrlo{-1} &\qw & \ghost{M_{p,q}} & \control \cwx \cw\\
      &&\lstick{g_1}&\lstick{g_2}&&\lstick{g_3}&\lstick{g_4}
    }
  }
  \caption{Other forms of control for gates}
  \label{qc:zctrlAndClassicalControl}
\end{figure}


% subsubsection _0_control_and_control_by_bits (end)

\subsubsection{Multi-\protect{\qubit} lines} % (fold)
\label{ssub:multi_qubit_lines}


It is common to represent multiple \qubit{}s on one line. A gate applied to a multi-\qubit{} line
must be a tensor product of gates of the correct dimensions. The circuit in
\vref{qc:manyqubitsOneLine} shows $n$ \qubit{}s on one line with the \Had{} gate (tensored with
itself $n$ times) applied to all of them. That is followed by a unary gate $U_R$ tensored with
$I^{\otimes (n-2)}$ and tensored with itself again. This will have the effect of applying an $U_R$
gate to the first and last \qubits{} on the line.

\begin{figure}[htbp]
  \centerline{
    \Qcircuit @C=1em @R=.7em {
    & {/^{{}^n}} \qw  & \gate{H^{\otimes n}} & \gate{U_R\otimes I^{\otimes (n-2)} \otimes U_R} & \qw
    }
  }
  \caption{$n$ \qubit{}s on one line}
  \label{qc:manyqubitsOneLine}
\end{figure}

% subsubsection multi_qubit_lines (end)

\subsubsection{Other common circuit symbols} % (fold)
\label{ssub:other_common_circuit_symbols}

Two other symbols that are regularly used are the swap and controlled-\Z, shown in the circuit in
\vref{qc:swapAndCtrlZ}. Note that swap is just shorthand for a series of three controlled-\nottr{}
gates with the control \qubit{} changing. This can also be seen directly by multiplying the
matrices for the controlled-\nottr{} gates as shown in \vref{eq:cnotcnotcnot}.

\begin{figure}[htbp]
  \centerline{
    \begin{tabular}{c}
      \Qcircuit @C=1em @R=.7em {
        & \qswap & \qw &  \raisebox{-3.5ex}{=} & & \ctrl{1} & \targ &\ctrl{1}  &\qw \\
        & \qswap \qwx &\qw &  & & \targ &\ctrl{-1} & \targ &\qw \\
      } \\
      \\
      \Qcircuit @C=1em @R=.7em {
        & \ctrl{1} &\qw  & \raisebox{-3.5ex}{=} & & \Box \qw &\qw \\
        & \gate{Z} & \qw & & & \Box \qw \qwx & \qw
      }
    \end{tabular}
  }
  \caption{Swap and controlled-Z}
  \label{qc:swapAndCtrlZ}
\end{figure}

\begin{equation}
  \begin{singlespace}
    \begin{bmatrix}
      1&0&0&0\\
      0&0&1&0\\
      0&1&0&0\\
      0&0&0&1
    \end{bmatrix} =
    \begin{bmatrix}
      1&0&0&0\\
      0&1&0&0\\
      0&0&0&1\\
      0&0&1&0
    \end{bmatrix}
    \begin{bmatrix}
      1&0&0&0\\
      0&0&0&1\\
      0&0&1&0\\
      0&1&0&0
    \end{bmatrix}
    \begin{bmatrix}
      1&0&0&0\\
      0&1&0&0\\
      0&0&0&1\\
      0&0&1&0
    \end{bmatrix}.\label{eq:cnotcnotcnot}
  \end{singlespace}
\end{equation}
% subsubsection other_common_circuit_symbols (end)

% subsection contents_of_quantum_circuits (end)

\subsection{Syntax of quantum circuits} % (fold)
\label{sub:syntax_of_quantum_circuits}


Quantum circuits were originally introduced by David Deutsch in \cite{deutsch89:qciruits}. He
extended the idea of standard classical based gate diagrams to encompass the quantum cases. In his
paper, he introduced the concepts of quantum gates, sources of \bit{}s, sinks and universal gates.
One interesting point of the original definition is that it \emph{does} allow loops. Currently, the
general practice is not to allow loops of \qubit{}s. The commonly used elements of a circuit are
summarized in \vref{tab:qcdSyntax}.

\begin{table}
  \setlength\extrarowheight{4pt}
  \begin{tabular}{|>{\raggedright\hspace{0pt}}p{.9in}|p{3in}|>{$}c<{$}|}
    \hline
    \textbf{Desired element} & \textbf{Element in a quantum circuit diagram.} &
    \mbox{\textbf{Example}}\\
    \hline \hline
    \qubit{} & A single horizontal line. &
    \Qcircuit @C=1em @R=.7em {
      & \qw
    } \\\hline
    Classical \bit{} & A double horizontal line.
    &\Qcircuit @C=1em @R=.7em {
      & \cw
    } \\\hline
    Single-\qubit{} gates & A box with the gate name ($G$) inside it, one wire attached on its left
    and one wire attached on the right. &
    \Qcircuit @C=1em @R=.7em {
      & \gate{G} & \qw
    }  \\[4pt]\hline
    Multi-\qubit{} gates & A box with the gate name ($R$) inside it, $n$ wires on the left side and
    the same number of wires on the right. &
    \Qcircuit @C=1em @R=.7em {
      & \multigate{1}{R} & \qw \\
      & \ghost{R} & \qw
    }  \\[4pt]\hline
    Controlled \qubit{} gates & A box with the gate name ($H$, $W$) inside, with a solid
    (1-control) or open (0-control) dot on the control wire with a vertical wire between the dot
    and the second gate.
    &\Qcircuit @C=1em @R=.7em {
      & \ctrl{1} & \gate{W} & \qw \\
      & \gate{H} & \ctrlo{-1} &\qw
    }  \\[4pt]\hline
    Controlled-Not gates & A \emph{target} $\oplus$, with a solid (1-control) or open (0-control)
    dot on the control wire with a vertical wire between the dot and the gate.
    &\Qcircuit @C=1em @R=.7em {
      & \ctrl{1} & \qw \\
      & \targ &\qw
    }  \\[4pt]\hline
    Measurement & A \emph{D} shaped node with optional names or comments inside. One to $n$
    single wires are attached on the left (\qubit{}s coming in) and $0$ to $m$ classical \bit{}
    wires on the right where $m\le n$. Classical \bits{} may be dropped as desired.
    &\Qcircuit @C=1em @R=.7em {
      & \ctrl{1} & \multimeasureD{1}{q,r} & \cw \\
      & \gate{H} & \ghost{q,r}
    }  \\[4pt]\hline
    Classical control & Control bullets are attached to horizontal classical wires, with vertical
    classical wires attached to the controlled gate.
    &\Qcircuit @C=1em @R=.7em {
      &  \measureD{r} & \control \cw \cwx[1] \\
      & \qw & \gate{X} & \qw
    }  \\[4pt]\hline
    Multiple \qubit{}s & Annotate the line with the number of \qubit{}s and use tensors on gates.
    &\Qcircuit @C=1em @R=.7em {
      & {/^{{}^n}} \qw  & \gate{H^{\otimes n}} & \qw
    }  \\[4pt]\hline
  \end{tabular}
  \caption{Syntactic elements of quantum circuit diagrams}\label{tab:qcdSyntax}
\end{table}

A valid quantum circuit must follow certain restrictions. As physics requires \qubits{} must not be
duplicated, circuits must enforce this rule. Therefore, three restrictions in circuits are the
\emph{no fan-out}, \emph{no fan-in} and \emph{no loops} rules. These conditions are a way to
express the \emph{linearity} of quantum algorithms. Variables (wires) may not be duplicated, may
not be destroyed without a specific operation and may not be amalgamated.
% subsection syntax_of_quantum_circuits (end)

\subsection{Examples of quantum circuits} % (fold)
\label{sub:examples_of_quantum_circuits}


This section will present three quantum algorithms and the associated circuits. Each of these
circuits presented may be found in \cite{neilsen2000:QuantumComputationAndInfo}.

First, \emph{quantum teleportation}, an algorithm which sends a quantum bit across a distance via
the exchange of two classical bits. This is followed by the \emph{Deutsch-Jozsa algorithm}, which
provides information about the global nature of a function with less work than a classical
deterministic algorithm can. The third example is circuits for the \emph{quantum Fourier
transformation} and its inverse.

\subsubsection{Quantum teleportation} % (fold)
\label{ssub:quantum_teleportation}

The standard presentation of this algorithm involves two participants $A$ and $B$. (Henceforth
known as Alice and Bob). Alice and Bob first initialize two \qubit{}s to \ket{00}, then place them
into what is known as an \emph{EPR} (for Einstein, Podolsky and Rosen) state. This is accomplished
by first applying the \Had{} gate to Alice's \qubit{}, followed by a \cnot{} to the pair of
\qubits{} controlled by Alice's \qubit{}.

Then, Bob travels somewhere distant from Alice, taking his \qubit{} with him\footnote{Notice that
all other physical constraints are ignored in this algorithm. There is no concern about how one
separates the \qubit{}s, transports the \qubit{} or potential decoherence of the \qubit{}s.}.

At this point, Alice receives a \qubit{}, $\nu$, in an unknown state and has to pass $\nu$ on to
Bob. She then uses $\nu$ as the control and applies a {\cnot{}} transform to this new pair. Alice
then applies a \Had{} transform applied to $\nu$.

Alice now measures the two \qubit{}s and sends the resulting two \bits{} of classical information
to Bob.

Bob then examines the two \bits{} that he receives from Alice. If the \bit{} resulting from
measuring Alice's original bit is $1$, he applies the \nottr{} (also referred to as $X$) gate
($\begin{singlespace}=\begin{bmatrix}0&1\\1&0\end{bmatrix}\end{singlespace}$) to his \qubit{}. If
the measurement result of $\nu$ is one, he applies the $Z$ gate
($\begin{singlespace}=\begin{bmatrix}1&0\\0&-1\end{bmatrix}\end{singlespace}$). Bob's \qubit{} is
now in the same state as the \qubit{} Alice wanted to send. The circuit for this is shown in
\vref{qc:quantumTeleportation}.

\begin{figure}[htbp]
  \centerline{%
    \Qcircuit @C=1em @R=.7em {
      \lstick{\ket{\nu}} & \ctrl{1} & \gate{H} & \measureD{M_1} & \cw &  \control \cw  \cwx[2] \\
      \lstick{\mbox{A}}  & \targ & \qw & \measureD{M_2} & \control \cw  \cwx[1] \\
      \lstick{\mbox{B}} & \qw &\qw & \qw & \gate{X} & \gate{Z} & \qw & \rstick{\ket{\nu}}\\
      & \rstick{\ket{s_1}}& \rstick{\ket{s_2}} & & \lstick{\ket{s_3}} & &
      \lstick{\ket{s_4}}
    }
  }
  \caption{Quantum teleportation}
  \label{qc:quantumTeleportation}
\end{figure}
For comparison , see % todo lqpl program for \vref{fig:stackTeleportation}
showing how this would be implemented in
\lqpl{}.

% subsubsection quantum_teleportation (end)

\subsubsection{Deutsch-Jozsa algorithm} % (fold)
\label{ssub:deutsch_jozsa_algorithm}

The Deutsch-Jozsa algorithm describes a way of determining whether a function $f$\footnote{The
obvious pre-condition for the Deutsch-Jozsa algorithm is that the function $f$ is \emph{either}
balanced or constant and not some general function. The results are not well-defined if $f$ does
not fit into one of the two possible categories.} is \emph{constant} (i.e. always $0$ or $1$) or
\emph{balanced} (i.e. produces an equal number of $0$ or $1$ results) based on applying it to one
quantum bit. The function takes $n$ \bit{}s as input and produces a single \bit.

$f$ is assumed to be an expensive function, therefore, a desired effect is to evaluate $f$ as few
times as possible before determining if $f$ is balanced or constant. The worst case scenario when
evaluating $f$ classically is that determining the result requires $2^{n-1} + 1$ invocations of the
function. The best possible case is $2$ invocations, which occurs when $f$ is balanced and the
first two inputs chosen produce different results.

The quantum circuit requires only one application of the function to $n+1$ \qubits{} which have
been suitably prepared to make the decision.

The algorithm relies on being able to construct an $n+1$ order unitary operator based upon $f$. In
general, a unitary operator like this may be constructed by mapping the state $\ket{a,b}$ to
$\ket{a,b\oplus f(a)}$ where $\oplus$ is the exclusive-or operator and $a$ is $n$ \bit{} values. If
we name this operator $U_f$, the circuit in \vref{qc:djAlgorithm} will solve the problem with just
one application. See the appendix, %todo lqpl program {vref{appsubsec:djalgorithm}}
for how this would be done in \lqpl.
\begin{figure}[htbp]
  \centerline{%
    \Qcircuit @C=1em @R=.7em {
      \lstick{\ket{0}} & {/^{{}^n}} \qw & \gate{H^{\otimes n}} & \multigate{1}{U_f} & \gate{H^{\otimes n}}& \qw &\multimeasureD{1}{DJ} \\
      \lstick{\ket{1}} & \qw  & \gate{H} & \ghost{U_f} & \qw & \qw&\ghost{DJ}
    }
  }
  \caption{Circuit for the Deutsch-Jozsa algorithm}
  \label{qc:djAlgorithm}
\end{figure}

The idea of quantum parallelism is what makes this and many other quantum algorithms work. The
initial state of the system is set to $\ket{0^{\otimes n} \otimes 1}$ after which the \Had{} gate
is applied to all of the \qubit{s}. This places the input \qubit{}s into a superposition of all
possible input values and the answer \qubit{} is a superposition of 0 and 1. At this point, the
unitary transformation $U_f$ is applied to the \qubit{}s. Then the \Had{} transform is applied
again to the input \qubit{}s.

To complete the algorithm, measure \emph{all} the \qubit{}s. It can be shown that if $f$ is
constant, the input \qubit{}s will all measure to 0, while if it is balanced, at least one of those
\qubit{}s will be 1.

% subsubsection deutsch_jozsa_algorithm (end)

\subsubsection{Quantum Fourier transform} % (fold)
\label{ssub:quantum_fourier_transform}

The circuits for the quantum Fourier transformation and its inverse are in \vref{qc:qft} and
\vref{qc:inverseqft} respectively. These transforms are used extensively in many quantum algorithms,
including Shor's factoring algorithm.

%For comparison, we first review the algorithm for the classical
fast Fourier transform. (FFT) \marginnote{Is this a good idea or too much
digression?}

The FFT allows one to compute the \emph{Discrete Fourier Transform} -- 
$\mbox{DFT}_\omega$, a linear map from $\C^n \rightarrow \C^n$
which maps a polynomial $f$ to its evaluation at powers of $\omega$ which 
is a primitive $n^{th}$ root of unity. ($f$ is identified with the 
$n-$tuple of its coefficients for the purposes of the map). The algorithm is
presented in \vref{fig:classicalfft}.

\begin{figure}
\begin{description}
\item{Calling:} $r_k = \mbox{DFT}_\omega(n, [\omega, \ldots, \omega^{n-1}], [f_0, f_1, \ldots, f_{n-1}])$
\item{Inputs:} \begin{itemize}
\item{} The powers of $\omega$, where $\omega$ is a primitive  $n^{th}$ root of unity.
\item{} The value of $n$.
\item{} The coefficients of $f=f_0+f_1x+f_2x^2+\cdots+f_{n-1}x^{n-1}$.
\end{itemize}
\item{Output: } $(f(1), f(\omega), \ldots , f(\omega^{n-1})$.
\item{Method: } \begin{enumerate}
\item{} if $n=1$, return $f_0$. (Base of recursion).
\item{} Set $g=\sum_{0\le i \le n/2}(f_i + f_{i+n/2})x^i$
\item{} Recursively call: $gr_k = \mbox{DFT}_\omega(n/2, [\omega^2, \ldots, (\omega^2)^{n-1/2}], [g_0, \ldots, g_{n/2-1}])$
\item{} Set $h=\sum_{0\le i \le n/2}(f_i - f_{i+n/2})\omega^i x^i$
\item{} Recursively call: $hr_k = \mbox{DFT}_\omega(n/2, [\omega^2, \ldots, (\omega^2)^{n-1/2}], [h_0, \ldots, h_{n/2-1}])$
\item{} return $(gr_0, hr_0, gr_1, hr_1, \ldots, gr_{n/2}, hr_{n/2})$
\end{enumerate}
\end{description}
\caption{Algorithm for classical FFT.}\label{fig:classicalfft}
\end{figure}

The FFT algorithm can be used to reduce the time of polynomial multiplication
from \BigO{n^2} to \BigO{n\log n}. 


The quantum Fourier transform is definable on an arbitrary number of \qubits. This is typically
presented by eliding the $3^{\text{rd}}$ to the $n-3^{\text{rd}}$ lines and interior processing.
The \lqpl{} code for the quantum Fourier transform is in the appendix,
%todo lqpl for quantum fourier transform {vref{fig:qft}}.
In this circuit, the parametrized transform $R_n$ is the rotation transform, given by:
\[
  R_n=
  \begin{singlespace}
    \begin{bmatrix}1&0\\
      0&e^{\frac{2\pi i}{2^n}}
    \end{bmatrix}
  \end{singlespace}
\]
\begin{figure}[htbp]
  \centerline{%
    \Qcircuit @C=1em @R=.7em {
      &\qw & \gate{H} & \gate{R_2} & \cdots& &\gate{R_{n-1}}&\gate{R_{n}}&\qw&\qw&\qw&\qw&\qw&\cdots& &\qw&\qw&\qw\\
      &\qw & \qw & \ctrl{-1} & \cdots& &\qw&\qw&\gate{H} & \cdots& &\gate{R_{n-2}}&\gate{R_{n-1}}&\cdots& &\qw&\qw&\qw\\
      &\vdots &  & \vdots & & & & & &\vdots & & & & & & & &\\
      &\qw & \qw & \qw & \qw&\qw&\ctrl{-3}&\qw&\qw & \cdots& &\ctrl{-2}&\qw&\cdots& &\gate{H}&\gate{R_2}&\qw\\
      &\qw & \qw & \qw & \qw&\qw&\qw&\ctrl{-4}&\qw & \cdots& &\qw&\ctrl{-3}&\cdots& &\qw&\ctrl{-1}&\gate{H}
    }
  }
  \caption{Circuit for the quantum Fourier transform}
  \label{qc:qft}
\end{figure}


The inverse of a circuit is determined by traversing the circuit from right to left. This process
changes the original quantum Fourier circuit to its inverse as shown in \vref{qc:inverseqft}. The
\lqpl{} code for the inverse quantum Fourier transform is in the appendix,
%todo lqpl code for {vrefrange{fig:inverseqft}{fig:inverserotate}}.
\begin{figure}[htbp]
  \centerline{%
    \Qcircuit @C=1em @R=.7em {
      &\qw & \qw & \qw & \cdots& &\qw&\qw&\qw&\qw&\qw&\gate{R_{n}^{-1}}&\gate{R_{n-1}^{-1}}&\cdots& &\gate{R_2^{-1}}&\gate{H}&\qw\\
      &\qw & \qw & \qw& \cdots & &\gate{R_{n-1}^{-1}}&\gate{R_{n-2}^{-1}}&\cdots& &\gate{H}&\qw&\qw& \cdots& &\ctrl{-1}&\qw&\qw\\
      &\vdots & & & \vdots & & & &\vdots& & & & & &  & &\\
      &\qw &\gate{R_2^{-1}}&\gate{H} & \cdots & & \qw & \ctrl{-2} & \cdots& &\qw&\qw&\ctrl{-3}&\qw  &\qw&\qw&\qw  &\qw\\
      &\gate{H}&\ctrl{-1}&\qw & \cdots & &\ctrl{-3} & \qw & \cdots& &\qw&\ctrl{-4}&\qw&\qw  &\qw&\qw&\qw  &\qw
    }
  }
  \caption{Circuit for the inverse quantum Fourier transform}
  \label{qc:inverseqft}
\end{figure}

% subsubsection quantum_fourier_transform (end)

% subsection examples_of_quantum_circuits (end)

% section quantum_circuits (end)



\section{Extensions to quantum circuits} % (fold)
\label{sec:extensions_to_quantum_circuits}
To facilitate the transition to the programming language \lqpl, this section introduces three
extensions to quantum circuits. The extensions are \emph{renaming}, \emph{wire bending and
crossing}, and \emph{scoped control}. Each extension adds expressive power to quantum circuits but
does not change the semantic power. For each of the extensions, examples of how to re-write the
extension in standard quantum circuit terminology will be provided.

\subsection{Renaming} % (fold)
\label{sub:renaming}


Quantum circuits currently allow renaming to be an implicit part of any gate. The circuit in
\vref{qc:renaming} gives an operation to explicitly do this and its rewriting in standard circuit
notation.

\begin{figure}[htbp]
  \centerline{%
    \Qcircuit @C=1em @R=.7em {
      &\qw_y& \gate{x:=y}  & \qw& \qw_x &\equiv &  &\qw_y& \gate{I} & \qw& \qw_x
    }
  }
  \caption{Renaming of a \protect{\qubit} and its equivalent diagram}\label{qc:renaming}
\end{figure}
% subsection renaming (end)

\subsection{Wire crossing} % (fold)
\label{sub:wire_crossing}

Crossing and bending of wires in a circuit diagram is added to allow a simpler presentation of
algorithms. The circuit in \vref{qc:benditbaby} illustrates the concept of re-organizing and
bending of wires.

\begin{figure}[htbp]
  \[
    \begin{array}{c}
      \Qcircuit @C=1em @R=.7em {
        &\qw& \multigate{1}{U_1}&\qw \ar @{-} [ddr]\\
        &\qw& \ghost{U_1}  & \qw & \qw & \qw & \qw & \qw & \qw \\
        &   &              &                & &\qw & \multigate{1}{U_3} &\qw \\
        &\qw& \multigate{1}{U_2}  & \qw \ar @{-} [dr] & &  \qw &\ghost{U_3} & \qw \\
        &\qw& \ghost{U_2}&\qw \ar @{-}  [ur]&&\qw &\qw &\qw
      }
    \end{array}
  \]
  \caption{Bending}\label{qc:benditbaby}
\end{figure}

% subsection wire_crossing (end)

\subsection{Scoped control} % (fold)
\label{sub:scoped_control}

This extension allows us to group different operations in a circuit and show that all of them are
controlled by a particular \qubit. This is the same as attaching separate control wires to each of
the gates in the grouped operations. Measurements are not affected by control.
\Vref{qc:scopedcontrol} shows a scoped control box on the left which includes a measure. The right
hand side of the same figure shows the circuit translated back to standard circuit notation, with
the measure not being affected by the control.

\begin{figure}[htbp]
  \[
    \begin{array}{ccc}
      \Qcircuit @C=1em @R=.7em {
        &\qw& \gate{H}\gategroup{1}{2}{4}{4}{1.25em}{-}  & \gate{Z}& \qw&\qw\\
        &\qw& \qswap  & \qw& \qw&\qw\\
        &\qw& \qswap \qwx  & \qw& \qw&\qw\\
        &\qw& \qw  & \measureD{M}& \cw&\cw\\
        &  &    &      &     &  \\
        &\qw&  \control \ar @{-} [-1,0]+<0ex,.1ex> \qw& \qw&\qw&\qw
      }
      & \qquad \raisebox{-3em}{$\equiv$} \qquad \qquad &
      \Qcircuit @C=1em @R=.7em {
        & \gate{H} &\qw& \gate{Z}& \qw&\qw\\
        &\qw& \qswap  & \qw& \qw&\qw\\
        &\qw& \qswap \qwx  & \qw& \qw&\qw\\
        &\qw& \qw  & \qw&\measureD{M}& \cw\\
        &  &    &      &     &  \\
        &\ctrl{-5}&\ctrl{-3}&  \ctrl{-5}&\qw&\qw
      }
    \end{array}
  \]
  \caption{Scope of control}\label{qc:scopedcontrol}
\end{figure}

Scoping boxes correspond to procedures and blocks in \lqpl.

Naturally, both scoping and bending may be combined as in \vref{qc:allexts}.


\begin{figure}[htbp]
  \[
    \begin{array}{c}
      \Qcircuit @C=1em @R=.7em {
        &\qw& \multigate{1}{U_1}&\qw&\qw&\qw&\multigate{1}{U_3}\gategroup{1}{7}{3}{8}{.8em}{-}  & \gate{Z}& \qw&\qw&\qw\\
        &\qw& \ghost{U_1}  & \multigate{1}{U_2}&\qw \ar @{-}  [dddr]&&\ghost{U_3} &\measureD{M} &\controlo \cw \cwx[1] &\control \cw \cwx[1] &\cw\\
        &\qw& \qw  & \ghost{U_2}&\qw \ar @{-}  [ur]&& & &\nowiregate{U_4} &\gate{U_5} &\qw \\
        &  &    &      &  & & & &   &  \\
        & & & & & & \control \ar @{-} [-2,0]-<0ex,.9ex> \qw &\qw \ar @{-} @/_/ [uur] &
      }
    \end{array}
  \]
  \caption{Extensions sample}\label{qc:allexts}
\end{figure}

However, note that exchanging wires is not the same as swap. Exchanging a pair of wires is not
affected by control, but a swap is affected by control, as shown in \vref{qc:bendisnotswap}.
%TODO - consider rewriting qcircuits in tikz
\begin{figure}[htbp]
  \[
    \Qcircuitnocompile @C=1em @R=1.5em {
      &\qw&\qswap  &\qw \\
      &\qw&\qswap \qwx &\qw\\
      &\qw& \control \qwx \qw & \qw
    }
    \quad \raisebox{-1.9em}{$\equiv$} \quad
    \Qcircuitnocompile @C=1em @R=1.5em {
      &\qw&\qswap \ghost{U} {\POS"1,2"."2,2"."1,3"."2,3"!C*+<2ex>\frm{-}}
      &\qw &\raisebox{-4em}{$\neq$}&
      &\qw \ar @{-} [dr] & {\hphantom{ }} \gategroup{1}{7}{2}{8}{2ex}{-} &\qw\\
      &\qw&\qswap \qwx &\qw & &&\qw \ar @{-}[ur] &&\qw\\
      &\qw& \control \ar @{-} [-1,0]-<0ex,1.1ex> \qw &\qw & &
      &\qw&\control \ar @{-} [-1,0]-<0ex,1.1ex> \qw &\qw
    }
    \quad \raisebox{-1.9em}{$\equiv$} \quad
    \Qcircuitnocompile @C=1em @R=1.5em {
      &\qw \ar @{-} [dr] &  &\qw\\
      &\qw \ar @{-}[ur] &&\qw\\
      &\qw&\qw &\qw
    }
  \]
  \caption{Swap in control vs. exchange in control}\label{qc:bendisnotswap}
\end{figure}

% subsection scoped_control (end)

\subsection{Circuit identities} % (fold)
\label{sub:circuit_identities}

Circuits allow the writing of the same algorithm in multiple ways. This sub-section will list some
of the circuit identities that hold with the extended notation.

First, note that although a measure may appear inside a control box, it is not affected by the
control, as in \vref{qc:measurenotaffectedbycontrol}. Conversely, a measurement commutes with
control of a circuit as in \vref{qc:measurenotaffectedbycontrol}.

\begin{figure}[htbp]
  \centerline{%
    \Qcircuitnocompile @C=1em @R=1.5em {
      &\qw&\measureD{M}\gategroup{1}{2}{1}{3}{2ex}{-} &\qw &\raisebox{-4em}{$\equiv$}& &\qw&\measureD{M}& \qw\\
      &\qw& \control \ar @{-} [-1,0]-<0ex,2.35ex> \qw &\qw & &&\qw&\qw&\qw
    }
  }
  \caption{Measure is not affected by control}\label{qc:measurenotaffectedbycontrol}
\end{figure}


\begin{figure}[htbp]
  \centerline{%
    \Qcircuitnocompile @C=1em @R=1.5em {
      &\qw & \measureD{M} & \control \cwx[1] \cw &\cw & & \raisebox{-4em}{$\equiv$}& &\qw &\ctrl{1}  & \measureD{M}& \cw\\
      &\qw& \qw & \gate{C_1} & \qw & & & & \qw & \gate{C_1} &\qw &\qw
    }
  }
  \caption{Control  is not affected by measure}\label{qc:controlnotaffectedbymeasure}
\end{figure}

One of the notations introduced earlier was that of \emph{0-control}. This type of control is the
same as applying a $Not$ transform before and after a \emph{1-control}, as shown in
\vref{qc:zctrlisonecontrol}.


\begin{figure}[htbp]
  \centerline{%
    \Qcircuitnocompile @C=1em @R=1.5em {
      &\qw & \gate{U} & \qw & & \raisebox{-4em}{$\equiv$}& &\qw &\qw & \gate{U} &\qw & \qw \\
      &\qw& \ctrlo{-1} & \qw & & & & \qw & \targ & \ctrl{-1} &\targ  &\qw
    }
  }
  \caption{Zero control is syntactic sugar}\label{qc:zctrlisonecontrol}
\end{figure}


\Vref{qc:scopedctrleqparallelcontrol} shows that scoped control of multiple transforms is the same
as controlling those transforms individually.
\begin{figure}[htbp]
  \centerline{%
    \Qcircuitnocompile @C=1em @R=1.5em {
      &\qw&\gate{U_1}\gategroup{1}{2}{2}{3}{1ex}{-} &\qw &\raisebox{-4em}{$\equiv$}& &\qw&\gate{U_1}&\qw&\qw\\
      &\qw&\gate{U_2}&\qw & &&\qw&\qw&\gate{U_2}&\qw\\
      &\qw& \control \ar @{-} [-1,0]-<0ex,2.1ex> \qw &\qw & &&\qw&\ctrl{-2}&\ctrl{-1}&\qw
    }
  }
  \caption{Scoped control is parallel control}\label{qc:scopedctrleqparallelcontrol}
\end{figure}
\Vref{qc:scopedctrleqserialcontrol} similarly shows that scoped control of multiple transforms of
the same \qubit{} is the same as controlling those transforms serially.
\begin{figure}[htbp]
  \centerline{%
    \Qcircuitnocompile @C=1em @R=1.5em {
    &\qw&\gate{U_1}\gategroup{1}{2}{1}{4}{1ex}{-} &\gate{U_2}&\qw &\raisebox{-2em}{$\equiv$}& &\qw&\gate{U_1}&\gate{U_2}&\qw\\
    &\qw& \control \ar @{-} [-1,0]-<0ex,2.1ex> \qw &\qw &\qw &&&\qw&\ctrl{-1}&\ctrl{-1}&\qw
    }
  }
  \caption{Scoped control is serial control}\label{qc:scopedctrleqserialcontrol}
\end{figure}

Multiple control commutes with scoping as shown in
\vrefrange{qc:scopecommutecontrolone}{qc:scopecommutecontroltwo}.
\begin{figure}[htbp]
  \centerline{%
    \Qcircuitnocompile @C=1em @R=1.5em {
      &\qw&\gate{U_1}\gategroup{1}{2}{2}{3}{2ex}{-} &\qw &\raisebox{-4em}{$\equiv$}& &\qw&\gate{U_1}&\qw\\
      &\qw&\ctrl{-1}&\qw & &&\qw&\ctrl{-1}&\qw\\
      &\qw& \control \ar @{-} [-1,0]-<0ex,1.1ex> \qw &\qw & &&\qw&\ctrl{-1}&\qw
    }
  }
  \caption{Multiple control}\label{qc:scopecommutecontrolone}
\end{figure}
\begin{figure}[htbp]
  \centerline{%
    \Qcircuitnocompile @C=1em @R=1.5em {
      &\qw&\ctrl{1}\gategroup{1}{2}{2}{3}{2ex}{-} &\qw &\raisebox{-4em}{$\equiv$}& &\qw&\control \ar @{-} [1,0]+<0ex,2.5ex> \qw&\qw\\
      &\qw&\gate{U_1}&\qw & &&\qw&\gate{U_1}\gategroup{2}{7}{3}{8}{2ex}{-} &\qw\\
      &\qw& \control \ar @{-} [-1,0]-<0ex,2.6ex> \qw &\qw & &&\qw&\ctrl{-1}&\qw
    }
  }
  \caption{Control scopes commute}\label{qc:scopecommutecontroltwo}
\end{figure}
% subsection circuit_identities (end)

% section extensions_to_quantum_circuits (end)





\section{An alternate description of quantum circuits} % (fold)
\label{sec:an_alternate_description_of_quantum_circuits}

In order to explore transforms of quantum circuits, it is helpful to have a algebraic description
which will allow manipulation of the circuits. The goals of the algebraic description are:
\begin{itemize}
  \item Represent \qubits and \bits;
  \item represent gates;
  \item allow algorithmic manipulations of the circuit;
  \item allow proving that correctness of manipulations.
\end{itemize}

Note that measurement is not included in this representation.

\subsection{Base types} % (fold)
\label{sub:base_types}
The base types \bit and \qubit are needed for any circuit. These will be taken as elemental and
correspond to the classical notions of ``bit'' (i.e., $0$ or $1$) and ``qubit'' (i.e.,
$\alpha\ket{0} + \beta\ket{1}$).
% subsection basic_types (end)
\begin{definition}\label{def:endpoint}
  The type \type{endpoint} is either \bit or \qubit.
\end{definition}

\begin{definition}\label{def:qarity}
  The type \type{Q-Arity} is a partial map from $W\subset\nat$ to \type{Endpoint}.
\end{definition}

\begin{definition}\label{def:wire}
  A \emph{wire} is an element of $W\subset\nat$. A \emph{typed wire} is a wire together with a
  specified \type{Q-Arity}.
\end{definition}

\begin{definition}\label{def:control}
  A \emph{control wire} is a pair $(w,b)$ where $w$ is a typed wire and $b::\type{Bool}$.
\end{definition}

A quantum program uses \emph{typed wire}s as its data.

\begin{definition}\label{def:gate}
  A \emph{gate} is a function from $W_1$, a set of typed wires, to $W_2$, another set of typed
  wires. The gate function must be a superoperator as defined in \cite{selinger04:qpl}.
\end{definition}

\subsection{Types and Shapes} % (fold)
\label{sub:types_and_shapes}
Although circuits, gates and low level subroutines are defined at the level of wires,
programmatically, we would like to refer to groupings of wires. For example, a list of \qubit or a
register (tuple) of \qubit. These groupings may have a required leaf type (e.g., \qubit or \bit) or
it may be considered polymorphic. Additionally, the leaf type may be mixed or homogenous.

%todo {Read and work on definitions of shapes and QCData}

\begin{definition}\label{def:singletons}
  The singleton type \type{B} is defined as having the instance \Bs and the singleton type \type{Q}
  is defined as having the singleton instance \Qs.
\end{definition}

\begin{definition}\label{def:qcdata}
  The type family \type{QCData} consists of algebraic data types built up from \bit or \qubit.
\end{definition}

Examples of \type{QCData} include $(\bit,\qubit,[\qubit])$ and $\bit$.

\begin{definition}\label{def:shape}
  Given an instance $I$ of a type \type{T} in \type{QCData}, the \emph{shape} of $I$ is the
  instance $I_s$ obtained from $I$ by replacing all terms of type \bit by \Bs and all terms of type
  \qubit by \Qs.
\end{definition}

For example, a 2 element list of \qubits has shape $[\Qs,\Qs]$ while a pair of a \qubit and \bit
has shape $(\Qs,\Bs)$


The concepts of shape and the type \type{QCData} allow us to group the wires of a quantum circuit
into higher order types.
% subsection types_and_shapes (end)
% section an_alternate_description_of_quantum_circuits (end)
