%!TEX root = /Users/gilesb/UofC/thesis/phd-thesis/phd-thesis.tex
\section{Disjoint sums} % (fold)
\label{sub:disjoint_sums}

As noted at the beginning of this chapter, we now take the reverse direction and determine what
conditions will allow us to define a disjoint sum tensor in an inverse category which already has a
disjoint join. As we shall see, it is sufficient to have ``enough'' \emph{disjoint sum} objects, as
defined below.

\begin{definition}\label{def:disjoint_sum}
  In an inverse category with disjoint joins, an object $X$ is the \emph{disjoint sum} of $A$ and
  $B$ when there exist maps $i_1,\ i_2,\ \xa,\ \xb$ such that:
  \begin{enumerate}[{(}i{)}]
    \item $i_1$ and $i_2$ are monic;
    \item $i_1 : A \to X$, $i_2: B \to X$, $\xa: X \to A$ and $\xb: X \to B$.
    \item $\inv{i_1} = \xa$ and $\inv{i_2} = \xb$.
    \item $\inv{i_1}i_1 \perp \inv{i_2}i_2$ and $\inv{i_1}i_1 \djoin \inv{i_2}i_2 = 1_X$.
  \end{enumerate}
  $i_1$ and $i_2$ will be referred to as the \emph{injection} maps of the disjoint sum.
\end{definition}

\begin{lemma}\label{lem:disjoint_sum_is_unique}
  The disjoint sum $X$ of $A$ and $B$ is unique up to isomorphism.
\end{lemma}
\begin{proof}
  Assume we have two disjoint sums over $A$ and $B$:
  \[
    \xymatrix{
      A\ar[r]^{i_1} &X\ar@/^9pt/[l]^{x_0} \ar@/_9pt/[r]_{x_1} & B \ar[l]_{i_2}
    }
    \qquad  \text{ and  }\qquad
    \xymatrix{
      A\ar[r]^{\jay_1} &Y\ar@/^9pt/[l]^{y_1} \ar@/_9pt/[r]_{y_2} & B \ar[l]_{\jay_2}
    }.
  \]
  We will show that the map $x_1 \jay_1 \djoin x_2 \jay_2 : X \to Y$ is an isomorphism.

  Note by the fact that $i_2$ is monic, we may conclude from the definition that
  $0 = \rst{x_1 i_1 x_2}$ and therefore $0 = x_1 i_1 x_2$. Then, given that $x_1$ is the inverse
  of the monic $i_1$, we may calculate
  $0 = \rg{0} = \wrg{x_2 i_1 x_2} = \rst{\inv{x_2}\, \inv{i_1}\, i_1} =
  \rst{\inv{x_2}\, \inv{i_1}} = \wrg{i_1 x_2}$. From this we see $i_1 x_2 = 0$. Similarly, we have
  $i_2 x_2 = 0$, $j_1 y_2 =0$ and $j_2 y_1 = 0$.

  Next, by Lemma~\ref{lem:disjointness_various}, we know that $\rst{\xa}\cdperp\rst{\xb}$ as both
  $i_1$ and $i_2$ are monic. By the same lemma, $\rg{\jay_1} \cdperp \rg{\jay_2}$ as $y_1, y_2$
  are the inverses of monic maps.  Then, from \axiom{Dis}{7}, we have
  $x_2 \jay_1 \cdperp x_2 \jay_2$, hence we may form $x_2 \jay_1 \djoin x_2 \jay_2 : X \to Y$.

  Similarly, we may form the map $y_1 i_1 \djoin y_2 i_2 : Y \to X$. Computing their composition:
  \begin{align*}
    (x_2 \jay_1 \djoin x_2 \jay_2)(y_1 i_1 \djoin y_2 i_2)
      &= (x_2 \jay_1 (y_1 i_1 \djoin y_2 i_2))\djoin (x_2 \jay_2(y_1 i_1 \djoin y_2 i_2))\\
      &= x_2 \jay_1 y_1 i_1 \djoin x_2 \jay_1 y_2 i_2 \djoin x_2 \jay_2 y_1 i_1 \djoin x_2
        \jay_2 y_2 i_2 \\
      &= x_2\, 1\, i_1 \djoin x_2\, 0\, i_2 \djoin x_2\, 0\, i_1 \djoin x_2\, 1\, i_2\\
      &= x_2 i_1 \djoin x_2 i_2 = 1.
  \end{align*}
  Computing the other direction,
  \begin{align*}
    (y_1 i_1 \djoin y_2 i_2)(x_2 \jay_1 \djoin x_2 \jay_2)
      &= (y_1 i_1 (x_2 \jay_1 \djoin x_2 \jay_2) )\djoin (y_2 i_2(x_2 \jay_1 \djoin x_2 \jay_2))\\
      &= y_1 i_1 x_2 \jay_1 \djoin y_1 i_1 x_2 \jay_2 \djoin y_2 i_2 x_2 \jay_1
        \djoin y_2 i_2 x_2 \jay_2\\
      &= y_1\, 1\, \jay_1 \djoin y_1\, 0\, \jay_2 \djoin y_2\, 0\, \jay_1 \djoin y_2\, 1\, \jay_2\\
      &= y_1 \jay_1 \djoin y_2 \jay_2 = 1.\\
  \end{align*}
  This shows that the map between any two disjoint sums over the same two objects is an isomorphism.
\end{proof}

% TODO: Is this needed??? Where? Why?
%\begin{lemma}\label{lem:all_maps_to_disjoint_sum_are_disjoint}
%  Suppose $X$ is the disjoint sum of $A$ and $B$ in the inverse category \X. Then for all maps
%  $f:C \to A$ and $g: C \to B$, the composition with the injections is disjoint, that is,
%  $f i_1 \perp g i_2$. (This is not right - need further thought...)
%\end{lemma}
%\begin{proof}
%  First note $f i_1 = f i_1 \wrg{i_1} = f i_1 \inv{i_1} i_1$ and similarly, $g i_2 = g i_2
%  \inv{i_2} i_2$.
%\end{proof}

\begin{proposition}\label{prop:a_disjoint_sum_tensor_gives_disjoint_sums}
  A disjoint sum tensor in an inverse category \X gives disjoint sums, i.e., for each pair of
  objects $A,B$, $A\+B$ is a disjoint sum.
\end{proposition}
\begin{proof}
  We claim that setting $i_i = \cp{i}$ and $x_i = \icp{i}$ and setting $X = A\+B$ produces a disjoint
  sum in \X. We show this satisfies the four conditions of Definition~\ref{def:disjoint_sum}.
  \begin{enumerate}[{(}i{)}]
    \item From Lemma~\ref{lem:tensor_identities}, we know that $\cpa$ and $\cpb$ are monic maps.
    \item $\cpa : A \to A\+B$, $\cpb: B \to A\+B$, $\icpa: A\+B \to A$ and $\icpb: A\+B \to B$.
    \item $\inv{\cpa} = \icpa$ and $\inv{\cpb} = \icpb$.
    \item $\inv{i_1}i_1 = 1\+ 0 \tperp 0\+1 = \inv{i_2}i_2$ as
      $1\+0 \tjdown 0\+1 = (\inv{\upr}\+\inv{\upl})$ and
      $1\+0 \tjup 0\+1 = (\icpa\+\icpb)$. For their join,
      $(1\+0)\tjoin (0\+1) = (\inv{\upr}\+\inv{\upl})(\icpa\+\icpb) =
      \inv{\upr}\icpa \+ \inv{\upl}\icpb = 1\+1 = 1$.
  \end{enumerate}
\end{proof}
% Following is really a consequence of the above....
% \begin{lemma}\label{lem:disjoint_sums_have_identity_zero}
%   If $A$ is an object in \X, a disjoint sum category, then $A+0$ is isomorphic to $A$.
% \end{lemma}
% \begin{proof}
%   We write the disjoint sum diagram:
%   \[
%     \xymatrix{
%       A\ar[r]^{1} &A\ar@/^9pt/[l]^{1} \ar@/_9pt/[r]_{0} & 0 \ar[l]_{0}.
%     }
%   \]
% \end{proof}

\begin{lemma}\label{lem:functor_preserving_joins_preserves_disjoint_sums}
  disjoint sums are absolute with respect to functors which preserve disjoint joins.
\end{lemma}
\begin{proof}
  Suppose we are given \X is an inverse category with disjoint sums.
  In \X, consider the disjoint sum over $A$ and $B$,
  \[
    \xymatrix{
      A\ar[r]^{i_1} &X\ar@/^9pt/[l]^{x_0} \ar@/_9pt/[r]_{x_1} & B \ar[l]_{i_2}
    }.
  \]
  The functor $F$ maps this as follows:
  \[
    \xymatrix@R+10pt@C+10pt{
      F(A)\ar[r]^{F(i_1)} &F(X)\ar@/^13pt/[l]^{F(x_0)} \ar@/_13pt/[r]_{F(x_1)} & F(B)\ar[l]_{F(i_2)}
    }.
  \]
  As $F$ is a restriction functor, we immediately have $F(x_0) = F(\inv{i_1}) = \inv{F(i_1)}$ and
  $F(x_1) = \inv{F(i_2)}$. Since $F$ preserves the disjoint join, we also have
  $\inv{F(i_1)}F(i_1) \cdperp\inv{F(i_2)}F(i_2)$ and
  $\inv{F(i_1)}F(i_1) \djoin \inv{F(i_2)}F(i_2) = 1$.

  Finally, as $F$ is a restriction functor, it preserves monics, hence $F(i_1)$ and $F(i_2)$ are
  both monic and therefore $F(X)$ is the disjoint sum of $F(A)$ and $F(B)$.

\end{proof}
\begin{lemma}\label{lem:disjoint_sum_maps_are_perp}
  Given \X a disjoint sum category and maps $f:A \to C$ and $g:B\to D$ in \X. Then
  $\inv{i_1} f i_1 \cdperp \inv{i_2} g i_2 : \invsum{A}{B}\to \invsum{A}{B}$.
\end{lemma}
\begin{proof}
  Note that $\rst{\inv{i_1} f i_1} = \rst{\inv{i_1} f} \le \rst{\inv{i_1}}$ and similarly
  $\rst{\inv{i_2} g i_2} \le \rst{\inv{i_2}}$. Then, by \axiom{Dis}{3}, we have
  $\rst{\inv{i_1} f i_1} \cdperp \rst{\inv{i_2} g i_2}$.
  As $\wrg{\inv{i_1} f i_1} \le \wrg{i_1}$ and  $\wrg{\inv{i_2} g i_2} \le \wrg{i_2}$, we
  have $\wrg{\inv{i_1} f i_1} \cdperp \wrg{\inv{i_2} g i_2}$ and by
  Lemma~\ref{lem:disjointness_various}, this means  $\inv{i_1} f i_1\cdperp \inv{i_2} g i_2$.
\end{proof}


\begin{lemma}\label{lem:disjoint_sums_have_unique_maps_out}
  Given $\X$ is a disjoint sum category. Denote the disjoint sum of objects $A,B$ of $\X$ by
  $\invsum{A}{B}$. Then for objects $A,B$ and $X$ with maps $f:A\to X$ and $g:B\to X$ such that
  $\rg{f} \cdperp \rg{g}$, there exists a unique map $h$ making the following diagram commute.
  \[
    \xymatrix@R+10pt@C+28pt {
      A \ar[dr]^{f} \ar[d]_{i_1}
        \\
      \invsum{A}{B} \ar@{.>}[r]^{h}
        & X\text{.}\\
      B \ar[ur]_{g} \ar[u]_{i_2}
    }
  \]
  We use the notation $f\rgp g$ for the unique map $h$.
\end{lemma}
\begin{proof}
  As $\rg{f} \cdperp \rg{g}$ and $\rst{\inv{i_1}} \cdperp \rst{\inv{i_2}}$ we may form the map $h'
  = \inv{i_1}f \djoin \inv{i_2}g$. By its construction, $h'$ is a map from $\invsum{A}{B}$ to $X$
  which makes the diagram commute. Suppose now that both maps $v$ and $w$ are such maps. Then we
  have
  \[
    (\inv{i_1}i_1) v = (\inv{i_1}i_1) w \quad\text{ and }\quad(\inv{i_2}i_2) v = (\inv{i_2}i_2) w.
  \]
  As $\inv{i_1}i_1 \cdperp \inv{i_2}i_2$, by Lemmas~\ref{lem:disjointness_various} and
  \ref{lem:join_is_associative_and_commutative_monoid}, we know that $(\inv{i_1}i_1) v \cdperp
  (\inv{i_2}i_2) v$ and $(\inv{i_1}i_1) w \cdperp (\inv{i_2}i_2) w$ allowing us to form their
  respective disjoint joins. As the disjoint joins of equal maps remains equal, we have
  \begin{align*}
    (\inv{i_1}i_1) v \djoin (\inv{i_2}i_2) v &=     (\inv{i_1}i_1) w \djoin (\inv{i_2}i_2) w \\
    (\inv{i_1}i_1 \djoin \inv{i_2}i_2 )v &=     (\inv{i_1}i_1 \djoin \inv{i_2}i_2) w \\
    (1 )v &=     (1) w \\
        v &=  w.
  \end{align*}
\end{proof}

\begin{corollary}\label{cor:disjoint_sums_have_unique_maps_in}
  Given $\X$ is a disjoint sum category. Then for objects $A,B$ and $X$ with maps $f:X\to A$
  and $g:X\to B$ such that
  $\rst{f} \cdperp \rst{g}$, there exists a unique map $h$ making the following diagram commute.
  \[
    \xymatrix@R+10pt@C+28pt {
      &A \ar[d]_{i_1}
        \\
      X   \ar[dr]_{g} \ar[ur]^{f}  \ar@{.>}[r]^{h} & \invsum{A}{B}\\
      & B \ar[u]_{i_2}\text{.}
    }
  \]
  We use the notation $f\rstp g$ for the unique map $h$.
\end{corollary}
\begin{proof}
  This is simply the dual of Lemma~\ref{lem:disjoint_sums_have_unique_maps_out}. The unique map $h$
  in this case is $f i_1 \djoin g i_2$.
\end{proof}

\begin{corollary}\label{cor:disjoint_sums_have_unique_maps}
  Suppose $\X$ is a disjoint sum category. Then for objects $A,B,C$ and $D$ with maps $f:A\to C$
  and $g:B\to D$, there exists a unique map $h$ making the following diagram commute.
  \[
    \xymatrix@R+10pt@C+28pt {
      A \ar[r]^{f} \ar[d]_{i_1}
        & C \ar[d]_{i_1} \\
      \invsum{A}{B} \ar@{.>}[r]^{h}
        & \invsum{C}{D}\\
      B \ar[r]_{g} \ar[u]_{i_2}
        & D \ar[u]_{i_2}\text{.}
    }
  \]
  We use the notation $\invsum{f}{g}$ for the map $h$.
\end{corollary}
\begin{proof}
  This follows directly from Lemma~\ref{lem:disjoint_sums_have_unique_maps_out} by setting
  $X = \invsum{C}{D}$. The unique map in this case is $\inv{i_1} f i_1 \djoin \inv{i_2} g i_2$.
\end{proof}
\begin{lemma}\label{lem:functor_preserving_disjoint_sums_preserves_joins}
  Suppose $\X$ and $\Y$ are disjoint sum categories and $F:\X \to \Y$ is a restriction functor which
  preserves disjoint sums. Then, $F$ preserves disjoint joins.
\end{lemma}
\begin{proof}
  By stating that $F$ preserves the disjoint sum, we mean it preserves diagrams derived via the
  properties of the disjoint sum, and specifically, it will preserve the diagrams of
  Lemma~\ref{lem:disjoint_sums_have_unique_maps_out} and
  Corollaries~\ref{cor:disjoint_sums_have_unique_maps_in} and
  \ref{cor:disjoint_sums_have_unique_maps}.

  Suppose we are given $f, g: A \to B$ with $f \cdperp g$. In the disjoint sum category, we know that
  $f \djoin g = (\rst{f}i_1 \djoin \rst{g} i_2) (\inv{i_1} f i_1 \djoin \inv{i_2} g i_2)
  (\inv{i_1} \rg{f} \djoin \inv{i_2} \rg{g})$, as this follows by:
  \begin{enumerate}
    \item Apply Corollary~\ref{cor:disjoint_sums_have_unique_maps_in} to $\rst{f}$ and $\rst{g}$;
    \item then apply Corollary~\ref{cor:disjoint_sums_have_unique_maps} to $f, g$;
    \item finally apply Lemma~\ref{lem:disjoint_sums_have_unique_maps_out} to $\rg{f}, \rg{g}$.
  \end{enumerate}

  Thus, we have that $f\djoin g =
  (\rst{f} \rstp \rst{g})(\invsum{f}{g})(\rg{f} \rgp \rg{g})$. As $F$ preserves
  the disjoint sum, this gives us:
  \begin{align*}
    F(f\djoin g) &= F(\rst{f}\rstp \rst{g})F(\invsum{f}{g})F(\rg{f} \rgp \rg{g})\\
    &=(F(\rst{f})\rstp F(\rst{g}))(\invsum{F(f)}{F(g)})(F(\rg{f}) \rgp F(\rg{g}))\\
    &=(\rst{F(f)}\rstp \rst{F(g)})(\invsum{F(f)}{F(g)})(\wrg{F(f)} \rgp \wrg{F(g)})\\
    &=F(f) \djoin F(g).
  \end{align*}

  The last line is due to \Y being a disjoint sum category as well.

\end{proof}


% subsection disjoint_sums (end)

\subsection{disjoint sum tensor} % (fold)
\label{sub:disjoint_sum_tensor}
\begin{definition}\label{def:disjoint_sum_tensor}
  An \emph{disjoint sum tensor} in an inverse category \X with disjoint joins $\djoin$ based
  on a disjointness relation $\cdperp$ and a restriction zero
  is given by a tensor combined with two restriction monics, $\cpa$ and $\cpb$. The data for the
  tensor is:
 \begin{align*}
    \_ \+ \_ &: \X \times \X \to \X\ \ \text{(a restriction functor preserving disjoint joins)}\\
    0 &: \boldsymbol{1}\to \X \\
    \upl &: 0 \+ A \to A\\
    \upr &: A \+ 0 \to A\\
    a_\+ &: (A \+ B) \+ C \to A \+ (B \+ C) \\
    c_\+ &: A \+ B \to B \+ A\\
    \cpa &: A \to A \+ B\\
    \cpb &: B \to A \+ B
  \end{align*}
  where $\upl, \upr, a_\+, c_\+$ are all isomorphisms and the standard symmetric monoidal
  equations and coherence diagrams hold. The unit of the tensor, $0: \boldsymbol{1}\to \X$, is the
  restriction zero of the category. We specifically note that preserving disjoint joins means the
  tensor obeys the following two equations:
  \begin{align}
    f \cdperp g, \ h \cdperp k&\text{ implies } f\+ h \cdperp g \+ k
    \label{eq:invsum_preserve_perp}\\
    f \cdperp g, \ h \cdperp k &\text{ implies } (f\djoin g)\+(h\djoin k) = (f \+h )\djoin (g\+k).
    \label{eq:invsum_preserve_join}
  \end{align}

\end{definition}


\begin{lemma}\label{lem:a_disjoint_sum_tensor_is_an_disjoint_sum_tensor}
  Given an inverse category \X with a disjoint sum tensor $\+$ as in
  Definition~\ref{def:disjoint_sum_tensor}, then $\+$ is a disjoint sum tensor.
\end{lemma}
\begin{proof}
  From the data of the disjoint sum tensor, the only thing remaining to show is that the tensor
  preserves the disjoint join.

  Suppose we have $f \tperp g$ and $\ h \tperp k$.
  From Lemma~\ref{lem:properties_of_tjdown_and_tjup}\ref{lemitem:l_r_preserve_tensor}, we
  know both $(f\+h)\tjdown(g\+k)$ and $(f\+h)\tjup(g\+k)$ exist, hence $(f\+h)\tperp(g\+k)$.
  This shows Condition~\ref{eq:invsum_preserve_perp}.

  For Condition~\ref{eq:invsum_preserve_join}, we compute from the right hand side:
  \begin{align*}
    (f\+h)\tjoin(g\+k) &= (f\+h)\tjdown (g\+k) \wrg{(f\+h)}\tjup \wrg{(g\+k)}\\
    &= \left( (f\tjdown g)\+ (h \tjdown k) \right)
       \left( (\rg{f}\+\rg{h}) \tjup (\rg{g}\+\rg{k}) \right) \\
    &= \left( (f\tjdown g)\+ (h \tjdown k) \right)
       \left( (\rg{f}\tjup \rg{g}) \+ (\rg{h}\tjup\rg{k}) \right) \\
    &= \left( (f\tjdown g)(\rg{f}\tjup \rg{g})\right) \+
       \left((h \tjdown k) (\rg{h}\tjup\rg{k}) \right) \\
    &=  (f\tjoin g)\+(h \tjoin k).
  \end{align*}
  The second and third lines above again use
  Lemma~\ref{lem:properties_of_tjdown_and_tjup}~\ref{lemitem:l_r_preserve_tensor}.

\end{proof}

\begin{lemma}\label{lem:an_disjoint_sum_tensor_gives_disjoint_sums}
  If $\+$ is a disjoint sum tensor in the inverse category \X, then $A\+B \cong A+B$, an inverse
  sum of $A$ and $B$.
\end{lemma}
\begin{proof}
  As $\+$ is a restriction functor from $\X\times\X$ to $\X$, this actually follows immediately
  from Lemma~\ref{lem:functor_preserving_joins_preserves_disjoint_sums}. It may also be proven
  directly:

  Consider the disjoint sum diagram:
  \[
    \xymatrix @C+60pt{
      A\ar[r]^{i_1=\inv{\upr} (1\+0)}
        & A\+B \ar@/^15pt/[l]^{x_0=(1\+0)\upr} \ar@/_15pt/[r]_{x_1=(0\+1)\upl}
          & B \ar[l]_{i_2=\inv{\upl} (0\+1)}
    }.
  \]
  Therefore, we have $\inv{i_1}i_1 =
  (1\+0)\upr \inv{\upr} (1\+0) = (1\+0)(1\+0) = (1\+0)$. Similarly, $\inv{i_2} i_2 = (0\+1)$.
  Since $0\cdperp 1$, we have $\inv{i_1}i_1 \cdperp \inv{i_2} i_2$.

  By the functorality of $\+$ and that it preserves disjoint joins, we have $(1\+0) \djoin (0 \+1)
  = (1\djoin 0 ) \+ (0\djoin 1) = 1 \+ 1 = 1_{A\+B}$. Hence $A\+ B$ is a disjoint sum of $A$ and
  $B$ and by Lemma~\ref{lem:disjoint_sum_is_unique} it is isomorphic to $A+B$.
\end{proof}

Conversely, we can show that given a tensor which produces disjoint sums, that tensor will
be a disjoint sum tensor.

\begin{lemma}\label{lem:disjoint_sums_give_disjoint_sum_tensors}
  Suppose we have an inverse category $\X$ with restriction zero, a disjointness relation $\cdperp$, a
  disjoint join $\djoin$ and a symmetric monoidal tensor $\+$, with natural restriction monics
  $\cpa: A\to A\+B$ and $\cpb: B \to A\+B$. Further suppose that $A\+B$ is a disjoint sum under $\cpa$ and
  $\cpb$. When $f,g :A \to B$ and $h,k:C\to D$ with $f \cdperp g$ and $h \cdperp k$, then $f
  \+h \cdperp g\+k$ and $(f\+h)\djoin (g\+k) = (f\djoin g)\+ (h\djoin k)$.
\end{lemma}
\begin{proof}
  Similarly, this follows immediately from
  Lemma~\ref{lem:functor_preserving_disjoint_sums_preserves_joins}. We show it directly below:
  \begin{equation}
    \xymatrix@R+10pt@C+28pt {
      A \ar[r]^{\rst{f}} \ar[d]_{\cpa} \ar@/^20pt/[rr]^f
        & A \ar[d]_{\cpa} \ar[r]^{f\djoin g} & B \ar[d]_{\cpa}\\
      A\+C \ar[r]^{\rst{f}\+\rst{h}}_{(1)}
        & A\+C  \ar[r]^{f\djoin g \+ h\djoin k}_{(2)} & B \+D \\
      C \ar[r]_{\rst{h}} \ar[u]_{\cpb} \ar@/_20pt/[rr]_h
        & C \ar[u]_{\cpb} \ar[r]_{h\djoin k} & D \ar[u]_{\cpb}
    }\label{dia:sum_preserve_joins1}
  \end{equation}
  Consider $\inv{\cpa} \rst{f}\cpa$. As this is idempotent and we are in an inverse category, we
  know that $\inv{\cpa} \rst{f}\cpa = \rst{\inv{\cpa} \rst{f}\cpa} = \rst{\inv{\cpa}\rst{f}} =
  \wrg{\rst{f}\cpa}$. Similarly, $\inv{\cpb}\rst{h}\cpb = \wrg{\rst{h}\cpb}$. By \axiom{Dis}{5} and
  \axiom{Dis}{6}, we know that $\wrg{\rst{f}\cpa} \cdperp \wrg{\rst{g}\cpa}$ and $\wrg{\rst{h}\cpb}
  \cdperp \wrg{\rst{k}\cpb}$. As shown in the proof of
  Lemma~\ref{lem:disjoint_sum_is_unique}, we know $\wrg{\cpa}\cdperp \wrg{\cpb}$. Hence, by
  \axiom{Dis}{3}, we have $\wrg{\rst{x}\cpa} \cdperp \wrg{\rst{y}\cpb}$ for any maps $x,y$.

  Hence, we can form the map $\wrg{\rst{f}\cpa} \djoin \wrg{\rst{h}\cpb}$. Referring to the
  Diagram~\ref{dia:sum_preserve_joins1}, by Corollary~\ref{cor:disjoint_sums_have_unique_maps} there is
  a unique map at location $(1)$ which makes the diagram commute --- currently given as
  $\rst{f}\+\rst{h}$. The map $\wrg{\rst{f}\cpa} \djoin \wrg{\rst{h}\cpb}$ also makes the diagram commute.
  Hence, we have $\wrg{\rst{f}\cpa} \djoin \wrg{\rst{h}\cpb} = \rst{f}\+\rst{h}$. Similarly,
  $\wrg{\rst{g}\cpa} \djoin \wrg{\rst{k}\cpb} = \rst{g}\+\rst{k}$. By
  Lemma~\ref{lem:arbitrary_disjoint_joins}, this means $\rst{f\+h} \cdperp \rst{g\+k}$.

  Using a similar argument based on the diagram

  \begin{equation}
    \xymatrix@R+10pt@C+28pt {
      A \ar[r]^{f\djoin g} \ar[d]_{\cpa} \ar@/^20pt/[rr]^f
        & B \ar[r]^{f} \ar[d]_{\cpa}  & B \ar[d]_{\cpa}\\
      A\+C  \ar[r]^{f\djoin g \+ h\djoin k}_{(3)}
        & B\+D \ar[r]^{\rg{f}\+\rg{h}}_{(4)}   & B \+D \\
      C  \ar[r]_{h\djoin k} \ar[u]_{\cpb} \ar@/_20pt/[rr]_h
        & D\ar[r]_{\rg{h}} \ar[u]_{\cpb}  & D \ar[u]_{\cpb}
    }\label{dia:sum_preserve_joins2}
  \end{equation}
  we can show $\wrg{f\+h} \cdperp \wrg{g\+k}$ and therefore $f\+h \cdperp g\+k$.

  This allows us to form the map $(f\+h) \djoin (g\+k)$. Once again, as the objects are inverse
  sums, the map at $(3)$ in Diagram~\ref{dia:sum_preserve_joins2} is unique.
  However, we see that both $f\djoin g \+ h\djoin k$ and $(f\+h) \djoin (g\+k)$ fulfill this
  requirement and hence they are equal.
\end{proof}


\begin{definition}\label{def:disjoint_sum_tensor_category}
  An inverse category $\X$ with restriction zero, a disjointness relation $\cdperp$, a disjoint
  join $\djoin$ and a disjoint sum tensor $\+$ is called an \emph{disjoint sum tensor category}.
\end{definition}

\begin{corollary}\label{cor:disjoint_sum_of_maps_is_disjoint_join}
  In a disjoint sum tensor category, $f \+g$ is given by $\inv{i_1} f i_1 \djoin \inv{i_2} g i_2$.
\end{corollary}
\begin{proof}
  Recall that in the proof of Lemma~\ref{lem:disjoint_sum_is_unique} that we showed $\rst{\inv{i_1}}
  \cdperp \rst{\inv{i_2}}$ and $\wrg{i_1}\perp \wrg{i_2}$. Since $\rst{x f} \le \rst{x}$, by
  \axiom{Dis}{3} and \axiom{Dis}{7}, we know that $\inv{i_1} f i_1 \perp \inv{i_2} g i_2$ and we
  can therefore form the disjoint join.
\end{proof}


%%% Local Variables:
%%% mode: latex
%%% TeX-master: "../../phd-thesis"
%%% End:

%%% Local Variables:
%%% mode: latex
%%% TeX-master: "../phd-thesis"
%%% End:
