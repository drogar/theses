
In \cite{coecke08structures}, Coecke et.al. build on the results of \cite{coeckeetal08:ortho}
to start from a $\dagger$-symmetric monoidal category and construct the minimal machinery needed to
model quantum and classical computations. For the rest of this section, $\cD$ will be assumed to be
such a category, with $\*$ the monoid tensor and $I$ the unit of the monoid.

\begin{definition}\label{def:compact_structure}
  A compact structure on an object $A$ in the category $\cD$ is given by the object $A$, an object
  $A^{*}$ called its \emph{dual} and the maps $\eta:I \to A^{*}\* A$, $\epsilon: A\* A^{*} \to I$
  such that the diagrams
  \[
    \xymatrix@C+20pt{
      A^{*} \ar[dr]^{id} \ar[d]_{\eta\*A^{*}} \\
      A^{*} \*A\*A^{*}  \ar[r]_(.6){A^{*} \*\epsilon} & A^{*}
    }
    \text{ and }
    \xymatrix@C+20pt{
      A \ar[r]^(.4){A\*\eta} \ar[dr]_{id} & A\* A^{*}\* A \ar[d]^{\epsilon\*A}\\
      & A
    }
  \]
  commute.
\end{definition}

\begin{definition}\label{def:quantumstructure}
  A \emph{quantum structure} is an object $A$ and map $\eta:I\to A\*A$ such that
  $(A,A,\eta,\dgr{\eta})$ form a compact structure.
\end{definition}
Note that $A$ is self-dual in definition \ref{def:quantumstructure}.

This allows the creation of the category $\cD_{q}$ which has as objects quantum structures and maps
are the maps in $\cD$ between the objects in the quantum structures.

In the category $\cD_{q}$, it is now possible to define the upper and lower $*$ operations on maps,
such that $(f_{*})^{*}= (f^{*})_{*} = \dgr{f}$:
\begin{eqnarray*}
&f^{*} &:= (\eta_{A}\*1) (1 \* f\*1) (1\*\dgr{\eta}_{B}),\\
&f_{*} &:= (\eta_{B}\*1) (1 \* \dgr{f}\*1) (1\*\dgr{\eta}_{A}).
\end{eqnarray*}

Next, define a classical structure on \cD.
\begin{definition}\label{def:classicalstructure}
  A \emph{classical structure} in \cD{} is an object $X$ together with two maps, $\Delta :X \to X\* X$,
  $\epsilon:X\to I$ such that $(X,\dgr{\Delta},\dgr{\epsilon},\Delta,\epsilon)$ forms a special
  Frobenius algebra.
\end{definition}

As above, this allows us to define $\cD_{c}$, the category whose objects are the classical
structures of $\cD$. The maps in $\cD_{c}$ are given by the maps in $\cD$ between the
objects of the classical structure.

Note that a classical structure will induce a quantum structure, setting $\eta_{X}$ to be
$\dgr{\epsilon_{X}}\, \Delta_{X}$.


%%% Local Variables:
%%% mode: latex
%%% TeX-master: "../phd-thesis"
%%% End:
