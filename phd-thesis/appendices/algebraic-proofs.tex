\chapter{Selected algebraic proofs}

\section{Algebraic proofs for Lemma~\ref{lem:properties_of_delta_and_tensor_in_a_discrete_inverse_category}}

\begin{lemma}
  In a discrete inverse category \X with the inverse product $\*$ and $\Delta$, where
  $e=\rst{e}$ is a restriction idempotent and $f,g,h$ are arrows in \X, the following are true:
  \begin{enumerate}[{(}i{)}]
    \item{}$e=\Delta (e\* 1) \inv{\Delta}$.\label{le:eisde1}
    \item{}$e\Delta (f \* g) = \Delta (e f \* g) $ (and $= \Delta (f \* e g) $ and
      $ = \Delta (e f \* e g)$.)\label{le:deltaefg}
    \item{}$ (f \* g e) \inv{\Delta} =(f \* g) \inv{\Delta} e $ (and $= (f e\* g) \inv{\Delta}$ and
      $ = (f e\* g e)\inv{\Delta}$.)\label{le:efginvdelta}
    \item{}$\restr{\Delta (f \* g) \inv{\Delta}} =
       \Delta(1\* g \inv{f})\inv{\Delta}$. \label{le:restfg}
    \item{} If $\Delta (h \* g) \inv{\Delta} = \restr{\Delta (h \* g) \inv{\Delta}}$ then
      $(\Delta (h \* g) \inv{\Delta}) h = \Delta (h \* g) \inv{\Delta}$.\label{le:hge}
    \item{}$\Delta (f\*1) = \Delta (g\*1) \implies f = g$.\label{le:dfgisfg}
    \item{}$(f\*1) = (g\*1) \implies f = g$.\label{le:fgisfg}
  \end{enumerate}
\end{lemma}
\begin{proof}
  \prepprooflist
  \begin{enumerate}[{(}i{)}]
    \item[\ref{le:eisde1}]This is shown by proving both sides
      equal $\Delta (e\* 1) \inv{\Delta} \Delta (e\* 1) \inv{\Delta}$.
      \begin{align*}
        \Delta (e\* 1) &\inv{\Delta} \Delta (e\* 1) \inv{\Delta} \\
        &= \Delta (e\* 1) \inv{\Delta} \Delta (1\* e) \inv{\Delta}&\text{co-commutativity}\\
        &=\Delta(e \Delta \* 1) (1\*\inv{\Delta} e) \inv{\Delta} &\text{Frobenius}\\
        &=\Delta(\Delta \* 1) (e\*e\*1) (1\*\inv{\Delta}e)\inv{\Delta}&\Delta\text{natural}\\
        &=\Delta(\Delta \* 1) (e\*e\*1) \\
          & \qquad(1\*e\*e)(1\*\inv{\Delta})\inv{\Delta}&\inv{\Delta}\text{ natural}\\
        &=\Delta (\Delta \*1) (e\* e\* e)(1\* \inv{\Delta})\inv{\Delta}&e\text{ idempotent}\\
        &=\Delta (\Delta \*1) (e\* \inv{\Delta}e)\inv{\Delta}&\inv{\Delta}\text{ natural}\\
        &=\Delta (\Delta \*1) (1\* \inv{\Delta})\inv{\Delta}e&\inv{\Delta}\text{ natural}\\
        &=\Delta \inv{\Delta}\Delta \inv{\Delta}e&\text{Frobenius}\\
        &=e&\Delta\text{ total}.
      \end{align*}
      At the same time,
      \begin{align*}
        \Delta (e\* 1) &\inv{\Delta} \Delta (e\* 1) \inv{\Delta}
        =\Delta(e \Delta \* 1) (e\*\inv{\Delta} 1) \inv{\Delta} &\text{Frobenius}\\
        &=\Delta(\Delta \* 1) (e\*e\*1) (e\*\inv{\Delta})\inv{\Delta}&\Delta\text{natural}\\
        &=\Delta(\Delta \* 1) (e\*e\*1) (e\*1\*1)(1\*\inv{\Delta})\inv{\Delta}&\inv{\Delta}\text{ natural}\\
        &=\Delta (\Delta \*1) (e\* e\* 1)(1\* \inv{\Delta})\inv{\Delta}&e\text{ idempotent}\\
        &=\Delta  (e\Delta\* 1)(1\*\inv{\Delta})\inv{\Delta}&\Delta\text{natural}\\
        &=\Delta (e \* 1) \inv{\Delta}\Delta\inv{\Delta} &\text{Frobenius}\\
        &=\Delta (e \* 1) \inv{\Delta} &\Delta\text{ total}\
      \end{align*}
      which gives $e = \Delta (e \* 1) \inv{\Delta}$.
    \item[\ref{le:deltaefg}]This equality starts by using the previous equality:
      \begin{align*}
        e\Delta &(f \* g) = \Delta (e\* 1) \inv{\Delta} \Delta(f \* g)
          &\text{by part \ref{le:eisde1}}\\
        &=\Delta(e  \* 1) \rst{\inv{\Delta}}(f\*g)&\\
        &=\Delta\rst{\inv{\Delta}}(e \*1)(f\*g)
          & \text{\rtwo}\\
        &=\Delta (e f \* g) &\text{ ($f\inv{f} = f$)}.
      \end{align*}
      The second and third equalities follow by co-commutativity and naturality of $\Delta$ and $e$
      being a restriction idempotent.
    \item[\ref{le:efginvdelta}] As in (\ref{le:deltaefg}), details are only given for the
      first equality.
      \begin{align*}
        (f \* g)&\inv{\Delta} e \\
        &= (f \* g) \inv{\Delta}\Delta (1\*e) \inv{\Delta}   &\text{part \ref{le:eisde1}}\\
        &=(f\*g)\rst{\inv{\Delta}}(1\*e) \inv{\Delta}&\\
        &=(f\*g)(1\*e) \rst{\inv{\Delta}}\inv{\Delta}&\rtwo\\
        &=(f\*g e)\inv{\Delta}&\rone
      \end{align*}
      The other equalities follow from co-commutativity and naturality of $\Delta$ and $e$ being
      a restriction idempotent.
    \item[\ref{le:restfg}]Here, we start by using the fact all maps have a partial inverse:
      \begin{align*}
        ~&\restr{\Delta (f \* g) \inv{\Delta} } \\
        &=\Delta (f \* g) \inv{\Delta} \Delta (\inv{f} \* \inv{g}) \inv{\Delta} \\
        &=\Delta (g \* f) \inv{\Delta} \Delta (\inv{g} \* \inv{f}) \inv{\Delta}& \text{co-commutative} \\
        &=\Delta(g\Delta \*f)(\inv{g}\*\inv{\Delta}\inv{f})\inv{\Delta}&\text{Frobenius}\\
        &=\Delta (\Delta\*1)(g \* g \* f)(\inv{g}\*\inv{\Delta}\inv{f})\inv{\Delta}&\Delta \text{ natural}\\
        &=\Delta (\Delta\*1)(g \* g \* f)\\
          &\qquad(\inv{g} \* \inv{f}\* \inv{f})(1\* \inv{\Delta})\inv{\Delta}&\inv{\Delta} \text{ natural}\\
        &=\Delta (\Delta\*1)(\restr{g} \* g \inv{f}\* \restr{f})(1\* \inv{\Delta})\inv{\Delta}&\text{combine maps}\\
        &=\Delta (\Delta\*1)(\restr{g} \* \restr{g}\,g \inv{f}\restr{f}\* \restr{f})(1\* \inv{\Delta})\inv{\Delta}&\text{restriction axioms}\\
        &=\Delta (\restr{g}\Delta\*1)(1\*g\inv{f}\restr{f}\* \restr{f})(1\*\inv{\Delta}) \inv{\Delta}&\Delta \text{ natural}\\
        &=\Delta (\restr{g}\Delta\*1)(1\*g\inv{f}\*1)(1\*\inv{\Delta}\restr{f}) \inv{\Delta}&\inv{\Delta} \text{ natural}\\
        &=\Delta (\Delta\*1)(1 \* \restr{g}g \inv{f}\*1)(1\*\inv{\Delta}\restr{f}) \inv{\Delta}&\text{This lemma(\ref{le:deltaefg})}\\
        &=\Delta (\Delta\*1)(1 \* \restr{g}g \inv{f}\restr{f}\*1)(1\*\inv{\Delta}) \inv{\Delta}&\text{This lemma(\ref{le:efginvdelta})}\\
        &=\Delta (\Delta\*1)(1 \*g \inv{f}\*1)(1\*\inv{\Delta}) \inv{\Delta}&\text{restriction axioms}\\
        &=\Delta c_{A,A}(\Delta\*1)(1 \*g \inv{f}\*1)(1\*\inv{\Delta}) \inv{\Delta}&\text{co-commutative}\\
        &=\Delta (1\*\Delta)c_{A,A\*A}(1 \*g \inv{f}\*1)(1\*\inv{\Delta}) \inv{\Delta}&c_{\*}\text{natural}\\
        &=\Delta (1\*\Delta)(1\*1\*g \inv{f})c_{A,A\*A}(1\*\inv{\Delta}) \inv{\Delta}&c_{\*}\text{natural}\\
        &=\Delta (1\*\Delta)(1\*1\*g \inv{f})(\inv{\Delta}\*1)c_{A,A} \inv{\Delta}&c_{\*}\text{natural}\\
        &=\Delta (1\*\Delta)(1\*1\*g \inv{f})(\inv{\Delta}\*1) \inv{\Delta}&c_{\*}\text{co-commutative}\\
        &=\Delta \inv{\Delta}\Delta (1 \* g \inv{f})\inv{\Delta}&\text{Frobenius}\\
        &=\Delta(1 \* g \inv{f}) \inv{\Delta}&\Delta\text{ total}
      \end{align*}
      Note the pattern in the last few lines of using the co-commutativity of $\Delta$, the
      naturality of the commutativity isomorphism and finishing with the co-commutativity of
      $\inv{\Delta}$. In future proofs, these steps will be combined to a single line and referred to
      as commutativity.
    \item[\ref{le:hge}]Beginning with the assumption,
      \begin{align*}
        (\Delta &(h \* g)\inv{\Delta})h  = \rst{\Delta (h \* g) \inv{\Delta}}h&\\
        &=\Delta(1 \* g \inv{h}) \inv{\Delta}h&\text{This lemma(\ref{le:restfg})}\\
        &=\Delta(1 \* g \inv{h}) (h\*h)\inv{\Delta}&\inv{\Delta}\text{ natural}\\
        &=\Delta(h \* g \rst{\inv{h}}) \inv{\Delta}&\\
        &=\Delta(h\rst{\inv{h}} \* g) \inv{\Delta}&\text{part (\ref{le:deltaefg})}\\
         &=\Delta(h\*g)\inv{\Delta}&\text{property of inverse}.
      \end{align*}

    \item[\ref{le:dfgisfg}]As $\Delta$ is total and natural, we start with:
      \begin{align*}
        f&=\Delta(f\*f)\inv{\Delta}  & \\
        &= \Delta(f\*1)(1\*f)\inv{\Delta} &  \\
        &= \Delta(g\*1)(1\*f)\inv{\Delta} &\text{assumption} \\
        &= \Delta(1\*f)(g\*1)\inv{\Delta} &\text{Identities commute}\\
        &= \Delta(1\*g)(g\*1)\inv{\Delta} &\text{assumption, co-commutative}\\
        &= \Delta(g\*g)\inv{\Delta} \\
        &= g\Delta\inv{\Delta} & \Delta \text{ natural}\\
        &= g&\Delta\text{ total}.
      \end{align*}
    \item[\ref{le:fgisfg}] Immediate from part \ref{le:dfgisfg}.
  \end{enumerate}
\end{proof}

\section{Algebraic proof for Proposition \ref{prop:discrete_inverse_category_has_meets}}

\begin{proposition}
  A discrete inverse category has meets, where $f\cap g =\Delta (f\* g) \inv{\Delta}$.
\end{proposition}
\begin{proof}
  $f\cap g \le f$:

  \begin{align*}
    f\cap g&= \Delta (f\*g) \inv{\Delta}&\text{Definition of }\cap \\
    &= \Delta (f\restr{\inv{f}}\*g) \inv{\Delta} &\text{property of inverse}\\
    &= \Delta (f \* g\restr{\inv{f}}) \inv{\Delta} &\text{by lemma \ref{lem:properties_of_delta_and_tensor_in_a_discrete_inverse_category}(\ref{le:efginvdelta})}\\
    &= \Delta (f \* g\inv{f}f) \inv{\Delta} &\text{definition of inverse}\\
    &= \Delta (1 \* g\inv{f}) \inv{\Delta} f &\inv{\Delta}\text{ natural}\\
    &=\restr{f \cap g} f &\text{by lemma \ref{lem:properties_of_delta_and_tensor_in_a_discrete_inverse_category}(\ref{le:restfg})}\\
  \end{align*}

  $f\cap f = f$:
  \begin{equation*}
    f\cap f = \Delta(f\* f) \inv{\Delta} =f \Delta \inv{\Delta} = f\Delta.
  \end{align*}

  $h(f\cap g) = h f \cap h g$:
  \begin{align*}
    h(f\cap g) &= h \Delta(f\* g) \inv{\Delta}& \text{Definition of }\cap\\
    &= \Delta(h \* h) (f \* g) \inv{\Delta} &\Delta\text{ natural}\\
    &= \Delta(h f\* h g) \inv{\Delta} &\text{compose maps}\\
    &= h f \cap h g&\text{Definition of }\cap.
  \end{align*}
\end{proof}

\section{Algebraic proof for Theorem~\ref{thm:cfrob_is_a_discrete_inverse_category}}
\begin{theorem}
  When \X is a symmetric monoidal category, CFrob(\X) is a discrete inverse category.
\end{theorem}
\begin{proof}
  For $f:X \to Y$, define $\inv{f}$ as
  \[
    Y \xrightarrow{1\*\eta} Y\*X \xrightarrow{1\*\Delta}
      Y\*X\*X \xrightarrow{1\*f\*1} Y\*Y\*X \xrightarrow{\nabla\*1}
      Y\*X \xrightarrow{\epsilon\*1}X.
  \]
  Using a result from \cite{cockett2002:restcategories1}, we need only show:
  \begin{align}
    \inv{(\inv{f})} &= f\label{eq:finvinv_is_f}\\
    f\inv{f}f &= f\label{eq:ffinvf_is_f}\\
    f\inv{f}g\inv{g} &=g\inv{g} f\inv{f}.\label{eq:ffinv_commutes_gginv}
  \end{align}
  We also use the following two identities from \cite{kock04}:
  \begin{align}
    (1\*\eta)\nabla &= id\\
    \Delta(1\*\epsilon) &= id.
  \end{align}
  Proof of Equation~\ref{eq:finvinv_is_f}:
  \begin{align*}
    \inv{\inv{f}} &=(1\*\eta)(1\*\Delta)(1\*(\inv{f})\*1)(\nabla\*1)(\epsilon\*1) \\
    &=(1\*\eta)(1\*\Delta)(1\*((1\*\eta)(1\*\Delta)(1\*f\*1)(\nabla\*1)(\epsilon\*1))\*1)\\
    &\qquad\qquad(\nabla\*1)(\epsilon\*1) \\
    &=(1\*\eta)(1\*\Delta)(1\*1\*\eta\* 1)(1\*1\*\Delta\*1)(1\*1\*f\*1\*1)(1\*\nabla\*1\*1)\\
    &\qquad\qquad(1\*\epsilon\*1\*1) (\nabla\*1)(\epsilon\*1)\\
    &=(\eta\*1)(\Delta\*1)(1\*\nabla)(f\*1)(((\eta)(\Delta\*1)(1\*\nabla)(1\*\epsilon))\*1)
      ((1\*\epsilon)\\
    &=(1\*\eta)\nabla \Delta(1\*\epsilon)f(\eta\*1)\nabla\Delta(1\*\epsilon)\\
    &=id_{x}id_{x}\  f \ id_{y} id_{y}\\
    &=f.
  \end{align*}
  Proof of Equation~\ref{eq:ffinvf_is_f}:
  \begin{align*}
    f\inv{f}f &= f(1\*\eta)(1\*\Delta)(1\*f\*1)(\nabla\*1)(\epsilon\*1)f\\
    &=(1\*\eta)(1\*\Delta)(f\*f\*1)(\nabla\*1)(1\*f)(\epsilon\*1)\\
    &=(1\*\eta)(1\*\Delta)(\nabla\*1)(f\*f)(\epsilon\*1)\\
    &=(1\*\eta)\nabla\Delta(f\*f)(\epsilon\*1)\\
    &=\Delta(f\*f)(\epsilon\*1)\\
    &=f\Delta(\epsilon\*1)\\
    &=f.
  \end{align*}
  Proof of Equation~\ref{eq:ffinv_commutes_gginv}:
  \begin{align*}
    f(1\*\eta)&(1\*\Delta)(1\*f\*1)(\nabla\*1)(\epsilon\*1)g(1\*\eta)(1\*\Delta)(1\*g\*1)
      (\nabla\*1)(\epsilon\*1)\\
    &=(1\*\eta)(1\*\Delta)(\nabla\*1)(f\*1)(\epsilon\*1)(1\*\eta)(1\*\Delta)(\nabla\*1)(g\*1)
      (\epsilon\*1)\\
    &=(1\*\eta)\nabla\Delta(f\*1)(\epsilon\*1)(1\*\eta)\nabla\Delta(g\*1)(\epsilon\*1)\\
    &=\Delta(f\*1)(\epsilon\*1)\Delta(g\*1)(\epsilon\*1)\\
    &=\Delta(1\*\Delta)(f\*g\*1)(\epsilon\*\epsilon\*1)\\
    &=\Delta(1\*\Delta)(g\*f\*1)(\epsilon\*\epsilon\*1)\qquad\qquad\qquad\text{co-commutativity}\\
    &=g\inv{g}f\inv{f}.
  \end{align*}

\end{proof}

\begin{lemma}
  Algebraic proof of Lemma~\ref{lem:mediating_map_equivalence_is_symmetric_reflexive_and_transitive}.
  Definition~\ref{def:xequivalence} gives a symmetric, reflexive equivalence class of maps in \X.
\end{lemma}
\begin{proof}
  \prepprooflist
  \begin{description}
    \itembf{Reflexivity: } Choose $h$ as the identity map.
    \itembf{Symmetry: } Suppose $f\xequiv{h}g$. Then, $\restr{f} = \restr{g}$ and $f k = g$ where
      \[
        k = (\Delta\* 1) \, a_{\*}\, (1\*h)\, \inv{a_{\*}}\,(\inv{\Delta}\* 1).
      \] Applying $\inv{k}$,
      which is
      \[
        (\Delta\* 1) \, a_{\*}\, (1\*\inv{h})\, \inv{a_{\*}}\,(\inv{\Delta}\* 1),
      \]
      we have
      \[
        g \inv{k} = f k \inv{k} = f \restr{k} = \restr{f k} f
        = \restr{g} f = \restr{f} f = f.
      \]

      Thus, $g\xequiv{\inv{h}} f$.

    \itembf{Transitivity: } Suppose $f\xequiv{h} f'$ and $f' \xequiv{k} f''$. Consider the
      compositions of the mediating portions of the equivalences:
      \[
        \ell = ((\Delta \* 1)  a_{\*}  (1 \* h ) \inv{a_{\*}} (\inv{\Delta}\* 1))
          ( (\Delta \* 1) a_{\*}  (1 \* k) \inv{a_{\*}} (\inv{\Delta}\* 1)).
      \]
      By pasting the diagrams which give the above equivalences, we see that $f \ell = f''$.
      However, $\ell$ is not in the form of a mediating map as presented.

      The claim is that $\ell$ is the actual mediating map for $f$ and $f''$. That is, that we have
      $f(\Delta \* 1)a_{\*}(1 \* \ell)\inv{a_{\*}}(\inv{\Delta}\*1) = f''$. In the interest of some
      brevity, this is shown below with the associativity maps elided from the equations.

      We need to show that $(\Delta \* 1)(1 \* \ell)(\inv{\Delta}\*1) = \ell$.
      \begin{align*}
        (\Delta &\* 1)(1 \* \ell)(\inv{\Delta}\*1) \\
        &=
          (\Delta \* 1)(1\*\Delta \* 1)  (1\*1 \* h ) (1\*\inv{\Delta}\* 1))\\
          &\qquad
            (1\*\Delta \* 1) (1\* 1 \* k) (1\*\inv{\Delta}\* 1)(\inv{\Delta}\* 1)\\
        &=
          (\Delta \* 1)(\Delta \* 1\*1)  (1\*1 \* h ) (1\*\inv{\Delta}\* 1))\\
          & \qquad
            (1\*\Delta \* 1) (1\* 1 \* k) (\inv{\Delta}\*1\* 1)(\inv{\Delta}\* 1)
              &\text{co-associativity}\\
        &=
          (\Delta \* 1)  (1 \* h ) (\Delta \* 1\*1)(1\*\inv{\Delta}\* 1)) \\
          &\qquad
            (1\*\Delta \* 1)  (\inv{\Delta}\*1\* 1)(1 \* k)(\inv{\Delta}\* 1)&\text{Naturality}\\
        &=
          (\Delta \* 1)  (1 \* h ) (\inv{\Delta} \*1)(\Delta\* 1)) \\
          &\qquad
            (\inv{\Delta} \* 1)  (\Delta \* 1)(1 \* k)(\inv{\Delta}\* 1)&\text{Frobenius}\\
        &=
          (\Delta \* 1)  (1 \* h ) (\inv{\Delta} \*1)
          (\Delta \* 1)(1 \* k)(\inv{\Delta}\* 1)&\Delta\text{ Total}\\
        &= \ell
      \end{align*}
  \end{description}
  and therefore $f\xequiv{\ell}f''$.
\end{proof}

\begin{corollary}
  Algebraic proof of Corollary~\ref{cor:equivalence_simplified_diagram}
  If $\restr{f} = \restr{g}$ in \X, a discrete inverse category, and the diagram
  \[
    \xymatrix @C=40pt @R=15pt{
      & & B \* C \ar@{.>}[dd]^{1\*h}\\
      A \ar[urr]^f \ar[drr]_{g}\\
      && B\* C'
    }
  \]
  commutes for some $h$, then there is a $h'$ such that $f\xequiv{h'}g$.
\end{corollary}
\begin{proof}

  Consider
  \begin{align*}
    (\Delta \*1)\,a_{\*}\,(1\*(1\*h))\,&\inv{a_{\*}}(\inv{\Delta}\*1)\\
    & = (\Delta \*1)\,((1\*1)\*h)\,a_{\*}\inv{a_{\*}}(\inv{\Delta}\*1) & \text{Naturality}\\
    &=(\Delta \*1)\,((1\*1)\*h)\,(\inv{\Delta}\*1) & \text{Isomorphism Inverse}\\
    &=(\Delta (1\*1) \inv{\Delta})\* h & \text{Naturality of }\*\\
    &=(1\* h) & \Delta  \inv{\Delta}=1\\
  \end{align*}
  and therefore we can set $h' = 1 \* h$.
\end{proof}

%%% Local Variables:
%%% mode: latex
%%% TeX-master: ../phd-thesis.tex
%%% End:
