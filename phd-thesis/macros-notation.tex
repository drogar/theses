%!TEX root = /Users/gilesb/UofC/thesis/phd-thesis/phd-thesis.tex
\newcommand{\bproofenum}{\vspace{12pt} \hspace{-12pt} \renewcommand{\theenumi}{(\roman{enumi})}\renewcommand{\labelenumi}{\theenumi}\begin{enumerate}}
\newcommand{\eproofenum}{\end{enumerate}\renewcommand{\theenumi}{\arabic{enumi}}\renewcommand{\labelenumi}{\theenumi.}}
\newcommand{\bproofitemize}{ \vspace{12pt} \hspace{-12pt} \begin{itemize}}
\newcommand{\eproofitemize}{\end{itemize}}
\newcommand{\bproofdesc}{ \vspace{12pt} \hspace{-12pt} \begin{description}}
\newcommand{\eproofdesc}{\end{description}}
\newcommand{\bi}{\begin{itemize}}
\newcommand{\ei}{\end{itemize}}
\newcommand{\be}{\begin{enumerate}}
\newcommand{\ee}{\end{enumerate}}
\newcommand{\bd}{\begin{description}}
\newcommand{\ed}{\end{description}}
\newcommand{\itembf}[1]{\item{\textbf{#1}}}
\newcommand{\itemem}[1]{\item{\emph{#1}}}
\newcommand{\itemtt}[1]{\item{\texttt{#1}}}

\newcommand{\rone}{[\emph{\bfseries R.1}]\xspace}
\newcommand{\rtwo}{[\emph{\bfseries R.2}]\xspace}
\newcommand{\rthree}{[\emph{\bfseries R.3}]\xspace}
\newcommand{\rfour}{[\emph{\bfseries R.4}]\xspace}

\DeclareMathOperator*{\dom}{dom}
\DeclareMathOperator*{\rng}{range}
\newcommand{\nothing}{\phi}
\newcommand{\nin}{\ensuremath{\in\hspace{-0.8em}/\hspace{0.2em}}}

\newcommand{\BigO}[1]{\ensuremath{\mathscr{O}({#1})}}

\newcommand{\type}[1]{\ensuremath{\mathbf{{#1}}}\xspace}
\newcommand{\bit}{\type{bit}}
\newcommand{\qubit}{\type{qubit}}
\newcommand{\qubits}{\qubit{s}\xspace}
\newcommand{\bits}{\bit{s}\xspace}

\newcommand{\Qs}{\ensuremath{Q_{()}}\xspace}
\newcommand{\Bs}{\ensuremath{B_{()}}\xspace}

\newcommand{\mbf}{\ensuremath{\mathbf{f}}}
\newcommand{\mbg}{\ensuremath{\mathbf{g}}}
\newcommand{\mbh}{\ensuremath{\mathbf{h}}}

\newcommand{\restr}[1]{\overline{#1}}
\newcommand{\rst}[1]{\restr{#1}}
\newcommand{\cp}[1]{\amalg_{#1}}
\newcommand{\cpa}{\cp{0}}
\newcommand{\cpb}{\cp{1}}
\let\*\otimes
\let\+\oplus
\let\<\langle
\let\>\rangle
\newcommand{\uts}{\hspace{0.1pt}}


\newcommand{\spl}[2]{\ensuremath{\text{K}_{#1}(#2)}}
\newcommand{\capspl}{\ensuremath{\cap_{\text{K}}}\xspace}
\newcommand{\inv}[1]{\ensuremath{{#1}^{(-1)}}}
\newcommand{\dual}[1]{\ensuremath{{#1}^{*}}}
\newcommand{\dgr}[1]{\ensuremath{{#1}^{\dagger}}}

\newcommand{\uleft}[1]{\ensuremath{u_{#1}^l}}
\newcommand{\uright}[1]{\ensuremath{u_{#1}^r}}
\newcommand{\usl}{\uleft{\*}}
\newcommand{\usr}{\uright{\*}}
\newcommand{\upl}{\uleft{+}}
\newcommand{\upr}{\uright{+}}

\newcommand{\Xt}{\ensuremath{\widetilde{\X}}\xspace}


\newcommand{\hypXt}{\texorpdfstring{\Xt}{Xt}}
\newcommand{\hypX}{\texorpdfstring{\X}{X}}

\newcommand{\xtdmn}[2]{\ensuremath{{#1}_{|{#2}}}}


\newcommand{\xequiv}[1]{\ensuremath{\overset{\scriptscriptstyle #1}{\simeq}}}
\newcommand{\quest}[1]{\fbox{\fbox{\parbox{5in}{\footnotesize \textbf{ (Question): }#1}}}}
\newcommand{\note}[1]{\par\fbox{\fbox{\parbox{5in}{\footnotesize \textbf{(Internal Note): }#1}}}}
%\newcommand{\todo}[1]{\par\fbox{\fbox{\parbox{4in}{\footnotesize \textbf{2DO: }#1}}}}


\newcommand{\nm}{\ensuremath{n\times{}m}}
\newcommand{\mn}{\ensuremath{m\times{}n}}

\newcommand{\specialcat}[1]{\textsc{#1}\xspace}
\newcommand{\ltrcat}[1]{\ensuremath{\mathfrak{#1}}\xspace}
\newcommand{\ltrcatbb}[1]{\ensuremath{\mathbb{#1}}\xspace}
\newcommand{\B}{\ltrcatbb{B}}
\newcommand{\C}{\ltrcatbb{C}}
\newcommand{\D}{\ltrcatbb{D}}
\newcommand{\E}{\ltrcatbb{E}}
\newcommand{\Lat}{\ltrcatbb{L}}
\newcommand{\X}{\ltrcatbb{X}}

\newcommand{\Y}{\ltrcatbb{Y}}
\newcommand{\Z}{\ltrcatbb{Z}}

\newcommand{\complex}{\C}
\newcommand{\integers}{\Z}

\newcommand{\typecontext}{\ltrcatbb{T}}

\newcommand{\Hil}{\ensuremath{\mathcal{H}}\xspace}
\newcommand{\nat}{\ensuremath{\ltrcatbb{N}}\xspace}

\newcommand{\sets}{\specialcat{Sets}}
\newcommand{\rel}{\specialcat{Rel}}
\newcommand{\fdh}{\specialcat{FdHilb}}
\newcommand{\mon}{\specialcat{Mon}}
\newcommand{\ring}{\specialcat{Rng}}
\newcommand{\Par}{\specialcat{Par}}
\newcommand{\cring}{\specialcat{CRng}}
\newcommand{\cat}{\specialcat{Cat}}
\newcommand{\poset}{\specialcat{Poset}}
\newcommand{\preorder}{\specialcat{Preorder}}
\newcommand{\mcat}{\ensuremath{\mathcal{M}\text{\cat}}\xspace}

\newcommand{\Mstab}{\ensuremath{\mathcal{M}}\xspace}

\newcommand{\obj}[1]{\ensuremath{#1_{obj}}}
\newcommand{\bottom}[1]{\perp_{#1}}
\newcommand{\finpower}{\mathscr{P}_{fin}}

\newcommand{\category}[4]{%
\begin{description}%
\item{\textbf{Objects: }}{#1}%
\item{\textbf{Maps: }}{#2}%
\item{\textbf{Identity: }}{#3}%
\item{\textbf{Composition: }}{#4}%
 \end{description}%
%
}
\newcommand{\rcategory}[5]{%
\begin{description}%
\item{\textbf{Objects: }}{#1}%
\item{\textbf{Maps: }}{#2}%
\item{\textbf{Identity: }}{#3}%
\item{\textbf{Composition: }}{#4}%
\item{\textbf{Restriction: }}{#5}%
\end{description}%
%
}

\newcommand{\rcategoryequiv}[5]{%
\begin{description}%
\item{\textbf{Objects: }}{#1}%
\item{\textbf{Equivalence Classes of Maps: }}{#2}%
\item{\textbf{Identity: }}{#3}%
\item{\textbf{Composition: }}{#4}%
\item{\textbf{Restriction: }}{#5}%
\end{description}%
%
}

\newcommand{\pfcategory}[3]{%
\begin{description}%
\item{\textbf{Well-Defined: }}{#1}%
\item{\textbf{Identities: }}{#2}%
\item{\textbf{Associativity: }}{#3}%
\end{description}}



% general math symbols
\newcommand{\union}{\ensuremath{\bigcup}}
\newcommand{\disjunion}{\ensuremath{\sqcup}}
\newcommand{\intersect}{\ensuremath{\bigcap}}
\newcommand{\logor}{\ensuremath{\lor}}
\newcommand{\logand}{\ensuremath{\land}}
\newcommand{\lognand}{\ensuremath{\barwedge}}
\newcommand{\natmap}{\ensuremath{\Rightarrow}}
\newcommand{\fctrmap}{\ensuremath{\to}}
\newcommand{\fnctrmap}{\fctrmap}
\newcommand{\fmap}{\ensuremath{\to}}
\newcommand{\produces}{\ensuremath{\to}}
\newcommand{\ladjoint}{\ensuremath{\dashv}}
\newcommand{\pproj}{\ensuremath{\preceq_p}}

\newcommand{\tr}[1]{\ensuremath{\text{tr}(#1)}}

\newcommand{\colvec}[1]{\ensuremath{
\begin{pmatrix}#1\end{pmatrix}}}

\newcommand{\tbtmatrix}[4]{\ensuremath{
\begin{pmatrix}
#1&#2\\
#3&#4
\end{pmatrix}
}}

\newcommand{\quadmatrix}[4]{\ensuremath{
\left(
\begin{array}{c|c}
#1&#2\\
\hline
#3&#4
\end{array}
\right)
}}


\newcommand{\nowiregate}[1]{*{\xy *+<.6em>{#1};p\save+LU;+RU **\dir{-}\restore\save+RU;+RD **\dir{-}\restore\save+RD;+LD **\dir{-}\restore\POS+LD;+LU **\dir{-}\endxy}}

%-------------------------------------------------
% subscripting

\newcommand{\jay}{\ensuremath{j}\xspace}
\newcommand{\kay}{\ensuremath{k}\xspace}