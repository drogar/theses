\documentclass[12pt]{article}
\usepackage{palatino,fancyhdr}
\pagestyle{fancy}
\setcounter{tocdepth}{2}
\setlength{\headheight}{15pt}
\date{2006-04-06}
\author{Brett Giles}
\lhead{\bfseries 2006-04-06}
\chead{\thepage}
\rhead{\slshape \rightmark}
\lfoot{\emph{Reviewed - revised after 04-06 meeting}}
\cfoot{}
\rfoot{}
\renewcommand{\headrulewidth}{0.4pt}
\renewcommand{\footrulewidth}{0.4pt}
\begin{document}
\section{New instructions}\label{sec:newinstructions}
\begin{description}
\item{RENAME}
Renames the top element of the stack to the name in the 'nameRegister'.

Typical usage is to rename the result of an expression. For example,
code like \verb|m=5| would generate:
\begin{verbatim}
UseName c99
AllocInt
SetInt 5
UseName m
Rename
\end{verabtim}
\item{GENRENAME}
Generalized the renaming, allowing a target / source relationship. This 
will traverse the stack and rename the first encounter of the name associated
with the instruction to the name currently in the name register.

To rename 'x' to 'y':
\begin{verbatim}
UseName y
GenRename x
\end{verbatim}
\item{ALIASTOP $nm$}
\emph{DEPRECATED}

Aliases $nm$ to the name on the top of the list.

This is needed primarily for function and constructor expressions,
due to the way I am generating code for the argument expressions. An 
expression will leave its value in the top stack element, possibly 
with a newly 
generated name. This new name generation is being done at code gen time.

This is then used to alias the elements to argument names for function calls.

\item{\emph{OP}}
For example, $ADD$, which I envision as adding the top two nodes of the
stack, discarding both and leaving a new node with a newly generated name.

There either would be individual operation codes for each arithmetic 
function, or we could do something like $OP\ ADD$. 

Alternately, these operations could create a new element and leave the 
previous two in existance, requiring us to generate explicit DISCARD 
instructions.  See section 
\ref{sec:expressionhandling} on page \pageref{sec:expressionhandling} for 
further discussion.

Decided at 0406 meeting not to have this - revisiting at 0419 meeting.

\item{USE, ENDUSE}
This would be similar to a switch / case operation. The integer on
top of the stack would have its value(s) placed in a special register
which would then be available non-destructively for the instructions
contained in the USE / ENDUSE pair. The group of instructions would then
be executed on each subtree of the integer. 

Typically, we would then generate a DISCARD operation for the 
top element of the stack (which was USE subject).  See section 
\ref{sec:usehandling} on page \pageref{sec:usehandling} for 
further discussion.

\item{USE\emph{OP}}
For example $USEADD$ which I envision as adding the current use register
to the top element of the
stack.

Again, there either would be individual operation codes for each arithmetic 
function, or we could do something like $OP\ ADD$.
\item{ALLOCINT}
Similar to ALLOCBIT, allocates an integer at the top of the stack, 
initialized to $0$.
\item{SETINT}
Similar to SETBIT, sets the integer on the top of the stack to the value
in the immediate register.

\end{description}
\section{Modified instructions}\label{sec:modifiedinstructions}
\begin{description}
\item{LOADIMMED $n$}
Loads any interger value into the immediate register. Used to only do 
bit values.

\end{description}
\section{Expression handling}\label{sec:expressionhandling}
My intent in handling expressions is that the code generating an expression
would leave a single value at the top of the stack. This implies discarding
values as the expression is generated.

For example, the expression $5+i$ would generate code like:
\begin{verbatim}
UseName c1
AllocInt
LoadImmed 5
SetInt
UseName i
Pullup
UseName c2
Add  -- computes the value(s) of i+5, discarding the 
     -- top two nodes and creates a new node 'c2' with 
     -- that result at the top of the stack. 
\end{verbatim}
Continuing on, if the full statement had been i = 5+i, the code would conclude
with:
\begin{verbatim}
UseName i
Rename
     -- Code generation knows that 'i' had been discarded 
     -- due to its use in the expression.
\end{verbatim}

The biggest question is if the binary operation should do the 
discards automatically. It is probably the easiest thing to implement, but
it seems to do a lot for one instruction. Generating the actual 
discards would not be hard to do.
\section{USE handling}\label{sec:usehandling}
Currently the new instructions only allow a single ``USE'' at a time.
This is probably not realistic. We could easily want to USE two or more
items at a time.

However, this would tend to make the USE\emph{OP} instructions more 
complex than the rest. We would either need to include a name or 
level (some way of addressing USE variables at a higher level). 

Syntactically in the language, we should consider how this is done as well.
We could allow ``use n,m,l in \{ ... \}'' only, with no embeded USE statements
or just allow them anywhere, such as 
\begin{verbatim}
use n in {
  m = n;
  use m in {
    x = 2*m +n
  }
}
\end{verbatim}
In the above case, only 'x' would be available after the completion of the 
USE statements. The code generated for the last assignment would have to
look something like:
\begin{verbatim}
Usename c1
LoadImm 2
SetInt
USEMul 
USEAdd -1 -- Or perhaps even USEAdd n, the name.
Usename x
Rename
\end{verbatim}
In other cases the nesting would be to produce multiple new variables.
\begin{verbatim}
use n in {
  m = n;
  use j in {
    x = 2*j +n
  }
}
\end{verbatim}
In this case, both  'x' and 'm' would be available after the completion of the 
USE statements. The code generated for the last assignment would be the same
as the previous one. The difference would be after the inner USE statement, 
where 'm' is discarded in the first example and 'j' in the second.

\end{document}