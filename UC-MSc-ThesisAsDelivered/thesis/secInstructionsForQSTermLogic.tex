\section{Semantics of basic \protect{\lqpl} statements}\label{sec:programmingaquantumstack}
\Ref{sec:sembasiclqpl} introduced basic \lqpl{} statements, giving the
judgements for their formation. This section introduces the concept of
a statement modifier. Modifiers will be one of 
\emph{IdOnly, Left}, or \emph{Right}. The judgement  for formation
of statements with a modifier is:
\[ \infer[\text{modifier}]
   {\Gamma;\Gamma' \ientail \qmod \qins}
   {\qmod\in\{\text{\texttt{IdOnly, Left, Right}}\} &
     \Gamma; \Gamma' \ientail \qins}
\] 

An operational semantics for basic \lqpl{} statements is given
by the judgement diagrams in \vref{fig:qsstatementtransitionspartone}
and \vref{fig:qsstatementtransitionsparttwo}.
\Ref{tab:qstacktransitionnotation} gives the additional notation
used in the  judgements.



\begin{table}
\centerline{
\begin{tabular}{|>{$}l<{$}|p{3.5in}|}
\hline
\textbf{Notation} & \textbf{Meaning}\\
\hline \hline 
\qvec{z}&A list of names \\[12pt]
\hline
\qop{\Gamma;\Gamma'\ientail\qins}{\Gamma\Gamma''\qentail S}&Application 
of the statements \qins{} to the
quantum stack $S$.\\[12pt]
\hline
\qop{(\Gamma;\Gamma'\ientail\qcontrolled{\qins}{\qvec{z}})}{\Gamma\qentail S} &
Application of the statements \qins{} to the
quantum stack $S$, controlled by the \bits{} and \qbits{} in \qvec{z}.\\[12pt]
\hline
\qresultsin{\qop{(\Gamma;\Gamma'\ientail\qins)}{\Gamma\qentail S^t}}
{\Gamma'\qentail {S'}^{t'}}&
Application of the statements \qins{} with inputs $\Gamma$ and
output $\Gamma'$ to the quantum stack $S$ in context $\Gamma$  results 
in the quantum stack $S'$ in context $\Gamma'$.\\
\hline
\qresultsin{\Gamma\qentail \qop{\qins}{ S^t}}
{\Gamma'\qentail {S'}^{t'}}&
Alternate way to write the application of the statements \qins{} 
to the quantum stack $S$. $\Gamma$ will contain the input variables of
$\qins$ and $\Gamma'$ will contain the output variables.\\
\hline
\end{tabular}
}
\caption{Notation used in judgements for operational semantics}\label{tab:qstacktransitionnotation}
\end{table} 

The  modifiers ( \emph{IdOnly, Left, Right}) come from examining the effect of
a controlled transformation in a general setting. When 
a controlled transform and a general density matrix are multiplied:
\begin{equation}
\begin{pmatrix}1&0\\0&U\end{pmatrix} 
\begin{pmatrix}A&B\\C&D\end{pmatrix} 
\begin{pmatrix}1&0\\0&U^{*}\end{pmatrix} =
\begin{pmatrix}A&B U^{*}\\U C&U D U^{*}\end{pmatrix}\label{eq:controlledtransform}
\end{equation} 
Equating the density matrix to the quantum stack notation, a controlled
transformation is 
controlled by the \qbit{} at the top of the quantum stack. It has
the four sub-stacks $A,B,C$ and $D$ and the $controlled-U$ 
transform is applied as in \ref{eq:controlledtransform}.
The resulting quantum stack, with the same \qbit{} at the 
top now has the four sub-stacks $A,B U^{*},U C$ and $U D U^{*}$. This 
leads to the obvious terminology of ``\IdOnly $U$'' for the action on 
$A$, ``\Right $U$'' for the action on
$B$ and ``\Left $U$'' for the action on $C$.  The action on $D$ is
a normal application of the transform $U$ to $D$. As such, it does
not need a modifier to describe the action.

\Vref{fig:qsstatementtransitionspartone} presents the operational semantics
of the the \lqpl{} statements \emph{identity, new, discard, measure, compose}  and 
\emph{control}.
\Vref{fig:qsstatementtransitionsparttwo} presents the operational semantics 
of \emph{unitary} and the effects of statement modifiers.

\begin{figure}[htbp]
\[
\begin{gathered}
\infer[{\raisebox{2ex}{identity}}]
   { \qresultsin{\Gamma \qentail\qop{\varepsilon}{S}} 
         { \Gamma \qentail S}}
   {}\qquad
\infer[{\raisebox{4ex}{new \bit}}]
   { \qresultsinUp{\Gamma \qentail\qop{(b=0)}{S}} 
         {(b{::}\bit),\Gamma\qentail\qsnodeij{b}{0}{S}{t}}}
   {}\\[12pt]
\infer[{\raisebox{2ex}{new \qbit}}]
   {\qresultsin{\Gamma \qentail \qop{(q=\ket{0})}{S}} 
         {(q{::}\qbit),\Gamma\qentail \qsnodeij{q}{00}{S}{t}}}
   {}\\[12pt]
\infer[{\raisebox{2ex}{delete \bit}}]
   {\qresultsin{(b{::}\bit),\Gamma\qentail \qop{(\text{disc }b)}{\qsbit{b}{S_0}{S_1}{t}}} 
         {\Gamma\qentail  (S_0+S_1)^{t}}}
   {}\\[12pt]
\infer[{\raisebox{2ex}{delete \qbit}}]
   {\qresultsin{(q{::}\qbit),\Gamma \qentail 
        \qop{(\text{disc }q)}
          {\qsnodeij{q}{ij}{S_{ij}}{t}} 
%     {\qsqbitUp{q}{S_{00}}{S_{01}}{S_{10}}{S_{11}}{t}}
     }
     {\Gamma\qentail  (S_{00}+S_{11})^{t}}} 
   {}\\[12pt]
\infer[{\raisebox{6.5ex}{measure}}]
   {{\qresultsin{(q{::}\qbit)\Gamma\qentail \qop{\qmeas{q}{I_0}{I_1}}{\qsnodeij{q}{ij}{S_{ij}}{t_0+t_1}}}
          {\Gamma'\qentail (S_0+S_1)^{(t'_0+t'_1)}}}}
 {\qresultsin{\Gamma\qentail \qop{I_0}{S_{00}^{t_0}}}{\Gamma'\qentail S_0^{t'_0}}\quad
    \qresultsin{\Gamma\qentail\qop{I_1}{S_{11}^{t_1}}}{\Gamma'\qentail S_1^{t'_1}}}\\[12pt]
\infer[{\raisebox{1ex}{compose}}]
   { \qresultsin{\Gamma\qentail\qop{(\qcompose{Op_1}{Op_2})}{S^t}}
       {\Gamma''\qentail S''^{t''}}}
 { \qresultsin{\Gamma \qentail\qop{Op_1}{S^t}}{\Gamma'\qentail {S'}^{t'}}\quad
   \qresultsin{\Gamma' \qentail \qop{Op_2}{{S'}^{t'}}}{\Gamma''\qentail {S''}^{t''}}}\\[12pt]
\infer[{\raisebox{1ex}{tensor}}]
   { \qresultsin{\Gamma_1 \Gamma_2\qentail
        \qop{(\qtensor{Op_1}{Op_2})}
             {(S_1\otimes S_2)^{t_1\times t_2}}}
       {\Gamma_1' \Gamma'_2\qentail (S_1'\otimes S_2')^{t'_1\times t'_2}}}
 { \qresultsin{\Gamma_1 \qentail\qop{Op_1}{S_1^{t_1}}}
        {\Gamma_1'\qentail {S_1'}^{t_1'}}\quad
   \qresultsin{\Gamma_2 \qentail \qop{Op_2}{{S_2}^{t_2}}}
           {\Gamma'_2\qentail {S'_2}^{t'_2}}}\\[12pt]
\infer[{\raisebox{10ex}{control \bit}}]
   { \qresultsinUp{(z{::}\bit),\Gamma \qentail \qop{(\qcontrolled{\qins}{\qvec{z},z})}{\qsbitUp{z}{S_0^{t_0}}{S_1^t}{}}}
     {\qquad\qquad\qquad(z{::}\bit),\Gamma\qentail\qsbitUp{z}{S_0}{S'_1}{t_0+t'}}}
{\qresultsin{\Gamma \qentail  \qop{\qcontrolled{\qins}{\qvec{z}}}{S_1^t}}
   {\Gamma\qentail {S'}_1^{t'}}}\\[12pt]
\infer[{\raisebox{8ex}{control \qbit}}]
   {\qresultsinUp{(z{::}\qbit),\Gamma \qentail \qop{(\qcontrolled{\qins}{\qvec{z},z})}
                    {\qsnodeij{z}{ij}{S_{ij}}{}}}
         {\qquad\qquad\quad(z{::}\qbit),\Gamma \qentail \qsqbitUp{z}{S'_{00}}{S'_{01}}{S'_{10}}{S'_{11}}{t_{00}+t'_{11}}}}
   {\stacktwo{\qresultsin{\Gamma \qentail  
                \qop{(\qcontrolled{(\IdOnly\ \qins)}{\qvec{z}})}{S_{00}}}
                {\Gamma\qentail S'_{00}}}
    {\stackthree{\qresultsin{\Gamma \qentail  
                \qop{(\qcontrolled{(\Left\ \qins)}{\qvec{z}})}{S_{10}}}
                {\Gamma\qentail S'_{10}}}
   {\qresultsin{\Gamma \qentail  
                \qop{(\qcontrolled{(\Right\ \qins)}{\qvec{z}})}{S_{01}}}
                {\Gamma\qentail S'_{01}}}
    {\qresultsin{\Gamma \qentail  
                \qop{(\qcontrolled{\qins}{\qvec{z}})}{S_{11}^{t_{11}}}}
                {\Gamma\qentail {S'}_{11}^{t'_{11}}}}}}
\end{gathered}
\]
\caption{Operational semantics of basic \lqpl{} statements}\label{fig:qsstatementtransitionspartone}
\end{figure}



\begin{figure}[htbp]
\[
\begin{gathered}
\infer[{\raisebox{7ex}{$U$ a transform}}]
   { \qresultsinUp{(q{::}\qbit),\Gamma\qentail
         \qop{U\ q}{\qsnodeij{q}{ij}{S_{ij}}{t}}} 
         {\quad(q{::}\qbit),\Gamma\qentail 
	\qsqbitUp{q}{S_{00}'}{S_{01}'}{S_{10}'}{S_{11}'}{t}}}
   { U\begin{pmatrix}S_{00}&S_{01}\\
  S_{10}&S_{11}\end{pmatrix} U^{*} = \begin{pmatrix}S_{00}'&S_{01}'\\
  S_{10}'&S_{11}'\end{pmatrix}}\\ 
\\
\infer[{\raisebox{6ex}{$\stacktwo{\IdOnly\ U,}{U \text{ a transform}}$}}]
   { \qresultsinUp{(q{::}\qbit),\Gamma\qentail
         \qop{\IdOnly\ U\ q}{\qsnodeij{q}{ij}{S_{ij}}{t}}} 
         {\quad(q{::}\qbit),\Gamma\qentail 
	{\qsnodeij{q}{ij}{S_{ij}}{t}}}}
    {} \\ 
\\
\infer[{\raisebox{8ex}{$\stacktwo{\Left\ U,}{U \text{ a transform}}$}}]
   { \qresultsinUp{(q{::}\qbit),\Gamma\qentail
         \qop{\Left\ U\ q}{\qsnodeij{q}{ij}{S_{ij}}{t}}} 
         {\quad(q{::}\qbit),\Gamma\qentail 
	\qsqbitUp{q}{S_{00}'}{S_{01}'}{S_{10}'}{S_{11}'}{t}}}
    { U\begin{pmatrix}S_{00}&S_{01}\\
  S_{10}&S_{11}\end{pmatrix} = \begin{pmatrix}S_{00}'&S_{01}'\\
  S_{10}'&S_{11}'\end{pmatrix}} \\ 
\\
\infer[{\raisebox{8ex}{$\stacktwo{\Right\ U,}{U \text{ a transform}}$}}]
   { \qresultsinUp{(q{::}\qbit),\Gamma\qentail
         \qop{\Right\ U\ q}{\qsnodeij{q}{ij}{S_{ij}}{t}}} 
         {\quad(q{::}\qbit),\Gamma\qentail 
	\qsqbitUp{q}{S_{00}'}{S_{01}'}{S_{10}'}{S_{11}'}{t}}}
    { \begin{pmatrix}S_{00}&S_{01}\\
  S_{10}&S_{11}\end{pmatrix}U^{*} = \begin{pmatrix}S_{00}'&S_{01}'\\
  S_{10}'&S_{11}'\end{pmatrix}}\\      
\\
\infer[\raisebox{1ex}{\qmod \qins, (\qins{} not a transform)}]
   { \qresultsin{\Gamma\qentail
         \qop{(\qmod \qins)}{S}}
         {\Gamma'\qentail S'}}
    {\qresultsin{\Gamma\qentail \qop{\qins}{S}}
           {\Gamma'\qentail S'}}\\
\\
\infer[\raisebox{2ex}{modifier reduction}]
   { \qresultsin{\Gamma\qentail
         \qop{(\qmod_0\ \qmod_1\cdots\qmod_n)\qins}{S}}
         {\Gamma'\qentail S'}}
    {\qresultsin{\Gamma\qentail \qop{(\qmod_1\cdots\qmod_n)\qins}{S}}
           {\Gamma'\qentail S'} & \qmod_0 \equiv \qmod_1}\\
\\
\infer[\raisebox{2ex}{modifier elimination}]
   { \qresultsin{\Gamma\qentail
         \qop{((\qmod_0\ \qmod_1\cdots\qmod_n)\qins)}{S}}
         {\Gamma'\qentail S'}}
    {\qmod_0 \neq \qmod_1 &
        \qresultsin{\Gamma\qentail \qop{(\text{IdOnly }\qins)}{S}}
           {\Gamma'\qentail S'} }
\end{gathered}
\]
\caption{Operational semantics of basic \lqpl{} statements, continued}\label{fig:qsstatementtransitionsparttwo}
\end{figure}

\subsection{Operational semantics commentary}\label{subsec:operationalsemanticscommentary}
\subsubsection{Identity}
This has no effect on  the quantum stack or the context.

\subsubsection{New \bit{} and new \qbit}
These two statements transform the quantum stack $S$ by adding a 
new node at the top of $S$.
 Both \bit{} and \qbit{} nodes are added with a single 
sub-branch which points to $S$. In the case of
a \bit{} the new sub-branch is labelled with $0$, for a \qbit{} the
new sub-branch is labelled with $00$. These
labels mean the \bit{} is $0$ with $100\%$ probability and the 
\qbit{} is $1.0 \ket{0} + 0.0 \ket{1}$.


\subsubsection{Delete \bit{} and delete \qbit}
On the quantum stack $S$, deletion removes  the top node $x$, merging
 the appropriate branches below $x$. Deletion of a \bit{} merges
both the $0$ and $1$ branches. Deletion of a 
\qbit{} merges only the $00$ and $11$ branches.


\subsubsection{Measure}\label{subsubsec:measuretransitions}
This statement applies separate groups of statements 
to  the $00$ and $11$ sub-stacks of a \qbit{} node, $q$. After applying those
statements, the node $q$ is discarded.


\subsubsection{Compose}\label{subsubsec:composeqstacktransitions}
Composition of the operations on a quantum circuit are done in left to right 
order of their application.


\subsubsection{Control by \bit}
When the statement $\qins_b$ is controlled by the \bit{} $z$, the effect is to
apply $\qins_b$ only to the $1$ sub-branch of $z$.

\subsubsection{Control by \qbit{}}
If the statement $\qins_q$ is controlled 
by the \qbit{} $w$, different operations
are applied to each of $w$'s sub-branches. The modifiers \texttt{IdOnly, Left}
and \texttt{Right} are used: (\texttt{IdOnly} $\qins_q$) is applied to the
the $00$ sub-branch of $w$; (\texttt{Left} $\qins_q$) to the $10$ sub-branch;
(\texttt{Right} $\qins_q$) to the $01$ sub-branch and the unmodified
statement, $\qins_q$, is applied to the $11$ sub-branch.

\subsubsection{Unitary transforms}
Given the unitary transform $U$ and a quantum stack $S$ having the
\qbit{} $q$ at the top, transforming $q$ by $U$ is defined in
terms of matrix multiplication. Note that while 
\ref{fig:qsstatementtransitionsparttwo} only gives the 
operational semantics for a single \qbit{} transform, the
same process is applied for higher degree transforms. A 
quantum stack with two \qbits{} at its top may be converted to 
the $ 4\times4 $ density matrix of those two \qbits{} in the obvious way.
A similar process may be used to extend the operation to more \qbits.

\subsubsection{Modifiers of unitary transforms}
With $U$ a unitary transform, the three rules for \texttt{IdOnly} $U$,
\texttt{Left} $U$ and \texttt{Right} $U$ are given in
terms of the appropriate matrix multiplications. As above, these may
be extended to multiple \qbit{} transformation in the obvious way.

\subsubsection{Modifiers of statements other than unitary transforms}
Modifiers have no effect on any statement of \lqpl{} except for
unitary transformations.

\subsubsection{Combinations of modifiers: reduction and elimination}
When controlling by multiple \qbits, it is possible to have more than
one modifier applied to a statement. The modifier reduction rule 
states that applying the same modifier twice in a row has the same
effect as applying it once.

Modifier elimination, however, reduces the list of modifiers to 
the single modifier \texttt{IdOnly}
whenever  two different modifiers follow one another in the modifier list.









