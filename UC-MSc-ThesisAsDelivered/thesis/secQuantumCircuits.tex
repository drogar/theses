\section{Quantum circuits}\label{sec:QuantumCircuits}
\subsection{Contents of quantum circuits}\label{subsec:contentsOfCircuitDiagrams}
Currently a majority of quantum algorithms are defined and documented
using \emph{quantum circuits}. These are wire-type diagrams with a series of
\qbit{}s input on the left of the diagram and output on the right. Various 
graphical elements are used to describe quantum gates, measurement, control 
and classical \bits. 
\subsubsection{Gates and \qbit{}s}\label{subsubsec:gatesAndQbits}
The simplest circuit  is  a single wire with no
action:
\[
\Qcircuit @C=1em @R=.7em {
  & \qw& \qw_x&\qw\\
}
\]
The next  simplest circuit is one \qbit{} and
one gate. The \qbit{} is represented by a single wire, while the 
gate is represented by a box with its name, $G$,
inside it. This is shown in the circuit in \vref{qc:simpleqcircuit}.
In general, the name of the wire which is input to the gate $G$ may be
different from the name of $G$'s output wire.
Circuit diagrams may also contain constant components as input
to gates as in 
the circuit in \vref{qc:cnotcircuit}. 

\begin{figure}[htbp]
\centerline{%
\Qcircuit @C=1em @R=.7em {
  &\qw_x& \gate{G}  & \qw& \qw_y\\
}}
\caption{Simple single gate circuit}\label{qc:simpleqcircuit}
\end{figure}


\begin{figure}[htbp]
\centerline{%
\Qcircuit @C=1em @R=.7em {
 &\qw_x & \gate{H} &\qw_{\quad x'} & \ctrl{1} & \qw_{\ x'}\\
 &\qw_y  & \qw & \qw & \targ & \qw_{\ y'}\\
}}
\caption{Entangling two \qbit{}s.}
\label{qc:appEntangle}
\end{figure}


\begin{figure}[htbp]
\centerline{%
\Qcircuit @C=1em @R=.7em {
\lstick{\ket{1}} & \targ & \rstick{\ket{0}} \qw \\
\lstick{\ket{1}}  & \ctrl{-1} & \rstick{\ket{1}} \qw \\
}}
\caption{Controlled-\nottr{} of $\ket{1}$ and $\ket{1}$}
\label{qc:cnotcircuit}
\end{figure}

Future diagrams will drop the wire labels except when
they are important to the concept under discussion.

Controlled gates, where the gate action depends upon another \qbit,
 are shown by attaching a wire between the wire of  the control \qbit{} and
the controlled gate. 
The circuit in \vref{qc:appEntangle} shows two \qbit{}s, where a
\Had{} is applied to the top \qbit{}, followed by 
 a \Cnot{} applied to the second \qbit{}.
In circuits, the control \qbit{} is on  the vertical wire with the 
solid dot. This is then connected via a horizontal wire to the
gate being controlled.

A list of common gates, their circuits and corresponding matrices is given
in \vref{tab:qgatesAndRep}.

\begin{table}
\centerline{
\begin{tabular}{|l|c|l|}
\hline
\text{\textbf{Gate}}&\text{\textbf{Circuit}}&\text{\textbf{Matrix}}\\
\hline
& & \\
Not ($X$) &
$\Qcircuit @C=1em @R=.7em {
 & \targ & \qw 
}$ & $ 
\begin{bmatrix}
0&1\\
1&0
\end{bmatrix}$ \\ & & \\\hline
 & &  \\
$Z$ &
$\Qcircuit @C=1em @R=.7em {
 & \gate{Z} & \qw 
}$ & $ 
\begin{bmatrix}
1&0\\
0&-1
\end{bmatrix}$ \\ & & \\\hline
 & &  \\
Hadamard &
$\Qcircuit @C=1em @R=.7em {
 & \gate{H} & \qw 
}$ & $ 
\frac{1}{\sqrt{2}}\begin{bmatrix}
1&1\\
1&-1
\end{bmatrix}$ \\ & & \\\hline
& & \\
Swap &
$\Qcircuit @C=1em @R=.7em {
 & \qswap & \qw \\
 & \qswap \qwx & \qw 
}$ & $ 
\begin{bmatrix}
1&0&0&0\\
0&0&1&0\\
0&1&0&0\\
0&0&0&1
\end{bmatrix}$ \\ & & \\\hline
& & \\
Controlled-Not &
$\Qcircuit @C=1em @R=.7em {
 & \ctrl{1} & \qw \\
 & \targ & \qw 
}$ & $ 
\begin{bmatrix}
1&0&0&0\\
0&1&0&0\\
0&0&0&1\\
0&0&1&0
\end{bmatrix}$ \\ & & \\\hline
 & &  \\
Toffoli &
$\Qcircuit @C=1em @R=.7em {
 & \ctrl{1} & \qw \\
 & \ctrl{1} & \qw \\
 & \targ & \qw 
}$ & $ 
\begin{bmatrix}
1&0&0&0&0&0&0&0\\
0&1&0&0&0&0&0&0\\
0&0&1&0&0&0&0&0\\
0&0&0&1&0&0&0&0\\
0&0&0&0&1&0&0&0\\
0&0&0&0&0&1&0&0\\
0&0&0&0&0&0&0&1\\
0&0&0&0&0&0&1&0
\end{bmatrix}$ \\ & & \\\hline
\end{tabular}
} % end centerline
\caption{Gates, circuit notation and matrices}\label{tab:qgatesAndRep}
\end{table}
\subsubsection{Measurement}\label{subsubsec:measurement}
Measurement is used to transform the quantum data to classical data so that it
may be then used in classical computing (e.g. for output). The act of 
measurement
is placed at  the last part of the quantum algorithm in many 
circuit diagrams and is sometimes just implicitly considered to be there.

While there are multiple notations used for 
measurement in quantum circuit diagrams, this thesis will
standardize on the \emph{D-box} style of
measurement as shown in \ref{qc:dboxmeasure}.

%\begin{figure}[htbp]
%\centerline{%
%\Qcircuit @C=1em @R=.7em {
% & \gate{H} &\ctrl{1} & \meter & \cw &\rstick{\text{meter}}\\
% & \ctrl{1} & \targ & \measure{\mbox{bit}}& \cw  &\rstick{\text{oval}}\\
% & \gate{V} &\qw & \measuretab{M}  & &\rstick{\text{tab}}\\
% & \gate{W} & \qw & \measureD{\chi}& &\rstick{\text{D-box}}
% }}
%\caption{Examples of different measure notations}
%\label{qc:manymeasures}
%\end{figure}

\begin{figure}[htbp]
\centerline{%
\Qcircuit @C=1em @R=.7em {
 & \gate{W} & \qw & \measureD{\chi}& \cw & &\rstick{\text{D-box}}
}}
\caption{Measure notation in quantum circuits}
\label{qc:dboxmeasure}
\end{figure}

A measurement may have a double line leaving it, signifying a  \bit, or
nothing, signifying a destructive measurement.

Operations affecting multiple \qbit{}s at the same time are shown
by extending the gate or measure box to encompass all desired wires.
In the circuit in \vref{qc:multibox},  the gate $U$ 
applies to all of the first three \qbits{} and 
 the measurement applies to the first two \qbits{}.
\begin{figure}[htbp]
\centerline{%
\Qcircuit @C=1em @R=.7em {
 & \multigate{2}{U} & \multimeasureD{1}{\mbox{bits}} & \cw \\ 
& \ghost{U} &  \ghost{\mbox{bits}} &\cw \\ 
& \ghost{U} & \qw &\qw }}
\caption{Examples of multi-\qbit{} gates and measures}
\label{qc:multibox}
\end{figure}
\subsubsection{$0$-control and control by \protect{\bits}}\label{subsubsec:otherElementsOfCircuits}
The examples above have only shown  control based upon
a \qbit{} being \ket{1}. Circuits also allow control on a 
\qbit{} being \ket{0} and upon classical values. The circuit
in \vref{qc:zctrlAndClassicalControl} with four \qbit{}s $(r_1, r_2, p$ and 
$q)$ illustrates all these forms of control.

At $g_1$, a  \Had{} is  $1-$controlled by $r_2$ and
  is applied to each of $r_1$ and $p$. This is followed in 
column $g_2$ with the \nottr{} transform applied to $r_2$ 
being $0-$controlled by $r_1$ . In the same column,
a  $Z$ gate is $0-$controlled by $q$ and applied to $p$. 
$p$ and $q$ are then measured in column $g_3$ 
and their corresponding classical values are
used for control in $g_4$.
In $g_4$, the $U_R$ gate is applied to both $r_1$ and $r_2$, 
but only when the
measure result of $p$ is $0$ and the measure result of $q$ is $1$.

\begin{figure}[htbp]
\centerline{%
\Qcircuit @C=1em @R=.7em {
&\qw_{r_1} & \gate{H} &\ctrlo{1} & \qw & \qw & \multigate{1}{U_R} & \qw \\
&\qw_{r_2} & \ctrl{-1} \qwx[1] &\targ & \qw  & \qw & \ghost{U_R} & \qw \\
&\qw_{p} & \gate{H} & \gate{Z} &\qw & \multimeasureD{1}{M_{p,q}} & \controlo \cw \cwx \\
&\qw_{q} & \qw & \ctrlo{-1} &\qw & \ghost{M_{p,q}} & \control \cwx \cw\\
&&\lstick{g_1}&\lstick{g_2}&&\lstick{g_3}&\lstick{g_4}
}}
\caption{Other forms of control for gates}
\label{qc:zctrlAndClassicalControl}
\end{figure}
\subsubsection{Multi-\protect{\qbit} lines}
It is common to represent multiple \qbit{}s  on one line. 
A gate
 applied to a multi-\qbit{} line must be a tensor product of gates of the 
correct dimensions. The circuit 
in \vref{qc:manyQbitsOneLine} shows $n$ \qbit{}s on one line with the
\Had{} gate (tensored with itself $n$ times) applied to all of them. 
That is  followed by a unary gate $U_R$ tensored with $I^{\otimes (n-2)}$
and tensored with itself again. This will have the effect of applying
an $U_R$ gate to the first and last \qbits{} on the line.

\begin{figure}[htbp]
\centerline{
\Qcircuit @C=1em @R=.7em {
 & {/^{{}^n}} \qw  & \gate{H^{\otimes n}} & \gate{U_R\otimes I^{\otimes (n-2)} \otimes U_R} & \qw
}
}
\caption{$n$ \qbit{}s on one line}
\label{qc:manyQbitsOneLine}
\end{figure}
\subsubsection{Other common circuit symbols}

Two other symbols that are regularly used are the swap and controlled-\Z, shown
in the circuit in \vref{qc:swapAndCtrlZ}. Note that swap is just shorthand for
a series of three controlled-\nottr{} gates with the control \qbit{} changing. 
This can also be seen directly by multiplying the matrices for the
controlled-\nottr{} gates  as shown in \vref{eq:cnotcnotcnot}.

\begin{figure}[htbp]
\centerline{
\begin{tabular}{c}
\Qcircuit @C=1em @R=.7em {
 & \qswap & \qw &  \raisebox{-3.5ex}{=} & & \ctrl{1} & \targ &\ctrl{1}  &\qw \\
 & \qswap \qwx &\qw &  & & \targ &\ctrl{-1} & \targ &\qw \\
} \\
  \\
\Qcircuit @C=1em @R=.7em {
 & \ctrl{1} &\qw  & \raisebox{-3.5ex}{=} & & \Box \qw &\qw \\
 & \gate{Z} & \qw & & & \Box \qw \qwx & \qw
}
\end{tabular}
}
\caption{Swap and controlled-Z}
\label{qc:swapAndCtrlZ}
\end{figure}

{\begin{equation}
\begin{singlespace}
\begin{bmatrix}
1&0&0&0\\
0&0&1&0\\
0&1&0&0\\
0&0&0&1
\end{bmatrix} =
\begin{bmatrix}
1&0&0&0\\
0&1&0&0\\
0&0&0&1\\
0&0&1&0
\end{bmatrix}
\begin{bmatrix}
1&0&0&0\\
0&0&0&1\\
0&0&1&0\\
0&1&0&0
\end{bmatrix}
\begin{bmatrix}
1&0&0&0\\
0&1&0&0\\
0&0&0&1\\
0&0&1&0
\end{bmatrix}.\label{eq:cnotcnotcnot}
\end{singlespace}
\end{equation}}


\subsection{Syntax of quantum circuits}\label{subsec:SyntaxOfQCD}
Quantum circuits were originally introduced by David 
Deutsch in \cite{deutsch89:qciruits}. 
He extended the idea of standard classical based gate diagrams to 
encompass the quantum cases. In his paper, he introduced the concepts
of quantum gates, sources of \bit{}s, sinks and universal gates. One
interesting point of the original definition is that it \emph{does} allow
loops. Currently, the general practice is not to allow loops of \qbit{}s.
The commonly used elements of a circuit are summarized in \vref{tab:qcdSyntax}.

\begin{table}
\setlength\extrarowheight{4pt}
\begin{tabular}{|>{\raggedright\hspace{0pt}}p{.9in}|p{3in}|>{$}c<{$}|}
\hline
\textbf{Desired element} & \textbf{Element in a quantum circuit diagram.} & \mbox{\textbf{Example}}\\
\hline \hline 
\qbit{} & A single horizontal line. &
\Qcircuit @C=1em @R=.7em {
 & \qw
} \\\hline
Classical \bit{} & A double horizontal line. 
&\Qcircuit @C=1em @R=.7em {
 & \cw
} \\\hline
Single-\qbit{} gates & A box with the gate name ($G$) inside it, one wire 
attached on its left and one wire attached on the right. &
\Qcircuit @C=1em @R=.7em {
 & \gate{G} & \qw
}  \\[4pt]\hline
Multi-\qbit{} gates & A box with the gate name ($R$) inside it, 
$n$ wires on the left side and the same number of wires on the right. &
\Qcircuit @C=1em @R=.7em {
 & \multigate{1}{R} & \qw \\
 & \ghost{R} & \qw
}  \\[4pt]\hline
Controlled \qbit{} gates & A box with the gate name  ($H$, $W$) inside, with
a solid (1-control) or open (0-control) dot on the control wire 
with a vertical wire between the dot and the second gate.
&\Qcircuit @C=1em @R=.7em {
 & \ctrl{1} & \gate{W} & \qw \\
 & \gate{H} & \ctrlo{-1} &\qw
}  \\[4pt]\hline
Controlled-Not gates & A \emph{target} $\oplus$, with
a solid (1-control) or open (0-control) dot on the control wire 
with a vertical wire between the dot and the gate.
&\Qcircuit @C=1em @R=.7em {
 & \ctrl{1} & \qw \\
 & \targ &\qw
}  \\[4pt]\hline
Measurement & A \emph{D-box} shaped node with optional names or comments
inside. One to $n$ single wires are attached on the left 
(\qbit{}s coming in) and $0$ to $n$ classical \bit{} wires 
on the left. Classical \bits{} may be dropped as desired.
&\Qcircuit @C=1em @R=.7em {
 & \ctrl{1} & \multimeasureD{1}{q,r} & \cw \\
 & \gate{H} & \ghost{q,r} 
}  \\[4pt]\hline
Classical control & Control bullets attached to 
horizontal classical wires, with vertical classical wires attached to
the controlled gate.
&\Qcircuit @C=1em @R=.7em {
 &  \measureD{r} & \control \cw \cwx[1] \\
 & \qw & \gate{X} & \qw
}  \\[4pt]\hline
Multiple \qbit{}s & Annotate the line with  the number of \qbit{}s
and use tensors on gates.
&\Qcircuit @C=1em @R=.7em {
 & {/^{{}^n}} \qw  & \gate{H^{\otimes n}} & \qw
}  \\[4pt]\hline
\end{tabular}
\caption{Syntactic elements of quantum circuit diagrams}\label{tab:qcdSyntax}
\end{table}

A valid quantum circuit must follow certain restrictions. As physics requires
\qbits{} must not be duplicated, circuits must enforce this rule.
 Therefore, three restrictions in circuits are the \emph{no fan-out}, 
\emph{no fan-in} and \emph{no loops} rules. These conditions are  a
way to express the \emph{linearity} of quantum algorithms. Variables (wires)
may not be duplicated, may not be destroyed without a specific
operation and may not be amalgamated.


\subsection{Examples of quantum circuits}\label{subsec:exampleQuantumCircuits}
This section will present three quantum algorithms
and the associated circuits. 
Each of these circuits presented may be found  in \cite{neilsen2000:QuantumComputationAndInfo}.

First, \emph{quantum teleportation}, an algorithm which sends a quantum bit
 across a distance via the exchange of two classical bits. This is
followed by the \emph{Deutsch-Jozsa algorithm}, 
which provides information about the 
global nature of a function with less work than a classical deterministic 
algorithm can. The third example is  circuits for the 
\emph{quantum Fourier transformation} and its inverse.

\subsubsection{Quantum teleportation}\label{subsubsec:quantumTeleportation}
The standard presentation of this algorithm involves two 
participants $A$ and $B$. 
(Henceforth known as Alice and Bob). Alice and Bob first 
initialize two \qbit{}s
to \ket{00}, then place them into what is known
 as an \emph{EPR} (for Einstein, 
Podolsky and Rosen) state. This is accomplished by first applying the \Had{} 
gate to Alice's \qbit{}, followed by a \Cnot{} to the pair of \qbits{}
 controlled
by Alice's \qbit{}.

Then, Bob travels somewhere distant from Alice, taking his \qbit{} with 
him\footnote{Notice that all other physical constraints are ignored in this 
algorithm. There is no concern about how one separates the \qbit{}s, transports
the \qbit{} or potential decoherence of the \qbit{}s.}.

At this point,  Alice receives a 
 \qbit{}, $\nu$,  in an unknown state and has to  pass $\nu$
 on to Bob. She then uses $\nu$ as  the control and applies a
{\Cnot{}} transform to this new pair. Alice then applies a \Had{}
transform applied to $\nu$.

Alice now measures the two \qbit{}s and sends the resulting
two \bits{} of classical information to Bob.

Bob then examines the two \bits{} that he receives from Alice.
If the \bit{} resulting from measuring Alice's original bit is $1$,
he applies the \nottr{} (also referred to as $X$) gate 
($\begin{singlespace}=\begin{bmatrix}0&1\\1&0\end{bmatrix}\end{singlespace}$) to his \qbit{}.
If  the measurement result of $\nu$ is one,
he applies the $Z$ gate 
($\begin{singlespace}=\begin{bmatrix}1&0\\0&-1\end{bmatrix}\end{singlespace}$). 
Bob's \qbit{} is now in the same state as the \qbit{} Alice wanted to send.
The circuit for this is shown in \vref{qc:quantumTeleportation}.

\begin{figure}[htbp]
\centerline{%
\Qcircuit @C=1em @R=.7em {
\lstick{\ket{\nu}} & \ctrl{1} & \gate{H} & \measureD{M_1} & \cw &  \control \cw  \cwx[2] \\
\lstick{\mbox{A}}  & \targ & \qw & \measureD{M_2} & \control \cw  \cwx[1] \\
\lstick{\mbox{B}} & \qw &\qw & \qw & \gate{X} & \gate{Z} & \qw & \rstick{\ket{\nu}}\\
& \rstick{\ket{s_1}}& \rstick{\ket{s_2}} & & \lstick{\ket{s_3}} & &
\lstick{\ket{s_4}} }}
\caption{Quantum teleportation}
\label{qc:quantumTeleportation}
\end{figure}
For comparison , see \vref{fig:stackTeleportation} 
 showing how this would be implemented
in \lqpl{}.



\subsubsection{Deutsch-Jozsa algorithm}\label{subsubsec:djAlgorithm}
The Deutsch-Jozsa algorithm describes a way of determining whether
a function $f$\footnote{The obvious pre-condition for the 
Deutsch-Jozsa algorithm is that the function $f$ is \emph{either}
balanced or constant and not some general function. The results
are not well-defined if $f$ does not fit into one of the two 
possible categories.} is \emph{constant} (i.e. always $0$ or $1$) or 
\emph{balanced} (i.e. produces an equal number of $0$ or $1$ results)
 based on applying it to one quantum bit. The function 
takes $n$ \bit{}s as input and produces a single \bit. 


 $f$ is assumed to be an expensive function, therefore, a desired effect is to
evaluate $f$ as few times as possible before determining if
$f$ is balanced or constant. 
The worst case scenario when evaluating $f$ classically is that determining
the result
requires $2^{n-1} + 1$ invocations of the function.
The best possible case is $2$ invocations, which occurs 
when $f$ is balanced and
the first two inputs chosen  produce  different results.

The quantum circuit requires only one application of the function
to $n+1$ \qbits{} which have been  suitably prepared to make the decision. 

The algorithm relies on being able to construct an $n+1$ order unitary 
operator based upon $f$. In general, a unitary operator like this
may be constructed by mapping the 
state $\ket{a,b}$ to $\ket{a,b\oplus f(a)}$ 
where $\oplus$ is the exclusive-or operator and $a$ is $n$ \bit{} values. 
If we name this operator 
$U_f$,  the circuit in \vref{qc:djAlgorithm} will solve the problem
with  just one application. See the appendix, \vref{appsubsec:djalgorithm}
for how this would be done in \lqpl.
\begin{figure}[htbp]
\centerline{%
\Qcircuit @C=1em @R=.7em {
\lstick{\ket{0}} & {/^{{}^n}} \qw & \gate{H^{\otimes n}} & \multigate{1}{U_f} & \gate{H^{\otimes n}}& \qw &\multimeasureD{1}{DJ} \\
\lstick{\ket{1}} & \qw  & \gate{H} & \ghost{U_f} & \qw & \qw&\ghost{DJ}
}
}
\caption{Circuit for the Deutsch-Jozsa algorithm}
\label{qc:djAlgorithm}
\end{figure}

The idea of quantum parallelism is what makes  this
 and many other quantum algorithms work.
 The initial state of the system is set to 
$\ket{0^{\otimes n} \otimes 1}$ after which  the
\Had{} gate is applied 
to all of the \qbit{s}. This places the input \qbit{}s into
a superposition of all possible input values and the answer \qbit{} is
a superposition of 0 and 1. At this point, the unitary transformation $U_f$
 is applied to the \qbit{}s. Then the \Had{} transform is
applied again to the input \qbit{}s.

To complete the algorithm, measure \emph{all} the \qbit{}s. It can be
shown that if $f$ is constant, the input \qbit{}s will all measure
to 0, while if it is balanced, at least one of those \qbit{}s will be
1.

\subsubsection{Quantum Fourier transform}\label{subsubsec:QFTcircuit}
The  circuits for the quantum Fourier 
transformation and its inverse are in  \ref{qc:qft} and 
 \ref{qc:inverseqft} respectively.
 These transforms are used extensively in many quantum 
algorithms, including Shor's factoring algorithm.

%\include{omits/classicalFFT.tex
The quantum Fourier transform is definable  on an arbitrary number of \qbits. 
This is typically presented by eliding the $3^{\text{rd}}$ to the 
$n-3^{\text{rd}}$ lines and interior processing. The \lqpl{} code for
the quantum Fourier transform  is in the appendix, \vref{fig:qft}.
 In this circuit, the parametrized transform $R_n$ is
the rotation transform, given by:
\[R_n={\begin{singlespace}\begin{bmatrix}1&0\\
   0&e^{\frac{2\pi i}{2^n}}\end{bmatrix}\end{singlespace}}\]
\begin{figure}[htbp]
\centerline{%
\Qcircuit @C=1em @R=.7em {
 &\qw & \gate{H} & \gate{R_2} & \cdots& &\gate{R_{n-1}}&\gate{R_{n}}&\qw&\qw&\qw&\qw&\qw&\cdots& &\qw&\qw&\qw\\
 &\qw & \qw & \ctrl{-1} & \cdots& &\qw&\qw&\gate{H} & \cdots& &\gate{R_{n-2}}&\gate{R_{n-1}}&\cdots& &\qw&\qw&\qw\\
 &\vdots &  & \vdots & & & & & &\vdots & & & & & & & &\\
 &\qw & \qw & \qw & \qw&\qw&\ctrl{-3}&\qw&\qw & \cdots& &\ctrl{-2}&\qw&\cdots& &\gate{H}&\gate{R_2}&\qw\\
 &\qw & \qw & \qw & \qw&\qw&\qw&\ctrl{-4}&\qw & \cdots& &\qw&\ctrl{-3}&\cdots& &\qw&\ctrl{-1}&\gate{H}
}
}
\caption{Circuit for the quantum Fourier transform}
\label{qc:qft}
\end{figure}


The inverse of a circuit is determined by traversing the circuit 
from right to left. This process changes the original quantum Fourier 
circuit to its inverse as  shown in 
\vref{qc:inverseqft}. The \lqpl{} code for the 
inverse quantum Fourier transform is in the
appendix, \vrefrange{fig:inverseqft}{fig:inverserotate}.
\begin{figure}[htbp]
\centerline{%
\Qcircuit @C=1em @R=.7em {
 &\qw & \qw & \qw & \cdots& &\qw&\qw&\qw&\qw&\qw&\gate{R_{n}^{-1}}&\gate{R_{n-1}^{-1}}&\cdots& &\gate{R_2^{-1}}&\gate{H}&\qw\\
 &\qw & \qw & \qw& \cdots & &\gate{R_{n-1}^{-1}}&\gate{R_{n-2}^{-1}}&\cdots& &\gate{H}&\qw&\qw& \cdots& &\ctrl{-1}&\qw&\qw\\
 &\vdots & & & \vdots & & & &\vdots& & & & & &  & &\\
 &\qw &\gate{R_2^{-1}}&\gate{H} & \cdots & & \qw & \ctrl{-2} & \cdots& &\qw&\qw&\ctrl{-3}&\qw  &\qw&\qw&\qw  &\qw\\
 &\gate{H}&\ctrl{-1}&\qw & \cdots & &\ctrl{-3} & \qw & \cdots& &\qw&\ctrl{-4}&\qw&\qw  &\qw&\qw&\qw  &\qw
}
}
\caption{Circuit for the inverse quantum Fourier transform}
\label{qc:inverseqft}
\end{figure}
