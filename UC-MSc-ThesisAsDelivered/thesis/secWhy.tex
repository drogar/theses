\section{Why a quantum programming language?}\label{sec:why}
Currently, it is not  clear that there will ever be a quantum computer
having a number of \qbits{} comparable to the number of \bits{} available 
on classical computers. Thus, it is reasonable to ask what is the
point of a quantum programming language.
 The compiler for \lqpl{} presented in this thesis
targets a virtual machine --- the quantum stack machine, which
is implemented with a classical language on a classical computer. Given
that, it is highly likely that any significant quantum algorithm written in 
\lqpl{} will be woefully inefficient.


However, there \emph{are} many reasons to create and use a quantum programming
language.

\paragraph{Theory of algorithms.} The current understanding of the 
limits of practical computability has led researchers and practitioners
to probabilistic (and quantum) algorithms to solve problems. This has 
increased the understanding of those algorithms and led back to
classical algorithms. An example of this is the recent 
polynomial time algorithm for primality testing in \cite{agrawal04:primes}.

Quantum algorithms subsume probabilistic algorithms in that
providing language support for quantum computing also gives probabilistic 
support. A simple example of this can be seen in the program to 
generate a coin toss given in \vref{fig:defsec:coin}.

\paragraph{Quantum algorithm experimentation.}
Thinking about quantum algorithms is enhanced by having 
a high level way  of expressing these algorithms.
 A high level language such as \lqpl{}
allows the researcher or practitioner to design an algorithm at an altogether 
different level than the standard \bits{} and \qbits{} of quantum circuits.
A simulator and virtual machine as provided by this 
thesis allows experimentation with quantum algorithms, allowing a
broader exploration of the field.

\paragraph{Quantum computer design.}
The thesis presents a novel view of quantum computation using a quantum 
stack machine. This suggests a way of organizing
 quantum computation with a  quantum stack machine as the central 
element. This may stimulate others to consider how such a machine can be 
realized efficiently.