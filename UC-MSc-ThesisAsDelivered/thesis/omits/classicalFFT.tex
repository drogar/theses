For comparison, we first review the algorithm for the classical
fast Fourier transform. (FFT) \marginnote{Is this a good idea or too much
digression?}

The FFT allows one to compute the \emph{Discrete Fourier Transform} -- 
$\mbox{DFT}_\omega$, a linear map from $\C^n \rightarrow \C^n$
which maps a polynomial $f$ to its evaluation at powers of $\omega$ which 
is a primitive $n^{th}$ root of unity. ($f$ is identified with the 
$n-$tuple of its coefficients for the purposes of the map). The algorithm is
presented in \vref{fig:classicalfft}.

\begin{figure}
\begin{description}
\item{Calling:} $r_k = \mbox{DFT}_\omega(n, [\omega, \ldots, \omega^{n-1}], [f_0, f_1, \ldots, f_{n-1}])$
\item{Inputs:} \begin{itemize}
\item{} The powers of $\omega$, where $\omega$ is a primitive  $n^{th}$ root of unity.
\item{} The value of $n$.
\item{} The coefficients of $f=f_0+f_1x+f_2x^2+\cdots+f_{n-1}x^{n-1}$.
\end{itemize}
\item{Output: } $(f(1), f(\omega), \ldots , f(\omega^{n-1})$.
\item{Method: } \begin{enumerate}
\item{} if $n=1$, return $f_0$. (Base of recursion).
\item{} Set $g=\sum_{0\le i \le n/2}(f_i + f_{i+n/2})x^i$
\item{} Recursively call: $gr_k = \mbox{DFT}_\omega(n/2, [\omega^2, \ldots, (\omega^2)^{n-1/2}], [g_0, \ldots, g_{n/2-1}])$
\item{} Set $h=\sum_{0\le i \le n/2}(f_i - f_{i+n/2})\omega^i x^i$
\item{} Recursively call: $hr_k = \mbox{DFT}_\omega(n/2, [\omega^2, \ldots, (\omega^2)^{n-1/2}], [h_0, \ldots, h_{n/2-1}])$
\item{} return $(gr_0, hr_0, gr_1, hr_1, \ldots, gr_{n/2}, hr_{n/2})$
\end{enumerate}
\end{description}
\caption{Algorithm for classical FFT.}\label{fig:classicalfft}
\end{figure}

The FFT algorithm can be used to reduce the time of polynomial multiplication
from \BigO{n^2} to \BigO{n\log n}. 
