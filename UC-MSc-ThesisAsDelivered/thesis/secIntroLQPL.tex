\lstset{style=inlinqpl}
\section{Introduction to \lqpl}\label{sec:introlqpl}
This chapter  presents an overview of the linear quantum
program language. Explanations of \lqpl{} programs, statements, 
and expressions are given. The explanations are done
using expository presentation with  many short examples to
illustrate relevant points.


\lqpl{} is  a language for experimenting with  quantum
algorithms. The language provides an expressive syntax for creating functions 
and working with different datatypes. \lqpl{} has \qbit{}s as first class
citizens of the language, together with quantum control. 
Classical operations and classical control are also available to 
work with classical data.

\subsection{Linearity of \lqpl}\label{subsec:lqpllinearity}
The language \lqpl{} treats all quantum variables as \emph{linear}. This 
means that any variable \emph{may only be used once}. 
The primary reason for implementing this is the underlying aspect of 
linearity of quantum systems, as exemplified by 
 the \emph{no-duplication} rule which must be respected at all times. 
This allows us to provide
compile-time checking that enforces this rule.

The compiler and language do
 provide ways to ``ease the burden'' of linear thinking. 
For example, function calls (\vref{subsec:functioncalls}) 
provide a specialized syntax for variables
which are both input and output to a function. The classical 
use statements (\vref{subsec:usestatements}) place values on to the
classical stack where the values may be used multiple times.

\begin{figure}[htbp]
\lstinputlisting[style=linqpl]{examplecode/LengthList.qpl}
\caption{\lqpl{} code to return the length of the list}\label{fig:lenSQPL}
\end{figure}

An example illustrating linearity is given in \vref{fig:lenSQPL}. In
line \ref{line:lengthlist:len}, the function \inlqpl{len} is defined
as taking one argument of type \inlqpl{List (a)} and returning
a variable of type \inlqpl{Int}. The input only argument, \inlqpl{listIn},
\emph{must be destroyed in the function}. When the
case statement refers to \inlqpl{listIn}, the argument is 
destroyed, fulfilling the requirement of the function to do so.



