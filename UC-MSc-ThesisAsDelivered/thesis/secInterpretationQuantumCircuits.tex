\section{Example --- putting it all together on teleport}\label{subsec:semanticsOfQCD}

Circuit diagrams are interpreted in the algebra of
 quantum stacks. The preceding sections have given a translation
of quantum circuits to \lqpl{} programs and an operational semantics of 
\lqpl{} programs. From this, we have an operational
 semantics of quantum circuits.


\subsection{Translation of teleport to basic \lqpl}

The quantum circuit for teleportation makes a 
variety of assumptions, including
that the two starting \qbits{} are placed in an EPR state and that the
target \qbit{} is in some unknown state. 
\Vref{fig:semantics:quantumTeleportation} shows
the standard teleport circuit assuming 
that $\nu$ is in some unknown state, but that
the \qbits{}
\emph{alice} and \emph{bob} still need to be placed into an EPR state.


\begin{figure}[htbp]
\centerline{%
\Qcircuit @C=1em @R=.7em {
\lstick{\nu} & \qw&\qw &\ctrl{1} & \gate{H} & \measureD{M_1} & \cw &  \control \cw  \cwx[2] \\
\lstick{\mbox{\emph{alice}}=\ket{0}}  &\gate{H}&\ctrl{1}& \targ & \qw & \measureD{M_2} & \control \cw  \cwx[1] \\
\lstick{\mbox{\emph{bob}}=\ket{0}} & \qw&\targ& \qw &\qw & \qw & \gate{X} & \gate{Z} & \qw \\
\lstick{s_0}& \rstick{s_1}& \rstick{s_2} &\rstick{s_3} & \rstick{s_4} &\rstick{s_5} &\rstick{s_6}&\rstick{s_7} }}
\caption{Preparation and quantum teleportation}
\label{fig:semantics:quantumTeleportation}
\end{figure}


The translation of this circuit to a basic \lqpl{} program is 
straightforward and is given in \vref{fig:translationofteleport}.

\begin{figure}
\begin{align}
 &\text{\emph{alice}} = \ket{0};\notag  \\
   &\text{\emph{bob}} = \ket{0}; \tag{$\qins_{s_0}$} \\
  &Had\ {\text{\emph{alice}}} ;\tag{$\qins_{s_1}$} \\
  &\qcontrolled{(Not\ \text{\emph{bob}})}{\text{\emph{alice}}} ;\tag{$\qins_{s_2}$} \\
 &\qcontrolled{Not\ \text{\emph{alice}}}{\nu}; \tag{$\qins_{s_3}$}\\
   &Had\ \nu;\tag{$\qins_{s_4}$}\\
 &\qmeasnob{\nu}{\nu_b = 0}{\nu_b =1} ;\notag\\
   &\qmeasnob{\text{\emph{alice}}}{\text{\emph{alice}}_b = 0}{\text{\emph{alice}}_b=1};\tag{$\qins_{s_5}$}\\
  &\qcontrolled{Not\ \text{\emph{bob}}}{\text{\emph{alice}}_b} ;\notag\\
   &\text{disc \emph{alice}}_b;\tag{$\qins_{s_6}$} \\
 &\qcontrolled{Z\ \text{\emph{bob}}}{\nu_b}; \notag\\
   &\text{disc }\nu_b \tag{$\qins_{s_7}$}
\end{align}
\caption{Basic \lqpl{} program for teleport}\label{fig:translationofteleport}
\end{figure}

\newcommand{\nustack}[1]{\qsqbitUp{\nu}{#1 \alpha\overline{\alpha}}{#1 \alpha\overline{\beta}}
                       {#1 \beta\overline{\alpha}}{#1 \beta\overline{\beta}}{}}


\subsection{Unfolding the operational semantics on teleport}
This section will step through the program, using the operational semantics
to show the changes to the quantum stack at each of the labelled stages.
Input to the algorithm is assumed to be a single \qbit{}, $\nu$, which
is in an unknown state, $\alpha \ket{0} +\beta\ket{1}$. Therefore,
the starting stack is
\[\nustack{}. \]
When the instructions $\qins_{s_0}$ are applied, the stack will change
to
\[
 \qsnodeij{\text{\emph{bob}}}{00}{\qsnodeij{\text{\emph{alice}}}{00}{\nustack{}}{}}{}.
\]
The next instruction, $\qins_{s_1}$, applies the 
Hadamard transform to \emph{alice}. \emph{alice} is
first rotated to the top and then the transformation is applied.
The stack will then be in the state
\[
\qsqbitAllUp{\text{\emph{alice}}}
	 {\qsnodeij{\text{\emph{bob}}}{00}{\nustack{.5\times}}{}}
	 {\qsnodeij{\text{\emph{bob}}}{00}{\nustack{.5\times}}{}}
	 {\qsnodeij{\text{\emph{bob}}}{00}{\nustack{.5\times}}{}}
	 {\qsnodeij{\text{\emph{bob}}}{00}{\nustack{.5\times}}{}}
         {}.
\]

The preparation of EPR state is completed by  doing a \Cnot{}
between \emph{alice} and \emph{bob}. This results in
\[
\qsqbitAllUp{\text{\emph{alice}}}
	 {\qsnodeij{\text{\emph{bob}}}{00}{\nustack{.5\times}}{}}
	 {\qsnodeij{\text{\emph{bob}}}{10}{\nustack{.5\times}}{}}
	 {\qsnodeij{\text{\emph{bob}}}{01}{\nustack{.5\times}}{}}
	 {\qsnodeij{\text{\emph{bob}}}{11}{\nustack{.5\times}}{}}
         {}.
\]

This is followed by the application of
 the \Cnot{} to $\nu$ and \emph{alice}, giving 
\[
\qsqbitAllUp{\nu}
    {\qsqbitUp{\text{\emph{alice}}}
	 {\qsnodeij{\text{\emph{bob}}}{00}{.5\times\alpha\overline{\alpha}}{}}
	 {\qsnodeij{\text{\emph{bob}}}{10}{.5\times\alpha\overline{\alpha}}{}}
	 {\qsnodeij{\text{\emph{bob}}}{01}{.5\times\alpha\overline{\alpha}}{}}
	 {\qsnodeij{\text{\emph{bob}}}{11}{.5\times\alpha\overline{\alpha}}{}}
	{}}
{\qsqbitUp{\text{\emph{alice}}}
	 {\qsnodeij{\text{\emph{bob}}}{01}{.5\times\alpha\overline{\beta}}{}}
	 {\qsnodeij{\text{\emph{bob}}}{11}{.5\times\alpha\overline{\beta}}{}}
	 {\qsnodeij{\text{\emph{bob}}}{00}{.5\times\alpha\overline{\beta}}{}}
	 {\qsnodeij{\text{\emph{bob}}}{10}{.5\times\alpha\overline{\beta}}{}}
	{}}
{\qsqbitUp{\text{\emph{alice}}}
	 {\qsnodeij{\text{\emph{bob}}}{00}{.5\times\beta\overline{\alpha}}{}}
	 {\qsnodeij{\text{\emph{bob}}}{10}{.5\times\beta\overline{\alpha}}{}}
	 {\qsnodeij{\text{\emph{bob}}}{01}{.5\times\beta\overline{\alpha}}{}}
	 {\qsnodeij{\text{\emph{bob}}}{11}{.5\times\beta\overline{\alpha}}{}}
	{}}
{\qsqbitUp{\text{\emph{alice}}}
	 {\qsnodeij{\text{\emph{bob}}}{11}{.5\times\beta\overline{\beta}}{}}
	 {\qsnodeij{\text{\emph{bob}}}{01}{.5\times\beta\overline{\beta}}{}}
	 {\qsnodeij{\text{\emph{bob}}}{10}{.5\times\beta\overline{\beta}}{}}
	 {\qsnodeij{\text{\emph{bob}}}{00}{.5\times\beta\overline{\beta}}{}}
	{}}{}.
\]

The \Had{} transform is now applied to $\nu$. Unfortunately, the notation
used to this point is somewhat too verbose to show the complete 
resulting quantum stack. It begins with:
\begin{align*}
\qsnodeij{\nu}{00} {\qsnodeij{\text{\emph{alice}}}{00}
                   {\qsqbitUp{\text{\emph{bob}}}
                             {\frac{\alpha\overline{\alpha}}{4}} 
		             {\frac{\alpha\overline{\beta}}{4}} 
                             {\frac{\beta\overline{\alpha}}{4}} 
                             {\frac{\beta\overline{\beta}}{4}}{}\ldots}{}\ldots}{}\\
\vdots
\end{align*}

In the next stage, \emph{alice} is measured and a \nottr{} transform is 
applied to \emph{bob} depending on the result of the measure. 
The bit $\text{\emph{alice}}_b$ is then discarded, resulting in
\[\qsqbitAllUp{\nu}
{\qsqbitUp{\text{\emph{bob}}}{\alpha\overline{\alpha}/2}{\alpha\overline{\beta}/2}{\beta\overline{\alpha}/2}{\beta\overline{\beta}/2}{}}
{\qsqbitUp{\text{\emph{bob}}}{\alpha\overline{\alpha}/2}{-\alpha\overline{\beta}/2}{\beta\overline{\alpha}/2}{-\beta\overline{\beta}/2}{}}
{\qsqbitUp{\text{\emph{bob}}}{\alpha\overline{\alpha}/2}{\alpha\overline{\beta}/2}{-\beta\overline{\alpha}/2}{-\beta\overline{\beta}/2}{}}
{\qsqbitUp{\text{\emph{bob}}}{\alpha\overline{\alpha}/2}{-\alpha\overline{\beta}/2}{-\beta\overline{\alpha}/2}{\beta\overline{\beta}/2}{}}{}.
\]

Then, $\nu$ is measured and a $RhoZ$ transform is applied when the resulting 
\bit{} is 1. The \bit{} $\nu_b$ is then discarded. The
resulting quantum stack is
\[\qsqbitUp{\text{\emph{bob}}}
           {\alpha\overline{\alpha}}{\alpha\overline{\beta}}
           {\beta\overline{\alpha}}{\beta\overline{\beta}}{},
\]
which shows \emph{bob} has been changed to the same state $\nu$ was
in at the start of the program.