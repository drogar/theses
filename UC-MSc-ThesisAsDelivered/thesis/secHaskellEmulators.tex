\section{Haskell based quantum computation emulators}
\label{sec:haskellemulators}
\subsection{Functional quantum programming}
Shin-Cheng Mu and Richard Bird wrote 
\cite{MuBird2001:Functionalquantumprogramming} in 2001, which
explores ways to write quantum algorithms in Haskell. In this paper,
$n$ \qbits{} are represented as a list of $2^n$ values. The paper refers
to these as \emph{quregs}.

The paper provides Haskell code to perform unitary operations and measurements.

Two examples of quantum algorithms; the Deutsch-Jozsa and Grover's search
algorithm are provided.

Bird and Mu then make the point that a \emph{qureg} resembles a 
Haskell monad, in that a \emph{join} function and \emph{return} 
function may be defined on \emph{quregs}. This resemblance is used 
to restate the Deutsch-Jozsa algorithm in a monadic format. The importance
of this approach is that it encourages programmers to consider the 
algorithm used, rather than concentrating on the multitude of possible
values a \emph{qureg} may assume.

The approach given in \cite{MuBird2001:Functionalquantumprogramming}
 is not used by the simulator introduced in this
thesis.

\subsection{Modelling quantum computation in Haskell}
In 2003, Amr Sabry \cite{sabry03:qcinH} described a way
to emulate quantum computation in Haskell.

\subsubsection{Representation of quantum values}
In the paper a \qbit{} is represented as a map from \emph{basis} values 
to values in \complex. The \emph{basis}
 is a list of elements that can be used as
an orthogonal basis of a vector space. Examples presented in the paper include
\inlhskl{[False,True]}, \inlhskl{[Up,Down]} and \inlhskl{[Red,}
 \inlhskl{Green,} \inlhskl{Blue]}, 
where the latter two
are made up of constructors for some of the 
new types introduced in the paper. 

This allows a programmer to represent any \qbit{} once the basis is chosen. 
For example, a ``False'' \qbit{} is represented as $\{False: 1.0+0.0i\}$ and
an indeterminate ``False - True'' \qbit{} may be represented as
\[\{False:\frac{1}{\sqrt{2}}+0.0i;True:\frac{1}{\sqrt{2}}+0.0i\}\]
The Haskell type used for this is \inlhskl{type\ QV\ a = FiniteMap\ a\ (Complex Double)}.

\subsubsection{Entanglement}
The concept of \emph{entangled} \qbits{} is handled by extending the 
basis to pairs of basis elements. Given a \emph{basis} definition
for elements of type $a$, extend this to a definition of a \emph{basis}
for  elements of 
type \inlhskl{(a,a)} by using the product type.

As an example, the standard EPR pair under the \inlhskl{(Bool, Bool)}
 basis would be
represented as
\[\{(False,False):\frac{1}{\sqrt{2}}+0.0i;(True,True):\frac{1}{\sqrt{2}}+0.0i\}.\]
\subsubsection{Unitary operations}
Unitary operations are defined as functions of 
type~\inlhskl{QV Bool -> QV Bool}. The 
definition of the \Had{} gate is given as 

{\begin{singlespace}
\begin{lstlisting}[style=hskl]
had v = let a = pr v False
            b = pr v True
	in qv [(False, a+b), (True, a-b)]
\end{lstlisting}
\end{singlespace}
}

In the above \inlhskl{pr} retrieves the value of a 
basis element from a \qbit{} and 
\inlhskl{qv} is a smart constructor. \inlhskl{qv} takes a 
list of \inlhskl{(basis, Complex Double)} pairs
and turns the list into an element of type \inlhskl{QV basis}.

It is possible to lift operations on standard classical values to ones
on quantum values \emph{provided the operation is reversible}.  The paper 
provides Haskell functions
 for lifting of the base case where the classical
function is of type \inlhskl{(a->b)}.
\subsubsection{Measurement}
Measurements of quantum values are done by  ``collapsing'' the 
measured value. The code as supplied in the paper requires that measurable 
quantum values be stored in \inlhskl{IORef} variables. \inlhskl{IORef}
variables are the Haskell idiom for mutable variables. The measurement
will generate a random value to pick one of the basis elements (\inlhskl{a})
 as the result of the measure and then update the variable so 
that it is now:
\[\{a:1.0+0.0i\}\]
This also ensures all future measures will return the same value.
\subsubsection{Multiple \qbits}
Operations on multiple \qbits{} are discussed next in the paper and the
issues with isolating specific sets of \qbits{} to work with are addressed.
In  Dr. Sabry's simulator, \emph{adaptor}s are required to 
restructure the form of multiple \qbits. A specific adaptor would,
for example, take a quantum value of type \inlhskl{QV (a,(b,c))} to
one of \inlhskl{QV ((a,c),b)}. This then allows the application of 
a unitary transformation to the first and third \qbits{} of the first type.
Dr. Sabry assumes that it will be possible to generate any needed adaptors
rather than writing code for each individual one.
Candidates for doing this would be template Haskell or a pre-compiler that 
creates adaptors as required.
\subsection{Comparison with and contrast to the quantum stack machine}
Dr. Sabry's paper provided the inspiration for the use of a 
\inlhskl{class Basis} and a Haskell 
\inlhskl{Map} from basis elements to values
 when defining the simulator presented in this thesis. It is used in a 
slightly different way in that the elements of a \qbit{}'s density
matrix are indexed by pairs of basis elements as opposed to a 
single basis element. 

Classical data and constructed datatypes
 in the quantum stack machine also use 
the idea of a Haskell \inlhskl{Map} from their 
``basis'' elements to values. The basis elements for classical data
are the integers and Booleans. The basis elements for a constructed data
type are the constructors together with bound nodes.

In contrast to Dr. Sabry's approach, multiple
\qbit{}s are handled by increasing the depth of the tree, classical and
quantum data are easily mixed and measurement is approached in a very 
different way.

One of the largest differences is the  approach to handling
quantum values. The simulator presented in 
this thesis (QSM)
keeps all the probabilities of values as the program proceeds. This allows 
us to perform the quantum computation just once, and then pick random 
values as needed to evaluate different results. In Dr. Sabry's presentation,
the actual quantum algorithm needs to be re-run to determine
the different possible results.