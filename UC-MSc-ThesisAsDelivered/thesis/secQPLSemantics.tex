\section{Semantics of QPL}\label{subsec:semanticsOfQPL}
Each statement in QPL is to be viewed as a function from tuples of
matrices to tuples of matrices. For each \bit{} in the programs' context
there will be two tuple elements. For each \qbit{} within a \bit{}'s 
sphere of control the size of the 
matrices in the corresponding tuple position will grow exponentially. 
Given $n$ \bit{}s and $m_i$ \qbit{}s for each \bit,
 the context will be
 $\C^{m_1 \times m_1} \times \cdots \times \C^{m_{2n} \times m_{2n}}$. 
As explained in Dr. Selinger's paper, \bit{} values 
correspond to \emph{classical control}. For each distinct bit value, there may
be a different number of \qbit{}s. Consider the program
fragment in \vref{fig:fragmentShowingDifferentBits}.

\begin{figure}[htbp]
\begin{qplcodethesis}
new bit b:= 0;
new qbit q := 0;
q *= Had;
measure q then
   b:=0
 else
   b:= 1;

if b then
  new qbit r = 0;
  new qbit s = 0;
  r *= Had; \label{progline:manymatrices}
  r,s *= Cnot
 else
  skip;\label{progline:fewmatrics}
\end{qplcodethesis}
\caption{Fragment of QPL code showing different contexts}
\label{fig:fragmentShowingDifferentBits}
\end{figure}

Measurement in QPL is not destructive, therefore the \qbit{} \texttt{q} 
remains in scope throughout the program.
In the fragment at line \ref{progline:manymatrices}, the 
context has three \qbit{}s: (\texttt{q,r,s}). The \bit{} $b$ is not available
to the program inside an \texttt{if}. At 
line \ref{progline:fewmatrics} only one \qbit{} (\texttt{q}) is in context.
The context after completion of the $if$ statement is the merge of these
two contexts and would therefore be:
\[ \C^{8\times 8} \times \C^{2 \times 2}.\]


The system can be regarded as being in $\C^{8\times8}$ when the
\bit{} $b$ is $0$, and in  $\C^{2 \times 2}$ when it is $1$.

Each statement defined for QPL has a realization in these contexts. 

\subsection{Formal semantics for QPL}\label{subsec:formalsemanticsqpl}
Dr. Selinger presents a categorical semantics for QPL as the
category $Q$ of superoperators over signatures. 
I will give an outline of the major points of this semantics. This 
section uses concepts from linear algebra that are introduced later in this
thesis in \vref{sec:linearalgebra}.

\subsubsection{Signatures}
\emph{Signatures} are the objects of $Q$. Each signature is a 
list of non-zero natural numbers:\[\ell=[n_1,\ldots,n_m].\]
 Each of these
signatures is identified with a vector space $V_\ell$ over the complex
numbers \[V_\ell = \complex^{n_1\times n_1}\times\cdots\times\complex^{n_m\times n_m}.\]
This vector space is tuples of matrices over the complex numbers. 


Specific objects in $Q$ are given names:
\begin{align*}
\bit&= [1,1] &
\qbit&=[2]\\
I&=[1] &
0&=[\ ]
\end{align*}


\subsubsection{QPL realizations in $Q$}
Given the details of the category $Q$ it is now
straightforward to associate specific maps in $Q$ (superoperators) with
specific QPL statements. Dr. Selinger gives a complete list. 
\Vref{tab:instructionsforQPL} repeats a few of these.

{\begin{singlespace}
\begin{table}[htbp]
\begin{tabular}{|>{$}p{.7in}<{$}>{$}p{.95in}<{$}>{$}p{1in}<{$}>{$}l<{$}|}
\hline
\mathrm{\textbf{QPL}}&\mathrm{\textbf{Map}}&\mathrm{\textbf{Typing}}&\mathrm{\textbf{Result}}\\
\hline
b := 0 & set_0: & \bit\to\bit&set_0(a,b) = (a+b,0)\\
b := 1 & set_1: & \bit\to\bit&set_1(a,b) = (0,a+b)\\
merge & merge: & \bit\to I&merge(a,b) = a+b\\
\hline
\end{tabular}
\caption{Instructions for QPL and their meaning}\label{tab:instructionsforQPL}
\end{table}
\end{singlespace}
}

