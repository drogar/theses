\section{Miscellaneous examples}\label{sec:miscExamples}
\subsection{Quantum teleportation function}\label{subsubsec:quantumTeleportationExample}
The \lqpl{} program shown in \vref{fig:stackTeleportation}
 is an implementation of a function that will accomplish 
quantum teleportation as per the circuit shown previously in 
\vref{qc:quantumTeleportation}. It also provides a separate function
to place two \qbits{} into the EPR state.

Note that the teleport function, similarly to the circuit,
 does not check the precondition that \qbit{}s $a$ and $b$ are in the
EPR state, which is required to actually have teleportation work.

\begin{figure}[htbp]
\lstinputlisting[style=linqpl]{examplecode/teleport.qpl}
\caption{\lqpl{} code for a teleport routine}\label{fig:stackTeleportation}
\end{figure}

\subsection{Quantum Fourier transform}
The \lqpl{} program to implement the quantum Fourier transform in 
\vref{fig:qft} uses
two recursive routines, \inlqpl{qft} and \inlqpl{rotate}. These
functions assume the \qbits{} to transform are in a \inlqpl{List}.

The routine \inlqpl{qft} first applies the \Had{} transform to the
\qbit{} at the head of the list, then uses the \inlqpl{rotate} routine
to recursively apply the correct $Rot$ transforms controlled by the
other \qbits{} in the list. \inlqpl{qft} then recursively calls itself on
the remaining \qbits{} in the list.

\begin{figure}[htbp]
\lstinputlisting[style=linqpl]{examplecode/qft.qpl}
\caption{\lqpl{} code for a quantum Fourier transform}\label{fig:qft}
\end{figure}

\subsection{Deutsch-Jozsa algorithm}\label{appsubsec:djalgorithm}
The \lqpl{} program to implement the Deutsch-Jozsa algorithm is in 
\vref{fig:dj} with supporting routines in \ref{fig:initlist}, 
\ref{fig:addnzerps}, \ref{fig:measureinps}. The 
\inlqpl{hadList} function is defined in \vref{fig:hadList}.

The algorithm decides if a function is balanced or constant on 
$n$ \bits. This implementation requires supplying the number of
\bits{} / \qbits{} used by the function, so that the input can 
be prepared. Additionally, it currently requires an extension to the 
language where a $n+1-\qbit$ function can be defined from
a $n-\bit$ function. Assuming the existance of this planned extension,
the rest of the algorithm is straight-forward.

The function \inlqpl{dj} creates an input list for the function, 
applies the Hadamard transform to all the elements of that list
and  applies
the candidate function. When that is completed, the initial segment
of the list is transformed again by Hadamard and then measured. 


\begin{figure}[htbp]
\lstinputlisting[style=linqpl]{examplecode/dj.qpl}
\caption{\lqpl{} code for the Deutsch-Jozsa algorithm}\label{fig:dj}
\end{figure}

The function \inlqpl{addNZeroqbs} creates a list of \qbits{} when given
a length and the last value. Assuming the parameters passed to the 
function were $3$ and \ket{1}, this would return the list:
\[[\ket{0},\ket{0},\ket{0},\ket{1}]\]
\begin{figure}[htbp]
\lstinputlisting[style=linqpl]{examplecode/addnzeros.qpl}
\caption{\lqpl{} code to prepend $n$ \ket{0}'s to a \qbit}\label{fig:addnzerps}
\end{figure}

The \inlqpl{initList} function removes the last element of a list.
\begin{figure}[htbp]
\lstinputlisting[style=linqpl]{examplecode/initList.qpl}
\caption{\lqpl{} code accessing initial part of list}\label{fig:initlist}
\end{figure}

The \inlqpl{measureInputs} function recursively measures the 
\qbits{} in a list. If any of them measure to $1$, it returns the 
value \inlqpl{Balanced}. If all of them measure to $0$, it returns
the value \inlqpl{Constant}.

\begin{figure}[htbp]
\lstinputlisting[style=linqpl]{examplecode/measureInps.qpl}
\caption{\lqpl{} code to measure a list of  \qbits}\label{fig:measureinps}
\end{figure}

\subsection{Quantum adder}\label{appsubsec:quantumadder}
This section provides subroutines that perform \emph{carry-save}
arithmetic on \qbits{}. The algorithm is from
\cite{Vedral:1995ga}. The \inlqpl{carry} and \inlqpl{sum}
routines in \vref{fig:carrysum} function as gates on four \qbits{}
and three \qbits{} respectively.


\begin{figure}[htbp]
\lstinputlisting[style=linqpl]{../testdata/arithmetic/carrysumgates.qpl}
\caption{\lqpl{} code to implement carry and sum gates}\label{fig:carrysum}
\end{figure}

The addition algorithm adds two lists of \qbits{} and an input
carried \qbit. The first list
is unchanged by the algorithm and the second list is changed to hold the 
sum of the lists, as shown in \vref{fig:quantumadder}.

\begin{figure}[htbp]
\lstinputlisting[style=linqpl]{../testdata/arithmetic/adder.qpl}
\caption{\lqpl{} code to add two lists of \qbits}\label{fig:quantumadder}
\end{figure}

The program proceeds down the lists $A$ and $B$ of input \qbits, first applying
the \inlqpl{carry} to the input carried \qbit, the heads of $A$ and $B$ and
a new zeroed \qbit, $c_1$. When the ends of the lists are reached, a controlled
not and the \inlqpl{sum} are applied. The output $A+B$ list is then started
with $c_1$. Otherwise, the program recurses, calling itself
with $c_1$ and the tails of the input lists. When that returns, \inlqpl{carry}
and \inlqpl{sum} are applied, the results are ``Consed'' to the existing
tails of the lists, $c_1$ is discarded and the program returns.