\chapter{Future Work}\label{chap:futurework}
As with any piece of research there are many avenues of exploration 
still left open with respect to \lqpl{} and its quantum stack machine.

\section{Language extensions}\label{sec:languageextensions}
The programming language \lqpl{}, while currently interesting and
useful, could use additional features to make it a fully functional and
powerful tool for developing quantum algorithms.

\subsection{Refinements to the type system}\label{subsec:typesystem}
In addition to the current \inlqpl{qdata} construction, a corresponding
construction for classical data creation could be added. This 
declared classical data would be held in a classical node on the
quantum stack, but could be moved back and forth to the classical stack.

A type \emph{aliasing} declaration and potentially a \emph{class} system
to allow for closed types may also be useful.

\subsection{Transform definition}\label{subsec:transformdefinition}
Currently the language has a finite set of built-in transforms, two of
which are parametrized by integer values. However, one common feature of 
quantum algorithms seems to be the application of generic reversible 
classical computation on \bit{}s as a unitary transformation on \qbit{}s.

This is typically done by using the equivalences:
\[
\genfrac{}{}{}{}{\bit^n \rightarrow \bit}{%
\genfrac{}{}{}{0}{\bit^{n+1} \rightarrow \bit^{n+1}}{%
\qbit^{n+1} \rightarrow \qbit^{n+1}}}
\]

We would like to add supporting syntax and semantics to the language to 
accomplish this, without sacrificing the compile time type safety.

\subsection{Input/Output}\label{subsec:inputoutput}
The ability to allow a program to read in values and print out results, 
likely outside of the bounds of any \qbit{} manipulation, is a necessity 
for future development.

\subsection{Miscellaneous enhancements}\label{subsec:miscellaneousenhancements}
Additional base types in the language, such as characters and floating
point numbers, would be useful in some algorithms and in I/O.
