\section{Basic \protect{\lqpl} statements}\label{sec:sembasiclqpl}
Quantum circuits can be translated into basic \lqpl{} statements, which
 form a core fragment of the \lqpl{}
language.  Later sections in this chapter will provide the semantics
for these statements and progressively build up to 
providing a semantics of \lqpl.

The language is introduced in a series of judgements which give
the construction of valid \lqpl{} statements.  Judgements have
the following form:
\[\Gamma ; \Gamma' \ientail \qins \]
where $\Gamma$ is the context input to the 
statement $\qins$ and $\Gamma'$ is the context
after it is completed.

Basic \lqpl{} has seven distinct statements: 
\emph{identity, new, assign, discard, measure, control} and \emph{unitary}. 
Statements may also be composed, creating a \emph{block} of 
statements which may be used wherever a statement is required.


The  judgements for the creation of basic \lqpl{} statements are given in 
\vref{fig:judgementscreatestatements}.

The \emph{identity} statement is a do-nothing statement that does
not change the stack or context. The statements \emph{new} and
\emph{assign} share a similar syntax and create new variables in the output
context. \emph{Assign} will remove an existing variable, while  \emph{new}
just creates the new variable. \emph{Discard} removes a variable from the
input context. The \emph{measure} statement performs a destructive 
measure of a \qbit, while applying different sets of dependent statements
contingent on the result of the measure. \emph{Control} modifies the
execution of dependent statements, contingent upon the value of the
attached control variable. The final statement, \emph{unitary}, applies
unitary transformations to a \qbit.

\begin{figure}[htbp]
\[
\begin{gathered}
\infer[{\raisebox{1ex}{identity}}]
   {\Gamma ; \Gamma \ientail \varepsilon}
   {}\\
\\
\infer[{\raisebox{2ex}{new \bit}}]
   {\Gamma ; (x{::}\bit),\Gamma \ientail x = i}
   {i\in\{0,1\}}\qquad
\infer[{\raisebox{2ex}{new \qbit}}]
   {\Gamma ; (q{::}\qbit),\Gamma \ientail x = \ket{k}}
   {k\in\{0,1\}}\\ 
\\
\infer[{\raisebox{2ex}{assign}}]
   {(x{::}\tau),\Gamma;(y{::}\tau),\Gamma\ientail x = y}
   {}\qquad
\infer[{\raisebox{2ex}{discard}}]
   {(x{::}\tau),\Gamma ; \Gamma\ientail \text{disc }x}
   {}\\ 
\\
\infer[{\raisebox{1ex}{measure}}]
   {(x{::}\qbit),\Gamma; \Gamma'\ientail \text{meas }x\ \ket{0} =>\qins_0\ 
    \ket{1} => \qins_1}
   {\Gamma;\Gamma'\ientail \qins_0 & \Gamma; \Gamma' \ientail \qins_1}\\
\\
\infer[{\raisebox{2ex}{control}}]
   {(z{::}\tau),\Gamma;(z{::}\tau),\Gamma'\ientail \qins <= z}
   {\Gamma; \Gamma' \ientail \qins}\qquad
\infer[{\raisebox{1ex}{compose}}]
   {\Gamma; \Gamma'' \ientail (\qins_1 ; \qins_2)}
   {\Gamma;\Gamma'\ientail \qins_1 & \Gamma'; \Gamma'' \ientail \qins_2}\\
\\
\infer[{\raisebox{2ex}{tensor compose}}]
    {\Gamma_1 \Gamma_2; \Gamma_1' \Gamma_2' \ientail (\qins_1 ;; \qins_2)}
   {\Gamma_1;\Gamma_1'\ientail \qins_1 & \Gamma_2; \Gamma_2' \ientail \qins_2}\\
\\
\infer[{\raisebox{2ex}{unitary}}]
   {(x{::}\qbit),\Gamma;(x{::}\qbit),\Gamma\ientail U\ x}{}
\end{gathered}
\]
\caption[Judgements for statement creation]{Judgements for creation of basic \lqpl{} statements.}\label{fig:judgementscreatestatements}
\end{figure}






\subsection{Examples of basic \protect{\lqpl} programs}
The examples in this sub-section use the statements as introduced
in \ref{fig:judgementscreatestatements} to give some example programs.

\paragraph{Program to swap two \qbits.}
This program will swap two \qbits{} by successive renaming. Both the
input and output contexts have two \qbits.

$q,v::\qbit{} ;\ v,q::\qbit{} \ientail$
\begin{Verbatim}
      w = q;
      q = v;
      v = w;
\end{Verbatim}


\paragraph{Program to do a coinflip.}
The program has an empty input context and a single \bit{} in its 
output context. At the end, the \bit{} \terminalio{b} will be $0$ with
probability $.5$ and $1$ with an equal probability.


$\emptyset ;\ b::\bit{} \ientail$
\begin{Verbatim}
      q = |0>;
      Had q;
      meas q
       |0> => {b = 0}
       |1> => {b = 1}
\end{Verbatim}


\paragraph{Program to entangle two \qbits{}.} This program places
two input \qbits{} into an EPR state.


$q, r::\qbit{} ;\  q, r::\qbit{}  \ientail$
\begin{Verbatim}
      Had q;
      Not r <= q;
\end{Verbatim}


