\section{Operational semantics for the language \lqpl}
\label{sec:operationalsemanticslqpl}
In this section, I provide an operational semantics of \lqpl.
\subsection{Syntactic sets}\label{subsec:opsyntacticsets}
\lqpl{} has a variety of different syntactic sets as it is a quantum 
language with classical features. Traditional syntactic sets are composed
of unique elements, such as $N$ being the natural numbers. In \lqpl, the
base elements are sets of (value,probability) pairs, where the sum
of the probabilities is less than or equal to one. (A probability sum 
being strictly less than one signifies a chance of non-termination). 
This is required as
quantum operations, when a measure is applied, will create results of 
varying probabilities.

Certain items in the language are uniquely valued, for example, a constant.
In these cases, the constant will evaluate to the number it represents with
$100\%$ probability.

In the following list of syntactic sets, $p$ is always a probability
value between $0$ and $1$ inclusive. The required syntactic sets are:
\begin{itemize}
\item{} \n, the set of sets of $(n,p)$ pairs, where $n$ is an integer.
\item{} \T, the set of sets of $(t,p)$ pairs, 
where $t \in \{TRUE, FALSE\}$.
\item{} \Q, the set of quantum values as a single $2\times2$ matrix. ??????
\item{} \Data, the set of sets of $(d,p)$ pairs, where $d$ is a
defined constructor and associated items. 
\item{} \Qloc, the set of quantum locations.
\item{} \Cloc, the set of classical locations.
\item{} \Aexp, the set of arithmetic expressions.
\item{} \Bexp, the set of Boolean expressions.
\item{} \Cexp, the set of constructor expressions.
\item{} \Stm, the set of statements.
\end{itemize}

Following common practice, various letters will be used to range over these
sets. Sub-scripting and adding of primes to the letter does not change the 
category they belong to.
\begin{itemize}
\item{} $(n,p),(m,p)$ for \n.
\item{} $q,r$ for \Q.
\item{} $(d,p)$ for \Data.
\item{} $Z,Q$ for \Qloc.
\item{} $X,Y$ for \Cloc.
\item{} $a$ for \Aexp.
\item{} $b$ for \Bexp.
\item{} $c$ for \Cexp.
\item{} $e$ for $\Aexp + \Bexp + \Cexp$.
\item{} $s$ for \Stm.
\end{itemize}

The execution of statements in \lqpl, while functional, may be thought of
as a function from a state to the next state. That state is composed of the
disjoint 
union of the elements in \n, \T, \Q{} and \Data. This set is denoted as 
\R. Additionally, \Loc{} will denote the disjoint union of 
\Cloc{} and \Qloc{}. $r$ will range over \R{} and $L$ over \Loc{} 
where required.
The \emph{state} $\Sigma$ of a 
program at a particular point is a function
 $\sigma: \Loc \to\R$.

As operational semantics are an \emph{evaluation} based semantics,
the evaluation of a pair consists of an item in one of our 
expression sets(\Aexp, \Bexp{} and \Cexp)
 and a state to a value in one of the base sets(\n,\T, \Q{} and \Data).
I denote this evaluation triplet by:
\[\eval{e}{\sigma}{r}.\]

\subsection{Operational semantics of arithmetic expressions}\label{subsec:oparithmetic}
All rules that follow are presented in the premise / conclusion format.
\subsubsection{Constants and variables}
\[\infer[\mathrm{numbers}]{\eval{n}{\sigma}{(n,1.0)}}{}\]
\[
\infer[\mathrm{Classical}]{\eval{X}{\sigma}{\sigma(X)}}{}
\qquad
\infer[\mathrm{Quantum}]{\eval{Z}{\sigma}{\sigma(Z)}}{}
\]
\subsubsection{Binary operations}
\[
\infer[{n=n_0+n_1}]{\eval{a_0 + a_1}{\sigma}{(n,p_0 \times p_1)}}
    {\eval{a_0}{\sigma}{(n_0,p_0)}\quad\eval{a_1}{\sigma}{(n_1,p_1)} }
\]
\[
\infer[{n=n_0-n_1}]{\eval{a_0 - a_1}{\sigma}{(n,p_0 \times p_1)}}
    {\eval{a_0}{\sigma}{(n_0,p_0)}\quad\eval{a_1}{\sigma}{(n_1,p_1)} }
\]
\[
\infer[{n=n_0\times n_1}]{\eval{a_0 * a_1}{\sigma}{(n,p_0 \times p_1)}}
    {\eval{a_0}{\sigma}{(n_0,p_0)}\quad\eval{a_1}{\sigma}{(n_1,p_1)} }
\]
\[
\infer[{n=n_0 \div n_1}]{\eval{a_0 / a_1}{\sigma}{(n,p_0 \times p_1)}}
    {\eval{a_0}{\sigma}{(n_0,p_0)}\quad\eval{a_1}{\sigma}{(n_1,p_1)} }
\]
\[
\infer[{n=n_0 - (n_1 \times (n_0 \div n_1))}]{\eval{a_0 \bmod a_1}{\sigma}{(n,p_0 \times p_1)}}
    {\eval{a_0}{\sigma}{(n_0,p_0)}\quad\eval{a_1}{\sigma}{(n_1,p_1)} }
\]
\[
\infer[{n=n_0 \times 2^{n_1}}]{\eval{a_0 <\!< a_1}{\sigma}{(n,p_0 \times p_1)}}
    {\eval{a_0}{\sigma}{(n_0,p_0)}\quad\eval{a_1}{\sigma}{(n_1,p_1)} }
\]
\[
\infer[{n=n_0 \div 2^{n_1}}]{\eval{a_0 >\!> a_1}{\sigma}{(n,p_0 \times p_1)}}
    {\eval{a_0}{\sigma}{(n_0,p_0)}\quad\eval{a_1}{\sigma}{(n_1,p_1)} }
\]


\subsection{Operational semantics of Boolean expressions}\label{subsec:opboolean}

\subsubsection{Constants and variables}
\[\infer{\eval{\true}{\sigma}{(\true,1.0)}}{}
\qquad
\infer{\eval{\false}{\sigma}{(\false,1.0)}}{}
\]
\[
\infer[\mathrm{Classical}]{\eval{X}{\sigma}{\sigma(X)}}{}
\qquad
\infer[\mathrm{Quantum}]{\eval{Z}{\sigma}{\sigma(Z)}}{}
\]
\subsubsection{Binary operations}
\[
\infer[\mathrm{when\ }n\mathrm{\ and\ }m\mathrm{\ are\ equal}]
     {\eval{a_0 = a_1}{\sigma}{(\true,p_0 \times p_1)}}
    {\eval{a_0}{\sigma}{(n,p_0)}\quad\eval{a_1}{\sigma}{(m,p_1)} }
\]