\section{Program definition}\label{sec:bnfProgramDefinition}
\lqpl{} programs consist of a series of definitions at the
global level. These are either \emph{data definitions} which
give a description of an algebraic data type \emph{procedure definitions}
which define executable code.

\begin{singlespace}
\begin{bnf}
   <Linearqplprogram>   :: <global definitions>
  
   <global_definitions> :: <global_definitions> <global_definition>
        | empty 
  
   <global_definition> :: <data_definition>
        | <procedure_definition>
\end{bnf}
\end{singlespace}

\section{Data definition}\label{sec:bnfDataDefinition}
A data definition consists of declaring a type name, with an optional
list of type variables and a list of constructors for that type. 
It is a semantic error to have different types having the same 
constructor name, or to redeclare a type name. 

Constructor definitions allow either fixed types or uses of the
type variables mentioned in the type declaration. 

\begin{singlespace}
\begin{bnf}
   <data_definition> :: <type_definition> '=' 
                        '\{' <constructor_list> '\}'
  
   <type_definition> :: 'type' <constructorid> <id_list>
  
   <constructor_list>:: <constructor> <more_constructor_list>
      
   <more_constructor_list> :: 
        '|' <constructor> <more_constructor_list>
        | \{- empty -\}
  
   <constructor> :: <constructorid> '(' <typevar_list> ')'
        | <constructorid>
  
   <typevar_list> :: <typevar> <moretypevar_list>
             
   <moretypevar_list> :: ',' typevar moretypevar_list
        | \{- empty -\} 

   <typevar> :: <identifier> 
        | <identifier>
        | <constructor>
        | <constructor> '(' <typevar_list> ')' 
        | <builtintype>

   <builtintype>:: 'Qbit'  | 'Int' | 'Bool'
\end{bnf}
\end{singlespace}

\section{Procedure definition}\label{sec:bnfProcedureDefinition}
Procedures may only be defined at  the global level
in a \lqpl{} program. The definition consists of
a procedure name, its input and output formal parameters and
a body of code. Note that a procedure may have either no input, no
outputs or neither.

The classical and quantum inputs are  separated by a '|'. 
Definitions with either no parameters or no classical parameters are
specific special cases. 

\begin{singlespace}
\begin{bnf}
   <procedure_definition> :: <identifier> '::' 
            '(' <parameter_definitions> '|' 
	        <parameter_definitions> ';' 	
                <parameter_definitions>  ')'
                    '='  <block>
        | <identifier> '::' 
            '(' <parameter_definitions> ';' 	
                <parameter_definitions>  ')'
                    '='  <block>
        | <identifier> '::' '(' ')' '=' <block>
  
   <parameter_definitions> :: <parameter_definition>
             <more_parameter_definitions>
        | \{- empty -\}
  
   <more_parameter_definitions> :: ',' <parameter_definition> 
	     <more_parameter_definitions>
        | \{- empty -\}
  
   <parameter_definition> :: <identifier> ':' <constructorid>
             '(' <typevar_list> ')' 
        | <identifier> ':' <constructorid>
        | <identifier> ':' <builtintype>
     
\end{bnf}
\end{singlespace}

\section{Statements}\label{sec:bnfStatementDefinition}
Although \lqpl{} is a functional  language, the language retains the
concept of \emph{statements} which provide an execution flow for the
program. 

The valid collections of statements are \emph{blocks} which
are lists of statements.

\begin{singlespace}
\begin{bnf}   

   <block> :: '\{' <stmtlist> '\}' 
  
   <stmtlist>:: <stmtlist> ';' <stmt>
        | <stmtlist> ';'
        | <stmt>
        | \{- empty -\}
     
\end{bnf}
\end{singlespace}

Statements are broadly grouped into a few classes.
\subsection{Assignment} 
Variables are created by assigning to them. There is no 
ability to separately declare them. Type unification will determine the
appropriate type for the variable.

\begin{singlespace}
\begin{bnf}   
   <stmt> :: <identifier> '=' <exp>
\end{bnf}
\end{singlespace}

\subsection{Case statements} 
These are \inlqpl{measure, case, use, discard} and
the classical assign, \inlqpl{:=}. 
These statements give the programmer the capability to
specify different processing on the sub-stacks of
a quantum variable. This is done with dependant statements. 
For  \inlqpl{measure} 
and \inlqpl{case},
the dependent statements are in the block specified in the 
statement. For \inlqpl{use}, they may be
specified explicitly, or they may be all the 
statements following the \inlqpl{use} to the end
of the enclosing block. The classical assign is syntactic sugar for an
assignment followed by a \inlqpl{use} with no explicit dependent statements.

The \inlqpl{discard} statement is grouped here due to the quantum effects.
Doing a discard of a \qbit{} is equivalent to measuring the \qbit{} and 
ignoring the results. This same pattern is followed for discarding 
quantum variables of all types.

\begin{singlespace}
\begin{bnf} 
   (<stmt> continued)    
        | 'case' <exp> 'of' <cases>  
        | 'measure' <exp> 'of' <zeroalt> <onealt> 
	| <identifier> ':=' <exp>
        | 'use' <identifier_list> <block>
        | 'use' <identifier_list> 
\end{bnf}
\end{singlespace}

\subsection{Functions} 
This category includes procedures and
transforms. A variety of calling syntax is available, however, there
are no semantic differences between them.

\begin{singlespace}
\begin{bnf}   
   (<stmt> continued)  
        | '(' <identifier_list> ') '=' 
	      <callable>  '(' <exp_list> ')'
        | '(' <identifier_list> ') '='
	      <callable>  '(' <exp_list> '|' <exp_list> ')'
        | <callable> <ids>
        | <callable> '(' <exp_list> ')' <ids>
        | <callable> '(' <exp_list> '|' <exp_list> ';' 
	      <ids> ')'
\end{bnf}
\end{singlespace}

\subsection{Blocks} 
The block statement allows grouping of a 
series of statements by enclosing them with \inlqpl{\{} and \inlqpl{\}}.
An empty statement is also valid.

\begin{singlespace}
\begin{bnf}   
   (<stmt> continued)  
        |  <block>  
	| \{- empty -\}  
\end{bnf}
\end{singlespace}

\subsection{Control} 
\lqpl{} provides a statement for classical control and
one for quantum control. Note that quantum control affects only the semantics
of any transformations applied within the control. Classical control requires
the expressions in its guards (see below) to be classical and not quantum.

\begin{singlespace}
\begin{bnf}   
   (<stmt> continued)  
        | 'if' guards
	| <stmt> '<=' <identifier_list>
\end{bnf}
\end{singlespace}

\subsection{Divergence}
 This signifies that this portion of the program does not
terminate. Statements after this will have no effect. 
\begin{singlespace}
\begin{bnf}   
   (<stmt> continued)  
        | 'zero'                       
\end{bnf}
\end{singlespace}

\section{Parts of statements}\label{sec:bnfStatementPartsDefinition}
The portions of statements are explained below. First is
\emph{callable} which can be either a procedure name or a 
particular unitary transformation.

\begin{singlespace}
\begin{bnf}  
   <callable> :: <identifer> | <transform>
\end{bnf}
\end{singlespace}

The alternatives of a measure statement consist of choice indicators 
for the base of the measure followed by a block of statements.

\begin{singlespace}
\begin{bnf} 
   <zeroalt> ::  '|0>' '=>' <block>
  
   <onealt> :: '|1>' '=>' <block>
\end{bnf}
\end{singlespace}

The \inlqpl{if} statement requires a list of \emph{guards} following it.
Each guard is composed of a  classical expression that will evaluate to 
\inlqpl{true} or \inlqpl{false}, followed by a block of guarded statements. 
The statements guarded by the expression will be executed only when
the expression in the guard is true. 
The list of guards must end with a  default guard
called \inlqpl{else}. Semantically, this is equivalent to putting a guard of
\inlqpl{true}.

\begin{singlespace}
\begin{bnf} 
   <guards> :: <freeguards> <owguard>

   <freeguards> :: <freeguard> <freeguards>
        | \{- empty -\}

   <freeguard> :: <exp> '=>' <block>

   <owguard> :: 'else' '=>' <block>
\end{bnf}
\end{singlespace}

When deconstructing a data type with a \inlqpl{case} statement, a
pattern match is used to determine which set of dependent 
 statements are executed. The patterns allow the programmer
to either throw away the data element (using the '\inlqpl{\_}' special 
pattern), or assign it to a new identifier.

\begin{singlespace}
\begin{bnf}   
   <cases> :: <case> <more_cases> 
  
   <more_cases> :: \{- empty -\}
        | <case> <more_cases>
  
   <case> :: <caseclause> '=>' <block>
  
   <caseclause> ::  <constructorid> '(' <pattern_list> ')' 
        | <constructorid>
  
   <pattern_list>:: <pattern> <more_patterns>

   <more_patterns> :: ',' <pattern> <more_patterns>
        | \{- empty -\} 

   <pattern> :: <identifier> | '_'
\end{bnf}
\end{singlespace}

\section{Expressions}\label{sec:bnfExpressionDefinition}
\lqpl{} provides  standard expressions, with the restriction that
 arithmetic expressions 
 may be done only on classical values. That is, they must be 
on the classical stack or a constant.

The results of comparisons are Boolean values that will be held on
the classical stack.

\begin{singlespace}
\begin{bnf}   

   <exp>:: <exp0> 
  
   <exp0>:: <exp0> <or_op> <exp1> | <exp1>
  
   <exp1>:: <exp1> '&&' <exp2> | <exp2> 
  
   <exp2>:: '~' <exp2> | <exp3>  | <exp3> <compare_op> <exp3>
  
   <exp3>:: <exp3> <add_op> <exp4> | <exp4>
  
   <exp4>:: <exp4> <mul_op> <exp5> | <exp5> 
  
   <exp5>:: <exp5> <shift_op> <exp6> | <exp6>
  
   <exp6>:: <identifier> | <number> | 'true' | 'false' 
        | '(' <exp> ')' 
        | <constructorid> '(' <exp_list> ')' 
        | <constructorid>
	| <identifier> '('  ')'
        | <identifier> '(' <exp_list> ')'  
        | <identifier> '(' <exp_list> ';' ids ')'
	| '|0>' | '|1>'
  
   <exp_list>:: <exp> <more_exp_list>
  
   <more_exp_list>:: ',' <exp> <more_exp_list>  
        | \{- empty -\} 
\end{bnf}
\end{singlespace}

\section{Miscellaneous and lexical}\label{sec:bnfMiscLexicalDefinition}
These are the basic elements of the language as used above. Many of these
items are differentiated at the lexing stage of the compiler.

\begin{singlespace}
\begin{bnf}   

   <idlist> :: <identifier> more_ids
        | \{- empty -\}
 
   <more_ids> :: <identifier> more_ids
        | \{- empty -\}
 
   <identifier_list>:: <identifier> <more_idlist>
        | \{- empty -\}

   <more_idlist>:: ',' <identifier> <more_idlist> 
        | \{- empty -\}
    
  
   <or_op>:: '||' | '^'
  
   <compare_op>:: '==' | '<' | '>' | '=<' | '>=' | '=/=' 

   <add_op>::'+' | '-' 

   <mul_op>::'*' | 'div'  | 'rem'

   <shift_op>::'>>' | '<<'

   <transform>:: <gate> 
        | <transform> *o* transform 

   <gate> :: 'Had' | 'T'  | 'Phase' | 'Not' |  'RhoX'  
	| 'Swap' | 'Rot '| 'RhoY ' | '  RhoZ ' | 'Inv-'<gate>

   <identifier> :: <lower> | <identifier><letterOrDigit>
   <constructorid :: <upper> | <constructorid><letterOrDigit>
   <letterOrDigit> :: <upper>|<lower>|<digit>
   <number> :: ['+'|'-'] <digit>+
   <lower> ::   'a' | 'b' | 'c' | 'd' | 'e' | 'f' | 'g' 
        | 'h' | 'i' | 'j' | 'k' | 'l' | 'm' | 'n' | 'o' 
        | 'p' | 'q' | 'r' | 's' | 't' | 'u' | 'v' | 'w' 
	| 'x' | 'y' | 'z'
   <upper> ::   'A' | 'B' | 'C' | 'D' | 'E' | 'F' | 'G' 
        | 'H' | 'I' | 'J' | 'K' | 'L' | 'M' | 'N' | 'O' 
        | 'P' | 'Q' | 'R' | 'S' | 'T' | 'U' | 'V' | 'W' 
        | 'X' | 'Y' | 'Z'
   <digit> ::   '0' | '1' | '2' | '3' | '4' | '5' | '6' 
        | '7' | '8' | '9'
\end{bnf}
\end{singlespace}