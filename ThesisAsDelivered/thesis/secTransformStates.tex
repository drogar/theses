\section{Transforming between the states}\label{sec:statetransforming}

As introduced in \vref{sec:qsmstate}, there are four different 
state descriptions for the transition diagrams.
They are:
\begin{gather*}
\bms = (\cd,S,Q,D,N)\\
\bms' = (\cd,S,L(Q),D,N)\\
\cms = (\cd, C , [(S,L(Q),D,N)])\\
\ms = \il{(\cd, C , [(S,L(Q),D,N)])} 
\end{gather*}

These actual states are used in the Haskell implementation for the 
quantum stack machine. In that implementation, 
\vref{subsec:QSM:machinedescription}  gives the 
details of the  higher order functions which
lift a function on one of the base states (\bms, \bms', \cms) to the 
machine state (\il{(\cd, C , [(S,L(Q),D,N)])}).

\subsection{Lifting functions on \bms{} to \bms'.}\label{subsec:liftbmstobmsprime}
An endomorphism of \bms{} is changed to an endomorphism of \bms{}' by 
ignoring the labelling. That is, lifting of the endomorphism is accomplished
by commuting it
with the labelling. So, if
\[ f\,(\cd,S,Q,D,N) = (\cd_f,S_f,Q_f,D_f,N_f)\]
then 
\[ (\text{lift}\ f)(\cd,S,L(Q),D,N) = (\cd_f,S_f,L(Q_f),D_f,N_f).\]


\subsection{Lifting functions on \bms{}' to \cms}\label{subsec:liftbmstocms}
This sub-section describes how to lift a restricted type of
 endomorphism on $\bms' = (\cd,S,L(Q),D,N)$ to
an endomorphism on $\cms = (\cd, C , [(S,L(Q),D,N)])$.

When moving from \bms{}' to \cms, the lifting of the function maps it 
across the list of tuples in \cms, while ignoring the control object $C$. We
require one restriction on the endomorphism for this to work. 
The restriction is that the code result of the endomorphism 
only depends on the code of the input. That is,
given $f::\bms' \to\bms'$ where
 $f (\cd,S,L(Q),D,N) = (\cd_f,S_f,L(Q)_f,D_f,N_f) $ then
$\cd_f = f_1(\cd)$ for some function $f_1$.

The lifting function that does this is defined in a series of smaller steps.
First, consider
$(\cd, C , [(S,L(Q),D,N)])$, an
object in \cms. This can be transformed to
\[(C,[(\cd,S,L(Q),D,N)])\]
by duplication $\cd$ and ``pasting'' it in front of each tuple in the list.
Given the endomorphism $f$ of $\bms'$, it may now be applied to each of 
the elements of the list, as each of them is an object in \bms'. 

Since  $f$ is restricted as above, it is allowed to reverse
the ``pasting'' and bring the code portion out of the list of tuples. As
each code portion was equal before 
applying $f$ and the code result
only depends on the code input for $f$, 
it is safe to simply use the code from the
first tuple in the list. Removing the code entry from all of the other tuples
 completes the transformation back to an  object of \cms.


\subsection{Lifting functions on \cms{} to \ms}\label{subsec:liftcmstoms}
This sub-section describes how to lift an
 endomorphism on  $\cms = (\cd, C , [(S,L(Q),D,N)])$ to 
an endomorphism on $\ms = \il{(\cd, C , [(S,L(Q),D,N)])}$.

The final stage, from \cms{} to \ms{}, makes use of the fact that
the infinite list functor is known to be a monad. Thus, 
the Kleisli lifting will transform any endomorphism in \cms{} to an
endomorphism in \ms{}.

Infinite lists are given by the co-inductive type 
defined with the destructors $\hd$, which returns the base type, and
$\tl$, which returns a new infinite list.
\[\ila=\nu x.\{\hd: A, \tl: x\}\]
Categorically this corresponds to the diagram in \vref{fig:inflistalgebra} 
which defines infinite lists.
\begin{figure}[htbp]
\[
\xymatrix{
A \ar@{=}[d] &
    C \ar[l]_{h} \ar[r]^{t} \ar@{.>}[d]^{f}&
    C \ar@{.>}[d]^{f}\\
A &
    \ila \ar[l]^{\hd} \ar[r]_{\tl}&
    \ila
 }
\]
\caption{Initial algebra diagram for infinite lists}\label{fig:inflistalgebra}
\end{figure}

Recall that for a monad, given the functor $T$, (which is $\il{\_}$ in
this case) the natural transformations 
$\eta::I\to T, \mu::T^2\to T$ so that the
diagrams in \vref{fig:opstandardmonad} commute must be supplied.
\begin{figure}[htbp]
\[
\xymatrix{
T^3 \ar[r]^{T\mu} \ar[d]_{\mu T} &
    T^2 \ar[d]^{\mu} \\
T^2 \ar[r]_{\mu} &
    T
 }\quad \mathrm{\ and\ }\quad
\xymatrix{
IT \ar@{=}[dr] \ar[r]^{\eta T} &
    T^2 \ar[d]^{\mu} &
    TI \ar[l]_{T\eta} \ar@{=}[dl]\\
 & T &
 }
\]
\caption{Standard monad coherence diagrams}\label{fig:opstandardmonad}
\end{figure}

In the case of the infinite list, $\mu=$\emph{diagonal} 
and $\eta=$\emph{constant}, that is:
\begin{gather*}
\mu([[a_{00},a_{01},\ldots],[a_{10},a_{11},\ldots],\ldots]) = 
[a_{00},a_{11},a_{22},\ldots] \\
\eta(a) = [a,a,a,\ldots]
\end{gather*}

What this means is that for any endomorphism in \cms, it is  lifted to \ms{}
 by
just applying it across the entire infinite list. When composing these with 
native endomorphisms of \cms{} (in the quantum stack machine, 
only the \qsmins{Call} instruction generates
one of these), composition is done by
 taking the diagonal of the resulting infinite list of infinite lists.
\TODO{ - seems a bit light....}