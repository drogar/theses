\section{Introduction to the quantum stack machine}\label{sec:introStackMachine}
The quantum stack machine provides an execution environment where
 quantum and classical data may be manipulated. The primary component
of this machine is the \emph{quantum stack}, which stores both quantum and
probabilistic data.

The quantum stack  has the same function as a  classical stack 
in that it provides the basic
operations and data structures
 required for quantum computation. 
\Ref{chap:semantics} initially showed how quantum circuits can be
interpreted as acting on  a simple quantum stack consisting
of \bits{} and \qbits. Later sections of \ref{chap:semantics}
extended the quantum stack with datatype and classical data nodes,
together with operations on those nodes. \Ref{sec:semanticsiteration}
gave the interpretation of recursive functions acting on a quantum stack.

This chapter  describes a machine using this full quantum stack and other
data structures to provide an execution environment for \lqpl{} programs.

