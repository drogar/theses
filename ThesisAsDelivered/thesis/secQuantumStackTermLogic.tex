\section{Quantum stacks}\label{sec:quantumstacksemantics}
This section will define a quantum stack and use quantum stacks to 
provide an operational semantics
for \lqpl{} statements. The previous section has 
provided a translation of quantum circuits  into these statements. 
\subsection{Definition of a quantum stack}\label{subsec:definitionquantumstack}
In classical computing, a \emph{stack} is an object together with operations
for pushing new items onto the stack, retrieving items from the stack and 
examining whether the stack is empty or not. Refinements are often made in
terms of adding linear addressing to the stack, multiple push-pull operations
or defining specific data elements that are allowed on the stack.

A \emph{quantum stack} is a somewhat more complex object, due to the properties
of entanglement and superpositions. However, the basic 
idea of an object together with
methods of adding and removing elements is retained. The initiating 
idea of the quantum stack was to find a stack-like  way to represent the
state of a quantum system and therefore the density matrix describing it.

The construction of a quantum stack is given by four judgements, shown
in \vref{fig:formingaquantumstack}. These judgements show the quantum stack
is a pair, $\Gamma \qentail S$.  The $\Gamma$, called the \emph{context},
is a list of names and their types. The $S$, called the \emph{stack}, contains
the actual stack nodes and traces. The stack will often be written as $S^t$ 
where the $t$ is an explicitly shown trace.

The judgement rules do not 
restrict quantum stacks to be Hermitian (i.e., correspond
to a tuple of matrices, all of which are Hermitian)
 or to have a trace less than 1.
However, when applied to a quantum stack with these properties, 
the statements used to interpret quantum circuits
 do not increase the trace and take Hermitian stacks to Hermitian stacks.
\begin{figure}[htbp]
\[
\begin{gathered}
\infer[\text{}]
	{ \qentail \emptyset^{0}}{} \qquad
\infer[\text{scalar}]
	{\qentail \alpha^{\sqrt{\alpha\overline{\alpha}}}}{}\\ 
\\
\infer[\text{\qbit}]
	{q{::}\qbit, \Gamma \qentail 
          \qsnodeij{q}{ij}{S_{ij}}{t_{00} + t_{11}}}
        {\Gamma \qentail  S_{00}^{t_{00}},S_{01}^{t_{01}},S_{10}^{t_{10}},S_{11}^{t_{11}}}\\
\\
\infer[\text{\bit}]
	{b{::}\bit, \Gamma \qentail  \qsnodeij{b}{i}{S_{i}}{t_{0} + t_{1}}}
        {\Gamma \qentail  S_{0}^{t_{0}},S_{1}^{t_{1}} }\\
\end{gathered}
\]
\caption{Judgements  for a quantum stack}\label{fig:formingaquantumstack}
\end{figure}



\subsubsection{Description of the  quantum stack}
A quantum stack is a tree where each node is 
associated with a particular \bit{} or \qbit{}.
 The nodes associated with
a \qbit{}  have four branches, corresponding to the density matrix 
representation of a \qbit. The nodes associated with a \bit{} 
have two branches, corresponding to the probabilistic representation of a bit.
(i.e., \bit\ $b=\alpha 0 + \beta 1$, where $\alpha + \beta \le 1.0$). 
All nodes have a \emph{trace}. 
Data values, which are the result of multiplying the 
 probabilities of \bits{} and density matrix entries for \qbits{},
 are  stored at the leaves of the tree.

The trace of a quantum stack is a non-negative real number. It is defined recursively as shown in 
\vref{fig:definitionoftrace}.

\begin{figure}[htbp]
{\begin{singlespace}
\[  \text{trace}(Q) = 
\begin{cases}
\text{trace} (Q_{00}) + \text{trace} (Q_{11}) & \text{when}\ Q=
\qsqbitUp{q}{Q_{00}}{Q_{01}}{Q_{10}}{Q_{11}}{}\\
\text{trace} (B_{0}) + \text{trace} (B_{1}) & \text{when}\ Q=b\{0\qsp B_0; 1\qsp B_1\}\\
\sqrt{v\overline{v}} & \text{when}\ Q=v\ \text{(a scalar at the leaf)}
\end{cases}
\]
\end{singlespace}
}
\caption{Definition of the trace of a quantum stack}\label{fig:definitionoftrace}
\end{figure}

\subsubsection{Examples}

The quantum stack of a single \bit{}, $b$, which is $0$ with a probability
of $.25$ and $1$ with a probability of $.75$, is represented as: 
\[\qsbit{b}{.25}{.75}{1.0}.\]
A \qbit{} $q$ that was $0$ and then subjected to the Hadamard transform 
followed by a Phase transformation
is represented as:
\[\qsqbit{q}{.5}{-.5i}{.5i}{.5}{1.0}.\]
A general \qbit{} $r$ having the density matrix $\begin{singlespace}\begin{bmatrix}a_{00}&a_{01}\\a_{10}&a_{11}\end{bmatrix}\end{singlespace}$ is
represented as
\[\qsnodeij{r}{ij}{a_{ij}}{(a_{00} + a_{11})}\]
where the $ij$ of the labelling correspond to
 the row and column of $q$'s density matrix.
The number at the upper right of the node representation is its trace.

Given a single \bit{} $b$ as above and a \qbit{} $q_1$ which is $0$,
the representation is
\[\qsbit{b}{\qsnodeij{q_1}{00}{.25}{.25}}{\qsnodeij{q_1}{00}{.75}{.75}}{1.0}.\]
As shown in this example,
 values that are zero probabilities or zero entries of the 
matrix can be elided without loss of meaning. 

\subsection{Quantum stack equivalence}\label{subsec:quantumstackequivalance}
The equality judgements for quantum stacks are given in 
\vref{fig:qstackequality}.
\begin{figure}[htbp]
\[
\begin{gathered}
\infer[\text{reflexivity}]
	{ \Gamma \qentail S = \Gamma\qentail S}{\Gamma\qentail S} \qquad
\infer[\text{symmetry}]
	{\Gamma\qentail S_1 = \Gamma\qentail S_2}
        {\Gamma\qentail S_2 = \Gamma\qentail S_1}\\ 
\\
\infer[\text{transitivity}]
	{\Gamma\qentail S_1 = \Gamma\qentail S_3}
        {\Gamma\qentail S_1 = \Gamma\qentail S_2 & \Gamma\qentail S_2 = \Gamma\qentail S_3}\\
\\\infer[\raisebox{1ex}{rotation}]
	{ \Gamma,(t{::}\tau),\Gamma'\qentail S =
     \qop{(\text{rotate }t)}{\Gamma,(t{::}\tau),\Gamma'\qentail S }}
{\Gamma,(t{::}\tau),\Gamma'\qentail S}
\end{gathered}
\]
\caption{Judgements  for quantum stack equality}\label{fig:qstackequality}
\end{figure}


Rotation  brings a desired node to the top of the stack and 
then reorganizes the tree so that the 
Huffman-like encoding of the leaves 
is invariant.   Given a tree with four 
nodes ($a,b,c,d$), each of the leaves  can be encoded by the indexes
depending on the node names. e.g., $v_{i_a;i_b;i_c;i_d}$. Then, after
rotating a tree, following a particular index path in the tree will
take you to the same node. That is, if the tree order is now ($d,b,c,a$) with
$d,c$ \qbits, $a,b$ \bits, then following the path $01\to1\to10\to0$ will
take you to the leaf $v_{0;1;10;01}$, where the labelling is from
the original $a,b,c,d$ ordering. An example of this in the 
\bit, \qbit{} case is provided below.

\subsubsection{Rotation example}
Consider the following
single \bit, single \qbit{} quantum stack,
\[
{\begin{singlespace}
\qsbitUp{c}
    {\qsqbitUp{r}{v_{0;00}}{v_{0;01}}{v_{0;10}}{v_{0;11}}{v_{0;00}+v_{0;11}}}
    {\qsqbitUp{r}{v_{1;00}}{v_{1;01}}{v_{1;10}}{v_{1;11}}{v_{1;00}+v_{1;11}}}
    {\sum v}
\end{singlespace}
},
\]
where $\sum v = v_{0;00}+v_{0;11}+v_{1;00}+v_{1;11}$.
If the \qbit{} $r$ is rotated to the  top of the quantum stack, the
resulting stack will be
\[{\begin{singlespace}\qsqbitUp{r}{\qsbit{c}{v_{0;00}}{v_{1;00}}{v'_{00}}}%
   {\qsbit{c}{v_{0;01}}{v_{1;01}}{v'_{01}}}%
   {\qsbit{c}{v_{0;10}}{v_{1;10}}{v'_{10}}}%
   {\qsbit{c}{v_{0;11}}{v_{1;11}}{v'_{11}}}%
   {\sum v}
\end{singlespace}
},
\]
where $\sum v $ is as before and $v'_{ij} = v_{0;ij} + v_{1;ij}$.



\subsubsection{Details of rotation on a quantum stack}
Rotation is defined recursively on a quantum stack. Given a 
target name $t$ and a quantum stack $\Gamma\qentail S$,
 rotation brings the node named $t$ to the top and 
thus changes $\Gamma\qentail S$  to the quantum stack 
$(t:\tau),\Gamma' \qentail S'$. The quantum stack  
$\Gamma\qentail S$ is equivalent to  $(t:\tau),\Gamma' \qentail S'$.

If a quantum stack has duplicate names, only the highest node with that 
name will be rotated up.
Rotation's definition can
be broken into cases depending on
whether the target name is at the top, the second element or somewhere lower
in the stack and whether the nodes being affected are \bits{} or \qbits.

When the target name is the top element of the stack, there is no 
change to the stack and rotation is just an identity transformation of
the quantum stack.

In the case where the target name is the second element of the stack,
 there are four separate cases  to
consider: \bit{} above a  \bit; \bit{} above a \qbit; 
\qbit{} above a \qbit{} and \qbit{} above a \bit{}.

{\begin{footnotesize}
\begin{singlespace}
\bit{} above a \bit{}:
\begin{equation*}
 \begin{footnotesize}\qop{(\text{rotate}_1\ b_2)}
           {\qsbitUp{b_1}{\qsbit{b_2}{S_{0;0}}{S_{0;1}}{}}%
				{\qsbit{b_2}{S_{1;0}}{S_{1;1}}{}}{t}} =
      {\qsbitUp{b_2}{\qsbit{b_1}{S_{0;0}}{S_{1;0}}{}}%
				{\qsbit{b_1}{S_{0;1}}{S_{1;1}}{}}{t}}.\end{footnotesize}
\end{equation*}
\bit{} above a \qbit{}
\begin{align*}
&\begin{footnotesize} \qop{(\text{rotate}_1\ q_2)}
    {\qsbitUp{b_1}{\qsqbitUp{q_2}{S_{0;00}}{S_{0;01}}{S_{0;10}}{S_{0;11}}{}}%
   {\qsqbitUp{q_2}{S_{1;00}}{S_{1;01}}{S_{1;10}}{S_{1;11}}{}}{t}} \end{footnotesize} \\
&\begin{footnotesize}    = \qsqbitUp{q_2}{\qsbit{b_1}{S_{0;00}}{S_{1;00}}{}}%
         {\qsbit{b_1}{S_{0;01}}{S_{1;01}}{}}%
         {\qsbit{b_1}{S_{0;10}}{S_{1;10}}{}}%
         {\qsbit{b_1}{S_{0;11}}{S_{1;11}}{}}{t}.\end{footnotesize}
\end{align*}
\qbit{} above a \bit{}:
\begin{align*}
&\begin{footnotesize}\qop{(\text{rotate}_1\ b_2)}
{\qsqbitAllUp{q_1}{\qsbit{b_2}{S_{00;0}}{S_{00;1}}{}}%
         {\qsbit{b_2}{S_{01;0}}{S_{01;1}}{}}%
         {\qsbit{b_2}{S_{10;0}}{S_{10;1}}{}}%
         {\qsbit{b_2}{S_{11;0}}{S_{11;1}}{}}{t}} \end{footnotesize} \\
&\begin{footnotesize}=\qsbitUp{b_2}{\qsqbit{q_1}{S_{00;0}}{S_{01;0}}{S_{10;0}}{S_{11;0}}{}}%
   {\qsqbit{q_1}{S_{00;1}}{S_{01;1}}{S_{10;1}}{S_{11;1}}{}}{t}.\end{footnotesize}
\end{align*}
\qbit{} above a \qbit{}:
\begin{align*}
&\begin{footnotesize}\qop{(\text{rotate}_1\ q_2)}
  {\qsqbitAllUp{q_1}{\qsqbitUp{q_2}{S_{00;00}}{S_{00;01}}{S_{00;10}}{S_{00;11}}{}}%
   {\qsqbitUp{q_2}{S_{01;00}}{S_{01;01}}{S_{01;10}}{S_{01;11}}{}}% 
   {\qsqbitUp{q_2}{S_{10;00}}{S_{10;01}}{S_{10;10}}{S_{10;11}}{}}% 
   {\qsqbitUp{q_2}{S_{11;00}}{S_{11;01}}{S_{11;10}}{S_{11;11}}{}}% 
   {t}} \end{footnotesize}\\
&\begin{footnotesize}=\qsqbitAllUp{q_2}{\qsqbitUp{q_1}{S_{00;00}}{S_{01;00}}{S_{10;00}}{S_{11;00}}{}}%
    {\qsqbitUp{q_1}{S_{00;01}}{S_{01;01}}{S_{10;01}}{S_{11;01}}{}}%
     {\qsqbitUp{q_1}{S_{00;10}}{S_{01;10}}{S_{10;10}}{S_{11;10}}{}}%
     {\qsqbitUp{q_1}{S_{00;11}}{S_{01;11}}{S_{10;11}}{S_{11;11}}{}}{t}.\end{footnotesize}
\end{align*}
\end{singlespace}
\end{footnotesize}
}

When the node to be rotated up is deeper in the quantum stack, we
recurse down. Otherwise, we just apply $\text{rotate}_1$. That is:
\begin{align*}
&\qop{(\text{rotate}\ n_3)}{\qsnodeij{n_1}{\ell_k}
    {\qsnodeij{n_2}{\ell'_m}{S}{}}{}} \\
&=  \qop{(\text{rotate}_1\ n_3)}
      {\qsnodeij{n_1}{\ell_k}{\qop{(\text{rotate}\ n_3)}{\qsnodeij{n_2}{\ell'_m}{S}{}}}{}}
\end{align*}
and
\begin{equation*}
\qop{(\text{rotate}\ n_2)}{\qsnodeij{n_1}{\ell_k}
    {\qsnodeij{n_2}{\ell'_m}{S}{}}{}} =
  \qop{(\text{rotate}_1\ n_2)}
      {\qsnodeij{n_1}{\ell_k}{\qsnodeij{n_2}{\ell'_m}{S}{}}{}}.
\end{equation*}

In this way, the desired node ``bubbles'' to the top of the quantum stack.



\subsection{Basic operations on a quantum stack}\label{subsec:quantumstackbasicops}

Quantum stacks may be added, tensored and multiplied by scalars. 
Multiplication of a quantum stack $\Gamma\qentail S$
 by a scalar $\alpha$ recurses down the stack of $S$
 until the leaves are reached. Each leaf value is then 
multiplied by $\alpha$. This will also have the effect of multiplying the
trace of $S$ by $\sqrt{\alpha\overline{\alpha}}$.

When adding two stacks, $\Gamma \qentail S$ and $\Gamma \qentail T$,
the first step is to align the stacks. Aligning is done by recursively
rotating nodes in the second stack to correspond with the
same named nodes in the first stack. When the alignment is completed,
the values at the corresponding leaves are added to produce a 
result stack. The trace
of the resulting quantum stack will be the sum of the traces of $S$ and $T$.

Tensoring of two quantum stacks $\Gamma_1 \qentail S_1$ and
$\Gamma_2 \qentail S_2$ proceeds by first creating a 
new context $\Gamma$, which is the concatenation of
$\Gamma_1$ and $\Gamma_2$. The new stack portion $S$ is
created by replacing each leaf value of $S_1$ with
a copy of $S_2$ multiplied by the previous leaf value. The 
trace of the tensor will be the product of the traces of 
$S_1$ and $S_2$. The notation for this operation is
\[S=S_1\otimes S_2.\]
The judgements for these stack operations are shown 
in \vref{fig:judgementsbasicstackops}.

\begin{figure}[htbp]
\[
\begin{gathered}
\infer[\raisebox{1em}{scalar multiply}]
   {\Gamma \qentail  (\alpha S)^{t\sqrt{\alpha\overline{\alpha}}}}
   {\Gamma \qentail  S^t}\\ 
\\
\infer[\raisebox{1.5em}{add}]
   {\Gamma \qentail (S_1 + S_2)^{t_1+t_2}} 
   {\Gamma \qentail  S_1^{t_1}, S_2^{t_2}}\\ 
\\
\infer[\raisebox{1.5em}{tensor}]
   {\Gamma_1 \Gamma_2 \qentail (S_1 \otimes S_2)^{t_1\times t_2}} 
   {\Gamma_1 \qentail  S_1^{t_1} & \Gamma_2 \qentail S_2^{t_2}}
\end{gathered}
\]
\caption{Judgements for basic stack operations}\label{fig:judgementsbasicstackops}
\end{figure}


When recursing down stacks to do an addition or a scalar multiplication,
any elided branches are treated as having the required structure with 
leaf values of $0$.

As an
example, consider the following two quantum stacks:
%WARN - Macros not used for quantumstack display
\begin{align*}
S_1 &= q\{00\qsp \qsbit{b}{.1}{.1}{.2};11\qsp b{::}\{1\qsp.1\}^{.25}\}^{.45}\\
S_2 &= q\{00\qsp \qsbit{b}{.01}{.1}{.15};11\qsp \qsbit{b}{.15}{.25}{.4}\}^{.55}.
\end{align*}
In the first stage, recursively add the $00$ branches of each, i.e.:
\begin{align*}
S_1[00] &=  \qsbit{b}{.1}{.1}{.2}\\
S_2[00] &=  \qsbit{b}{.05}{.1}{.15}.
\end{align*}
When adding these two branches, each of the labels immediately 
point to leaves. As leaves are the base case for addition, this gives
a result of:
\begin{align*}
S'[00] &=  \qsbit{b}{.15}{.2}{.35}.
\end{align*}
Similarly, apply the addition process in the same manner  to the $11$ branches
of $S_1$ and $S_2$:
\begin{align*}
S_1[11] &=  \qsnodeij{b}{1}{.25}{.25}\\
S_2[11] &=  \qsbit{b}{.15}{.25}{.4}.
\end{align*}
Note that in this case, the first sub-stack does not have a branch labelled
by $0$, due to elision of zero leaf values. The result of adding this is
\begin{align*}
S'[11] &= \qsbit{b}{.15}{.5}{.65},
\end{align*}
which gives us the final result of:
\begin{align*}
S&=q\{00\qsp \qsbit{b}{.15}{.2}{.35}; 11\qsp \qsbit{b}{.15}{.5}{.65}\}^{1}.
\end{align*}

The general cases for  addition are given in 
\vref{fig:definitionofqstackaddtion}. In the figure, the 
token $ \langle empty \rangle$ stands for any sub-branches that
have been elided because all leaves below that node are $0$.

\begin{figure}[htbp]
\begin{singlespace}
\begin{align*}
S_1  + \langle empty \rangle &= S_1\\
\langle empty \rangle + S_2 &= S_2\\
v_1 + v_2 &= v_1 + v_2 \\
v_1 + \langle empty \rangle  &= v_1 \\
\langle empty \rangle + v_2 &=  v_2 \\
nm{::}\{\ell_{i} \qsp S_i\} +  nm{::}\{\ell_{i} \qsp S'_i\}
    &= nm{::}\{\ell_i \qsp (S_i + S'_i)\}\\
S_1 + S_2 &= S_1 +\qop{(\text{rotate}\ nm_{S_1})}{S_2}.
\end{align*}
\caption{Definition of addition of quantum stacks}\label{fig:definitionofqstackaddtion}
\end{singlespace}
\end{figure} 



