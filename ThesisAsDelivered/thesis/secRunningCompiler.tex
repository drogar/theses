\section{Running the \lqpl{} compiler}\label{appsec:runningthecompiler}
The compiler is run from the command line as:
\begin{Verbatim}
  lqplc <options> <infiles>
\end{Verbatim}

This will run the compiler on each of the input files \emph{infiles} which 
are expected to have a suffix '\terminalio{.qpl}'. The compiled files will be 
written with the suffix '\terminalio{.qpo}'.

The options allowed for the compiler are:
\begin{description}
\item{\terminalio{-e, {-}{-}echo\_code}} Echo the input files to 
 \terminalio{stderr}.
\item{\terminalio{-s, {-}{-}syntactic}} This option will cause the 
compiler to print a syntax parse tree on \terminalio{stderr}.
\item{\terminalio{-r, {-}{-}ir\_print}} This compiler option
will force the printing of the intermediate representation generated
during the semantic analysis phase.
\item{\terminalio{-h, {-}{-}help}} This prints a 
help message describing these
options on \terminalio{stderr}.
\item{\terminalio{-V, -?, {-}{-}version}} This option prints the version information of the 
compiler on \terminalio{stderr}.
\item{\terminalio{-o[FILE], {-}{-}output[=FILE]}} This will
cause the compiler to write the compiled QSM code to 
\terminalio{FILE}.
\item{\terminalio{-i[DIRLIST], {-}{-}includes=DIRLIST}} For this option,
\terminalio{DIRLIST} is expected to be a list of semi-colon separated
directories. The compiler will use the directory list when searching for
any  import files.
\end{description}


