\section{Semantics of datatypes}\label{sec:semanticsdatatypes}
To this point, this chapter has only considered quantum stacks with
two node types, \bit{} and \qbit{}. This section will add nodes for
constructed datatypes and  probabilistic classical data and statements
that operate on these nodes.

Constructed datatypes will allow the addition of algebraic datatypes
to our quantum stacks. This includes   sum, product,  and recursive 
data types. For example, types
such as \emph{List} (a recursive type), \emph{Either} (a sum type)
and \emph{Pair} (a product type) are now definable. 

\subsection{Statements for constructed  datatypes and classical data}
Two new statements are required to implement datatypes on the
quantum stack: \emph{node construction} and a \emph{case} statement. 
A third statement \emph{discard data} is also added. \emph{Discard data}
 is semantic sugar for doing a \emph{case} without any
dependent statements.

With the addition of classical data, arithmetic and logical 
\emph{expressions} are  added to \lqpl{}. 
The standard \emph{new} and \emph{discard} statements
as well as a \emph{use} statement are added to the language. The
\emph{if-else} statement is added to allow choices based on
classical expressions.

The judgements for expressions are given in \ref{fig:judgeexpressions}
and for the new statements 
in \ref{fig:judgementdatatypestatements}. Judgements for expressions
use the notation
\[\Gamma_c|\Gamma \eentail e::\tcls,\]
which means that given the context $\Gamma_c$ of 
classical variables in the porch and the context $\Gamma$ of
quantum variables, $e$ is a valid expression, with the classical
type $\tcls$. Judgements for statements
will also carry the classical context in the porch as above. Statements
introduced to this point have not been affected by the classical context,
nor have their operational semantics affected the classical context.
The new format of judgements for statements will be:
\[ \Gamma_c | \Gamma ; \Gamma_c'|\Gamma' \ientail \qins.\]

For the operational semantics, the classical context must also now
be carried in the porch. Additionally, the data in the classical context
is not stored in the quantum stack, but in an adjacent standard stack. The
full format of the semantics of a statement operating on the 
stacks  will now be:
\[ \qresultsin{\Gamma_c | \Gamma \qentail \qop{\qins}{(S,C)}}
          {\Gamma_c' |\Gamma' \qentail (S',C')}
\]
In many cases, the classical portions are not involved in the
semantics and thus may be elided, resulting in
our original syntax for the operational semantics.

\begin{figure}[htbp]
\[
\begin{gathered}
\infer[{\raisebox{3ex}{integers}}]
   { \Gamma_c|\Gamma
      \eentail n::Int}
   {n\in\integers }\qquad
\infer[{\raisebox{3ex}{Booleans}}]
   { \Gamma_c|\Gamma
      \eentail b::Boolean}
   {n\in \{\text{true, false}\}}\\
\\
\infer[{\raisebox{2ex}{operations}}]
   { \Gamma_c|\Gamma
      \eentail (e_1\, op \, e_2)::\tau_{op}}
   { \Gamma_c|\Gamma \eentail e_1 &  \Gamma_c|\Gamma \eentail e_2}\\
\\
\infer[{\raisebox{2ex}{identifiers}}]
   {(n::\tcls), \Gamma_c|\Gamma \eentail n::\tcls}
   {}
\end{gathered}
\]
\caption{Judgements for expressions in \lqpl.}\label{fig:judgeexpressions}
\end{figure}






\begin{figure}[htbp]
\[
\begin{gathered}
\infer[{\raisebox{3ex}{new data}}]
   { (v_1::\tau_0, \dots v_n::\tau_n),\Gamma; (x::\tau(C)),\Gamma 
      \ientail x = C(v_1,\dots,v_n)}
   {}\\
\\
\infer[{\raisebox{4ex}{case}}]
   {(nd::\tau),\Gamma ; \Gamma' \ientail {\qcase{nd}{C_i(v_{ij})}{\qins_i}{C_i\in \tau(C_i)}}}
 {\{\Gamma; \Gamma' \ientail \qins_i\}_i}\\
\\
\infer[{\raisebox{2ex}{discard data}}]
   {(nd::\tau),\Gamma ; \Gamma \ientail {\text{disc }nd}}
{}\\
\\
\infer[{\raisebox{3ex}{new classical}}]
   { \Gamma_c | \Gamma; \Gamma_c|(x::\tcls),\Gamma 
      \ientail x = e}
   {\Gamma_c|\Gamma \eentail e}\\
\\
\infer[{\raisebox{2ex}{use}}]
   {\Gamma_c | (n::\tcls),\Gamma ; \Gamma_c|\Gamma' \ientail 
        {\quse{n}{\qins}}}
 {(n::\tcls),\Gamma_c|\Gamma; \Gamma_c|\Gamma' \ientail \qins}\\
\\
\infer[{\raisebox{7ex}{if-else}}]
   {\Gamma_c | \Gamma ; \Gamma_c|\Gamma' \ientail 
        {\qifelse{e_0}{\qins_0}{e_i}{\qins_i}{i}{\qins_n}}}
 {\{\Gamma_c|\Gamma \eentail e_i{::}Boolean\}_{i=0,\dots,n-1} &
    \{\Gamma_c| \Gamma ; \Gamma_c^{i} | \Gamma' \ientail \qins_i\}_{i=0,\dots,n}}\\
\\
\infer[{\raisebox{2ex}{discard classical}}]
   {\Gamma_c|(n::\tcls),\Gamma ;\Gamma_c'| \Gamma' \ientail {\text{disc }n}}
{}
\end{gathered}
\]
\caption{Judgements for datatype and classical data statements.}\label{fig:judgementdatatypestatements}
\end{figure}

\subsection{Operational semantics}
\subsubsection{Semantics for datatype statements}
In the quantum stack, a datatype node of 
type $\tau$ has multiple branches, where each branch is 
labelled with a datatype constructor $C$ of $\tau$ 
and the \emph{bound nodes} for
$C$. Constructors may require 0 or more bound nodes. 

The \emph{case} statement provides dependent statements to be executed
for each of the branches of a datatype node $d$. Upon completion of the 
case, the node $d$ is no longer available.

In discussing datatypes,  the notation $\theta_z$ is used for
 creating a ``fresh'' variable. The notation $\tau(Cns)$ for
the type of the  constructor $Cns$.

The operational semantics for the datatype statements 
  are given in \ref{fig:qsdatatypetransitions}. As \emph{new data}
and \emph{discard data} have no effect on the classical context, the 
original syntax of the semantics statements is retained for these two
statements. The
\emph{case} statement, however, does affect the classical context
in that the classical
context is reset at the beginning of execution of each set of dependent
 statements and at completion of the statement. Therefore, the semantics
for \emph{case} will use the full syntax introduced in this section.


\begin{figure}[htbp]
\[
\begin{gathered}
\infer[{\raisebox{4.8ex}{new data}}]
   { \qresultsinUp{(x_i{::}\tau_i)_{i=1,\dots m}, \Gamma \qentail
          \qop{(nd=Cns(x_1,\dots,x_m))}{S}} 
         {\quad(nd{::}\tau(Cns)), 
             (\theta_{x_i}{::}\tau_i)_{i=1,\dots m}, \Gamma'\qentail
             \qsnodeij{nd}{Cns(\theta_{x_1},\dots,\theta_{x_m})}{S'}{}}}
   {\qresultsin{(x_i{::}\tau_i)_{i=1,\dots m}, \Gamma \qentail
         \qop{([\theta_{x_1}/x_1];\dots;[\theta_{x_m};x_m])}{S}} 
         {(\theta_{x_i}{::}\tau_i)_{i=1,\dots m}, \Gamma'\qentail S'}}\\
\\
\infer[{\raisebox{3ex}{discard data}}]
   {{\qresultsin{(nd::\tau),\Gamma\qentail 
          \qop{\text{disc nd}}{\qsnodeij{nd}{C_i(x_{ij})}{S_{i}}{}}}
          {\Gamma'\qentail \sum S'_{i}}}}
 {\{\qresultsin{\Gamma\qentail 
       \qop{(\{\text{disc }x_{ij}\}_j)}{S_{i}}}{\Gamma'\qentail S'_i}\}_i}\\ 
\\
\infer[{\raisebox{7ex}{case}}]
   {{\qresultsinUp{\Gamma_c|(nd::\tau),\Gamma\qentail 
       \qop{\qcase{nd}{Cns_i(v_{ij})}{\qins_i}{}}
           {(\qsnodeij{nd}{Cns_i(x_{ij})}{S_{i}}{},C)}}
          {\qquad\qquad\qquad\qquad\qquad\qquad\qquad\qquad\Gamma_c|\Gamma'\qentail (\sum S'_{i},C)}}}
 {\{\qresultsin{\Gamma_c|\Gamma\qentail 
       \qop{([v_{ij}/x_{ij}])_j;\qins_i}{(S_{i},C)}}
               {\Gamma_c'|\Gamma'\qentail (S'_i,C_i)}\}_i}
\end{gathered}
\]
\caption{Operational semantics for datatype statements}\label{fig:qsdatatypetransitions}
\end{figure}

\subsubsection{Classical data}\label{subsec:quantumstackclassicaldata}
Adding probabilistic classical data led to adding a second
data structure for the operational semantics. In order to perform standard 
arithmetic and logical (or other classically defined) operations on this
data, a \emph{classical stack} is paired with the quantum stack. 

Classical data nodes extend the concept of probabilistic \bits{} to 
other types. For example, a node $c$ could hold
an integer that had value $7$ with probability $.3$, $5$ with probability
$.2$ and $37$ with probability $.5$. A probabilistic node is
 restricted to items of all the same type, e.g., all integers, all floats.

The classical stack  is used to operate on
non-probabilistic classical data and will interact with the classical nodes
 via \emph{classical construction} and \emph{use}.

The \emph{use} statement had a set of dependent 
statements, $\qins$, which are
executed for each sub-stack $S_i$ of a a classical node $c$. Prior to 
executing $\qins$, the classical value $k_i$ which labels the 
sub-branch $S_i$ is moved onto the top of the classical stack.
When the statements $\qins$ have been executed for each of $c$'s sub-stacks, 
$c$ is removed and the resulting sub-branches are added together.

The classical stack has  standard arithmetic and logic operations defined
on it. The \emph{if - else} statement is defined to execute various 
statements depending upon classical expressions.

The judgements for the classical note construction, discarding, use and
interaction with the classical stack are shown in 
\ref{fig:qsclassicaltransitions}.


\begin{figure}[htbp]
\[
\begin{gathered}
\infer[{\raisebox{2.5ex}{new classical}}]
   { \qresultsin{\Gamma_c|\Gamma \qentail\qop{(n=e)}{(S,C)}} 
         {\Gamma_c|(n{::}\tcls),\Gamma\qentail
             (\qsnodeij{n}{e}{S}{},C)}}
   {\Gamma_c|\Gamma \eentail e}\\ 
\\
\infer[{\raisebox{2.5ex}{use}}]
   {{\qresultsin{\Gamma_c|(n{::}\tcls),\Gamma\qentail \qop{\quse{nd}{\qins}}{(\qsnodeij{nd}{k_i}{S_{i}}{},C)}}
          {\Gamma_c|\Gamma'\qentail (\sum S'_{i},C)}}}
 {\{\qresultsin{(n{::}\tcls),\Gamma_c|\Gamma\qentail \qop{\qins}{(S_{i},k_i:C)}}{\Gamma_c'|\Gamma'\qentail (S'_i,C'_i)}\}_i}\\
\\
\infer[{\raisebox{2.5ex}{discard classical}}]
   {{\qresultsin{\Gamma_c|(n{::}\tcls),\Gamma\qentail 
         \qop{\text{disc n}}{(\qsnodeij{nd}{cv_i}{S_{i}}{},C)}}
          {\Gamma_c|\Gamma\qentail (\sum S_{i}, C)}}}
 {}\\
\\
\infer[{\raisebox{7ex}{if-else, $e_k$ true}}]
   {\qresultsin{\Gamma_c | \Gamma \qentail 
        \qop{\qifelse{e_0}{\qins_0}{e_i}{\qins_i}{i}{\qins_n}}{(S,C)}}
      {\Gamma_c^k | \Gamma' \qentail (S', C_k)}}
 { e_k = \text{\texttt{True}} & \{e_j = \text{\texttt{False}}\}_{j<k} & 
\qresultsin{\Gamma_c | \Gamma \qentail 
        \qop{\qins_k}{(S,C)}}
      {\Gamma_c^k | \Gamma' \qentail (S', C_k)}}\\
\\
\infer[{\raisebox{7ex}{if-else, else}}]
   {\qresultsin{\Gamma_c | \Gamma \qentail 
        \qop{\qifelse{e_0}{\qins_0}{e_i}{\qins_i}{i}{\qins_n}}{(S,C)}}
      {\Gamma_c^n | \Gamma' \qentail (S', C_n)}}
 { \{e_j = \text{\texttt{False}}\}_{j=0,\ldots,n-1} & 
\qresultsin{\Gamma_c | \Gamma \qentail 
        \qop{\qins_n}{(S,C)}}
      {\Gamma_c^n | \Gamma' \qentail (S', C_n)}}\\
\\
\infer[{\raisebox{2.5ex}{binary op}}]
   {{\qresultsin{i::\tau_1,j::\tau_2,\Gamma_c|\Gamma\qentail \qop{(\text{capp }op_c)}{(S,v_i:v_j:C)}}
       {k::\tau_3,\Gamma_c|\Gamma \qentail (S,v_k:C)}}}
          { v_i\,op_c\,v_j = v_k}
\end{gathered}
\]
\caption{Operational semantics for classical data statements}\label{fig:qsclassicaltransitions}
\end{figure}



An important point to note in the transitions for the \semins{use} statement
is that the classical stack is \emph{reset} at the beginning of each 
execution and in the final result. This enforces a block-like scoping on the 
classical stack. This  also applies to \semins{measure} and \semins{case}.
The  revised judgement for measure is given
in \vref{fig:revisedtransitionsformeasure}.



\begin{figure}[htbp]
\[
\begin{gathered}
\infer[{\raisebox{7.5ex}{measure'}}]
   {{\qresultsin{\Gamma_c|\Gamma\qentail \qop{\qmeas{q}{I_0}{I_1}}{(\qsnodeij{q}{ij}{S_{ij}}{},C)}}
          {\Gamma_c|\Gamma'\qentail ((S_0+S_1),C)}}}
 {\stacktwo{\qresultsin{\Gamma_c|\Gamma\qentail \qop{I_0}{(S_{00},C)}}
         {\Gamma_c'|\Gamma'\qentail (S_0,C_0)}}
    {\qresultsin{\Gamma_c|\Gamma\qentail\qop{I_1}{(S_{11},C)}}
      {\Gamma_c''|\Gamma'\qentail (S_1,C_1)}}}
\end{gathered}
\]
\caption{Revised semantics for measure with classical stack.}\label{fig:revisedtransitionsformeasure}
\end{figure}