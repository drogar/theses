\section{Shor's factoring algorithm}\label{sec:shorsFactoring}
Shor's algorithm is split into a classical computational part and a 
quantum computational.
The classical part can be considered as three steps:
\begin{itemize}
\item{}Setup;
\item{}Call order finding (a quantum algorithm);
\item{}Check to see if an answer has been found, repeat if not.
\end{itemize}

The code provided in this section is currently work in progress. It
represents my current understanding of the algorithm and how that
fits with the capabilities of \lqpl. Future \lqpl{} enhancements and
study of the algorithm will undoubtedly lead to changes and improvements.

\subsection{Setup for factoring}\label{subsec:setupforfactoring}
The algorithm begins by assuming input of a number $N$ for factoring. 
It then chooses a random number $a$ less than $N$ (see
\vref{fig:randomQPL}) and compute the $\gcd(a,N)$ (see \vref{fig:gcdQPL}).
If this is one, $a$ is a factor of $N$ and the algorithm is complete.

\begin{figure}[htbp]
\lstinputlisting[style=linqpl]{../testdata/shors/random.qpl}
\caption{\lqpl{} function to get a random number}\label{fig:randomQPL}
\end{figure}

\begin{figure}[htbp]
\lstinputlisting[style=linqpl]{../testdata/shors/gcd.qpl}
\caption{\lqpl{} function to compute the GCD}\label{fig:gcdQPL}
\end{figure}

\subsection{Order finding}\label{subsec:orderfinding}

The first part involve creating two \qbit{} ``registers'', which in 
\lqpl{} are items of type \inlqpl{List(Qbit)}. 
The lists need to be $\log_2 N$ 
\qbit{}s each,the first initialized to \ket{0}, the
second to \ket{1}, and individual Hadamard transforms must be applied
to the first register. This is accomplished by calling the functions
\inlqpl{intToZeroQbitList} and \inlqpl{hadList}
 as shown in \ref{fig:intToZeroqbitlist} and
\ref{fig:hadList}.


\begin{figure}[htbp]
\lstinputlisting[style=linqpl]{../testdata/shors/intToZeroQbitList.qpl}
\caption{Function to take an int and return a list of \qbits}
\label{fig:intToZeroqbitlist}
\end{figure}

\begin{figure}[htbp]
\lstinputlisting[style=linqpl]{../testdata/shors/hadList.qpl}
\caption{Function to apply Hadamard to all of a list of \qbits}\label{fig:hadList}
\end{figure}

The next step is to apply the unitary transform corresponding to the
function $f$, where $f(x) = a^x \mod N$. This is done via the 
specialized transform $UM$, which takes $x,N$ and $t$ as parametrization
inputs and then applies the transform to \emph{a list of \qbit{}s}, controlled
by the \qbit{s} in the first list. This is
%TODO - Dr. C - in Order find, can/should recursive call be controlled by q?
shown in \ref{fig:orderfind}.

\begin{figure}[htbp]
\lstinputlisting[style=linqpl]{../testdata/shors/orderFind.qpl}
\caption{Order finding algorithm}
\label{fig:orderfind}
\end{figure}

In the example program, the 
 matrix of the transformation $UM$ will be a $16\times 16$ matrix
as it operates on 4 \qbits. Let us determine what the matrix would be 
in the case $N=15, x=7$. 

First, compute $x^7\mod 15$ for $1,\ldots 14$.
\[\begin{array}{ccc}
1^7\equiv 1(= '0001')&6^7\equiv 6(= '0110')&11^7\equiv 11(='1011') \\
2^7\equiv 8(= '1000')&7^7\equiv 13(= '1110')&12^7\equiv 3(='0011') \\
3^7\equiv 12(= '1100')&8^7\equiv 2(= '0010')&13^7\equiv 7(='0111') \\
4^7\equiv 4(= '0100')&9^7\equiv 9(= '1001')&14^7\equiv 14(='1110') \\
5^7\equiv 5(= '0101')&10^7\equiv 10(= '1010')&\\
\end{array}\]
Note also that the bit strings $'0000', '1111'$ are mapped to themselves.

Identifying these numbers (i.e., using their corresponding bit strings)
 with the basis vectors of $\complex^4$,
 the function determines a permutation of the basis
vectors:
\[\begin{pmatrix}
\begin{array}{cccccccc}
0&1&2&3&4&5&6&7\\
0&1&8&12&4&5&6&13
\end{array}&
\begin{array}{cccccccc}
8&9&10&11&12&13&14&15\\
2&9&10&11&3&7&14&15
\end{array}
\end{pmatrix}
\]
This gives us the transformation matrix:
{\begin{singlespace}
\[\begin{bmatrix}
\begin{array}{cccccccc}
1&0&0&0&0&0&0&0\\
0&1&0&0&0&0&0&0\\
0&0&0&0&0&0&0&0\\
0&0&0&0&0&0&0&0\\
0&0&0&0&1&0&0&0\\
0&0&0&0&0&1&0&0\\
0&0&0&0&0&0&1&0\\
0&0&0&0&0&0&0&0
\end{array} &
\begin{array}{cccccccc}
0&0&0&0&0&0&0&0\\
1&0&0&0&0&0&0&0\\
0&0&0&0&1&0&0&0\\
0&0&0&0&0&0&0&0\\
0&0&0&0&0&0&0&0\\
0&0&0&0&0&0&0&0\\
0&0&0&0&0&0&0&0\\
0&0&0&0&0&1&0&0
\end{array} \\
&\\
\begin{array}{cccccccc}
0&1&0&0&0&0&0&0\\
0&0&0&0&0&0&0&0\\
0&0&0&0&0&0&0&0\\
0&0&0&0&0&0&0&0\\
0&0&1&0&0&0&0&0\\
0&0&0&0&0&0&0&0\\
0&0&0&0&0&0&1&0\\
0&0&0&0&0&0&0&0
\end{array} &
\begin{array}{cccccccc}
0&0&0&0&0&0&0&0\\
0&1&0&0&0&0&0&0\\
0&0&1&0&0&0&0&0\\
0&0&0&1&0&0&0&0\\
0&0&0&0&0&0&0&0\\
0&0&0&0&0&0&0&0\\
0&0&0&0&0&0&1&0\\
0&0&0&0&0&0&0&1
\end{array} 
\end{bmatrix}
\]
\end{singlespace}
}


The algorithm  then applies
the inverse quantum Fourier transform to the first 
\qbit{} list. This is given by the functions as shown in \ref{fig:inverseqft}
and \ref{fig:inverserotate}.

\begin{figure}[htbp]
\lstinputlisting[style=linqpl]{../testdata/shors/inverseQft.qpl}
\caption{Function to apply the inverse quantum Fourier transform}
\label{fig:inverseqft}
\end{figure}


\begin{figure}[htbp]
\lstinputlisting[style=linqpl]{../testdata/shors/inverseRotate.qpl}
\caption{Function to apply inverse rotations as part of the inverse QFT}
\label{fig:inverserotate}
\end{figure}

The algorithm  then measures
 the first list,  creating an integer from its value, as in
\vref{fig:qbitListToInt}. Call  the result of this $y$. 


\begin{figure}[htbp]
\lstinputlisting[style=linqpl]{../testdata/shors/qbitListToInt.qpl}
\caption{Function to measure a list and create a probabilistic integer}
\label{fig:qbitListToInt}
\end{figure}

The algorithm now takes $y$ and computes the 
denominator $r'$ of $y/N$ when reduced 
via a continued fraction algorithm. The code for the 
continued fraction algorithm is still under active development and 
not included at this time.

The final program is in \vref{fig:shorsfactoring}.


\begin{figure}[htbp]
\lstinputlisting[style=linqpl]{../testdata/shors/shor15.qpl}
\caption{Program to factor 15 via Shor's algorithm}
\label{fig:shorsfactoring}
\end{figure}
