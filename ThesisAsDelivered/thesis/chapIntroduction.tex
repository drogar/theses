\chapter{Introduction}\label{chap:introduction}
This thesis introduces a \emph{quantum stack} and its semantics 
together with a   quantum 
programming language  which has quantum control and classically controlled
data types. The work was motivated by the 
desire to provide a semantically correct programming language for
programming quantum algorithms at a higher level than \bits{} and \qbits.

The thesis starts with a brief comparison with, 
and contrast to other Haskell based
quantum simulators. This is followed by a short (re-)introduction of the 
basic concepts of linear algebra and quantum computing. 


The two main contributions of this thesis, the quantum stack and \lqpl{},
 may be reviewed independently of
each other. Some of the appendices refer to both the quantum stack and 
\lqpl. 

For those interested
in reviewing the introduction of the quantum stack machine and its 
implementation the recommended reading path is the chapter on
semantics, \ref{chap:semantics}, 
followed by   \ref{chap:quantumStackMachine}. In
 \ref{chap:qsmadditional}, further details are given on
quantum stack machine instructions and 
the translation of \lqpl{} into quantum stack machine code. 
The details of the Haskell
implementation of the machine are also covered in 
  \ref{chap:qsmimplementation}. 
Implementation  details of the quantum stack are 
in \ref{subsec:quantumstackdescription} while  details
of the staged implementation of the quantum stack machine are
 in   \ref{subsec:QSM:machinedescription}.

 \Ref{chap:informalintroductionLinearQuantumProgrammingLanguage} is
the primary required reading for learning \lqpl. All facets of the language
are presented in that chapter interspersed with a number of examples. 
Further examples of programs in \lqpl{} can be found  in 
 \ref{app:exampleprograms}. 
A complete BNF
description of \lqpl{} is available in  
 \ref{chap:formalSpecificationLinearQuantumProgrammingLanguage} and a
 description of how \lqpl{} is translated into quantum stack machine
code  in  \ref{sec:translationtoqsmcode}. 
 Details of running the compiler
 are in   \ref{app:usethesystem}.

