\section{Semantics of recursion}\label{sec:semanticsiteration}
This section will extend \lqpl{} with statements for recursion and provide
the operational semantics for them. 

\subsection{Statements for recursion}
The only statements added for recursion are  the \emph{proc} and
\emph{call} statements. \emph{Proc} is used to create a 
subroutine that is available for the \emph{call} statement.

All \emph{proc} statements are global in scope to a program, hence
a subroutine may \emph{call} itself or be called by other subroutines 
defined elsewhere in a program. The judgements for these two new statements
are given in \vref{fig:judgementsforprocandcall}. The judgements 
reuse the notation introduced earlier of 
$\qvec{z_j}$ signifying a vector of elements.


\begin{figure}[htbp]
\[
\begin{gathered}
\infer[{\raisebox{2ex}{proc}}]
   {\Gamma_c|\Gamma \ientail 
             \text{prc}_x ::(\qvec{c_i}{:}\qvec{{\tcls}_i} | 
                \qvec{qv_j}{:}\qvec{\tau_j} ; 
                 \qvec{r_k}{:}\qvec{\tau_k}) = \qins}
   {}\\ 
\\
\infer[{\raisebox{2ex}{call}}]
   {\Gamma_c | (\qvec{qv_j}::\qvec{\tau_j})\Gamma \ientail 
        (\qvec{r_k}) = \text{prc}_c(\qvec{e_i} | \qvec{qv_j})}
   {\{\Gamma_c | (\qvec{qv_j}::\qvec{\tau_j})\Gamma \eentail e_i\}_i}
\end{gathered}
\]
\caption{Judgements for formation of proc and call statements}\label{fig:judgementsforprocandcall}
\end{figure}


\subsection{Operational semantics for recursion}
In order to provide an operational semantics for recursion, the
 quantum stacks will now be considered a \emph{stream} (also known as an
infinite list) of quantum stacks. The semantics as given previously in
this chapter will stay the same, with the
provision that all of these apply across the entire stream. 

Additional notation is required  for calling subroutines, re-namings (formal
in and out parameters) and for applying
 a different transition dependent on the level 
in the stream. This additional notation is explained in
 \vref{tab:notationfortransitionsforinteration}. 



\begin{table}
\begin{tabular}{|>{$}c<{$}|p{3.75in}|}
\hline
\textbf{Notation} & \textbf{Meaning}\\
\hline \hline 
[z'/z]\Gamma \qentail [z'/z]S&Substituting $z'$ for $z$ in the context 
and stack. \\[12pt]
\hline
\qresultsindUp{d}{\Gamma_c|\Gamma\qentail\qop{\qins}{(S,C)}}
{\qquad\Gamma_c'\Gamma'\qentail (S',C')}&
Application of the statements $\Gamma_c|\Gamma;\Gamma_c'\Gamma' \ientail \qins{}$ to
to the quantum stack / classical stack pair $(S,C)$ in classical context $\Gamma_c$ and
context $\Gamma$  results 
in the quantum stack / classical stack pair  $(S',C')$ in classical context 
$\Gamma_c'$ and context $\Gamma'$ 
\emph{at the} $d^{\text{th}}$ \emph{iteration}.\\
\hline
\end{tabular}
\caption[Notation for quantum stack recursion]{Notation used in operational semantics once iteration is added}\label{tab:notationfortransitionsforinteration}
\end{table} 


The operational semantics for recursion is given in 
 \vref{fig:qsiterationtransitions}.
Recursion is done in the semantics by first \emph{diverging} at
the head of the infinite list and then calling a subroutine once at
the second element, calling twice in the third and so forth. In this 
way, the program gives a closer and closer approximation of the
actual results the further one looks down the stream of quantum stacks.

\begin{figure}[htbp]
\[
\begin{gathered}
\infer[{\raisebox{2ex}{rename}}]
   {\qresultsin{(x{::}T),\Gamma \qentail\qop{[x'/x]}{\qsnodeij{x}{v_i}{S_i}{t}}} 
         {(x'{::}T),\Gamma\qentail\qsnodeij{x'}{v_i}{S_i}{t}}}
   {}\\ 
\\
\infer[{\raisebox{2ex}{base call}}]
   {\qresultsind{0}{\Gamma_c| (\qvec{z_j}::\qvec{\tau_j})\Gamma \qentail 
              \qop{(\qvec{w_k}) = \text{prc}_a(\qvec{c_i}|\qvec{z_j})}{(S,C)}}
         {\Gamma_c|\emptyset\qentail (\emptyset^{0},C)}}
   {\text{prc}_a ::(\qvec{c_i}{:}\qvec{{\tcls}_i} | \qvec{z_j}{:}\qvec{\tau_j} ; 
        \qvec{w_k}{:}\qvec{\tau_k}) = \qins_a}\\ 
\\
\infer[{\raisebox{2ex}{call $d+1$}}]
   {\qresultsind{d+1}{\Gamma_c|(\qvec{z_j}::\qvec{\tau_j})\Gamma\qentail 
         \qop{(\qvec{w_k}) = \text{prc}_a(\qvec{c_i}|\qvec{z_j})}{(S,C)}}
         {\Gamma_c|\Gamma'\qentail  (S',C)}}
   {\stackthreec{\qresultsind{d}{\Gamma_c|\Gamma\qentail
             \qop{\{[z_j'/z_j]\}_j;\qins_a;\{[w_k/w_k']\}_k}{(S,C)}}
         {\Gamma_c'|(\qvec{w_k}::\qvec{\tau_k})\Gamma'\qentail (S',C')}}
       {\text{prc}_a ::(\qvec{c_i'}{:}\qvec{{\tcls}_i} | 
                 \qvec{z_j'}{:}\qvec{\tau_j} ; \qvec{w_k'}{:}\qvec{\tau_k}) = \qins_a}
{\{\Gamma_c | (\qvec{z_j}::\qvec{\tau_j})\Gamma \eentail e_i{::}{\tcls}_i\}_i}}
\end{gathered}
\]
\caption{Operational semantics for recursion}\label{fig:qsiterationtransitions}
\end{figure}

