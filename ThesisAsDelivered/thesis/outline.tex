\documentclass[12pt]{report}
\usepackage{palatino}
\usepackage{noweb}
\usepackage[T1]{fontenc}
\usepackage[latin1]{inputenc}
\usepackage[reqno]{amsmath}
\usepackage{amsthm}
\usepackage{amsfonts,amssymb}
\usepackage[mathscr]{euscript}
\usepackage[all]{xy}
\usepackage{stmaryrd}
\usepackage{graphicx}
%\usepackage{verbatim}
\usepackage{fancyvrb} 
\usepackage{proof}
\usepackage{makeidx}
%\usepackage{lucidabr}
%\usepackage{/home/grads/gilesb/latexFiles/proof}
\noweboptions{longxref,smallcode}
%include lhs2TeX.fmt
%include lhs2TeX.sty
%format ^* = "\ltimes"
%format *^ = "\rtimes"


\setlength{\textheight}{8.5in}
\setlength{\textwidth}{6.5in}
\setlength{\oddsidemargin}{-.3in}
\setlength{\evensidemargin}{-.3in}
% macros for processing of the literate files
\DefineVerbatimEnvironment%
 {happycodefirst}{Verbatim}{gobble=2,numbers=left,firstnumber=1,fontsize=\footnotesize}
\DefineVerbatimEnvironment%
 {happycodecont}{Verbatim}{gobble=2,numbers=left,firstnumber=last,fontsize=\footnotesize}
% import and redefinition of general macros
\input{/home/grads/gilesb/latexFiles/macros}
\newcommand{\inflist}[1]{\ensuremath{\mathbb{IL}({#1})}}
\renewcommand{\incsec}[1]{\subsection{#1}}
% end of redefinitions.
\newcommand{\incsubsec}[1]{\subsubsection{#1}}
\newcommand{\incsubsubsec}[1]{\paragraph{#1}}
\newcommand{\qplcode}[1]{\textbf{#1}}

 
\DefineVerbatimEnvironment%
{code}{Verbatim}{numbers=left,fontsize=\footnotesize,firstnumber=1}
\newcommand{\CodeResetNumbers}{\RecustomVerbatimEnvironment%
{code}{Verbatim}{numbers=left,fontsize=\footnotesize,firstnumber=1}}
\newcommand{\CodeContinueNumbers}{\RecustomVerbatimEnvironment%
{code}{Verbatim}{numbers=left,fontsize=\footnotesize,firstnumber=last}}

% End of macros for literate programming

\makeindex
\begin{document}
\author{Brett Giles}
\title{A Quantum Stack for Quantum Programming}
\maketitle
\tableofcontents{}
\chapter{Introduction}
\chapter{Previous Work}
\section{Quantum circuits}
\section{Semantics of QPL by Peter Selinger}
\section{Semantics of iteration}
\section{Semantics of data types}
\section{Haskell based Quantum computation emulators}
\chapter{Quantum Stack Machine}
\section{Representation of Qbits}
\section{Representation of Bits}
\section{Representation of general data types}
\section{Basic stack operations}
\section{Abstract semantics}
\chapter{A Quantum Programming Language}
\section{Definition}
\section{Syntax}
\section{Semantics}
\section{Translation to Stack Machine Code}
\chapter{Examples of Quantum Programs}
\section{Miscellaneous examples}
\section{Shor's Factoring Algorithm}



%\printindex
\bibliography{/home/grads/gilesb/latexFiles/BibTexFiles/computingScience}
\bibliographystyle{plain}

\end{document}