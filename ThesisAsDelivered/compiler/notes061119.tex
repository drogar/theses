\documentclass[12pt,letter]{article}
\usepackage{palatino,fancyhdr}
\pagestyle{fancy}
\setcounter{tocdepth}{2}
\setlength{\headheight}{15pt}
\date{2006-04-19}
\author{Brett Giles}
\lhead{\bfseries 2006-04-19}
\chead{\thepage}
\rhead{\slshape \rightmark}
\lfoot{\emph{For Review}}
\cfoot{}
\rfoot{}
\renewcommand{\headrulewidth}{0.4pt}
\renewcommand{\footrulewidth}{0.4pt}
\begin{document}
\section{Function Calls}\label{sec:functioncalls}
Currently there are a multitude of ways to call functions. In the 
following discussion, we assume we have the following function definitions.
\begin{verbatim}
f3in2out ::(n:Int, h:Qbit, l:List (Qbit); h:Qbit, l:List (Qbit))
= {...}

f2in0out :: (i:Int, j:Int;)
= {...}

f2in2out ::(i:Int, j:Int; i:Int, j:Int)
= {...}

f2in1out ::(i:Int, j:Int; j:Int)
= {...}
\end{verbatim}

The language currently allows the following equivalent methods for calling
these functions.

For \verb|f3in2out|:
\begin{verbatim}
l = f3in2out(n, h,l;h) ;
f3in2out(n, h, l; h,l);
f3in2out(n,h;h) l;
f3in2out(n) h l;
\end{verbatim}
Another suggestion is to remove the first option, but allow 
\begin{verbatim}
(h,l) = f3in2out(n,h,l);
\end{verbatim}

My opinion is we have too many options now.

For \verb|f2in0out| there is only one option:
\begin{verbatim}
f2in0out(3,4);
\end{verbatim}

The function \verb|f2in2out| has as many options as our first example:
\begin{verbatim}
j = f2in2out(i,j;i);
f2in2out(i,j;i,j);
f2in2out(i;i) j;
f2in2out() i j;
f2in2out i j;
\end{verbatim}
The same pair assignment syntax is suggested here.

The function \verb|f2in1out| has the following equivalent calls:
\begin{verbatim}
j = f2in1out(i,j);
f2in1out(i,j;j);
f2in1out(i) j;
\end{verbatim}
\section{Expressions}\label{sec:Expressions}
Last time we discussed allowing arithmetic only in the context of an Use 
statement. I don't thing this is the way to go. For example, in the
function to get the length of a list,
\begin{verbatim}
len ::(l: List (a) ; i:Int) =
{
   case l of
     Nil => {i = 0}
     Cons (_, l1) => {
         i = 1 + len(l1)
     } 
}
\end{verbatim}
the assigment to 'i' at the end really should be allowed. I think that we 
should just allow general arithmetic, with the realization that if you 
are not in a Use statement, then any variables are used up by the expression.
\section{Int and Bit}\label{sec:intandbit}
The more I look at the system (and especially the machine) the less I think
we should keep BIT as at type (as you suggested earlier). I think it would
be best to just switch all BIT things to INT and make the machine work 
that way. 

This leads to three cases (and probably others)
\subsection{Allow IF, WHILE to work on INT}
We could continue to keep an 'IF' based on 'INT', where the 'THEN' clause
is selected when the 'INT' has value zero, the 'ELSE' clause otherwise.

This would be the fastest and easist to implement. Minor changes are needed
in the machine to handle this. 

Integer expressions in this case would always evaluate to another integer.
For example,(using the notation $(i:p_1,j:p_2,\ldots)$ to mean $i$ has 
percentage $p_1$, $j$ has percentage $p_2$, \ldots)
we would have
\begin{itemize}
\item{} $(3:1.00) > (2:1.00)$ would evaluate to $(1:1.00)$
\item{} $(3:.50, 0:.50) > (2:.50,-1:.50) = (1:.75,0:.25)$.
\end{itemize}

Similarly the 'WHILE' could continue to operate on an 'INT' where the loop 
continues whenever there is a non-zero value.
\subsection{Drop IF and WHILE}\label{sec:dropifwhile}
We could just drop the WHILE and IF statements, requiring 
CASE decomposition for booleans and recursive subroutines. I think this
leads to a fair bit of awkwardness in the language though.

\subsection{Builtin BOOL}
This would require a fairly significant overhaul of the machine. I
don't honestly think it is a viable thing to do right now.


\end{document}